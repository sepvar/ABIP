\documentclass[11pt,letter,openany,makeidx]{book}
\usepackage{amsmath}
\usepackage{macros}
\usepackage{comment}
\usepackage{graphicx}
\usepackage{microtype}
\usepackage{gfsdidot}
\usepackage[T1]{fontenc}
\usepackage{booktabs}
\usepackage{underscore}
\usepackage{caption}
\usepackage[within=chapter,chapterlistsgap=6pt]{newfloat}
\usepackage{tocloft}
\usepackage{xpicture}
\usepackage{xcolor}
%\usepackage[dvips]{xcolor}
%\GetGinDriver  % for xcolor to work well with hyperref
%\usepackage[\GinDriver]{hyperref}
\usepackage{ulem}%for \sout (done)
\usepackage[colorinlistoftodos]{todonotes}
%\usepackage[disable,colorinlistoftodos]{todonotes}
%\usepackage{layouts}
%\usepackage{showframe}
\usepackage{coordsys}
\usepackage[pdftex]{hyperref}

%\usepackage{cellpage}

\hypersetup{colorlinks=true,bookmarks=true,pdftitle=Algebra-Based Introductory Physics,pdfauthor=J.Christensen,pdfdisplaydoctitle}
% If using a bibliography, then include "backref" in list of \hypersetup items
% linkcolor=color of internal links (red); anchorcolor = color of anchor text (black); citecolor = bibliographic citations (green); filecolor = color for local URL files (cyan); menucolor = Acrobat menu (red); urlcolor = external links (magenta); hidelinks = remove all color
% citebordercolor = color of box for citations (0 1 0); fileborder = links to files box (0 .5 .5); linkbordercolor = normal links (1 0 0); menuborder; urlborder; allbordercolors; pdfborder

\includecomment{ForMe}
\includecomment{ForReviewer}
\includecomment{ForPublic}

\makeindex

\newlistof{example}{loe}{List of Examples}
\DeclareFloatingEnvironment[fileext=loe,listname="List of Examples",name=Example]{example}
\setlength{\cftexamplenumwidth}{1cm}
\newcounter{sample}
\newcounter{carrysample}
\renewcommand{\thesample}{Simple Example \arabic{sample}}
\renewcommand{\thecarrysample}{Simple Example \arabic{carrysample}}
\newenvironment{sample}{\color{rgb:red,0;green,2;blue,1}\begin{list}{\textbf{\thesample}:}{\usecounter{sample} \setcounter{sample}{\value{carrysample}} \leftmargin 12pt}}{\end{list}\setcounter{carrysample}{\value{sample}}}
\newcommand{\THREE}[6]{\vspace{-3pt}\begin{flushright} Select one:  \mbox{#1 (\ref{#4})},  \mbox{#2 (\ref{#5})}, or \mbox{#3 (\ref{#6})}.\end{flushright}}
\newcommand{\TWO}[4]{\begin{flushright} Select one:  \mbox{#1 (\ref{#3})} or \mbox{#2 (\ref{#4})}.\end{flushright}}
\newcommand{\YN}[2]{\TWO{Yes}{No}{#1}{#2}}
\newcommand{\TF}[2]{\TWO{True}{False}{#1}{#2}}
\newcommand{\return}[1]{{} \hfill \mbox{Return to \ref{#1}.}}
\newcommand{\autoreturn}[1]{{} \hfill \mbox{Return to \autoref{#1}.}}
\newcommand{\linkreturn}[2][a related idea]{{}\hfill \mbox{Return to the discussion of \protect{\hyperlink{#2}{#1}}.}}
\newcommand{\mmr}[1]{\mbox{[\protect{#1}]}}
\newcommand{\multireturn}[1]{{}\hfill Return to one of the following locations: \newline #1.}
\newcounter{AtIQ}
\renewcommand{\theAtIQ}{Answer \arabic{AtIQ}}
\newenvironment{AIQ}{\begin{list}{\textbf{Interactive \theAtIQ}:}{\usecounter{AtIQ} \leftmargin 12pt}}{\end{list}}

% Related (return), but not part of...
\newcommand{\mreturn}[1]{\note{Return to \protect{\ref{#1}}.}}
\newcommand{\mlinkreturn}[2][a related idea]{\note{Return to the discussion of \protect{\hyperlink{#2}{#1}}.}}
\newcommand{\mautoreturn}[1]{\note{Return to \protect{\autoref{#1}}.}}
\newcommand{\mmultireturn}[1]{\note{Return to one of the following locations: \newline #1.}}


\newlistof{adventure}{loa}{List of Adventures}
\DeclareFloatingEnvironment[fileext=loa,listname="List of Adventures",name=Adventure]{adventure}
\setlength{\cftadventurenumwidth}{1cm}
\newcounter{CYOA}
\renewcommand{\theCYOA}{Plan \Alph{CYOA}}
\newenvironment{CYOA}{\begin{list}{\textbf{\theCYOA}:}{\usecounter{CYOA}}}{\end{list}}
\newcounter{storyline}
\renewcommand{\thestoryline}{Storyline \arabic{storyline}}
\newenvironment{Story}{\begin{list}{\textbf{\thestoryline}:}{\usecounter{storyline} \leftmargin 12pt}}{\end{list}}

\newlistof{reallife}{irl}{List of Real Life Patterns}
%\DeclareFloatingEnvironment[fileext=irl,listname="List of Real Life Patterns",chapterlistsgaps=off,name=Real Life Patterns]{reallife}
\DeclareFloatingEnvironment[fileext=irl,listname="List of Real Life Patterns",name=Real Life Patterns]{reallife}
\setlength{\cftreallifenumwidth}{1cm}
\newcounter{IRL}
%\renewcommand{\theIRL}{\arabic{IRL}}
\newenvironment{realtable}{%\renewcommand{\arraystretch}{2}
                           %\hspace{-.2in}
                            \begin{tabular}{@{}lll@{}} \toprule Do This & Notice This & Ask This  \\ }
                            {\bottomrule \end{tabular} }%\renewcommand{\arraystretch}{.5}}
\newcommand{\dna}[3]{\midrule \begin{minipage}{4cm}\raggedright #1 \end{minipage}
                   & \begin{minipage}{4cm}\raggedright #2 \end{minipage}
                   & \begin{minipage}{4cm}\raggedright #3 \end{minipage} \\ }
\newcommand{\multidna}[1]{\multicolumn{3}{|c|}{\begin{minipage}{13cm}\center #1 \end{minipage}} \\ \midrule }


\newlistof{story}{los}{The Stories of the Equations}
\DeclareFloatingEnvironment[fileext=los,listname="The Stories of the Equations",name=This Equation's Story]{story}
\setlength{\cftstorynumwidth}{1cm}
\newcommand{\thestoryof}[1]{\marginpar{\raggedright \footnotesize The story of \\ \fcolorbox{black}{yellow}{\begin{minipage}[c]{1.5in} \center $\deq #1$ \end{minipage}}}}
\newcommand{\EqStory}[2]{\left[ {\color{rgb:red,1;green,1;blue,4} \begin{minipage}{#1}\raggedright\begin{center} #2 \end{center}\end{minipage}} \right]}
\newcommand{\EqStoryOver}[3]{\overbrace{\EqStory{#1}{#2}}^{\displaystyle #3}}
\newcommand{\EqStoryUnder}[3]{\underbrace{\EqStory{#1}{#2}}_{\displaystyle #3}}
\newcommand{\EqStoryFrac}[5]{\frac{\overbrace{\EqStory{#1}{#2}}^{\displaystyle #3}}
                                 {\underbrace{\EqStory{#1}{#4}}_{\displaystyle #5}}}


%%%%%%%%%%%%%%%%%%%%%%%%%%%%%%%%%%%%%%%%%%%%%%%%%%%%%%%%%%%%
%
%\presetkeys{todonotes}{fancyline,color=blue!15}{}
\presetkeys{todonotes}{color=blue!15,linecolor=blue!75,size=\footnotesize}{}
%
\newcounter{todocounter}
\newcommand{\dothis}[2][]
{\stepcounter{todocounter}\todo[color=green!30, #1]{\thetodocounter: #2}}
\newcommand{\docaption}[3][]
{\stepcounter{todocounter}\todo[color=green!30, prepend, caption={\thetodocounter: \underline{#2}}, #1]{#3}}
\newcommand{\addlink}[2][]
{\stepcounter{todocounter}\todo[prepend, caption={\thetodocounter: \underline{Add Link}}, #1]{#2}}
\newcounter{todourgentcounter}
\newcommand{\urgent}[2][]
{\stepcounter{todourgentcounter}\todo[color=orange!50, #1]{\thetodourgentcounter: #2}}
\newcommand{\urgcap}[3][]
{\stepcounter{todourgentcounter}\todo[color=orange!50, prepend, caption={\thetodourgentcounter: \underline{#2}}, #1]{#3}}
\newcommand{\done}[2][]
{\todo[color=yellow!10, #1]{\sout{#2}}}
%
%\newcommand{\new}[2]{}%
\newcommand{\new}[2]{\marginpar{\raggedright \footnotesize New to #1 \\ \fcolorbox{blue}{yellow!10}{\begin{minipage}[c]{1.5in} \center {\color{blue} #2 } \end{minipage}}}}%
%%%%%%%%%%%%%%%%%%%%%%%%%%%%%%%%%%%%%%%%%%%%%%%%%%%%%%%%%%%%


%%%%%%%%%%%%%%%%%%%%%%%%%%%%%%%%%%%%%%%%%%%%%%%%%%%%%%%%%%%%
%
%\newcommand{\deq}{\displaystyle}
%\newcommand{\txtfrac}[2]{{}^{#1}\!/_{\!#2}}
%
%%%%%%%%%%%%%%%%%%%%%%%%%%%%%%%%%%%%%%%%%%%%%%%%%%%%%%%%%%%%



%%%%%%%%%%%%%%%%%%%%%%%%%%%%%%%%%%%%%%%%%%%%%%%%%%%%%%%%%%%%
%
% PEOPLE AND PRONOUNS
%
% According to https://www.cdc.gov/nchs/fastats/body-measurements.htm
% Measured average height, weight, and waist circumference for adults ages 20 years and over
% Men:
% Height (inches): 69.3                 = 1.760 m
% Weight (pounds): 195.5                = 88.86 kg
% Waist circumference (inches): 39.7    = 1.01 m
% Women:
% Height (inches): 63.8                 = 1.621 m
% Weight (pounds): 166.2                = 75.55 kg
% Waist circumference (inches): 37.5    = 0.9525 m
% Source: Anthropometric Reference Data for Children and Adults: United States, 2007-2010, tables 4, 6, 10, 12, 19, 20[PDF - 1.7 MB]
%  https://www.cdc.gov/nchs/data/series/sr_11/sr11_252.pdf
%
\newcommand{\studentA}{Abdul}       \newcommand{\massA}{\mbox{$85.0\unit{kg}$}}
\newcommand{\studentB}{Beth}        \newcommand{\massB}{\mbox{$75.0\unit{kg}$}}
\newcommand{\studentC}{Carl}        \newcommand{\massC}{\mbox{$90.0\unit{kg}$}}
\newcommand{\studentD}{Diane}       \newcommand{\massD}{\mbox{$80.0\unit{kg}$}}
\newcommand{\studentE}{Erik}        \newcommand{\massE}{\mbox{$95.0\unit{kg}$}}
\newcommand{\studentF}{Frances}       \newcommand{\massF}{\mbox{$85.0\unit{kg}$}}
\newcommand{\studentX}{Xerxes}       \newcommand{\massX}{\mbox{$62.5\unit{kg}$}}
\newcommand{\studentZ}{Zambert}     \newcommand{\massZ}{\mbox{$95.0\unit{kg}$}}
% Male
\newcommand{\heA}{he}\newcommand{\himA}{him}\newcommand{\hisA}{his}\newcommand{\himselfA}{himself}
\newcommand{\HeA}{He}\newcommand{\HimA}{Him}\newcommand{\HisA}{His}
\newcommand{\heC}{he}\newcommand{\himC}{him}\newcommand{\hisC}{his}\newcommand{\himselfC}{himself}
\newcommand{\HeC}{He}\newcommand{\HimC}{Him}\newcommand{\HisC}{His}
\newcommand{\heE}{he}\newcommand{\himE}{him}\newcommand{\hisE}{his}\newcommand{\himselfE}{himself}
\newcommand{\HeE}{He}\newcommand{\HimE}{Him}\newcommand{\HisE}{His}
\newcommand{\heZ}{he}\newcommand{\himZ}{him}\newcommand{\hisZ}{his}\newcommand{\himselfZ}{himself}
\newcommand{\HeZ}{He}\newcommand{\HimZ}{Him}\newcommand{\HisZ}{His}
% Female
\newcommand{\heB}{she}\newcommand{\himB}{her}\newcommand{\hisB}{her}\newcommand{\himselfB}{herself}
\newcommand{\HeB}{She}\newcommand{\HimB}{Her}\newcommand{\HisB}{Her}
\newcommand{\heD}{she}\newcommand{\himD}{her}\newcommand{\hisD}{her}\newcommand{\himselfD}{herself}
\newcommand{\HeD}{She}\newcommand{\HimD}{Her}\newcommand{\HisD}{Her}
\newcommand{\heF}{she}\newcommand{\himF}{her}\newcommand{\hisF}{her}\newcommand{\himselfF}{herself}
\newcommand{\HeF}{She}\newcommand{\HimF}{Her}\newcommand{\HisF}{Her}
%
\newcommand{\heX}{\studentX}\newcommand{\himX}{\studentX}\newcommand{\hisX}{\studentX's}\newcommand{\himselfX}{the person of \studentX}
\newcommand{\HeX}{\studentX}\newcommand{\HimX}{\studentX}\newcommand{\HisX}{\studentX's}
%%%%%%%%%%%%%%%%%%%%%%%%%%%%%%%%%%%%%%%%%%%%%%%%%%%%%%%%%%%%%


%%%%%%%%%%%%%%%%%%%%%%%%%%%%%%%%%%%%%%%%%%%%%%%%%%%%%%%%%%%%
%
% Book macros
%
\newcommand{\aside}[2]{\marginpar{\raggedright \footnotesize\textbf{#1}: #2}}
\newcommand{\important}[1]{\\ \fcolorbox{black}{yellow}{\begin{minipage}[c]{4.925in} \center #1 \end{minipage}}\\}
\newcommand{\inlife}{\marginpar[\scriptsize \raggedright How you might observe $\Rightarrow$ this in your life.]
                               {\scriptsize \raggedleft $\Leftarrow$ How you might observe this in your life.}}
\newcommand{\touchstone}{\marginpar[\scriptsize \raggedright Where have I seen this $\Rightarrow$ before?]
                                   {\scriptsize \raggedleft $\Leftarrow$ Where have I seen this before?}}
\newcommand{\foreshadow}{\marginpar[\scriptsize \raggedright When will I ever use this? $\Rightarrow$]
                                   {\scriptsize \raggedleft $\Leftarrow$ When will I ever use this?}}
\newcommand{\foreshadowR}{\reversemarginpar
                          \marginpar[\scriptsize \raggedright When will I ever use this? $\Rightarrow$]
                                    {\scriptsize \raggedleft $\Leftarrow$ When will I ever use this?}}
\newcommand{\Touchstone}[1]{\marginpar[\scriptsize \raggedright Where have I seen this $\Rightarrow$ \\ before? #1]
                                      {\scriptsize \raggedleft $\Leftarrow$ Where have I seen this before? #1}}
\newcommand{\Foreshadow}[1]{\marginpar[\scriptsize \raggedright When will I ever use this? $\Rightarrow$ \\ #1]
                                      {\scriptsize \raggedleft $\Leftarrow$ When will I ever use this? #1}}
%
%%%%%%%%%%%%%%%%%%%%%%%%%%%%%%%%%%%%%%%%%%%%%%%%%%%%%%%%%%%%


\begin{document}

%\title{Algebra-Based Introductory Physics}
%\author{J Christensen}
%\date{Jan 2017}
%\maketitle
%\pagestyle{cellpage}

\begin{titlepage}
	\centering
%	\includegraphics[width=0.15\textwidth]{example-image-1x1}\par\vspace{1cm}
	{\Huge\bfseries Physics Connected\par}
	\vspace{1cm}
	{\Large\bfseries An Algebra-Based Introductory Physics Textbook\par}
	\vspace{1cm}
	{\large Learn like you think: an interconnected view of physics\par}
	\vspace{2cm}
	{\Large\itshape by: J Christensen\par}
	\vfill
\begin{ForReviewer}
	Version 2.3\par
	{\footnotesize
    \begin{itemize}
    \item Ideas yet to implement:
        \begin{itemize}
        \item The examples are phrased as descriptions, not examples like the homework problems.  Need to consider rephrasing these, not calling them examples, or adding actual examples that better show how to respond to the way homework problems are written.
        \item Define a different page dimension that fits on a cell phone display.  (Enhance possible cell-phone reading.)
        \end{itemize}
    \item version 2.3: June 16-28, 2017
        \begin{itemize}
        \item Updated Section 81. $F=mg$ and Section 8.2 Normal Force
        \item Added specific list of Flame Challenges
        \item Rearranged some of the subsections in the ``Seeing Physics'', added references
        \item Equations of motion for constant acceleration (Need the Story Of)
        \item Added a section to Chap 5 (1-D motion) that gives examples of solutions that require multiple steps  (one equation is insufficient)
        \item Developed the weight and mass discussion and examples
        \item Ladder leaning example in torque, plus some homework problems
        \item Added some Conceptual Homework to weight/mass
        \item Added placeholders to the Gravity chapter
        \item Removed indicators of v1.7 changes
        \end{itemize}
    \item version 2.2: June 16, 2017
        \begin{itemize}
        \item Created conversation about $F=mg$ for Chapter on types of forces.  Caused modifications in lots of places
            \begin{itemize}
            \item Added freefall to the motion chapter
            \item Created IRL and Example dropping objects to see acceleration in $F=mg$, then moved to freefall section -- new Answers to interactive questions
            \item Commented on air resistance
            \item Comments about precision in language (need to do more with precision in mathematics)
            \item Started a couple of ideas about effective theories.  (need to decide where it goes)
            \item Added detail about SI, and specifically the pound-force, pound-mass, and kilogram. to sections \ref{s:SI-MKS} and \ref{ss:weightmass}
            \item Added NIST and BIPM references (found in Wikipedia and then searched further)
            \item conversation about weight and mass.  (required reference to the chapter on Fluids and density)
            \item Moved Google search about significant figures
            \end{itemize}
        \item Added comments about fundamental forces to the section on types of force
        \item Removed indicators of v1.5 and v1.6 changes
        \end{itemize}
    \item version 2.1: June 10, 2017
        \begin{itemize}
        \item Re-commented the $\backslash$new command
        \item Started the chapters on Seeing Physics [\autoref{c:physics}] and Deeper Dive [\autoref{c:revisted}] (These should be renamed)
        \item Moved some sections on fundamental interactions
        \end{itemize}
    \item version 2.0: April 10, 2017
        \begin{itemize}
        \item Re-enabled v1.8 hides
        \item Added a link to \textit{Spacepod}, \textit{Physics Footnotes}, and \textit{Sixty Symbols}
        \item Fixed a $\backslash$dothis that was inside an $\backslash$important, causing a compile error.
        \item Removed indicators of v1.4 changes
        \end{itemize}
    \item version 1.8: April 1, 2017
        \begin{itemize}
        \item Prepare for "the public": "Disabled" the To-Do items, "Hid" the $\backslash$new revision notes, Hid the List of Tables (have none yet)
        \end{itemize}
    \end{itemize}
    }
\end{ForReviewer}
\begin{ForPublic}
{\flushleft
\textbf{Note to the reviewers:}\new{v1.8}{Added the note}
My goal with this book is to create an electronically viewable book that makes use of the advantages of being electronic.  While current e-books have the advantage of being viewable on various devices with having to carry a physical book around, most e-textbooks do not take advantage of hyperlinked text.  With this book I hope to integrate links both forward and backward.  The forward links will be used to motivate curious students.  The backward links will be used to support students who lose track of previous topics.  The integration of these will also provide a convenient opportunity for students to browse through topics they are interested in.
\newpar

At this time, I am providing a single chapter to gauge the viability.  The chapter I am providing is on Newton's Laws.  However, as you read this document, you will find many, many more partially written chapters.  All of the partial chapters and sections are intended to be place-holders for the forward- and backward-links that \autoref{c:force} depends upon.
\newpar

I created this as a PDF that, I believe, can be easily viewed on a computer or tablet.  Since some of my students also seem to read on their phone, I verified that I am also able to view the text in a reasonable manner on my Samsung phone in landscape mode.  In each case, the links should be active and easily manageable.

}
\end{ForPublic}
	\vfill

% Bottom of the page
	{\large \today\par}
\end{titlepage}

\tableofcontents
\newpage
\begin{ForReviewer}
\listoftables
\vfill
\end{ForReviewer}
%\newpage
\listoffigures
\vfill
%\newpage
%\listofstorys
%\newpage
\listofexamples
\vfill
%\newpage
\listofadventures
\vfill
%\newpage
\listofreallifes
\vfill
\newpage

\listoftodos

\newpage



\part{Prerequisites}

<chapter><title>The Story of Science</title>

</chapter><chapter><title></title>Seeing Physics}\label{c-physics}

</chapter><chapter><title></title>Why so much math?}

</chapter><chapter><title></title>Estimating and Uncertainty}

\part{Introducing Motion, Force, and Energy}

</chapter><chapter><title></title>One-Dimensional Motion}\label{c-motion}

</chapter><chapter><title></title>Two-Dimensional Motion}

</chapter><chapter><title></title>Force}\label{c-force}

</section><section><title></title>How Physicists Use the Words (Notation)}\label{s-forcewords}

</section><section><title></title>Connecting the Concepts: Newton's Laws}\label{s-Newton}<aside><title>Referenced by</title> <p>Discussion of <xref provisional=""></xref></p></aside>[how to describe forces]{d-Newtonahead}

    </subsection><subsection><title></title>Translating Newton's First Law: The Law of Inertia}\label{ss-NI}<aside><title>Referenced by</title> <p>Discussion of <xref provisional=""></xref></p></aside>[how to describe forces]{d-Newtonahead}

    </subsection><subsection><title></title>Translating Newton's Second Law: The Equation Law}\label{ss-NII}<aside><title>Referenced by</title> <p></p></aside>{\mmr{<xref ref="" />sss-vectorequations}}, \mmr{<xref ref=""></xref>{d-Newtonahead}{how to describe forces}}, \mmr{<xref ref=""></xref>{d-atrestinmotion}{Newton's first law}}, \mmr{<xref ref="" />d-Fgball}}}

    </subsection><subsection><title></title>Translating Newton's Third Law: Action \& Reaction}\label{ss-NIII}<aside><title>Referenced by</title> <p></p></aside>{\mmr{<xref ref=""></xref>{d-Newtonahead}{how to describe forces}}, \mmr{<xref ref="" />ex-braced}}, \mmr{<xref ref="" />ex-unbraced}}}

</section><section><title></title>Examples} \label{s-NewtonExamples}<aside><title>Referenced by</title> <p><xref provisional="" /></p></aside>{sss-NIItogether}

</section><section><title></title>Summary and Homework}


</chapter><chapter><title></title>The Many Types of Force}\label{c-forcetype}<aside><title>Referenced by</title> <p>Discussion of <xref provisional=""></xref></p></aside>[subscript notation of forces]{d-interaction}

</section><section><title></title>Gravity at the Surface of the Earth}\label{s-Fg}<aside><title>Referenced by</title> <p></p></aside>{\mmr{<xref ref=""></xref>{d-accgrav}{freefall}}, \mmr{<xref ref="" />f-firstFBD}}}<!-- -->\new{v2.2}{Adding detail}

</subsection><subsection><title></title>Weight versus Mass}\label{ss-weightmass}<aside><title>Referenced by</title> <p></p></aside>{\mmr{<xref ref="" />ss-convertunits}}, \mmr{<xref ref="" />s-sigfig}}}<idx></idx>{Weight}<!-- -->\new{v2.2}{Added detail.  Moved the previous version to \protect{<xref ref="" />s-sigfig}} to smooth the transition to \protect{<xref ref="" />ss-equivmm}}.}

</subsection><subsection><title></title>Calculating the weight}\label{ss-local-mg}<!-- -->\new{v2.2}{renamed this section and added detail}


</section><section><title></title>Fundamental Forces}\label{s-fundamental}<idx><h></h><h></h></idx>{Force!Fundamental}<!-- -->\new{v2.1}{Started the section on fundamental interactions.  Link ahead, rather than detailling here.}


</section><section><title></title>Normal Force}\label{s-FN}<aside><title>Referenced by</title> <p></p></aside>{\mmr{<xref ref="" />f-firstFBD}}, \mmr{<xref ref="" text="type-global" />A-floor}}, \mmr{<xref ref="" />s-FT}}}<!-- -->\new{v2.2}{Added detail}<idx><h></h><h></h></idx>{Force!Normal}

</section><section><title></title>Tension}\label{s-FT}<aside><title>Referenced by</title> <p>Discussion of <xref provisional=""></xref></p></aside>[<m>F=ma</m>]{d-fma}<idx><h></h><h></h></idx>{Force!Tension}
</subsection><subsection><title></title>Tension as a Support Force}\label{ss-tension-support}
</subsection><subsection><title></title>Tension as Dragging Force}\label{ss-tension-drag}



% YOU ARE HERE


</subsection><subsection><title></title>Pulleys}

While the flexibility of ropes makes them inconvenient for pushing, their flexibility makes them <em></em>very useful} for changing the direction of the pull.  The mechanism for changing the direction is the pulley.  Furthermore, by allowing us to change the direction of the pull, we are also able to double, triple, or further improve the strength of the pull.  The term for this is <q>the mechanical advantage</q> of a pulley-system.

First we will consider three simple cases of redirecting the force.  In each of these cases, I will <xref ref="" text="title"></xref>[s-effective2]{assume} that the pulley and rope have no mass and that there is no friction in the turning of the pulley (assume it is trivially easy to spin).  If we do not make this assumption, then the problem gets significantly more complicated.<todo></todo>add a reference to the section (problem?) where this is considered.}{}

\begin{minipage}[c]{3.25in}
\begin{sample}
</p></li><li><p>\studentA<idx><h sortby="\studentA">\studentA</h></idx> decides to hold a box that weighs <m>20\unit N</m> using a pulley-system.  What is the tension in the rope?

Since the mass is in equilibrium, the net force is zero and the tension must balance the weight.  This tells us that the tension in the rope is <m>20 \unit N</m>.

If the pulley were difficult to turn (had friction) that stickiness could help support the mass and the tension on \studentA's side might be less than <m>20\unit N</m>; but since we assumed the pulley to be frictionless, \studentA\ must provide the full <m>20\unit N</m> of tension to the rope.
\end{sample}
\end{minipage}
\hfill
\begin{minipage}{1in}
\begin{picture}(100,120)(0,7)
\put(31,105){\oval(36,36)[t]}
\put(31,105){\circle{33}}
\put(31,106){\line(0,1){29}}
\put(49,105){\line(0,-1){62}}
\put(13,105){\line(0,-1){70}}
\put(-30,7){\line(1,0){100}} % floor
\put(0,135){\line(1,0){62}} % ceiling
\multiput(5,135)(10,0){6}{\line(1,1){5}} % immovable
%
\drawbox{-26}{8}{20}{50} %\studentA
\drawbox{-6}{32}{18}{5} %\studentA's arms
\put(-26,60){\scriptsize \studentA}
%
%\drawbox{5}{19}{16}{16}
%\put(6,25){\small <m>m_1</m>}
%
\drawbox{41}{19}{16}{24}
\put(42,25){\small <m>m</m>}
%
%\put(49,-1){\vector(0,1){20}}
%\put(49,-1){\vector(0,-1){20}}
%\put(51,-1){\tiny <m>12\unit m</m>}
\end{picture}
\end{minipage}
%

\noindent
The interesting aspect is that \studentA\ must pull <em></em>down} in order to produce the <em></em>upward} tension on the box.  This means that both \studentA\ and the mass are pulling down.  Since the rope is draped over the pulley, the pulley feels <m>40\unit N</m> downwards, <m>20\unit N</m> from the tension supporting the mass and <m>20\unit N</m> from \studentA\ who is creating the tension that supports the mass.  This means that the second rope that is connecting the pulley to the ceiling must be supporting the full <m>40\unit N</m> in order to keep the pulley in equilibrium.



</subsection><subsection><title></title>Interesting Complications}

</subsubsection><subsubsection><title></title>What is the net force on the rope itself?}
The answer to this depends on how complicated you want the answer to be (recall the discussion about effective theories in <xref ref="" />s-effective2}).  Some reasonable answers are:
<ul>
</p></li><li><p> If the rope (and the attachments) are static, then the net force on the rope must be zero even while it maintains the tension.  It is also possible that the rope is accelerating, in which case the net force on the rope while it transfers the forces between the objects at each end is whatever is necessary to produce the acceleration <m>\vec F_\mathrm{net} = m_\mathrm{rope} \vec a_\mathrm{rope}</m>.
</p></li><li><p> A different answer is to assume that the mass of the rope is small enough that whether it is in equilibrium or accelerating, it does not require a net force and it merely passes its tension through to the object at the other end.
</ul>

</subsubsection><subsubsection><title></title>Multiple Masses}\label{sss-multiple-mass}<aside><title>Referenced by</title> <p><xref provisional="" /></p></aside>{ss-tension-support}

Now that we have a few examples of tension under our belts, we can consider some more interesting examples.

<xref ref="" />ex-multiweight-tension} considers the case of hanging multiple masses, which extends the ideas of <xref ref="" />ss-tension-support}.
%
\begin{example}[hbpt]
\fcolorbox{black}{yellow!10}{\begin{minipage}{4.925in}
\caption{\label{ex-multiweight-tension} How many weights?}
While preparing to hang some ornament on a tree, you chain them from a hook on the wall.  You hang ornament 1 from ornament 2 from ornament 3.  What is the tension in each subsequent string?

\color{blue}
The first thing we should do is notice what information is given to us and make sure that everything is in consistent units.  I will convert everything to <xref ref="" text="title"></xref>[ss-convertunits]{SI units}.

\color{black}
<assemblage>Return to: </assemblage> <xref provisional="" />{sss-multiple-mass}
\end{minipage}}
\end{example}
%
<xref ref="" />ex-multidrag-tension} considers the case of dragging multiple masses, which extends the ideas of <xref ref="" />ss-tension-drag}.
%
\begin{example}[hbpt]
\fcolorbox{black}{yellow!10}{\begin{minipage}{4.925in}
\caption{\label{ex-multidrag-tension} Caravan}
While pulling a sled on which your son sits, your son's sled is tied to a sled on which your dog sits.  Your dog's sled is then connected to a sled with provisions for the day.  What is the tension in each subsequent string?

\color{blue}
The first thing we should do is notice what information is given to us and make sure that everything is in consistent units.  I will convert everything to <xref ref="" text="title"></xref>[ss-convertunits]{SI units}.

\color{black}
<assemblage>Return to: </assemblage> <xref provisional="" />{sss-multiple-mass}
\end{minipage}}
\end{example}
%
You should note that these examples are essentially expressing the same idea in two different contexts.

</subsubsection><subsubsection><title></title>Atwood's Machine}\label{sss-Atwood}

The<todo></todo>imported a homework problem from Giordano.  Need to modify it to fit my purposes.}{} two crates in the figure (p. 114) hang over a pulley (in what is called an <q>Atwood's machine</q>).  I will select <m>m_1=35\unit{kg}</m> (because it looks smaller) and <m>m_2=85\unit{kg}</m> (because it looks bigger).  We will assume that the pulley is massless and frictionless (so that the tension is the same throughout the rope).  Find the acceleration and the time it takes <m>m_2</m> to accelerate down for the <m>12\unit m</m> to the floor.

\begin{minipage}{1in}
\begin{picture}(100,150)(0,-50)
\put(31,80){\oval(36,36)[t]}
\put(31,80){\circle{33}}
\put(31,81){\line(0,1){29}}
\put(49,80){\line(0,-1){62}}
\put(13,80){\line(0,-1){70}}
%
\put(5,-6){\line(0,1){16}}
\put(5,-6){\line(1,0){16}}
\put(21,10){\line(0,-1){16}}
\put(21,10){\line(-1,0){16}}
\put(6,0){\small <m>m_1</m>}
%
\put(41,-6){\line(0,1){24}}
\put(41,-6){\line(1,0){16}}
\put(57,18){\line(0,-1){24}}
\put(57,18){\line(-1,0){16}}
\put(42,0){\small <m>m_2</m>}
%
\put(49,-26){\vector(0,1){20}}
\put(49,-26){\vector(0,-1){20}}
\put(51,-26){\tiny <m>12\unit m</m>}
\end{picture}
\end{minipage}
\hfill
\begin{minipage}{4.5in}
The easy way to do this is to say that <m>m_1</m> pulls down on the left with <m>F_{g1} = (35\unit{kg})(9.81\unitfrac{m}{s^2})=\sig{34}{3.4}{N}</m> and <m>m_2</m> pulls down on the right with <m>F_{g2}=(85\unit{kg})(9.81\unitfrac{m}{s^2})=\sig{83}{3.5}{N}</m> for a difference of <m>F_{net} = \sig{49}{0}{N}</m> down to the right.  Since this has to move both <m>m_1</m> and <m>m_2</m>, the acceleration is
<me> a = \frac{F_{\rm net}}{m_1+m_2} = \frac{\sig{49}{0}{N}}{(35\unit{kg})+(85\unit{kg})} = \frac{\sig{49}{0}{N}}{\sig{120}{}{kg}} = \sigfrac{4.0}{87}{m}{s^2} </me>
This acceleration then causes <m>m_2</m> to drop and the time it takes is found from the equation that include distance and time, \\
<me>y_f \ = \  y_i + v_i \, t + \frac{1}{2} a \,  t^2 </me>
<me> (0\unit m) \ = \ (12\unit m) + (0\unitfrac ms) \, t + \frac{1}{2} (-\sigfrac{4.0}{9}{m}{s^2}) \,  t^2 </me>
\end{minipage}
which we can solve for time:
<me> t \ = \ \sqrt{ \frac{-(12\unit m)}{\frac{1}{2} (-\sigfrac{4.0}{9}{m}{s^2})} } \ = \  \sqrt{ \sig{5.8}{7}{s^2}} \ = \ \sig{2.4}{2}{s} </me>

\footnoterule
\small
However, this does not show what the tension is, and many students make a mistake with the tension.  So, I will also answer the question about the tension. We can draw three free-body diagrams. The equation for <m>m_1</m> is as follows, where I am putting the sign in by
<!-- -->\newpar

\begin{minipage}{4.5in}
hand to indicate the direction: \hfill
<m>\displaystyle (-F_{g1}) + (+F_T) = m_1 (+a) </m> \\
The equation for <m>m_2</m> is as follows: \hfill
<m>\displaystyle (-F_{g2}) + (+F_T) = m_2 (-a) </m> \\
Since we know the weights and the masses, these two equations and two unknowns can be written as
<md>
(-\sig{34}{3}{N}) + (+F_T) & = & (35\unit{kg}) (+a) \\
(-\sig{83}{3}{N}) + (+F_T) & = & (85\unit{kg}) (-a)
</md>
There are many ways to solve two equations and two unknowns.
If we subtract the second equation from the first, then we get the equation on the left.
But, if we solve the first equation for <m>a</m> and plug it into the second, then we get the equation on the right
\end{minipage}
\hfill
\begin{minipage}{1in}
\begin{picture}(100,150)(0,-50)
%\put(31,80){\oval(36,36)[t]}
\put(31,80){\circle{33}}
\put(31,81){\vector(0,1){50}}
\put(47.5,80){\vector(0,-1){30}}
\put(14.5,80){\vector(0,-1){30}}
\put(50,55){\tiny <m>F_T</m>}
\put(15,55){\tiny <m>F_T</m>}
%
\put(5,-6){\line(0,1){16}}
\put(5,-6){\line(1,0){16}}
\put(21,10){\line(0,-1){16}}
\put(21,10){\line(-1,0){16}}
\put(13,4){\vector(0,1){30}}
\put(13,0){\vector(0,-1){20}}
\put(14,20){\tiny <m>F_T</m>}
\put(14,-15){\tiny <m>F_{g1}</m>}
%
\put(49,8){\vector(0,1){30}}
\put(49,4){\vector(0,-1){40}}
\put(41,-6){\line(0,1){24}}
\put(41,-6){\line(1,0){16}}
\put(57,18){\line(0,-1){24}}
\put(57,18){\line(-1,0){16}}
\put(51,30){\tiny <m>F_T</m>}
\put(51,-15){\tiny <m>F_{g1}</m>}
\end{picture}
\end{minipage}

<me> \begin{array}{ccc}
\deq
(-\sig{34}{3}{N}) - (-\sig{83}{3}{N}) \ = \ \left[ (35\unit{kg}) + (85\unit{kg}) \right] (a) &&
\deq
(-\sig{83}{3}{N}) + (F_T) \ = \  - (85\unit{kg}) \left[ \frac{(-\sig{34}{3}{N}) + (F_T)}{(35\unit{kg})} \right] \\
\deq
a \ = \ \frac{\sig{49}{0}{N}}{(35\unit{kg})+(85\unit{kg})} = \sigfrac{4.0}{87}{m}{s^2} &&
\deq
F_T \ = \ \frac{-(35\unit{kg})(-\sig{83}{3}{N})-(85\unit{kg})(-\sig{34}{3}{N})}{[(35\unit{kg})+(85\unit{kg})]} \ = \ \sig{48}{6}{N}
\end{array} </me>
The acceleration is as above.  The tension is not enough to support <m>m_2</m> (so it falls) and more than enough to lift <m>m_1</m> (so it rises).
You should note that
<m>\left[(\sig{48}{6}{N}-\sig{34}{3}{N})/(35\unit{kg})=\sigfrac{4.0}{9}{m}{s^2}\right]</m>
\hfill and \hfill
<m>\left[(\sig{83}{3}{N}-\sig{48}{6}{N})/(85\unit{kg})=\sigfrac{4.0}{9}{m}{s^2}\right]</m>.

\normalsize

</subsubsection><subsubsection><title></title>Surface Tension}

As a <p xml:id=""></p>d-surf-tension}{final note}, <xref ref="" text="title"></xref>[s-surface-tension]{surface tension} is something else entirely.  See <xref ref="" />sss-tea} for a comment on the contribution to hot versus cold spoon noises.

</section><section><title></title>Frictional Force}\label{s-Ff}<aside><title>Referenced by</title> <p></p></aside>{\mmr{<xref ref="" text="type-global" />A-chair2}}, \mmr{<xref ref="" text="type-global" />A-chair6}}, \mmr{<xref ref="" text="type-global" />A-chair7}}, \mmr{<xref ref="" />A-fly.balls}}}

%
\begin{reallife}[bthp]
\hspace{-.2in}
\fcolorbox{black}{green!10}{\begin{minipage}{5.29in} \center
\caption{\label{irl-poolfriction}<idx><h></h><h></h></idx>{Pool!Real Life} Rolling pool balls and friction.}
\begin{minipage}{4.925in}
\studentD<idx><h sortby="\studentD">\studentD</h></idx> is relaxing with the local physics club, playing pool.  \HeD\ hits the cue ball and counts the number of walls \heD\ can hit in one shot.
\end{minipage}
\begin{realtable}
\dna{Hit the cue-ball off of a bumper in the manner intended for
\protect{<url href=""></url>{http://c.ymcdn.com/sites/bca-pool.com/resource/resmgr/imported/BCAEquipmentSpecifications_2008.pdf}{testing cushions}}.}
    {Compare the strength of the hit to the distance travelled}
    {How much is the total distance affected by the number of bumpers hit? \\
     Does it matter if you shoot along the length of the table versus the width of the table?  \\
     Why does friction slow the ball down instead of just make it turn <m>v=\omega r</m> (no slip)}
\end{realtable}
\begin{minipage}{4.925in}
Billiard tables have a lot of interesting physics, which can help us see a wide variety of physics, for example:
<xref ref="" text="title"></xref>[irl-poolnormal]{normal force}, <xref ref="" text="title"></xref>[irl-poolelastic]{elastic versus inelastic collisions}, <xref ref="" text="title"></xref>[irl-poolrotmot]{rotational motion}, and <xref ref="" text="title"></xref>[irl-poolangmom]{angular momentum}.
\end{minipage}

%\flushright
%<assemblage>Return to: </assemblage> <xref provisional=""></xref>[pool]{d-bank-shot}
\end{minipage}}
\end{reallife}
%

</section><section><title></title>Spring Force}\label{s-springs}<aside><title>Referenced by</title> <p></p></aside>{\mmr{<xref ref=""></xref>{d-fma}{<m>F=ma</m>}}, \mmr{<xref ref=""></xref>{d-usesofFma}{uses of <m>F=ma</m>}}, \mmr{<xref ref="" />ss-scales}}}

</section><section xml:id="s-FA"><title></title>Applied Force}

The term <q>an applied force</q> is used to describe any force applied by any object when there isn't really a formula to find it.  So this is kind of a <q>any other force</q> category.  I will use this type of force to describe forces exerted by people.  We have seen some examples where a person throws an object.  We can now revisit those examples and consider the force exerted (applied) by the person who threw the object.
\begin{sample}
</p></li><li><p>\label{se-throw-up} \studentC<idx><h sortby="\studentC">\studentC</h></idx> recalls that one time \heC\ got bored one day in physics class (what?!?) and tossed a baseball (<m>m_b = 0.145\unit{kg}</m>) at the ceiling<ellipsis /> a little too hard <ellipsis /> as recounted in <xref ref="" />ex-ceiling}.  Recall that <xref ref="" text="type-global" />se-ceiling} found the normal force by the ceiling on the ball.  Please now find the force \studentC\ applied while throwing and catching the ball assuming that the throw took <m>0.200\unit{s}</m> to gain the speed of <m>5.00\unitfrac ms</m> and the catch took <m>0.250\unit s</m> to slow the ball from <m>4.73\unitfrac ms</m> to rest.

There are five stages to the motion: (a) throwing, (b) falling up, (c) hitting the ceiling, (d) falling down, and (e) catching show the forces involved. \\
\fbox{\begin{minipage}[b]{55pt}
\begin{picture}(50,100)(0,0)
\put(25,25){\circle{10}}
\put(25,26){\vector(0,1){25}}
\put(25,24){\vector(0,-1){15}}
\put(28,35){<m>F_\mathrm{throw}</m>}
\put(28,10){<m>F_g</m>}
\end{picture}
\centering{(a) throwing}
\end{minipage}}
\hfill
\color{lightgray}
\fbox{\begin{minipage}[b]{55pt}
\begin{picture}(50,100)(0,0)
\put(25,50){\circle{10}}
\put(25,50){\vector(0,-1){15}}
\put(28,35){<m>F_g</m>}
\end{picture}
\centering{(b) falling up}
\end{minipage}}
\hfill
\fbox{\begin{minipage}[b]{55pt}
\begin{picture}(50,100)(0,0)
\put(25,95){\circle{10}}
\put(26,95){\vector(0,-1){25}}
\put(24,95){\vector(0,-1){15}}
\put(28,75){<m>F_N</m>}
\put(10,75){<m>F_g</m>}
\end{picture}
\centering{(c) \\ hitting}
\end{minipage}}
\hfill
\fbox{\begin{minipage}[b]{55pt}
\begin{picture}(50,100)(0,0)
\put(25,50){\circle{10}}
\put(25,50){\vector(0,-1){15}}
\put(28,35){<m>F_g</m>}
\end{picture}
\centering{(d) falling down}
\end{minipage}}
\hfill
\color{rgb:red,0;green,2;blue,1}
\fbox{\begin{minipage}[b]{55pt}
\begin{picture}(50,100)(0,0)
\put(25,25){\circle{10}}
\put(25,26){\vector(0,1){25}}
\put(25,24){\vector(0,-1){15}}
\put(28,35){<m>F_\mathrm{catch}</m>}
\put(28,10){<m>F_g</m>}
\end{picture}
\centering{(e) catching}
\end{minipage}}
\\
In this particular problem, we are only concerned with steps (a) and (e) because that's where \studentC\ throws and catches the ball. In each case, we need the acceleration: \\
\begin{minipage}[b]{150pt}
<md>
\vec a_\mathrm{throw} & = & \frac{(+5.00\unitfrac ms \jhat)-(0\unitfrac ms \jhat)}{0.200\unit s} \\
& = & +\sigfrac{25.0}{0}{m}{s^2} \jhat
</md>
\end{minipage}
\hfill
\begin{minipage}[b]{150pt}
<md>
\vec a_\mathrm{catch} & = & \frac{(0\unitfrac ms \jhat)-(-4.73\unitfrac ms \jhat)}{0.250\unit s} \\
& = & +\sigfrac{18.9}{2}{m}{s^2} \jhat
</md>
\end{minipage}

During each step, we have the actual acceleration, which tells us about the net force.  We will also need to know the weight of the baseball <m>F_g=\sig{1.42}{2}{N}</m>, because gravity is still acting during the collision.  Let's consider the throwing part first.
<md>
\vec F_N + \vec F_g & = &  \vec F_\mathrm{net} \ = \ m \vec a \\
\vec F_A  & = &  m \vec a - \vec F_g \\
\vec F_A  & = &  \left[ (0.145\unit{kg})(+\sigfrac{25.0}{0}{m}{s^2}\jhat) \right] - \left[  - \sig{1.42}{2}{N} \jhat \right] \\
\vec F_A  & = &  \left[ +\sig{3.62}{5}{N} \jhat \right] - \left[  - \sig{1.42}{2}{N} \jhat \right] \ = \ +\sig{5.04}{7}{N} \jhat
</md>
You can see that the upward applied force <m>(\sig{5.04}{7}{N})</m> has to be large enough so that when it is combined with the downward gravitational force <m>(\sig{1.42}{2}{N})</m> they can together result in the necessary (but smaller) upward net force <m>(\sig{3.62}{5}{N})</m> to get it going upwards.

For the catching part, the ball is moving downwards and needs to be stopped, so the catching applied force must be upwards.
<md>
\vec F_A + \vec F_g & = &  \vec F_\mathrm{net} \ = \ m \vec a \\
\vec F_A  & = &  m \vec a - \vec F_g \\
\vec F_A  & = &  \left[ (0.145\unit{kg})(+\sigfrac{18.9}{2}{m}{s^2}\jhat) \right] - \left[  - \sig{1.42}{2}{N} \jhat \right] \\
\vec F_A  & = &  \left[ +\sig{2.74}{3}{N} \jhat \right] - \left[  - \sig{1.42}{2}{N} \jhat \right] \ = \ +\sig{4.16}{5}{N} \jhat
</md>
You can see that the upward applied force <m>(\sig{4.16}{5}{N})</m> has to be large enough so that when it is combined with the downward gravitational force <m>(\sig{1.42}{2}{N})</m> they can together result in the necessary upward net force <m>(\sig{2.74}{3}{N})</m> to stop it from continuing downwards.
\end{sample}

</section><section><title></title>Putting it Together, <m>F_\mathrm{net}</m>}\label{s-Fnet}

</subsection><subsection><title></title>Translational Equilibrium}

blah blah blah
\phantomsection\label{ss-transeq} Translational equilibrium: <m>F_\mathrm{net} = m \cancel{0}{a}</m>.  blah blah blah

</subsection><subsection><title></title>Static Equilibrium}

</subsection><subsection><title></title>Dynamic Equilibrium}


</section><section><title></title>Summary and Homework}

</subsection><subsection><title></title>Summary of Concepts and Equations}<!-- -->\new{v2.3}{Created this section}

<ellipsis />

</subsection><subsection><title></title>Conceptual Questions}<!-- -->\new{v2.3}{Added two conceptual problems.}
%\vspace{-24pt}
<ol>
</p></li><li><p>\label{c-weightmass} Estimate, preferably without using the internet, the mass of the following: (a) a four-door sedan, (b) dishwasher, (c) a pair of glasses, (d) a cell phone.  You should be able to estimate to within one significant digit.
</p></li><li><p>\label{c-massweight} List at least one object, preferably without using the internet, that has the following mass: (a) <m>2500\unit{kg}</m> (b) <m>41\unit{kg}</m>, (c) <m>3\unit{kg}</m>, (d) <m>50\unit{g}</m>.
</ol>
</subsection><subsection><title></title>Problems}<!-- -->\new{v2.3}{Created section.}<todo></todo>Add more problems.}
%\vspace{-24pt}
<ol>
 </p></li><li><p><ellipsis />
</ol>


</chapter><chapter><title></title>Energy and the Transfer of Energy}

<p xml:id=""></p>d-energynoun}{Energy is a noun}<idx><h></h><h></h></idx>{Energy!noun}; objects can <em></em>have} energy.  <p xml:id=""></p>d-workverb}{Work is a verb}<idx><h></h><h></h></idx>{Work!verb}<aside><title>Referenced by</title> <p>Discussion of <xref provisional=""></xref></p></aside>[heat as a verb]{d-heatverb}; doing work is the process of <em></em>exchanging} energy.

</section><section><title></title>Objects Can Have Energy}

</section><section><title></title>A Force Can Transfer Energy} \label{s-work}<aside><title>Referenced by</title> <p>Discussion of <xref provisional=""></xref></p></aside>[the direction of forces]{d-pushvector}

</section><section><title></title>Dissipating Energy} \label{s-Wfr}

pool balls on cushion/bumper

</section><section><title></title>Conserving Energy} \label{s-PE}

%
\begin{reallife}[bthp]
\hspace{-.2in}
\fcolorbox{black}{green!10}{\begin{minipage}{5.29in} \center
\caption{\label{irl-poolelastic}<idx><h></h><h></h></idx>{Pool!Real Life} 1-D elastic collisions of pool balls.  inelastic collisions off the bumper.}
\begin{minipage}{4.925in}
\studentD<idx><h sortby="\studentD">\studentD</h></idx> is relaxing with the local physics club, playing pool.  \HeD\ hits the cue ball and counts the number of walls \heD\ can hit in one shot.
\end{minipage}
\begin{realtable}
\dna{collide balls.}
    {where does it hit}
    {<m>90^\circ</m> output}
\end{realtable}
\begin{minipage}{4.925in}
Billiard tables have a lot of interesting physics, which can help us see a wide variety of physics, for example:
<xref ref="" text="title"></xref>[irl-poolnormal]{normal force}, <xref ref="" text="title"></xref>[irl-poolelastic]{elastic versus inelastic collisions}, <xref ref="" text="title"></xref>[irl-poolrotmot]{rotational motion}, and <xref ref="" text="title"></xref>[irl-poolangmom]{angular momentum}.
\end{minipage}

%\flushright
%<assemblage>Return to: </assemblage> <xref provisional=""></xref>[pool]{d-bank-shot}
\end{minipage}}
\end{reallife}
%

</subsection><subsection><title></title>Gravitational Potential Energy}\label{ss-PEg}<aside><title>Referenced by</title> <p><xref provisional="" /></p></aside>{s-PEG}
See also <xref ref="" text="type-global" />s-PEG}
</subsection><subsection><title></title>Spring Potential Energy}\label{ss-PEs}
</subsection><subsection><title></title>Conservative Forces in General}

\part{Interesting Uses of Motion, Force, and Energy}

</chapter><chapter><title></title>Momentum: A Better Way to Describe Force}\label{c-momentum}<aside><title>Referenced by</title> <p></p></aside>{\mmr{<xref ref=""></xref>{d-objectinmotion}{objects in motion}}, \mmr{<xref ref="" />sss-inertia}}, \mmr{<xref ref="" />ss-NIII}}, \mmr{<xref ref="" text="type-global" />A-chair6}}}

Useful to include?
<url href=""></url>{https://www.wired.com/2017/06/physics-bullets-versus-wonder-womans-bracelets/}{The Physics of Bullets Vs. Wonder Woman's Bracelets}

</section><section><title></title>Revising Newton's First and Second Laws}

</subsection><subsection><title></title>Inertia and Momentum}\label{ss-inertia}<aside><title>Referenced by</title> <p><xref provisional="" /></p></aside>{sss-inertia}
Recall <xref ref="" />sss-inertia}.

</section><section><title></title>Revising Newton's Third Law: Conservation of Momentum}\label{s-conservemom}<aside><title>Referenced by</title> <p><xref provisional="" /></p></aside>{ss-NIII}

</section><section><title></title>Two-Dimensional Collisions}\label{s-2Dcollisions}<aside><title>Referenced by</title> <p><xref provisional="" /></p></aside>{sss-vectorequations}

pool balls?  What about rolling?
%
\begin{reallife}[bthp]
\hspace{-.2in}
\fcolorbox{black}{green!10}{\begin{minipage}{5.29in} \center
\caption{\label{irl-pool2Dcollision}<idx><h></h><h></h></idx>{Pool!Real Life} 2-D collisions of pool balls.}
\begin{minipage}{4.925in}
\studentD<idx><h sortby="\studentD">\studentD</h></idx> is relaxing with the local physics club, playing pool.  \HeD\ hits the cue ball and counts the number of walls \heD\ can hit in one shot.
\end{minipage}
\begin{realtable}
\dna{collide balls.}
    {where does it hit}
    {<m>90^\circ</m> output}
\end{realtable}
\begin{minipage}{4.925in}
Billiard tables have a lot of interesting physics, which can help us see a wide variety of physics, for example:
<xref ref="" text="title"></xref>[irl-poolnormal]{normal force}, <xref ref="" text="title"></xref>[irl-poolelastic]{elastic versus inelastic collisions}, <xref ref="" text="title"></xref>[irl-poolrotmot]{rotational motion}, and <xref ref="" text="title"></xref>[irl-poolangmom]{angular momentum}.
\end{minipage}

%\flushright
%<assemblage>Return to: </assemblage> <xref provisional=""></xref>[pool]{d-bank-shot}
\end{minipage}}
\end{reallife}
%


</chapter><chapter><title></title>Rotational Motion}

</section><section><title></title>The Equations of Rotational Motion}

%
\begin{reallife}[bthp]
\hspace{-.2in}
\fcolorbox{black}{green!10}{\begin{minipage}{5.29in} \center
\caption{\label{irl-poolrotmot}<idx><h></h><h></h></idx>{Pool!Real Life} Rolling pool balls.}
\begin{minipage}{4.925in}
\studentD<idx><h sortby="\studentD">\studentD</h></idx> is relaxing with the local physics club, playing pool.  \HeD\ hits the cue ball and counts the number of walls \heD\ can hit in one shot.
\end{minipage}
\begin{realtable}
\dna{Roll a striped ball along the table.}
    {Use the stripe to notice the rate of rotation}
    {How does the rotation compare to the translation?}
\dna{Roll a striped ball along the table.}
    {Notice the distance the ball travels}
    {Why does friction slow the ball down instead of just make it turn <m>v=\omega r</m> (no slip)}
\end{realtable}
\begin{minipage}{4.925in}
Billiard tables have a lot of interesting physics, which can help us see a wide variety of physics, for example:
<xref ref="" text="title"></xref>[irl-poolnormal]{normal force}, <xref ref="" text="title"></xref>[irl-poolelastic]{elastic versus inelastic collisions}, <xref ref="" text="title"></xref>[irl-poolrotmot]{rotational motion}, and <xref ref="" text="title"></xref>[irl-poolangmom]{angular momentum}.
\end{minipage}

%\flushright
%<assemblage>Return to: </assemblage> <xref provisional=""></xref>[pool]{d-bank-shot}
\end{minipage}}
\end{reallife}
%

</section><section><title></title>Angular Momentum}

%
\begin{reallife}[bthp]
\hspace{-.2in}
\fcolorbox{black}{green!10}{\begin{minipage}{5.29in} \center
\caption{\label{irl-poolangmom}<idx><h></h><h></h></idx>{Pool!Real Life} Rolling pool balls.}
\begin{minipage}{4.925in}
\studentD<idx><h sortby="\studentD">\studentD</h></idx> is relaxing with the local physics club, playing pool.  \HeD\ hits the cue ball and counts the number of walls \heD\ can hit in one shot.
\end{minipage}
\begin{realtable}
\dna{Roll a striped ball along the table.}
    {Use the stripe to notice the rate of rotation}
    {How does the rotation compare to the translation?}
\end{realtable}
\begin{minipage}{4.925in}
Billiard tables have a lot of interesting physics, which can help us see a wide variety of physics, for example:
<xref ref="" text="title"></xref>[irl-poolnormal]{normal force}, <xref ref="" text="title"></xref>[irl-poolelastic]{elastic versus inelastic collisions}, <xref ref="" text="title"></xref>[irl-poolrotmot]{rotational motion}, and <xref ref="" text="title"></xref>[irl-poolangmom]{angular momentum}.
\end{minipage}

%\flushright
%<assemblage>Return to: </assemblage> <xref provisional=""></xref>[pool]{d-bank-shot}
\end{minipage}}
\end{reallife}
%


</section><section><title></title>Non-inertial Rotational Reference Frames} \label{s-noninertial}<aside><title>Referenced by</title> <p></p></aside>{\mmr{<xref ref="" />ss-noninertial}}, \mmr{<xref ref=""></xref>{d-NewtonInertial}{non-inertial reference frames}}, \mmr{<xref ref="" />ss-NI}}}
<idx><h></h><h></h></idx>{Reference Frames!Inertial}
<idx><h></h><h></h></idx>{Reference Frames!Non-inertial}

Because the Earth <p xml:id=""></p>d-noninertial}{rotates}<aside><title>Referenced by</title> <p><xref provisional="" /></p></aside>{ss-NII}, we are actually in a non-inertial reference frame.  In fact, we can prove that the Earth rotates by observing the effects, such as the <xref ref=""></xref>{d-coriolis}{Coriolis effect}, that in our non-inertial frame seem to require unexplainable forces but which, in a non-rotating frame, follow the expected laws of physics.

</subsection><subsection><title></title>The Coriolis Effect}\label{ss-coriolis}<aside><title>Referenced by</title> <p></p></aside>{\mmr{<xref ref=""></xref>{d-NewtonInertial}{non-inertial reference frames}}, \mmr{<xref ref=""></xref>{d-noninertial}{Non-inertial Rotational Reference Frames}}}

<p xml:id=""></p>d-coriolis}{weather, etc}
<!-- -->\newpar

In her podcast<!-- -->\new{v2.0}{<em></em>Spacepod}}, <em></em>Spacepod}<fn xml:id=""></fn>{Nugent, Carrie (Producer, Host). <em></em>Spacepod} [Audio podcast], episode 89 (19 May, 2017).  Retrieved from <xref ref="" text="title"></xref>{http://spacepod.libsyn.com/}{T4LTFdOxHD5WWzdD}{99}{\nolinkurl{http://spacepod.libsyn.com/}}
on 9 Apr. 2017.} Dr. Carrie Nugent interviews Dr. Andy Thompson about <q>underwater flying objects</q> that investigate the ocean.  He notes that ocean waters, because they are such a large-scale system, can see the effect of the rotation of the Earth.

</subsection><subsection><title></title>The Foucault Pendulum}\label{ss-Foucault}

See <url href=""></url>{https://www.youtube.com/watch?v=sWDi-Xk3rgw}{youtube video} by <url href=""></url>{http://sixtysymbols.com/}{Sixty Symbols}.<!-- -->\new{v2.0}{Foucault video}




</chapter><chapter><title></title>Circular Motion and Centripetal Force}

</section><section><title></title>Circular Motion}
</section><section><title></title>Centripetal Force}\label{s-centripetal}<aside><title>Referenced by</title> <p>Discussion of <xref provisional=""></xref></p></aside>[<m>F=ma</m>]{d-fma}




</chapter><chapter><title></title>Torque and the <m>F=ma</m> of Rotations}\label{c-torque}<aside><title>Referenced by</title> <p><xref provisional="" /></p></aside>{a-NIIIaction}<!-- -->\new{v2.3}{Added an example that is computable here, but helps introduce normal force in \protect{<xref ref="" />s-FN}}.}

</section><section><title></title>Leverage}\label{s-leverarm}<aside><title>Referenced by</title> <p><xref provisional="" /></p></aside>{ss-scales}

</section><section><title></title>Putting it all together, <m>\tau_\mathrm{net}</m>}

</subsection><subsection><title></title>Rotational Equilibrium}

blah blah blah
\phantomsection\label{ss-roteq} Rotational equilibrium: <m>\tau_\mathrm{net} = I \cancelto{0}{\alpha}</m>.  blah blah blah

</subsection><subsection><title></title>Static (Rotational) Equilibrium}

</subsection><subsection><title></title>Dynamic (Rotational) Equilibrium}

<!-- -->\new{v2.3}{Answered \protect{<xref ref="" />ex-ladder2}} and its related problems.}
\begin{example}[p]
\fcolorbox{black}{yellow!10}{\begin{minipage}{4.925in}\setlength{\parskip}{3pt}
\caption{\label{ex-ladder2} \studentC<idx><h sortby="\studentC">\studentC</h></idx> uses a ladder}
\begin{quote}
\studentC\ leans a <m>22.7\unit{kg}</m> ladder against a wall at an angle of <m>75.5^\circ</m>, consistent with \protect{<url href=""></url>{https://www.osha.gov/}{OSHA}} standard \protect{<url href=""></url>{https://www.osha.gov/pls/oshaweb/owadisp.show_document?p_table=standards&p_id=10839}{1926.1053(a)(1)(ii)}}.
The coefficient of friction between the ladder and the floor is <m>\mu_f=0.31</m>.
The coefficient of friction between the ladder and the wall is <m>\mu_w=0.19</m>.
Use the rotational and translational equilibrium to determine if the ladder slides.
\end{quote}

Since we are asked to distinguish between two cases that cannot both be true, we should assume one (the easier one to calculate is that the ladder does not slip) and then verify that the result is consistent with that assumption.

<em></em>What do we know?}
We know that the floor has a normal force <m>(F_{Nf})</m> upwards and a frictional force <m>(F_{ff})</m> to the left.
We know that the wall
\\[2pt]
\begin{minipage}{3.2in}
has a normal force <m>(F_{Nw})</m> to the right and a frictional force <m>(F_{fw})</m> up (keeping the ladder from sliding down).
We know the weight is
<m> F_g = mg = (22.7\unit{kg})(9.81\unitfrac{m}{s^2}) = \sig{222}{.69}{N} </m>
<em></em>What do we want to know?}  We want to know about the the magnitudes of both normal
\end{minipage}
\hfill
\begin{minipage}{100pt}
\begin{picture}(100,100)(-10,-5)
% Dimensions and offset: (width,height)(x offset,y offset)
% Insert picture commands (\line,\circle, etc...) here:
\put(0,0){\line(0,1){100}}
\put(0,0){\line(1,0){75}}
\put(20,0){\color{blue}\line(-1,4){20}}     % ladder
\put(10,40){\color{red}\vector(0,-1){30}}   % Fg
\put(20,1){\color{red}\vector(0,1){25}}     % FNf
\put(19,1){\color{red}\vector(-1,0){12}}
\put(1,80){\color{red}\vector(1,0){12}}
\put(1,81){\color{red}\vector(0,1){8}}
\end{picture}
\end{minipage}
%\hfill {}
\\[3pt]
forces and both frictional forces.
Can we easily deduce the magnitude of <m>F_{Nf}</m>? <xref ref="" text="type-global" />A-ladderNf}.

<em></em>How are these related?}  The forces acting on any body are related by static <xref ref="" text="title"></xref>[ss-transeq]{translational equilibrium}
<md>
x: \hspace{.5cm} 0 & = & \cancel{0}{F_{gx}} + \zero{F_{Nfx}}{0} + F_{ffx} + F_{Nwx} + \cancelto{0}{F_{fwx}} \\
y: \hspace{.5cm} 0 & = & F_{gy} + F_{Nfy} + \cancelto{0}{F_{ffy}} + \cancelto{0}{F_{Nwy}} + F_{fwy}
</md>
and static <xref ref="" text="title"></xref>[ss-roteq]{rotational equilibrium}, assuming the pivot point is at the ground, and using the relationship <m>F_f=\mu F_N</m>, we find
<md>
0 & = & \tau_{g} + \cancelto{0}{\tau_{Nf}} + \cancelto{0}{\tau_{ff}} + \tau_{Nw} + \tau_{fw} \\
0 & = & \left[ F_g \frac{l}{2} \sin 14.5^\circ \right] + \left[ - F_{Nw} l \sin(75.5^\circ) \right] + \left[ - F_{fw} l \sin(14.5^\circ) \right] \\
F_{Nw} & = & \left[ F_g \frac{l}{2} \sin 14.5^\circ \right] / \left[  l \sin(75.5^\circ) + \mu_w l \sin(14.5^\circ) \right]
</md>
%<em></em>Free-Body Diagrams:}  Since the picture is so simple, we will not draw the free-body diagram.


{}\hfill {\footnotesize\autoref*{ex-ladder2} continued on next page<ellipsis />}
\end{minipage}}
\end{example}
\begin{example}[p]
\fcolorbox{black}{yellow!10}{\begin{minipage}{4.925in}\setlength{\parskip}{3pt}
{\footnotesize \autoref*{ex-ladder2} continued from previous page<ellipsis />}

<em></em>Concepts to Consider:}  First, the length of the ladder cancels from the expression; what matters is the angle at which it is propped.

Second, every force value will be linearly dependent on the mass of the ladder.  So once we solve this problem, we can easily scale the answers to any mass.

Third, the friction with the wall is, by far, the smallest effect and it might be interesting to approximate all of this with <m>\mu_w=0</m>.  You can check your calculation against <xref ref="" text="type-global" />A-nowall}.

<em></em>Solution to the example:}  When we worry about significant figures,
<md>
F_{Nw} & = & \frac{\left[ (\sig{222}{.7}{N})(\txtfrac{1}{2}) (0.\sig{250}{4}{}) \right]}{\left[  (0.\sig{968}{2}{}) + (0.19) (0.\sig{250}{4}{}) \right]}
\ = \ \frac{\left[ (\sig{27.8}{8}{N})\right]}{\left[  (0.\sig{968}{2}{}) + (0.0\sig{47}{6}{}) \right]} \\
F_{Nw} & = & \frac{\left[ (\sig{27.8}{8}{N})\right]}{\left[  (\sig{1.015}{7}{}) \right]}
\ = \ \sig{27.4}{4}{N} \\
F_{fw,\mathrm{max}} & = & (0.19)(\sig{27.4}{4}{N}) \ = \ \sig{5.2}{15}{N} \\
F_{Nf} & = & F_g - F_{fw} = (\sig{222}{.7}{N})-(\sig{5.2}{15}{N}) \ = \ \sig{217}{.5}{N} \\
F_{ff,\mathrm{max}} & = & (0.31)(\sig{217}{.5}{N}) \ = \ \sig{672}{.4}{N}
</md>
Since <m>F_{ff} >F_{Nw}</m>, the friction is sufficient to hold the ladder in place, as assumed.

%\begin{quote}
<em></em>Aside:} Since <m>F_{ff}</m> only needs to be <m>\sig{27.4}{4}{N}</m> to hold the ladder in place, it is possible for the ladder to not slide on a floor that only has
<m>\mu_\mathrm{min} = (\sig{27.4}{4}{N})/(\sig{217}{.5}{N}) = 0.\sig{126}{2}{}</m>; but that would not allow a person to climb the ladder.

<em></em>Homework:} Homework problem~<xref ref="" text="type-global" />h-ladderC} asks you to determine if the ladder slides when \studentC\ climbs to different locations on the ladder.
%\end{quote}
\flushright
<assemblage>Return to: </assemblage>{\mmr{<xref ref="" text="type-global" />se-ladderN}}, \mmr{<xref ref="" />ss-roteq}}}
\end{minipage}}
\end{example}

</section><section><title></title>Torsion}\label{s-torsion}<aside><title>Referenced by</title> <p><xref provisional="" /></p></aside>{s-FT}<!-- -->\new{v2.4}{Created this section}

</section><section><title></title>Summary and Homework}

</subsection><subsection><title></title>Summary of Concepts and Equations}<!-- -->\new{v2.3}{Created this section}

<ellipsis />

</subsection><subsection><title></title>Conceptual Questions}<todo></todo>Add conceptual problems.}
%\vspace{-24pt}
<ol>
</p></li><li><p><ellipsis />
</ol>
</subsection><subsection><title></title>Problems}<!-- -->\new{v2.3}{Added problems.}<todo></todo>Add more problems.}
%\vspace{-24pt}
<ol>
 </p></li><li><p>\label{h-ladderC} \studentC\ leans a <m>22.7\unit{kg}</m> ladder against a wall at an angle of <m>75.5^\circ</m>, consistent with \protect{<url href=""></url>{https://www.osha.gov/}{OSHA}} standard \protect{<url href=""></url>{https://www.osha.gov/pls/oshaweb/owadisp.show_document?p_table=standards&p_id=10839}{1926.1053(a)(1)(ii)}}.<!-- -->\new{v2.3}{Answered \protect{<xref ref="" text="type-global" />h-ladderC}} and its related problems.}
The coefficient of friction between the ladder and the floor is <m>\mu_f=0.31</m>.
The coefficient of friction between the ladder and the wall is <m>\mu_w=0.19</m>.
Use the rotational and translational equilibrium to determine if the ladder slides when \studentC\ (<m>\massC</m>) climbs to
<ol>
</p></li><li><p> the third-rung from the top of the ladder, so that he is <m>1.53\unit m</m> from the bottom of the ladder.
    (See <xref ref="" text="type-global" />A-nowallC} for that answers if <m>\mu_w = 0</m>.)
\begin{ForMe}
\color{blue} Answers:
<md>
F_{Nw} & = & \sig{163}{.9}{N} \\
F_{fw,\mathrm{max}} & = & (0.19)(\sig{163}{.9}{N}) \ = \ \sig{31}{.14}{N} \\
F_{Nf} & = & \sig{1074}{.4}{N} \\
F_{ff,\mathrm{max}} & = & \sig{333}{.0}{N} < \sig{163}{.9}{N}
</md>
<m>\mu_\mathrm{min} = 0.\sig{152}{56}{}</m>
\color{black}
\end{ForMe}
</p></li><li><p> the third-rung from the bottom of the ladder, so that he is <m>0.914\unit m</m> from the bottom of the ladder.
\begin{ForMe}
\color{blue}
Answers:
<md>
F_{Nw} & = & \sig{108}{.97}{N} \\
F_{fw,\mathrm{max}} & = & (0.19)(\sig{108}{.97}{N}) \ = \ \sig{20}{.70}{N} \\
F_{Nf} & = & \sig{1084}{.9}{N} \\
F_{ff,\mathrm{max}} & = & \sig{336}{.3}{N} < \sig{108}{.97}{N}
</md>
<m>\mu_\mathrm{min} = 0.\sig{100}{45}{}</m>

If <m>\mu_w = 0</m>.
<md>
F_{Nw} & = & \sig{114}{.3}{N} \\
F_{fw,\mathrm{max}} & = & 0 \unit N \\
F_{Nf} & = & \sig{1105}{.6}{N} \\
F_{ff,\mathrm{max}} & = & \sig{342}{.7}{N} < \sig{114}{.3}{N}
</md>
<m>\mu_\mathrm{min} = 0.\sig{103}{4}{}</m>
\color{black}
\end{ForMe}
</ol>
</ol>


</chapter><chapter><title></title>Energy of Rotating Objects}
</section><section><title></title>Rotational Kinetic Energy}
pool balls

</chapter><chapter><title></title>The Gravitational Force on a Large Scale}\label{c-gravity}<aside><title>Referenced by</title> <p></p></aside>{\mmr{<xref ref=""></xref>{d-accgrav}{freefall}}, \mmr{<xref ref=""></xref>{d-fundamental}{fundamental forces}}}

</section><section><title></title>Gravitational Force and Field}\label{s-Gfield}<aside><title>Referenced by</title> <p>Discussion of <xref provisional=""></xref></p></aside>[<m>F=ma</m>]{d-fma}<!-- -->\new{v2.3}{Added some placeholders}

The value of the acceleration due to gravity  varies according to the mass and size of any celestial body.<todo></todo>Reference a table of <m>g</m> on other planets and compute the weight of a space craft at each planet.}
This means that, as was seen in <xref ref="" text="type-global" />se-gworld}, your weight can change even when your mass remains the same.
\begin{sample}
</p></li><li><p>\label{se-gplanets} In conversation with a visiting alien, \studentX<idx><h sortby="\studentX">\studentX</h></idx>, you find that \studentX\ has been to the moon and several planets both within and outside of our solar system.  In addition to the Earth, \studentX\ has visited our moon, Mars, Pluto, and Planet X.  Using <xref ref="" />t-gplanets}, compute \studentX's weight are each location, assuming \hisX\ mass is \massX.
<ol>
</p></li><li><p>[Earth] <m>F_g = (\massX)\left[ \frac{ G M_E}{R_E^2} \right] = (\massX)(9.825\unitfrac{m}{s^2}) \ = \ \sig{933}{.4}{N}</m>
</p></li><li><p>[moon] <m>F_g = (\massX)\left[ \frac{ G M_m}{R_m^2} \right] = (\massX)(9.782\unitfrac{m}{s^2}) \ = \ \sig{929}{.3}{N}</m>
</p></li><li><p>[Mars] <m>F_g = (\massX)\left[ \frac{ G M_M}{R_M^2} \right] = (\massX)(9.763\unitfrac{m}{s^2}) \ = \ \sig{927}{.5}{N}</m>
</p></li><li><p>[Pluto] <m>F_g = (\massX)\left[ \frac{ G M_P}{R_P^2} \right] = (\massX)(9.763\unitfrac{m}{s^2}) \ = \ \sig{927}{.5}{N}</m>
</p></li><li><p>[Planet X] <m>F_g = (\massX)\left[ \frac{ G M_X}{R_X^2} \right] = (\massX)(9.763\unitfrac{m}{s^2}) \ = \ \sig{927}{.5}{N}</m>
</ol>
\end{sample}
%
\begin{table}[bhtp]
\hrule\hrule
\begin{center}
\caption[Properties of various celestial bodies]{\label{t-gplanets} Properties of various celestial bodies.
<assemblage>Return to: </assemblage> <xref provisional="" />{se-gplanets}
}
\begin{tabular}{lccr}
Planet & Mass (kg) & Mean Radius (m) & <m>g (\unitfrac{m}{s^2})</m> \\
\end{tabular}
\end{center}
\hrule\hrule
\end{table}
%


</subsection><subsection><title></title>Inertial Mass versus Gravitational Mass}\label{ss-equivmm}<aside><title>Referenced by</title> <p><xref provisional="" /></p></aside>{ss-weightmass}<!-- -->\new{v2.2}{Moved this here, might need to move it back.}

</section><section><title></title>Gravitational Potential Energy} \label{s-PEG}<aside><title>Referenced by</title> <p><xref provisional="" /></p></aside>{ss-PEg}

Recall <xref ref="" text="type-global" />ss-PEg}

</section><section><title></title>Making Connections}\label{s-Gconnection}<aside><title>Referenced by</title> <p><xref provisional="" /></p></aside>{s-Econnection}

<me> \begin{array}{ccccc}
& & \vec F = m \vec g & & \\
& \deq F = G \frac{m_1 m_2}{R^2} & \leftrightarrow & \deq g = G \frac{m}{R^2} & \\
\Delta \PE = -\vec F \cdot \Delta\vec x & \updownarrow & & \updownarrow & \mbox{\scriptsize [for later]} \\
& \deq \PE = G \frac{m_1 m_2}{R} & \leftrightarrow & \mbox{[for later]} & \\
& & \mbox{\scriptsize [for later]} & &
\end{array} </me>
(Look ahead to the parallel with the electrical interaction in <xref ref="" />s-Econnection}.)

</section><section><title></title>Orbits}


\part{Making Waves}

</chapter><chapter><title></title>Fluids}<!-- -->\new{v2.2}{Placeholder}
</section><section><title></title>Density}\label{s-density}<idx></idx>{Density}<aside><title>Referenced by</title> <p><xref provisional="" /></p></aside>{ss-weightmass}

</section><section><title></title>Surface Tension}\label{s-surface-tension}<aside><title>Referenced by</title> <p>Discussion of <xref provisional=""></xref></p></aside>{d-surf-tension}



</chapter><chapter><title></title>Oscillations}\label{c-SHM}
</section><section><title></title>Oscillating Springs}\label{c-SHMspring}<aside><title>Referenced by</title> <p>Discussion of <xref provisional=""></xref></p></aside>[<m>F=ma</m>]{d-fma}
</section><section><title></title>Oscillating Pendulums}\label{c-SHMpend}

</section><section><title></title>Other Examples of Oscillations}\label{s-SHMother}

On 13 April, 2017,<!-- -->\new{v2.3}{New source of info}
<url href=""></url>{http://www.cbc.ca/podcasting}{CBC Broadcasting} published a
<url href=""></url>{http://www.cbc.ca/podcasting/includes/quirks.xml}{<em></em>Quirks and Quarks}} episode discussing how we can find
<url href=""></url>{https://podcast-a.akamaihd.net/mp3/podcasts/quirks_20170415_12100.mp3}{solutions to health issues caused by swaying office towers and vibrating floors}.

</chapter><chapter><title></title>Sound}
</subsection><subsection><title></title>Musical Instruments}\label{ss-stringed-instruments} <aside><title>Referenced by</title> <p><xref provisional="" /></p></aside>{A-swing-tension}



\part{Is It Hot in Here?}

</chapter><chapter><title></title>The flow of thermal energy}

\phantomsection\label{find-heatwarm}
Energy is a noun<idx><h></h><h></h></idx>{Energy!noun}; objects can <em></em>have} energy.  <p xml:id=""></p>d-heatverb}{Heat is a verb}<idx><h></h><h></h></idx>{Heat!verb}; heating is a process of <em></em>exchanging} energy.  Recall our <xref ref=""></xref>{d-forcenoun}{discussions of force}<idx><h></h><h></h></idx>{Force!noun} and <xref ref=""></xref>{d-workverb}{work}<idx><h></h><h></h></idx>{Work!verb}.

</section><section><title></title>Specific Heat Capacity}\label{s-specificheat}

<p xml:id=""></p>d-heatwarm}{Heating (positive <m>Q</m>)} can warm (positive <m>\Delta T</m>) a material.
<men>\label{eq-Q=mcDT}
Q = m c \, \Delta T
</men>
but <xref ref="" />eq-Q=mL} (as one example) shows that it is possible to heat (positive <m>Q</m>) a material without warming it (constant <m>T</m>). When we get to <xref ref="" />s-PV} we will see other examples of <q>isothermal processes</q> that have a non-zero <m>Q</m> (heat the system or heat the surroundings) without warming or cooling the system.

</section><section><title></title>Latent Heat}

Heating might also change the phase of a material.<aside><title>Referenced by</title> <p>Discussion of <xref provisional=""></xref></p></aside>[heating versus warming]{d-heatwarm}
<men>\label{eq-Q=mL}
Q = \pm mL
</men>

</section><section><title></title>The Flow of Thermal Energy}

</subsection><subsection><title></title>Thermal Conductivity}\label{ss-thermalconductivity}<aside><title>Referenced by</title> <p><xref provisional="" /></p></aside>{s-story}

<men>\label{eq-thermalconductivity}
\frac{Q}{\Delta t} = \kappa A \, \frac{\Delta T}{\Delta x}
</men>

\begin{example}
\fcolorbox{black}{yellow!10}{\begin{minipage}{4.925in}
\caption{\label{ex-baking}\studentA\protect{<idx><h sortby="\studentA">\studentA</h></idx>} warms \hisA\ oven.}
\studentA\protect{<idx><h sortby="\studentA">\studentA</h></idx>} decides to bake some bread for the dinner party at \studentB\protect{<idx><h sortby="\studentB">\studentB</h></idx>}'s house, but \heA\ is on a tight schedule.  In order to set \hisA\ schedule, \heA\ needs to know how long it will take \hisA\ oven to <xref ref="" text="title"></xref>[find-heatwarm]{warm up}.

<assemblage>Return to: </assemblage> <xref provisional="" />{s-story}
\end{minipage}}
\end{example}

</subsection><subsection><title></title>Convection}
</subsection><subsection><title></title>Radiation}

</chapter><chapter><title></title>Ideal Gas Law}
</section><section><title></title><m>P</m>-<m>V</m> Diagrams}\label{s-PV}<aside><title>Referenced by</title> <p>Discussion of <xref provisional=""></xref></p></aside>[heating versus warming]{d-heatwarm}

\part{Let There be Light!}

</chapter><chapter><title></title>The Electrical Interaction}\label{c-electric}<aside><title>Referenced by</title> <p>Discussion of <xref provisional=""></xref></p></aside>[fundamental forces]{d-fundamental}
</section><section><title></title>Electrical Charge}\label{s-Echarge}<!-- -->\new{v2.1}{Decide where this should go.}

</section><section><title></title>The Big Picture}

</subsection><subsection><title></title>Electric Forces and Fields}\label{ss-Efield}<aside><title>Referenced by</title> <p></p></aside>{\mmr{<xref ref="" />sss-vectorequations}}, \mmr{<xref ref=""></xref>{d-fma}{<m>F=ma</m>}}}

pst-electricfield

</subsubsection><subsubsection><title></title>Examples}

</subsection><subsection><title></title>Electric Forces, Fields, and Potential Energy}

</subsection><subsection><title></title>Electric Fields, Potential Energy, and Potential}

</section><section><title></title>Making Connections}\label{s-Econnection}<aside><title>Referenced by</title> <p><xref provisional="" /></p></aside>{s-Gconnection}

<me> \begin{array}{ccccc}
& & \vec F = q \vec E & & \\
& \deq F = k \frac{q_1 q_2}{r^2} & \leftrightarrow & \deq E = k \frac{q}{r^2} & \\
\Delta \PE = -\vec F \cdot \Delta\vec x & \updownarrow & & \updownarrow & \Delta V = -\vec E \cdot \Delta\vec x  \\
& \deq \PE = k \frac{q_1 q_2}{r} & \leftrightarrow & \deq V = k \frac{q}{r} & \\
& & \Delta \PE = q \Delta V & &
\end{array} </me>
(Recall the parallel with the gravitational interaction in <xref ref="" />s-Gconnection}.)

</chapter><chapter><title></title>Electricity}

</chapter><chapter><title></title>The Magnetic Interaction}

pst-magneticfield

</chapter><chapter><title></title><q>Magnicity?</q>}

</chapter><chapter><title></title>Light}

</chapter><chapter><title></title>Optics}

\part{What Have You Done for Me Lately?}

</chapter><chapter><title></title>Relativity}
</chapter><chapter><title></title>Quantum Mechanics}<!-- -->\new{v2.1}{Decide if these subsections should be chapters in and of themselves.  These are now labeled.}
</section><section><title></title>Atomic Physics} </subsection><subsection><title></title>The Periodic Table and Quantum Numbers}
</section><section><title></title>Nuclear Physics} </subsection><subsection><title></title>Nuclear Decay}\label{ss-nucleardecay}
</subsection><subsection><title></title>The Strong Nuclear Force}\label{ss-strong}<aside><title>Referenced by</title> <p>Discussion of <xref provisional=""></xref></p></aside>[fundamental forces]{d-fundamental}
</subsection><subsection><title></title>The Weak Nuclear Force}\label{ss-weak}<aside><title>Referenced by</title> <p>Discussion of <xref provisional=""></xref></p></aside>[fundamental forces]{d-fundamental}
</section><section><title></title>Particle Physics}\label{s-particle}
</subsection><subsection><title></title>Field Theory}
</subsection><subsection><title></title>Quantum Electrodynamics}\label{ss-QED}<aside><title>Referenced by</title> <p>Discussion of <xref provisional=""></xref></p></aside>[fundamental forces]{d-fundamental}
</subsection><subsection><title></title>Quantum Chromodynamics}\label{ss-QCD}<aside><title>Referenced by</title> <p>Discussion of <xref provisional=""></xref></p></aside>[fundamental forces]{d-fundamental}
</subsection><subsection><title></title>The Standard Model}\label{ss-StandardModel}
</subsection><subsection><title></title>Particle Decay}\label{ss-particledecay}
</chapter><chapter><title></title>Condensed Matter}
</chapter><chapter><title></title>Astronomy}
</chapter><chapter><title></title>Cosmology}

\part{Supplements}

</chapter><chapter><title></title>Deeper Dive}\label{c-revisted}<!-- -->\new{v2.1}{This chapter should mirror \protect{<xref ref="" />c-physics}}.}

This is where I will put the fuller explanations.

</subsection><subsection><title></title>The Sun}\label{sss-sun}
The bright, shiny sun, which keeps us all alive, is a nice example of a rather complex system that allows us to verify our various theories of the world around us.  We can consider the existence of a star in three phases: the birth of a star, the life of the star, and the death of the star.

</subsubsection><subsubsection><title></title>The Birth of a Star}
</subsubsection><subsubsection><title></title>The Life of a Star}
</subsubsection><subsubsection><title></title>The Death of a Star}


</subsection><subsection><title></title>Kitchen Appliances}
</subsubsection><subsubsection><title></title>Oven}
</subsubsection><subsubsection><title></title>Refrigerator}
</subsubsection><subsubsection><title></title>Microwave}
</subsubsection><subsubsection><title></title>Television}

</subsection><subsection><title></title>Automobile}
</subsubsection><subsubsection><title></title>Coolant and Antifreeze}
</subsubsection><subsubsection><title></title>Tires}
</subsubsection><subsubsection><title></title>Torque}

</subsection><subsection><title></title>Cool Ideas}
</subsubsection><subsubsection><title></title>Black Holes}\label{sss-blackhole2}<aside><title>Referenced by</title> <p><xref provisional="" /></p></aside>{ss-weightmass}

On 7 April, 2017,<!-- -->\new{v2.3}{New source of info}
<url href=""></url>{http://www.cbc.ca/podcasting}{CBC Broadcasting} published a
<url href=""></url>{http://www.cbc.ca/podcasting/includes/quirks.xml}{<em></em>Quirks and Quarks}} episode discussing how we can
<url href=""></url>{https://podcast-a.akamaihd.net/mp3/podcasts/quirks_20170408_51226.mp3}{turn our planet into a giant telescope to get a photo of a black hole}.
The results should be available by the early 2018.<todo></todo>Follow-up in 2018 to find the results.}

</subsubsection><subsubsection><title></title>Quantum Mechanics}
</subsubsection><subsubsection><title></title>Relativity}
</subsubsection><subsubsection><title></title>String Theory}



</chapter><chapter><title></title>Podcasts and Videos}\label{c-videos}\label{c-podcasts}

</section><section><title></title>Podcasts}\label{s-podcasts}
<xref ref="" text="title"></xref>{http://spacepod.libsyn.com/}{T4LTFdOxHD5WWzdD}{99}{Spacepod with Carrie Nugent} \\
<url href=""></url>{http://www.sciencefriday.com/}{Science Friday with Ira Flatow}

</section><section><title></title>Videos}\label{s-videos}
<url href=""></url>{http://physicsfootnotes.com/}{Physics Footnotes} \\
<url href=""></url>{http://sixtysymbols.com/}{Sixty Symbols}

</section><section><title></title>Websites}\label{s-websites}
<url href=""></url>{http://www.aldakavlilearningcenter.org/practice/flame-challenge}{The Flame Challenge}

</chapter><chapter><title></title>Answers to Interactive Questions}

\begin{AIQ}
</p></li><li><p>\label{A-hbf} TOOK <assemblage>Return to: </assemblage> <xref provisional="" />{IQ-holdbook}
</p></li><li><p>\label{A-chair1} TOOK <assemblage>Return to: </assemblage> <xref provisional="" />{irl-NI}
</p></li><li><p>\label{A-chair2} TOOK  <assemblage>Return to: </assemblage> <xref provisional="" />{irl-NI}
</p></li><li><p>\label{A-weight.loss} TOOK <assemblage>Return to: </assemblage> <xref provisional="" />{irl-scale}
</p></li><li><p> \label{A-ladderNf} Since the full weight of the ladder, <m>F_g = \sig{222}{.69}{N}</m>, is still pressing downwards into the floor (as a normal force), it is tempting to say that <xref ref="" text="title"></xref>[ss-NIII]{Newton's third law} implies that the floor pushes the ladder upwards with a normal force of <m>\sig{222}{.69}{N}</m> but this would not account for the frictional force on the wall, <m>F_{fw}</m>.  If there were no friction between the ladder and the wall, then we could deduce <m>F_{Nf}</m>, but at this point, we cannot. <assemblage>Return to: </assemblage> <xref provisional="" />{ex-ladder2}
</p></li><li><p>\label{A-hbnof}  TOOK  <assemblage>Return to: </assemblage> <xref provisional="" />{IQ-holdbook}
</p></li><li><p>\label{A-netF-a} TOOK <assemblage>Return to: </assemblage> <xref provisional="" />{se-weightA}
</p></li><li><p>\label{A-chair3} TOOK <assemblage>Return to: </assemblage> <xref provisional="" />{irl-NI}
</p></li><li><p>\label{A-weight.gain} TOOK <assemblage>Return to: </assemblage> <xref provisional="" />{irl-scale}
</p></li><li><p>\label{A-nowall} If we consider <m>\mu_w\rightarrow 0</m>, then <m>F_{fw}=0\unit N</m>,  <m>\vec F_{Nf} = -\vec F_g = \sig{222}{.7}{N} \jhat</m>, and <m>\vec F_{Nw} = - \vec F_{ff} = \sig{28.7}{9}{N} \ihat</m>.  In this case, <m>\mu_f</m> could be as small as <m>0.\sig{129}{3}{}</m> and still hold the ladder in place, unless \studentC<idx><h sortby="\studentC">\studentC</h></idx> climbs the ladder, in which case see <xref ref="" text="type-global" />A-nowallC}.  <assemblage>Return to: </assemblage> <xref provisional="" />{ex-ladder2}
</p></li><li><p>\label{A-true1} TOOK  <assemblage>Return to: </assemblage> <xref provisional="" />{IQ-holdbook}
</p></li><li><p>\label{A-chair4} TOOK <assemblage>Return to: </assemblage> <xref provisional="" />{irl-NI}
</p></li><li><p>\label{A-scale.increase} TOOK  <assemblage>Return to: </assemblage> <xref provisional="" />{irl-scale}
</p></li><li><p>\label{A-nowallC} If we consider <m>\mu_w\rightarrow 0</m> with \studentC<idx><h sortby="\studentC">\studentC</h></idx> (<m>m=\massC</m>) at the third-rung-from-the-top of the ladder, (<m>1.53\unit m</m> up the ladder), then <m>F_{fw}=0\unit N</m>,  <m>\vec F_{Nf} = \sig{1105}{.6}{N} \jhat</m>, and <m>\vec F_{Nw} = - \vec F_{ff} = \sig{171}{.97}{N} \ihat</m>.  In this case, <m>\mu_f</m> could be as small as <m>0.\sig{155}{5}{}</m> and still hold the ladder in place. <assemblage>Return to: </assemblage>{\mmr{<xref ref="" text="type-global" />A-nowall}}, \mmr{<xref ref="" />ex-ladder2}}}
</p></li><li><p>\label{A-gworld} TOOK <assemblage>Return to: </assemblage>{\mmr{<xref ref="" text="type-global" />A-gpeaks}}, \mmr{<xref ref="" />t-gworld}}}
</p></li><li><p>\label{A-false1} TOOK <assemblage>Return to: </assemblage> <xref provisional="" />{IQ-holdbook}
</p></li><li><p>\label{A-chair5} TOOK <assemblage>Return to: </assemblage> <xref provisional="" />{irl-NI}
</p></li><li><p>\label{A-scale.measure} TOOK  <assemblage>Return to: </assemblage> <xref provisional="" />{irl-scale}
</p></li><li><p>\label{A-gpeaks} TOOK  <assemblage>Return to: </assemblage> <xref provisional="" />{t-gworld}
</p></li><li><p>\label{A-falls}  TOOK   <assemblage>Return to: </assemblage> <xref provisional="" />{IQ-holdbook}
</p></li><li><p>\label{A-chair6} TOOK <assemblage>Return to: </assemblage> <xref provisional="" />{irl-NI}
</p></li><li><p>\label{A-fly.balls} TOOK <assemblage>Return to: </assemblage> <xref provisional="" />{irl-nonparabolic}
</p></li><li><p>\label{A-hitY} TOOK  <assemblage>Return to: </assemblage> <xref provisional="" />{IQ-holdbook}
</p></li><li><p>\label{A-noFT} TOOK <assemblage>Return to: </assemblage> <xref provisional="" />{A-chair5}
</p></li><li><p>\label{A-scale.ramp} TOOK   <assemblage>Return to: </assemblage> <xref provisional="" />{irl-scale}
</p></li><li><p>\label{A-pitches.side} TOOK  <assemblage>Return to: </assemblage> <xref provisional="" />{irl-nonparabolic}
</p></li><li><p>\label{A-hitN} TOOK <assemblage>Return to: </assemblage> <xref provisional="" />{IQ-holdbook}
</p></li><li><p>\label{A-chair7} TOOK <assemblage>Return to: </assemblage> <xref provisional="" />{irl-NI}
</p></li><li><p>\label{A-pitches.top} TOOK <assemblage>Return to: </assemblage> <xref provisional="" />{irl-nonparabolic}
</p></li><li><p>\label{A-landedY} TOOK  <assemblage>Return to: </assemblage> <xref provisional="" />{IQ-holdbook}
</p></li><li><p>\label{A-gravity}  TOOK <assemblage>Return to: </assemblage> <xref provisional="" />{A-hbf}
</p></li><li><p>\label{A-chair8} TOOK <assemblage>Return to: </assemblage> <xref provisional="" />{irl-NI}
</p></li><li><p>\label{A-pool-roll} TOOK <assemblage>Return to: </assemblage> <xref provisional="" />{irl-poolcushion}
</p></li><li><p>\label{A-landedN} TOOK <assemblage>Return to: </assemblage> <xref provisional="" />{IQ-holdbook}
</p></li><li><p>\label{A-FT} TOOK  <assemblage>Return to: </assemblage> <xref provisional="" />{A-chair5}
</p></li><li><p>\label{A-pool-bumper} TOOK <assemblage>Return to: </assemblage> <xref provisional="" />{irl-poolcushion}.
</p></li><li><p>\label{A-zero} TOOK <assemblage>Return to: </assemblage> <xref provisional="" />{IQ-holdbook}
</p></li><li><p>\label{A-firstfall} TOOK  <assemblage>Return to: </assemblage> <xref provisional="" />{irl-freefall}
</p></li><li><p>\label{A-floor}  TOOK  <assemblage>Return to: </assemblage> <xref provisional="" />{se-FNB}
</p></li><li><p>\label{A-firstwhy} TOOK  <assemblage>Return to: </assemblage> <xref provisional="" />{irl-freefall}
</p></li><li><p>\label{A-noncue} TOOK <assemblage>Return to: </assemblage> <xref provisional="" />{irl-poolcushion}
</p></li><li><p>\label{A-one} TOOK  <assemblage>Return to: </assemblage> <xref provisional="" />{IQ-holdbook}
</p></li><li><p>\label{A-fallv} TOOK  <assemblage>Return to: </assemblage> <xref provisional="" />{irl-freefall}
</p></li><li><p>\label{A-pool-spin} TOOK <assemblage>Return to: </assemblage> <xref provisional="" />{irl-poolcushion}
</p></li><li><p>\label{A-second} TOOK  <assemblage>Return to: </assemblage> <xref provisional="" />{se-FNB}
</p></li><li><p>\label{A-falla} TOOK  <assemblage>Return to: </assemblage> <xref provisional="" />{irl-freefall}
</p></li><li><p>\label{A-pool-later} TOOK <assemblage>Return to: </assemblage> <xref provisional="" />{irl-poolcushion}
</p></li><li><p>\label{A-two} TOOK  <assemblage>Return to: </assemblage> <xref provisional="" />{IQ-holdbook}
</p></li><li><p>\label{A-third} TOOK  <assemblage>Return to: </assemblage> <xref provisional="" />{se-FNB}

</p></li><li><p>\label{A-swing-tension} TOOK <assemblage>Return to: </assemblage>{\mmr{<xref ref="" />irl-tension}}, \mmr{<xref ref="" text="type-global" />A-chandelier-tension}}}
</p></li><li><p>\label{A-fan-tension} TOOK <assemblage>Return to: </assemblage> <xref provisional="" />{irl-tension}
</p></li><li><p>\label{A-chandelier-tension} TOOK   <assemblage>Return to: </assemblage> <xref provisional="" />{A-fan-tension}
\end{AIQ}

</chapter><chapter><title></title>Adventures}


Throughout the book, there are examples and adventures.  The follow-up stories are contained below.
\begin{Story}
</p></li><li><p>\label{a-parkandwalk}  TOOK
</p></li><li><p>\label{a-NIIIaction} TOOK
</p></li><li><p>\label{a-coastindrive} TOOK
</p></li><li><p>\label{a-NIIIreaction} TOOK
</p></li><li><p>\label{a-coastinneutral} TOOK
</p></li><li><p>\label{a-NIIIconcern} TOOK
</p></li><li><p>\label{a-NIdrive} TOOK
</p></li><li><p>\label{a-NIIIexperiment} TOOK
</p></li><li><p>\label{a-nogas} TOOK
</p></li><li><p>\label{a-NIIIrestraint} TOOK
</p></li><li><p>\label{a-parkandwalk2}  TOOK
</p></li><li><p>\label{a-NIIIfaculty} TOOK
</p></li><li><p>\label{a-intosunset} TOOK
</p></li><li><p>\label{a-NIIIsecurity} TOOK
</p></li><li><p>\label{a-NIresult} TOOK
</p></li><li><p>\label{a-guilty} TOOK
</p></li><li><p>\label{a-nogas2} TOOK
</p></li><li><p>\label{a-intosunset2} TOOK
\end{Story}

%%%%%%%%%%%%%%%%%%%%%%%%%%%%%%%%%%%%%%%%%%%%%%%%%%%%%%%%%%%%%%%%%%%%%%%%%%%%%%%%%%%%%%%%%%%%%%%%%%%%%%

</chapter><chapter><title></title>Characters}

This textbook has five characters who follow you throughout the book.  They appear in the examples and some homework problems.  They also remember previous experiences.  I need to adjust the examples in <xref ref="" />c-force} such that the people pushing boxes are helping the reader rearrange furniture.

The index lists<todo></todo>The index will recognize the people in two different formats.  One is by my name for them, which is <backslash />studentX (where X is A, B, C, D, <ellipsis /> Z).  The other is by the name assigned to that variable.  So these show up in different places in the Index.}{} the pages that the characters appear.  The point of this chapter is to highlight some of the primary adventures of the characters according to their own perspectives.  <em></em>None of the links in this chapter will be given a corresponding return link.}  This chapter is for me to track relationships and will likely go away when the book is ready for publication.
%
I can, at the header of the code, define the name, gender, mass, and dimensions of each individual.\dothis[inline]{<url href=""></url>{http://malveyauthor.com/}{Madeline Alvey}, the author of  \protect{<url href=""></url>{http://escapepod.org/2017/03/09/ep566-honey-and-bone-artemis-rising-3/}{<q>Honey and Bone</q> at EscapePod}} is a physics and English undergraduate student at UK in Lexington.  I might consider hiring(?) her to help storyboard the characters.}

</section><section><title></title>\studentA<idx><h></h><h></h></idx>{\studentA!inside}}<idx><h></h><h></h></idx>{\studentA!outside}<todo></todo>The index-call that is <em></em>outside} of the section title registers as <backslash />studentA, which puts the name alphabetically under <backslash />studentA, rather than \studentA.  The index-call that is <em></em>inside} of the section title registers as \studentA, which puts the name alphabetically under \studentA, rather than <backslash />studentA.}
<idx></idx>{\studentA|(} % Begin page-range
<ul>
</p></li><li><p> In <xref ref="" />s-forcewords}, \studentB{} gives \studentA{} a good-natured shove in the arm in order to get the language clarified and begin the conversation about the on-by notation.
</p></li><li><p> In <xref ref="" text="type-global" />se-FBD-AB} \studentA{} helps \studentB{}<ellipsis />
    <ul>
    </p></li><li><p> (in the current version) push an object to make it accelerate and feel a reaction force causing \himA{} to accelerate backwards.
    </p></li><li><p> (in the future version) will help the reader move into or out of their residence hall by pushing on heavier furniture.
    </p></li><li><p>[NOTE:] This is all drawn in <xref ref="" />f-firstFBD}, which is updated in <xref ref="" />f-firstFBDupdate}.
    </ul>
</p></li><li><p> In <xref ref="" text="type-global" />se-weightA}, \studentA{} falls from a small height.  (maybe he is jumping off a short ledge while taking a short-cut to class?)
</p></li><li><p> In <xref ref="" />ex-baking}, \studentA{} decides to bake some bread for a party at \studentB's house, measuring the time it takes to warm his oven.
</ul>

<idx></idx>{\studentA|)} % end page-range
</section><section><title></title>\studentB<idx><h sortby="\studentB">\studentB</h></idx>}
<idx></idx>{\studentB|(} % Begin page-range

<ul>
</p></li><li><p> \studentB{} is a passenger in the reader's car in <xref ref="" />ex-slowcar} when the reader runs out of gas and coasts to a stop.
</p></li><li><p> \studentB{} is a passenger in the reader's car in <xref ref="" />ex-coasting} and speculates about how fast to go before putting the car in neutral to coast to a stop.
</p></li><li><p> \studentB{} joins the reader on a road trip in <xref ref="" />cyoa-NI} and runs out of gas.  This results in multiple possible adventures:
<ul>
    </p></li><li><p> <xref ref="" text="type-global" />a-parkandwalk}, which leads to either an end at <xref ref="" text="type-global" />a-nogas} or an end at <xref ref="" text="type-global" />a-intosunset}.
    </p></li><li><p> <xref ref="" text="type-global" />a-coastindrive}, which leads to either <xref ref="" text="type-global" />a-NIdrive} (choose <xref ref="" text="type-global" />a-coastinneutral} or end with <xref ref="" text="type-global" />a-intosunset2}) or <xref ref="" text="type-global" />a-parkandwalk2} (choose <xref ref="" text="type-global" />a-intosunset} or end at <xref ref="" text="type-global" />a-nogas2})
    </p></li><li><p> <xref ref="" text="type-global" />a-coastinneutral}, which leads to an end at <xref ref="" text="type-global" />a-NIresult}.
</ul>
</p></li><li><p> In <xref ref="" />s-forcewords}, \studentB{} gives \studentA{} a good-natured shove in the arm in order to get the language clarified and begin the conversation about the on-by notation.
</p></li><li><p> In <xref ref="" />se-FBD-AB}, \studentB{} helps \studentA{}<ellipsis />
    <ul>
    </p></li><li><p> (in the current version) pull an object to make it accelerate and feel a reaction force causing \himB{} to accelerate backwards.
    </p></li><li><p> (in the future version) will help the reader move into or out of their residence hall by pushing on heavier furniture.
    </p></li><li><p>[NOTE:] This is all drawn in <xref ref="" />f-firstFBD}, which is updated in <xref ref="" />f-firstFBDupdate}.
    </ul>
</p></li><li><p> In <xref ref="" text="type-global" />se-FNB}, \studentB{} has a normal force supporting \himB.  (This touches <xref ref="" text="type-global" />A-floor}, <xref ref="" text="type-global" />A-second}, and <xref ref="" text="type-global" />A-third}.)
</p></li><li><p> At some point, \studentB{} has a party, because in <xref ref="" />ex-baking}, \studentA{} decides to bake some bread for a party at \studentB's house.
</ul>

<idx></idx>{\studentB|)} % end page-range
</section><section><title></title>\studentC}
<idx></idx>{\studentC|(} % Begin page-range

<idx></idx>{\studentC|)} % end page-range
</section><section><title></title>\studentD}
<idx></idx>{\studentD|(} % Begin page-range

<idx></idx>{\studentD|)} % end page-range
</section><section><title></title>\studentE}
<idx></idx>{\studentE|(} % Begin page-range

<idx></idx>{\studentE|)} % end page-range
</section><section><title></title>\studentF}
<idx></idx>{\studentF|(} % Begin page-range

<idx></idx>{\studentF|)} % end page-range
</section><section><title></title>\studentZ}
<idx></idx>{\studentZ|(} % Begin page-range

<idx></idx>{\studentZ|)} % end page-range



%%%%%%%%%%%%%%%%%%%%%%%%%%%%%%%%%%%%%%%%%%%%%%%%%%%%%%%%%%%%%%%%%%%%%%%%%%%%%%%%%%%%%%%%%%%%%%%%%%%%%%

\addcontentsline{toc}{chapter}{Index}
%\printindex
\input{ABIP.ind}

<!-- -->\newpage
Note: You can do some more fancy indexing with the formatting found at
\url{https://en.wikibooks.org/wiki/LaTeX/Indexing}

\end{document}

<!-- -->\newpage

%\verb[\marginparwidth]:
\printinunitsof{in}\prntlen{\marginparwidth}

%\verb[\marginparwidth]:
\printinunitsof{mm}\prntlen{\marginparwidth}

%\verb[\marginparwidth]:
\printinunitsof{pt}\prntlen{\marginparwidth}

\pagediagram


