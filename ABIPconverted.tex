\documentclass[11pt,letter,openany,makeidx]{book}
\usepackage{amsmath}
\usepackage{macros}
\usepackage{comment}
\usepackage{graphicx}
\usepackage{microtype}
\usepackage{gfsdidot}
\usepackage[T1]{fontenc}
\usepackage{booktabs}
\usepackage{underscore}
\usepackage{caption}
\usepackage[within=chapter,chapterlistsgap=6pt]{newfloat}
\usepackage{tocloft}
\usepackage{xpicture}
\usepackage{xcolor}
%\usepackage[dvips]{xcolor}
%\GetGinDriver  % for xcolor to work well with hyperref
%\usepackage[\GinDriver]{hyperref}
\usepackage{ulem}%for \sout (done)
\usepackage[colorinlistoftodos]{todonotes}
%\usepackage[disable,colorinlistoftodos]{todonotes}
%\usepackage{layouts}
%\usepackage{showframe}
\usepackage{coordsys}
\usepackage[pdftex]{hyperref}

%\usepackage{cellpage}

\hypersetup{colorlinks=true,bookmarks=true,pdftitle=Algebra-Based Introductory Physics,pdfauthor=J.Christensen,pdfdisplaydoctitle}
% If using a bibliography, then include "backref" in list of \hypersetup items
% linkcolor=color of internal links (red); anchorcolor = color of anchor text (black); citecolor = bibliographic citations (green); filecolor = color for local URL files (cyan); menucolor = Acrobat menu (red); urlcolor = external links (magenta); hidelinks = remove all color
% citebordercolor = color of box for citations (0 1 0); fileborder = links to files box (0 .5 .5); linkbordercolor = normal links (1 0 0); menuborder; urlborder; allbordercolors; pdfborder

\includecomment{ForMe}
\includecomment{ForReviewer}
\includecomment{ForPublic}

\makeindex

\newlistof{example}{loe}{List of Examples}
\DeclareFloatingEnvironment[fileext=loe,listname="List of Examples",name=Example]{example}
\setlength{\cftexamplenumwidth}{1cm}
\newcounter{sample}
\newcounter{carrysample}
\renewcommand{\thesample}{Simple Example \arabic{sample}}
\renewcommand{\thecarrysample}{Simple Example \arabic{carrysample}}
\newenvironment{sample}{\color{rgb:red,0;green,2;blue,1}\begin{list}{\textbf{\thesample}:}{\usecounter{sample} \setcounter{sample}{\value{carrysample}} \leftmargin 12pt}}{\end{list}\setcounter{carrysample}{\value{sample}}}
\newcommand{\THREE}[6]{\vspace{-3pt}\begin{flushright} Select one:  \mbox{#1 (\ref{#4})},  \mbox{#2 (\ref{#5})}, or \mbox{#3 (\ref{#6})}.\end{flushright}}
\newcommand{\TWO}[4]{\begin{flushright} Select one:  \mbox{#1 (\ref{#3})} or \mbox{#2 (\ref{#4})}.\end{flushright}}
\newcommand{\YN}[2]{\TWO{Yes}{No}{#1}{#2}}
\newcommand{\TF}[2]{\TWO{True}{False}{#1}{#2}}
\newcommand{\return}[1]{{} \hfill \mbox{Return to \ref{#1}.}}
\newcommand{\autoreturn}[1]{{} \hfill \mbox{Return to \autoref{#1}.}}
\newcommand{\linkreturn}[2][a related idea]{{}\hfill \mbox{Return to the discussion of \protect{\hyperlink{#2}{#1}}.}}
\newcommand{\mmr}[1]{\mbox{[\protect{#1}]}}
\newcommand{\multireturn}[1]{{}\hfill Return to one of the following locations: \newline #1.}
\newcounter{AtIQ}
\renewcommand{\theAtIQ}{Answer \arabic{AtIQ}}
\newenvironment{AIQ}{\begin{list}{\textbf{Interactive \theAtIQ}:}{\usecounter{AtIQ} \leftmargin 12pt}}{\end{list}}

% Related (return), but not part of...
\newcommand{\mreturn}[1]{\note{Return to \protect{\ref{#1}}.}}
\newcommand{\mlinkreturn}[2][a related idea]{\note{Return to the discussion of \protect{\hyperlink{#2}{#1}}.}}
\newcommand{\mautoreturn}[1]{\note{Return to \protect{\autoref{#1}}.}}
\newcommand{\mmultireturn}[1]{\note{Return to one of the following locations: \newline #1.}}


\newlistof{adventure}{loa}{List of Adventures}
\DeclareFloatingEnvironment[fileext=loa,listname="List of Adventures",name=Adventure]{adventure}
\setlength{\cftadventurenumwidth}{1cm}
\newcounter{CYOA}
\renewcommand{\theCYOA}{Plan \Alph{CYOA}}
\newenvironment{CYOA}{\begin{list}{\textbf{\theCYOA}:}{\usecounter{CYOA}}}{\end{list}}
\newcounter{storyline}
\renewcommand{\thestoryline}{Storyline \arabic{storyline}}
\newenvironment{Story}{\begin{list}{\textbf{\thestoryline}:}{\usecounter{storyline} \leftmargin 12pt}}{\end{list}}

\newlistof{reallife}{irl}{List of Real Life Patterns}
%\DeclareFloatingEnvironment[fileext=irl,listname="List of Real Life Patterns",chapterlistsgaps=off,name=Real Life Patterns]{reallife}
\DeclareFloatingEnvironment[fileext=irl,listname="List of Real Life Patterns",name=Real Life Patterns]{reallife}
\setlength{\cftreallifenumwidth}{1cm}
\newcounter{IRL}
%\renewcommand{\theIRL}{\arabic{IRL}}
\newenvironment{realtable}{%\renewcommand{\arraystretch}{2}
                           %\hspace{-.2in}
                            \begin{tabular}{@{}lll@{}} \toprule Do This & Notice This & Ask This  \\ }
                            {\bottomrule \end{tabular} }%\renewcommand{\arraystretch}{.5}}
\newcommand{\dna}[3]{\midrule \begin{minipage}{4cm}\raggedright #1 \end{minipage}
                   & \begin{minipage}{4cm}\raggedright #2 \end{minipage}
                   & \begin{minipage}{4cm}\raggedright #3 \end{minipage} \\ }
\newcommand{\multidna}[1]{\multicolumn{3}{|c|}{\begin{minipage}{13cm}\center #1 \end{minipage}} \\ \midrule }


\newlistof{story}{los}{The Stories of the Equations}
\DeclareFloatingEnvironment[fileext=los,listname="The Stories of the Equations",name=This Equation's Story]{story}
\setlength{\cftstorynumwidth}{1cm}
\newcommand{\thestoryof}[1]{\marginpar{\raggedright \footnotesize The story of \\ \fcolorbox{black}{yellow}{\begin{minipage}[c]{1.5in} \center $\deq #1$ \end{minipage}}}}
\newcommand{\EqStory}[2]{\left[ {\color{rgb:red,1;green,1;blue,4} \begin{minipage}{#1}\raggedright\begin{center} #2 \end{center}\end{minipage}} \right]}
\newcommand{\EqStoryOver}[3]{\overbrace{\EqStory{#1}{#2}}^{\displaystyle #3}}
\newcommand{\EqStoryUnder}[3]{\underbrace{\EqStory{#1}{#2}}_{\displaystyle #3}}
\newcommand{\EqStoryFrac}[5]{\frac{\overbrace{\EqStory{#1}{#2}}^{\displaystyle #3}}
                                 {\underbrace{\EqStory{#1}{#4}}_{\displaystyle #5}}}


%%%%%%%%%%%%%%%%%%%%%%%%%%%%%%%%%%%%%%%%%%%%%%%%%%%%%%%%%%%%
%
%\presetkeys{todonotes}{fancyline,color=blue!15}{}
\presetkeys{todonotes}{color=blue!15,linecolor=blue!75,size=\footnotesize}{}
%
\newcounter{todocounter}
\newcommand{\dothis}[2][]
{\stepcounter{todocounter}\todo[color=green!30, #1]{\thetodocounter: #2}}
\newcommand{\docaption}[3][]
{\stepcounter{todocounter}\todo[color=green!30, prepend, caption={\thetodocounter: \underline{#2}}, #1]{#3}}
\newcommand{\addlink}[2][]
{\stepcounter{todocounter}\todo[prepend, caption={\thetodocounter: \underline{Add Link}}, #1]{#2}}
\newcounter{todourgentcounter}
\newcommand{\urgent}[2][]
{\stepcounter{todourgentcounter}\todo[color=orange!50, #1]{\thetodourgentcounter: #2}}
\newcommand{\urgcap}[3][]
{\stepcounter{todourgentcounter}\todo[color=orange!50, prepend, caption={\thetodourgentcounter: \underline{#2}}, #1]{#3}}
\newcommand{\done}[2][]
{\todo[color=yellow!10, #1]{\sout{#2}}}
%
%\newcommand{\new}[2]{}%
\newcommand{\new}[2]{\marginpar{\raggedright \footnotesize New to #1 \\ \fcolorbox{blue}{yellow!10}{\begin{minipage}[c]{1.5in} \center {\color{blue} #2 } \end{minipage}}}}%
%%%%%%%%%%%%%%%%%%%%%%%%%%%%%%%%%%%%%%%%%%%%%%%%%%%%%%%%%%%%


%%%%%%%%%%%%%%%%%%%%%%%%%%%%%%%%%%%%%%%%%%%%%%%%%%%%%%%%%%%%
%
%\newcommand{\deq}{\displaystyle}
%\newcommand{\txtfrac}[2]{{}^{#1}\!/_{\!#2}}
%
%%%%%%%%%%%%%%%%%%%%%%%%%%%%%%%%%%%%%%%%%%%%%%%%%%%%%%%%%%%%



%%%%%%%%%%%%%%%%%%%%%%%%%%%%%%%%%%%%%%%%%%%%%%%%%%%%%%%%%%%%
%
% PEOPLE AND PRONOUNS
%
% According to https://www.cdc.gov/nchs/fastats/body-measurements.htm
% Measured average height, weight, and waist circumference for adults ages 20 years and over
% Men:
% Height (inches): 69.3                 = 1.760 m
% Weight (pounds): 195.5                = 88.86 kg
% Waist circumference (inches): 39.7    = 1.01 m
% Women:
% Height (inches): 63.8                 = 1.621 m
% Weight (pounds): 166.2                = 75.55 kg
% Waist circumference (inches): 37.5    = 0.9525 m
% Source: Anthropometric Reference Data for Children and Adults: United States, 2007-2010, tables 4, 6, 10, 12, 19, 20[PDF - 1.7 MB]
%  https://www.cdc.gov/nchs/data/series/sr_11/sr11_252.pdf
%
\newcommand{\studentA}{Abdul}       \newcommand{\massA}{\mbox{$85.0\unit{kg}$}}
\newcommand{\studentB}{Beth}        \newcommand{\massB}{\mbox{$75.0\unit{kg}$}}
\newcommand{\studentC}{Carl}        \newcommand{\massC}{\mbox{$90.0\unit{kg}$}}
\newcommand{\studentD}{Diane}       \newcommand{\massD}{\mbox{$80.0\unit{kg}$}}
\newcommand{\studentE}{Erik}        \newcommand{\massE}{\mbox{$95.0\unit{kg}$}}
\newcommand{\studentF}{Frances}       \newcommand{\massF}{\mbox{$85.0\unit{kg}$}}
\newcommand{\studentX}{Xerxes}       \newcommand{\massX}{\mbox{$62.5\unit{kg}$}}
\newcommand{\studentZ}{Zambert}     \newcommand{\massZ}{\mbox{$95.0\unit{kg}$}}
% Male
\newcommand{\heA}{he}\newcommand{\himA}{him}\newcommand{\hisA}{his}\newcommand{\himselfA}{himself}
\newcommand{\HeA}{He}\newcommand{\HimA}{Him}\newcommand{\HisA}{His}
\newcommand{\heC}{he}\newcommand{\himC}{him}\newcommand{\hisC}{his}\newcommand{\himselfC}{himself}
\newcommand{\HeC}{He}\newcommand{\HimC}{Him}\newcommand{\HisC}{His}
\newcommand{\heE}{he}\newcommand{\himE}{him}\newcommand{\hisE}{his}\newcommand{\himselfE}{himself}
\newcommand{\HeE}{He}\newcommand{\HimE}{Him}\newcommand{\HisE}{His}
\newcommand{\heZ}{he}\newcommand{\himZ}{him}\newcommand{\hisZ}{his}\newcommand{\himselfZ}{himself}
\newcommand{\HeZ}{He}\newcommand{\HimZ}{Him}\newcommand{\HisZ}{His}
% Female
\newcommand{\heB}{she}\newcommand{\himB}{her}\newcommand{\hisB}{her}\newcommand{\himselfB}{herself}
\newcommand{\HeB}{She}\newcommand{\HimB}{Her}\newcommand{\HisB}{Her}
\newcommand{\heD}{she}\newcommand{\himD}{her}\newcommand{\hisD}{her}\newcommand{\himselfD}{herself}
\newcommand{\HeD}{She}\newcommand{\HimD}{Her}\newcommand{\HisD}{Her}
\newcommand{\heF}{she}\newcommand{\himF}{her}\newcommand{\hisF}{her}\newcommand{\himselfF}{herself}
\newcommand{\HeF}{She}\newcommand{\HimF}{Her}\newcommand{\HisF}{Her}
%
\newcommand{\heX}{\studentX}\newcommand{\himX}{\studentX}\newcommand{\hisX}{\studentX's}\newcommand{\himselfX}{the person of \studentX}
\newcommand{\HeX}{\studentX}\newcommand{\HimX}{\studentX}\newcommand{\HisX}{\studentX's}
%%%%%%%%%%%%%%%%%%%%%%%%%%%%%%%%%%%%%%%%%%%%%%%%%%%%%%%%%%%%%


%%%%%%%%%%%%%%%%%%%%%%%%%%%%%%%%%%%%%%%%%%%%%%%%%%%%%%%%%%%%
%
% Book macros
%
\newcommand{\aside}[2]{\marginpar{\raggedright \footnotesize\textbf{#1}: #2}}
\newcommand{\important}[1]{\\ \fcolorbox{black}{yellow}{\begin{minipage}[c]{4.925in} \center #1 \end{minipage}}\\}
\newcommand{\inlife}{\marginpar[\scriptsize \raggedright How you might observe $\Rightarrow$ this in your life.]
                               {\scriptsize \raggedleft $\Leftarrow$ How you might observe this in your life.}}
\newcommand{\touchstone}{\marginpar[\scriptsize \raggedright Where have I seen this $\Rightarrow$ before?]
                                   {\scriptsize \raggedleft $\Leftarrow$ Where have I seen this before?}}
\newcommand{\foreshadow}{\marginpar[\scriptsize \raggedright When will I ever use this? $\Rightarrow$]
                                   {\scriptsize \raggedleft $\Leftarrow$ When will I ever use this?}}
\newcommand{\foreshadowR}{\reversemarginpar
                          \marginpar[\scriptsize \raggedright When will I ever use this? $\Rightarrow$]
                                    {\scriptsize \raggedleft $\Leftarrow$ When will I ever use this?}}
\newcommand{\Touchstone}[1]{\marginpar[\scriptsize \raggedright Where have I seen this $\Rightarrow$ \\ before? #1]
                                      {\scriptsize \raggedleft $\Leftarrow$ Where have I seen this before? #1}}
\newcommand{\Foreshadow}[1]{\marginpar[\scriptsize \raggedright When will I ever use this? $\Rightarrow$ \\ #1]
                                      {\scriptsize \raggedleft $\Leftarrow$ When will I ever use this? #1}}
%
%%%%%%%%%%%%%%%%%%%%%%%%%%%%%%%%%%%%%%%%%%%%%%%%%%%%%%%%%%%%


\begin{document}

%\title{Algebra-Based Introductory Physics}
%\author{J Christensen}
%\date{Jan 2017}
%\maketitle
%\pagestyle{cellpage}

\begin{titlepage}
	\centering
%	\includegraphics[width=0.15\textwidth]{example-image-1x1}\par\vspace{1cm}
	{\Huge\bfseries Physics Connected\par}
	\vspace{1cm}
	{\Large\bfseries An Algebra-Based Introductory Physics Textbook\par}
	\vspace{1cm}
	{\large Learn like you think: an interconnected view of physics\par}
	\vspace{2cm}
	{\Large\itshape by: J Christensen\par}
	\vfill
\begin{ForReviewer}
	Version 2.3\par
	{\footnotesize
    \begin{itemize}
    \item Ideas yet to implement:
        \begin{itemize}
        \item The examples are phrased as descriptions, not examples like the homework problems.  Need to consider rephrasing these, not calling them examples, or adding actual examples that better show how to respond to the way homework problems are written.
        \item Define a different page dimension that fits on a cell phone display.  (Enhance possible cell-phone reading.)
        \end{itemize}
    \item version 2.3: June 16-28, 2017
        \begin{itemize}
        \item Updated Section 81. $F=mg$ and Section 8.2 Normal Force
        \item Added specific list of Flame Challenges
        \item Rearranged some of the subsections in the ``Seeing Physics'', added references
        \item Equations of motion for constant acceleration (Need the Story Of)
        \item Added a section to Chap 5 (1-D motion) that gives examples of solutions that require multiple steps  (one equation is insufficient)
        \item Developed the weight and mass discussion and examples
        \item Ladder leaning example in torque, plus some homework problems
        \item Added some Conceptual Homework to weight/mass
        \item Added placeholders to the Gravity chapter
        \item Removed indicators of v1.7 changes
        \end{itemize}
    \item version 2.2: June 16, 2017
        \begin{itemize}
        \item Created conversation about $F=mg$ for Chapter on types of forces.  Caused modifications in lots of places
            \begin{itemize}
            \item Added freefall to the motion chapter
            \item Created IRL and Example dropping objects to see acceleration in $F=mg$, then moved to freefall section -- new Answers to interactive questions
            \item Commented on air resistance
            \item Comments about precision in language (need to do more with precision in mathematics)
            \item Started a couple of ideas about effective theories.  (need to decide where it goes)
            \item Added detail about SI, and specifically the pound-force, pound-mass, and kilogram. to sections \ref{s:SI-MKS} and \ref{ss:weightmass}
            \item Added NIST and BIPM references (found in Wikipedia and then searched further)
            \item conversation about weight and mass.  (required reference to the chapter on Fluids and density)
            \item Moved Google search about significant figures
            \end{itemize}
        \item Added comments about fundamental forces to the section on types of force
        \item Removed indicators of v1.5 and v1.6 changes
        \end{itemize}
    \item version 2.1: June 10, 2017
        \begin{itemize}
        \item Re-commented the $\backslash$new command
        \item Started the chapters on Seeing Physics [\autoref{c:physics}] and Deeper Dive [\autoref{c:revisted}] (These should be renamed)
        \item Moved some sections on fundamental interactions
        \end{itemize}
    \item version 2.0: April 10, 2017
        \begin{itemize}
        \item Re-enabled v1.8 hides
        \item Added a link to \textit{Spacepod}, \textit{Physics Footnotes}, and \textit{Sixty Symbols}
        \item Fixed a $\backslash$dothis that was inside an $\backslash$important, causing a compile error.
        \item Removed indicators of v1.4 changes
        \end{itemize}
    \item version 1.8: April 1, 2017
        \begin{itemize}
        \item Prepare for "the public": "Disabled" the To-Do items, "Hid" the $\backslash$new revision notes, Hid the List of Tables (have none yet)
        \end{itemize}
    \end{itemize}
    }
\end{ForReviewer}
\begin{ForPublic}
{\flushleft
\textbf{Note to the reviewers:}\new{v1.8}{Added the note}
My goal with this book is to create an electronically viewable book that makes use of the advantages of being electronic.  While current e-books have the advantage of being viewable on various devices with having to carry a physical book around, most e-textbooks do not take advantage of hyperlinked text.  With this book I hope to integrate links both forward and backward.  The forward links will be used to motivate curious students.  The backward links will be used to support students who lose track of previous topics.  The integration of these will also provide a convenient opportunity for students to browse through topics they are interested in.
\newpar

At this time, I am providing a single chapter to gauge the viability.  The chapter I am providing is on Newton's Laws.  However, as you read this document, you will find many, many more partially written chapters.  All of the partial chapters and sections are intended to be place-holders for the forward- and backward-links that \autoref{c:force} depends upon.
\newpar

I created this as a PDF that, I believe, can be easily viewed on a computer or tablet.  Since some of my students also seem to read on their phone, I verified that I am also able to view the text in a reasonable manner on my Samsung phone in landscape mode.  In each case, the links should be active and easily manageable.

}
\end{ForPublic}
	\vfill

% Bottom of the page
	{\large \today\par}
\end{titlepage}

\tableofcontents
\newpage
\begin{ForReviewer}
\listoftables
\vfill
\end{ForReviewer}
%\newpage
\listoffigures
\vfill
%\newpage
%\listofstorys
%\newpage
\listofexamples
\vfill
%\newpage
\listofadventures
\vfill
%\newpage
\listofreallifes
\vfill
\newpage

\listoftodos

\newpage



\part{Prerequisites}

<chapter><title>The Story of Science</title>

</chapter><chapter><title></title>Seeing Physics}\label{c-physics}

</chapter><chapter><title></title>Why so much math?}

</chapter><chapter><title></title>Estimating and Uncertainty}

\part{Introducing Motion, Force, and Energy}

</chapter><chapter><title></title>One-Dimensional Motion}\label{c-motion}

</chapter><chapter><title></title>Two-Dimensional Motion}

</chapter><chapter><title></title>Force}\label{c-force}
</section><section><title></title>How Physicists Use the Words (Notation)}\label{s-forcewords}
</section><section><title></title>Connecting the Concepts: Newton's Laws}\label{s-Newton}<aside><title>Referenced by</title> <p>Discussion of <xref provisional=""></xref></p></aside>[how to describe forces]{d-Newtonahead}
    </subsection><subsection><title></title>Translating Newton's First Law: The Law of Inertia}\label{ss-NI}<aside><title>Referenced by</title> <p>Discussion of <xref provisional=""></xref></p></aside>[how to describe forces]{d-Newtonahead}
    </subsection><subsection><title></title>Translating Newton's Second Law: The Equation Law}\label{ss-NII}<aside><title>Referenced by</title> <p></p></aside>{\mmr{<xref ref="" />sss-vectorequations}}, \mmr{<xref ref=""></xref>{d-Newtonahead}{how to describe forces}}, \mmr{<xref ref=""></xref>{d-atrestinmotion}{Newton's first law}}, \mmr{<xref ref="" />d-Fgball}}}
    </subsection><subsection><title></title>Translating Newton's Third Law: Action \& Reaction}\label{ss-NIII}<aside><title>Referenced by</title> <p></p></aside>{\mmr{<xref ref=""></xref>{d-Newtonahead}{how to describe forces}}, \mmr{<xref ref="" />ex-braced}}, \mmr{<xref ref="" />ex-unbraced}}}
</section><section><title></title>Examples} \label{s-NewtonExamples}<aside><title>Referenced by</title> <p><xref provisional="" /></p></aside>{sss-NIItogether}
</section><section><title></title>Summary and Homework}

</chapter><chapter><title></title>The Many Types of Force}\label{c-forcetype}<aside><title>Referenced by</title> <p>Discussion of <xref provisional=""></xref></p></aside>[subscript notation of forces]{d-interaction}
</section><section><title></title>Gravity at the Surface of the Earth}\label{s-Fg}<aside><title>Referenced by</title> <p></p></aside>{\mmr{<xref ref=""></xref>{d-accgrav}{freefall}}, \mmr{<xref ref="" />f-firstFBD}}}<!-- -->\new{v2.2}{Adding detail}
    </subsection><subsection><title></title>Weight versus Mass}\label{ss-weightmass}<aside><title>Referenced by</title> <p></p></aside>{\mmr{<xref ref="" />ss-convertunits}}, \mmr{<xref ref="" />s-sigfig}}}<idx></idx>{Weight}<!-- -->\new{v2.2}{Added detail.  Moved the previous version to \protect{<xref ref="" />s-sigfig}} to smooth the transition to \protect{<xref ref="" />ss-equivmm}}.}
    </subsection><subsection><title></title>Calculating the weight}\label{ss-local-mg}<!-- -->\new{v2.2}{renamed this section and added detail}
</section><section><title></title>Fundamental Forces}\label{s-fundamental}<idx><h></h><h></h></idx>{Force!Fundamental}<!-- -->\new{v2.1}{Started the section on fundamental interactions.  Link ahead, rather than detailling here.}
</section><section><title></title>Normal Force}\label{s-FN}<aside><title>Referenced by</title> <p></p></aside>{\mmr{<xref ref="" />f-firstFBD}}, \mmr{<xref ref="" text="type-global" />A-floor}}, \mmr{<xref ref="" />s-FT}}}<!-- -->\new{v2.2}{Added detail}<idx><h></h><h></h></idx>{Force!Normal}
</section><section><title></title>Tension}\label{s-FT}<aside><title>Referenced by</title> <p>Discussion of <xref provisional=""></xref></p></aside>[<m>F=ma</m>]{d-fma}<idx><h></h><h></h></idx>{Force!Tension}
    </subsection><subsection><title></title>Tension as a Support Force}\label{ss-tension-support}
    </subsection><subsection><title></title>Tension as Dragging Force}\label{ss-tension-drag}
    </subsection><subsection><title></title>Pulleys}
    </subsection><subsection><title></title>Interesting Complications}
</section><section><title></title>Frictional Force}\label{s-Ff}<aside><title>Referenced by</title> <p></p></aside>{\mmr{<xref ref="" text="type-global" />A-chair2}}, \mmr{<xref ref="" text="type-global" />A-chair6}}, \mmr{<xref ref="" text="type-global" />A-chair7}}, \mmr{<xref ref="" />A-fly.balls}}}
</section><section><title></title>Spring Force}\label{s-springs}<aside><title>Referenced by</title> <p></p></aside>{\mmr{<xref ref=""></xref>{d-fma}{<m>F=ma</m>}}, \mmr{<xref ref=""></xref>{d-usesofFma}{uses of <m>F=ma</m>}}, \mmr{<xref ref="" />ss-scales}}}
<section xml:id="s-FA"><title>Applied Force</title>
</section><section><title></title>Putting it Together, <m>F_\mathrm{net}</m>}\label{s-Fnet}
</section><section><title></title>Summary and Homework}

</chapter><chapter><title></title>Energy and the Transfer of Energy}


% YOU ARE HERE



\part{Interesting Uses of Motion, Force, and Energy}

</chapter><chapter><title></title>Momentum: A Better Way to Describe Force}\label{c-momentum}<aside><title>Referenced by</title> <p></p></aside>{\mmr{<xref ref=""></xref>{d-objectinmotion}{objects in motion}}, \mmr{<xref ref="" />sss-inertia}}, \mmr{<xref ref="" />ss-NIII}}, \mmr{<xref ref="" text="type-global" />A-chair6}}}

Useful to include?
<url href=""></url>{https://www.wired.com/2017/06/physics-bullets-versus-wonder-womans-bracelets/}{The Physics of Bullets Vs. Wonder Woman's Bracelets}

</section><section><title></title>Revising Newton's First and Second Laws}

</subsection><subsection><title></title>Inertia and Momentum}\label{ss-inertia}<aside><title>Referenced by</title> <p><xref provisional="" /></p></aside>{sss-inertia}
Recall <xref ref="" />sss-inertia}.

</section><section><title></title>Revising Newton's Third Law: Conservation of Momentum}\label{s-conservemom}<aside><title>Referenced by</title> <p><xref provisional="" /></p></aside>{ss-NIII}

</section><section><title></title>Two-Dimensional Collisions}\label{s-2Dcollisions}<aside><title>Referenced by</title> <p><xref provisional="" /></p></aside>{sss-vectorequations}

pool balls?  What about rolling?
%
\begin{reallife}[bthp]
\hspace{-.2in}
\fcolorbox{black}{green!10}{\begin{minipage}{5.29in} \center
\caption{\label{irl-pool2Dcollision}<idx><h></h><h></h></idx>{Pool!Real Life} 2-D collisions of pool balls.}
\begin{minipage}{4.925in}
\studentD<idx><h sortby="\studentD">\studentD</h></idx> is relaxing with the local physics club, playing pool.  \HeD\ hits the cue ball and counts the number of walls \heD\ can hit in one shot.
\end{minipage}
\begin{realtable}
\dna{collide balls.}
    {where does it hit}
    {<m>90^\circ</m> output}
\end{realtable}
\begin{minipage}{4.925in}
Billiard tables have a lot of interesting physics, which can help us see a wide variety of physics, for example:
<xref ref="" text="title"></xref>[irl-poolnormal]{normal force}, <xref ref="" text="title"></xref>[irl-poolelastic]{elastic versus inelastic collisions}, <xref ref="" text="title"></xref>[irl-poolrotmot]{rotational motion}, and <xref ref="" text="title"></xref>[irl-poolangmom]{angular momentum}.
\end{minipage}

%\flushright
%<assemblage>Return to: </assemblage> <xref provisional=""></xref>[pool]{d-bank-shot}
\end{minipage}}
\end{reallife}
%


</chapter><chapter><title></title>Rotational Motion}

</section><section><title></title>The Equations of Rotational Motion}

%
\begin{reallife}[bthp]
\hspace{-.2in}
\fcolorbox{black}{green!10}{\begin{minipage}{5.29in} \center
\caption{\label{irl-poolrotmot}<idx><h></h><h></h></idx>{Pool!Real Life} Rolling pool balls.}
\begin{minipage}{4.925in}
\studentD<idx><h sortby="\studentD">\studentD</h></idx> is relaxing with the local physics club, playing pool.  \HeD\ hits the cue ball and counts the number of walls \heD\ can hit in one shot.
\end{minipage}
\begin{realtable}
\dna{Roll a striped ball along the table.}
    {Use the stripe to notice the rate of rotation}
    {How does the rotation compare to the translation?}
\dna{Roll a striped ball along the table.}
    {Notice the distance the ball travels}
    {Why does friction slow the ball down instead of just make it turn <m>v=\omega r</m> (no slip)}
\end{realtable}
\begin{minipage}{4.925in}
Billiard tables have a lot of interesting physics, which can help us see a wide variety of physics, for example:
<xref ref="" text="title"></xref>[irl-poolnormal]{normal force}, <xref ref="" text="title"></xref>[irl-poolelastic]{elastic versus inelastic collisions}, <xref ref="" text="title"></xref>[irl-poolrotmot]{rotational motion}, and <xref ref="" text="title"></xref>[irl-poolangmom]{angular momentum}.
\end{minipage}

%\flushright
%<assemblage>Return to: </assemblage> <xref provisional=""></xref>[pool]{d-bank-shot}
\end{minipage}}
\end{reallife}
%

</section><section><title></title>Angular Momentum}

%
\begin{reallife}[bthp]
\hspace{-.2in}
\fcolorbox{black}{green!10}{\begin{minipage}{5.29in} \center
\caption{\label{irl-poolangmom}<idx><h></h><h></h></idx>{Pool!Real Life} Rolling pool balls.}
\begin{minipage}{4.925in}
\studentD<idx><h sortby="\studentD">\studentD</h></idx> is relaxing with the local physics club, playing pool.  \HeD\ hits the cue ball and counts the number of walls \heD\ can hit in one shot.
\end{minipage}
\begin{realtable}
\dna{Roll a striped ball along the table.}
    {Use the stripe to notice the rate of rotation}
    {How does the rotation compare to the translation?}
\end{realtable}
\begin{minipage}{4.925in}
Billiard tables have a lot of interesting physics, which can help us see a wide variety of physics, for example:
<xref ref="" text="title"></xref>[irl-poolnormal]{normal force}, <xref ref="" text="title"></xref>[irl-poolelastic]{elastic versus inelastic collisions}, <xref ref="" text="title"></xref>[irl-poolrotmot]{rotational motion}, and <xref ref="" text="title"></xref>[irl-poolangmom]{angular momentum}.
\end{minipage}

%\flushright
%<assemblage>Return to: </assemblage> <xref provisional=""></xref>[pool]{d-bank-shot}
\end{minipage}}
\end{reallife}
%


</section><section><title></title>Non-inertial Rotational Reference Frames} \label{s-noninertial}<aside><title>Referenced by</title> <p></p></aside>{\mmr{<xref ref="" />ss-noninertial}}, \mmr{<xref ref=""></xref>{d-NewtonInertial}{non-inertial reference frames}}, \mmr{<xref ref="" />ss-NI}}}
<idx><h></h><h></h></idx>{Reference Frames!Inertial}
<idx><h></h><h></h></idx>{Reference Frames!Non-inertial}

Because the Earth <p xml:id=""></p>d-noninertial}{rotates}<aside><title>Referenced by</title> <p><xref provisional="" /></p></aside>{ss-NII}, we are actually in a non-inertial reference frame.  In fact, we can prove that the Earth rotates by observing the effects, such as the <xref ref=""></xref>{d-coriolis}{Coriolis effect}, that in our non-inertial frame seem to require unexplainable forces but which, in a non-rotating frame, follow the expected laws of physics.

</subsection><subsection><title></title>The Coriolis Effect}\label{ss-coriolis}<aside><title>Referenced by</title> <p></p></aside>{\mmr{<xref ref=""></xref>{d-NewtonInertial}{non-inertial reference frames}}, \mmr{<xref ref=""></xref>{d-noninertial}{Non-inertial Rotational Reference Frames}}}

<p xml:id=""></p>d-coriolis}{weather, etc}
<!-- -->\newpar

In her podcast<!-- -->\new{v2.0}{<em></em>Spacepod}}, <em></em>Spacepod}<fn xml:id=""></fn>{Nugent, Carrie (Producer, Host). <em></em>Spacepod} [Audio podcast], episode 89 (19 May, 2017).  Retrieved from <xref ref="" text="title"></xref>{http://spacepod.libsyn.com/}{T4LTFdOxHD5WWzdD}{99}{\nolinkurl{http://spacepod.libsyn.com/}}
on 9 Apr. 2017.} Dr. Carrie Nugent interviews Dr. Andy Thompson about <q>underwater flying objects</q> that investigate the ocean.  He notes that ocean waters, because they are such a large-scale system, can see the effect of the rotation of the Earth.

</subsection><subsection><title></title>The Foucault Pendulum}\label{ss-Foucault}

See <url href=""></url>{https://www.youtube.com/watch?v=sWDi-Xk3rgw}{youtube video} by <url href=""></url>{http://sixtysymbols.com/}{Sixty Symbols}.<!-- -->\new{v2.0}{Foucault video}




</chapter><chapter><title></title>Circular Motion and Centripetal Force}

</section><section><title></title>Circular Motion}
</section><section><title></title>Centripetal Force}\label{s-centripetal}<aside><title>Referenced by</title> <p>Discussion of <xref provisional=""></xref></p></aside>[<m>F=ma</m>]{d-fma}




</chapter><chapter><title></title>Torque and the <m>F=ma</m> of Rotations}\label{c-torque}<aside><title>Referenced by</title> <p><xref provisional="" /></p></aside>{a-NIIIaction}<!-- -->\new{v2.3}{Added an example that is computable here, but helps introduce normal force in \protect{<xref ref="" />s-FN}}.}

</section><section><title></title>Leverage}\label{s-leverarm}<aside><title>Referenced by</title> <p><xref provisional="" /></p></aside>{ss-scales}

</section><section><title></title>Putting it all together, <m>\tau_\mathrm{net}</m>}

</subsection><subsection><title></title>Rotational Equilibrium}

blah blah blah
\phantomsection\label{ss-roteq} Rotational equilibrium: <m>\tau_\mathrm{net} = I \cancelto{0}{\alpha}</m>.  blah blah blah

</subsection><subsection><title></title>Static (Rotational) Equilibrium}

</subsection><subsection><title></title>Dynamic (Rotational) Equilibrium}

<!-- -->\new{v2.3}{Answered \protect{<xref ref="" />ex-ladder2}} and its related problems.}
\begin{example}[p]
\fcolorbox{black}{yellow!10}{\begin{minipage}{4.925in}\setlength{\parskip}{3pt}
\caption{\label{ex-ladder2} \studentC<idx><h sortby="\studentC">\studentC</h></idx> uses a ladder}
\begin{quote}
\studentC\ leans a <m>22.7\unit{kg}</m> ladder against a wall at an angle of <m>75.5^\circ</m>, consistent with \protect{<url href=""></url>{https://www.osha.gov/}{OSHA}} standard \protect{<url href=""></url>{https://www.osha.gov/pls/oshaweb/owadisp.show_document?p_table=standards&p_id=10839}{1926.1053(a)(1)(ii)}}.
The coefficient of friction between the ladder and the floor is <m>\mu_f=0.31</m>.
The coefficient of friction between the ladder and the wall is <m>\mu_w=0.19</m>.
Use the rotational and translational equilibrium to determine if the ladder slides.
\end{quote}

Since we are asked to distinguish between two cases that cannot both be true, we should assume one (the easier one to calculate is that the ladder does not slip) and then verify that the result is consistent with that assumption.

<em></em>What do we know?}
We know that the floor has a normal force <m>(F_{Nf})</m> upwards and a frictional force <m>(F_{ff})</m> to the left.
We know that the wall
\\[2pt]
\begin{minipage}{3.2in}
has a normal force <m>(F_{Nw})</m> to the right and a frictional force <m>(F_{fw})</m> up (keeping the ladder from sliding down).
We know the weight is
<m> F_g = mg = (22.7\unit{kg})(9.81\unitfrac{m}{s^2}) = \sig{222}{.69}{N} </m>
<em></em>What do we want to know?}  We want to know about the the magnitudes of both normal
\end{minipage}
\hfill
\begin{minipage}{100pt}
\begin{picture}(100,100)(-10,-5)
% Dimensions and offset: (width,height)(x offset,y offset)
% Insert picture commands (\line,\circle, etc...) here:
\put(0,0){\line(0,1){100}}
\put(0,0){\line(1,0){75}}
\put(20,0){\color{blue}\line(-1,4){20}}     % ladder
\put(10,40){\color{red}\vector(0,-1){30}}   % Fg
\put(20,1){\color{red}\vector(0,1){25}}     % FNf
\put(19,1){\color{red}\vector(-1,0){12}}
\put(1,80){\color{red}\vector(1,0){12}}
\put(1,81){\color{red}\vector(0,1){8}}
\end{picture}
\end{minipage}
%\hfill {}
\\[3pt]
forces and both frictional forces.
Can we easily deduce the magnitude of <m>F_{Nf}</m>? <xref ref="" text="type-global" />A-ladderNf}.

<em></em>How are these related?}  The forces acting on any body are related by static <xref ref="" text="title"></xref>[ss-transeq]{translational equilibrium}
<md>
x: \hspace{.5cm} 0 & = & \cancel{0}{F_{gx}} + \zero{F_{Nfx}}{0} + F_{ffx} + F_{Nwx} + \cancelto{0}{F_{fwx}} \\
y: \hspace{.5cm} 0 & = & F_{gy} + F_{Nfy} + \cancelto{0}{F_{ffy}} + \cancelto{0}{F_{Nwy}} + F_{fwy}
</md>
and static <xref ref="" text="title"></xref>[ss-roteq]{rotational equilibrium}, assuming the pivot point is at the ground, and using the relationship <m>F_f=\mu F_N</m>, we find
<md>
0 & = & \tau_{g} + \cancelto{0}{\tau_{Nf}} + \cancelto{0}{\tau_{ff}} + \tau_{Nw} + \tau_{fw} \\
0 & = & \left[ F_g \frac{l}{2} \sin 14.5^\circ \right] + \left[ - F_{Nw} l \sin(75.5^\circ) \right] + \left[ - F_{fw} l \sin(14.5^\circ) \right] \\
F_{Nw} & = & \left[ F_g \frac{l}{2} \sin 14.5^\circ \right] / \left[  l \sin(75.5^\circ) + \mu_w l \sin(14.5^\circ) \right]
</md>
%<em></em>Free-Body Diagrams:}  Since the picture is so simple, we will not draw the free-body diagram.


{}\hfill {\footnotesize\autoref*{ex-ladder2} continued on next page<ellipsis />}
\end{minipage}}
\end{example}
\begin{example}[p]
\fcolorbox{black}{yellow!10}{\begin{minipage}{4.925in}\setlength{\parskip}{3pt}
{\footnotesize \autoref*{ex-ladder2} continued from previous page<ellipsis />}

<em></em>Concepts to Consider:}  First, the length of the ladder cancels from the expression; what matters is the angle at which it is propped.

Second, every force value will be linearly dependent on the mass of the ladder.  So once we solve this problem, we can easily scale the answers to any mass.

Third, the friction with the wall is, by far, the smallest effect and it might be interesting to approximate all of this with <m>\mu_w=0</m>.  You can check your calculation against <xref ref="" text="type-global" />A-nowall}.

<em></em>Solution to the example:}  When we worry about significant figures,
<md>
F_{Nw} & = & \frac{\left[ (\sig{222}{.7}{N})(\txtfrac{1}{2}) (0.\sig{250}{4}{}) \right]}{\left[  (0.\sig{968}{2}{}) + (0.19) (0.\sig{250}{4}{}) \right]}
\ = \ \frac{\left[ (\sig{27.8}{8}{N})\right]}{\left[  (0.\sig{968}{2}{}) + (0.0\sig{47}{6}{}) \right]} \\
F_{Nw} & = & \frac{\left[ (\sig{27.8}{8}{N})\right]}{\left[  (\sig{1.015}{7}{}) \right]}
\ = \ \sig{27.4}{4}{N} \\
F_{fw,\mathrm{max}} & = & (0.19)(\sig{27.4}{4}{N}) \ = \ \sig{5.2}{15}{N} \\
F_{Nf} & = & F_g - F_{fw} = (\sig{222}{.7}{N})-(\sig{5.2}{15}{N}) \ = \ \sig{217}{.5}{N} \\
F_{ff,\mathrm{max}} & = & (0.31)(\sig{217}{.5}{N}) \ = \ \sig{672}{.4}{N}
</md>
Since <m>F_{ff} >F_{Nw}</m>, the friction is sufficient to hold the ladder in place, as assumed.

%\begin{quote}
<em></em>Aside:} Since <m>F_{ff}</m> only needs to be <m>\sig{27.4}{4}{N}</m> to hold the ladder in place, it is possible for the ladder to not slide on a floor that only has
<m>\mu_\mathrm{min} = (\sig{27.4}{4}{N})/(\sig{217}{.5}{N}) = 0.\sig{126}{2}{}</m>; but that would not allow a person to climb the ladder.

<em></em>Homework:} Homework problem~<xref ref="" text="type-global" />h-ladderC} asks you to determine if the ladder slides when \studentC\ climbs to different locations on the ladder.
%\end{quote}
\flushright
<assemblage>Return to: </assemblage>{\mmr{<xref ref="" text="type-global" />se-ladderN}}, \mmr{<xref ref="" />ss-roteq}}}
\end{minipage}}
\end{example}

</section><section><title></title>Torsion}\label{s-torsion}<aside><title>Referenced by</title> <p><xref provisional="" /></p></aside>{s-FT}<!-- -->\new{v2.4}{Created this section}

</section><section><title></title>Summary and Homework}

</subsection><subsection><title></title>Summary of Concepts and Equations}<!-- -->\new{v2.3}{Created this section}

<ellipsis />

</subsection><subsection><title></title>Conceptual Questions}<todo></todo>Add conceptual problems.}
%\vspace{-24pt}
<ol>
</p></li><li><p><ellipsis />
</ol>
</subsection><subsection><title></title>Problems}<!-- -->\new{v2.3}{Added problems.}<todo></todo>Add more problems.}
%\vspace{-24pt}
<ol>
 </p></li><li><p>\label{h-ladderC} \studentC\ leans a <m>22.7\unit{kg}</m> ladder against a wall at an angle of <m>75.5^\circ</m>, consistent with \protect{<url href=""></url>{https://www.osha.gov/}{OSHA}} standard \protect{<url href=""></url>{https://www.osha.gov/pls/oshaweb/owadisp.show_document?p_table=standards&p_id=10839}{1926.1053(a)(1)(ii)}}.<!-- -->\new{v2.3}{Answered \protect{<xref ref="" text="type-global" />h-ladderC}} and its related problems.}
The coefficient of friction between the ladder and the floor is <m>\mu_f=0.31</m>.
The coefficient of friction between the ladder and the wall is <m>\mu_w=0.19</m>.
Use the rotational and translational equilibrium to determine if the ladder slides when \studentC\ (<m>\massC</m>) climbs to
<ol>
</p></li><li><p> the third-rung from the top of the ladder, so that he is <m>1.53\unit m</m> from the bottom of the ladder.
    (See <xref ref="" text="type-global" />A-nowallC} for that answers if <m>\mu_w = 0</m>.)
\begin{ForMe}
\color{blue} Answers:
<md>
F_{Nw} & = & \sig{163}{.9}{N} \\
F_{fw,\mathrm{max}} & = & (0.19)(\sig{163}{.9}{N}) \ = \ \sig{31}{.14}{N} \\
F_{Nf} & = & \sig{1074}{.4}{N} \\
F_{ff,\mathrm{max}} & = & \sig{333}{.0}{N} < \sig{163}{.9}{N}
</md>
<m>\mu_\mathrm{min} = 0.\sig{152}{56}{}</m>
\color{black}
\end{ForMe}
</p></li><li><p> the third-rung from the bottom of the ladder, so that he is <m>0.914\unit m</m> from the bottom of the ladder.
\begin{ForMe}
\color{blue}
Answers:
<md>
F_{Nw} & = & \sig{108}{.97}{N} \\
F_{fw,\mathrm{max}} & = & (0.19)(\sig{108}{.97}{N}) \ = \ \sig{20}{.70}{N} \\
F_{Nf} & = & \sig{1084}{.9}{N} \\
F_{ff,\mathrm{max}} & = & \sig{336}{.3}{N} < \sig{108}{.97}{N}
</md>
<m>\mu_\mathrm{min} = 0.\sig{100}{45}{}</m>

If <m>\mu_w = 0</m>.
<md>
F_{Nw} & = & \sig{114}{.3}{N} \\
F_{fw,\mathrm{max}} & = & 0 \unit N \\
F_{Nf} & = & \sig{1105}{.6}{N} \\
F_{ff,\mathrm{max}} & = & \sig{342}{.7}{N} < \sig{114}{.3}{N}
</md>
<m>\mu_\mathrm{min} = 0.\sig{103}{4}{}</m>
\color{black}
\end{ForMe}
</ol>
</ol>


</chapter><chapter><title></title>Energy of Rotating Objects}
</section><section><title></title>Rotational Kinetic Energy}
pool balls

</chapter><chapter><title></title>The Gravitational Force on a Large Scale}\label{c-gravity}<aside><title>Referenced by</title> <p></p></aside>{\mmr{<xref ref=""></xref>{d-accgrav}{freefall}}, \mmr{<xref ref=""></xref>{d-fundamental}{fundamental forces}}}

</section><section><title></title>Gravitational Force and Field}\label{s-Gfield}<aside><title>Referenced by</title> <p>Discussion of <xref provisional=""></xref></p></aside>[<m>F=ma</m>]{d-fma}<!-- -->\new{v2.3}{Added some placeholders}

The value of the acceleration due to gravity  varies according to the mass and size of any celestial body.<todo></todo>Reference a table of <m>g</m> on other planets and compute the weight of a space craft at each planet.}
This means that, as was seen in <xref ref="" text="type-global" />se-gworld}, your weight can change even when your mass remains the same.
\begin{sample}
</p></li><li><p>\label{se-gplanets} In conversation with a visiting alien, \studentX<idx><h sortby="\studentX">\studentX</h></idx>, you find that \studentX\ has been to the moon and several planets both within and outside of our solar system.  In addition to the Earth, \studentX\ has visited our moon, Mars, Pluto, and Planet X.  Using <xref ref="" />t-gplanets}, compute \studentX's weight are each location, assuming \hisX\ mass is \massX.
<ol>
</p></li><li><p>[Earth] <m>F_g = (\massX)\left[ \frac{ G M_E}{R_E^2} \right] = (\massX)(9.825\unitfrac{m}{s^2}) \ = \ \sig{933}{.4}{N}</m>
</p></li><li><p>[moon] <m>F_g = (\massX)\left[ \frac{ G M_m}{R_m^2} \right] = (\massX)(9.782\unitfrac{m}{s^2}) \ = \ \sig{929}{.3}{N}</m>
</p></li><li><p>[Mars] <m>F_g = (\massX)\left[ \frac{ G M_M}{R_M^2} \right] = (\massX)(9.763\unitfrac{m}{s^2}) \ = \ \sig{927}{.5}{N}</m>
</p></li><li><p>[Pluto] <m>F_g = (\massX)\left[ \frac{ G M_P}{R_P^2} \right] = (\massX)(9.763\unitfrac{m}{s^2}) \ = \ \sig{927}{.5}{N}</m>
</p></li><li><p>[Planet X] <m>F_g = (\massX)\left[ \frac{ G M_X}{R_X^2} \right] = (\massX)(9.763\unitfrac{m}{s^2}) \ = \ \sig{927}{.5}{N}</m>
</ol>
\end{sample}
%
\begin{table}[bhtp]
\hrule\hrule
\begin{center}
\caption[Properties of various celestial bodies]{\label{t-gplanets} Properties of various celestial bodies.
<assemblage>Return to: </assemblage> <xref provisional="" />{se-gplanets}
}
\begin{tabular}{lccr}
Planet & Mass (kg) & Mean Radius (m) & <m>g (\unitfrac{m}{s^2})</m> \\
\end{tabular}
\end{center}
\hrule\hrule
\end{table}
%


</subsection><subsection><title></title>Inertial Mass versus Gravitational Mass}\label{ss-equivmm}<aside><title>Referenced by</title> <p><xref provisional="" /></p></aside>{ss-weightmass}<!-- -->\new{v2.2}{Moved this here, might need to move it back.}

</section><section><title></title>Gravitational Potential Energy} \label{s-PEG}<aside><title>Referenced by</title> <p><xref provisional="" /></p></aside>{ss-PEg}

Recall <xref ref="" text="type-global" />ss-PEg}

</section><section><title></title>Making Connections}\label{s-Gconnection}<aside><title>Referenced by</title> <p><xref provisional="" /></p></aside>{s-Econnection}

<me> \begin{array}{ccccc}
& & \vec F = m \vec g & & \\
& \deq F = G \frac{m_1 m_2}{R^2} & \leftrightarrow & \deq g = G \frac{m}{R^2} & \\
\Delta \PE = -\vec F \cdot \Delta\vec x & \updownarrow & & \updownarrow & \mbox{\scriptsize [for later]} \\
& \deq \PE = G \frac{m_1 m_2}{R} & \leftrightarrow & \mbox{[for later]} & \\
& & \mbox{\scriptsize [for later]} & &
\end{array} </me>
(Look ahead to the parallel with the electrical interaction in <xref ref="" />s-Econnection}.)

</section><section><title></title>Orbits}


\part{Making Waves}

</chapter><chapter><title></title>Fluids}<!-- -->\new{v2.2}{Placeholder}
</section><section><title></title>Density}\label{s-density}<idx></idx>{Density}<aside><title>Referenced by</title> <p><xref provisional="" /></p></aside>{ss-weightmass}

</section><section><title></title>Surface Tension}\label{s-surface-tension}<aside><title>Referenced by</title> <p>Discussion of <xref provisional=""></xref></p></aside>{d-surf-tension}



</chapter><chapter><title></title>Oscillations}\label{c-SHM}
</section><section><title></title>Oscillating Springs}\label{c-SHMspring}<aside><title>Referenced by</title> <p>Discussion of <xref provisional=""></xref></p></aside>[<m>F=ma</m>]{d-fma}
</section><section><title></title>Oscillating Pendulums}\label{c-SHMpend}

</section><section><title></title>Other Examples of Oscillations}\label{s-SHMother}

On 13 April, 2017,<!-- -->\new{v2.3}{New source of info}
<url href=""></url>{http://www.cbc.ca/podcasting}{CBC Broadcasting} published a
<url href=""></url>{http://www.cbc.ca/podcasting/includes/quirks.xml}{<em></em>Quirks and Quarks}} episode discussing how we can find
<url href=""></url>{https://podcast-a.akamaihd.net/mp3/podcasts/quirks_20170415_12100.mp3}{solutions to health issues caused by swaying office towers and vibrating floors}.

</chapter><chapter><title></title>Sound}
</subsection><subsection><title></title>Musical Instruments}\label{ss-stringed-instruments} <aside><title>Referenced by</title> <p><xref provisional="" /></p></aside>{A-swing-tension}



\part{Is It Hot in Here?}

</chapter><chapter><title></title>The flow of thermal energy}

\phantomsection\label{find-heatwarm}
Energy is a noun<idx><h></h><h></h></idx>{Energy!noun}; objects can <em></em>have} energy.  <p xml:id=""></p>d-heatverb}{Heat is a verb}<idx><h></h><h></h></idx>{Heat!verb}; heating is a process of <em></em>exchanging} energy.  Recall our <xref ref=""></xref>{d-forcenoun}{discussions of force}<idx><h></h><h></h></idx>{Force!noun} and <xref ref=""></xref>{d-workverb}{work}<idx><h></h><h></h></idx>{Work!verb}.

</section><section><title></title>Specific Heat Capacity}\label{s-specificheat}

<p xml:id=""></p>d-heatwarm}{Heating (positive <m>Q</m>)} can warm (positive <m>\Delta T</m>) a material.
<men>\label{eq-Q=mcDT}
Q = m c \, \Delta T
</men>
but <xref ref="" />eq-Q=mL} (as one example) shows that it is possible to heat (positive <m>Q</m>) a material without warming it (constant <m>T</m>). When we get to <xref ref="" />s-PV} we will see other examples of <q>isothermal processes</q> that have a non-zero <m>Q</m> (heat the system or heat the surroundings) without warming or cooling the system.

</section><section><title></title>Latent Heat}

Heating might also change the phase of a material.<aside><title>Referenced by</title> <p>Discussion of <xref provisional=""></xref></p></aside>[heating versus warming]{d-heatwarm}
<men>\label{eq-Q=mL}
Q = \pm mL
</men>

</section><section><title></title>The Flow of Thermal Energy}

</subsection><subsection><title></title>Thermal Conductivity}\label{ss-thermalconductivity}<aside><title>Referenced by</title> <p><xref provisional="" /></p></aside>{s-story}

<men>\label{eq-thermalconductivity}
\frac{Q}{\Delta t} = \kappa A \, \frac{\Delta T}{\Delta x}
</men>

\begin{example}
\fcolorbox{black}{yellow!10}{\begin{minipage}{4.925in}
\caption{\label{ex-baking}\studentA\protect{<idx><h sortby="\studentA">\studentA</h></idx>} warms \hisA\ oven.}
\studentA\protect{<idx><h sortby="\studentA">\studentA</h></idx>} decides to bake some bread for the dinner party at \studentB\protect{<idx><h sortby="\studentB">\studentB</h></idx>}'s house, but \heA\ is on a tight schedule.  In order to set \hisA\ schedule, \heA\ needs to know how long it will take \hisA\ oven to <xref ref="" text="title"></xref>[find-heatwarm]{warm up}.

<assemblage>Return to: </assemblage> <xref provisional="" />{s-story}
\end{minipage}}
\end{example}

</subsection><subsection><title></title>Convection}
</subsection><subsection><title></title>Radiation}

</chapter><chapter><title></title>Ideal Gas Law}
</section><section><title></title><m>P</m>-<m>V</m> Diagrams}\label{s-PV}<aside><title>Referenced by</title> <p>Discussion of <xref provisional=""></xref></p></aside>[heating versus warming]{d-heatwarm}

\part{Let There be Light!}

</chapter><chapter><title></title>The Electrical Interaction}\label{c-electric}<aside><title>Referenced by</title> <p>Discussion of <xref provisional=""></xref></p></aside>[fundamental forces]{d-fundamental}
</section><section><title></title>Electrical Charge}\label{s-Echarge}<!-- -->\new{v2.1}{Decide where this should go.}

</section><section><title></title>The Big Picture}

</subsection><subsection><title></title>Electric Forces and Fields}\label{ss-Efield}<aside><title>Referenced by</title> <p></p></aside>{\mmr{<xref ref="" />sss-vectorequations}}, \mmr{<xref ref=""></xref>{d-fma}{<m>F=ma</m>}}}

pst-electricfield

</subsubsection><subsubsection><title></title>Examples}

</subsection><subsection><title></title>Electric Forces, Fields, and Potential Energy}

</subsection><subsection><title></title>Electric Fields, Potential Energy, and Potential}

</section><section><title></title>Making Connections}\label{s-Econnection}<aside><title>Referenced by</title> <p><xref provisional="" /></p></aside>{s-Gconnection}

<me> \begin{array}{ccccc}
& & \vec F = q \vec E & & \\
& \deq F = k \frac{q_1 q_2}{r^2} & \leftrightarrow & \deq E = k \frac{q}{r^2} & \\
\Delta \PE = -\vec F \cdot \Delta\vec x & \updownarrow & & \updownarrow & \Delta V = -\vec E \cdot \Delta\vec x  \\
& \deq \PE = k \frac{q_1 q_2}{r} & \leftrightarrow & \deq V = k \frac{q}{r} & \\
& & \Delta \PE = q \Delta V & &
\end{array} </me>
(Recall the parallel with the gravitational interaction in <xref ref="" />s-Gconnection}.)

</chapter><chapter><title></title>Electricity}

</chapter><chapter><title></title>The Magnetic Interaction}

pst-magneticfield

</chapter><chapter><title></title><q>Magnicity?</q>}

</chapter><chapter><title></title>Light}

</chapter><chapter><title></title>Optics}

\part{What Have You Done for Me Lately?}

</chapter><chapter><title></title>Relativity}
</chapter><chapter><title></title>Quantum Mechanics}<!-- -->\new{v2.1}{Decide if these subsections should be chapters in and of themselves.  These are now labeled.}
</section><section><title></title>Atomic Physics} </subsection><subsection><title></title>The Periodic Table and Quantum Numbers}
</section><section><title></title>Nuclear Physics} </subsection><subsection><title></title>Nuclear Decay}\label{ss-nucleardecay}
</subsection><subsection><title></title>The Strong Nuclear Force}\label{ss-strong}<aside><title>Referenced by</title> <p>Discussion of <xref provisional=""></xref></p></aside>[fundamental forces]{d-fundamental}
</subsection><subsection><title></title>The Weak Nuclear Force}\label{ss-weak}<aside><title>Referenced by</title> <p>Discussion of <xref provisional=""></xref></p></aside>[fundamental forces]{d-fundamental}
</section><section><title></title>Particle Physics}\label{s-particle}
</subsection><subsection><title></title>Field Theory}
</subsection><subsection><title></title>Quantum Electrodynamics}\label{ss-QED}<aside><title>Referenced by</title> <p>Discussion of <xref provisional=""></xref></p></aside>[fundamental forces]{d-fundamental}
</subsection><subsection><title></title>Quantum Chromodynamics}\label{ss-QCD}<aside><title>Referenced by</title> <p>Discussion of <xref provisional=""></xref></p></aside>[fundamental forces]{d-fundamental}
</subsection><subsection><title></title>The Standard Model}\label{ss-StandardModel}
</subsection><subsection><title></title>Particle Decay}\label{ss-particledecay}
</chapter><chapter><title></title>Condensed Matter}
</chapter><chapter><title></title>Astronomy}
</chapter><chapter><title></title>Cosmology}

\part{Supplements}

</chapter><chapter><title></title>Deeper Dive}\label{c-revisted}<!-- -->\new{v2.1}{This chapter should mirror \protect{<xref ref="" />c-physics}}.}

This is where I will put the fuller explanations.

</subsection><subsection><title></title>The Sun}\label{sss-sun}
The bright, shiny sun, which keeps us all alive, is a nice example of a rather complex system that allows us to verify our various theories of the world around us.  We can consider the existence of a star in three phases: the birth of a star, the life of the star, and the death of the star.

</subsubsection><subsubsection><title></title>The Birth of a Star}
</subsubsection><subsubsection><title></title>The Life of a Star}
</subsubsection><subsubsection><title></title>The Death of a Star}


</subsection><subsection><title></title>Kitchen Appliances}
</subsubsection><subsubsection><title></title>Oven}
</subsubsection><subsubsection><title></title>Refrigerator}
</subsubsection><subsubsection><title></title>Microwave}
</subsubsection><subsubsection><title></title>Television}

</subsection><subsection><title></title>Automobile}
</subsubsection><subsubsection><title></title>Coolant and Antifreeze}
</subsubsection><subsubsection><title></title>Tires}
</subsubsection><subsubsection><title></title>Torque}

</subsection><subsection><title></title>Cool Ideas}
</subsubsection><subsubsection><title></title>Black Holes}\label{sss-blackhole2}<aside><title>Referenced by</title> <p><xref provisional="" /></p></aside>{ss-weightmass}

On 7 April, 2017,<!-- -->\new{v2.3}{New source of info}
<url href=""></url>{http://www.cbc.ca/podcasting}{CBC Broadcasting} published a
<url href=""></url>{http://www.cbc.ca/podcasting/includes/quirks.xml}{<em></em>Quirks and Quarks}} episode discussing how we can
<url href=""></url>{https://podcast-a.akamaihd.net/mp3/podcasts/quirks_20170408_51226.mp3}{turn our planet into a giant telescope to get a photo of a black hole}.
The results should be available by the early 2018.<todo></todo>Follow-up in 2018 to find the results.}

</subsubsection><subsubsection><title></title>Quantum Mechanics}
</subsubsection><subsubsection><title></title>Relativity}
</subsubsection><subsubsection><title></title>String Theory}



</chapter><chapter><title></title>Podcasts and Videos}\label{c-videos}\label{c-podcasts}

</section><section><title></title>Podcasts}\label{s-podcasts}
<xref ref="" text="title"></xref>{http://spacepod.libsyn.com/}{T4LTFdOxHD5WWzdD}{99}{Spacepod with Carrie Nugent} \\
<url href=""></url>{http://www.sciencefriday.com/}{Science Friday with Ira Flatow}

</section><section><title></title>Videos}\label{s-videos}
<url href=""></url>{http://physicsfootnotes.com/}{Physics Footnotes} \\
<url href=""></url>{http://sixtysymbols.com/}{Sixty Symbols}

</section><section><title></title>Websites}\label{s-websites}
<url href=""></url>{http://www.aldakavlilearningcenter.org/practice/flame-challenge}{The Flame Challenge}

</chapter><chapter><title></title>Answers to Interactive Questions}

\begin{AIQ}
</p></li><li><p>\label{A-hbf} TOOK <assemblage>Return to: </assemblage> <xref provisional="" />{IQ-holdbook}
</p></li><li><p>\label{A-chair1} TOOK <assemblage>Return to: </assemblage> <xref provisional="" />{irl-NI}
</p></li><li><p>\label{A-chair2} TOOK  <assemblage>Return to: </assemblage> <xref provisional="" />{irl-NI}
</p></li><li><p>\label{A-weight.loss} TOOK <assemblage>Return to: </assemblage> <xref provisional="" />{irl-scale}
</p></li><li><p> \label{A-ladderNf} Since the full weight of the ladder, <m>F_g = \sig{222}{.69}{N}</m>, is still pressing downwards into the floor (as a normal force), it is tempting to say that <xref ref="" text="title"></xref>[ss-NIII]{Newton's third law} implies that the floor pushes the ladder upwards with a normal force of <m>\sig{222}{.69}{N}</m> but this would not account for the frictional force on the wall, <m>F_{fw}</m>.  If there were no friction between the ladder and the wall, then we could deduce <m>F_{Nf}</m>, but at this point, we cannot. <assemblage>Return to: </assemblage> <xref provisional="" />{ex-ladder2}
</p></li><li><p>\label{A-hbnof}  TOOK  <assemblage>Return to: </assemblage> <xref provisional="" />{IQ-holdbook}
</p></li><li><p>\label{A-netF-a} TOOK <assemblage>Return to: </assemblage> <xref provisional="" />{se-weightA}
</p></li><li><p>\label{A-chair3} TOOK <assemblage>Return to: </assemblage> <xref provisional="" />{irl-NI}
</p></li><li><p>\label{A-weight.gain} TOOK <assemblage>Return to: </assemblage> <xref provisional="" />{irl-scale}
</p></li><li><p>\label{A-nowall} If we consider <m>\mu_w\rightarrow 0</m>, then <m>F_{fw}=0\unit N</m>,  <m>\vec F_{Nf} = -\vec F_g = \sig{222}{.7}{N} \jhat</m>, and <m>\vec F_{Nw} = - \vec F_{ff} = \sig{28.7}{9}{N} \ihat</m>.  In this case, <m>\mu_f</m> could be as small as <m>0.\sig{129}{3}{}</m> and still hold the ladder in place, unless \studentC<idx><h sortby="\studentC">\studentC</h></idx> climbs the ladder, in which case see <xref ref="" text="type-global" />A-nowallC}.  <assemblage>Return to: </assemblage> <xref provisional="" />{ex-ladder2}
</p></li><li><p>\label{A-true1} TOOK  <assemblage>Return to: </assemblage> <xref provisional="" />{IQ-holdbook}
</p></li><li><p>\label{A-chair4} TOOK <assemblage>Return to: </assemblage> <xref provisional="" />{irl-NI}
</p></li><li><p>\label{A-scale.increase} TOOK  <assemblage>Return to: </assemblage> <xref provisional="" />{irl-scale}
</p></li><li><p>\label{A-nowallC} If we consider <m>\mu_w\rightarrow 0</m> with \studentC<idx><h sortby="\studentC">\studentC</h></idx> (<m>m=\massC</m>) at the third-rung-from-the-top of the ladder, (<m>1.53\unit m</m> up the ladder), then <m>F_{fw}=0\unit N</m>,  <m>\vec F_{Nf} = \sig{1105}{.6}{N} \jhat</m>, and <m>\vec F_{Nw} = - \vec F_{ff} = \sig{171}{.97}{N} \ihat</m>.  In this case, <m>\mu_f</m> could be as small as <m>0.\sig{155}{5}{}</m> and still hold the ladder in place. <assemblage>Return to: </assemblage>{\mmr{<xref ref="" text="type-global" />A-nowall}}, \mmr{<xref ref="" />ex-ladder2}}}
</p></li><li><p>\label{A-gworld} TOOK <assemblage>Return to: </assemblage>{\mmr{<xref ref="" text="type-global" />A-gpeaks}}, \mmr{<xref ref="" />t-gworld}}}
</p></li><li><p>\label{A-false1} TOOK <assemblage>Return to: </assemblage> <xref provisional="" />{IQ-holdbook}
</p></li><li><p>\label{A-chair5} TOOK <assemblage>Return to: </assemblage> <xref provisional="" />{irl-NI}
</p></li><li><p>\label{A-scale.measure} TOOK  <assemblage>Return to: </assemblage> <xref provisional="" />{irl-scale}
</p></li><li><p>\label{A-gpeaks} TOOK  <assemblage>Return to: </assemblage> <xref provisional="" />{t-gworld}
</p></li><li><p>\label{A-falls}  TOOK   <assemblage>Return to: </assemblage> <xref provisional="" />{IQ-holdbook}
</p></li><li><p>\label{A-chair6} TOOK <assemblage>Return to: </assemblage> <xref provisional="" />{irl-NI}
</p></li><li><p>\label{A-fly.balls} TOOK <assemblage>Return to: </assemblage> <xref provisional="" />{irl-nonparabolic}
</p></li><li><p>\label{A-hitY} TOOK  <assemblage>Return to: </assemblage> <xref provisional="" />{IQ-holdbook}
</p></li><li><p>\label{A-noFT} TOOK <assemblage>Return to: </assemblage> <xref provisional="" />{A-chair5}
</p></li><li><p>\label{A-scale.ramp} TOOK   <assemblage>Return to: </assemblage> <xref provisional="" />{irl-scale}
</p></li><li><p>\label{A-pitches.side} TOOK  <assemblage>Return to: </assemblage> <xref provisional="" />{irl-nonparabolic}
</p></li><li><p>\label{A-hitN} TOOK <assemblage>Return to: </assemblage> <xref provisional="" />{IQ-holdbook}
</p></li><li><p>\label{A-chair7} TOOK <assemblage>Return to: </assemblage> <xref provisional="" />{irl-NI}
</p></li><li><p>\label{A-pitches.top} TOOK <assemblage>Return to: </assemblage> <xref provisional="" />{irl-nonparabolic}
</p></li><li><p>\label{A-landedY} TOOK  <assemblage>Return to: </assemblage> <xref provisional="" />{IQ-holdbook}
</p></li><li><p>\label{A-gravity}  TOOK <assemblage>Return to: </assemblage> <xref provisional="" />{A-hbf}
</p></li><li><p>\label{A-chair8} TOOK <assemblage>Return to: </assemblage> <xref provisional="" />{irl-NI}
</p></li><li><p>\label{A-pool-roll} TOOK <assemblage>Return to: </assemblage> <xref provisional="" />{irl-poolcushion}
</p></li><li><p>\label{A-landedN} TOOK <assemblage>Return to: </assemblage> <xref provisional="" />{IQ-holdbook}
</p></li><li><p>\label{A-FT} TOOK  <assemblage>Return to: </assemblage> <xref provisional="" />{A-chair5}
</p></li><li><p>\label{A-pool-bumper} TOOK <assemblage>Return to: </assemblage> <xref provisional="" />{irl-poolcushion}.
</p></li><li><p>\label{A-zero} TOOK <assemblage>Return to: </assemblage> <xref provisional="" />{IQ-holdbook}
</p></li><li><p>\label{A-firstfall} TOOK  <assemblage>Return to: </assemblage> <xref provisional="" />{irl-freefall}
</p></li><li><p>\label{A-floor}  TOOK  <assemblage>Return to: </assemblage> <xref provisional="" />{se-FNB}
</p></li><li><p>\label{A-firstwhy} TOOK  <assemblage>Return to: </assemblage> <xref provisional="" />{irl-freefall}
</p></li><li><p>\label{A-noncue} TOOK <assemblage>Return to: </assemblage> <xref provisional="" />{irl-poolcushion}
</p></li><li><p>\label{A-one} TOOK  <assemblage>Return to: </assemblage> <xref provisional="" />{IQ-holdbook}
</p></li><li><p>\label{A-fallv} TOOK  <assemblage>Return to: </assemblage> <xref provisional="" />{irl-freefall}
</p></li><li><p>\label{A-pool-spin} TOOK <assemblage>Return to: </assemblage> <xref provisional="" />{irl-poolcushion}
</p></li><li><p>\label{A-second} TOOK  <assemblage>Return to: </assemblage> <xref provisional="" />{se-FNB}
</p></li><li><p>\label{A-falla} TOOK  <assemblage>Return to: </assemblage> <xref provisional="" />{irl-freefall}
</p></li><li><p>\label{A-pool-later} TOOK <assemblage>Return to: </assemblage> <xref provisional="" />{irl-poolcushion}
</p></li><li><p>\label{A-two} TOOK  <assemblage>Return to: </assemblage> <xref provisional="" />{IQ-holdbook}
</p></li><li><p>\label{A-third} TOOK  <assemblage>Return to: </assemblage> <xref provisional="" />{se-FNB}

</p></li><li><p>\label{A-swing-tension} TOOK <assemblage>Return to: </assemblage>{\mmr{<xref ref="" />irl-tension}}, \mmr{<xref ref="" text="type-global" />A-chandelier-tension}}}
</p></li><li><p>\label{A-fan-tension} TOOK <assemblage>Return to: </assemblage> <xref provisional="" />{irl-tension}
</p></li><li><p>\label{A-chandelier-tension} TOOK   <assemblage>Return to: </assemblage> <xref provisional="" />{A-fan-tension}
\end{AIQ}

</chapter><chapter><title></title>Adventures}


Throughout the book, there are examples and adventures.  The follow-up stories are contained below.
\begin{Story}
</p></li><li><p>\label{a-parkandwalk}  TOOK
</p></li><li><p>\label{a-NIIIaction} TOOK
</p></li><li><p>\label{a-coastindrive} TOOK
</p></li><li><p>\label{a-NIIIreaction} TOOK
</p></li><li><p>\label{a-coastinneutral} TOOK
</p></li><li><p>\label{a-NIIIconcern} TOOK
</p></li><li><p>\label{a-NIdrive} TOOK
</p></li><li><p>\label{a-NIIIexperiment} TOOK
</p></li><li><p>\label{a-nogas} TOOK
</p></li><li><p>\label{a-NIIIrestraint} TOOK
</p></li><li><p>\label{a-parkandwalk2}  TOOK
</p></li><li><p>\label{a-NIIIfaculty} TOOK
</p></li><li><p>\label{a-intosunset} TOOK
</p></li><li><p>\label{a-NIIIsecurity} TOOK
</p></li><li><p>\label{a-NIresult} TOOK
</p></li><li><p>\label{a-guilty} TOOK
</p></li><li><p>\label{a-nogas2} TOOK
</p></li><li><p>\label{a-intosunset2} TOOK
\end{Story}

%%%%%%%%%%%%%%%%%%%%%%%%%%%%%%%%%%%%%%%%%%%%%%%%%%%%%%%%%%%%%%%%%%%%%%%%%%%%%%%%%%%%%%%%%%%%%%%%%%%%%%

</chapter><chapter><title></title>Characters}

This textbook has five characters who follow you throughout the book.  They appear in the examples and some homework problems.  They also remember previous experiences.  I need to adjust the examples in <xref ref="" />c-force} such that the people pushing boxes are helping the reader rearrange furniture.

The index lists<todo></todo>The index will recognize the people in two different formats.  One is by my name for them, which is <backslash />studentX (where X is A, B, C, D, <ellipsis /> Z).  The other is by the name assigned to that variable.  So these show up in different places in the Index.}{} the pages that the characters appear.  The point of this chapter is to highlight some of the primary adventures of the characters according to their own perspectives.  <em></em>None of the links in this chapter will be given a corresponding return link.}  This chapter is for me to track relationships and will likely go away when the book is ready for publication.
%
I can, at the header of the code, define the name, gender, mass, and dimensions of each individual.\dothis[inline]{<url href=""></url>{http://malveyauthor.com/}{Madeline Alvey}, the author of  \protect{<url href=""></url>{http://escapepod.org/2017/03/09/ep566-honey-and-bone-artemis-rising-3/}{<q>Honey and Bone</q> at EscapePod}} is a physics and English undergraduate student at UK in Lexington.  I might consider hiring(?) her to help storyboard the characters.}

</section><section><title></title>\studentA<idx><h></h><h></h></idx>{\studentA!inside}}<idx><h></h><h></h></idx>{\studentA!outside}<todo></todo>The index-call that is <em></em>outside} of the section title registers as <backslash />studentA, which puts the name alphabetically under <backslash />studentA, rather than \studentA.  The index-call that is <em></em>inside} of the section title registers as \studentA, which puts the name alphabetically under \studentA, rather than <backslash />studentA.}
<idx></idx>{\studentA|(} % Begin page-range
<ul>
</p></li><li><p> In <xref ref="" />s-forcewords}, \studentB{} gives \studentA{} a good-natured shove in the arm in order to get the language clarified and begin the conversation about the on-by notation.
</p></li><li><p> In <xref ref="" text="type-global" />se-FBD-AB} \studentA{} helps \studentB{}<ellipsis />
    <ul>
    </p></li><li><p> (in the current version) push an object to make it accelerate and feel a reaction force causing \himA{} to accelerate backwards.
    </p></li><li><p> (in the future version) will help the reader move into or out of their residence hall by pushing on heavier furniture.
    </p></li><li><p>[NOTE:] This is all drawn in <xref ref="" />f-firstFBD}, which is updated in <xref ref="" />f-firstFBDupdate}.
    </ul>
</p></li><li><p> In <xref ref="" text="type-global" />se-weightA}, \studentA{} falls from a small height.  (maybe he is jumping off a short ledge while taking a short-cut to class?)
</p></li><li><p> In <xref ref="" />ex-baking}, \studentA{} decides to bake some bread for a party at \studentB's house, measuring the time it takes to warm his oven.
</ul>

<idx></idx>{\studentA|)} % end page-range
</section><section><title></title>\studentB<idx><h sortby="\studentB">\studentB</h></idx>}
<idx></idx>{\studentB|(} % Begin page-range

<ul>
</p></li><li><p> \studentB{} is a passenger in the reader's car in <xref ref="" />ex-slowcar} when the reader runs out of gas and coasts to a stop.
</p></li><li><p> \studentB{} is a passenger in the reader's car in <xref ref="" />ex-coasting} and speculates about how fast to go before putting the car in neutral to coast to a stop.
</p></li><li><p> \studentB{} joins the reader on a road trip in <xref ref="" />cyoa-NI} and runs out of gas.  This results in multiple possible adventures:
<ul>
    </p></li><li><p> <xref ref="" text="type-global" />a-parkandwalk}, which leads to either an end at <xref ref="" text="type-global" />a-nogas} or an end at <xref ref="" text="type-global" />a-intosunset}.
    </p></li><li><p> <xref ref="" text="type-global" />a-coastindrive}, which leads to either <xref ref="" text="type-global" />a-NIdrive} (choose <xref ref="" text="type-global" />a-coastinneutral} or end with <xref ref="" text="type-global" />a-intosunset2}) or <xref ref="" text="type-global" />a-parkandwalk2} (choose <xref ref="" text="type-global" />a-intosunset} or end at <xref ref="" text="type-global" />a-nogas2})
    </p></li><li><p> <xref ref="" text="type-global" />a-coastinneutral}, which leads to an end at <xref ref="" text="type-global" />a-NIresult}.
</ul>
</p></li><li><p> In <xref ref="" />s-forcewords}, \studentB{} gives \studentA{} a good-natured shove in the arm in order to get the language clarified and begin the conversation about the on-by notation.
</p></li><li><p> In <xref ref="" />se-FBD-AB}, \studentB{} helps \studentA{}<ellipsis />
    <ul>
    </p></li><li><p> (in the current version) pull an object to make it accelerate and feel a reaction force causing \himB{} to accelerate backwards.
    </p></li><li><p> (in the future version) will help the reader move into or out of their residence hall by pushing on heavier furniture.
    </p></li><li><p>[NOTE:] This is all drawn in <xref ref="" />f-firstFBD}, which is updated in <xref ref="" />f-firstFBDupdate}.
    </ul>
</p></li><li><p> In <xref ref="" text="type-global" />se-FNB}, \studentB{} has a normal force supporting \himB.  (This touches <xref ref="" text="type-global" />A-floor}, <xref ref="" text="type-global" />A-second}, and <xref ref="" text="type-global" />A-third}.)
</p></li><li><p> At some point, \studentB{} has a party, because in <xref ref="" />ex-baking}, \studentA{} decides to bake some bread for a party at \studentB's house.
</ul>

<idx></idx>{\studentB|)} % end page-range
</section><section><title></title>\studentC}
<idx></idx>{\studentC|(} % Begin page-range

<idx></idx>{\studentC|)} % end page-range
</section><section><title></title>\studentD}
<idx></idx>{\studentD|(} % Begin page-range

<idx></idx>{\studentD|)} % end page-range
</section><section><title></title>\studentE}
<idx></idx>{\studentE|(} % Begin page-range

<idx></idx>{\studentE|)} % end page-range
</section><section><title></title>\studentF}
<idx></idx>{\studentF|(} % Begin page-range

<idx></idx>{\studentF|)} % end page-range
</section><section><title></title>\studentZ}
<idx></idx>{\studentZ|(} % Begin page-range

<idx></idx>{\studentZ|)} % end page-range



%%%%%%%%%%%%%%%%%%%%%%%%%%%%%%%%%%%%%%%%%%%%%%%%%%%%%%%%%%%%%%%%%%%%%%%%%%%%%%%%%%%%%%%%%%%%%%%%%%%%%%

\addcontentsline{toc}{chapter}{Index}
%\printindex
\documentclass[11pt,letter,openany,makeidx]{book}
\usepackage{amsmath}
\usepackage{macros}
\usepackage{comment}
\usepackage{graphicx}
\usepackage{microtype}
\usepackage{gfsdidot}
\usepackage[T1]{fontenc}
\usepackage{booktabs}
\usepackage{underscore,cancel}
\usepackage{caption}
\usepackage[within=chapter,chapterlistsgap=6pt]{newfloat}
\usepackage{tocloft}
\usepackage{xpicture}
\usepackage{xcolor}
%\usepackage[dvips]{xcolor}
%\GetGinDriver  % for xcolor to work well with hyperref
%\usepackage[\GinDriver]{hyperref}
\usepackage{ulem}%for \sout (done)
\usepackage[colorinlistoftodos]{todonotes}
%\usepackage[disable,colorinlistoftodos]{todonotes}
%\usepackage{layouts}
%\usepackage{showframe}
\usepackage{coordsys}
\usepackage{tikz}
\usepackage[pdftex]{hyperref}

%\usepackage{cellpage}

\hypersetup{colorlinks=true,bookmarks=true,pdftitle=Algebra-Based Introductory Physics,pdfauthor=J.Christensen,pdfdisplaydoctitle}
% If using a bibliography, then include "backref" in list of \hypersetup items
% linkcolor=color of internal links (red); anchorcolor = color of anchor text (black); citecolor = bibliographic citations (green); filecolor = color for local URL files (cyan); menucolor = Acrobat menu (red); urlcolor = external links (magenta); hidelinks = remove all color
% citebordercolor = color of box for citations (0 1 0); fileborder = links to files box (0 .5 .5); linkbordercolor = normal links (1 0 0); menuborder; urlborder; allbordercolors; pdfborder

\includecomment{ForMe}
\includecomment{ForReviewer}
\includecomment{ForPublic}

\makeindex

\newlistof{example}{loe}{List of Examples}
\DeclareFloatingEnvironment[fileext=loe,listname="List of Examples",name=Example]{example}
\setlength{\cftexamplenumwidth}{1cm}
\newcounter{sample}
\newcounter{carrysample}
\renewcommand{\thesample}{Simple Example \arabic{sample}}
\renewcommand{\thecarrysample}{Simple Example \arabic{carrysample}}
\newenvironment{sample}{\color{rgb:red,0;green,2;blue,1}\begin{list}{\textbf{\thesample}:}{\usecounter{sample} \setcounter{sample}{\value{carrysample}} \leftmargin 12pt}}{\end{list}\setcounter{carrysample}{\value{sample}}}
\newcommand{\THREE}[6]{\vspace{-3pt}\begin{flushright} Select one:  \mbox{#1 (\ref{#4})},  \mbox{#2 (\ref{#5})}, or \mbox{#3 (\ref{#6})}.\end{flushright}}
\newcommand{\TWO}[4]{\begin{flushright} Select one:  \mbox{#1 (\ref{#3})} or \mbox{#2 (\ref{#4})}.\end{flushright}}
\newcommand{\YN}[2]{\TWO{Yes}{No}{#1}{#2}}
\newcommand{\TF}[2]{\TWO{True}{False}{#1}{#2}}
\newcommand{\return}[1]{{} \hfill \mbox{Return to \ref{#1}.}}
\newcommand{\autoreturn}[1]{{} \hfill \mbox{Return to \autoref{#1}.}}
\newcommand{\linkreturn}[2][a related idea]{{}\hfill \mbox{Return to the discussion of \protect{\hyperlink{#2}{#1}}.}}
\newcommand{\mmr}[1]{\mbox{[\protect{#1}]}}
\newcommand{\multireturn}[1]{{}\hfill Return to one of the following locations: \newline #1.}
\newcounter{AtIQ}
\renewcommand{\theAtIQ}{Answer \arabic{AtIQ}}
\newenvironment{AIQ}{\begin{list}{\textbf{Interactive \theAtIQ}:}{\usecounter{AtIQ} \leftmargin 12pt}}{\end{list}}

% Related (return), but not part of...
\newcommand{\mreturn}[1]{\note{Return to \protect{\ref{#1}}.}}
\newcommand{\mlinkreturn}[2][a related idea]{\note{Return to the discussion of \protect{\hyperlink{#2}{#1}}.}}
\newcommand{\mautoreturn}[1]{\note{Return to \protect{\autoref{#1}}.}}
\newcommand{\mmultireturn}[1]{\note{Return to one of the following locations: \newline #1.}}


\newlistof{adventure}{loa}{List of Adventures}
\DeclareFloatingEnvironment[fileext=loa,listname="List of Adventures",name=Adventure]{adventure}
\setlength{\cftadventurenumwidth}{1cm}
\newcounter{CYOA}
\renewcommand{\theCYOA}{Plan \Alph{CYOA}}
\newenvironment{CYOA}{\begin{list}{\textbf{\theCYOA}:}{\usecounter{CYOA}}}{\end{list}}
\newcounter{storyline}
\renewcommand{\thestoryline}{Storyline \arabic{storyline}}
\newenvironment{Story}{\begin{list}{\textbf{\thestoryline}:}{\usecounter{storyline} \leftmargin 12pt}}{\end{list}}

\newlistof{reallife}{irl}{List of Real Life Patterns}
%\DeclareFloatingEnvironment[fileext=irl,listname="List of Real Life Patterns",chapterlistsgaps=off,name=Real Life Patterns]{reallife}
\DeclareFloatingEnvironment[fileext=irl,listname="List of Real Life Patterns",name=Real Life Patterns]{reallife}
\setlength{\cftreallifenumwidth}{1cm}
\newcounter{IRL}
%\renewcommand{\theIRL}{\arabic{IRL}}
\newenvironment{realtable}{%\renewcommand{\arraystretch}{2}
                           %\hspace{-.2in}
                            \begin{tabular}{@{}lll@{}} \toprule Do This & Notice This & Ask This  \\ }
                            {\bottomrule \end{tabular} }%\renewcommand{\arraystretch}{.5}}
\newcommand{\dna}[3]{\midrule \begin{minipage}{4cm}\raggedright #1 \end{minipage}
                   & \begin{minipage}{4cm}\raggedright #2 \end{minipage}
                   & \begin{minipage}{4cm}\raggedright #3 \end{minipage} \\ }
\newcommand{\multidna}[1]{\multicolumn{3}{|c|}{\begin{minipage}{13cm}\center #1 \end{minipage}} \\ \midrule }


\newlistof{story}{los}{The Stories of the Equations}
\DeclareFloatingEnvironment[fileext=los,listname="The Stories of the Equations",name=This Equation's Story]{story}
\setlength{\cftstorynumwidth}{1cm}
\newcommand{\thestoryof}[1]{\marginpar{\raggedright \footnotesize The story of \\ \fcolorbox{black}{yellow}{\begin{minipage}[c]{1.5in} \center $\deq #1$ \end{minipage}}}}
\newcommand{\EqStory}[2]{\left[ {\color{rgb:red,1;green,1;blue,4} \begin{minipage}{#1}\raggedright\begin{center} #2 \end{center}\end{minipage}} \right]}
\newcommand{\EqStoryOver}[3]{\overbrace{\EqStory{#1}{#2}}^{\displaystyle #3}}
\newcommand{\EqStoryUnder}[3]{\underbrace{\EqStory{#1}{#2}}_{\displaystyle #3}}
\newcommand{\EqStoryFrac}[5]{\frac{\overbrace{\EqStory{#1}{#2}}^{\displaystyle #3}}
                                 {\underbrace{\EqStory{#1}{#4}}_{\displaystyle #5}}}


%%%%%%%%%%%%%%%%%%%%%%%%%%%%%%%%%%%%%%%%%%%%%%%%%%%%%%%%%%%%
%
%\presetkeys{todonotes}{fancyline,color=blue!15}{}
\presetkeys{todonotes}{color=blue!15,linecolor=blue!75,size=\footnotesize}{}
%
\newcounter{todocounter}
\newcommand{\dothis}[2][]
{\stepcounter{todocounter}\todo[color=green!30, #1]{\thetodocounter: #2}}
\newcommand{\docaption}[3][]
{\stepcounter{todocounter}\todo[color=green!30, prepend, caption={\thetodocounter: \underline{#2}}, #1]{#3}}
\newcommand{\addlink}[2][]
{\stepcounter{todocounter}\todo[prepend, caption={\thetodocounter: \underline{Add Link}}, #1]{#2}}
\newcounter{todourgentcounter}
\newcommand{\urgent}[2][]
{\stepcounter{todourgentcounter}\todo[color=orange!50, #1]{\thetodourgentcounter: #2}}
\newcommand{\urgcap}[3][]
{\stepcounter{todourgentcounter}\todo[color=orange!50, prepend, caption={\thetodourgentcounter: \underline{#2}}, #1]{#3}}
\newcommand{\done}[2][]
{\todo[color=yellow!10, #1]{\sout{#2}}}
%
%\newcommand{\new}[2]{}%
\newcommand{\new}[2]{\marginpar{\raggedright \footnotesize New to #1 \\ \fcolorbox{blue}{yellow!10}{\begin{minipage}[c]{1.5in} \center {\color{blue} #2 } \end{minipage}}}}%
%%%%%%%%%%%%%%%%%%%%%%%%%%%%%%%%%%%%%%%%%%%%%%%%%%%%%%%%%%%%


%%%%%%%%%%%%%%%%%%%%%%%%%%%%%%%%%%%%%%%%%%%%%%%%%%%%%%%%%%%%
%
%\newcommand{\deq}{\displaystyle}
%\newcommand{\txtfrac}[2]{{}^{#1}\!/_{\!#2}}
%
%%%%%%%%%%%%%%%%%%%%%%%%%%%%%%%%%%%%%%%%%%%%%%%%%%%%%%%%%%%%



%%%%%%%%%%%%%%%%%%%%%%%%%%%%%%%%%%%%%%%%%%%%%%%%%%%%%%%%%%%%
%
% PEOPLE AND PRONOUNS
%
% According to https://www.cdc.gov/nchs/fastats/body-measurements.htm
% Measured average height, weight, and waist circumference for adults ages 20 years and over
% Men:
% Height (inches): 69.3                 = 1.760 m
% Weight (pounds): 195.5                = 88.86 kg
% Waist circumference (inches): 39.7    = 1.01 m
% Women:
% Height (inches): 63.8                 = 1.621 m
% Weight (pounds): 166.2                = 75.55 kg
% Waist circumference (inches): 37.5    = 0.9525 m
% Source: Anthropometric Reference Data for Children and Adults: United States, 2007-2010, tables 4, 6, 10, 12, 19, 20[PDF - 1.7 MB]
%  https://www.cdc.gov/nchs/data/series/sr_11/sr11_252.pdf
%
\newcommand{\studentA}{Abdul}       \newcommand{\massA}{\mbox{$85.0\unit{kg}$}}
\newcommand{\studentB}{Beth}        \newcommand{\massB}{\mbox{$75.0\unit{kg}$}}
\newcommand{\studentC}{Carl}        \newcommand{\massC}{\mbox{$90.0\unit{kg}$}}
\newcommand{\studentD}{Diane}       \newcommand{\massD}{\mbox{$80.0\unit{kg}$}}
\newcommand{\studentE}{Erik}        \newcommand{\massE}{\mbox{$95.0\unit{kg}$}}
\newcommand{\studentF}{Frances}       \newcommand{\massF}{\mbox{$85.0\unit{kg}$}}
\newcommand{\studentX}{Xerxes}       \newcommand{\massX}{\mbox{$62.5\unit{kg}$}}
\newcommand{\studentZ}{Zambert}     \newcommand{\massZ}{\mbox{$95.0\unit{kg}$}}
% Male
\newcommand{\heA}{he}\newcommand{\himA}{him}\newcommand{\hisA}{his}\newcommand{\himselfA}{himself}
\newcommand{\HeA}{He}\newcommand{\HimA}{Him}\newcommand{\HisA}{His}
\newcommand{\heC}{he}\newcommand{\himC}{him}\newcommand{\hisC}{his}\newcommand{\himselfC}{himself}
\newcommand{\HeC}{He}\newcommand{\HimC}{Him}\newcommand{\HisC}{His}
\newcommand{\heE}{he}\newcommand{\himE}{him}\newcommand{\hisE}{his}\newcommand{\himselfE}{himself}
\newcommand{\HeE}{He}\newcommand{\HimE}{Him}\newcommand{\HisE}{His}
\newcommand{\heZ}{he}\newcommand{\himZ}{him}\newcommand{\hisZ}{his}\newcommand{\himselfZ}{himself}
\newcommand{\HeZ}{He}\newcommand{\HimZ}{Him}\newcommand{\HisZ}{His}
% Female
\newcommand{\heB}{she}\newcommand{\himB}{her}\newcommand{\hisB}{her}\newcommand{\himselfB}{herself}
\newcommand{\HeB}{She}\newcommand{\HimB}{Her}\newcommand{\HisB}{Her}
\newcommand{\heD}{she}\newcommand{\himD}{her}\newcommand{\hisD}{her}\newcommand{\himselfD}{herself}
\newcommand{\HeD}{She}\newcommand{\HimD}{Her}\newcommand{\HisD}{Her}
\newcommand{\heF}{she}\newcommand{\himF}{her}\newcommand{\hisF}{her}\newcommand{\himselfF}{herself}
\newcommand{\HeF}{She}\newcommand{\HimF}{Her}\newcommand{\HisF}{Her}
%
\newcommand{\heX}{\studentX}\newcommand{\himX}{\studentX}\newcommand{\hisX}{\studentX's}\newcommand{\himselfX}{the person of \studentX}
\newcommand{\HeX}{\studentX}\newcommand{\HimX}{\studentX}\newcommand{\HisX}{\studentX's}
%%%%%%%%%%%%%%%%%%%%%%%%%%%%%%%%%%%%%%%%%%%%%%%%%%%%%%%%%%%%%


%%%%%%%%%%%%%%%%%%%%%%%%%%%%%%%%%%%%%%%%%%%%%%%%%%%%%%%%%%%%
%
% Book macros
%
\newcommand{\aside}[2]{\marginpar{\raggedright \footnotesize\textbf{#1}: #2}}
\newcommand{\important}[1]{\\ \fcolorbox{black}{yellow}{\begin{minipage}[c]{4.925in} \center #1 \end{minipage}}\\}
\newcommand{\inlife}{\marginpar[\scriptsize \raggedright How you might observe $\Rightarrow$ this in your life.]
                               {\scriptsize \raggedleft $\Leftarrow$ How you might observe this in your life.}}
\newcommand{\touchstone}{\marginpar[\scriptsize \raggedright Where have I seen this $\Rightarrow$ before?]
                                   {\scriptsize \raggedleft $\Leftarrow$ Where have I seen this before?}}
\newcommand{\foreshadow}{\marginpar[\scriptsize \raggedright When will I ever use this? $\Rightarrow$]
                                   {\scriptsize \raggedleft $\Leftarrow$ When will I ever use this?}}
\newcommand{\foreshadowR}{\reversemarginpar
                          \marginpar[\scriptsize \raggedright When will I ever use this? $\Rightarrow$]
                                    {\scriptsize \raggedleft $\Leftarrow$ When will I ever use this?}}
\newcommand{\Touchstone}[1]{\marginpar[\scriptsize \raggedright Where have I seen this $\Rightarrow$ \\ before? #1]
                                      {\scriptsize \raggedleft $\Leftarrow$ Where have I seen this before? #1}}
\newcommand{\Foreshadow}[1]{\marginpar[\scriptsize \raggedright When will I ever use this? $\Rightarrow$ \\ #1]
                                      {\scriptsize \raggedleft $\Leftarrow$ When will I ever use this? #1}}
%
%%%%%%%%%%%%%%%%%%%%%%%%%%%%%%%%%%%%%%%%%%%%%%%%%%%%%%%%%%%%


\begin{document}

%\title{Algebra-Based Introductory Physics}
%\author{J Christensen}
%\date{Jan 2017}
%\maketitle
%\pagestyle{cellpage}

\begin{titlepage}
	\centering
%	\includegraphics[width=0.15\textwidth]{example-image-1x1}\par\vspace{1cm}
	{\Huge\bfseries Physics Connected\par}
	\vspace{1cm}
	{\Large\bfseries An Algebra-Based Introductory Physics Textbook\par}
	\vspace{1cm}
	{\large Learn like you think: an interconnected view of physics\par}
	\vspace{2cm}
	{\Large\itshape by: J Christensen\par}
	\vfill
\begin{ForReviewer}
	Version 2.3\par
	{\footnotesize
    \begin{itemize}
    \item Ideas yet to implement:
        \begin{itemize}
        \item The examples are phrased as descriptions, not examples like the homework problems.  Need to consider rephrasing these, not calling them examples, or adding actual examples that better show how to respond to the way homework problems are written.
        \item Define a different page dimension that fits on a cell phone display.  (Enhance possible cell-phone reading.)
        \end{itemize}
    \item version 2.3: June 16-28, 2017
        \begin{itemize}
        \item Updated Section 81. $F=mg$ and Section 8.2 Normal Force
        \item Added specific list of Flame Challenges
        \item Rearranged some of the subsections in the ``Seeing Physics'', added references
        \item Equations of motion for constant acceleration (Need the Story Of)
        \item Added a section to Chap 5 (1-D motion) that gives examples of solutions that require multiple steps  (one equation is insufficient)
        \item Developed the weight and mass discussion and examples
        \item Ladder leaning example in torque, plus some homework problems
        \item Added some Conceptual Homework to weight/mass
        \item Added placeholders to the Gravity chapter
        \item Removed indicators of v1.7 changes
        \end{itemize}
    \item version 2.2: June 16, 2017
        \begin{itemize}
        \item Created conversation about $F=mg$ for Chapter on types of forces.  Caused modifications in lots of places
            \begin{itemize}
            \item Added freefall to the motion chapter
            \item Created IRL and Example dropping objects to see acceleration in $F=mg$, then moved to freefall section -- new Answers to interactive questions
            \item Commented on air resistance
            \item Comments about precision in language (need to do more with precision in mathematics)
            \item Started a couple of ideas about effective theories.  (need to decide where it goes)
            \item Added detail about SI, and specifically the pound-force, pound-mass, and kilogram. to sections \ref{s:SI-MKS} and \ref{ss:weightmass}
            \item Added NIST and BIPM references (found in Wikipedia and then searched further)
            \item conversation about weight and mass.  (required reference to the chapter on Fluids and density)
            \item Moved Google search about significant figures
            \end{itemize}
        \item Added comments about fundamental forces to the section on types of force
        \item Removed indicators of v1.5 and v1.6 changes
        \end{itemize}
    \item version 2.1: June 10, 2017
        \begin{itemize}
        \item Re-commented the $\backslash$new command
        \item Started the chapters on Seeing Physics [\autoref{c:physics}] and Deeper Dive [\autoref{c:revisted}] (These should be renamed)
        \item Moved some sections on fundamental interactions
        \end{itemize}
    \item version 2.0: April 10, 2017
        \begin{itemize}
        \item Re-enabled v1.8 hides
        \item Added a link to \textit{Spacepod}, \textit{Physics Footnotes}, and \textit{Sixty Symbols}
        \item Fixed a $\backslash$dothis that was inside an $\backslash$important, causing a compile error.
        \item Removed indicators of v1.4 changes
        \end{itemize}
    \item version 1.8: April 1, 2017
        \begin{itemize}
        \item Prepare for "the public": "Disabled" the To-Do items, "Hid" the $\backslash$new revision notes, Hid the List of Tables (have none yet)
        \end{itemize}
    \end{itemize}
    }
\end{ForReviewer}
\begin{ForPublic}
{\flushleft
\textbf{Note to the reviewers:}\new{v1.8}{Added the note}
My goal with this book is to create an electronically viewable book that makes use of the advantages of being electronic.  While current e-books have the advantage of being viewable on various devices with having to carry a physical book around, most e-textbooks do not take advantage of hyperlinked text.  With this book I hope to integrate links both forward and backward.  The forward links will be used to motivate curious students.  The backward links will be used to support students who lose track of previous topics.  The integration of these will also provide a convenient opportunity for students to browse through topics they are interested in.
\newpar

At this time, I am providing a single chapter to gauge the viability.  The chapter I am providing is on Newton's Laws.  However, as you read this document, you will find many, many more partially written chapters.  All of the partial chapters and sections are intended to be place-holders for the forward- and backward-links that \autoref{c:force} depends upon.
\newpar

I created this as a PDF that, I believe, can be easily viewed on a computer or tablet.  Since some of my students also seem to read on their phone, I verified that I am also able to view the text in a reasonable manner on my Samsung phone in landscape mode.  In each case, the links should be active and easily manageable.

}
\end{ForPublic}
	\vfill

% Bottom of the page
	{\large \today\par}
\end{titlepage}

\tableofcontents
\newpage
\begin{ForReviewer}
\listoftables
\vfill
\end{ForReviewer}
%\newpage
\listoffigures
\vfill
%\newpage
%\listofstorys
%\newpage
\listofexamples
\vfill
%\newpage
\listofadventures
\vfill
%\newpage
\listofreallifes
\vfill
\newpage

\listoftodos

\newpage

\chapter*{Preface}\new{v1.8}{Modified this for the public distribution}

The purpose of creating this book is to make better use of the technology that electronic texts allow for without losing the functionality of a print book.  While this text should be comparable to any other print text, when this is provided in the online format it will provide links back and forth between early and later topics.  Linking from later material to earlier material will allow students to refresh their memory of what was previously discussed.  Linking from earlier material to later material will inspire students to look ahead to how that topic will be used in more interesting scenarios.

Having these links will allow for some other interesting features that can be placed in the back of the book and accessed through links.  Examples of this might be:
\begin{enumerate}
\item ``Dig Deeper'' where some of the more tedious and some of the more interesting aspects can be investigated. For example in \autoref{c:motion} on the equations of motion, one might see how these equations are direct applications of calculus for those students who happen to have taken that course (which is common for biology and pre-medical students).
\item ``Every Equation Tells a Story'' which discusses how the description-in-English and the description-with-math interrelate to build intuition in both directions.
\item ``Examples'', with the difference from a traditional textbook being that students can interact with the example as: ``If you have this question, then go here. If you have that question, then go there.''
\item ``In the `Real World''' where students see how the concept lives in the messy real world and why physicists simplify or ignore complicating aspects.
\item ``Connections'', which might take one of three forms:
\begin{enumerate}
\item ``Where have I seen this before?'' (linking back to earlier material)
\item ``When will I ever use this?'' (linking ahead to later material)
\item ``Why is this interesting?'' (linking to popular or complex topics)
\end{enumerate}
\end{enumerate}
The goal of the book is to encourage curiosity in the reader. Since there is an expectation that students will explore the material on their own, advanced topics will explicitly note where the reader can look for supporting material and basic topics will be motivated with links to more advanced topics.  To help maintain the interest of the reader, recurring characters will be featured in the examples.  These characters will live a storyline\dothis{storyboard the characters and how they develop}{} and interact with each other.  It is possible to read the examples as a separate storyline for the N\urgent{Decide how many characters}{} interacting characters.

I am choosing the approach described above based on the assumption that students will prefer to develop their knowledge by building a world-view that connects to their current understanding, their interests, and their world-view. Providing the cross-referencing links without distracting students with all of the information at once will enable them to explore the information. Writing the text in a narrative style that helps students see the explanations for the world they live in will encourage them to explore ``what happens when I do this'' in their real life. Fostering this spirit of exploration will enable the instructors to bring their own active-learning techniques into the classroom.

This textbook is in several Parts\urgcap{book layout}{Here we should add information about Adventures, Examples, Equation-Stories, and IRLs.}:  \textbf{Part I} is for the preliminaries, including descriptions of science in general, physics in particular, and the use of math.  \textbf{Part II} is intended to introduce three fundamental and powerful concepts.  These concepts are motion, force, and energy.  I have found that if a student can understand these ideas sufficiently well, then they can quickly pick up any other idea that we introduce, even if the idea seems initially unfamiliar.  \textbf{Part III} develops the ideas in Part II by introducing momentum, circular motion, rotational motion, torque, and the Newtonian theory of gravitation.  \textbf{Parts IV} and \textbf{V} are oscillations and thermodynamics.  With the traditional organization of the two-semester introductory physics, these parts can be covered in either order and can be chosen to be put in either semester. \textbf{Part VI} covers electricity, magnetism, light, and optics.  This is traditionally the meat of the second semester. \textbf{Part VII} touches on the topics that are usually referred to as ``modern physics''.  The goal with including these chapters is to provide some inspiration for what some students see as the tedium of the standard material.  These chapters will be linked to throughout the book as examples of how the traditional material supports the material that may be in the news and is more talked about in popular science.  The last final part, \textbf{Part VIII}, holds the answers to the interactive examples mentioned above, the bulk of the adventures the reader can investigate in order to test their understanding of the material, and the story lines of each of the characters in the text.

\textbf{A note about viewing the PDF online:}  If you are viewing this as a PDF set to view ``single page,'' then the links will take you to the top of the relevant page, rather than to the specific topic.  If, on the other hand, you are viewing this in ``continuous view'' then you should go directly to the location of interest. If you are viewing this in ``two-page'' mode (whether continuous or not), it might not be immediately obvious to which page (left or right) you have jumped.  Most of the PDF viewers I have encountered allow you to follow links and to return to your previous location.  On most PCs, the way to return to your previous location is by holding the [ctrl] key and pressing the $[\leftarrow]$ button.  There are a few PDF viewers that do not allow you to ``go back'' to the location you linked from.  Whether or not you have that capability, I have placed ``return links'' in the margins so that you can get back to the place from which you linked.


\part{Prerequisites}

\chapter{The Story of Science}

Once upon a time\done{start the book}{} somebody saw the world around them and thought something equivalent to ``well, that's an interesting pattern\ldots'' and predictions were born.  Every human and many animals build their own world of expectations such as: objects will fall down, food will arrive at mealtime, or certain people will smile at me.  Scientists study the patterns in the world around us and do so in a fairly specific way.  Novelists, sociologists, historians, and cartoonists also look at the world around us in a very particular way.  The story of humanity is a story about observing the world around us.

Scientists, in general, observe patterns through careful, detailed measurements \ldots\dothis{Add description of science.}{}

Physicists, in particular, consider the patterns in the physical world around us.\dothis{Add description of physics}{}

\hypertarget{d:physicspatterns}{Some patterns} that you might experience help us take very different experiences and group them together.  For example\inlife, there are ways in which \hyperlink{d:freefall}{dropping your keys} and \hyperlink{d:ballistic}{throwing a dog toy} are very similar.  They both fall, even thought the fall along rather different paths.  There are also patterns that you experience that might look very similar but can be treated very differently.  For example\inlife, \hyperref[irl:nonparabolic]{the path of baseball pitch} is very different for a fast ball compared to a slider, a curve ball, or a knuckleball.

\begin{figure}[h]
\hrule\hrule
  \missingfigure{Photograph a park with tennis courts and basketball hoops in the background and falling car keys and a dog in the foreground. I think we could do that at Boone County park.}
\begin{ForPublic}
\centering
\fbox{\begin{minipage}{4in}
\vspace{1in}
This will be a photograph of a park with tennis courts and basketball hoops in the background and falling car keys and a dog in the foreground.
\vspace{1in}
\end{minipage}}
\end{ForPublic}
  \caption{\label{Fig:BoonePark} Life is full of examples of physics all around us. }
\hrule\hrule
\end{figure}

%\todo[due=2017-4-1]{this one has a due date}

\section{Careful, Detailed Observation}

[Discussion of ``\hypertarget{d:casual}{casual observer}\mautoreturn{ss:NI}'' as intuition versus ``scientific observing'' and mathematical modelling]\dothis{Consider the ``casual to the obvious observer'' joke}{}

\noindent
[Discussion of common student comment: ``in physics class it is this way, but in \textit{real life} it is that way.'']

\section{Theory versus Law}\label{s:law}\mautoreturn{s:Newton}


\chapter{Seeing Physics}\label{c:physics}\new{v2.1}{Filled in the details a little. This chapter should mirror \protect{\autoref{c:revisted}}.}

\section{The Flame Challenge and Other Brief Descriptions}\label{s:flame}

What you will find in this book is a series of chapters that, on the surface, feel like a list of isolated topics.  Each chapter will have examples that focus your attention on examples of that specific concept.  However, the really interesting aspect of physics is that these descriptions of the world around us come together in different ways to explain complex systems that might feel unrelated.  For example, the thermodynamics of making your refrigerator work on Earth comes from the same theories of thermodynamics that help us understand the heat flow of the sun.  Furthermore, in order to understand the sun, we also need to understand the gravitational interaction, which also describes how baseballs fly through the air.

This chapter will introduce a set of quick-overview explanations of phenomena to indicate how different ideas tie together in some complex systems.  The point  is specifically to over-simplify complex ideas in order to ``get the idea''.  You will also be pointed to the various chapters that go into the details of the relevant physics where you can learn more.  Then, at the end of the book in~\autoref{c:revisted}, we will revisit each of these ideas and go into the description in more depth assuming you have understood each of the relevant chapters, with reference back to the sections that provide the basis of our understanding.

\textbf{Caution}: Since this particular chapter is intended to be background introduction, rather than a place to study details, none of the links to other sections here will have return links in the rest of the text.  So, if you intend to use this as a jumping off point, you might want to create a bookmark here so that you can return after you read the details in other sections.

\subsection{The Flame Challenge}\label{ss:flame}\new{v2.3}{Added the questions.  These might be better in their respective sections.}
\href{http://www.aldakavlilearningcenter.org/practice/flame-challenge}{The Flame Challenge}

Useful?  \href{https://newsstand.google.com/articles/CAIiEBF_HbPTdq-9q-hjA0W51WYqFggEKg4IACoGCAow9vBNMK3UCDDq0Rc}{How Alan Alda Makes Science Understandable}

\href{http://www.aldakavlilearningcenter.org/practice/flame-challenge/what-is-a-flame}{2012: What is a flame?} \\
\href{http://www.aldakavlilearningcenter.org/flame-challenge/past-challenges/what-time}{2013: What is time?} \\
\href{http://www.aldakavlilearningcenter.org/practice/flame-challenge/past-challenges/what-is-color}{2014: What is color?} \\
\href{http://www.aldakavlilearningcenter.org/practice/flame-challenge/past-challenges/what-is-sleep}{2015: What is sleep?} \\
\href{http://www.aldakavlilearningcenter.org/practice/flame-challenge/past-challenges/what-is-sound}{2016: What is sound?} \\
\href{http://www.aldakavlilearningcenter.org/practice/flame-challenge/past-challenges/energy}{2017: What is energy?}


\subsection{The Forming of Matter in the Universe}\label{ss:matter}\new{v2.1}{Started this section to give a sense\ldots}

In the early ages of the universe, which is an entirely different story that could be told, there were a ridiculously large number of particles created and drifting around.  There were a variety of types (\autoref{ss:StandardModel}), some being positively charged (\autoref{s:Echarge}), some negatively charged, and some were neutral; but the larger ones tended to gradually decay (\autoref{ss:particledecay}) into smaller ones.  The smaller of the positively-charged baryons (\autoref{s:particle}), which we call protons, and the smallest of the negatively-charged leptons (\autoref{s:particle}), which we call electrons, also tended to stick together because of their electrical charges (\autoref{s:Echarge}), forming hydrogen atoms.  You may note that as this happens, sometimes the more ambitious of the particles form larger clumps of two protons and two neutrons, making helium atoms that are held together by the strong nuclear force ([need ref])\dothis{Stopped mid-stream.  This is a good place to jump back in when I am stuck someplace else.}{}

\subsection{Things in the Sky}\new{v2.3}{Rearranged sections}

\subsubsection{The Sun}\label{sss:sun}\new{v2.1}{This point of this will be to connect gravity-thermo-nuclear and to do it in 1-2 paragraphs (a la the flame challenge).}
The bright, shiny sun, which keeps us all alive, is a nice example of a rather complex system that allows us to verify our various theories of the world around us.  As an over-simplification of the process, we can consider the existence of a star in three phases: the ignition (some have said ``birth'') of a star, the shining (some would say ``life'') of the star, and the snuffing (``death''?) of the star.

\subsection{Things on the ground}

\subsubsection{Hot Tea and Iced Tea}\label{sss:tea}\mautoreturn{s:surface.tension}

On 28 April, 2017,\new{v2.3}{New source of info}
\href{http://www.cbc.ca/podcasting}{CBC Broadcasting} published a
\href{http://www.cbc.ca/podcasting/includes/quirks.xml}{\textit{Quirks and Quarks}} episode discussing why
\href{https://podcast-a.akamaihd.net/mp3/podcasts/quirks_20170429_19254.mp3}{hot water sounds different from cold water when they are poured}.
Spoiler Alert: It is due to surface tension, size of droplets when heated, and auditory perception.


%\subsection{Kitchen Appliances}
\subsubsection{Oven}
\subsubsection{Refrigerator}
\subsubsection{Microwave}
\subsubsection{Television}

\subsection{Automobile}
\subsubsection{Coolant and Antifreeze}
\subsubsection{Tires}
\subsubsection{Torque}

\subsection{Cool Ideas}
\subsubsection{Black Holes}\label{sss:blackhole1}
\subsubsection{Quantum Mechanics}
\subsubsection{Relativity}
\subsubsection{String Theory}
\subsubsection{Fusion}

On 28 April, 2017,\new{v2.3}{New source of info}
\href{http://www.cbc.ca/podcasting}{CBC Broadcasting} published a
\href{http://www.cbc.ca/podcasting/includes/quirks.xml}{\textit{Quirks and Quarks}} episode discussing a
\href{https://podcast-a.akamaihd.net/mp3/podcasts/quirks_20170429_51936.mp3}{documentary compares the massive scale ITER approach to fusion with the much smaller approach by a Canadian company}.
\textbf{I don't think I want to use this, but it might be helpful to listen again to the nice summary of fusion.}  Maybe get some resources on ``state of the art''.


\section{Effective Theory}\label{s:effective1}\dothis{Should this be here or in \protect{\autoref{s:effective2}}?}{}

All of our explanations are approximations.  This section will describe some physics in the world around us in one or two paragraphs with links to the sections in the book that provide the detailed understanding of that piece which connects to the mathematics and the underlying foundation.  Each topic will also link to a more detailed discussion at the end of the book with a longer conversation that gets into more nitty-gritty details which assume you have learned the details from the book.  In short, this section looks forward to what is possible to understand and that chapter looks back at how you do understand.  Each of these topics will also be accompanied by a five-minute podcast describing the topic.

The term ``effective theory'' is used in physics to describe a wide-reaching phenomenon which can be approximated by a simpler theory in a smaller circumstance.  So, for example, Einstein's theory of general relativity as a complex description of the gravitational interaction.  It would be unwieldy and impractical to use that to describe our day-to-day interactions with the gravitational interaction.  On the other hand, Newton's theory of the gravitational interaction is a special case of Einstein's general theory of relativity that works perfectly well so long as you behave yourself and do not try to travel at a significant fraction of the speed of light.  We can say that Newton's theory of gravity is an effective theory for Einstein's theory of gravity that accounts for acceleration at low speeds.  Likewise, Einstein's special theory of relativity is an effective theory of the general theory of relativity.  The special theory is relevant when you do not allow for acceleration, but do allow for faster speeds.  Once you reach beyond the limitations of the effective theory, the description ``breaks down''.


\chapter{Why so much math?}

\section{Every equation tells a story}\label{s:story}

Mathematics is its own language.  It is the language of patterns.  Humans are very adept at tracking patterns.  Physics is the study of patterns in the physical world.  It turns out that the language of physics provides a natural and concise mechanism for expressing patterns in a uniquely precise manner.  Equations allow us to connect physical reality to very specific predictions.  For example, the equation for thermal conductivity, \autoref{eq:thermalconductivity} in \autoref{ss:thermalconductivity}, allows \studentA\index{\studentA} to predict the time it takes for \hisA\ oven to warm up to a specific temperature because
$\displaystyle \frac{Q}{\Delta t} = \kappa A \, \frac{\Delta T}{\Delta x}$\todo{I would love for this to be a mouse-over in the equation}{} says that {the rate at which energy flows} {depends on} {how well air allows energy to flow,} {the size of the oven,} and {the amount the temperature needs to change} {across the height of the oven} as follows:\dothis{Consider ``chunking'' the ``story'' with colors to indicate the pieces.}{}
\[\begin{array}{ccccc}
\deq \frac{Q}{\Delta t} & = & \deq \kappa & \deq A & \deq \frac{\Delta T}{\Delta x} \\
\EqStoryOver{45pt}{the rate at which energy flows}{}
& \EqStoryOver{40pt}{depends on}{}
& \EqStoryOver{50pt}{how well air allows energy to flow,}{}
& \EqStoryOver{50pt}{the size of the oven,}{}
& \EqStoryFrac{75pt}{and the amount the temperature needs to change}{}
                    {across the height of the oven}{}
\end{array}\]
We will see this particular story in more detail with \autoref{ex:baking} (pg.~\pageref{ex:baking}) when \studentA\index{\studentA} prepares to bake some bread for \hisA\ friends.  Some of the more important equations are listed below.  By jumping between these narratives, you can get a better sense of how to think about physics in general.

\begin{ForPublic}
\begin{table}[h]
\centering
\begin{tabular}{ccc}
\hyperref[st:F=ma]{$\deq \vec F_\mathrm{net} = m \vec a$} & .......... & \pageref{st:F=ma}
\end{tabular}
\end{table}
\end{ForPublic}
\begin{ForMe}
\dothis{Decide if should use the ``public version'' or the ``me version'' (which uses $\backslash$listofstorys).}{}
\listofstorys
\vfill
\end{ForMe}


\section{The Metric System}\label{s:SI-MKS}\mautoreturn{ss:weightmass}\new{v2.2}{Added detail}

The International System of Units (SI)
% https://en.wikipedia.org/wiki/International_System_of_Units
was adopted in 1960 at the
\href{http://www.bipm.org/jsp/en/ListCGPMResolution.jsp?CGPM=11}{eleventh meeting}
of the
\href{http://www.bipm.org/en/about-us/}{International Bureau of Weights and Measures (BIPM)}.\footnote{In French this organization is the Bureau International des poids et mesures, so the acronym is BIPM.}
%  https://en.wikipedia.org/wiki/General_Conference_on_Weights_and_Measures

In 1901 at the
\href{http://www.bipm.org/jsp/en/ListCGPMResolution.jsp?CGPM=3}{third meeting}
of the BIPM, it
\href{http://www.bipm.org/en/CGPM/db/3/2/}{was declared} that
\begin{enumerate}
\item The kilogram is the unit of mass; it is equal to the mass of the international prototype of the kilogram;
\item The word ``weight'' denotes a quantity of the same nature as a ``force'': the weight of a body is the product of its mass and the acceleration due to gravity; in particular, the standard weight of a body is the product of its mass and the standard acceleration due to gravity;
\item The value adopted in the International Service of Weights and Measures for the standard acceleration due to gravity is $980.665 \unitfrac{cm}{s^2}$, value already stated in the laws of some countries.
\end{enumerate}
The 11th meeting (1960) redefined the meter in terms of wavelengths of light.
The 13th meeting (1967) redefined the second in terms of the frequency of radiation from $^{133}$Cs.
The 17th meeting (1983) redefined the meter in terms of the speed of light and seconds.
The 24th (2011) and 25th (2014) meeting discussed redefining the kilogram in terms of the Planck constant, with an expectation that it will be redefined at the 26th meeting (Nov, 2018).  See note in \autoref{ss:units}.

Note \href{https://www.nist.gov/sites/default/files/documents/2016/11/10/appb-17-hb44-final.pdf}{Handbook 44, page B-6} talks about SI.\new{v2.2}{References to NIST}

Note
\href{https://www.nist.gov/pml/weights-and-measures/publications/nist-handbooks/handbook-44}{Handbook 44 webpage}
still links to
\href{https://www.nist.gov/sites/default/files/documents/2016/11/10/appc-17-hb44-final.pdf}{the 2016 pdf}
instead of the
\href{https://www.nist.gov/sites/default/files/documents/2017/04/28/AppC-12-hb44-final.pdf}{the 2017 pdf}
even though it says it was updated in 2017.

There is also
\href{https://www.nist.gov/pml/special-publication-811-extended-contents}{a special publication} from NIST that summarizes the use and conversation between units in the SI.

\subsection{Units Quantify Dimensions}

\subsection{Conversion from English Units}\label{ss:convertunits}\mmultireturn{\mmr{\autoref{s:sigfig}}, \mmr{\autoref{ex:slowcar}}}

Note internet search comments in \autoref{ss:weightmass} regarding the ``conversion'' of kilograms-to-pounds, with special attention to \hyperref[s:sigfig]{significant digits}.\index{Significant Digits}\dothis{rephrase this.  I moved that discussion to \protect{\autoref{s:sigfig}}.}


\subsection{Fundamental Units versus Derived Units}\label{ss:units}\mmultireturn{\mmr{\autoref{sss:unit-N}}, \mmr{\autoref{ss:weightmass}}}

Note conversation in \autoref{sss:unit-N} about the Newton.

See\new{v2.2}{Possible redefinition of the kilogram.}\mautoreturn{s:SI-MKS}
\href{https://scitechdaily.com/researchers-to-redefine-the-kilogram-in-terms-of-plancks-constant/}{the 2012 article from SciTechDaily.com}
and
\href{https://www.nist.gov/physical-measurement-laboratory/plancks-constant}{the NIST explanation} about redefining the kilogram in terms of the Planck constant at the 26th meeting (Nov, 2018) of BIPM.


\section{A graph is worth a thousand pictures}

\subsection{Coordinate Systems}

\noindent
\begin{itemize}
\item Discussion of the choice of origin (possible reference to zero-value of the potential energy)
\item Discussion of the choice of the positive-direction (possible reference to falling objects and using positive-up versus positive-down)
\item \hypertarget{d:referenceframe}{Definition of a reference frame}\mmultireturn{\mmr{\autoref{ss:addvel}}, \mmr{\autoref{ss:noninertial}}, \mmr{\hyperlink{d:NewtonInertial}{Newton's Laws}}}
\begin{itemize}
    \item (different locations) The view from the roof versus from the ground
    \item (different speeds) The view from the sidewalk versus from a moving car  (See also \autoref{ss:noninertial}.)
    \item (different types of motion) The view from a park bench versus from a merry-go-round.  (See also \autoref{s:noninertial}.)
\end{itemize}
\end{itemize}

\subsection{The Vocabulary of Graphs}

[Quick review of parameters and variables of $y=mx+b$ and $y=ax^2+bx+c$.]

\begin{center}
\setlength{\unitlength}{1cm}
\begin{Picture}(-2.5,-5.5)(3.5,3.5)
\cartesiangrid(-2,-5)(3,3)
\pictcolor{blue}
%\qbezier(-1,-7.405)(0.306,9.322)(1.612,-7.405)
%\qbezier(-1,-10.405)(0.612,15.075)(2.223,-10.405)
\qbezier(-0.5,-3.726)(0.612,8.396)(1.723,-3.726)
\end{Picture}
\end{center}

\section{Trigonometry and Vectors}

\subsection{Trigonometry}
\subsection{Vectors}\label{ss:vectors}\mmultireturn{\mmr{\hyperlink{d:pushvector}{the direction of force}}, \mmr{\autoref{sss:netforce}}}
\begin{ForMe}
\begin{figure}
\hrule\hrule
  \centering
  \caption{\LaTeX\ lines and vectors.  This will be deleted, but is here for reference.}\label{f:lines}
\begin{picture}(300,500)(0,0)
\put(0,0){\line(1,0){300}} \put(301,-2){(1,0) $0^\circ$}
\put(0,0){\line(6,1){300}} \put(301,48){(6,1) $9.46^\circ$}
\put(0,0){\line(5,1){300}} \put(301,58){(5,1) $11.31^\circ$}
\put(0,0){\line(4,1){300}} \put(301,73){(4,1) $14.04^\circ$}
\put(0,0){\line(3,1){300}} \put(301,98){(3,1) $18.43^\circ$}
\put(0,0){\line(2,1){300}} \put(301,148){(2,1) $26.57^\circ$}
\put(0,0){\line(1,1){250}} \put(251,248){(1,1) $45^\circ$}
%\put(0,0){\line(6,2){300}} \put(301,98){(6,2) $18.43^\circ$}
\put(0,0){\line(5,2){300}} \put(301,118){(5,2) $21.8^\circ$}
%\put(0,0){\line(4,2){300}} \put(301,148){(4,2) $26.57^\circ$}
\put(0,0){\line(3,2){300}} \put(301,198){(3,2) $33.69^\circ$}
%\put(0,0){\line(2,2){250}} \put(251,248){(2,2) $45^\circ$}
\put(0,0){\line(1,2){171}} \put(172,340){(1,2) $63.43^\circ$}
%\put(0,0){\line(6,3){300}} \put(301,148){(6,3) $26.57^\circ$}
\put(0,0){\line(5,3){300}} \put(301,178){(5,3) $30.96^\circ$}
\put(0,0){\line(4,3){300}} \put(301,223){(4,3) $36.87^\circ$}
%\put(0,0){\line(3,3){250}} \put(251,248){(3,3) $45^\circ$}
\put(0,0){\line(2,3){204}} \put(205,304){(2,3) $56.31^\circ$}
\put(0,0){\line(1,3){128}} \put(129,382){(1,3) $71.57^\circ$}
%\put(0,0){\line(6,4){300}} \put(301,198){(6,4) $33.69^\circ$}
\put(0,0){\line(5,4){300}} \put(301,238){(5,4) $38.66^\circ$}
%\put(0,0){\line(4,4){250}} \put(251,248){(4,4) $45^\circ$}
\put(0,0){\line(3,4){218}} \put(219,288.666666666667){(3,4) $53.13^\circ$}
%\put(0,0){\line(2,4){171}} \put(172,340){(2,4) $63.43^\circ$}
\put(0,0){\line(1,4){101}} \put(102,402){(1,4) $75.96^\circ$}
\put(0,0){\line(6,5){300}} \put(301,248){(6,5) $39.81^\circ$}
%\put(0,0){\line(5,5){250}} \put(251,248){(5,5) $45^\circ$}
\put(0,0){\line(4,5){225}} \put(226,279.25){(4,5) $51.34^\circ$}
\put(0,0){\line(3,5){192}} \put(193,318){(3,5) $59.04^\circ$}
\put(0,0){\line(2,5){146}} \put(147,363){(2,5) $68.2^\circ$}
\put(0,0){\line(1,5){84}} \put(85,418){(1,5) $78.69^\circ$}
%\put(0,0){\line(6,6){250}} \put(251,248){(6,6) $45^\circ$}
\put(0,0){\line(5,6){230}} \put(231,264){(5,6) $50.19^\circ$}
%\put(0,0){\line(4,6){204}} \put(205,304){(4,6) $56.31^\circ$}
%\put(0,0){\line(3,6){171}} \put(172,340){(3,6) $63.43^\circ$}
%\put(0,0){\line(2,6){128}} \put(129,382){(2,6) $71.57^\circ$}
\put(0,0){\line(1,6){71}} \put(72,434){(1,6) $80.54^\circ$}
%
\put(0,0){\line(0,1){450}} \put(-5,451){(0,1) $90^\circ$}
%
%
%
\put(0,0){\vector(1,0){225}}
\put(0,0){\vector(6,1){221.9}}
\put(0,0){\vector(5,1){220.6}}
\put(0,0){\vector(4,1){218.3}}
\put(0,0){\vector(3,1){213.5}}
\put(0,0){\vector(5,2){208.9}}
\put(0,0){\vector(2,1){201.2}}
\put(0,0){\vector(5,3){192.9}}
\put(0,0){\vector(3,2){187.2}}
\put(0,0){\vector(4,3){180}}
\put(0,0){\vector(5,4){175.7}}
\put(0,0){\vector(6,5){172.8}}
\put(0,0){\vector(1,1){159.1}}
\put(0,0){\vector(5,6){144}}
\put(0,0){\vector(4,5){140.6}}
\put(0,0){\vector(3,4){135}}
\put(0,0){\vector(2,3){124.8}}
\put(0,0){\vector(3,5){115.8}}
\put(0,0){\vector(1,2){100.6}}
\put(0,0){\vector(2,5){83.6}}
\put(0,0){\vector(1,3){71.2}}
\put(0,0){\vector(1,4){54.6}}
\put(0,0){\vector(1,5){44.1}}
\put(0,0){\vector(1,6){37}}
\put(0,0){\vector(0,1){225}}
%
\end{picture}
%\hrule\hrule
\end{figure}
\begin{figure}
%\hrule\hrule
  \centering
  \caption{\LaTeX\ lines and vectors.  This will be deleted, but is here for reference.}\label{f:lines2}
\begin{tikzpicture}
\draw [<->, rounded corners, thick, gray] (10,0) -- (0,0) --(0,10);
\draw [lightgray] (0,6) arc [radius=6, start angle=90, end angle=0];  % start at the (+y) of the circle, end at the (+x) of the circle
\draw [lightgray] (9,1) arc [radius=17, start angle=-10, end angle=52];
\draw [->] (0,0) -- (5.92,0.99); \draw (0,0) -- (9.08,1.51);  \node [right] at (9.08,1.51) {(6,1) $9.46^\circ$};
\draw [->] (0,0) -- (5.88,1.18); \draw (0,0) -- (9.12,1.82);  \node [right] at (9.12,1.82) {(5,1) $11.31^\circ$};
\draw [->] (0,0) -- (5.82,1.46); \draw (0,0) -- (9.18,2.29);  \node [right] at (9.18,2.29) {(4,1) $14.04^\circ$};
\draw [->] (0,0) -- (5.69,1.9); \draw (0,0) -- (9.24,3.08);  \node [right] at (9.24,3.08) {(3,1) $18.43^\circ$};
% \draw [->] (0,0) -- (5.69,1.9); \draw (0,0) -- (9.24,3.08);  \node [right] at (9.24,3.08) {(6,2) $18.43^\circ$};
\draw [->] (0,0) -- (5.57,2.23); \draw (0,0) -- (9.26,3.7);  \node [right] at (9.26,3.7) {(5,2) $21.8^\circ$};
\draw [->] (0,0) -- (5.37,2.68); \draw (0,0) -- (9.25,4.62);  \node [right] at (9.25,4.62) {(2,1) $26.57^\circ$};
% \draw [->] (0,0) -- (5.37,2.68); \draw (0,0) -- (9.25,4.62);  \node [right] at (9.25,4.62) {(4,2) $26.57^\circ$};
% \draw [->] (0,0) -- (5.37,2.68); \draw (0,0) -- (9.25,4.62);  \node [right] at (9.25,4.62) {(6,3) $26.57^\circ$};
\draw [->] (0,0) -- (5.14,3.09); \draw (0,0) -- (9.19,5.51);  \node [right] at (9.19,5.51) {(5,3) $30.96^\circ$};
\draw [->] (0,0) -- (4.99,3.33); \draw (0,0) -- (9.12,6.08);  \node [right] at (9.12,6.08) {(3,2) $33.69^\circ$};
% \draw [->] (0,0) -- (4.99,3.33); \draw (0,0) -- (9.12,6.08);  \node [right] at (9.12,6.08) {(6,4) $33.69^\circ$};
\draw [->] (0,0) -- (4.8,3.6); \draw (0,0) -- (9.02,6.77);  \node [right] at (9.02,6.77) {(4,3) $36.87^\circ$};
\draw [->] (0,0) -- (4.69,3.75); \draw (0,0) -- (8.95,7.16);  \node [right] at (8.95,7.16) {(5,4) $38.66^\circ$};
\draw [->] (0,0) -- (4.61,3.84); \draw (0,0) -- (8.9,7.42);  \node [right] at (8.9,7.42) {(6,5) $39.81^\circ$};
\draw [->] (0,0) -- (4.24,4.24); \draw (0,0) -- (8.61,8.61);  \node [right] at (8.61,8.61) {(1,1) $45^\circ$};
% \draw [->] (0,0) -- (4.24,4.24); \draw (0,0) -- (8.61,8.61);  \node [right] at (8.61,8.61) {(2,2) $45^\circ$};
% \draw [->] (0,0) -- (4.24,4.24); \draw (0,0) -- (8.61,8.61);  \node [right] at (8.61,8.61) {(3,3) $45^\circ$};
% \draw [->] (0,0) -- (4.24,4.24); \draw (0,0) -- (8.61,8.61);  \node [right] at (8.61,8.61) {(4,4) $45^\circ$};
% \draw [->] (0,0) -- (4.24,4.24); \draw (0,0) -- (8.61,8.61);  \node [right] at (8.61,8.61) {(5,5) $45^\circ$};
% \draw [->] (0,0) -- (4.24,4.24); \draw (0,0) -- (8.61,8.61);  \node [right] at (8.61,8.61) {(6,6) $45^\circ$};
\draw [->] (0,0) -- (3.84,4.61); \draw (0,0) -- (8.2,9.85);  \node [right] at (8.2,9.85) {(5,6) $50.19^\circ$};
\draw [->] (0,0) -- (3.75,4.69); \draw (0,0) -- (8.1,10.12);  \node [right] at (8.1,10.12) {(4,5) $51.34^\circ$};
\draw [->] (0,0) -- (3.6,4.8); \draw (0,0) -- (7.92,10.56);  \node [right] at (7.92,10.56) {(3,4) $53.13^\circ$};
\draw [->] (0,0) -- (3.33,4.99); \draw (0,0) -- (7.57,11.35);  \node [right] at (7.57,11.35) {(2,3) $56.31^\circ$};
% \draw [->] (0,0) -- (3.33,4.99); \draw (0,0) -- (7.57,11.35);  \node [right] at (7.57,11.35) {(4,6) $56.31^\circ$};
\draw [->] (0,0) -- (3.09,5.14); \draw (0,0) -- (7.22,12.03);  \node [right] at (7.22,12.03) {(3,5) $59.04^\circ$};
\draw [->] (0,0) -- (2.68,5.37); \draw (0,0) -- (6.57,13.13);  \node [right] at (6.57,13.13) {(1,2) $63.43^\circ$};
% \draw [->] (0,0) -- (2.68,5.37); \draw (0,0) -- (6.57,13.13);  \node [right] at (6.57,13.13) {(2,4) $63.43^\circ$};
% \draw [->] (0,0) -- (2.68,5.37); \draw (0,0) -- (6.57,13.13);  \node [right] at (6.57,13.13) {(3,6) $63.43^\circ$};
\draw [->] (0,0) -- (2.23,5.57); \draw (0,0) -- (5.73,14.32);  \node [right] at (5.73,14.32) {(2,5) $68.2^\circ$};
\draw [->] (0,0) -- (1.9,5.69); \draw (0,0) -- (5.05,15.15);  \node [right] at (5.05,15.15) {(1,3) $71.57^\circ$};
% \draw [->] (0,0) -- (1.9,5.69); \draw (0,0) -- (5.05,15.15);  \node [right] at (5.05,15.15) {(2,6) $71.57^\circ$};
\draw [->] (0,0) -- (1.46,5.82); \draw (0,0) -- (4.05,16.2);  \node [right] at (4.05,16.2) {(1,4) $75.96^\circ$};
\draw [->] (0,0) -- (1.18,5.88); \draw (0,0) -- (3.36,16.82);  \node [right] at (3.36,16.82) {(1,5) $78.69^\circ$};
\draw [->] (0,0) -- (0.99,5.92); \draw (0,0) -- (2.87,17.23);  \node [right] at (2.87,17.23) {(1,6) $80.54^\circ$};
\end{tikzpicture}
%\hrule\hrule
\end{figure}
\end{ForMe}
\subsubsection{Scalar Quantities versus Vector Quantities} \label{sss:scalarvector}\mlinkreturn[the direction of forces]{d:pushvector}
\subsubsection{Vector Equations}\label{sss:vectorequations}\mmultireturn{\mmr{\hyperlink{d:2Dmotion}{the ballistic freefall}}, \mmr{\hyperlink{d:f=ma}{$F=ma$}}}
\ldots If $\vec A = 3 \vec B$, then this is true for each component.
\begin{eqnarray}
A_x & = & 3 B_x \\
A_y & = & 3 B_y \\
A_z & = & 3 B_z
\end{eqnarray}

This can also be written in two different ways:
\[ \vect{A_x}{+A_y}{+A_z} \ = \ 3 \left( \vect{B_x}{+B_y}{+B_z} \right) \ = \ \vect{(3B_x)}{+(3B_y)}{+(3B_z)} \]

This will be useful when\foreshadow{} we are discussing \hyperref[ss:ballistic]{ballistics} (2-dimensional motion), \hyperref[ss:NII]{Newton's second law} (combining multiple forces pushing on an object), \hyperref[s:2Dcollisions]{2-dimensional collisions}, and the calculation of \hyperref[ss:Efield]{electrical fields}.

\subsubsection{Multiplication, but Not Division}\label{sss:vectorproducts}

[define dot product]

\noindent
[define cross-product]

\noindent
Can do magnitude-equations like $F=ma$ or $m=F/a$.  But for vector equations, while you can do $\vec F=m\vec a$, you cannot do something like
\hypertarget{d:dividevectors}{$\deq m = \frac{3\ihat+4\jhat}{2\ihat-5\jhat}$}\mreturn{se:netF-m}; but, in that case, you can use the magnitudes as follows
\[ m = \frac{\sqrt{(3)^2+(4)^2}}{\sqrt{(2)^2+(-5)^2}} = \frac{\sqrt{25}}{\sqrt{29}} = \sqrt{\frac{25}{29}} = 0.\sig{9}{28}{} \]

\chapter{Estimating and Uncertainty}

\section{Precision and Accuracy}\label{s:precision}\index{Precision}\new{v2.2}{Added section, added some detail}

In this section, we will consider the benefits of being precise both in measurements and in our language.  Sometimes people confuse the words precise and accurate, but they mean different things.  It may help to remember that the opposite of precise is vague.  Being precise makes it easier to determine if a statement is accurate.  If we already know the answer, then we can know if a result is accurate.  However, the exciting aspect of science is to study that which we do not already know.  In this case, gauging accuracy can be tricky.  If we are do not already know an answer, then we can try to be consistent within our accepted precision.

Since physics has its roots in the natural philosophy of the ancient Greeks and developed mathematically with Galileo and Newton, it has been around long enough for the technical language to both evolve (Newton used the word ``action'' for what we refer to as ``force'') and to be absorbed into everyday (colloquial) language. Words like force and energy have taken on broader meanings in English.  In this text, we will try to be precise with the language.  Hopefully we can avoid using the dismissive phrase, ``Oh, you \textit{know} what I \textit{mean}.''

One example\mautoreturn{ss:weightmass} of not being careful with the language comes when people use the term ``massive'' to mean ``big.''  The word massive actually means ``has a large amount of material'' whereas big means ``takes up a large amount of space'' (which might be replaced by the word ``voluminous'' rather than ``massive''). These are related by \hyperref[s:density]{the density}\index{Density} but it is possible to be massive and not voluminous (see, for example, the discussion of \hyperref[s:blakhole2]{black holes}).  While it is \textit{technically} inaccurate to use massive to mean big, ``we'' know what ``we'' mean.

\section{Significant Figures}\label{s:sigfig}\mmultireturn{\mmr{\autoref{ss:convertunits}}, \mmr{\autoref{ss:weightmass}}}

Note the comments in \autoref{ss:weightmass} regarding an internet search on the ``\hyperref[ss:convertunits]{conversion}'' of kilograms-to-pounds.\index{Significant Digits}\dothis{Remove this sentence and make the next paragraph sensible.  It makes more sense here than in \protect{\autoref{ss:weightmass}}.}

A short Google\textsuperscript{tm} search by the author found that the conversation rate between pounds and kilograms was $1 \unit{kg} = 2.2046226218 \unit{lbs}$.  Several sites go on to list about 10 decimal places for all of the conversions.  First, you should recall our discussion about \hyperref[s:sigfig]{significant digits}\index{Significant Digits}\dothis{Refocus this paragraph as an \textit{example} about significant digits.}\new{v2.2}{moved this conversation here.}.  Second you should note that the unit of pounds is a measure of force (how much the Earth pulls on you)\index{Weight}, whereas the unit of kilogram is a measure of mass (how much ``stuff'' there is)\index{Mass}.  These are related in proportion to the strength of the gravitational field, which varies in the third digit (on the order of about 1\%\addlink{variation in $g$}) around the globe.  Some sites indicate that they are shortening their conversion factor to 3 digits for convenience, but this is not an issue of convenience, it is an issue of precision\index{Precision}.

\section{Scientific Notation}


\section{Effective Theories}\label{s:effective2}\mmultireturn{\mmr{\hyperref[ss:noninertial]{non-inertial reference frames}}, \mmr{air resistance \autoref{ss:airresistance}}, \mmr{air resistance \autoref{ss:ballisticairresistance}}, \mmr{\hyperref[ss:NI]{Newton's first law}}, \mmr{\hyperref[ss:NII]{Newton's second law}}, \mmr{\autoref{s:Fg}}, \mmr{\hyperlink{d:fundamental}{fundamental forces}}, \mmr{\autoref{s:FT}}}\new{v2.2}{Added section to indicate approximate truth}\dothis{Should this be here or in \protect{\autoref{s:effective1}}?}{}

Life is complicated.  One mechanism that scientists in general and physicists in particular use to simplify their descriptions of the world around us is to build an effective theory.  These are not intended to be true (accurate) to as many decimal places as can be calculated, but rather are intended to be good enough.  In this context, good enough is most likely to mean something like: true to a reasonable number of decimal places.

A colloquial example of this is when you wear a smile to give the impression of happiness even if you are not in the mood.  Most of your casual interactions will be the same as when you are in a good mood, but your friends who know you better will recognize the small discrepancies.

A technical example of this is that Newton's theory of gravity is very precise as long as none of the objects being described are travelling ``close'' to the speed of light.  How close counts as close depends on the level of precision the measurement needs to be.  If any of the objects are moving close to the speed of light, then we need Einstein's general theory of relativity.  One way to describe this is that Newton's theory is a special case of Einstein's Theory.  Another way is describe it is that Newton's theory is an effective theory for Einstein's theory, effective when the speed is low.  It is possible for us to measure the difference between Newton's theory and Einstein's theory, but it is often not worth the effort of using the more complex theory in the cases where the simpler one will do; it is effectively true (rather than actually true).

Another technical example is that Einstein's special theory of relativity is a special case of the general theory of relativity.  The aspect that makes it a special case is that the special theory only considers motion without acceleration.

One final case that should be mentioned up front is to notice that humans experience the Earth \textit{as if} it were stationary.\dothis{Decide if this should be filled out more or if it should reference the variety of places where the text fills out these kinds of details.}{}

\part{Introducing Motion, Force, and Energy}

The trio of topics in this part of the book are fundamental and powerful concepts\footnote{The idea of Fundamental and Powerful Concepts (F\&PC) is taken from Dr. Gerald Nosich, \textit{Learning to Think Things Through}, Prentice Hall, 2012.}.  These are fundamental in that most other topics in physics are built upon them.  They are powerful in that if they are well-understood, then one is empowered to use them to understand and develop an intuition for nearly any other topic that is experienced.  It has been my experience that with these topics, students can jump into a surprisingly wide variety of other, significantly more esoteric, topics and develop a reasonable grasp of the key concepts.  Furthermore, the development of understanding of these ideas introduce the language and thought processes of being a professional physicist such that it nicely bridges the language barrier that might otherwise exist due to the jargon of physics.

\chapter{One-Dimensional Motion}\label{c:motion}

\section{How Physicists Use the Words (Notation)}

\begin{itemize}
\item Position = where is it?  Also discuss location as a vector and giving directions as defining a coordinate system (locate a common origin and unit-vector, then give a series of magnitudes and directions).
\begin{itemize}
\item This chapter will distinguish location versus distance.
\item This chapter will distinguish distance traveled versus displacement.
\end{itemize}
\item Velocity = which way did it go?  Is its position changing?
\begin{itemize}
\item This chapter will distinguish speed and velocity.
\item Introduce the language of ``at rest''.
\end{itemize}
\item Acceleration = Is its velocity changing?
\begin{itemize}
\item This chapter will clarify acceleration, deceleration, and changing direction.
\item This chapter will distinguish distance traveled versus displacement.
\end{itemize}
\end{itemize}

\section{Connecting the Concepts: distance equals rate times time}

\subsection{Position}

Identifying the position requires identifying a common known position (which we could call ``the origin''), a distance from that known location (which we could call ``a magnitude''), and a direction from the origin in which to travel such a distance.  The common example\inlife{} that identifies the location as ``I am in my room'' references ``your room'' as the common, known origin.  If the author of this text were to tell you that he was in his room, then your next obvious question is: ``OK, but where is your room?''

Position can be seen to be a vector when you describe a meeting place or destination to a friend who has never been to that location:  ``Well, you know where the bookstore is, right?'' (establishes a common origin).  ``OK, so, if you face the sports gear shop\ldots'' (sets the coordinate axis and defines the position direction) ``\ldots turn left and walk a block'' (defines the magnitude and the direction).

\subsection{Speed versus Velocity}

When you are not moving, physicists will describe you as being \hypertarget{d:atrest}{``at rest''}\mlinkreturn[Newton's First Law]{d:atrestinmotion}.  When you drive to the store\inlife, your car ``starts from rest'' and then travels some distance in some time.  When you arrive at the store, your car ``ends at rest'' when you arrive at your destination.

When you are moving\ldots\dothis{Discussion of speed as $\txtfrac{\Delta x}{\Delta t}$}{}

To be moving, you must be moving in a particular direction.\dothis{Discussion of velocity as a vector}{}


\subsection{Adding Velocities}\label{ss:addvel}

Comment on inertial \hyperlink{d:referenceframe}{reference frames}.\index{Reference Frames!Inertial}

\section{Extending the Concepts: Changing How You Move}\label{s:acceleration}\mmultireturn{\mmr{\hyperlink{d:NewtonInertial}{non-inertial reference frames}}, \mmr{\hyperlink{d:f=ma}{$F=ma$}}}

\subsection{Moving versus Speeding Up}\label{ss:acceleration}

\begin{itemize}
\item Description of \hypertarget{d:motion}{``moving''} as \textit{moving at constant velocity}.
    \mmultireturn{\mmr{\hyperlink{d:objectinmotion}{objects in motion}}, \mmr{\hyperlink{d:atrestinmotion}{Newton's First Law}}}

\item Description of \hypertarget{d:acc}{\textit{accelerating}} as either ``accelerating'', ``decelerating'', or ``turning.''\mautoreturn{s:forcewords}

\end{itemize}

Discussion of \autoref{ex:slowcar}  (pg.~\pageref{ex:slowcar}) and \autoref{ex:coasting}  (pg.~\pageref{ex:coasting}).

\section{Connecting the English to the Math}\label{s:EOM}\mreturn{se:netF-a}

\hypertarget{d:EOM}{The equations of constant acceleration}\mautoreturn{ex:ceiling} can be summarized as\new{v2.3}{referenced.   Need the story of these equations}
\begin{eqnarray*}
x_f & = & x_i + v_i t + \frac{1}{2} a t^2 \\
v_f & = & v_i + a t \\
v_f^2 & = & v_i^2 + 2 a \, \Delta x
\end{eqnarray*}

\section{Examples}

\begin{example}[hbpt]
\fcolorbox{black}{yellow!10}{\begin{minipage}{4.925in}
\caption{\label{ex:slowcar} How far will you go?}
You and your friend, \studentB\index{\studentB}, are driving along at $55.0\unit{mph}$ and run out of gas $2.25\unit{mi}$ from a gas station.  You leave the car in gear and find that after $t_1=1.00\unit{min}$, you are travelling $v_1=30\unit{mph}$.  Will you make it to the gas station?

\color{blue}
The first thing we should do is notice what information is given to us and make sure that everything is in consistent units.  I will convert everything to \hyperref[ss:convertunits]{SI units}.
\begin{eqnarray*}
v_i & = & 55.0\unitfrac{mi}{hr} \convert{1609 \unit{ft}}{1.0000 \unit{mi}}_{4\unit{sig}} \convert{1\unit{hr}}{3600\unit{s}}_\mathrm{exact} = \sigfrac{24.5}{8}{m}{s} \\
\Delta x & = & 2.25\unit{mi} \convert{1609 \unit{ft}}{1.0000 \unit{mi}}_{4\unit{sig}} = \sig{362}{0.3}{m} = \sig{3.62}{0\ten{3}}{m} \\
\end{eqnarray*}
[This example is not done, but the work will result in the following numbers:
With $t_1$ and $v_1$, you can find $a=-1500\unitfrac{mi}{hr^2}$.  From that you can find, for $v_f=0\unitfrac ms$, that $t=2.2\unit{min}$ and $\Delta x = \sig{1.00}{8}{mi}$.]

You do not make it to the gas station.

\autoreturn{ss:acceleration}
\color{black}
\end{minipage}}
\end{example}
\begin{example}[hbpt]
\fcolorbox{black}{yellow!10}{\begin{minipage}{4.925in}
\caption{\label{ex:coasting} How fast should you start?}
You and your friend, \studentB\index{\studentB}, are driving along at $55.0\unit{mph}$ and run out of gas $2.25\unit{mi}$ from a gas station.  You put the car in neutral because you know that the car will slow down with an acceleration of $a=500\unitfrac{mi}{hr^2}$.  With what speed should you be going when you put your car into neutral in order to coast to a stop at the gas station?

\color{blue}
The first thing we should do is notice what information is given to us and make sure that everything is in consistent units.  I will convert everything to metric.
\begin{eqnarray*}
v_i & = & 55.0\unitfrac{mi}{hr} \convert{1609 \unit{ft}}{1.0000 \unit{mi}}_{4\unit{sig}} \convert{1\unit{hr}}{3600\unit{s}}_\mathrm{exact} = \sigfrac{24.5}{8}{m}{s} \\
\Delta x & = & 2.25\unit{mi} \convert{1609 \unit{ft}}{1.0000 \unit{mi}}_{4\unit{sig}} = \sig{362}{0.3}{m} = \sig{3.62}{0\ten{3}}{m} \\
\end{eqnarray*}
[This example is not done, but the work will result in the following numbers:
With $a=-500\unitfrac{mi}{hr^2}$.  You can find, for $v_f=0\unitfrac ms$, that $t=6.6\unit{min}$ and $\Delta x = 3.025\unit{mi}$.  You clearly make it to the gas station.  You can also find that for $\Delta x = 2.25\unit{mi}$, $t=\sig{3.25}{9}{min}$ and $v_f=\sigfrac{27.8}{388}{mi}{hr}$.]

So, if you start at $55.0\unitfrac{mi}{hr} - 27.8\unitfrac{mi}{hr} = \sigfrac{27.1}{6}{mi}{hr}$ you should make it exactly.

\autoreturn{ss:acceleration}
\color{black}
\end{minipage}}
\end{example}

\subsection{Freefall}\label{ss:freefall}\index{Freefall}\new{v2.2}{Adding detail}\mautoreturn{ex:ceiling}

Since acceleration is the change in velocity (magnitude and/or direction), it is possible to select your own rate of change while driving your car.  However, that acceleration is difficult to measure directly.  Your speedometer measures the speed and you have to compute your acceleration based on how quickly your speed changes.  It turns out that there is a convenient way to start from the acceleration and compute the expected velocity:  \hypertarget{d:freefall}{Drop a ball or your keys}\mlinkreturn[the description of physics]{d:physicspatterns}\inlife{}.  To convince yourself that objects do, in fact, accelerate when they fall, we can consider dropping items.  One of the complications during such an experiment will be discussed in \autoref{ss:airresistance}.  If we drop a sheet of paper, air resistance causes an obvious effect (fluttering).  For this section, I will assume that the mass-to-surface-area ratio is large enough that we can effectively\Touchstone{Recall \protect{\hyperref[s:effective2]{effective theories}}.}{} ignore the air resistance.

The \hypertarget{d:accgrav}{patterns} that you see when you drop objects is that objects fall faster than humans are used to paying attention to.  The green box of \autoref{irl:freefall} (on page~\pageref{irl:freefall})\footnote{\protect{\href{https://www.osha.gov/}{OSHA}} standard \protect{\href{https://www.osha.gov/pls/oshaweb/owadisp.show_document?p_table=standards&p_id=10839}{1926.1053(a)(3)(i)}} says ``Rungs \ldots of portable \ldots  and fixed ladders \ldots shall be spaced not less than 10 inches (25 cm) apart, nor more than 14 inches (36 cm) apart \ldots ''} shows you how you can pay close attention to the patterns that result from observing falling objects.
%
\begin{reallife}[bthp]
\hspace{-.2in}
\fcolorbox{black}{green!10}{\begin{minipage}{5.29in} \center
\caption{\label{irl:freefall}\index{Freefall!Real Life} The motion of dropped objects.}
\begin{minipage}{4.925in}
Because \studentC\index{\studentC} is a pitcher on the local baseball team, \heC\ decides to drop a ball and watch what happens.  You and \studentD\index{\studentD} decide to join him.  \studentD\ provides a few other objects that can also be dropped: a tennis ball, a hammer, a small Wonder Woman toy, and a broken cell phone.  Some of these are dropped at the same time.  \studentD\ notices that it is important to release the objects at exactly the same time.
\studentC\ notices that it is important to have the objects line up at the bottom so that if they travel at the same speed, then they hit at the same time.
\end{minipage}
\begin{realtable}
\dna{Drop \textit{any} two objects at the same time from the same height}
    {Are there any objects that always hit first or last? \ref{A:firstfall}}
    {If so, what are the properties of those objects? \ref{A:firstwhy}}
\dna{Drop one of these objects from about eye-level}
    {Observe the speed of the object as it falls}
    {Is the object moving at a constant speed? \ref{A:fallv}}
\dna{Climb a tall ladder, drop the ball from at least eight-feet high}
    {Observe the time it takes the object to pass four rungs near the top of the ladder and compare it to the time it takes the object to pass four rungs near the bottom of the ladder}
    {Is one set of four-rungs a shorter time or are they the same amount of time? \ref{A:falla}}
\end{realtable}
\begin{minipage}{4.925in}
You and your friends should get together to see if you can come up with a way to measure the acceleration due to the gravitational force.
\end{minipage}

\flushright
\multireturn{\mmr{\hyperlink{d:accgrav}{freefall}}, \mmr{\hyperlink{d:Fgrav}{the force of gravity}}}
\end{minipage}}
\end{reallife}
%
You should go do those experiments before reading further.  Go ahead.  I'll wait.

You did do them, right?  You're not just reading ahead?  Really?  OK.  Doing that experiment will help you see (1) that everything falls at the same rate and (2) that objects accelerate as they fall.
It turns out that, ignoring the effect of \hyperref[ss:airresistance]{air resistance}, all objects fall with the same acceleration (due to the gravity), $a_g = 9.81\unitfrac{m}{s^2}$ downwards.
In this book,\index{Freefall}
\important{``being in freefall'' will mean moving only under the influence of gravity and having an acceleration of $a_g = 9.81\unitfrac{m}{s^2}$ downwards.}
We will start to discuss the reason for this in \autoref{s:Fg} and then get into more detail in \autoref{c:gravity}.
For now, \autoref{ex:freefall} shows the type of experiment that can allow you to calculate the acceleration due to gravity.
%
\begin{example}[hbpt]
\fcolorbox{black}{yellow!10}{\begin{minipage}{4.925in}
\caption{\label{ex:freefall} How quickly does it fall?}
Your friend, \studentC\index{\studentC}, is a baseball player and is curious to learn about the rate that baseballs fly through the air.  You get on a $12\unit{ft}$ ladder and \heC\ lays on the ground below you aiming his radar gun (which measures speed) upwards.  Each rung is $1.0\unit{ft}$ apart and his gun is at the first rung.  When you drop the ball three rungs above the gun, he measures the final velocity to be $4.24\unitfrac ms$.  When you drop the ball six rungs above the gun, he measures the final velocity to be $6.00\unitfrac ms$.  When you drop the ball eleven  rungs above the gun, he measures the final velocity to be $8.11\unitfrac ms$.  Find the acceleration of the ball in each case.

\color{blue}
The first thing we should do is notice what information is given to us and make sure that everything is in consistent units.  I will convert everything to metric.
\begin{eqnarray*}
3\unit{rungs}  & = & 3.00\unit{ft} \convert{0.3048 \unit m}{1.00000\unit{ft}} = 0.\sig{914}{4}{m} \\
6\unit{rungs}  & = & 6.00\unit{ft} \convert{0.3048 \unit m}{1.00000\unit{ft}} = \sig{1.82}{9}{m} \\
11\unit{rungs}  & = & 11.00\unit{ft} \convert{0.3048 \unit m}{1.00000\unit{ft}} = \sig{3.35}{3}{m}
\end{eqnarray*}
To find the acceleration in each case, we can solve $v_f^2 = v_i^2 + 2 a \, \Delta x$ for the acceleration:
\begin{eqnarray*}
a_3 & = & \frac{(4.23\unitfrac ms)^2 - (0 \unitfrac ms)^2}{2(0.\sig{914}{4}{m})} \ = \ \sigfrac{9.83}{1}{m}{s^2} \\
a_6 & = & \frac{(6.00\unitfrac ms)^2 - (0 \unitfrac ms)^2}{2(\sig{1.82}{9}{m})} \ = \ \sigfrac{9.84}{1}{m}{s^2} \\
a_11 & = & \frac{(8.11\unitfrac ms)^2 - (0 \unitfrac ms)^2}{2(\sig{3.35}{3}{m})} \ = \ \sigfrac{9.80}{8}{m}{s^2}
\end{eqnarray*}
Notice that these have some variation due to the rounding.  It turns out that the variation in the value of acceleration depends on the composition of the earth in your location as well as your altitude above sea-level.  That will be discussed in detail in \autoref{c:gravity}, for simplicity we will assume that all objects accelerate at the rate of $9.81\unitfrac{m}{s^2}$ when they are solely under the influence of gravity.

\multireturn{\mmr{\hyperlink{d:accgrav}{freefall}}, \mmr{\autoref{d:Fgball}}}
\color{black}
\end{minipage}}
\end{example}
%
It also turns out that you can also see this acceleration where you throw an object straight up into the air.


\section{Complications}
\subsection{Non-Inertial Accelerated Reference Frames} \label{ss:noninertial}\mmultireturn{\mmr{\hyperlink{d:referenceframe}{Reference Frames}}, \mmr{\autoref{ss:NI}}}
\index{Reference Frames!Inertial}
\index{Reference Frames!Non-inertial}

[Discuss non-rotating linearly accelerating \hyperlink{d:referenceframe}{reference frames}. See also \autoref{s:noninertial} for a discussion on rotating reference frames.]

[Comment on the Earth as essentially stationary?  See \autoref{s:effective2} on effective theories.\new{v2.2}{Effective theories}{}]

\subsection{Air Resistance}\label{ss:airresistance}\mmultireturn{\mmr{\hyperlink{d:accgrav}{freefall}}, \mmr{\ref{A:falls}}, \mmr{\autoref{s:Fg}}}
Terminal velocity\ldots
When do we include air resistance and when can we ignore it?  \ldots
[Comment on air resistance being a small effect in some cases?  See \autoref{s:effective2} on effective theories.\new{v2.2}{Reference effective theories}{}]

\subsection{Multi-Step Solutions}\new{v2.3}{new section, new example}

\begin{example}[hbpt]
\fcolorbox{black}{yellow!10}{\begin{minipage}{4.925in}
\caption{\label{ex:ceiling} \studentC\ hits the ceiling!}
\studentC\index{\studentC} gets bored one day in physics class (what?!?) and tossed a baseball ($m_b = 0.145\unit{kg}$) at the ceiling\ldots a little too hard.  The initial velocity is $v_i = +5.00\unitfrac ms \jhat$ and it leaves \hisC\ hand $1.00\unit{m}$ below the ceiling.  The ball hits the ceiling and when it returns to \hisC\ hand, it is travelling $\vec v_f=-4.73\unitfrac ms \jhat$, slower than \heC\ expected.  (a) Assuming that the ball is in contact with the ceiling for $\Delta t = 0.142\unit{s}$, find the acceleration of the ball during the collision.  (b) On the other hand, if the ceiling had not been there, then how high would the ball have gone and how fast would it have been going when it returned to \studentC's hand?

\color{blue}
In order to solve part (a) for the acceleration, we need to recognize that (1) there are five stages to the motion of the baseball and that (2) \hyperlink{d:EOM}{the equations of constant acceleration} assume that the acceleration is constant.  The five stages of the motion are: the throw, the ball moving from \studentC's hand up to (but not yet hitting) the ceiling, the ball hitting the ceiling, the ball falling from the ceiling down to (but not yet touching) \studentC's hand, and the catching of the ball.  The acceleration is $\deq a = \frac{v_f-v_i}{\Delta t}$, but the story of this equation says that since the acceleration is only during the interaction with the ceiling, then the velocities in this equation are just-before the ball hits and just-after the ball hits (not the very beginning velocity and not the very final velocity).  Similarly, the $\Delta t$ in this equation is only the time during which it was interacting with the ceiling, not the entire flight.

During \underline{the first stage}, the ball is accelerating upwards and \studentC\ is interacting with the ball.  We are not going to consider this part of the motion at all because we are given the velocity that ends this stage (and begins the next stage).

\underline{The second stage} of the motion is while the ball moves from \studentC's hand up to the ceiling. During this stage only the gravitational force is acting on the ball.  Since it has left \studentC's hand, \heC\ is not interacting with it.  Since it has not yet hit the ceiling, the ceiling is not interacting with it.  We can therefore use \hyperlink{d:EOM}{the equations of constant acceleration} to describe the motion.  During this portion of the motion we know that the velocity at the bottom is $\vec v_\mathrm{bot} = +5.00\unitfrac ms \jhat$, that it travels $\Delta \vec x = +1.00\unit{m} \jhat$, and that (because it is in \href{ss:freefall}{freefall}) it is accelerating at $\vec a_g = -9.81\unitfrac{m}{s^2} \jhat$.

\color{black}
{}\hfill {\footnotesize \autoref*{ex:ceiling} continued on next page\ldots}
\end{minipage}}
\end{example}
\begin{example}[p]
\fcolorbox{black}{yellow!10}{\begin{minipage}{4.925in}\setlength{\parskip}{3pt}
{\footnotesize \autoref*{ex:ceiling} continued from previous page\ldots}
\color{blue}

We can find the time of flight (not useful) and the velocity when the ball reaches the ceiling:
\begin{eqnarray*}
v_\mathrm{top} & = & \sqrt{ v_\mathrm{bot}^2 + 2 a \, \Delta x} \\
v_\mathrm{top} & = & \sqrt{ (+5.00\unitfrac ms)^2 + 2 (-9.81\unitfrac{m}{s^2})(+1.00\unit{m})} \\
v_\mathrm{top} & = & +\sigfrac{2.31}{9}{m}{s}
\end{eqnarray*}
Note: When you take the square root, you have to decide if you should take the positive sign or the negative sign.  In this case, the ball is still moving upwards, so we \textit{choose} the positive sign.

\underline{The third stage} is while it is interacting with the ceiling.  In order to find the acceleration during this motion, we need to know the velocity immediately before hitting the ceiling (which we just found) and the velocity just after it finishes hitting the ceiling (which we have not yet found).  We will come back to this step.

\underline{The fourth stage}, like the second, is while the ball moves from the ceiling down to \studentC's hand.  During this portion of the motion we know that the velocity at the bottom (final) is $\vec v_\mathrm{bot} = -1.67\unitfrac ms \jhat$, that it travels $\Delta \vec x = -1.00\unit{m} \jhat$, and that (because it is in \href{ss:freefall}{freefall}) it is accelerating at $\vec a_g = -9.81\unitfrac{m}{s^2} \jhat$.  We can find the time of flight (not useful) and the velocity when the ball leaves the ceiling (initial), solving $v_\mathrm{bot}^2 = v_\mathrm{top}^2 + 2 a \, \Delta x$ for $v_\mathrm{top}$:
\begin{eqnarray*}
v_\mathrm{top} & = & \sqrt{ v_\mathrm{bot}^2 - 2 a \, \Delta x} \\
v_\mathrm{top} & = & \sqrt{ (-1.67\unitfrac ms)^2 - 2 (-9.81\unitfrac{m}{s^2})(+1.00\unit{m})} \\
v_\mathrm{top} & = & -\sigfrac{4.73}{4}{m}{s}
\end{eqnarray*}
Note: When you take the square root, you have to decide if you should take the positive sign or the negative sign.  In this case, the ball is now moving downwards, so we \textit{choose} the negative sign.

\color{black}
{}\hfill {\footnotesize \autoref*{ex:ceiling} continued on next page\ldots}
\end{minipage}}
\end{example}
\begin{example}[p]
\fcolorbox{black}{yellow!10}{\begin{minipage}{4.925in}\setlength{\parskip}{3pt}
{\footnotesize \autoref*{ex:ceiling} continued from previous page\ldots}
\color{blue}

During \underline{the fifth stage}, the ball is accelerating upwards while moving downwards and so \studentC\ is stopping the ball.  We are not going to consider this part of the motion at all. \\

Now that we have the velocities immediately before and after the collision with the ceiling, we can find the acceleration:
\[ a = \frac{v_f-v_i}{\Delta t} = \frac{(-\sigfrac{4.73}{4}{m}{s})-(+\sigfrac{2.31}{9}{m}{s})}{(0.142\unit s)} = -\sigfrac{28.0}{9}{m}{s^2} \,\jhat \]
Notice that the acceleration is negative because the ball went from going up to going down.

To solve part (b), we can just consider from after-thrown to before-caught.  During this motion, assuming there is no ceiling, the entire motion is in freefall, so we can use $v_f^2 = v_i^2 + 2 a \, \Delta x$ and solve for $\Delta x$.  However, we only want to consider from the lowest point to the highest point, not all the way back to \studentC's hand.

\centering{THIS NEEDS TO BE FINISHED}

\flushright
\multireturn{\mmr{\ref{se:ceiling}}, \mmr{\ref{se:throw-up}}}
\color{black}
\end{minipage}}
\end{example}\dothis{finish \protect{\autoref{ex:ceiling}}.  Maybe make it two examples, instead of one?}


\chapter{Two-Dimensional Motion}

\subsection{Ballistic Freefall}\label{ss:ballistic}\mautoreturn{sss:vectorequations}

\hypertarget{d:ballistic}{Discussion about throwing a ball\ldots}\mlinkreturn[the description of physics]{d:physicspatterns}\inlife.

\hypertarget{d:2Dmotion}{For 2-dimensional motion}\mlinkreturn[$F=ma$]{d:f=ma}, we will use \hyperref[sss:vectorequations]{vector equations} to describe the relationships.
When we write $\vec v_f = \vec v_i + \vec a t$, we mean that this relationship holds for the $x$-components and separately for the $y$-components:
\[ v_{fx} = v_{ix} + a_x t \hspace{1cm} v_{fy} = v_{iy} + a_y t \]

\section{Complications}
\subsection{Air Resistance}\label{ss:ballisticairresistance}
Terminal velocity\ldots non-parabolic paths \ldots \autoref{irl:nonparabolic} (pg.~\pageref{irl:nonparabolic})
\begin{reallife}[bhp]
\hspace{-.2in}
\fcolorbox{black}{green!10}{\begin{minipage}{5.29in} \center
\caption{\label{irl:nonparabolic} Baseball pitches are not usually parabolic.}
\begin{minipage}{4.925in}
\studentC\index{\studentC} is a pitcher on the local baseball team.  \HeC\ throws a fast ball, a slider, a curve ball, and a knuckleball.
\end{minipage}
\begin{realtable}
\dna{Go to a baseball game on a calm day.  Sit near third base.}
    {The path of fly balls to left field}
    {Are they parabolic? \ref{A:fly.balls}}
\dna{Go to a baseball game on a calm day.  Sit near third base.}
    {The path of pitch towards home plate.}
    {Are they parabolic? \ref{A:pitches.side}}
\dna{Go to a baseball game.  Sit up high behind home plate.}
    {The path of the baseball for various pitches.}
    {Do they all fly straight over the plate? \ref{A:pitches.top}}
\end{realtable}

\flushright
\multireturn{\mmr{\hyperlink{d:physicspatterns}{the description of physics}}, \mmr{\autoref{ss:ballisticairresistance}}}
\end{minipage}}
\end{reallife}

[Comment on air resistance being a small effect in some cases?  See \autoref{s:effective2} on effective theories.\new{v2.2}{Reference Effective theories}{}]



\chapter{Force}\label{c:force}

\section{How Physicists Use the Words (Notation)}\label{s:forcewords}

The technical term \textbf{force}\index{Force!description} refers to the general idea of pushing or pulling.  In the same way that\touchstone{} physicists use the word \hyperlink{d:acc}{acceleration} (technically \textit{changing the velocity}) to mean \textbf{either} \textit{speeding up} (colloquially ``acceleration'') \textbf{or} \textit{slowing down} (colloquially ``deceleration'') \textbf{or} \textit{changing the direction} (colloquially ``turning''), we will use \hypertarget{d:forcenoun}{\textbf{force} as a noun}\index{Force!noun}\mlinkreturn[heat as a verb]{d:heatverb} referring to the act of pushing or pulling.

\hypertarget{d:interaction}{You should note}\mmultireturn{\mmr{\autoref{ex:braced}}, \mmr{\hyperref[d:Fgball]{the falling ball}}} that you can't have a push or pull without \textbf{both} a thing that pushes or pulls \textbf{and} a thing that is pushed or pulled.\aside{Push or Pull}{By now you may have noticed that it is tedious to keep saying ``pushed or pulled,'' so I will only say ``push'' even when I am including the possibility of ``pushing or pulling''.}{}
\important{Forces are \underline{necessarily} an interaction\index{Interaction}\index{Force!Interaction} between two objects.}
Sometimes we care about the thing doing the pushing or pulling, sometimes we don't.  We always care about the thing being pushed or pulled.  We will \underline{distinguish these objects} by referring to the object that is pushing as the object ``causing the force'' or ``exerting the force'' and by referring to the object that is being pushed as the object ``feeling the force''.  We will \underline{distinguish these forces} as follows: Let's imagine that \studentB\index{\studentB} gives \studentA\index{\studentA} a good-natured shove in the arm.  The following are useful descriptions and are different ways of describing the same action.
\begin{itemize}\itemsep 1pt
\item \studentB\ exerted a force \underline{on} \studentA.
\item \studentA\ felt a force \underline{from} \studentB.
\item There was a force \underline{on} \studentA\ \underline{by} \studentB.
\end{itemize}
The notation for this will be $F_{A,B}$ where the first subscript is the person who felt the force (who the force is ``on'') and the second subscript is who exerted the force (who the force is ``by'').  In those instances when we only care about who is feeling the force and not who is exerting the force, we might just use one subscript $F_A$.  In some cases, there may be two forces acting on one person (or object).  In that case, it will be obvious who is feeling the force and we will use the subscript to distinguish which force it is, such as $F_1$ or $F_2$, rather than who feels the force.  This will be more relevant when we discuss in \autoref{c:forcetype} the types of forces that might be applied.

\hypertarget{d:Newtonahead}{Looking ahead} to \hyperref[s:Newton]{Newton's Laws}\index{Laws!Newton}\Touchstone{Recall the distinction between \protect{\hyperref[s:law]{theories and laws}}.}, you should be ready to notice that the \hyperref[ss:NI]{first law} is about objects that are not feeling a force, the \hyperref[ss:NII]{second law} is about a specific object that is feeling a force, and the \hyperref[ss:NIII]{third law} is about the interaction between the two objects.  In all three of these, we care about the object feeling the force.   It is only in the third law that we care about the object exerting the force.

\hypertarget{d:pushvector}{Looking back}, forces are \hyperref[sss:scalarvector]{vectors}:
\important{Pushing on something intrinsically involves a direction.}
You will use this property to show that multiple people pushing\inlife{} in the same direction increases the effect, whereas multiple people pushing in opposite directions reduces the effect.  One might say that people who push an object in opposite directions work\footnote{After you study \protect{\autoref{s:work}}, this play on words will be hilarious!} against each other.  Because the force is a vector, whenever you are answering a question about a force, you should always expect to give the strength of the force (the \hyperref[ss:vectors]{magnitude}) and \hyperref[ss:vectors]{the direction} of the force (relative to some specific axis, usually the positive $x$-axis).

\section{Connecting the Concepts: Newton's Laws}\label{s:Newton}\mlinkreturn[how to describe forces]{d:Newtonahead}

Newton's Laws (recall \hyperref[s:law]{Theory versus Law}) describe our observations about three questions\inlife:
\begin{enumerate}\itemsep 1pt
  \item What happens to an object when I \textit{don't} push on it?
  \item What happens \textit{to an object} when I do push on it?
  \item What happens \textit{to me} when I push on an object?
\end{enumerate}
The answer to these questions have a precise, concise, technical language and the point of the next three subsections are to translate that into (modern) English, into math, and into intuition.  The statement of these laws has slightly different versions in different texts to emphasize different points.  We will state them as follows\index{Newton!Laws}\index{Laws!Newton}:
\begin{enumerate}
    \item \hypertarget{sum:Newton'sLaws}{When} viewed from an inertial reference frame, an object with no forces acting on it will maintain its velocity, which may be zero.
    \item When viewed from an inertial reference frame, the vector-sum of all forces acting on an object will cause that object to accelerate in proportion to its mass: $\vec F_\mathrm{net} = m \vec a$.
    \item For every force acting (the ``action'') on one object by an other object, there is an equal-in-magnitude reaction-force acting on the other object in the opposite direction.
\end{enumerate}
\hypertarget{d:NewtonInertial}{There are a few terms} that should be clarified in these laws.  Being in an inertial \hyperlink{d:referenceframe}{reference frame}\index{Reference Frames!Inertial} essentially means being in a place in which you do not have to hold on in order to maintain your position.  If you are \hyperref[s:acceleration]{accelerating} (recall that this means \textit{speeding up}, \textit{slowing down}, or \textit{changing direction}) then you are not in an inertial reference frame, but rather are in a non-inertial reference frame.  In this case, you will misinterpret the forces acting.  This will be discussed in more detail in \autoref{s:noninertial} when we discuss centripetal force and in \autoref{ss:coriolis} when we discuss the Coriolis effect.

Sometimes Newton's first law is written to include the phrase ``an \hypertarget{d:objectinmotion}{object in motion}'', which I will be careful to link directly to \hyperlink{d:motion}{velocity}, as was done above.  However, it technically should reference the momentum\foreshadow, which is discussed in \autoref{c:momentum}.

The way Newton's third law is often written and referred to includes the words ``action'' and ``reaction''.  Newton was referring to forces with these words and to keep it clear in our discussion, we will use the word force, with the occasional clarification of the action-force or the reaction-force.

\subsection{Translating Newton's First Law: The Law of Inertia}\label{ss:NI}\mlinkreturn[how to describe forces]{d:Newtonahead}

%\important{} is not designed to start a new paragraph
\ \vspace{-12pt}
%\begin{quote}
\important{\textbf{Newton's First Law}:\index{Newton!First Law} When viewed from an inertial reference frame, an object with no forces acting on it will maintain its velocity, which may be zero.}
%\end{quote}
Let's take this apart and connect it to your daily experiences.  Looking ahead, we will discuss the surface of the Earth as a \hyperlink{d:noninertial}{non-inertial rotating reference frame}\Touchstone{You might also recall the discussion in \protect{\autoref{ss:noninertial}}.}; however, the effect is small enough that for most of what we \hyperlink{d:casual}{casually observe}, we can safely pretend that the Earth is stationary and that we are actually at rest while sitting on the curb watching the world go by.  This is so true that our human brains already interpret everything around us as though it were an inertial reference frame.  This psychological perspective is exactly the feature that both allows us to make fairly reliable predictions about the world around us \textit{and} causes us to make incorrect judgements when we encounter non-inertial reference frames.  That is to say, as long as we don't measure our world too closely\Touchstone{Recall \protect{\hyperref[s:effective2]{effective theories}}.}, we are viewing it from an essentially inertial reference frame.  This point is so implicit, that many books do not even include the portion of the statement referring to the reference frame.
\begin{ForMe}
\todo{Consider adding comment about Newton not including ``inertial reference frame'' to his laws.}
%\footnote{Newton himself did not include it in his original statement.}%\urgcap{Newton's ``inertial frames''}{check the implications of this}{}
\end{ForMe}

\hypertarget{d:atrestinmotion}{The rest of this statement} is often written a little differently (and less concisely) as ``an object at rest remains at rest unless acted on by an external force and an object in motion remains in motion unless acted on by an external force.''  Since being \hyperlink{d:atrest}{``at rest''} is a statement about the velocity ($\vec v=0$) and being \hyperlink{d:motion}{``in motion''} is also a statement about the velocity, each of these statements can be understood as saying that
\important{Forces are those things that cause a change in the velocity.}
In other words, Newton's first law says that without a force, the velocity will not change.  In the discussion of \hyperref[sss:equilibrium]{equilibrium}, we will note that this is often extended to say that without a \hyperref[sss:netforce]{net force} the velocity will not change, but that is a special case of \hyperref[ss:NII]{Newton's second law}.

\subsubsection{Inertia}\label{sss:inertia}\index{Inertia}

This law is often called the ``law of inertia''.  The concept of inertia can be described as \textit{the tendency of an object to maintain its velocity}.  This is describing how the object behaves when you don't do anything to it.  The inertia is not a quantity that physicists calculate, but physicists do refer to objects as having a lot of inertia, usually to indicate that it will take a large force to change the object's motion, or as having a small amount of inertia, usually to indicate that it should be relatively easy to change the object's motion.  However, the inertia does not actually refer to the force needed.  Instead, the inertia most often refers to the ``inertial mass'' of an object, which shows up in the second law.
\important{Inertia is not a force.}
Sometimes when physicists are not being careful with their language, they will appear to use the word inertia interchangeably with the term \hyperref[c:momentum]{momentum}, which we will discuss in more detail in \autoref{ss:inertia}.

\subsubsection{How the Laws Work Together}\label{sss:NItogether}

You should notice that Newton's First Law is about what happens when you are \textit{not pushing} on the object, which is to say, the tendency of an object to maintain its own motion without a force acting on it; this is the inertia of the object.  On the other hand, Newton's Second Law is about what happens \textit{to the object} when you \textit{do push} on an it.  This is what we will consider next.  After that, Newton's third law will describe what happens \textit{to the thing pushing} rather than just to the thing being pushed.  \hyperref[sss:NIItogether]{The section} at the end of \autoref{ss:NII} will explore these ideas further.



\subsection{Translating Newton's Second Law: The Equation Law}\label{ss:NII}\mmultireturn{\mmr{\autoref{sss:vectorequations}}, \mmr{\hyperlink{d:Newtonahead}{how to describe forces}}, \mmr{\hyperlink{d:atrestinmotion}{Newton's first law}}, \mmr{\autoref{d:Fgball}}}

%\important{} is not designed to start a new paragraph
\ \vspace{-12pt}
%\begin{quote}
\important{\textbf{Newton's Second Law}\index{Newton!Second Law}: When viewed from an inertial reference frame, the vector-sum of all forces acting on an object will cause that object to accelerate in proportion to its mass: $\vec F_\mathrm{net} = m \vec a$.}\docaption{Inline-math formatting}{Why does an equation that starts a new line get indented slightly?}
%\end{quote}
Let's take this apart and connect it to your daily experiences.  As with Newton's First Law, the \hyperlink{d:noninertial}{non-inertial rotating reference frame} of the surface of the Earth is a small enough effect that, as long as we don't measure our world too closely\Touchstone{Recall \protect{\hyperref[s:effective2]{effective theories}}.}, we can pretend that we are viewing it essentially from an inertial reference frame.

\hypertarget{d:f=ma}{For this law}, it is often sufficient to write down the equation and know that the words are there for back-up.  While most people have no trouble remembering $F=ma$, it is important to pay attention to two aspects:
\begin{itemize}
\item This is a vector-equation, which means\Touchstone{Recall (1) the generic explanation for \protect{\hyperref[sss:vectorequations]{vector equations}} and (2) the \protect{\hyperlink{d:2Dmotion}{vector-equations}} for two-dimensional motion} that
    \begin{itemize}
    \item the equation is true for each component separately, and
    \item the direction of $\vec F_\mathrm{net}$ is the same as the direction of $\vec a$ (which, of course, might be different than the direction of the velocity).
    \end{itemize}
\item The force in this equation is the \textit{net force}, which means that we must consider \underline{all} forces that are acting on this object and only those forces that are acting on \underline{this} object.
\end{itemize}
\addcontentsline{los}{story}{$\vec F_\mathrm{net} =m \vec a$}
\phantomsection\label{st:F=ma}\thestoryof{\vec F_\mathrm{net} = m \vec a}\mautoreturn{ex:2Dforce}
This equation is all about what happens to a specific object, $m$.  If the object, $m$, is accelerating in a particular direction, $\vec a$, then it is because the combination of forces, $\vec F_\mathrm{net}$, do not entirely cancel each other out.  This also can be expressed as: if the combination of forces, $\vec F_\mathrm{net}$, do not entirely cancel each other out, then our friend $m$ must be accelerating,~$\vec a$, in a particular direction.  Furthermore the resulting direction of the net force determines the direction of the acceleration.  \hypertarget{d:f=ma}{Connecting the English} and the math:
\[\begin{array}{cccc}
\deq \vec F_\mathrm{net} & = & \deq m & \deq \vec a \\
\EqStoryOver{65pt}{the combination of all forces acting on $m$}{}
& \EqStoryOver{33pt}{causes}{}
& \EqStoryOver{35pt}{that object}{}
& \EqStoryOver{40pt}{to change its velocity}{}
\end{array}\]
You should \hyperref[s:acceleration]{recall}\touchstone, that the direction of the acceleration does not determine the direction \textit{of the motion}, but rather determines the direction \textit{of the change} in motion.  That idea will be important\foreshadow{} when we discuss how a \hyperref[s:FT]{tension} acts as a \hyperref[s:centripetal]{centripetal force}, the relationship between velocity and acceleration in a \hyperref[s:springs]{spring} that \hyperref[c:SHMspring]{oscillates}, and objects that are propelled through either a \hyperref[s:Gfield]{gravitational} or an \hyperref[ss:Efield]{electrical} field.

\subsubsection{Units of Force}\label{sss:unit-N}\mautoreturn{ss:units}\index{Force!Units of}

Recall that the fundamental units of the \hyperref[ss:units]{SI-system} are meters, kilograms, and seconds (MKS)\addlink{maybe note the MKS-to-SI transition. maybe leave that in \protect{\autoref{ss:units}}}.  With our relationship connecting force to mass $\unit{(kg)}$ and acceleration $\left(\unitfrac{m}{s^2}\right)$, we can see that the units of force are ${}\unitfrac{kg \cdot m}{s^2}$.  This quantity is so common that we would like to have a shorthand for it.  Furthermore, Sir Isaac Newton did such ground-breaking work on the concept, that it was decided in 1948\footnote{According to: International Bureau of Weights and Measures (1977),The international system of units (330-331) (3rd ed.), U.S. Dept. of Commerce, National Bureau of Standards, \protect{\href{https://books.google.com/books?id=YvZNdSdeCnEC&pg=PA17\#v=onepage&q&f=false}{p. 17}},
%  ISBN 0745649742, Found at https://en.wikipedia.org/wiki/Newton_(unit)
which refers to
\protect{\href{http://www.bipm.org/jsp/en/ViewCGPMResolution.jsp?CGPM=9&RES=7}{the 7th resolution}} (Mar, 2017) of
\protect{\href{http://www.bipm.org/jsp/en/ListCGPMResolution.jsp?CGPM=9}{the 9th CGPM}} (Mar, 2017).} to name the unit the Newton, such that
\important{$1 \unit{N} = 1 \unitfrac{kg \cdot m}{s^2}$}

\subsubsection{Calculating the Net Force}\label{sss:netforce}\mlinkreturn[Newton's first law]{d:atrestinmotion}

The word ``net'' that goes with force is here to indicate the total, which is useful to think of as ``everything collected with the net.''\footnote{Although according to \protect{\href{http://www.etymonline.com/index.php?term=net&allowed_in_frame=0}{etymonline.com}} (Mar, 2017), it is actually from the Old French \textit{net} for ``neat'' or ``clean'', having the sense of trim and elegant.}  The intention here is that wherever there are multiple forces acting on a single object, we must combine them as \hyperref[ss:vectors]{vectors} as follows: \\
\begin{minipage}[c]{3.25in}
\begin{sample}
\item\label{se:netFadd} If there is a $5.0 \unit{N}$ force to the right and a $4.0 \unit{N}$ force to the right, then the net force is $9.0 \unit{N}$ to the right.
    \[ \vec F_\mathrm{net} = \vec F_1 + \vec F_2 = \left( 5.0\unit{N} \ihat\right) + \left( 4.0 \unit{N} \ihat\right) = +9.0\unit{N} \ihat \]
\end{sample}
\end{minipage}\mmultireturn{\mmr{\ref{se:netF-a}}, \mmr{\hyperref[sss:equilibrium]{Equilibrium}}, \mmr{\ref{se:FBD-AB}}, \mmr{\autoref{f:firstFBD}}, \mmr{\hyperlink{d:FBD-AB}{discussion about \ref*{se:FBD-AB}}}}
\hfill
\fbox{\begin{minipage}[c]{1.5in}
\begin{FBD}{10}{15}{15}{10}{object}
\twori{50}{$5\unit N$}{black}{40}{$4\unit N$}{black}
\end{FBD}
\end{minipage}}\dothis{Make this object a desk so that we can have \studentA\ and \studentB\ helping you rearrange your room in your residence hall.}{}
\begin{minipage}[c]{3.25in}
\begin{sample}
\item\label{se:netFsub} If there is a $5.0 \unit{N}$ force to the left and a $4.0 \unit{N}$ force to the right, then the net force is $1.0 \unit{N}$ to the left.
    \[ \vec F_\mathrm{net} = \vec F_1 + \vec F_2 = \left(-5.0\unit{N} \ihat\right) + \left( 4.0 \unit{N} \ihat\right) = -1.0\unit{N} \ihat \]
\end{sample}
\end{minipage}\mmultireturn{\mmr{\ref{se:netF-m}}, \mmr{\hyperref[sss:equilibrium]{Equilibrium}}}
\hfill
\fbox{\begin{minipage}[c]{1.5in}
\begin{FBD}{10}{15}{15}{10}{object}
\onele{50}{$5\unit N$}{black}
\oneri{40}{$4\unit N$}{black}
\end{FBD}
\end{minipage}}
\begin{minipage}[c]{3.25in}
\begin{sample}
\item\label{se:equi} If there is a $3.0 \unit{N}$ force to the right and a $3.0 \unit{N}$ force to the left, then the net force is $0.0 \unit{N}$.
    \[ \vec F_\mathrm{net} = \vec F_1 + \vec F_2 = \left( 3.0\unit{N} \ihat\right) + \left(-3.0 \unit{N} \ihat\right) = 0.0\unit{N} \ihat \]
    \mbox{In this case, the object is said to be ``\hyperref[sss:equilibrium]{in equilibrium}.''}
\end{sample}
\end{minipage}\mmultireturn{\mmr{\hyperref[sss:equilibrium]{Equilibrium}}, \mmr{\hyperlink{d:equi}{discussion about \ref*{se:equi}}}}
\hfill
\fbox{\begin{minipage}[b]{1.5in}
\begin{FBD}{10}{15}{15}{10}{object}
\onele{30}{$3\unit N$}{black}
\oneri{30}{$3\unit N$}{black}
\end{FBD}
\end{minipage}}
The images included in these examples will eventually\foreshadow{} be referred to as ``\hyperref[sss:FBD]{free-body diagrams}\index{Free-Body Diagrams!Images},'' but for now, you can just consider them images of the forces acting on the bodies.

\hypertarget{d:netforce}{Next}, we should do a couple of examples that show the math for situations with forces in two dimensions.   The first, \autoref{ex:2Dforce}  (pg.~\pageref{ex:2Dforce}), has one force in the $x$-direction and another in the $y$-direction.
%
\begin{example}[hb]
\fcolorbox{black}{yellow!10}{\begin{minipage}{4.925in}
\caption{\label{ex:2Dforce} An object is pushed by perpendicular forces.}
A $2.0\unit{kg}$ mass is being pushed north with $5.0\unit{N}$ and east with $4.0 \unit{N}$.  What is the net force?
\color{blue}

\vspace{9pt}
\begin{minipage}{3.25in}
Since we have multiple forces acting on a mass to cause an acceleration, it should be clear (recall the \hyperref[st:F=ma]{story}) that we need to use Newton's second law and find the net force in order to compute the acceleration.  We will, as usual, start with a free-body diagram (at right).  This example is made easier because the forces happen to be at right angles and so finding their $x$ and $y$ components is not difficult. By adding
\end{minipage}
\hfill
\fbox{\begin{minipage}{1.5in}
\begin{FBD}{10}{15}{15}{40}{object}
\oneup{50}{$5\unit N$}{black}
\oneri{40}{$4\unit N$}{black}
\end{FBD}
\end{minipage}}
\vspace{2pt}

\begin{minipage}[c]{1.95in}
\begin{forcetable}
\force{F_1}{ 0 \unit N}{+5 \unit N}
\force{F_2}{+4 \unit N}{ 0 \unit N} \hline\hline
\force{F_\mathrm{net}}{+4\unit N}{+5\unit{N}}
\end{forcetable}
\end{minipage}
\hfill
\begin{minipage}[c]{2.8in}
the $x$-components and separately adding the $y$-components, we have found the components of the net force.  From there, we can easily find the magnitude and direction of the net force.
\end{minipage}
\magdir{+4\unit N}{+5\unit{N}}{F_\mathrm{net}}{\sig{6.4}{0}{N}}{\theta}{\sig{51}{.3}{^\circ}}{N of E}
%
(The direction can be stated as $\theta = 51^\circ \textrm{N of E}$ or as $\phi = 39^\circ\textrm{E of N}$.)
\autoref{ex:2Dfa}  (pg.~\pageref{ex:2Dfa}) will use this calculation to find the acceleration.
\flushright
\multireturn{\mmr{the discussion of \hyperlink{d:netforce}{the net force}}, \mmr{\autoref{ex:2Dforce2}}}
\end{minipage}}
\end{example}
%
The second, \autoref{ex:2Dforce2}  (pg.~\pageref{ex:2Dforce2}), has one force in the $x$-direction and the other in the second quadrant.
%
\begin{example}[hb]
\fcolorbox{black}{yellow!10}{\begin{minipage}{4.925in}
\caption{\label{ex:2Dforce2} Three forces act on an object.}
A $2.0\unit{kg}$ mass is being pushed northwest with $5.0\unit{N}$ at an angle $63.4^\circ\textrm{N of W}$, southwest with $6.0\unit{N}$ at an angle of $21.8^\circ\textrm{S of W}$, and east with $4.0 \unit{N}$.  What is the net force?
\color{blue}

\vspace{9pt}
\begin{minipage}{3.25in}
This follows the same logic as \autoref{ex:2Dforce}  (pg.~\pageref{ex:2Dforce}), which I will not restate here. This example is slightly harder because the forces have to be split into their $x$ and $y$ components. By adding the $x$-components and separately adding the $y$-components, we have found the components of the net force.  From there, we can easily find the magnitude and direction of the net force.
\end{minipage}
\hfill
\fbox{\begin{minipage}{1.5in}
\begin{FBD}{10}{15}{25}{40}{object}
%\oneup{50}{$5\unit N$}{black}
%\put(24,56){\vector(-1,2){22.36}}
%\put(0,102){\color{black} \tiny $5\unit N$}
\oneul{22}{48}{-1}{2}{$5 \unit N$}{black}
\onedl{58}{22}{-5}{-2}{$6 \unit N$}{black}
\oneri{40}{$4\unit N$}{black}
\end{FBD}
\end{minipage}}
\vspace{2pt}

%\begin{minipage}[c]{1.95in}
\begin{forcetable}
\foTWO{F_1}{-(5.0\unit N)\cos(63.4^\circ) =}{-\sig{2.2}{4}{N}}{(5.0\unit N)\sin(63.4^\circ) =}{+\sig{4.4}{7}{N}}
\foTWO{F_2}{-(6.0\unit N)\cos(21.8^\circ) =}{-\sig{5.5}{7}{N}}{-(6.0\unit N)\sin(21.8^\circ) =}{-\sig{2.2}{3}{N}}
\force{F_3}{+4.0 \unit N}{ 0 \unit N} \hline\hline
\force{F_\mathrm{net}}{-\sig{3.8}{1}{N}}{+\sig{2.2}{4}{N}}
\end{forcetable}
%\end{minipage}
%
\magdir{-\sig{3.8}{1}{N}}{+\sig{2.2}{4}{N}}{F_\mathrm{net}}{\sig{4.4}{2}{N}}{\theta}{\sig{30}{.5}{^\circ}}{N of W}
%
(The direction can be stated as $\theta = 31^\circ \textrm{N of W}$ or as $\phi = 60^\circ\textrm{W of N}$.)
\autoref{ex:2Dfa2}  (pg.~\pageref{ex:2Dfa2}) will use this calculation to find the acceleration.

\linkreturn[the net force]{d:netforce}
\end{minipage}}
\end{example}


\subsubsection{Using the Net Force to Calculate Other Quantities}

Generally, the point of finding the net force is that it causes an object to change its velocity. Let's also consider a few simple examples of this calculation.
\begin{sample}
\item\label{se:netF-a}\mmultireturn{\mmr{\hyperlink{d:m=f/a}{finding $m$ from $F=ma$}}, \mmr{\ref{se:FBD-AB}}, \mmr{\autoref{f:firstFBD}},\mmr{\hyperlink{d:FBD-AB}{discussion about \ref*{se:FBD-AB}}}, \mmr{\ref{se:weightA}}, \mmr{\ref{A:netF-a}}, \mmr{\autoref{f:firstFBDupdate}}} If the forces in \ref{se:netFadd}\dothis{As before, make this object a desk so that we can have \studentA\ and \studentB\ helping you rearrange your room in your residence hall.  (Change the mass!)}{} are applied to an object with mass $2.0\unit{kg}$, then it will accelerate at the rate of
    \[ \vec a =\frac{\vec F_\mathrm{net}}{m} = \frac{+(9.0\unit N)\ihat}{2.0 \unit{kg}} = 4.5\unitfrac{N}{kg} \ihat = 4.5\unitfrac{kg \cdot m}{s^2 \cdot kg} \ihat \ = \ 4.5 \unitfrac{m}{s^2} \ihat \]
    which (recall \autoref{s:EOM}), after acting for $1.6\unit{s}$ on an object originally at rest, would result in a final speed of
    \[ v_f = (0\unitfrac{m}{s}) + (+4.5\unitfrac{m}{s^2})(1.6\unit{s}) = 7.2 \unitfrac{m}{s} \]
\end{sample}
We can do this same kind of procedure for the case when forces are in two dimensions.\dothis{Merge \autoref{ex:2Dfa} and \autoref{ex:2Dfa2}.  Also reference the example here.}
%
\begin{example}[p]
\fcolorbox{black}{yellow!10}{\begin{minipage}{4.925in} \setlength{\parsep}{3pt}
\caption{\label{ex:2Dfa} Moving a pushed box.}
A $2.0\unit{kg}$ mass is being pushed north with $5.0\unit{N}$ and east with $4.0 \unit{N}$.  What is the acceleration?
\color{blue}

\vspace{9pt}
\begin{minipage}{3.25in}
\autoref{ex:2Dforce}  (pg.~\pageref{ex:2Dforce}) already found the net force to be
\[ \vec F_\mathrm{net} = 4.0\unit{N} \ihat + 5.0\unit{N} \jhat \]
which is $F_\mathrm{net}=6.4\unit N$ at $51^\circ$ N of E.
What remains is to find the acceleration.  Since this is a vector, we can either find the components from the components of the net force
\end{minipage}
\hfill
\fbox{\begin{minipage}{1.5in}
\begin{FBD}{10}{15}{15}{40}{object}
\oneup{50}{$5\unit N$}{lightgray}
\oneri{40}{$4\unit N$}{lightgray}
\oneur{38}{48}{4}{5}{$F_\mathrm{net}$}{black}
\end{FBD}
\end{minipage}}
\vspace{2pt}
%
\[ \vec a =\frac{\vec F_\mathrm{net}}{m} = \frac{4.0\unit{N} \ihat + 5.0\unit{N} \jhat}{2.0\unit{kg}}  = 2.0\unitfrac{m}{s^2} \ihat + 2.5\unitfrac{m}{s^2} \jhat \]
or we can use the magnitude of the net force to find the magnitude of the acceleration
\[ a = \frac{6.4\unit N}{2.0\unit{kg}} = 3.2 \unitfrac{m}{s^2} \]
and know that the direction of the acceleration is the same as the acceleration of the net force: $51^\circ$ N of E.

You should notice that you can also use the components of the acceleration to find the magnitude and direction of the acceleration.
\magdir{+2.0\unitfrac{m}{s^2}}{+2.5\unitfrac{m}{s^2}}{a}{\sig{3.2}{0}{N}}{\theta}{\sig{51}{}{^\circ}}{N of E}

\color{black}
\end{minipage}}
\end{example}
%
\begin{example}[p]
\fcolorbox{black}{yellow!10}{\begin{minipage}{4.925in} \setlength{\parsep}{3pt}
\caption{\label{ex:2Dfa2} Moving a box pushed by three forces.}
A $2.0\unit{kg}$ mass is being pushed north with $5.0\unit{N}$ and east with $4.0 \unit{N}$.  What is the acceleration?
\color{blue}

\vspace{9pt}
\begin{minipage}{3.25in}
\autoref{ex:2Dforce2}  (pg.~\pageref{ex:2Dforce2}) already found the net force to be
\[ \vec F_\mathrm{net} = 3.8\unit{N} \ihat + 2.2\unit{N} \jhat \]
which is $F_\mathrm{net}=4.4\unit N$ at $31^\circ$ N of W.
What remains is to find the acceleration.  Since this is a vector, we can either find the components from the components of the net force
\end{minipage}
\hfill
\fbox{\begin{minipage}{1.5in}
\begin{FBD}{10}{15}{15}{40}{object}
%\oneup{50}{$5\unit N$}{black}
%\put(24,56){\vector(-1,2){22.36}}
%\put(0,102){\color{black} \tiny $5\unit N$}
\oneul{22}{48}{-1}{2}{$5 \unit N$}{lightgray}
\onedl{58}{22}{-5}{-2}{$6 \unit N$}{lightgray}
\oneri{40}{$4\unit N$}{lightgray}
\oneul{53}{22}{-5}{3}{$F_\mathrm{net}$}{black}
\end{FBD}
\end{minipage}}
\vspace{2pt}
%
\[ \vec a =\frac{\vec F_\mathrm{net}}{m} = \frac{3.8\unit{N} \ihat + 2.2\unit{N} \jhat}{2.0\unit{kg}}  = 1.9\unitfrac{m}{s^2} \ihat + 1.1\unitfrac{m}{s^2} \jhat \]
or we can use the magnitude of the net force to find the magnitude of the acceleration
\[ a = \frac{4.4\unit N}{2.0\unit{kg}} = 2.2 \unitfrac{m}{s^2} \]
and know that the direction of the acceleration is the same as the acceleration of the net force: $31^\circ$ N of W.

You should notice that you can also use the components of the acceleration to find the magnitude and direction of the acceleration.
\magdir{+1.9\unitfrac{m}{s^2}}{+1.1\unitfrac{m}{s^2}}{a}{\sigfrac{2.2}{}{m}{s^2}}{\theta}{\sig{31}{}{^\circ}}{N of W}

\color{black}
\end{minipage}}
\end{example}
%

\hypertarget{d:m=f/a}{In} \ref{se:netF-a}, we used the forces to find the acceleration.  It is also possible to use the forces to find the mass of an object, as follows:
\begin{sample}
\item\label{se:netF-m} If the forces in \ref{se:netFsub} are applied to an object with unknown mass and produce an acceleration of $3.2\unitfrac{m}{s^2}$, then what is the mass of the object?

    Naively, one might consider $\deq m = \frac{\vec F_\mathrm{net}}{\vec a}$, but it does not make mathematical sense to \hyperlink{d:dividevectors}{divide vectors}.  In this case, you \textit{must} consider the magnitudes of force and acceleration, knowing that their directions are the same. (We are \textit{not} ``cancelling'' the directions.)
    \[ m =\frac{F_\mathrm{net}}{a} = \frac{9.0\unit N}{3.2 \unitfrac{m}{s^2}} = \sigfrac{2.8}{1}{N \cdot s^2}{m} = 2.8\unitfrac{kg \cdot m \cdot s^2}{s^2 \cdot m} \ = \ 2.8 \unit{kg} \]
\end{sample}
\hypertarget{d:usesofF=ma}{Yet another example}\foreshadow{} of using this equation can be seen in many bathrooms.  The scale that people stand on uses a spring (introduced in \autoref{ss:scales} and discussed in detail in \autoref{s:springs}) to adjust the force provided until your acceleration is zero (placing you in equilibrium) and then tells you the force it needed to balance your weight.

It will be easier to visualize these ideas when we introduce the tool of a free-body diagram in \autoref{sss:FBD}.

\subsubsection{Equilibrium}\label{sss:equilibrium}\index{Equilibrium}\mmultireturn{\mmr{\ref{se:equi}}, \mmr{\hyperlink{d:atrestinmotion}{Newton's first law}}, \mmr{\autoref{d:Fgball}}}

This word can be traced back to Latin and Old English with the prefix \textit{equi-} for \textbf{equal} and the root \textit{libra} referring to a \textbf{pair of scales, as in a balance}, such as those depicted in images of the astronomical constellation Libra.  When the scales are equal, they are in equilibrium.  Since the second law asks us to calculate the sum of the forces acting on an object, one of the primary questions is to determine if those forces balance each other.  In \ref{se:netFadd} and \ref{se:netFsub}, the forces are not balanced, the object ``is not in equilibrium'', and it will be accelerated in a particular direction.  In the \ref{se:equi}, the forces are balanced, the object ``is in equilibrium'', and it will \textit{not change} its velocity (in accord with the first law).
\important{An object in equilibrium has $\vec F_\mathrm{net} = 0\unit N$ and $\vec a =0$.}

\subsubsection{How the Laws Work Together}\label{sss:NIItogether}\mautoreturn{sss:NItogether}

When forces act on an object, Newton's second law applies, so we usually start with the second law.  If those forces combine to give a net force of zero, such that the object is in equilibrium, then Newton's first law applies.  If we also care about the person or thing pushing, then the third law also applies.

To better understand how the first and second laws work together, \autoref{irl:NI} (pg.~\pageref{irl:NI}) provides some activities that you can do or consider in order to think about the patterns you can see when you are or aren't pushing on objects.  \autoref{cyoa:NI} (pg.~\pageref{cyoa:NI})
%
\begin{adventure}[bhpt]\fcolorbox{black}{blue!10}{\begin{minipage}{4.925in}\caption{\label{cyoa:NI} Out of gas}
On a long road trip with your friend \studentB\index{\studentB}, your car starts to sputter as it runs out of gas shortly before arriving in a new town.  You see a sign for a gas station in the distance and have to decide what to do.  You and \studentB\ can think of three options.
\begin{CYOA}
\item\label{c:parkandwalk} Pull over, park the car, walk to the gas station, buy a gas can, fill it up, carry it back to the car, and drive on!   If you follow this plan, then read \ref{a:parkandwalk}.
\item\label{c:coastindrive} Leave the car in drive, continue holding the gas-pedal down until there is absolutely no gas, and hope against all hope that you get the car to the gas station so that nobody needs to carry a heavy gas can. If you follow this plan, then read \ref{a:coastindrive}.
\item\label{c:coastinneutral} Speed up to just over the speed limit, put the car in neutral, turn on your blinking hazard-lights, coast as far as you can possibly coast, and hope against all hope that you get the car to the gas station so that nobody needs to carry a heavy gas can. If you follow this plan, then read \ref{a:coastinneutral}.
\end{CYOA}
\autoreturn{sss:NIItogether}
\end{minipage}}
\end{adventure}
%
will help you think through some of the consequences of the first and second law.  When you are ready to solve some problems, you can jump to \autoref{s:NewtonExamples}, but some of those examples will also reference Newton's third law.

\subsection{Translating Newton's Third Law: Action \& Reaction}\label{ss:NIII}\mmultireturn{\mmr{\hyperlink{d:Newtonahead}{how to describe forces}}, \mmr{\autoref{ex:braced}}, \mmr{\autoref{ex:unbraced}}}

%\important{} is not designed to start a new paragraph
\ \vspace{-12pt}
%\begin{quote}
\important{\textbf{Newton's Third Law}\index{Newton!Third Law}: $\overbrace{\mbox{For every force acting}}^{\mbox{\scriptsize ``For every action''}}$ \textit{on} one object \textit{by} an \underline{other object}, there is an equal-in-magnitude reaction-force acting \textit{on} the \underline{other object} in the opposite direction.}
%\end{quote}
This law is often shortened to ``For every action, there is an equal and opposite reaction.''  The statement given above is meant to emphasize several points:
\begin{itemize}\itemsep 0pt
\item These ``actions'' are specifically forces.
\item Forces are an interaction in which the acting force is \underline{on one object} by another and necessitates that there is a reaction force on the other object by the one.  That is to say, an object cannot feel a force without also exerting a force back on the other object.
\end{itemize}
Another way to say this is that all forces come in action/reaction pairs that necessarily have equal magnitude and opposite direction and necessarily act on different objects.
This law is the force-version of\Foreshadow{\protect{\hyperref[s:conservemom]{Conservation of momentum}}}{} the statement of the conservation of momentum, which will be discussed in \autoref{c:momentum}.

\hypertarget{d:NIIIbracing}{Let's take this apart} and connect it to your daily experiences.  Students of physics will often see the terms action and reaction and connect it to the way humans react to the actions of their friends.  However, this implies a causal response that is not true for Newton's forces.  That is to say, this is not a ``revenge law'' whereby if you push on me, then I will choose to push you back.  Instead, it is expressing that forces are intrinsically interactions between a pair of objects.  When you push on me, I am -- independent of my choosing to do so -- necessarily pushing back on you.  But, you might say, ``if that were true, then why am I able to sneak up on you and push you over without falling over myself?''  Well, you can think about how that works by reading \autoref{cyoa:NIII} (pg.~\pageref{cyoa:NIII}). After we introduce the tool of a free-body diagram in the next subsection, you can also explore this idea by comparing \autoref{ex:braced}  (pg.~\pageref{ex:braced}) to \autoref{ex:unbraced}  (pg.~\pageref{ex:unbraced}).

\subsubsection{The Free-Body Diagram (FBD)}\label{sss:FBD}\mmultireturn{\mmr{\ref{se:equi}}, \mmr{\hyperlink{d:usesofF=ma}{uses of $F=ma$}}}\index{Free-Body Diagrams}

In the more interesting situations where there are several forces acting, it can be easy to lose track of what is pushing whom where.  In order to better organize our information and direct our attention, we can make use of \textit{free-body diagrams}.  The basic idea is to make a diagram for each individual object that we care about in a given situation, and \textit{free} from the overall picture.  This allows us to identify the forces acting on a single object (relevant for Newton's second law) and more easily pair them with third-law pairs that act on different objects.

To see how this works, the next simple example will build on the previous simple examples to help us consider not only the (2nd law) forces on the object, but also the (3rd law) forces on the people doing the pushing and pulling.
\begin{sample}
\item\label{se:FBD-AB}\mautoreturn{f:firstFBD} As you revisit \ref{se:netFadd}\dothis{As before, make this object a desk so that we can have \studentA\ and \studentB\ helping you rearrange your room in your residence hall.  (Change the mass!  But, it is still nice to be moving a small mass so that it has a large acc, and the people have small acc.)}, imagine that \studentA\index{\studentA} is exerting $\vec F_1 = +5.0\unit{N}\ihat$ and \studentB\index{\studentB} is exerting $\vec F_2 = +4.0\unit{N}\ihat$.  \ref{se:netF-a} showed how the object moved (because Newton's second law focuses on the object to which the forces are applied).  Newton's third law tells us about the interaction between objects and, from this, we can say the following. In order to better describe the situation, let's assume \studentA\ is to the left of the object, pushing it to the right, and \studentB\ is to the right of the object, pulling it to the right.  See \autoref{f:firstFBD} on page~\pageref{f:firstFBD}.
    \begin{enumerate}
    \item Since \studentA\ exerts a force of $5.0\unit{N}$ to the right $(+\ihat)$ on the object, the third law reminds us that the object exerts a force of $5.0\unit{N}$ to the left  $(-\ihat)$ on \studentA.  Since \studentA\ has a mass of \massA, the second law reminds us that \heA\ is accelerated at the rate of
        \[ \vec a_1 = \frac{-5.0\unit{N} \ihat}{\massA} = -0.0\sigfrac{58}{8}{m}{s^2}\ihat = -5.9\ten{-2}\unitfrac{m}{s^2} \ihat\]
        which is to the left with a small enough value that it is easy for \himA\ to brace against.   Even if \heA\ doesn't brace, if \heA\ starts from rest, \heA\ will only be moving \\
        $v_{1f} = (0\unitfrac{m}{s})+(-0.0\sigfrac{58}{8}{m}{s^2})(1.6\unit s) = -0.0\sigfrac{94}{1}{m}{s}$.
    \item Since \studentB\ exerts a force of $4.0\unit{N}$ to the right on the object, the third law reminds us that the object exerts a force of $4.0\unit{N}$ to the left on \studentB. Since \studentB\ has a mass of \massB, the second law reminds us that \heB\ is accelerated at the rate of
        \[ \vec a_2 = \frac{-4.0\unit{N} \ihat}{\massB} = -0.0\sigfrac{53}{3}{m}{s^2}\ihat = -5.3\ten{-2}\unitfrac{m}{s^2}\]
        which is also to the left with a small enough value that it is easy for \himB\ to brace against.  Even if \heB\ doesn't brace, if \heB\ starts from rest, \heB\ will only be moving \\
        $v_{2f} = (0\unitfrac{m}{s})+(-0.0\sigfrac{53}{3}{m}{s^2})(1.6\unit s) = -0.0\sigfrac{85}{3}{m}{s}$.
    \end{enumerate}
\end{sample}
There are \hypertarget{d:FBD-AB}{a couple of important aspects} to take away from \ref{se:FBD-AB}:, especially as it builds on \ref{se:netFadd} (which told us about the forces on the object) and \ref{se:netF-a} (which told us
\begin{minipage}{3.25in}
about how the object moved).  The earlier examples were relevant to the 2nd law and only affected the object itself.  This information is captured in the middle free-body diagram of \autoref{f:firstFBD} and reproduced here.
\end{minipage}
\hfill
\fbox{\begin{minipage}{1.5in}
\begin{FBD}{10}{15}{15}{10}{object}
\twori{50}{$5\unit N$}{black}{40}{$4\unit N$}{black}
\end{FBD}
\end{minipage}}

As indicated above, the free-body diagram also helps us visualize the third-law (action/reaction) force pairs.  By reproducing the following image\index{Free-Body Diagrams!Images} from \autoref{f:firstFBD} here and adding color, we can see that the {\color{green} green forces} form an action-reaction pair and separately the {\color{blue} blue forces} form
\newline
\begin{minipage}{\textwidth}
\fbox{\begin{minipage}{1.5in}
\begin{FBD}{10}{25}{15}{10}{\studentA}
\onele{50}{$5\unit N$}{green}
\end{FBD}
\end{minipage}}
\hfill
\fbox{\begin{minipage}{1.5in}
\begin{FBD}{10}{15}{15}{10}{object}
\twori{50}{$5\unit N$}{green}{40}{$4\unit N$}{blue}
\end{FBD}
\end{minipage}}
\hfill
\fbox{\begin{minipage}{1.5in}
\begin{FBD}{10}{20}{15}{10}{\studentB}
\onele{40}{$4\unit N$}{blue}
\end{FBD}
\end{minipage}}
\end{minipage}
an action-reaction pair.  None of these three objects are in equilibrium.

If we now \hypertarget{d:equi}{do the same thing}\index{Free-Body Diagrams!Images} with the \ref{se:equi}, but let \studentC\index{\studentC} be the person on the left {\color{green} pushing to the right} and \studentD\index{\studentD} be the person on the right {\color{blue} pushing to the left}, then we see that while the {\color{green} green forces} form an action-reaction pair showing the third-law interaction between \studentC\ and the object and while the {\color{blue} blue forces} form an action-reaction pair showing the third-law interaction between \studentD\ and the object, it
%\newline
\begin{minipage}{\textwidth}
\fbox{\begin{minipage}{1.5in}
\begin{FBD}{10}{15}{15}{10}{\studentC}
\onele{30}{$3\unit N$}{green}
\end{FBD}
\end{minipage}}
\hfill
\fbox{\begin{minipage}{1.5in}
\begin{FBD}{10}{15}{15}{10}{object}
\oneri{30}{$3\unit N$}{green}
\onele{30}{$3\unit N$}{blue}
\end{FBD}
\end{minipage}}
\hfill
\fbox{\begin{minipage}{1.5in}
\begin{FBD}{10}{15}{15}{10}{\studentD}
\oneri{30}{$3\unit N$}{blue}
\end{FBD}
\end{minipage}}
\end{minipage}
is the combination of the {\color{blue} blue} and the {\color{green} green} forces, which only act on the object itself, that coincidentally cancel to leave the object in (second law) equilibrium.  In these images, \studentC\ and \studentD\ are \textit{not} in equilibrium.

To say this more specifically, the forces within one free-body diagram are described by Newton's second law. They do get added together to form the net-force (which is to say that we can add the object's green force to the object's blue force). They are able to cancel each other if they \textit{happen to} be equal in magnitude and opposite in direction. Finally, they will determine how that specific object accelerates.  On the other hand, Newton's third law describes any specific pair of forces that interact between free-body diagrams (each colored pair); they \textit{will necessarily be} equal in magnitude and opposite in direction, but they cannot be canceled because they cannot be added because they are on different objects.

\begin{figure}
\hrule\hrule
\caption{\label{f:firstFBD} A couple of people push a box.}\index{Free-Body Diagrams!Images}
First we will draw a picture of the situation described by \ref{se:FBD-AB}, which builds on \ref{se:netFadd}.  \studentA\index{\studentA} stands to the left of the object and exerts a force (pushes) to the right.  \studentB\index{\studentB} stands to the right of the object and exerts a force (pulls) to the right.  Both of these forces are \textit{on} the object and, by Newton's second law cause it to accelerate (as described in \ref{se:netF-a}).  By Newton's third law, we can learn about the forces on \studentA\ and \studentB.

\noindent
{}\hfill
\begin{minipage}{3.5in}
\begin{picture}(200,100)(-30,-25)
% Dimensions and offset: (width,height)(x offset,y offset)
% Insert picture commands (\line,\circle, etc...) here:
\put(0,0){\line(1,0){200}}
\put(60,2){\line(1,0){60}}
\drawbox{30}{1}{20}{50} %\studentA
\drawbox{50}{25}{18}{5} %\studentA's arms
\drawbox{70}{3}{20}{30} % object
\drawbox{150}{1}{20}{40} %\studentB
\drawbox{134}{25}{16}{5} %\studentB's arms
\put(90,27.5){\oval(2,2)[r]}
\put(91,27.5){\line(1,0){43}}
\put(30,53){\scriptsize \studentA}
\put(70,35){\scriptsize object}
\put(150,43){\scriptsize \studentB}
\put(60,-12){\begin{minipage}{60pt}
\scriptsize The object is on a sheet of ice.
\end{minipage}}
\end{picture}
\end{minipage}
\hfill {}

\noindent
Now we will draw a free-body diagram for each individual.  Notice that each \textit{free-body} diagram is on its own, free from the rest of the picture.  These diagrams will be discussed further in the \protect{\hyperlink{d:FBD-AB}{discussion about \ref*{se:FBD-AB}}}.

\noindent % \textwidth default is 5in for a book
\fbox{\begin{minipage}{1.5in}
\begin{FBD}{10}{25}{15}{10}{\studentA}
\onele{50}{$5\unit N$}{black}
\end{FBD}
\raggedright
Because \studentA\ pushes on the object to the right, \studentA\ feels a force \textit{on} \himA\ \textit{by} the object towards the left.
\end{minipage}}
\hfill
\fbox{\begin{minipage}{1.5in}
\begin{FBD}{10}{15}{15}{10}{object}
\twori{50}{$5\unit N$}{black}{40}{$4\unit N$}{black}
\end{FBD}
\raggedright
The object is in the middle.  Recall from \protect{\ref{se:netFadd}} that the $F_\mathrm{net} = 9.0 \unit N$ on this object.
\end{minipage}}
\hfill
\fbox{\begin{minipage}{1.5in}
\begin{FBD}{10}{20}{15}{10}{\studentB}
\onele{40}{$4\unit N$}{black}
\end{FBD}
\raggedright
Because \studentB\ pulls on the object to the right, \studentB\ feels a force \textit{on} \himB\ \textit{by} the object towards the left.
\end{minipage}}

\noindent
It turns out that this is a little oversimplified.  When we get to \protect{\autoref{s:Fg}} and \protect{\autoref{s:FN}}, we will see that we have to include a downwards force of gravity and an upwards support force.  This will be explained in \protect{\autoref{f:firstFBDupdate}} on page~\protect{\pageref{f:firstFBDupdate}}.
\flushright
\multireturn{\mmr{\autoref{ex:braced}}, \mmr{\autoref{ex:unbraced}}, \mmr{\hyperlink{d:rope.net}{rope-tension}}}
\hrule\hrule
\end{figure}

\section{Examples} \label{s:NewtonExamples}\mautoreturn{sss:NIItogether}

Next, we can consider a simple interactive example that is intended to help you think about how you know a force is acting.
%
\begin{sample}
\item\label{IQ:holdbook} You hold a book a little above your desk.  When you let go, it falls and then hits your desk.
    \begin{enumerate}
    \item While you are holding it, it has no acceleration.  Are there forces acting on it?  \YN{A:hbf}{A:hbnof}
    \item While you are holding it, is it in equilibrium?  \YN{A:true1}{A:false1}
    \item After you let go and while the book falls, it accelerates downwards.  Are there forces acting on it?  \YN{A:falls}{A:falls}
    \item While it is hitting the desk, is it accelerating?  \YN{A:hitY}{A:hitN}
    \item After it has landed and is sitting on the desk, is it in equilibrium? \YN{A:landedY}{A:landedN}
    \item After it has landed and is sitting on the desk, how many forces are acting on it? \THREE{Zero}{One}{Two}{A:zero}{A:one}{A:two}
    \end{enumerate}
\end{sample}
%
Next, we can \hypertarget{d:irlNI}{consider} pushing an object across the floor in \autoref{irl:NI} (pg.~\pageref{irl:NI}) to get a different sense of observations we can make that help us recognize patterns that are due to forces we might not have thought to look for.
%
\begin{reallife}[hp]
\hspace{-.2in}
\fcolorbox{black}{green!10}{\begin{minipage}{5.29in} \center
\caption{\label{irl:NI} Pushing an Object Across the Floor}
\begin{realtable}
\dna{Push a chair across a carpet floor}
    {When you stop pushing, it stops moving.}
    {Does force cause motion? \ref{A:chair1}}
\dna{Push a chair across a tile floor}
    {When you stop pushing, it probably stops moving.}
    {Does force cause motion? \ref{A:chair2}}
\dna{Push a chair \textit{with wheels} across a tile floor, with some strength, then let it go.}
    {What happens when you stop pushing? \ref{A:chair3}}
    {If force causes motion, why does the chair move after you stop touching it?  \ref{A:chair4}}
\dna{Push a chair \textit{with wheels} across a tile floor, change your behavior after you let it go.}
    {Do your actions when you are not touching the chair have \textit{any} impact on the chair? \ref{A:chair5}}
    {Is it possible that there is a ``residual effect'' that you have on the chair after letting it go? \ref{A:chair6}}
\multidna{Newton's First Law says that if you give the chair a velocity, it should keep that velocity.}
\dna{Repeat the first three suggestions}
    {Correlate the interaction-with-the-ground to the motion-after-you-push-and-release}
    {Is there a force that the chair feels after you release it?  \ref{A:chair7}}
\multidna{Newton's Second Law says that a net force will change the velocity.}
\dna{Push a chair gently across the floor}
    {A constant force (balanced by the force of friction) will move at a constant speed}
    {What if there were no friction? \ref{A:chair8}}
\dna{Push a chair forcefully across the floor}
    {A constant force (stronger than the force of friction) will accelerate the object away from your push}
    {Can you list surfaces that are essentially frictionless?}
\end{realtable}
\begin{minipage}{4.925in}
Notice in each case that you are not the only thing interacting with the chair.  The floor is also interacting with the chair.  The floor exerts a \hyperref[s:Ff]{force of friction} on the chair.  So, when you interpret how your force causes the chair to move, you \textit{must} also account for the interaction with the floor in your expectations.  We can minimize the effect of friction, by modifying the floor surface.  If you have ever driven on ice and felt out of control, you might have begun to develop your Newtonian intuition.
\flushright\vspace{-12pt}
\multireturn{\mmr{\autoref{sss:NIItogether}}, \mmr{\hyperlink{d:irlNI}{\autoref*{s:NewtonExamples} reference to \autoref*{irl:NI}}}}
\end{minipage}
\end{minipage}}
\end{reallife}
%
Building on that, it is useful to also consider how human beings behave when they are pushing or getting pushed.  Because people have \textit{intention} in their actions, we subconsciously balance ourselves and we don't always recognize that we are \hypertarget{d:cyoaNIII}{doing it}.  \autoref{cyoa:NIII}  (pg.~\pageref{cyoa:NIII}) provides an interactive storyline that starts to show some of the patterns that can lead to a recognition of how we balance ourselves.
%
\begin{adventure}[bpht]
\fcolorbox{black}{blue!10}{\begin{minipage}{4.925in}
\caption{\label{cyoa:NIII} The Town Bully}
\studentZ\index{\studentZ} is the town bully.  One day, he spies a biology student, \studentC\index{\studentC}, minding \hisC\ own business studying an interesting ecological phenomenon.  At the same time, you are standing across the street chatting with your friend \studentD\index{\studentD}, who happens to be taking a psychology class.  \studentD\ has been quite fascinated lately with watching the way others interact and points out the way \studentZ\ is menacingly approaching the unsuspecting \studentC.  You both predict that \studentZ\ is going to push \studentC\ over.  \studentD\ is mesmerized by the psychological effects and you, having just learned about Newton's laws, are excited to see if this action does indeed produce a reaction.
\begin{CYOA}
\item\label{c:one} If you watch the way \studentC\ is standing before, during, and after \studentZ\ pushes \himC, then read \ref{a:NIIIaction}.
\item\label{c:two} If you watch the way \studentZ\ is standing before, during, and after \heZ\ pushes \studentC, then read \ref{a:NIIIreaction}.
\item\label{c:three} If, on the other hand, you shout a warning to \studentC\ and a criticism to \studentZ, trying to keep the incident from becoming violent, then read \ref{a:NIIIconcern}.
\end{CYOA}
\flushright
\multireturn{\mmr{\hyperlink{d:NIIIbracing}{the discussion of action-reaction forces}}, \mmr{\hyperlink{d:cyoaNIII}{\autoref*{s:NewtonExamples} reference to \autoref*{cyoa:NIII}}}, \mmr{\autoref{ex:braced}}, \mmr{\autoref{ex:unbraced}}}
\end{minipage}}
\end{adventure}
%

\begin{example}[p]
\fcolorbox{black}{yellow!10}{\begin{minipage}{4.925in}\setlength{\parskip}{3pt}
\caption{\label{ex:braced} \studentZ\index{\studentZ} intentionally braces when pushing \studentC\index{\studentC}.}
(To better understand \hyperref[ss:NIII]{Newton's third law}, you should compare this example to \autoref{ex:unbraced}  [pg.~\pageref{ex:unbraced}].)
\begin{quote}
\studentZ\index{\studentZ}, the \hyperref[cyoa:NIII]{town bully} (with $m_Z=\massZ$), decides to vent \hisZ\ frustration on \studentC\index{\studentC}\ for all the times that \studentC\ makes \studentZ\ look bad in class.  While \studentC\ ($m_C=\massC$) has \hisC\ back turned, \studentZ\ walks up, leans in, and shoves \studentC\ with a force of $\vec F_{C,Z} = 215\unit{N}\ihat$.  How does \underline{\studentZ}{} accelerate during this exchange?
\end{quote}
%
%\begin{quote}
%Aside: Newton's second law tells us how this affects \studentC. See \ref{se:netF-a} and homework problem \ref{hmwk:pushbrace}.
%\end{quote}
%\noindent
\textbf{What do we know?}  As usual, it is convenient to start with a picture to help decide on the appropriate coordinate system.
We can also list
\\[2pt]
\begin{minipage}{2.6in}
the information that we know.
We know $m_Z$, which is useful for relating $F_{Z,\mathrm{net}}$ to $a_Z$.
We know $m_C$, which is useful for relating $F_{C,\mathrm{net}}$ to $a_C$.  (This is not asked for, but is asked in homework problem \ref{hmwk:pushbrace}.)
We know $F_{C,Z}$, how hard \studentZ\ pushes on \studentC.
\end{minipage}
\hfill
\begin{minipage}{150pt}
\begin{picture}(150,90)(-30,-25)
% Dimensions and offset: (width,height)(x offset,y offset)
% Insert picture commands (\line,\circle, etc...) here:
\drawbox{-10}{-20}{120}{20}  % Earth
\drawbox{25}{1}{20}{50} %\studentZ
\drawbox{45}{35}{18}{5} %\studentZ's arms
 %\studentZ's legs
    \put(24,19){\line(0,-1){6}}
    \put(24,19){\line(-1,-2){9}}
    \put(15,1){\line(1,0){4}}
    \put(19,1){\line(1,2){6}}
\drawbox{65}{1}{20}{45} %\studentC
\put(25,53){\scriptsize \studentZ}
\put(65,48){\scriptsize \studentC}
\put(40,-15){\scriptsize Earth}
\put(-40,24){\begin{minipage}{58pt}
\color{blue} \scriptsize \studentZ\ braces \himselfZ. \hfill $\searrow$
\end{minipage}}
\end{picture}
\end{minipage}
%\hfill {}
We also know that \studentC\ is not bracing \himselfC\ (because \heC\ ``has \hisC\ back turned'') so he only feels one force, and that \studentZ\ is bracing \himselfZ\ (because the problem states that \heZ\ ``leans in'') so he exerts multiple forces.

\textbf{What do we want to know?}  We want to know about the forces acting on \studentZ, in order to find  $F_{Z,\textrm{net}}$ and therefore $a_Z$.

\textbf{How are these related?}  First, since \studentZ\ exerts a force on \studentC, Newton's third law tells us that \studentZ\ feels a force of \mbox{$F_{Z,C}=-215 \unit{N}\ihat$}.
Second, because \studentZ\ \textit{knew} \heZ\ was going to feel this reaction force, \heZ\ compensates by bracing \himselfZ.  This means \heZ\ chooses to exert a force of $215\unit{N}$ on the Earth in the $-\ihat$ direction, probably by putting one leg behind \himselfZ\ and pushing the ground backwards with \hisZ\ foot.  Newton's third law then tells us that \studentZ\ feels a force of \mbox{$F_{Z,C}=+215 \unit{N}\ihat$} from the ground.

{}\hfill {\footnotesize \autoref*{ex:braced} continued on next page\ldots}
\end{minipage}}
\end{example}
\begin{example}[p]
\fcolorbox{black}{yellow!10}{\begin{minipage}{4.925in}\setlength{\parskip}{3pt}
{\footnotesize \autoref*{ex:braced} continued from previous page\ldots}

\textbf{Free-Body Diagrams:}  We are told of the force on \studentC.  We are told that \studentZ\ braces \himselfZ, which implies the force on the Earth. Newton's third law then helps us recognize the forces on \studentZ.  (Recall the \hyperlink{d:interaction}{``on-by'' notation}.)

\noindent % \textwidth default is 5in for a book
\fbox{\begin{minipage}{2.25in}
\begin{FBD}{10}{25}{15}{10}{\studentZ}
\onele{50}{$F_{Z,C}=215\unit N$}{black}
\oneri{50}{$F_{Z,E}=215\unit N$}{blue}
\end{FBD}
\vspace{-10pt}
\raggedright
\studentZ\ is pushed by \studentC\ to the left.  \studentZ\ is pushed to the right by the Earth.
\end{minipage}}
\hfill
\fbox{\begin{minipage}{2.25in}
\begin{FBD}{10}{23}{15}{10}{\studentC}
\oneri{50}{$F_{C,Z}=215\unit N$}{black}
\end{FBD}
\vspace{-10pt}
\raggedright
\studentC\ feels \studentZ\ push to the right.
\end{minipage}}
% \\
\fbox{\begin{minipage}{4.75in}
\begin{FBD}{60}{10}{15}{10}{Earth}
\onele{50}{$F_{E,Z}=215\unit N$}{blue}
\end{FBD}
\vspace{-10pt}
\raggedright
Earth feels a force by \studentZ\ to the left.
\end{minipage}}

\textbf{Concepts to Consider:}  Newton's third law guarantees that the action-reaction force pairs, such as $F_{Z,C}$ and $F_{C,Z}$ or $F_{Z,E}$ and $F_{E,Z}$, are equal and opposite.  There is no such guarantee on $F_{Z,C}$ and $F_{Z,E}$.  These are equal because \studentZ\ chose to make $F_{C,Z}$ and $F_{E,Z}$ equal.  \HeZ\ pushed on the two others in equal amounts so that the reaction forces that act on \himZ\ will balance for Newton's \textit{second} law so that \hisZ\ acceleration would be zero.

\textbf{Solution to the example:}  After using Newton's third law to find the forces on \studentZ, we can use Newton's second law to find \hisZ\ acceleration:
\[ a_Z = \frac{F_{Z,\mathrm{net}}}{m_Z} = \frac{\left[ \vec F_{Z,C} + \vec F_{Z,E} \right]}{\massZ} = \frac{\left[ \left( -215\unit N \ihat \right) + \left( +215\unit N \ihat \right) \right]}{\massZ} = 0 \unitfrac{m}{s^2}  \]

%\begin{quote}
\textbf{Aside:} This example only considers the left-right forces that act in order to make a point about our intuition regarding forces we intend to apply.  Please consider how \protect{\autoref{f:firstFBDupdate}} updates \autoref{f:firstFBD} to make yourself aware of the other forces that are acting here, but are being ignored.
%\end{quote}

\linkreturn[action-reaction]{d:NIIIbracing}
\end{minipage}}
\end{example}

\begin{example}[p]
\fcolorbox{black}{yellow!10}{\begin{minipage}{4.925in}\setlength{\parskip}{3pt}
\caption{\label{ex:unbraced} \studentD\index{\studentD} does not brace \himselfD\ when pushing \studentC\index{\studentC}.}
(To better understand \hyperref[ss:NIII]{Newton's third law}, you should compare this example to \autoref{ex:braced}  [pg.~\pageref{ex:braced}].)
\begin{quote}
In the lab room one day, while waiting for the instructor, \studentD\index{\studentD} (who has a mass of $m_D=\massD$) decides to try a physics experiment to test Newton's third law.  \HeD\ politely asks \hisD\ lab partner, \studentC\index{\studentC} ($m_C=\massC$), to turn \hisC\ back while \heD\ squares his feet underneath \himselfD\ and pushes with a force of $\vec F_{C,D} = 215\unit{N}\ihat$.  Despite the experience of \autoref{ex:braced} (as told in \autoref{cyoa:NIII}  [pg.~\pageref{cyoa:NIII}]), \studentC\ reluctantly agrees.  How does \underline{\studentD}{} accelerate during this exchange?
\end{quote}
%
%\begin{quote}
%Aside: Newton's second law tells us how this affects \studentC. See \ref{se:netF-a} and homework problem \ref{hmwk:pushbrace}.
%\end{quote}
%\noindent
\textbf{What do we know?}  As usual, it is convenient to start with a picture to help decide on the appropriate coordinate system.
We can also list
\\[2pt]
\begin{minipage}{2.6in}
the information that we know.
We know $m_D$, which is useful for relating $F_{D,\mathrm{net}}$ to $a_D$.
We know $m_C$, which is useful for relating $F_{C,\mathrm{net}}$ to $a_C$.  (This is not asked for, but is asked in homework problem \ref{hmwk:pushbrace}.)
We know $F_{C,D}$, how hard \studentD\ pushes on \studentC.
\end{minipage}
\hfill
\begin{minipage}{150pt}
\begin{picture}(150,90)(-30,-25)
% Dimensions and offset: (width,height)(x offset,y offset)
% Insert picture commands (\line,\circle, etc...) here:
\drawbox{-10}{-20}{120}{20}  % Earth
\drawbox{25}{1}{20}{40} %\studentD
\drawbox{45}{25}{18}{5} %\studentD's arms
\drawbox{65}{1}{20}{45} %\studentC
\put(25,43){\scriptsize \studentD}
\put(65,48){\scriptsize \studentC}
\put(40,-15){\scriptsize Earth}
\put(-40,24){\begin{minipage}{58pt}
\color{blue} \raggedright \scriptsize \studentD\ does not brace \himselfD. \\\hfill $\searrow$
\end{minipage}}
\end{picture}
\end{minipage}
%\hfill {}
\\[2pt]
We also know that neither person is bracing for the push. So, both \studentC\ and \studentD\ each only feel one force.

\textbf{What do we want to know?}  We want to know about the forces acting on \studentD, in order to find  $F_{D,\textrm{net}}$ and therefore $a_D$.

\textbf{How are these related?}  First, since \studentD\ exerts a force on \studentC, Newton's third law tells us that \studentD\ feels a force of \mbox{$F_{D,C}=-215 \unit{N}\ihat$}.
Second, unlike \studentZ\ in \autoref{ex:braced}  (pg.~\pageref{ex:braced}), \studentD\ chooses not to exert a force on the Earth in the $-\ihat$ direction.

\textbf{Free-Body Diagrams:}  We again draw free-body diagrams:

\noindent % \textwidth default is 5in for a book
\fbox{\begin{minipage}{2.25in}
\begin{FBD}{10}{20}{15}{10}{\studentD}
\onele{50}{$F_{D,C}=215\unit N$}{black}
\end{FBD}
\vspace{-10pt}
\raggedright
\studentD\ is pushed by \studentC\ to the left.
\end{minipage}}
\hfill
\fbox{\begin{minipage}{2.25in}
\begin{FBD}{10}{23}{15}{10}{\studentC}
\oneri{50}{$F_{C,D}=215\unit N$}{black}
\end{FBD}
\vspace{-10pt}
\raggedright
\studentC\ feels \studentD\ push to the right.
\end{minipage}}

{}\hfill {\footnotesize\autoref*{ex:unbraced} continued on next page\ldots}
\end{minipage}}
\end{example}
\begin{example}[p]
\fcolorbox{black}{yellow!10}{\begin{minipage}{4.925in}\setlength{\parskip}{3pt}
{\footnotesize \autoref*{ex:unbraced} continued from previous page\ldots}

\textbf{Concepts to Consider:}  Newton's third law guarantees that the action-reaction force pairs, $F_{D,C}$ and $F_{C,D}$, are equal and opposite.  Because these forces are not on the same person, we cannot add these forces.  Newton's second law will then indicate how each person accelerates.

\textbf{Solution to the example:}  After using Newton's third law to find the forces on \studentD, we can use Newton's second law to find \hisD\ acceleration:
\[ a_D = \frac{F_{D,\mathrm{net}}}{m_D} = \frac{\left[ \vec F_{D,C} \right]}{\massD} = \frac{\left[ \left( -215\unit N \ihat \right) \right]}{\massD} = -\sigfrac{2.68}{75}{m}{s^2} \ihat \]

%\begin{quote}
\textbf{Aside:} This example only considers the left-right forces that act in order to make a point about our intuition regarding forces we intend to apply.  Please consider how \protect{\autoref{f:firstFBDupdate}} updates \autoref{f:firstFBD} to make yourself aware of the other forces that are acting here, but are being ignored.
%\end{quote}
\flushright
\multireturn{\mmr{\hyperlink{d:NIIIbracing}{the discussion of action-reaction forces}}, \mmr{\autoref{ex:braced}}}
\end{minipage}}
\end{example}

\section{Summary and Homework}

\subsection{Summary of Concepts and Equations}

This chapter introduced the way physicists describe forces.  The concept of force encodes how objects interact.
After reading this chapter, you should be comfortable responding to the following questions or comments.
Unlike the other links in this book, if you follow the links in this summary section, there is no link to return to this page.  (This is on purpose to encourage you to answer these points without following these links.)
\begin{itemize}
\item State Newton's Laws. \hyperlink{sum:Newton'sLaws}{(Answer)}
\item How is the unit of Newton related to the fundamental units of the SI system?  \hyperref[sss:unit-N]{(Answer)}
\item How do you know when a system is in equilibrium? \hyperref[sss:equilibrium]{(Answer)}
\item You should know how to draw a free-body diagram.  \hyperref[f:firstFBD]{(Example)}
\end{itemize}

\subsection*{Conceptual Questions}\dothis{Add more conceptual questions}
%\vspace{-24pt}
\begin{enumerate}
\item In order to climb a tree, you reach up and grab a branch and pull.  Most people refer to this as ``pulling yourself up.'' In terms of Newton's third law, describe what is happening in more technical terms.
\item Some cars have a ``cruise-control'' feature that keeps your speed constant as you drive down the highway.  (a) If you are driving due north with the cruise-control on, are you in equilibrium?  (b) If, instead, you have the cruise-control set while you are following the road around a gradual curve of the road as it follows the shore of a lake, then are you in equilibrium?  (c) In both cases, how can you tell if you are in equilibrium?
\end{enumerate}
\subsection*{Problems}\dothis{Add more variety of problems.}
%\vspace{-24pt}
\begin{enumerate}
 \item\label{hmwk:pushbrace} If \studentZ, with $m_Z=\massZ$, braces \himselfZ\ (so that he does not accelerate) and pushes \studentC\ ($m_C=\massC$) with a force of $\vec F_{C,Z} = 215\unit{N}\ihat$, find the following:
\begin{enumerate}
    \item What is the acceleration of \studentC?  \answer{\mbox{$\deq\vec a_C = \frac{215 \unit{N}\ihat}{\massC} = \sigfrac{2.38}{9}{m}{s^2} \ihat$.}}
    \item What net force does \studentZ\ feel? \answer{$F_{Z,\mathrm{net}}=0\unit N$}
    \item If \studentZ\ braces \himselfZ\ against the Earth, then what must that bracing force be?  \answer{$\vec F_{E,Z} = -215\unit{N}\ihat$}
    \item What are the individual forces that \studentZ\ feels? \answer{$F_{Z,C}=-215\unit N \ihat$ and $F_{Z,E}=215\unit N \ihat$}
    \item What is the acceleration of the Earth?  \answer{\mbox{$\deq\vec a_E = \frac{-215 \unit{N}\ihat}{5.97\ten{24}\unit{kg}} = -\sigfrac{3.60}{1\ten{-23}}{m}{s^2} \ihat$.}}
    \item Which of Newton's laws allows you to answer each of these questions?
\end{enumerate}
\item If you apply a force of $4.65\unit N$ to a mass of $2.18\unit{kg}$, then how much will it accelerate?
\item How much force must you apply to cause a mass of $80.0\unit{kg}$ to accelerate at $a=0.795\unitfrac{m}{s^2}$?
\item You arrive home to find a box that came in the mail.  You find that you have to exert $54.3\unit N$ to cause it to accelerate $a=1.25\unitfrac{m}{s^2}$.  (a) What is its mass?  (b) Is that a heavy box or a light box?  (c) Is it likely that this box would fit in a mailbox?
\item Your $2538 \unit{kg}$ car has run out of gas.  So you ask your friend, \studentB{} who has a mass of $\massB$, to put it in neutral, sit inside, and steer while you push.  If you apply enough force to cause a net forward force of magnitude $37.5\unit N$, how much time will it take for the car to move faster than you can walk?  Assume your walking speed is $3.0\unitfrac{mi}{hr}$.  How far will the car have travelled in that time?
\item Find the components of the net force on a large crate if three forces are applied: $\vec F_1 = -3.0\unit N \ihat + 2.5 \unit N \jhat$, $\vec F_2 = -6.25\unit N \jhat$, and $\vec F_3 = 4.5\unit N \ihat + 1.63 \unit{N} \jhat$.
\item Find the components of the net force on a large crate if three forces are applied: $F_1 = 3.61 \unit N $ at $71.6^\circ$ north of east, $F_2 = 4.61\unit N$ due west, and $F_3 = 8.13\unit N$ at $21.8^\circ$ south of east.
\item Find the magnitude and direction of the net force on a large crate if three forces are applied: $\vec F_1 = 4.25\unit N \ihat - 4.66 \unit N \jhat$, $\vec F_2 = -2.65\unit N \jhat$, and $\vec F_3 = -5.4\unit N \ihat + 2.93 \unit{N} \jhat$.
\item Find the magnitude and direction of the net force on a large crate if three forces are applied: $F_1 = 2.65 \unit N $ at $26.6^\circ$ north of west, $F_2 = 2.22\unit N$ at $56.31^\circ$ south of west, and $F_3 = 7.12\unit N$ at $28.4^\circ$ north of east.
\end{enumerate}



\chapter{The Many Types of Force}\label{c:forcetype}\mlinkreturn[subscript notation of forces]{d:interaction}

\section{Gravity at the Surface of the Earth}\label{s:Fg}\mmultireturn{\mmr{\hyperlink{d:accgrav}{freefall}}, \mmr{\autoref{f:firstFBD}}}\new{v2.2}{Adding detail}

Perhaps the force that is the most obvious to humanity is the one that helps us fall when we stumble: the gravitational force\index{Gravity!Surface of Earth}.  This is one of the fundamental forces discussed in \autoref{s:fundamental}.  In addition, the details about how the planets, moon, and the sun experience this force will be discussed in \autoref{c:gravity}.  For now, we can consider how this interaction manifests itself on our daily lives.  In this section, we will start with how objects move when the gravitational force is the only force acting.  Subsections~\ref{ss:weightmass} and~\ref{ss:equivmm} will clarify some subtleties and then we'll jump into the examples in \autoref{ss:local.mg}.

We can investigate what happens when the gravitational force is the only force acting on an object by holding it in the air and dropping it\index{Freefall}.  One of the complications during such an experiment was discussed in \autoref{ss:airresistance}.  If we drop a sheet of paper, there is air resistance in addition to the gravitational force.  For this section, I will assume that the mass-to-surface-area ratio is large enough that we can effectively\Touchstone{Recall \protect{\hyperref[s:effective2]{effective theories}}.}{} ignore the air resistance.

Since objects fall faster than humans are used to paying attention to, the \hypertarget{d:Fgrav}{patterns} are difficult to see.  The green box of \autoref{irl:freefall} (on page~\pageref{irl:freefall}) shows you how you can pay close attention to the patterns that result from observing falling objects.
You should go do those experiments before reading further.  Go ahead.  I'll wait.

You did do them, right?  You're not just reading ahead?  Really?  OK.  Doing that experiment will help you see (1) that everything falls at the same rate and (2) that objects accelerate as they fall\phantomsection\label{d:Fgball}.  This first point is a bit less intuitive and will be discussed further in \autoref{ss:equivmm}.  This second point should be exactly what you expect, when you consider \hyperref[ss:NII]{Newton's second Law}: If there is only one force (the gravitational force), then the object cannot be in \hyperref[sss:equilibrium]{equilibrium} and it must be accelerating.  (You should notice that this is the language of \hyperref[st:F=ma]{the story of Newton's second law}.)

In order to evaluate this further, let's consider a specific object, like a baseball.  Our baseball has a mass of $m_b = 0.145\unit{kg}$.  If the only force acting \textit{on} the ball is the gravitational force \textit{by} the Earth, then the net force is the gravitational force: $\vec F_\mathrm{net} = \vec F_{bEg}$\Touchstone{\hyperlink{d:interaction}{the on-by notation}}.  Here the subscripts are $b$ (because the force is on the \underline{b}all), $E$ (because the force is exerted by the \underline{E}arth), and $g$ (because it is a \underline{g}ravitational force).  Since the acceleration is due to the gravitational force, I will use either $a_g$ (usually when the object is in \hyperref[ss:freefall]{freefall} and therefore accelerating at this rate) or $g$ (usually when the object is not actually accelerating at that rate).  With this notation, Newton's second law becomes:  \[ \vec F_{bEg} = m_b \vec a_g \]
At this point, we know the mass, but we don't know the force or the acceleration.  However, we have conveniently already done the experiment (recall \autoref{ex:freefall}) that will tell us the acceleration is $a_g = 9.81\unitfrac{m}{s^2}$ downwards.  (Recall that ``downwards'' is the direction of the vector, which can be expressed as $-\jhat$.)  If we know the mass and the acceleration, then we can compute the force.
\begin{sample}
\item\label{se:weightball} If a baseball with mass $m_b = 0.145\unit{kg}$ is dropped (allowed to \hyperref[ss:freefall]{fall freely}) so that it accelerates at $a_g = 9.81\unitfrac{m}{s^2}$ downwards, then while it falls it feels the gravitational force:
    \[ \vec F_g = m \vec g = (0.145\unit{kg}) [-(9.81\unitfrac{m}{s^2})\,\jhat] = -\sig{1.42}{24}{N} \jhat = -1.42 \unit N \jhat \]
\end{sample}
This is the force of the gravitational force on the baseball.  Although we computed the force while the ball was falling, the gravitational force does not magically vanish when the ball is sitting on the floor.  So, we can say that (as long as the ball is close to the surface of the Earth, as noted in \autoref{c:gravity}) the force always has this value.  Rather than continuing to say ``the force of gravity'' we call this force the weight\index{Weight}.
\important{The weight of an object is computed as its mass times the acceleration due to gravity, even when the object is not actually accelerating at that rate:  $\mathbf{F_g \equiv mg}$.}

\subsection{Weight versus Mass}\label{ss:weightmass}\mmultireturn{\mmr{\autoref{ss:convertunits}}, \mmr{\autoref{s:sigfig}}}\index{Weight}\new{v2.2}{Added detail.  Moved the previous version to \protect{\autoref{s:sigfig}} to smooth the transition to \protect{\autoref{ss:equivmm}}.}

Since all objects have the same acceleration due to gravity at the surface of the Earth, the weight of an object and the mass of an object are very closely correlated, but they are not the same quantity.  This tends to cause some confusion when the discussion is not explicitly technical.  Recall the discussion about \hyperref[s:precision]{being precise in our language}.  One complication for people in the United States is that there are two definitions of the pound; one is a unit of mass\footnote{There are also multiple versions of the pound-mass.  You can find these explained on the internet, but most of these are considered obsolete.  The one I will use is the ``avoirdupois-pound'', which is defined in the NIST publication
% found in https://en.wikipedia.org/wiki/Pound_(mass)
\protect{\href{https://www.nist.gov/sites/default/files/documents/2017/04/28/AppC-12-hb44-final.pdf}{Handbook 44}}, page C-19, as exactly $453.592 37\unit{g}$.}
and the other is a unit of force.  Since the pound-force\footnote{There is also a unit of force called the kilogram-force.} is defined as the standard unit of mass times the standard unit for the acceleration due to gravity,
% https://en.wikipedia.org/wiki/Pound_(force)
as discussed in \autoref{s:SI-MKS}\dothis{Update \protect{\autoref{s:SI-MKS}} with this information.}, the conversion directly from pound-force to Newtons will \underline{not} match the longer, but more appropriate, conversion from pound-mass to kilogram that gets multiplied by the local acceleration due to gravity (as opposed to the standard $g$) into Newtons.  It may also be useful to review the comments about unit-conversion in the section on \hyperref[s:sigfig]{significant digits}\index{Significant Digits}.

In the discussion about \hyperref[s:precision]{being precise in our language}, we distinguished ``massive'' (the amount) from ``voluminous'' (the size).  Now that we understand \hyperref[ss:NII]{Newton's second law}, we can distinguish ``massive''
%(an amount of material)
from ``weighty.'' %
(a strength needed to lift).
The concept that goes with
\important{mass is the amount of material,}
whereas, the concept that goes with
\important{weight is how strongly the gravitational force pulls on the object.}
Having mass affects both the inertia (ease of moving) and the weight (force of gravity).
Having weight expresses the gravitational force due to whichever large object (moon, planet, sun, etc.) you happen to be on or near.  Noticing that the \hyperref[s:SI-MKS]{SI-unit} is different for different types of quantities, such as a kilogram (a \hyperref[ss:units]{fundamental unit}) for mass and a Newton (a \hyperref[ss:units]{derived unit}) for weight, may help you remember that these are different kinds of quantities.

The interesting aspect of this relationship is that while having more mass makes an object harder to move (the same force produces less acceleration for more massive objects), when objects fall under the influence of the gravitational force, they accelerate at the \textit{same} rate.  This reveals that the gravitational force must be stronger for more massive objects \textit{by the exact amount} needed to compensate for that larger mass.  This is called the equivalence principle and is discussed in \autoref{ss:equivmm}.


\subsection{Calculating the weight}\label{ss:local.mg}\new{v2.2}{renamed this section and added detail}

When calculating the forces acting on a person or an object, we will often need to account for the force of gravity, while other forces may also be at work.  As mentioned above, the weight is found by multiplying the mass times the local acceleration due to gravity, even if the object is not actually accelerating at that rate.  Chapter~\ref{c:gravity} will clarify why it is true\footnote{The short answer is that the altitude (distance from the surface of the Earth) and local geology affect the strength of the gravitational field.  Since the Earth is slightly oblate (bulges at the equator), the altitude at different latitudes corresponds to a different distance from the center of the Earth.  In addition, while the spin of the Earth does not affect the strength of the gravitational field, it does affect how objects accelerate. The \protect{\href{http://www2.csr.utexas.edu/grace/gallery/animations/ggm01/ggm01_gif-200.html}{GRACE project}} has measured the variations across the globe.}, but for now please note that the acceleration due to gravity is (1) different according to where we are and also (2) the same for all objects at that location.\dothis{Gather values of $g$ at various locations.  Wiki has a list, but need to find the source.  Wolfram has numbers, but they seem to be calculated off a formula, not measurements.  \protect{\href{http://www.physics.montana.edu/demonstrations/video/1_mechanics/demos/localgravitychart.html}{U Montana}} has values but no reference.
\protect{\href{http://www.calpoly.edu/~gthorncr/ME302/documents/AccuracyofGravity.pdf}{Glen Thorncroft at Cal Poly}} has a formula and lists the level of each effect.}\index{Acceleration!Gravity}\index{Gravity!Acceleration}\done{Add a table of measured values of $g$ at various locations.  Compute the weight of a specific person at various locations.}

\hypertarget{d:weightmass}{Because} of the peculiarities in the definition of pound (\autoref{ss:weightmass}) it will be useful to build some intuition about masses in terms of kilograms and Newtons.  \autoref{t:weightmass} lists the mass of some common objects and, using the standard value for $g$, their corresponding weights.
%
\begin{table}[bhtp]
\hrule\hrule
\begin{center}
\caption[Comparison of masses and weights of common objects]{\label{t:weightmass} The list of objects is intended to give a sense of scale so that the reader can better estimate the value of the mass of an object.  You might notice that (except for the apple) each of these is between 4 and 4.5 times heavier than the previous object.  Note that these are rough estimates; for example, while the author weighs about $200\unit{lbs}$ this is not typical, nor average.
%\linkreturn[weight and mass]{d:weightmass}
% reference weight of an apple:  \url{http://www.applejournal.com/ref.htm}
}
\begin{tabular}{lrrr}
Object & pounds & mass (kg) & weight (N) \\ \hline
apple & 0.33 & 0.15 & 1.5 \\
lean, healthy cat & 10 & 4.6 & 45 \\
medium-sized dog & 44 & 20 & 196 \\
human & 200 & 91 & 890 \\
horse & 1000 & 362 & $3.56\ten{3}$ \\
large pick-up truck & 4000 & $1.81\ten{3}$ & $1.78\ten{4}$
\end{tabular}
\end{center}
\hrule\hrule
\end{table}
%
\hyperref[c:weightmass]{Conceptual Problem \ref{c:weightmass}} asks you to estimate the mass of some other common objects.  \hyperref[c:massweight]{Conceptual Problem \ref{c:massweight}} asks you to think of common objects with a specified mass.

Now let's do some calculations\ldots
\begin{sample}
\item \studentA\index{\studentA} notices that \heA\ needs to exert $F=1.5\unit{N}$ to support the apple listed in \autoref{t:weightmass}. \HeA\ then drops it  and notices its acceleration of $9.81\unitfrac{m}{s^2}$.  \HeA\ computes the mass to be
    \[ m = \frac{F_g}{a_a} \ = \ \frac{1.5\unit{N}}{9.81\unitfrac{m}{s^2}} \ = \ \frac{1.5\unitfrac{kg \cdot m}{s^2}}{9.81\unitfrac{m}{s^2}} \ = \ 0.\sig{15}{3}{kg} \]
    (If you know the weight, you can compute the mass, even if the mass is not actually in freefall.)
\item\label{se:weightA} \studentA\index{\studentA}\new{v2.3}{modified and supplemented}, who knows \hisA\ own mass ($\massA$), then imagines\mmultireturn{\mmr{\autoref{f:firstFBDupdate}}, \mmr{\autoref{f:firstFBDangle}}} dropping \himselfA\ (!) from a (small) height.  While \heA\ falls, \heA\ recognizes the gravitational force on \himA, which is computed to be
    \[ \vec F_g = m \vec g = (\massA) [-(9.81\unitfrac{m}{s^2})\,\jhat] = -\sig{833}{.85}{N} \jhat = -834 \unit N \jhat \]
    Since \heA\ is in freefall and there is only one force is acting on \himA, the net force is easy to compute:  $\vec F_\mathrm{net} = -834 \unit N$.
    However, if you know the mass something, you can compute the weight even if that object is not in freefall.
    You should repeat this calculation for the mass in \ref{se:netF-a}.  (\ref{A:netF-a})
\end{sample}
You should note that
\important{$F_\mathrm{net} \ (=ma)$ is always related to the actual acceleration of the object, \\ $F_g\ (=mg)$ is always related to the local acceleration due to gravity.}
You should also note that
\important{the actual acceleration is only equal to the local acceleration due to gravity if the object is in freefall.}
\begin{sample}
\item\label{se:FNB} If\mmultireturn{\mmr{\autoref{f:firstFBDupdate}}, \mmr{\autoref{f:firstFBDangle}}} \studentB\index{\studentB} is not falling, but rather standing safely on the floor, then the gravitational force is still acting.  It can be computed as
    \[ \vec F_g = m \vec g = (\massB) [-(9.81\unitfrac{m}{s^2})\,\jhat] = -\sig{735}{.75}{N} \jhat = -736 \unit N \jhat \]
    However, since we can see that \hisB\ acceleration is zero, the $\vec F_\mathrm{net}$ \textit{must be zero}.  The only way that can happen, though is if there is another force acting upwards on \studentB.  What could possibly be pushing up on \himB?  \ref{A:floor}.  Whatever it is pushing up on \himB, it is supplying a support force, which can be calculated since $\vec F_\mathrm{net} = \vec F_g + \vec F_\mathrm{support}$ and we can solve for
    \[ \vec F_\mathrm{support} = \vec F_\mathrm{net} - \vec F_g = m\left(0\unitfrac{m}{s^2}\right) - \left[ -(\sig{735}{.8}{N}) \jhat\right] = +736\unit N \jhat \]
    Because it is in the direction opposite to $\vec F_g$, it is upwards $(+\jhat)$.

    Can you identify \textit{why} the support force is equal in magnitude and opposite in direction to the gravitational force?
    \TWO{Newton's second law}{Newton's third law}{A:second}{A:third}
\end{sample}
As was mentioned earlier, the value of the acceleration due to gravity also varies across the surface, although this is less than about a percent or so (see~\autoref{t:gworld}).
Nonetheless, this means that your weight can change even when your mass remains the same.
\begin{sample}
\item\label{se:gworld} While talking to your friend \studentB\index{\studentB}, you learn that \hisB\ parents, \studentE\index{\studentE} and \studentF\index{\studentF}, grew up in Norway, visited Puerto Rico, and climbed Mount Everest before settling in the United States.  Using \autoref{t:gworld}, compute \studentE's weight are each location, assuming \hisE\ mass is \massE.
\begin{enumerate}
\item[Norway] $F_g = mg = (\massE)(9.825\unitfrac{m}{s^2}) \ = \ \sig{933}{.4}{N}$
\item[Puerto Rico] $F_g = mg = (\massE)(9.782\unitfrac{m}{s^2}) \ = \ \sig{929}{.3}{N}$
\item[Mount Everest] $F_g = mg = (\massE)(9.763\unitfrac{m}{s^2}) \ = \ \sig{927}{.5}{N}$
\end{enumerate}
\end{sample}
%
Because the variation is small, throughout this text when we are considering situations ``at the surface of the Earth'', we will assume that
\important{the acceleration due to gravity is $9.81\unitfrac{m}{s^2}$ to three significant figures.}
%
\begin{table}[bhtp]
\hrule\hrule
\begin{center}
\caption[Comparison of $g$ at a few places on Earth]{\label{t:gworld} Comparison of $g$ at a few places on Earth.  {\color{gray} [While both the latitude-longitude and the local value of $g$ were found using the
\href{https://www.wolframalpha.com/}{WolframAlpha$^R$ computational knowledge engine},
these $g$ values do not necessarily correspond to these coordinates.  The $g$ values are based on a theoretical model of the Earth.]}
You should look for a pattern as the latitude increases.  (\ref{A:gworld})
You might notice the values for  Mount Everest and Denver; Can you explain any peculiarity?  (\ref{A:gpeaks})
\return{se:gworld}
}
\begin{tabular}{lccr}
Location & latitude & longitude & local $g (\!\!\unitfrac{m}{s^2})$ \\ \hline
San Juan, Puerto Rico & $18^\circ 26' 24'' \unit{N}$  & $66^\circ 7' 48'' \unit W$ & $9.782 \unitfrac{m}{s^2}$ \\
Brownsville, TX & $26^\circ 1' 6'' \unit{N}$  & $97^\circ 27' 14'' \unit W$ & $9.788 \unitfrac{m}{s^2}$ \\
Mount Everest & $27^\circ 59' 17'' \unit{N}$  & $86^\circ 55' 31'' \unit E$ & $9.763\unitfrac{m}{s^2}$ \\
Cincinnati, OH & $39^\circ 8' 24'' \unit{N}$  & $84^\circ 30' 23'' \unit W$ & $9.801\unitfrac{m}{s^2}$ \\
Denver, CO & $39^\circ 45' 43'' \unit{N}$  & $104^\circ 52' 50'' \unit W$ & $9.798\unitfrac{m}{s^2}$ \\
Paris, France & $48^\circ 51' 36'' \unit{N}$  & $2^\circ 20' 24'' \unit E$ & $9.813\unitfrac{m}{s^2}$ \\
Oslo, Norway & $59^\circ 54' 36'' \unit{N}$  & $10^\circ 45' \phantom{24''} \unit E$ & $9.825\unitfrac{m}{s^2}$ \\
Anchorage, AK & $61^\circ 10' 39'' \unit{N}$  & $149^\circ 16' 28'' \unit E$ & $9.826\unitfrac{m}{s^2}$
\end{tabular}
\end{center}
\hrule\hrule
\end{table}
%



\section{Fundamental Forces}\label{s:fundamental}\index{Force!Fundamental}\new{v2.1}{Started the section on fundamental interactions.  Link ahead, rather than detailling here.}

The previous section describes our (macroscopic) experience of the gravitational interaction when standing on the surface of the Earth.  This is essentially the same across the surface, but does change with altitude and the difference can be measured on mountain tops and in caves.  In fact, one can use the differences from one location to another to predict where we might find a a pocket of oil.\new{v2.2}{Filled out the detail.  Changed the approach.}

In later \hypertarget{d:fundamental}{sections}, we will consider this and other interactions that depend on the physical properties, such as mass and charge.  All particles with the property of mass (which we will start to call gravitational charge) will interact according to the gravitational force; however, this description is better described by the mathematics in \autoref{c:gravity}.  All particles with the property of electrical charge will interact according to the electrical force.  The basic theory will be discussed in \autoref{c:electric}.  A more complicated version that incorporates quantum mechanics is called quantum electrodynamics (QED) and this will be touched on in \autoref{ss:QED}.  Particles like protons and neutrons (hadrons) are actually made up of other particles (quarks) that are held together by an interaction that is sometimes called the strong nuclear force (\autoref{ss:strong}) and is described by the theory of quantum chromodynamics (QCD); this will be touched on in \autoref{ss:QCD}.  Finally, in \autoref{ss:weak} another fundamental force, called the weak nuclear force, will be discussed.

For the most part, these theories describe the interaction between microscopic particles, so we will not discuss them in detail here.  However, the gravitational interaction is exception in a variety of ways.  In particular, the gravitational interaction does affect macroscopic objects.  These fundamental forces have a particular description that allows us to pretend (recall \hyperref[s:effective2]{effective theories}) that they are action-at-a-distance interactions.  All other forces (introduced next) will require physical contact in order to exert the force.

\section{Normal Force}\label{s:FN}\mmultireturn{\mmr{\autoref{f:firstFBD}}, \mmr{\ref{A:floor}}, \mmr{\autoref{s:FT}}}\new{v2.2}{Added detail}\index{Force!Normal}

The word ``normal'' \href{http://etymonline.com/index.php?term=normal}{originates}\index{Normal} with the idea of conformity to the pattern.  While in everyday life this the typical state of being, the origins actually refer to a carpenter's square, which put corners into a right angle.  In math and physics, the word is used to mean perpendicular.  In the context of forces,
\important{the normal force is the force that a surface exerts to keep objects from passing through them.  The direction of this force is always in the outward direction, normal (perpendicular) to the surface.}

Let's consider some specific situations\ldots\inlife{} In \ref{se:FNB}\dothis{DO we need to repeat the example here? no?}, \studentB\ felt the downwards gravitational force even while \heB\ was standing on the ground.  We noticed that \heB\ was not falling (and so not accelerating).  Colloquially, we say that the ground is supporting \studentB.  This support force is keeping \studentB\ from passing through the floor; this is a normal force.  The normal force from the floor is acting upwards, which is normal (perpendicular) to the surface of the floor.  \autoref{f:firstFBDupdate} updates the free-body diagrams of \autoref{f:firstFBD} to show how the gravitational and normal forces impact that calculation.
%
\begin{figure}
\hrule\hrule
\caption{\label{f:firstFBDupdate} An updated version of \protect{\autoref{f:firstFBD}}, people pushing a box.}\index{Free-Body Diagrams!Images}
Again, we can start by drawing a picture of the situation.  The description is the same as it was for \autoref{f:firstFBD}.  In addition to those forces, each of the three bodies has a downwards gravitational force.  This analogous to the calculation in \ref{se:weightA}, which was only for \studentA\index{\studentA}; but you can calculate the weight for the mass in \ref{se:netF-a} and \studentB\index{\studentB}'s weight was computed in \ref{se:FNB}.  In addition to the downward gravitational force (the weight), Newton's second law and the fact that nothing is accelerating up or down together tells us that

\noindent
\begin{minipage}[b]{150pt}
there must also be a normal force on each body.  This is analogous to the calculation in \ref{se:FNB}, which was only for \studentB; but you can deduce it for the object and for \studentA.
\end{minipage}
\hfill\begin{minipage}[b]{220pt}
\begin{picture}(220,85)(-10,-25)
\put(0,0){\line(1,0){200}}
\put(60,2){\line(1,0){60}}
\drawbox{30}{1}{20}{50} %\studentA
\drawbox{50}{25}{18}{5} %\studentA's arms
\put(30,53){\scriptsize \studentA}
\drawbox{70}{3}{20}{30} % object
\put(70,35){\scriptsize object}
\drawbox{150}{1}{20}{40} %\studentB
\drawbox{134}{25}{16}{5} %\studentB's arms
\put(150,43){\scriptsize \studentB}
\put(90,27.5){\oval(2,2)[r]}
\put(91,27.5){\line(1,0){43}}
\put(60,-12){\begin{minipage}{60pt}
\scriptsize The object is on a sheet of ice.
\end{minipage}}
\end{picture}
\end{minipage}


Now, as in \autoref{f:firstFBD}, we will draw a free-body diagram for each individual separately.  However, this time we will use \ref{se:weightA} and \ref{se:FNB} to include the gravitational force (the weight) and the normal force.

\noindent % \textwidth default is 5in for a book
\fbox{\begin{minipage}{1.5in}
\begin{FBD}{10}{25}{15}{80}{\studentA}
\onele{20}{$5\unit N$}{black}
\onedo{100}{$834\unit N$}{black}
\oneup{100}{$834\unit N$}{black}
\end{FBD}
\raggedright
Even with the vertical forces, \studentA\ still has a $\vec F_\mathrm{net} = -5.0\unit N \ihat$.
\end{minipage}}
\hfill
\fbox{\begin{minipage}{1.5in}
\begin{FBD}{10}{15}{15}{25}{object}
\twori{20}{$5\unit N$}{black}{16}{$4\unit N$}{black}
\onedo{35}{$20\unit N$}{black}
\oneup{35}{$20\unit N$}{black}
\end{FBD}
\raggedright
Even with the vertical forces, the object still has a $\vec F_\mathrm{net} = +9.0\unit N \ihat$.
\end{minipage}}
\hfill
\fbox{\begin{minipage}{1.5in}
\begin{FBD}{10}{20}{15}{75}{\studentB}
\onele{16}{$4\unit N$}{black}
\onedo{88}{$736\unit N$}{black}
\oneup{88}{$736\unit N$}{black}
\end{FBD}
\raggedright
Even with the vertical forces, \studentB\ still has a $\vec F_\mathrm{net} = -4.0\unit N \ihat$.
\end{minipage}}
\flushright
\multireturn{\mmr{\autoref{ex:braced}}, \mmr{\autoref{ex:unbraced}}, \mmr{\autoref{s:FN}}, \mmr{\hyperlink{d:rope.net}{rope-tension}}, \mmr{\autoref{f:firstFBDangle}}}
\hrule\hrule
\end{figure}

Let's consider some other specific situations\ldots If you decide to lean against a wall, the wall will provide a normal force that pushes horizontally, keeping you from moving through the wall.\new{v2.3}{Answered \protect{\ref{se:ladderN}} and its related problems.}
%
\begin{sample}
\item\label{se:ladderN} \studentC\ leans a $22.7\unit{kg}$ ladder against a wall at an angle of $75.5^\circ$, consistent with \protect{\href{https://www.osha.gov/}{OSHA}} standard \protect{\href{https://www.osha.gov/pls/oshaweb/owadisp.show_document?p_table=standards&p_id=10839}{1926.1053(a)(1)(ii)}}, so that about $\txtfrac{1}{8}$ of the weight is leaning into the wall.  \begin{enumerate}
\item Find the magnitude and direction of the normal force exerted by the wall on the ladder.
\item Find the magnitude and direction of the normal force exerted by the wall on the ladder.
    \end{enumerate}

Since the weight is $F_g = mg = (22.7\unit{kg})(9.81\unitfrac{m}{s^2}) = \sig{222}{.69}{N}$, an eighth of this is $\sig{27.8}{36}{N}$.  This force is pressing into the wall (horizontally, which I will choose as the $+\ihat$ direction).  By \hyperref[ss:NIII]{Newton's third law} if the ladder presses into the wall with $\sig{27.8}{36}{N}$ in the $+\ihat$ direction (this is also a normal force), then the wall pushes the ladder with a normal force of $\sig{27.8}{36}{N}$ in the $-\ihat$ direction.  \textbf{Notice that this is normal (perpendicular) to the surface of the wall.}

Since the full weight of the ladder, $F_g = \sig{222}{.69}{N}$, is still pressing downwards $(-\jhat)$ into the floor (as a normal force), \hyperref[ss:NIII]{Newton's third law} says that the floor pushes the ladder upwards $(+\jhat)$ with a normal force of $\sig{222}{.69}{N}$.  \textbf{Notice that this is normal (perpendicular) to the surface of the floor.}

\autoref{ex:ladder2} goes into the full details of how one calculates the necessary values.
\end{sample}
%
If you lose control of your car and run into a tree, the tree also provides a normal force pushing the car away from the tree; this normal force will stop the car.
%
\begin{sample}
\item\label{se:tree} \studentZ\index{\studentZ} is driving home after a late night of studying at the library.  \HeZ\ is kind of tired and drifts off during the drive.  While traveling $\vec v_i = 13.0\unitfrac ms \ihat$, \studentZ\ runs into a tree, bringing \hisZ\ car $(m=2.1\ten{3}\unit{kg})$ to a halt in $\Delta t = 0.243\unit s$.  (\studentZ\ remains unharmed because \heZ\ was awake enough to wear \hisZ\ seatbelt and
\noindent
\begin{minipage}[b]{240pt}
had a car with a functioning airbag.  Whew.)  Find the normal force by the tree on the car. \\

To be clear about what is happening, I will draw the picture. In order to find the force, we will first need to find the acceleration.
\end{minipage}
\hfill\begin{minipage}[b]{130pt}
\begin{picture}(120,80)(-10,-5)
\put(0,0){\line(1,0){100}}
\drawbox{70}{1}{20}{50} %\studentA
\drawbox{10}{5}{30}{20} % object
\put(15,3){\circle{5}}
\put(35,3){\circle{5}}
\put(0,40){\scriptsize $v=13.0\unitfrac ms$}
\put (10,35){\vector(1,0){30}}
\put(72,33){\scriptsize Tree}
\put(15,15){\scriptsize car}
\end{picture}
\end{minipage}
\[ \vec a = \frac{\vec v_f-\vec v_i}{\Delta t} = \frac{(0\unitfrac ms)-(13.0\unitfrac ms \ihat)}{0.243\unit s} = -\sigfrac{53.4}{9}{m}{s^2}\ihat \]
That the acceleration is in the direction opposite the velocity corresponds to the object slowing down.  Now we can find the \underline{net force} from Newton's second law:
\[ \vec F_\mathrm{net} = m \vec a = (2.1\ten{3}\unit{kg})(-\sigfrac{53.4}{9}{m}{s^2}\ihat) = -\sig{1.1}{2\ten{5}}{N} \ihat \]
There are three forces acting on the car, as can be seen in the free-body diagrams of \autoref{f:firstFBDupdate}.  So, we can draw a free-body diagram here as well.  The gravitational force on the car is
\[ \vec F_g = m\vec g = (2.1\ten{3}\unit{kg})(-9.81\unitfrac{m}{s^2}\jhat) = -\sig{2.0}{6\ten{4}}{N}\jhat \]
Since this in the vertical direction and the net force is in the horizontal direction, there must be an upwards normal force from the ground
\begin{minipage}[b]{240pt}
$ F_{N,\mathrm{ground}} = \sig{2.0}{6\ten{4}}{N}\jhat$.
\textbf{This is normal (perpendicular) to the surface of the ground.} \\

The remaining horizontal force is the normal force from the tree,
$ \deq F_{N,\mathrm{tree}} = -\sig{1.1}{2\ten{5}}{N} \ihat$.
\end{minipage}
\hfill\begin{minipage}[b]{130pt}
\fbox{\begin{minipage}[b]{100pt}
\begin{FBD}{15}{10}{15}{25}{car}
\onele{40}{$F_{N,\mathrm{tree}}$}{rgb:red,0;green,2;blue,1}
\onedo{30}{$F_g$}{rgb:red,0;green,2;blue,1}
\oneup{30}{$F_{N,\mathrm{ground}}$}{rgb:red,0;green,2;blue,1}
\end{FBD}
\end{minipage}}
\end{minipage}
\textbf{This is normal (perpendicular) to the surface of the tree.}
\end{sample}%
(Notice that \ref{se:tree} also shows why it is not always necessary to consider the vertical forces when we ``know'' that they cancel.)  If you throw a ball at the ceiling, the ceiling will provide a normal force downwards, keeping the ball from moving through the surface.
%
\begin{sample}
\item\label{se:ceiling} \studentC\index{\studentC} recalls that one time \heC\ got bored one day in physics class (what?!?) and tossed a baseball ($m_b = 0.145\unit{kg}$) at the ceiling\ldots a little too hard \ldots as recounted in \autoref{ex:ceiling}.  The acceleration during that collision with the ceiling was $\vec a = - \sigfrac{28.0}{9}{m}{s^2} \jhat$.  Find the normal force by the ceiling on the ball.

There are five stages to the motion: (a) throwing, (b) falling up, (c) hitting the ceiling, (d) falling down, and (e) catching show the forces involved. \\
\color{lightgray}
\fbox{\begin{minipage}[b]{55pt}
\begin{picture}(50,100)(0,0)
\put(25,25){\circle{10}}
\put(25,26){\vector(0,1){25}}
\put(25,24){\vector(0,-1){15}}
\put(28,35){$F_\mathrm{throw}$}
\put(28,10){$F_g$}
\end{picture}
\centering{(a) throwing}
\end{minipage}}
\hfill
\fbox{\begin{minipage}[b]{55pt}
\begin{picture}(50,100)(0,0)
\put(25,50){\circle{10}}
\put(25,50){\vector(0,-1){15}}
\put(28,35){$F_g$}
\end{picture}
\centering{(b) falling up}
\end{minipage}}
\hfill
\color{rgb:red,0;green,2;blue,1}
\fbox{\begin{minipage}[b]{55pt}
\begin{picture}(50,100)(0,0)
\put(25,95){\circle{10}}
\put(26,95){\vector(0,-1){25}}
\put(24,95){\vector(0,-1){15}}
\put(28,75){$F_N$}
\put(10,75){$F_g$}
\end{picture}
\centering{(c) \\ hitting}
\end{minipage}}
\hfill
\color{lightgray}
\fbox{\begin{minipage}[b]{55pt}
\begin{picture}(50,100)(0,0)
\put(25,50){\circle{10}}
\put(25,50){\vector(0,-1){15}}
\put(28,35){$F_g$}
\end{picture}
\centering{(d) falling down}
\end{minipage}}
\hfill
\fbox{\begin{minipage}[b]{55pt}
\begin{picture}(50,100)(0,0)
\put(25,25){\circle{10}}
\put(25,26){\vector(0,1){25}}
\put(25,24){\vector(0,-1){15}}
\put(28,35){$F_\mathrm{catch}$}
\put(28,10){$F_g$}
\end{picture}
\centering{(e) catching}
\end{minipage}}
\color{rgb:red,0;green,2;blue,1}
\\
In this particular problem, we are only concerned with step (c) when the ball hits the ceiling, because that is the only part where the normal force acts. \ref{se:throw-up} will describe what happens during steps (a) and (e).

During step (c), we have the actual acceleration, which tells us about the net force.  We will also need to know the weight of the baseball, because gravity is still acting during the collision.
\begin{eqnarray*}
\vec F_N + \vec F_g & = &  \vec F_\mathrm{net} \ = \ m \vec a \\
\vec F_N + m \vec g & = &  m \vec a \\
\vec F_N  & = &  m \vec a - m \vec g \\
\vec F_N  & = &  \left[ (0.145\unit{kg})(-\sigfrac{28.0}{9}{m}{s^2}\jhat) \right] - \left[ (0.145\unit{kg})(-9.81\unitfrac{m}{s^2}\jhat) \right] \\
\vec F_N  & = &  \left[ -\sig{4.07}{3}{N} \jhat \right] - \left[  - \sig{1.42}{2}{N} \jhat \right] \ = \ -\sig{2.65}{1}{N} \jhat
\end{eqnarray*}
You can see that the downward normal force $(\sig{2.65}{1}{N})$ combined with the downward gravitational force $(\sig{1.42}{2}{N})$ together create the downward net force $(\sig{4.07}{3}{N})$.
\end{sample}
%
If you make a \hypertarget{d:bank-shot}{``bank shot''} with either a basketball off the backboard or a pool ball\footnote{Resources for \protect{\href{http://wpapool.com/equipment-specifications/\#Balls-and-Ball-Rack}{specifications}} and
\protect{\href{http://c.ymcdn.com/sites/bca-pool.com/resource/resmgr/imported/BCAEquipmentSpecifications_2008.pdf}{a PDF version}}.
These provide:
    weight ($5.5\unit{oz}=0.\sig{155}{92}{kg}$ and $6.0\unit{oz}=0.\sig{170}{097}{kg}$ cue),
    diameter ($2.250\pm 0.005\unit{in}=\sig{5.71}{5}{cm}\pm 0.0127\unit{cm}$),
    rail height ($63.5 \%$ of the ball height, $= \sig{3.62}{9}{cm}$),  and
    dimension limits on the cue stick:
        $L_\mathrm{min}=40.00\unit{in}=1.016\unit{m}$,
        $m_\mathrm{max} = 25.0\unit{oz} = 0.\sig{708}{75}{kg}$, and
        tip-width $w_\mathrm{max}=1.4\unit{cm}$.
You might also consider the information and calculations at
\protect{\href{http://billiards.colostate.edu/technical_proofs/index.html}{Dr.~Dave's site}},
which gives
    slow ($1\unit{mph}$), medium ($3\unit{mph}$), and fast ($7\unit{mph}$);
    coefficient of friction ball-to-ball $\mu=0.06$; and
    ball-ball collision times as $250\unit{\mu s}$-$300\unit{\mu s}$.
}
off the bumper, then the surface provides a normal force that is perpendicular to the surface, in this case redirecting the ball rather than stopping it.  Unfortunately, the actual mechanism is somewhat more complicated than we are ready for; these are considered a little bit in the \autoref{irl:poolcushion} (pg.~\pageref{irl:poolcushion})\dothis{\protect{\autoref{irl:poolcushion}} should be moved to a section that has more about friction and angular momentum.  It is too complex for this section.}.
%
\begin{reallife}[bthp]
\hspace{-.2in}
\fcolorbox{black}{green!10}{\begin{minipage}{5.29in} \center
\caption{\label{irl:poolcushion}\index{Pool!Real Life} Pool balls and bumpers / cushions.}
\begin{minipage}{4.925in}
\studentD\index{\studentD} is relaxing with the local physics club, playing pool.  \HeD\ shoots a bank-shot and the ricochet reminds all of you about the normal force from the bumper on the ball.
\end{minipage}
\begin{realtable}
\dna{Find a billiards table}
    {Notice the felt, the bumpers (cushion), and the dimensions of the table}
    {Does the ball roll as far on felt as it does on hardwood?  \ref{A:pool.roll} \\
     How soft is the bumper? \ref{A:pool.bumper}}
\dna{Find a set of pool balls}
    {Compare the solid-colored balls, the striped balls, and the cue ball}
    {Are there differences in size of weight? \ref{A:noncue}}
\dna{Hit the cue-ball off of a bumper in the manner intended for
\protect{\href{http://c.ymcdn.com/sites/bca-pool.com/resource/resmgr/imported/BCAEquipmentSpecifications_2008.pdf}{testing cushions}}.}
    {Compare the angle it leaves the bumper (reflected angle) match the angle at which it came in (incident angle)}
    {Does the spin of the ball matter? \ref{A:pool.spin}}
\dna{Place a pool ball against the bumper and ricochet the cue ball off the pool ball instead of the bumper itself.}
    {Notice how the pool ball reacts. \ref{A:pool.later}}
    {Why does the pool ball jump off the bumper? \\
     Does the pool ball move along the wall? \\
     Where did you hit the pool ball?}
\end{realtable}
\begin{minipage}{4.925in}
Billiard tables have a lot of interesting physics, which can help us see a wide variety of physics, for example:
\hyperref[irl:poolfriction]{friction}, \hyperref[irl:poolelastic]{elastic versus inelastic collisions}, \hyperref[irl:poolrotmot]{rotational motion}, and \hyperref[irl:poolangmom]{angular momentum}.
\end{minipage}

\flushright
\linkreturn[pool]{d:bank-shot}
\end{minipage}}
\end{reallife}
%

\subsection{Bathroom Scales Measure the Normal Force}\label{ss:scales}\mlinkreturn[uses of $F=ma$]{d:usesofF=ma}

To get a good sense of what how the normal force works, it helps to consider the way a bathroom scale works.  Consider the concepts presented in the \autoref{irl:scale} (pg.~\pageref{irl:scale}).
%
\begin{reallife}[bthp]
\hspace{-.2in}
\fcolorbox{black}{green!10}{\begin{minipage}{5.29in} \center
\caption{\label{irl:scale}\index{Force!Normal} Playing with a scale.}
\begin{minipage}{4.925in}
While speaking to your friend, \studentB\index{\studentB} about \hisB\ recent accomplishment of losing $45\unit{N}$, you mention that your scale always gives a different number than the one in the doctor's office.  You suggest \heB\ gets on your scale to verify the calibration.  \studentB\ currently has a mass of $\massB$.
\end{minipage}
\begin{realtable}
\dna{Try to lose $45\unit{N}$.}
    {Compare this to your weight}
    {Is this a lot of weight to lose? \ref{A:weight.loss}}
\dna{Place your toe on the scale while \studentB\ weighs \himselfB}
    {This increases the value the scale reads}
    {Does \studentB\ weigh more? \ref{A:weight.gain}}
\dna{With your hands, press down on \studentB's shoulders while \heB\ stands on the scale}
    {Control the value read by the scale.  Increase the reading by $20\unit{N}$, $30\unit{N}$, etc.}
    {Does \studentB's weight change?  \ref{A:weight.gain} Are you adding weight to the scale? \ref{A:scale.increase}}
\dna{Have \studentB\ lean on a nearby table or counter while \heB\ stands on the scale}
    {Control the value read by the scale.  Decrease the reading by $20\unit{N}$, $30\unit{N}$, etc.}
    {Does \studentB's weight change?  \ref{A:weight.gain} }
\dna{Hold the scale against the wall and press into it.}
    {Control the value read by the scale.  Increase the reading by $20\unit{N}$, $30\unit{N}$, etc.}
    {What is the scale measuring? \ref{A:scale.measure}}
\dna{Imagine placing a scale on a ramp that can be laid flat or raised to any angle up to a vertical (making it a wall)}
    {Imagine standing on the scale on the ramp while it is lifted from horizontal (like a floor) to vertical (like a wall)}
    {Does the scale always read the same value while it is raised to different angles? \ref{A:scale.ramp}}
\end{realtable}
%\begin{minipage}{4.925in}
%If you can control the value read by the scale while at the same time not changing your actual mass, does the scale literally measure the weight of the object on the scale?  \ref{A:scale.measure}
%\end{minipage}

\flushright
\autoreturn{ss:scales}
\end{minipage}}
\end{reallife}
%
Some digital scales are inconvenient for understanding how they work because they don't display the value until it has come to something close to equilibrium.  If you have access to an analog scale, then you can watch the value change as it settles down and it might be easier to build your intuition.

As you consider the values that you read on the scale, you should consider what happens if you jump off of or land upon a scale.  \textbf{Note that actually doing this can decalibrate your scale, if not break it entirely.  Scales are not meant to be handled this way.} While you are jumping from your scale, it must provide not only the force necessary to support your weight, but also the upwards force require to accelerate you upwards.  While you are landing on the scale, it musty provide not only the force necessary to support your weight, but also the upwards force necessary to decelerate you.

Bathroom scales use leverage (i.e., \hyperref[s:leverarm]{torque}) and a \hyperref[s:springs]{spring}-system to balance the force pressing into them.  The mechanism can be seen at \href{http://home.howstuffworks.com/inside-scale.htm}{How Stuff Works}.

\section{Tension}\label{s:FT}\mlinkreturn[$F=ma$]{d:f=ma}\index{Force!Tension}

Where the \hyperref[s:FN]{normal force} is appropriate for pushing against surfaces,
\important{tension is the pulling force that is transmitted through materials \\ such as cable, chain, or rope.}
Tension is closely related to the compression force experienced by support beams.  One can simplistically think of tension as pulling\dothis{add a link to (and the section itself) to a section on the modulus and stress/strain.}{} and compression as pushing\dothis{Maybe add an IRL about a house settling and the compression forces.  Loading a pick-up truck and watching the bed sag as weight is added.  Hammock as an example of adding weight and increasing the tension force.}{} on the intermediate object that transmits force between the objects at either end.\footnote{It doesn't usually make sense to talk about the compression of a rope or chain.}
When engineers design the skeleton of bridges and buildings, one of the primary considerations is the tension and compression of the steel beams.  You can build your intuition by considering the \autoref{irl:tension} (pg.~\pageref{irl:tension}).\dothis{Still need to update the \protect{\autoref{irl:tension}}.}
%
\begin{reallife}[bthp]
\hspace{-.2in}
\fcolorbox{black}{green!10}{\begin{minipage}{5.29in} \center
\caption{\label{irl:tension}\index{Force!Tension} Pull my finger.}
\begin{minipage}{4.925in}
We talk about tension and stress in our daily lives.  This is an analogy to the physical version of tension, stress, and strain.  While \protect{\href{http://etymonline.com/index.php?allowed_in_frame=0&search=stress}{stress}} and \protect{\href{http://etymonline.com/index.php?search=strain&searchmode=&p=0&allowed_in_frame=0}{strain}} come from the the concept of tightening, tension \protect{\href{http://etymonline.com/index.php?allowed_in_frame=0&search=tension}{comes from}} the concept of stretching.
\end{minipage}
\begin{realtable}
\dna{Sit on a swing }
    {Notice the tightness of the support ropes/chains}
    {How tight are the supports when the swing is empty? When a small child is in the swing? When a full-sized adult is in the swing? \ref{A:swing.tension}}
\dna{Install a fan or light fixture that hangs from the ceiling}
    {You don't want the fan to be supported by the electrical wires, but rather by the metal shaft}
    {How is the fan supported? \ref{A:fan.tension}}
\dna{Pull on a doorknob}
    {Imagine replacing the knobs (inside and outside) with large knots on a rope that runs through the hole the doorknob used to occupy.}
    {What if the doorknob were replaced with a rope, knotted on either side of the door? [Answer]}
\dna{Take a dog for a walk on a leash}
    {Try to pay attention to Newton's second and third law when the dog changes its level of enthusiasm for pulling on the leash.}
    {If the dog pulls very hard on the leash and you balance that force without allowing the dog to move away from you, then describe the way the force connects you to the dog. [Answer]}
\end{realtable}
%\begin{minipage}{4.925in}
%If you can control the value read by the scale while at the same time not changing your actual mass, does the scale literally measure the weight of the object on the scale?  \ref{A:scale.measure}
%\end{minipage}

\flushright
\autoreturn{s:FT}
\end{minipage}}
\end{reallife}
%

When considering the tension in the rope, the context is generally that the rope is connecting two objects that are trying to pull on each other.  It is convenient to recognize that each object only ``sees'' the rope, not the object at the far side.  This can be seen in a couple of contexts.\new{v2.4}{Modified}

We will start with the \hyperref[s:effective2]{simplistic approximation} of ropes that only transmit the force.  As your understanding improves, we will add some examples where the tension in the rope also affects the rope itself.  In that more complicated situation, the tension will change across the rope\dothis{maybe add links}{} and the rope may stretch\dothis{maybe add links}{}.  Since ropes and cables are twisted strands while chains are links, ropes and cables can also introduce a \hyperref[s:torsion]{torsion}\foreshadow{} that tend not to occur in chains.

\subsection{Tension as a Support Force}\label{ss:tension.support}

Ropes and chains (and beams) can use tension to support (from above) dangling objects.
\begin{sample}
\item\label{se:purse} \studentD\index{\studentD} hangs her purse $(m=1.36\unit{kg})$ on a hook.  How much tension is in the shoulder strap to keep it from falling?

The strap connects the hook to the purse.  We can consider the interaction between the hook and the strap or between the purse and the strap.  We will consider the latter since we don't know anything about the hook.  Considering the forces on the purse, we know that there is a downwards gravitational force of $\deq F_g = (1.36\unit{kg})(9.81\unitfrac{m}{s^2}) = \sig{13.3}{4}{N}$ and that the net force must zero (because the purse is not accelerating). So, the strap must provide an upwards (tension) force.
\begin{eqnarray*}
\vec F_T + \vec F_g & = & m \cancelto{0}{\vec a} \\
\vec F_T + (-\sig{13.3}{4}{N} \jhat) & = & 0\unit N \\
\vec F_T & = & +\sig{13.3}{4}{N} \jhat
\end{eqnarray*}
This is the upwards force that the strap applies to the purse; however, the tension strap is doing two jobs: It is pulling up on the purse (as indicated above) \textbf{and} it is pulling down on the hook.
\end{sample}
The important thing to take away from \ref{se:purse} is not that we can compute the value (although that is, of course, a useful skill), but rather that
\important{the tension is conveying the force between the two objects.}  In the same way that the \hyperref[s:FN]{normal force} on a scale does not measure your weight, but rather the amount you press into the scale, the tension passes force on to the attached object.  The hook doesn't feel the weight of the purse, but does feel the tension required to support the purse.

In \hyperref[sss:multiple.mass]{an upcoming section}, we will consider what happens when multiple masses are hung from the rope.

\subsubsection{How Physicists Use the Words (Vocabulary)}

You can probably think of several examples of objects dangling: a purse on a hook, a flag on a pole, a shop sign attached to a post, a pendulum,\dothis{Add an image of an immovable surface to that section}{} \\
\begin{minipage}{4.25in}
a swing set, etc.  Since these are all similar in some ways (although different in other ways), \textbf{we can treat all of them as a mass at the end of a rope}.  Typically, because we do not want to deal with the complications that come from sagging supports, we will use the \hyperref[s:effective2]{approximation} of an ``\textbf{immovable support}.''  This will be indicated by hashing the surface.
\end{minipage}
\hfill
\begin{minipage}{30pt}
\begin{picture}(35,80)
\put(0,70){\line(1,0){25}}
\multiput(5,70)(5,0){4}{\line(1,1){5}}
\put(12.5,70){\line(0,-1){50}}
\put(7.5,20){\line(1,0){10}}
\put(7.5,20){\line(0,-1){10}}
\put(17.5,10){\line(-1,0){10}}
\put(17.5,10){\line(0,1){10}}
\end{picture}
\end{minipage}

\subsection{Tension as Dragging Force}\label{ss:tension.drag}

We can also consider the \hypertarget{d:rope.net}{tension} in a rope used to drag an object across the floor.  You may recall that in \autoref{f:firstFBD} (and the updated version, \autoref{f:firstFBDupdate}) \studentB\index{\studentB} pulled a box across a sheet of ice.  It is possible that  \studentB\ was grabbing the object itself, but it is more likely that \heB\ was pulling on a rope that was attached to the object.  In that case, the tension in the rope was $4.0 \unit N$.  This tension is what pulled \studentB\ leftwards \textbf{and} what pulled the object rightwards.

We can further update this by considering the case where \studentB\ pulls the rope up at an angle.  In that case, some of the tension is used to drag the box and some is used to reduce the normal force.  In \autoref{f:firstFBDangle}, we will have \studentA\ continue to push with $5.0\unit{N}$ horizontally and have \studentB\ pull with $4.0\unit{N}$ at a $14^\circ$ angle above the horizontal.
%
\begin{figure}
\hrule\hrule
\caption{\label{f:firstFBDangle} An updated version of \protect{\autoref{f:firstFBDupdate}}, people pushing a box.}\index{Free-Body Diagrams!Images}
Again, we can start by drawing a picture of the situation.  The description is the same as it was for \autoref{f:firstFBDupdate} except that \studentB\ pulls at a slight

\noindent
\begin{minipage}[b]{150pt}
angle upwards.  We will again need the gravitational force for \studentA\index{\studentA} (\ref{se:weightA}) and \studentB\index{\studentB} (\ref{se:FNB}).  As before, since nothing is accelerating up or down together, there must also be a normal force on each body.
\end{minipage}
\hfill\begin{minipage}[b]{220pt}
\begin{picture}(220,85)(-10,-25)
\put(0,0){\line(1,0){200}}
\put(60,2){\line(1,0){60}}
\drawbox{30}{1}{20}{50} %\studentA
\drawbox{50}{25}{18}{5} %\studentA's arms
\drawbox{70}{3}{20}{30} % object
\drawbox{150}{1}{20}{40} %\studentB
\drawbox{134}{25}{16}{5} %\studentB's arms
\put(90,16.5){\oval(2,2)[r]}
\put(91,16.5){\line(4,1){44}}
\put(30,53){\scriptsize \studentA}
\put(70,35){\scriptsize object}
\put(150,43){\scriptsize \studentB}
\put(60,-12){\begin{minipage}{60pt}
\scriptsize The object is on a sheet of ice.
\end{minipage}}
\end{picture}
\end{minipage}


Now, as in \autoref{f:firstFBDupdate}, we will draw a free-body diagram for each individual separately.  However, this time we will put the tension of the rope at the appropriate angle.  We will need to do a small calculation to find the value of the normal forces.

\noindent % \textwidth default is 5in for a book
\fbox{\begin{minipage}{1.5in}
\begin{FBD}{10}{25}{15}{80}{\studentA}
\onele{20}{$5\unit N$}{black}
\onedo{100}{$834\unit N$}{black}
\oneup{100}{$834\unit N$}{black}
\end{FBD}
\raggedright
The forces on \studentA\ have not changed.
\end{minipage}}
\hfill
\begin{minipage}{1.5in}
\fbox{\begin{minipage}{1.5in}
\begin{FBD}{10}{15}{15}{25}{object}
\oneri{20}{}{black}\put(43,30){\color{black}\tiny  $5\unit N$}
\onedo{35}{$20\unit N$}{black}
\oneup{35}{$F_N$}{black}
\put(26,41){\color{black} \vector(4,1){20}}
\put(43,46){\color{black} \tiny $4 \unit{N}$}
\end{FBD}
\raggedright
The forces on the object \textit{have} changed.
\end{minipage}}
\begin{picture}(100,60)
\put(0,10){\line(4,1){80}}
\put(0,10){\line(1,0){80}}
\put(80,10){\line(0,1){20}}
\put(15,10){\oval(5,8)[rt]}
\put(25,11){\tiny $14^\circ$}
\put(35,28){\tiny $F_T=4.0 \unit{N}$}
\put(82,30){\tiny $F_{Ty}=$}
\put(82,20){\tiny $= F_T\,\sin 14^\circ$}
\put(82,10){\tiny $ = 0.\sig{96}{8}{N}$}
\put(5,0){\tiny $F_{Tx}=F_T \, \cos 14^\circ = \sig{3.8}{8}{N}$}
\end{picture}
\end{minipage}
\hfill
\fbox{\begin{minipage}{1.5in}
\begin{FBD}{10}{20}{15}{75}{\studentB}
%\onele{16}{$4\unit N$}{black}
\onedo{88}{$736\unit N$}{black}
\oneup{88}{$F_N$}{black}
\put(24,94){\color{black} \vector(-4,-1){20}}
\put(0,92){\color{black} \tiny $4\unit{N}$}
\end{FBD}
\raggedright
The forces on \studentB\ \textit{have} changed.
\end{minipage}}

\noindent
\textbf{For the object}: Since the y-component of the net force is zero, we can find the normal force to be $F_N = -[(-20\unit{N})+(+0.\sig{96}{8}{N})] = 19\unit{N}$.
The x-component of the net force is $F_{\mathrm{net},x}=(5.0\unit N)+(\sig{3.8}{8}{N}) = \sig{8.8}{8}{N}$.

\textbf{For \studentB}: Since the y-component of the net force is zero, we can find the normal force to be $F_N = -[(-736\unit{N})+(-0.\sig{96}{8}{N}) = 737\unit{N}$.
The x-component of the net force is $F_{\mathrm{net},x}=(-\sig{3.8}{8}{N})$.

\flushright
\linkreturn[rope-tension]{d:rope.net}
\hrule\hrule
\end{figure}
%
You should note that since the tension on the object is pulling up, helping the normal force, this allows the normal force (what a scale would read) to be a little smaller.
You should also note that since the tension on \studentB\ is pulling down, counter-acting the normal force, this requires the normal force (what a scale would read) to be a little larger.

\subsection{Pulleys}

While the flexibility of ropes makes them inconvenient for pushing, their flexibility makes them \textit{very useful} for changing the direction of the pull.  The mechanism for changing the direction is the pulley.  Furthermore, by allowing us to change the direction of the pull, we are also able to double, triple, or further improve the strength of the pull.  The term for this is ``the mechanical advantage'' of a pulley-system.

First we will consider three simple cases of redirecting the force.  In each of these cases, I will \hyperref[s:effective2]{assume} that the pulley and rope have no mass and that there is no friction in the turning of the pulley (assume it is trivially easy to spin).  If we do not make this assumption, then the problem gets significantly more complicated.\dothis{add a reference to the section (problem?) where this is considered.}{}

\begin{minipage}[c]{3.25in}
\begin{sample}
\item\studentA\index{\studentA} decides to hold a box that weighs $20\unit N$ using a pulley-system.  What is the tension in the rope?

Since the mass is in equilibrium, the net force is zero and the tension must balance the weight.  This tells us that the tension in the rope is $20 \unit N$.

If the pulley were difficult to turn (had friction) that stickiness could help support the mass and the tension on \studentA's side might be less than $20\unit N$; but since we assumed the pulley to be frictionless, \studentA\ must provide the full $20\unit N$ of tension to the rope.
\end{sample}
\end{minipage}
\hfill
\begin{minipage}{1in}
\begin{picture}(100,120)(0,7)
\put(31,105){\oval(36,36)[t]}
\put(31,105){\circle{33}}
\put(31,106){\line(0,1){29}}
\put(49,105){\line(0,-1){62}}
\put(13,105){\line(0,-1){70}}
\put(-30,7){\line(1,0){100}} % floor
\put(0,135){\line(1,0){62}} % ceiling
\multiput(5,135)(10,0){6}{\line(1,1){5}} % immovable
%
\drawbox{-26}{8}{20}{50} %\studentA
\drawbox{-6}{32}{18}{5} %\studentA's arms
\put(-26,60){\scriptsize \studentA}
%
%\drawbox{5}{19}{16}{16}
%\put(6,25){\small $m_1$}
%
\drawbox{41}{19}{16}{24}
\put(42,25){\small $m$}
%
%\put(49,-1){\vector(0,1){20}}
%\put(49,-1){\vector(0,-1){20}}
%\put(51,-1){\tiny $12\unit m$}
\end{picture}
\end{minipage}
%

\noindent
The interesting aspect is that \studentA\ must pull \textit{down} in order to produce the \textit{upward} tension on the box.  This means that both \studentA\ and the mass are pulling down.  Since the rope is draped over the pulley, the pulley feels $40\unit N$ downwards, $20\unit N$ from the tension supporting the mass and $20\unit N$ from \studentA\ who is creating the tension that supports the mass.  This means that the second rope that is connecting the pulley to the ceiling must be supporting the full $40\unit N$ in order to keep the pulley in equilibrium.



\subsection{Interesting Complications}

\subsubsection{What is the net force on the rope itself?}
The answer to this depends on how complicated you want the answer to be (recall the discussion about effective theories in \autoref{s:effective2}).  Some reasonable answers are:
\begin{itemize}
\item If the rope (and the attachments) are static, then the net force on the rope must be zero even while it maintains the tension.  It is also possible that the rope is accelerating, in which case the net force on the rope while it transfers the forces between the objects at each end is whatever is necessary to produce the acceleration $\vec F_\mathrm{net} = m_\mathrm{rope} \vec a_\mathrm{rope}$.
\item A different answer is to assume that the mass of the rope is small enough that whether it is in equilibrium or accelerating, it does not require a net force and it merely passes its tension through to the object at the other end.
\end{itemize}

\subsubsection{Multiple Masses}\label{sss:multiple.mass}\mautoreturn{ss:tension.support}

Now that we have a few examples of tension under our belts, we can consider some more interesting examples.

\autoref{ex:multiweight.tension} considers the case of hanging multiple masses, which extends the ideas of \autoref{ss:tension.support}.
%
\begin{example}[hbpt]
\fcolorbox{black}{yellow!10}{\begin{minipage}{4.925in}
\caption{\label{ex:multiweight.tension} How many weights?}
While preparing to hang some ornament on a tree, you chain them from a hook on the wall.  You hang ornament 1 from ornament 2 from ornament 3.  What is the tension in each subsequent string?

\color{blue}
The first thing we should do is notice what information is given to us and make sure that everything is in consistent units.  I will convert everything to \hyperref[ss:convertunits]{SI units}.

\color{black}
\autoreturn{sss:multiple.mass}
\end{minipage}}
\end{example}
%
\autoref{ex:multidrag.tension} considers the case of dragging multiple masses, which extends the ideas of \autoref{ss:tension.drag}.
%
\begin{example}[hbpt]
\fcolorbox{black}{yellow!10}{\begin{minipage}{4.925in}
\caption{\label{ex:multidrag.tension} Caravan}
While pulling a sled on which your son sits, your son's sled is tied to a sled on which your dog sits.  Your dog's sled is then connected to a sled with provisions for the day.  What is the tension in each subsequent string?

\color{blue}
The first thing we should do is notice what information is given to us and make sure that everything is in consistent units.  I will convert everything to \hyperref[ss:convertunits]{SI units}.

\color{black}
\autoreturn{sss:multiple.mass}
\end{minipage}}
\end{example}
%
You should note that these examples are essentially expressing the same idea in two different contexts.

\subsubsection{Atwood's Machine}\label{sss:Atwood}

The\dothis{imported a homework problem from Giordano.  Need to modify it to fit my purposes.}{} two crates in the figure (p. 114) hang over a pulley (in what is called an ``Atwood's machine'').  I will select $m_1=35\unit{kg}$ (because it looks smaller) and $m_2=85\unit{kg}$ (because it looks bigger).  We will assume that the pulley is massless and frictionless (so that the tension is the same throughout the rope).  Find the acceleration and the time it takes $m_2$ to accelerate down for the $12\unit m$ to the floor.

\begin{minipage}{1in}
\begin{picture}(100,150)(0,-50)
\put(31,80){\oval(36,36)[t]}
\put(31,80){\circle{33}}
\put(31,81){\line(0,1){29}}
\put(49,80){\line(0,-1){62}}
\put(13,80){\line(0,-1){70}}
%
\put(5,-6){\line(0,1){16}}
\put(5,-6){\line(1,0){16}}
\put(21,10){\line(0,-1){16}}
\put(21,10){\line(-1,0){16}}
\put(6,0){\small $m_1$}
%
\put(41,-6){\line(0,1){24}}
\put(41,-6){\line(1,0){16}}
\put(57,18){\line(0,-1){24}}
\put(57,18){\line(-1,0){16}}
\put(42,0){\small $m_2$}
%
\put(49,-26){\vector(0,1){20}}
\put(49,-26){\vector(0,-1){20}}
\put(51,-26){\tiny $12\unit m$}
\end{picture}
\end{minipage}
\hfill
\begin{minipage}{4.5in}
The easy way to do this is to say that $m_1$ pulls down on the left with $F_{g1} = (35\unit{kg})(9.81\unitfrac{m}{s^2})=\sig{34}{3.4}{N}$ and $m_2$ pulls down on the right with $F_{g2}=(85\unit{kg})(9.81\unitfrac{m}{s^2})=\sig{83}{3.5}{N}$ for a difference of $F_{net} = \sig{49}{0}{N}$ down to the right.  Since this has to move both $m_1$ and $m_2$, the acceleration is
\[ a = \frac{F_{\rm net}}{m_1+m_2} = \frac{\sig{49}{0}{N}}{(35\unit{kg})+(85\unit{kg})} = \frac{\sig{49}{0}{N}}{\sig{120}{}{kg}} = \sigfrac{4.0}{87}{m}{s^2} \]
This acceleration then causes $m_2$ to drop and the time it takes is found from the equation that include distance and time, \\
$y_f \ = \  y_i + v_i \, t + \frac{1}{2} a \,  t^2 $
\[ (0\unit m) \ = \ (12\unit m) + (0\unitfrac ms) \, t + \frac{1}{2} (-\sigfrac{4.0}{9}{m}{s^2}) \,  t^2 \]
\end{minipage}
which we can solve for time:
\[ t \ = \ \sqrt{ \frac{-(12\unit m)}{\frac{1}{2} (-\sigfrac{4.0}{9}{m}{s^2})} } \ = \  \sqrt{ \sig{5.8}{7}{s^2}} \ = \ \sig{2.4}{2}{s} \]

\footnoterule
\small
However, this does not show what the tension is, and many students make a mistake with the tension.  So, I will also answer the question about the tension. We can draw three free-body diagrams. The equation for $m_1$ is as follows, where I am putting the sign in by
\newpar

\begin{minipage}{4.5in}
hand to indicate the direction: \hfill
$\displaystyle (-F_{g1}) + (+F_T) = m_1 (+a) $ \\
The equation for $m_2$ is as follows: \hfill
$\displaystyle (-F_{g2}) + (+F_T) = m_2 (-a) $ \\
Since we know the weights and the masses, these two equations and two unknowns can be written as
\begin{eqnarray*}
(-\sig{34}{3}{N}) + (+F_T) & = & (35\unit{kg}) (+a) \\
(-\sig{83}{3}{N}) + (+F_T) & = & (85\unit{kg}) (-a)
\end{eqnarray*}
There are many ways to solve two equations and two unknowns.
If we subtract the second equation from the first, then we get the equation on the left.
But, if we solve the first equation for $a$ and plug it into the second, then we get the equation on the right
\end{minipage}
\hfill
\begin{minipage}{1in}
\begin{picture}(100,150)(0,-50)
%\put(31,80){\oval(36,36)[t]}
\put(31,80){\circle{33}}
\put(31,81){\vector(0,1){50}}
\put(47.5,80){\vector(0,-1){30}}
\put(14.5,80){\vector(0,-1){30}}
\put(50,55){\tiny $F_T$}
\put(15,55){\tiny $F_T$}
%
\put(5,-6){\line(0,1){16}}
\put(5,-6){\line(1,0){16}}
\put(21,10){\line(0,-1){16}}
\put(21,10){\line(-1,0){16}}
\put(13,4){\vector(0,1){30}}
\put(13,0){\vector(0,-1){20}}
\put(14,20){\tiny $F_T$}
\put(14,-15){\tiny $F_{g1}$}
%
\put(49,8){\vector(0,1){30}}
\put(49,4){\vector(0,-1){40}}
\put(41,-6){\line(0,1){24}}
\put(41,-6){\line(1,0){16}}
\put(57,18){\line(0,-1){24}}
\put(57,18){\line(-1,0){16}}
\put(51,30){\tiny $F_T$}
\put(51,-15){\tiny $F_{g1}$}
\end{picture}
\end{minipage}

\[ \begin{array}{ccc}
\deq
(-\sig{34}{3}{N}) - (-\sig{83}{3}{N}) \ = \ \left[ (35\unit{kg}) + (85\unit{kg}) \right] (a) &&
\deq
(-\sig{83}{3}{N}) + (F_T) \ = \  - (85\unit{kg}) \left[ \frac{(-\sig{34}{3}{N}) + (F_T)}{(35\unit{kg})} \right] \\
\deq
a \ = \ \frac{\sig{49}{0}{N}}{(35\unit{kg})+(85\unit{kg})} = \sigfrac{4.0}{87}{m}{s^2} &&
\deq
F_T \ = \ \frac{-(35\unit{kg})(-\sig{83}{3}{N})-(85\unit{kg})(-\sig{34}{3}{N})}{[(35\unit{kg})+(85\unit{kg})]} \ = \ \sig{48}{6}{N}
\end{array} \]
The acceleration is as above.  The tension is not enough to support $m_2$ (so it falls) and more than enough to lift $m_1$ (so it rises).
You should note that
$\left[(\sig{48}{6}{N}-\sig{34}{3}{N})/(35\unit{kg})=\sigfrac{4.0}{9}{m}{s^2}\right]$
\hfill and \hfill
$\left[(\sig{83}{3}{N}-\sig{48}{6}{N})/(85\unit{kg})=\sigfrac{4.0}{9}{m}{s^2}\right]$.

\normalsize

\subsubsection{Surface Tension}

As a \hypertarget{d:surf.tension}{final note}, \hyperref[s:surface.tension]{surface tension} is something else entirely.  See \autoref{sss:tea} for a comment on the contribution to hot versus cold spoon noises.

\section{Frictional Force}\label{s:Ff}\mmultireturn{\mmr{\ref{A:chair2}}, \mmr{\ref{A:chair6}}, \mmr{\ref{A:chair7}}, \mmr{\autoref{A:fly.balls}}}

%
\begin{reallife}[bthp]
\hspace{-.2in}
\fcolorbox{black}{green!10}{\begin{minipage}{5.29in} \center
\caption{\label{irl:poolfriction}\index{Pool!Real Life} Rolling pool balls and friction.}
\begin{minipage}{4.925in}
\studentD\index{\studentD} is relaxing with the local physics club, playing pool.  \HeD\ hits the cue ball and counts the number of walls \heD\ can hit in one shot.
\end{minipage}
\begin{realtable}
\dna{Hit the cue-ball off of a bumper in the manner intended for
\protect{\href{http://c.ymcdn.com/sites/bca-pool.com/resource/resmgr/imported/BCAEquipmentSpecifications_2008.pdf}{testing cushions}}.}
    {Compare the strength of the hit to the distance travelled}
    {How much is the total distance affected by the number of bumpers hit? \\
     Does it matter if you shoot along the length of the table versus the width of the table?  \\
     Why does friction slow the ball down instead of just make it turn $v=\omega r$ (no slip)}
\end{realtable}
\begin{minipage}{4.925in}
Billiard tables have a lot of interesting physics, which can help us see a wide variety of physics, for example:
\hyperref[irl:poolnormal]{normal force}, \hyperref[irl:poolelastic]{elastic versus inelastic collisions}, \hyperref[irl:poolrotmot]{rotational motion}, and \hyperref[irl:poolangmom]{angular momentum}.
\end{minipage}

%\flushright
%\linkreturn[pool]{d:bank-shot}
\end{minipage}}
\end{reallife}
%

\section{Spring Force}\label{s:springs}\mmultireturn{\mmr{\hyperlink{d:f=ma}{$F=ma$}}, \mmr{\hyperlink{d:usesofF=ma}{uses of $F=ma$}}, \mmr{\autoref{ss:scales}}}

\section{Applied Force}

The term ``an applied force'' is used to describe any force applied by any object when there isn't really a formula to find it.  So this is kind of a ``any other force'' category.  I will use this type of force to describe forces exerted by people.  We have seen some examples where a person throws an object.  We can now revisit those examples and consider the force exerted (applied) by the person who threw the object.
\begin{sample}
\item\label{se:throw-up} \studentC\index{\studentC} recalls that one time \heC\ got bored one day in physics class (what?!?) and tossed a baseball ($m_b = 0.145\unit{kg}$) at the ceiling\ldots a little too hard \ldots as recounted in \autoref{ex:ceiling}.  Recall that \ref{se:ceiling} found the normal force by the ceiling on the ball.  Please now find the force \studentC\ applied while throwing and catching the ball assuming that the throw took $0.200\unit{s}$ to gain the speed of $5.00\unitfrac ms$ and the catch took $0.250\unit s$ to slow the ball from $4.73\unitfrac ms$ to rest.

There are five stages to the motion: (a) throwing, (b) falling up, (c) hitting the ceiling, (d) falling down, and (e) catching show the forces involved. \\
\fbox{\begin{minipage}[b]{55pt}
\begin{picture}(50,100)(0,0)
\put(25,25){\circle{10}}
\put(25,26){\vector(0,1){25}}
\put(25,24){\vector(0,-1){15}}
\put(28,35){$F_\mathrm{throw}$}
\put(28,10){$F_g$}
\end{picture}
\centering{(a) throwing}
\end{minipage}}
\hfill
\color{lightgray}
\fbox{\begin{minipage}[b]{55pt}
\begin{picture}(50,100)(0,0)
\put(25,50){\circle{10}}
\put(25,50){\vector(0,-1){15}}
\put(28,35){$F_g$}
\end{picture}
\centering{(b) falling up}
\end{minipage}}
\hfill
\fbox{\begin{minipage}[b]{55pt}
\begin{picture}(50,100)(0,0)
\put(25,95){\circle{10}}
\put(26,95){\vector(0,-1){25}}
\put(24,95){\vector(0,-1){15}}
\put(28,75){$F_N$}
\put(10,75){$F_g$}
\end{picture}
\centering{(c) \\ hitting}
\end{minipage}}
\hfill
\fbox{\begin{minipage}[b]{55pt}
\begin{picture}(50,100)(0,0)
\put(25,50){\circle{10}}
\put(25,50){\vector(0,-1){15}}
\put(28,35){$F_g$}
\end{picture}
\centering{(d) falling down}
\end{minipage}}
\hfill
\color{rgb:red,0;green,2;blue,1}
\fbox{\begin{minipage}[b]{55pt}
\begin{picture}(50,100)(0,0)
\put(25,25){\circle{10}}
\put(25,26){\vector(0,1){25}}
\put(25,24){\vector(0,-1){15}}
\put(28,35){$F_\mathrm{catch}$}
\put(28,10){$F_g$}
\end{picture}
\centering{(e) catching}
\end{minipage}}
\\
In this particular problem, we are only concerned with steps (a) and (e) because that's where \studentC\ throws and catches the ball. In each case, we need the acceleration: \\
\begin{minipage}[b]{150pt}
\begin{eqnarray*}
\vec a_\mathrm{throw} & = & \frac{(+5.00\unitfrac ms \jhat)-(0\unitfrac ms \jhat)}{0.200\unit s} \\
& = & +\sigfrac{25.0}{0}{m}{s^2} \jhat
\end{eqnarray*}
\end{minipage}
\hfill
\begin{minipage}[b]{150pt}
\begin{eqnarray*}
\vec a_\mathrm{catch} & = & \frac{(0\unitfrac ms \jhat)-(-4.73\unitfrac ms \jhat)}{0.250\unit s} \\
& = & +\sigfrac{18.9}{2}{m}{s^2} \jhat
\end{eqnarray*}
\end{minipage}

During each step, we have the actual acceleration, which tells us about the net force.  We will also need to know the weight of the baseball $F_g=\sig{1.42}{2}{N}$, because gravity is still acting during the collision.  Let's consider the throwing part first.
\begin{eqnarray*}
\vec F_N + \vec F_g & = &  \vec F_\mathrm{net} \ = \ m \vec a \\
\vec F_A  & = &  m \vec a - \vec F_g \\
\vec F_A  & = &  \left[ (0.145\unit{kg})(+\sigfrac{25.0}{0}{m}{s^2}\jhat) \right] - \left[  - \sig{1.42}{2}{N} \jhat \right] \\
\vec F_A  & = &  \left[ +\sig{3.62}{5}{N} \jhat \right] - \left[  - \sig{1.42}{2}{N} \jhat \right] \ = \ +\sig{5.04}{7}{N} \jhat
\end{eqnarray*}
You can see that the upward applied force $(\sig{5.04}{7}{N})$ has to be large enough so that when it is combined with the downward gravitational force $(\sig{1.42}{2}{N})$ they can together result in the necessary (but smaller) upward net force $(\sig{3.62}{5}{N})$ to get it going upwards.

For the catching part, the ball is moving downwards and needs to be stopped, so the catching applied force must be upwards.
\begin{eqnarray*}
\vec F_A + \vec F_g & = &  \vec F_\mathrm{net} \ = \ m \vec a \\
\vec F_A  & = &  m \vec a - \vec F_g \\
\vec F_A  & = &  \left[ (0.145\unit{kg})(+\sigfrac{18.9}{2}{m}{s^2}\jhat) \right] - \left[  - \sig{1.42}{2}{N} \jhat \right] \\
\vec F_A  & = &  \left[ +\sig{2.74}{3}{N} \jhat \right] - \left[  - \sig{1.42}{2}{N} \jhat \right] \ = \ +\sig{4.16}{5}{N} \jhat
\end{eqnarray*}
You can see that the upward applied force $(\sig{4.16}{5}{N})$ has to be large enough so that when it is combined with the downward gravitational force $(\sig{1.42}{2}{N})$ they can together result in the necessary upward net force $(\sig{2.74}{3}{N})$ to stop it from continuing downwards.
\end{sample}

\section{Putting it Together, $F_\mathrm{net}$}\label{s:Fnet}

\subsection{Translational Equilibrium}

blah blah blah
\phantomsection\label{ss:transeq} Translational equilibrium: $F_\mathrm{net} = m \cancelto{0}{a}$.  blah blah blah

\subsection{Static Equilibrium}

\subsection{Dynamic Equilibrium}


\section{Summary and Homework}

\subsection{Summary of Concepts and Equations}\new{v2.3}{Created this section}

\ldots

\subsection*{Conceptual Questions}\new{v2.3}{Added two conceptual problems.}
%\vspace{-24pt}
\begin{enumerate}
\item\label{c:weightmass} Estimate, preferably without using the internet, the mass of the following: (a) a four-door sedan, (b) dishwasher, (c) a pair of glasses, (d) a cell phone.  You should be able to estimate to within one significant digit.
\item\label{c:massweight} List at least one object, preferably without using the internet, that has the following mass: (a) $2500\unit{kg}$ (b) $41\unit{kg}$, (c) $3\unit{kg}$, (d) $50\unit{g}$.
\end{enumerate}
\subsection*{Problems}\new{v2.3}{Created section.}\dothis{Add more problems.}
%\vspace{-24pt}
\begin{enumerate}
 \item\ldots
\end{enumerate}


\chapter{Energy and the Transfer of Energy}

\hypertarget{d:energynoun}{Energy is a noun}\index{Energy!noun}; objects can \textit{have} energy.  \hypertarget{d:workverb}{Work is a verb}\index{Work!verb}\mlinkreturn[heat as a verb]{d:heatverb}; doing work is the process of \textit{exchanging} energy.

\section{Objects Can Have Energy}

\section{A Force Can Transfer Energy} \label{s:work}\mlinkreturn[the direction of forces]{d:pushvector}

\section{Dissipating Energy} \label{s:Wfr}

pool balls on cushion/bumper

\section{Conserving Energy} \label{s:PE}

%
\begin{reallife}[bthp]
\hspace{-.2in}
\fcolorbox{black}{green!10}{\begin{minipage}{5.29in} \center
\caption{\label{irl:poolelastic}\index{Pool!Real Life} 1-D elastic collisions of pool balls.  inelastic collisions off the bumper.}
\begin{minipage}{4.925in}
\studentD\index{\studentD} is relaxing with the local physics club, playing pool.  \HeD\ hits the cue ball and counts the number of walls \heD\ can hit in one shot.
\end{minipage}
\begin{realtable}
\dna{collide balls.}
    {where does it hit}
    {$90^\circ$ output}
\end{realtable}
\begin{minipage}{4.925in}
Billiard tables have a lot of interesting physics, which can help us see a wide variety of physics, for example:
\hyperref[irl:poolnormal]{normal force}, \hyperref[irl:poolelastic]{elastic versus inelastic collisions}, \hyperref[irl:poolrotmot]{rotational motion}, and \hyperref[irl:poolangmom]{angular momentum}.
\end{minipage}

%\flushright
%\linkreturn[pool]{d:bank-shot}
\end{minipage}}
\end{reallife}
%

\subsection{Gravitational Potential Energy}\label{ss:PEg}\mautoreturn{s:PEG}
See also \ref{s:PEG}
\subsection{Spring Potential Energy}\label{ss:PEs}
\subsection{Conservative Forces in General}

\part{Interesting Uses of Motion, Force, and Energy}

\chapter{Momentum: A Better Way to Describe Force}\label{c:momentum}\mmultireturn{\mmr{\hyperlink{d:objectinmotion}{objects in motion}}, \mmr{\autoref{sss:inertia}}, \mmr{\autoref{ss:NIII}}, \mmr{\ref{A:chair6}}}

Useful to include?
\href{https://www.wired.com/2017/06/physics-bullets-versus-wonder-womans-bracelets/}{The Physics of Bullets Vs. Wonder Woman's Bracelets}

\section{Revising Newton's First and Second Laws}

\subsection{Inertia and Momentum}\label{ss:inertia}\mautoreturn{sss:inertia}
Recall \autoref{sss:inertia}.

\section{Revising Newton's Third Law: Conservation of Momentum}\label{s:conservemom}\mautoreturn{ss:NIII}

\section{Two-Dimensional Collisions}\label{s:2Dcollisions}\mautoreturn{sss:vectorequations}

pool balls?  What about rolling?
%
\begin{reallife}[bthp]
\hspace{-.2in}
\fcolorbox{black}{green!10}{\begin{minipage}{5.29in} \center
\caption{\label{irl:pool2Dcollision}\index{Pool!Real Life} 2-D collisions of pool balls.}
\begin{minipage}{4.925in}
\studentD\index{\studentD} is relaxing with the local physics club, playing pool.  \HeD\ hits the cue ball and counts the number of walls \heD\ can hit in one shot.
\end{minipage}
\begin{realtable}
\dna{collide balls.}
    {where does it hit}
    {$90^\circ$ output}
\end{realtable}
\begin{minipage}{4.925in}
Billiard tables have a lot of interesting physics, which can help us see a wide variety of physics, for example:
\hyperref[irl:poolnormal]{normal force}, \hyperref[irl:poolelastic]{elastic versus inelastic collisions}, \hyperref[irl:poolrotmot]{rotational motion}, and \hyperref[irl:poolangmom]{angular momentum}.
\end{minipage}

%\flushright
%\linkreturn[pool]{d:bank-shot}
\end{minipage}}
\end{reallife}
%


\chapter{Rotational Motion}

\section{The Equations of Rotational Motion}

%
\begin{reallife}[bthp]
\hspace{-.2in}
\fcolorbox{black}{green!10}{\begin{minipage}{5.29in} \center
\caption{\label{irl:poolrotmot}\index{Pool!Real Life} Rolling pool balls.}
\begin{minipage}{4.925in}
\studentD\index{\studentD} is relaxing with the local physics club, playing pool.  \HeD\ hits the cue ball and counts the number of walls \heD\ can hit in one shot.
\end{minipage}
\begin{realtable}
\dna{Roll a striped ball along the table.}
    {Use the stripe to notice the rate of rotation}
    {How does the rotation compare to the translation?}
\dna{Roll a striped ball along the table.}
    {Notice the distance the ball travels}
    {Why does friction slow the ball down instead of just make it turn $v=\omega r$ (no slip)}
\end{realtable}
\begin{minipage}{4.925in}
Billiard tables have a lot of interesting physics, which can help us see a wide variety of physics, for example:
\hyperref[irl:poolnormal]{normal force}, \hyperref[irl:poolelastic]{elastic versus inelastic collisions}, \hyperref[irl:poolrotmot]{rotational motion}, and \hyperref[irl:poolangmom]{angular momentum}.
\end{minipage}

%\flushright
%\linkreturn[pool]{d:bank-shot}
\end{minipage}}
\end{reallife}
%

\section{Angular Momentum}

%
\begin{reallife}[bthp]
\hspace{-.2in}
\fcolorbox{black}{green!10}{\begin{minipage}{5.29in} \center
\caption{\label{irl:poolangmom}\index{Pool!Real Life} Rolling pool balls.}
\begin{minipage}{4.925in}
\studentD\index{\studentD} is relaxing with the local physics club, playing pool.  \HeD\ hits the cue ball and counts the number of walls \heD\ can hit in one shot.
\end{minipage}
\begin{realtable}
\dna{Roll a striped ball along the table.}
    {Use the stripe to notice the rate of rotation}
    {How does the rotation compare to the translation?}
\end{realtable}
\begin{minipage}{4.925in}
Billiard tables have a lot of interesting physics, which can help us see a wide variety of physics, for example:
\hyperref[irl:poolnormal]{normal force}, \hyperref[irl:poolelastic]{elastic versus inelastic collisions}, \hyperref[irl:poolrotmot]{rotational motion}, and \hyperref[irl:poolangmom]{angular momentum}.
\end{minipage}

%\flushright
%\linkreturn[pool]{d:bank-shot}
\end{minipage}}
\end{reallife}
%


\section{Non-inertial Rotational Reference Frames} \label{s:noninertial}\mmultireturn{\mmr{\autoref{ss:noninertial}}, \mmr{\hyperlink{d:NewtonInertial}{non-inertial reference frames}}, \mmr{\autoref{ss:NI}}}
\index{Reference Frames!Inertial}
\index{Reference Frames!Non-inertial}

Because the Earth \hypertarget{d:noninertial}{rotates}\mautoreturn{ss:NII}, we are actually in a non-inertial reference frame.  In fact, we can prove that the Earth rotates by observing the effects, such as the \hyperlink{d:coriolis}{Coriolis effect}, that in our non-inertial frame seem to require unexplainable forces but which, in a non-rotating frame, follow the expected laws of physics.

\subsection{The Coriolis Effect}\label{ss:coriolis}\mmultireturn{\mmr{\hyperlink{d:NewtonInertial}{non-inertial reference frames}}, \mmr{\hyperlink{d:noninertial}{Non-inertial Rotational Reference Frames}}}

\hypertarget{d:coriolis}{weather, etc}
\newpar

In her podcast\new{v2.0}{\textit{Spacepod}}, \textit{Spacepod}\footnote{Nugent, Carrie (Producer, Host). \textit{Spacepod} [Audio podcast], episode 89 (19 May, 2017).  Retrieved from \hyperref{http://spacepod.libsyn.com/}{T4LTFdOxHD5WWzdD}{99}{\nolinkurl{http://spacepod.libsyn.com/}}
on 9 Apr. 2017.} Dr. Carrie Nugent interviews Dr. Andy Thompson about ``underwater flying objects'' that investigate the ocean.  He notes that ocean waters, because they are such a large-scale system, can see the effect of the rotation of the Earth.

\subsection{The Foucault Pendulum}\label{ss:Foucault}

See \href{https://www.youtube.com/watch?v=sWDi-Xk3rgw}{youtube video} by \href{http://sixtysymbols.com/}{Sixty Symbols}.\new{v2.0}{Foucault video}




\chapter{Circular Motion and Centripetal Force}

\section{Circular Motion}
\section{Centripetal Force}\label{s:centripetal}\mlinkreturn[$F=ma$]{d:f=ma}




\chapter{Torque and the $F=ma$ of Rotations}\label{c:torque}\mreturn{a:NIIIaction}\new{v2.3}{Added an example that is computable here, but helps introduce normal force in \protect{\autoref{s:FN}}.}

\section{Leverage}\label{s:leverarm}\mautoreturn{ss:scales}

\section{Putting it all together, $\tau_\mathrm{net}$}

\subsection{Rotational Equilibrium}

blah blah blah
\phantomsection\label{ss:roteq} Rotational equilibrium: $\tau_\mathrm{net} = I \cancelto{0}{\alpha}$.  blah blah blah

\subsection{Static (Rotational) Equilibrium}

\subsection{Dynamic (Rotational) Equilibrium}

\new{v2.3}{Answered \protect{\autoref{ex:ladder2}} and its related problems.}
\begin{example}[p]
\fcolorbox{black}{yellow!10}{\begin{minipage}{4.925in}\setlength{\parskip}{3pt}
\caption{\label{ex:ladder2} \studentC\index{\studentC} uses a ladder}
\begin{quote}
\studentC\ leans a $22.7\unit{kg}$ ladder against a wall at an angle of $75.5^\circ$, consistent with \protect{\href{https://www.osha.gov/}{OSHA}} standard \protect{\href{https://www.osha.gov/pls/oshaweb/owadisp.show_document?p_table=standards&p_id=10839}{1926.1053(a)(1)(ii)}}.
The coefficient of friction between the ladder and the floor is $\mu_f=0.31$.
The coefficient of friction between the ladder and the wall is $\mu_w=0.19$.
Use the rotational and translational equilibrium to determine if the ladder slides.
\end{quote}

Since we are asked to distinguish between two cases that cannot both be true, we should assume one (the easier one to calculate is that the ladder does not slip) and then verify that the result is consistent with that assumption.

\textbf{What do we know?}
We know that the floor has a normal force $(F_{Nf})$ upwards and a frictional force $(F_{ff})$ to the left.
We know that the wall
\\[2pt]
\begin{minipage}{3.2in}
has a normal force $(F_{Nw})$ to the right and a frictional force $(F_{fw})$ up (keeping the ladder from sliding down).
We know the weight is
\[ F_g = mg = (22.7\unit{kg})(9.81\unitfrac{m}{s^2}) = \sig{222}{.69}{N} \]
\textbf{What do we want to know?}  We want to know about the the magnitudes of both normal
\end{minipage}
\hfill
\begin{minipage}{100pt}
\begin{picture}(100,100)(-10,-5)
% Dimensions and offset: (width,height)(x offset,y offset)
% Insert picture commands (\line,\circle, etc...) here:
\put(0,0){\line(0,1){100}}
\put(0,0){\line(1,0){75}}
\put(20,0){\color{blue}\line(-1,4){20}}     % ladder
\put(10,40){\color{red}\vector(0,-1){30}}   % Fg
\put(20,1){\color{red}\vector(0,1){25}}     % FNf
\put(19,1){\color{red}\vector(-1,0){12}}
\put(1,80){\color{red}\vector(1,0){12}}
\put(1,81){\color{red}\vector(0,1){8}}
\end{picture}
\end{minipage}
%\hfill {}
\\[3pt]
forces and both frictional forces.
Can we easily deduce the magnitude of $F_{Nf}$? \ref{A:ladderNf}.

\textbf{How are these related?}  The forces acting on any body are related by static \hyperref[ss:transeq]{translational equilibrium}
\begin{eqnarray*}
x: \hspace{.5cm} 0 & = & \cancelto{0}{F_{gx}} + \cancelto{0}{F_{Nfx}} + F_{ffx} + F_{Nwx} + \cancelto{0}{F_{fwx}} \\
y: \hspace{.5cm} 0 & = & F_{gy} + F_{Nfy} + \cancelto{0}{F_{ffy}} + \cancelto{0}{F_{Nwy}} + F_{fwy}
\end{eqnarray*}
and static \hyperref[ss:roteq]{rotational equilibrium}, assuming the pivot point is at the ground, and using the relationship $F_f=\mu F_N$, we find
\begin{eqnarray*}
0 & = & \tau_{g} + \cancelto{0}{\tau_{Nf}} + \cancelto{0}{\tau_{ff}} + \tau_{Nw} + \tau_{fw} \\
0 & = & \left[ F_g \frac{l}{2} \sin 14.5^\circ \right] + \left[ - F_{Nw} l \sin(75.5^\circ) \right] + \left[ - F_{fw} l \sin(14.5^\circ) \right] \\
F_{Nw} & = & \left[ F_g \frac{l}{2} \sin 14.5^\circ \right] / \left[  l \sin(75.5^\circ) + \mu_w l \sin(14.5^\circ) \right]
\end{eqnarray*}
%\textbf{Free-Body Diagrams:}  Since the picture is so simple, we will not draw the free-body diagram.


{}\hfill {\footnotesize\autoref*{ex:ladder2} continued on next page\ldots}
\end{minipage}}
\end{example}
\begin{example}[p]
\fcolorbox{black}{yellow!10}{\begin{minipage}{4.925in}\setlength{\parskip}{3pt}
{\footnotesize \autoref*{ex:ladder2} continued from previous page\ldots}

\textbf{Concepts to Consider:}  First, the length of the ladder cancels from the expression; what matters is the angle at which it is propped.

Second, every force value will be linearly dependent on the mass of the ladder.  So once we solve this problem, we can easily scale the answers to any mass.

Third, the friction with the wall is, by far, the smallest effect and it might be interesting to approximate all of this with $\mu_w=0$.  You can check your calculation against \ref{A:nowall}.

\textbf{Solution to the example:}  When we worry about significant figures,
\begin{eqnarray*}
F_{Nw} & = & \frac{\left[ (\sig{222}{.7}{N})(\txtfrac{1}{2}) (0.\sig{250}{4}{}) \right]}{\left[  (0.\sig{968}{2}{}) + (0.19) (0.\sig{250}{4}{}) \right]}
\ = \ \frac{\left[ (\sig{27.8}{8}{N})\right]}{\left[  (0.\sig{968}{2}{}) + (0.0\sig{47}{6}{}) \right]} \\
F_{Nw} & = & \frac{\left[ (\sig{27.8}{8}{N})\right]}{\left[  (\sig{1.015}{7}{}) \right]}
\ = \ \sig{27.4}{4}{N} \\
F_{fw,\mathrm{max}} & = & (0.19)(\sig{27.4}{4}{N}) \ = \ \sig{5.2}{15}{N} \\
F_{Nf} & = & F_g - F_{fw} = (\sig{222}{.7}{N})-(\sig{5.2}{15}{N}) \ = \ \sig{217}{.5}{N} \\
F_{ff,\mathrm{max}} & = & (0.31)(\sig{217}{.5}{N}) \ = \ \sig{672}{.4}{N}
\end{eqnarray*}
Since $F_{ff} >F_{Nw}$, the friction is sufficient to hold the ladder in place, as assumed.

%\begin{quote}
\textbf{Aside:} Since $F_{ff}$ only needs to be $\sig{27.4}{4}{N}$ to hold the ladder in place, it is possible for the ladder to not slide on a floor that only has
$\mu_\mathrm{min} = (\sig{27.4}{4}{N})/(\sig{217}{.5}{N}) = 0.\sig{126}{2}{}$; but that would not allow a person to climb the ladder.

\textbf{Homework:} Homework problem~\ref{h:ladderC} asks you to determine if the ladder slides when \studentC\ climbs to different locations on the ladder.
%\end{quote}
\flushright
\multireturn{\mmr{\ref{se:ladderN}}, \mmr{\autoref{ss:roteq}}}
\end{minipage}}
\end{example}

\section{Torsion}\label{s:torsion}\mautoreturn{s:FT}\new{v2.4}{Created this section}

\section{Summary and Homework}

\subsection{Summary of Concepts and Equations}\new{v2.3}{Created this section}

\ldots

\subsection*{Conceptual Questions}\dothis{Add conceptual problems.}
%\vspace{-24pt}
\begin{enumerate}
\item\ldots
\end{enumerate}
\subsection*{Problems}\new{v2.3}{Added problems.}\dothis{Add more problems.}
%\vspace{-24pt}
\begin{enumerate}
 \item\label{h:ladderC} \studentC\ leans a $22.7\unit{kg}$ ladder against a wall at an angle of $75.5^\circ$, consistent with \protect{\href{https://www.osha.gov/}{OSHA}} standard \protect{\href{https://www.osha.gov/pls/oshaweb/owadisp.show_document?p_table=standards&p_id=10839}{1926.1053(a)(1)(ii)}}.\new{v2.3}{Answered \protect{\ref{h:ladderC}} and its related problems.}
The coefficient of friction between the ladder and the floor is $\mu_f=0.31$.
The coefficient of friction between the ladder and the wall is $\mu_w=0.19$.
Use the rotational and translational equilibrium to determine if the ladder slides when \studentC\ ($\massC$) climbs to
\begin{enumerate}
\item the third-rung from the top of the ladder, so that he is $1.53\unit m$ from the bottom of the ladder.
    (See \ref{A:nowallC} for that answers if $\mu_w = 0$.)
\begin{ForMe}
\color{blue} Answers:
\begin{eqnarray*}
F_{Nw} & = & \sig{163}{.9}{N} \\
F_{fw,\mathrm{max}} & = & (0.19)(\sig{163}{.9}{N}) \ = \ \sig{31}{.14}{N} \\
F_{Nf} & = & \sig{1074}{.4}{N} \\
F_{ff,\mathrm{max}} & = & \sig{333}{.0}{N} < \sig{163}{.9}{N}
\end{eqnarray*}
$\mu_\mathrm{min} = 0.\sig{152}{56}{}$
\color{black}
\end{ForMe}
\item the third-rung from the bottom of the ladder, so that he is $0.914\unit m$ from the bottom of the ladder.
\begin{ForMe}
\color{blue}
Answers:
\begin{eqnarray*}
F_{Nw} & = & \sig{108}{.97}{N} \\
F_{fw,\mathrm{max}} & = & (0.19)(\sig{108}{.97}{N}) \ = \ \sig{20}{.70}{N} \\
F_{Nf} & = & \sig{1084}{.9}{N} \\
F_{ff,\mathrm{max}} & = & \sig{336}{.3}{N} < \sig{108}{.97}{N}
\end{eqnarray*}
$\mu_\mathrm{min} = 0.\sig{100}{45}{}$

If $\mu_w = 0$.
\begin{eqnarray*}
F_{Nw} & = & \sig{114}{.3}{N} \\
F_{fw,\mathrm{max}} & = & 0 \unit N \\
F_{Nf} & = & \sig{1105}{.6}{N} \\
F_{ff,\mathrm{max}} & = & \sig{342}{.7}{N} < \sig{114}{.3}{N}
\end{eqnarray*}
$\mu_\mathrm{min} = 0.\sig{103}{4}{}$
\color{black}
\end{ForMe}
\end{enumerate}
\end{enumerate}


\chapter{Energy of Rotating Objects}
\section{Rotational Kinetic Energy}
pool balls

\chapter{The Gravitational Force on a Large Scale}\label{c:gravity}\mmultireturn{\mmr{\hyperlink{d:accgrav}{freefall}}, \mmr{\hyperlink{d:fundamental}{fundamental forces}}}

\section{Gravitational Force and Field}\label{s:Gfield}\mlinkreturn[$F=ma$]{d:f=ma}\new{v2.3}{Added some placeholders}

The value of the acceleration due to gravity  varies according to the mass and size of any celestial body.\dothis{Reference a table of $g$ on other planets and compute the weight of a space craft at each planet.}
This means that, as was seen in \ref{se:gworld}, your weight can change even when your mass remains the same.
\begin{sample}
\item\label{se:gplanets} In conversation with a visiting alien, \studentX\index{\studentX}, you find that \studentX\ has been to the moon and several planets both within and outside of our solar system.  In addition to the Earth, \studentX\ has visited our moon, Mars, Pluto, and Planet X.  Using \autoref{t:gplanets}, compute \studentX's weight are each location, assuming \hisX\ mass is \massX.
\begin{enumerate}
\item[Earth] $F_g = (\massX)\left[ \frac{ G M_E}{R_E^2} \right] = (\massX)(9.825\unitfrac{m}{s^2}) \ = \ \sig{933}{.4}{N}$
\item[moon] $F_g = (\massX)\left[ \frac{ G M_m}{R_m^2} \right] = (\massX)(9.782\unitfrac{m}{s^2}) \ = \ \sig{929}{.3}{N}$
\item[Mars] $F_g = (\massX)\left[ \frac{ G M_M}{R_M^2} \right] = (\massX)(9.763\unitfrac{m}{s^2}) \ = \ \sig{927}{.5}{N}$
\item[Pluto] $F_g = (\massX)\left[ \frac{ G M_P}{R_P^2} \right] = (\massX)(9.763\unitfrac{m}{s^2}) \ = \ \sig{927}{.5}{N}$
\item[Planet X] $F_g = (\massX)\left[ \frac{ G M_X}{R_X^2} \right] = (\massX)(9.763\unitfrac{m}{s^2}) \ = \ \sig{927}{.5}{N}$
\end{enumerate}
\end{sample}
%
\begin{table}[bhtp]
\hrule\hrule
\begin{center}
\caption[Properties of various celestial bodies]{\label{t:gplanets} Properties of various celestial bodies.
\return{se:gplanets}
}
\begin{tabular}{lccr}
Planet & Mass (kg) & Mean Radius (m) & $g (\unitfrac{m}{s^2})$ \\
\end{tabular}
\end{center}
\hrule\hrule
\end{table}
%


\subsection{Inertial Mass versus Gravitational Mass}\label{ss:equivmm}\mautoreturn{ss:weightmass}\new{v2.2}{Moved this here, might need to move it back.}

\section{Gravitational Potential Energy} \label{s:PEG}\mautoreturn{ss:PEg}

Recall \ref{ss:PEg}

\section{Making Connections}\label{s:Gconnection}\mautoreturn{s:Econnection}

\[ \begin{array}{ccccc}
& & \vec F = m \vec g & & \\
& \deq F = G \frac{m_1 m_2}{R^2} & \leftrightarrow & \deq g = G \frac{m}{R^2} & \\
\Delta \PE = -\vec F \cdot \Delta\vec x & \updownarrow & & \updownarrow & \mbox{\scriptsize [for later]} \\
& \deq \PE = G \frac{m_1 m_2}{R} & \leftrightarrow & \mbox{[for later]} & \\
& & \mbox{\scriptsize [for later]} & &
\end{array} \]
(Look ahead to the parallel with the electrical interaction in \autoref{s:Econnection}.)

\section{Orbits}


\part{Making Waves}

\chapter{Fluids}\new{v2.2}{Placeholder}
\section{Density}\label{s:density}\index{Density}\mautoreturn{ss:weightmass}

\section{Surface Tension}\label{s:surface.tension}\mlinkreturn{d:surf.tension}



\chapter{Oscillations}\label{c:SHM}
\section{Oscillating Springs}\label{c:SHMspring}\mlinkreturn[$F=ma$]{d:f=ma}
\section{Oscillating Pendulums}\label{c:SHMpend}

\section{Other Examples of Oscillations}\label{s:SHMother}

On 13 April, 2017,\new{v2.3}{New source of info}
\href{http://www.cbc.ca/podcasting}{CBC Broadcasting} published a
\href{http://www.cbc.ca/podcasting/includes/quirks.xml}{\textit{Quirks and Quarks}} episode discussing how we can find
\href{https://podcast-a.akamaihd.net/mp3/podcasts/quirks_20170415_12100.mp3}{solutions to health issues caused by swaying office towers and vibrating floors}.

\chapter{Sound}
\subsection{Musical Instruments}\label{ss:stringed.instruments} \mautoreturn{A:swing.tension}



\part{Is It Hot in Here?}

\chapter{The flow of thermal energy}

\phantomsection\label{find:heatwarm}
Energy is a noun\index{Energy!noun}; objects can \textit{have} energy.  \hypertarget{d:heatverb}{Heat is a verb}\index{Heat!verb}; heating is a process of \textit{exchanging} energy.  Recall our \hyperlink{d:forcenoun}{discussions of force}\index{Force!noun} and \hyperlink{d:workverb}{work}\index{Work!verb}.

\section{Specific Heat Capacity}\label{s:specificheat}

\hypertarget{d:heatwarm}{Heating (positive $Q$)} can warm (positive $\Delta T$) a material.
\begin{equation}\label{eq:Q=mcDT}
Q = m c \, \Delta T
\end{equation}
but \autoref{eq:Q=mL} (as one example) shows that it is possible to heat (positive $Q$) a material without warming it (constant $T$). When we get to \autoref{s:PV} we will see other examples of ``isothermal processes'' that have a non-zero $Q$ (heat the system or heat the surroundings) without warming or cooling the system.

\section{Latent Heat}

Heating might also change the phase of a material.\mlinkreturn[heating versus warming]{d:heatwarm}
\begin{equation}\label{eq:Q=mL}
Q = \pm mL
\end{equation}

\section{The Flow of Thermal Energy}

\subsection{Thermal Conductivity}\label{ss:thermalconductivity}\mautoreturn{s:story}

\begin{equation}\label{eq:thermalconductivity}
\frac{Q}{\Delta t} = \kappa A \, \frac{\Delta T}{\Delta x}
\end{equation}

\begin{example}
\fcolorbox{black}{yellow!10}{\begin{minipage}{4.925in}
\caption{\label{ex:baking}\studentA\protect{\index{\studentA}} warms \hisA\ oven.}
\studentA\protect{\index{\studentA}} decides to bake some bread for the dinner party at \studentB\protect{\index{\studentB}}'s house, but \heA\ is on a tight schedule.  In order to set \hisA\ schedule, \heA\ needs to know how long it will take \hisA\ oven to \hyperref[find:heatwarm]{warm up}.

\autoreturn{s:story}
\end{minipage}}
\end{example}

\subsection{Convection}
\subsection{Radiation}

\chapter{Ideal Gas Law}
\section{$P$-$V$ Diagrams}\label{s:PV}\mlinkreturn[heating versus warming]{d:heatwarm}

\part{Let There be Light!}

\chapter{The Electrical Interaction}\label{c:electric}\mlinkreturn[fundamental forces]{d:fundamental}
\section{Electrical Charge}\label{s:Echarge}\new{v2.1}{Decide where this should go.}

\section{The Big Picture}

\subsection{Electric Forces and Fields}\label{ss:Efield}\mmultireturn{\mmr{\autoref{sss:vectorequations}}, \mmr{\hyperlink{d:f=ma}{$F=ma$}}}

pst-electricfield

\subsubsection{Examples}

\subsection{Electric Forces, Fields, and Potential Energy}

\subsection{Electric Fields, Potential Energy, and Potential}

\section{Making Connections}\label{s:Econnection}\mautoreturn{s:Gconnection}

\[ \begin{array}{ccccc}
& & \vec F = q \vec E & & \\
& \deq F = k \frac{q_1 q_2}{r^2} & \leftrightarrow & \deq E = k \frac{q}{r^2} & \\
\Delta \PE = -\vec F \cdot \Delta\vec x & \updownarrow & & \updownarrow & \Delta V = -\vec E \cdot \Delta\vec x  \\
& \deq \PE = k \frac{q_1 q_2}{r} & \leftrightarrow & \deq V = k \frac{q}{r} & \\
& & \Delta \PE = q \Delta V & &
\end{array} \]
(Recall the parallel with the gravitational interaction in \autoref{s:Gconnection}.)

\chapter{Electricity}

\chapter{The Magnetic Interaction}

pst-magneticfield

\chapter{``Magnicity?''}

\chapter{Light}

\chapter{Optics}

\part{What Have You Done for Me Lately?}

\chapter{Relativity}
\chapter{Quantum Mechanics}\new{v2.1}{Decide if these subsections should be chapters in and of themselves.  These are now labeled.}
\section{Atomic Physics} \subsection{The Periodic Table and Quantum Numbers}
\section{Nuclear Physics} \subsection{Nuclear Decay}\label{ss:nucleardecay}
\subsection{The Strong Nuclear Force}\label{ss:strong}\mlinkreturn[fundamental forces]{d:fundamental}
\subsection{The Weak Nuclear Force}\label{ss:weak}\mlinkreturn[fundamental forces]{d:fundamental}
\section{Particle Physics}\label{s:particle}
\subsection{Field Theory}
\subsection{Quantum Electrodynamics}\label{ss:QED}\mlinkreturn[fundamental forces]{d:fundamental}
\subsection{Quantum Chromodynamics}\label{ss:QCD}\mlinkreturn[fundamental forces]{d:fundamental}
\subsection{The Standard Model}\label{ss:StandardModel}
\subsection{Particle Decay}\label{ss:particledecay}
\chapter{Condensed Matter}
\chapter{Astronomy}
\chapter{Cosmology}

\part{Supplements}

\chapter{Deeper Dive}\label{c:revisted}\new{v2.1}{This chapter should mirror \protect{\autoref{c:physics}}.}

This is where I will put the fuller explanations.

\subsection{The Sun}\label{sss:sun}
The bright, shiny sun, which keeps us all alive, is a nice example of a rather complex system that allows us to verify our various theories of the world around us.  We can consider the existence of a star in three phases: the birth of a star, the life of the star, and the death of the star.

\subsubsection{The Birth of a Star}
\subsubsection{The Life of a Star}
\subsubsection{The Death of a Star}


\subsection{Kitchen Appliances}
\subsubsection{Oven}
\subsubsection{Refrigerator}
\subsubsection{Microwave}
\subsubsection{Television}

\subsection{Automobile}
\subsubsection{Coolant and Antifreeze}
\subsubsection{Tires}
\subsubsection{Torque}

\subsection{Cool Ideas}
\subsubsection{Black Holes}\label{sss:blackhole2}\mautoreturn{ss:weightmass}

On 7 April, 2017,\new{v2.3}{New source of info}
\href{http://www.cbc.ca/podcasting}{CBC Broadcasting} published a
\href{http://www.cbc.ca/podcasting/includes/quirks.xml}{\textit{Quirks and Quarks}} episode discussing how we can
\href{https://podcast-a.akamaihd.net/mp3/podcasts/quirks_20170408_51226.mp3}{turn our planet into a giant telescope to get a photo of a black hole}.
The results should be available by the early 2018.\dothis{Follow-up in 2018 to find the results.}

\subsubsection{Quantum Mechanics}
\subsubsection{Relativity}
\subsubsection{String Theory}



\chapter{Podcasts and Videos}\label{c:videos}\label{c:podcasts}

\section{Podcasts}\label{s:podcasts}
\hyperref{http://spacepod.libsyn.com/}{T4LTFdOxHD5WWzdD}{99}{Spacepod with Carrie Nugent} \\
\href{http://www.sciencefriday.com/}{Science Friday with Ira Flatow}

\section{Videos}\label{s:videos}
\href{http://physicsfootnotes.com/}{Physics Footnotes} \\
\href{http://sixtysymbols.com/}{Sixty Symbols}

\section{Websites}\label{s:websites}
\href{http://www.aldakavlilearningcenter.org/practice/flame-challenge}{The Flame Challenge}

\chapter{Answers to Interactive Questions}

\begin{AIQ}
\item\label{A:hbf} There are forces acting on it.  You should be able to tell this because you are exerting one of the forces. While it is true that there are forces on it, it is also true that there is no \textit{net force}.  If you are exerting an upward force on the book, can you guess (\ref{A:gravity}) what the downward force is?   \return{IQ:holdbook}
\item\label{A:chair1} If we refer to ``motion'' as describing the velocity, then no. Force causes a \textit{change in} velocity. When you stop pushing, the chair stops because there is a force from the carpet acting to oppose the force you apply while you push the chair. \autoreturn{irl:NI}
\item\label{A:chair2} This is essentially the same as \ref{A:chair1}, but the carpet exerts more force than the tile.  In either case, force causes a \textit{change in} velocity. You are trying to speed the chair up and the floor is trying to slow the chair down.  (Both are trying to change the velocity, but cancel to result in a constant velocity.)  When you stop pushing, the chair stops moving because there is a force from the tile acting to oppose the force you apply while you push the chair; when you let go, this force slows the chair until the chair stops and then the force stops acting. (See \autoref{s:Ff} for more details.) \autoreturn{irl:NI}
\item\label{A:weight.loss} Since \studentB\index{\studentB} weighs $(\massB)(9.81\unitfrac{m}{s^2})=736\unit{N}$, $45\unit{N}$ is about $6\%$ of her weight.  This is fairly substantial.  You should compute how much $6\%$ of your weight is and convert that to kilograms and Newtons.  \autoreturn{irl:scale}
\item \label{A:ladderNf} Since the full weight of the ladder, $F_g = \sig{222}{.69}{N}$, is still pressing downwards into the floor (as a normal force), it is tempting to say that \hyperref[ss:NIII]{Newton's third law} implies that the floor pushes the ladder upwards with a normal force of $\sig{222}{.69}{N}$ but this would not account for the frictional force on the wall, $F_{fw}$.  If there were no friction between the ladder and the wall, then we could deduce $F_{Nf}$, but at this point, we cannot. \autoreturn{ex:ladder2}
\item\label{A:hbnof}  It is true that while you hold the book, there is no \textit{net force}, but that does not mean that there is no force acting.  If there were no forces on the book, then your hand would not need to be there.  In fact, if you remove the force your hand provides, then the book falls. This shows that there is an upward force (by your hand on the book) and a downward force (of gravity by the Earth on the book).  \return{IQ:holdbook}
\item\label{A:netF-a} Since the object in \ref{se:netF-a} has a mass of $2.0\unit{kg}$, we can find the weight by
    \[ \vec F_g = m \vec g = (2.0\unit{kg}) [-(9.81\unitfrac{m}{s^2})\,\jhat] = -\sig{19}{.62}{N} \jhat = -20 \unit N \jhat \]
    \return{se:weightA}
\item\label{A:chair3} For a chair with wheels being pushed across a tile floor, when you stop pushing it probably continues to move across the floor for at least a short distance.  \autoreturn{irl:NI}
\item\label{A:weight.gain} When one person stands on the scale, the scale provides just enough of an upwards normal force to keep that person in equilibrium\dothis{link equilibrium?}.  In that case, the upwards force is balancing the weight of the person.  This gives the impression that the scale is telling you your weight; however, when you press down or help support whomever is standing on the scale, the scale adjusts the amount it must provide.  The scale is not trying to tell you your weight.  Rather the scale is trying to create equilibrium by balancing whatever force(s) are pressing into it.  Your weight is determined by the gravitational force\dothis{link the gravitational force?}{} and does not change when you press harder or lighter onto the scale.  \autoreturn{irl:scale}
\item\label{A:nowall} If we consider $\mu_w\rightarrow 0$, then $F_{fw}=0\unit N$,  $\vec F_{Nf} = -\vec F_g = \sig{222}{.7}{N} \jhat$, and $\vec F_{Nw} = - \vec F_{ff} = \sig{28.7}{9}{N} \ihat$.  In this case, $\mu_f$ could be as small as $0.\sig{129}{3}{}$ and still hold the ladder in place, unless \studentC\index{\studentC} climbs the ladder, in which case see \ref{A:nowallC}.  \autoreturn{ex:ladder2}
\item\label{A:true1} It is in equilibrium.  When the acceleration is zero, then the net force must be zero and those properties are what define equilibrium. \return{IQ:holdbook}
\item\label{A:chair4} The chair continues to move for the same reason that the chair without wheels and the chair on carpet \textit{all} continued to move when you let go.  The reason is that this is \textit{how all objects behave; they maintain their velocity when allowed to act without interference.}  (This is why Newton's first law says what it does.)   Because the chair with wheels has much less friction there is a smaller force trying to interfere with the motion and so it continues to move for a noticeable distance. The other chairs slowed to a stop almost immediately.  The wheel-less chair on tile might have continued for a short distance if it was moving fast enough that it required a long enough time to change its velocity to zero.  \autoreturn{irl:NI}
\item\label{A:scale.increase} When you press down on \studentB's shoulders, you are not adding weight.  Weight has a specific definition: it is specifically the value that the gravitational force\dothis{link the gravitational force}{} pulls on any object.  Pushing the person does not change their weight; it does, however, change the amount that they press into the Earth.  That is to say, it increases their downwards normal force, but not their weight.  \autoreturn{irl:scale}
\item\label{A:nowallC} If we consider $\mu_w\rightarrow 0$ with \studentC\index{\studentC} ($m=\massC$) at the third-rung-from-the-top of the ladder, ($1.53\unit m$ up the ladder), then $F_{fw}=0\unit N$,  $\vec F_{Nf} = \sig{1105}{.6}{N} \jhat$, and $\vec F_{Nw} = - \vec F_{ff} = \sig{171}{.97}{N} \ihat$.  In this case, $\mu_f$ could be as small as $0.\sig{155}{5}{}$ and still hold the ladder in place. \multireturn{\mmr{\ref{A:nowall}}, \mmr{\autoref{ex:ladder2}}}
\item\label{A:gworld} Because the Earth was spinning as it cooled (forming the crust), it formed an oblate spheroid\footnote{The equator is slightly further from the center than the poles are.}.  Since the strength of the gravitational interaction depends (among other things) on how far you are from the center (slightly weaker further away), the acceleration due to gravity is smaller when you are at smaller latitudes (closer to the equator).  \multireturn{\mmr{\ref{A:gpeaks}}, \mmr{\autoref{t:gworld}}}
\item\label{A:false1} The definition of equilibrium is that the forces balance.  The result of this is that the net force must be zero and the acceleration is then zero.  You can tell this is true because the velocity is \textit{not changing}.  It is not important that the velocity is zero, what is important is that the velocity \textit{stays} zero.  While you hold it, the book is in equilibrium. \return{IQ:holdbook}
\item\label{A:chair5} No. But if it does not matter what you do after you let go of the chair, then why do coaches (in basketball free-throws, tennis serves and swings, baseball pitches, and all manner of arm and leg propulsion) tell you to pay attention to your ``follow through''? \TWO{They have been fooled; follow-through doesn't matter}{they are right; follow-through does matter!}{A:noFT}{A:FT} \autoreturn{irl:NI}
\item\label{A:scale.measure} Since your weight is a force pulling downwards, having the scale on the wall shows that the scale cannot be balancing weight.  Since you are pushing into the wall, you are exerting a normal force into the scale and the scale is exerting a normal force back at you.  Both of these forces are horizontal (assuming the wall is plumb).  \autoreturn{irl:scale}
\item\label{A:gpeaks} In addition to being an oblate spheroid (\ref{A:gworld}), the Earth has mountains and valleys.  Since the strength of the gravitational interaction depends (among other things) on how far you are from the center (slightly weaker further away), the acceleration due to gravity is smaller when you are at at high altitudes, such as Denver, CO and Mount Everest.  \autoreturn{t:gworld}
\item\label{A:falls}  Both ``Yes'' and ``No'' bring you to this answer.  Yes, there is \underline{a force} on the book while it falls (the force of gravity), but no, there are not force\underline{s} (plural).  There is only one force.  ``But, wait!'' you say, ``What about the force of air resistance?''  Aha!   You are correct; there is a force of air resistance, but in this case, it is negligible and we will not consider it.  Please read \autoref{ss:airresistance} for more information about deciding when to use or ignore this phenomenon.  \return{IQ:holdbook}
\item\label{A:chair6} No.  When you throw a ball very high into the air, you can dance a jig or do any manner of things and it will obviously not affect the ball.  The force you exerted on the chair goes away the instant you stop touching the chair.  It is, however, true that your force gave the chair some velocity (actually \hyperref[c:momentum]{momentum}) and Newton's first law (inertia) says that the chair would prefer to keep that velocity.  Unfortunately, the friction with the ground slows it down.  The careful way to describe the situation is that your force gave the chair some velocity (actually \hyperref[c:momentum]{momentum}) and its characteristic inertia made it difficult for the \hyperref[s:Ff]{frictional force} to slow it down rapidly. \autoreturn{irl:NI}
\item\label{A:fly.balls} If you watch them carefully, you will notice that long fly balls are not parabolic.  It turns out that the air resistance is fairly complicated, but in the case of baseballs, the part that is relevant is that air resistance is strong when the ball is moving faster and weak when the ball is moving slower.  (This is different than the surface friction you will see in \autoref{s:Ff}.)  The effect of this is that the ball (usually) looks like it travels up into the air on a fairly straight path with a slight bend, which would produce a very wide parabola.  As it slows, the horizontal motion decreases, which tightens the parabola.  By the time the ball gets to its highest point, it is often travelling fairly slowly and has mostly all vertical motion by the time it drops into the outfielder's glove. \autoreturn{irl:nonparabolic}
\item\label{A:hitY} The book is accelerating.  The velocity \underline{is changing} from ``moving downwards'' to ``stopped''.  The book is not in equilibrium.  \return{IQ:holdbook}
\item\label{A:noFT} They haven't been fooled, but follow-through matters in a different way.  What does matter is not literally how you move \textit{after} the release, but rather how you move \textit{before} you release the ball. By paying attention to your follow-through, you are also changing the way you move before you release or impact the ball.  You want a smooth flow throughout the motion and a sloppy follow-through often implies a sloppy initiation of the motion.  \return{A:chair5}
\item\label{A:scale.ramp} When the scale is on the flat, horizontal floor, it balances your full weight.  When the scale is on the vertical wall it does not carry any of your weight.  At any angle in between those values, it carries some fraction of your weight while friction keeps you from sliding down the ramp\dothis{link to the section on friction and ramps}.  It will turn out that since the cosine function\dothis{link to the trig section}{} behaves in just the right way, we can use\dothis{link ``can use'' to the section on ramps}{} the cosine to find the component of the weight that the normal force from the scale has to support.   \autoreturn{irl:scale}
\item\label{A:pitches.side} The way a pitch travels is highly dependant on the way the pitcher releases the ball.  As the ball rolls out of the pitcher's hand, a spin is (usually) given to the ball and this spin interacts with the air to modify the direction that the air presses on the ball during the flight.  This will slightly affect the flight of the ball during the time it takes for the ball to get from the pitcher's mound to home plate.  In addition, fast balls have less time for the gravitational force to pull the ball down, so they will curve downwards less than a slower pitch.  This makes following the path of the ball somewhat difficult, but with some practice and careful attention, you should be able to see it.  All balls will drop somewhat, but the effect of the air resistance is exactly the mechanism for making a pitch unpredictable, so it is unlikely that you see the ball drop in a clean parabolic path.  \autoreturn{irl:nonparabolic}
\item\label{A:hitN} While it is hitting the desk, the velocity is changing from ``moving downwards'' to ``stopped''.  Since the velocity is changing, \underline{the book is accelerating}.  Since it is accelerating, the book is not in equilibrium.  \return{IQ:holdbook}
\item\label{A:chair7} The force that the chair feels after you release it is \hyperref[s:Ff]{friction}.  For the carpet, there is a lot of friction and the chair slows down very quickly (essentially instantaneously).  For the wheel-less chair on the tile floor, the chair slows rapidly although it may leave your hand.  The wheels provide the least amount of friction and that chair goes the furthest.  You may note that the friction slowing the chair-with-wheels is primarily between the rolling wheel and its axel (where it connects to the non-rolling chair leg) rather than between the wheel and the floor (although the friction between the wheel and the floor also plays a role).  This is discussed in more detail in \autoref{s:Ff}. \autoreturn{irl:NI}
\item\label{A:pitches.top} The way a pitch travels is highly dependant on the way the pitcher releases the ball.  As the ball rolls out of the pitcher's hand, a spin is (usually) given to the ball and this spin interacts with the air to modify the direction that the air presses on the ball during the flight.  In many cases, this will affect the flight of the ball (especially to the right or to the left) during the time it takes for the ball to get from the pitcher's mound to home plate.  If you watch from behind home plate, this sideways motion should be fairly clear.  \autoreturn{irl:nonparabolic}
\item\label{A:landedY} It is in equilibrium.  The book is at rest and \textit{continues to be} at rest on the desk. There are forces acting, but they cancel each other, resulting in no net force.  \return{IQ:holdbook}
\item\label{A:gravity}  It is the force of gravity. \return{A:hbf}
\item\label{A:chair8} If there were no friction, then you could start the chair and it would move on its own at a constant speed; you wouldn't need to continue pushing to keep it moving.  On the other hand, if you did continue to push, then the chair would continue to speed up and you would have to run faster and faster to keep up with it. On the other hand, if the chair were not experiencing friction, then you probably wouldn't either and you couldn't get enough traction to keep up with the chair, so it would sail away almost immediately, being then described by Newton's first law! \autoreturn{irl:NI}
\item\label{A:pool.roll} First, you should not roll a pool ball across just any floor; there is felt on the pool table for a reason.  However, if you have a clean, smooth surface and are able to reproduce your rolling speed, you will find that the pool ball rolls further on the stiff, nonyielding surface than it will on the felt.  The reason for this is beyond the scope of this textbook, but you can read more from \href{http://stacks.iop.org/0031-9120/30/i=3/a=009}{``Sliding and rolling: the physics of a rolling ball,'' J. Hierrezuelo and C. Carnero, Physics Education, Volume 30, Number 3} (unofficially at \href{http://billiards.colostate.edu/physics/Hierrezuelo_PhysEd_95_article.pdf}{this PDF}). \autoreturn{irl:poolcushion}
\item\label{A:landedN} After it has landed, the book stops moving.  Once the book comes to rest on the desk, it \textit{continues to stay at rest}.  This says that the velocity is not changing, so the book is not accelerating.  That means that the book is in equilibrium. There are forces acting, but they cancel each other, resulting in no net force. \return{IQ:holdbook}
\item\label{A:FT} What does matter is not literally how you move \textit{after} the release, but rather how you move \textit{before} you release the ball. By paying attention to your follow-through, you are also changing the way you move before you release or impact the ball.  You want a smooth flow throughout the motion and a sloppy follow-through often implies a sloppy initiation of the motion.  \return{A:chair5}
\item\label{A:pool.bumper} The cushion (sometimes called a bumper) is pretty still to the touch, but it is made of a springy rubber that allows the balls to bounce reasonably well.  The \protect{\href{http://c.ymcdn.com/sites/bca-pool.com/resource/resmgr/imported/BCAEquipmentSpecifications_2008.pdf}{document}} indicates that you should be able to firmly strike a ball at some angle to the far wall and have it bounce around the table four to four-and-a-half times.  If the bumpers were perfectly \hyperref[s:elastic]{elastic}, then the normal force would be normal to the restign surface; but since the bumper has some flexibility, when the ball hits the bumper with a glancing blow, then bumper bends inwards and the normal force is directed in a way that depends on the shape of the dent.  \autoreturn{irl:poolcushion}.
\item\label{A:zero} Recall the situation when you were holding the book.  Gravity is still pulling the book down and the desk is holding the book up.  There are two forces acting on the book while it is at rest on the desk. \return{IQ:holdbook}
\item\label{A:firstfall} As long as you are careful about releasing at the same time, you should not see any object consistently land first or consistently land last.  It is true that a piece of paper  will consistently land last, but this is because of the air resistance that we previously agreed to avoid.  If you crumple the paper into a tight ball (yes, it has to be a tight ball), then this will minimize the effect of air resistance and you might still be able to make the comparisons.   It is possible that some of the objects you are dropping (such as those in \ref{A:firstwhy}) have a shape that makes air resistance relevant.  \autoreturn{irl:freefall}
\item\label{A:floor}  I hope you guessed the floor.  That is the only thing pushing up on \studentB\index{\studentB}.  One useful way to think about it is that the floor is the thing keeping \himB\ from falling.  The direction of this force is \textit{normal} (perpendicular) to the horizontal floor, so it is in the vertical direction.  This will be discussed in more detail in \autoref{s:FN}. \return{se:FNB}
\item\label{A:firstwhy} As long as you are careful about releasing at the same time, it is unlikely that you will find anything consistently falling faster or slower than the others.  If you do notice a pattern, then the likely culprit is that air resistance is having an effect.  If you have something flat, like a computer (!) or a book that is falling more slowly than something else, like a hammer, then try dropping the flat object in different orientations to see if that affects the air resistance.  If you have something somewhat cylindrical, like a wine bottle (!) or a pencil that is falling more quickly than something else, like a hammer, then try dropping the cylindrical object in different orientations to see if that affects the air resistance. Remember that we are trying to eliminate differences due to air resistance so that we can study the effect of the gravitational force. \textbf{The effect you should notice is that so long as air resistance does not affect one object differently than the other, all objects fall at the same rate.}  \autoreturn{irl:freefall}
\item\label{A:noncue} The \href{http://wpapool.com/equipment-specifications/\#Balls-and-Ball-Rack}{specifications} show that there is no difference between the solids and stripes, but the cue ball weighs $9\%$ more that the other balls ($6.0\unit{oz}$ versus $5.5\unit{oz}$).  The colored balls and the cue ball are otherwise identical.  \autoreturn{irl:poolcushion}
\item\label{A:one} If there were only one force on the book, it could not be a balanced force, so the book could not be in equilibrium and the book would be accelerating.  The book is not accelerating, so there are either two forces (\ref{A:two}) or no forces (\ref{A:zero}).  \return{IQ:holdbook}
\item\label{A:fallv} When you drop something from eye-level, it takes less than a half-second for it to hit the ground.  Due to the limited need to gauge speed, it is very difficult for most humans to distinguish constant speed from accelerated motion in this small of a time interval.  Athletes can often tell is an object is moving fast or slow, but even then it is difficult to gauge acceleration.  Practice measuring the time-of-flight by counting out loud: ``one-one-thousand\ldots two-one-thousand\ldots''.  For this fall, you will likely only get to ``one-one-thou''. \autoreturn{irl:freefall}
\item\label{A:pool.spin} Because the bumper is covered in felt, it has a small grip on the ball.  Because the bumper has some give to it, it dents in when hit and provide more surface area, which increases the grip.  Both of these mean that the spin of the ball gets transferred to the pool table somewhat and change the way a spinning ball exits from the bumper collision.  \autoreturn{irl:poolcushion}
\item\label{A:second} You can tell that it is Newton's second law $(\vec F_\mathrm{net} = m \vec a)$ because the forces we are considering are acting on the \textit{same} object.  In this case, the gravitational force is caused by the Earth and the normal force is caused by the floor by they are both felt by \studentB\index{\studentB}.  These forces happen to be equal and opposite because \heB\ happens to be in equilibrium. \HeB\ does not \textit{have to be} in equilibrium, such as when \heB\ jumps, in which case the forces would not be equal and might not be opposite. \return{se:FNB}
\item\label{A:falla} Measure the time-of-flight by counting out loud: ``one-one-thousand\ldots two-one-thousand\ldots''.  For the four rungs near the top, you will likely only get to ``one-one-thou''.  For the four rungs near the bottom, you will likely only get to ``one-wa''.  Since those two distances are the same, it should be clear that the object is going faster at the bottom of the ladder. \textbf{Objects speed up (accelerate) while they fall.}  \autoreturn{irl:freefall}
\item\label{A:pool.later} This answer is getting \important{too complex for the section} it is in.  I need to move the IRL before I finish considering how to answer this question. \autoreturn{irl:poolcushion}
\item\label{A:two} There are two forces acting on the book while it is at rest on the desk. Similar to the situation when you were holding the book, gravity is pulling the book down and the desk is holding the book up.  \return{IQ:holdbook}
\item\label{A:third} If it were Newton's third law, then the two forces we were discussing would be acting on different objects and would be unrelated to the fact that the object (in this case, \studentB\index{\studentB}) is in equilibrium.  The gravitational force and the normal force in this case are both acting on \studentB, so although they happen to be equal and opposite, this is not due to Newton's third law.

    You should, however, note that the force that is reaction-paired to the gravitational force on \studentB\index{\studentB} by the Earth is a gravitational force on the Earth by \studentB.  Similarly, the reaction-paired force to the normal force on \studentB\ by the floor is a normal force on the floor by \studentB.  (Please note the ``on'' and ``by'' in each case.) \return{se:FNB}

\item\label{A:swing.tension} To make this comparison, let's consider a swing that is supported by chains.  If you are sitting in the swing and take hold of the chains at about shoulder height, you should be able to shake them in (towards your chest) and out (away from you, towards your neighbor swings).  You can do this same motion while standing next to the swing.  If you do this when the swing is empty, it is very easy to do this.  If you ask a series of successively larger people to sit in the swing, you will notice that it gets progressively more difficult to extend them very far.  The chains are increasing in tension; they are pulled more taut.  Your ability to move the chain in this way is exactly analogous to the way a bow draws across a violin or the way your fingers pluck a guitar, as described in \autoref{ss:stringed.instruments}. \multireturn{\mmr{\autoref{irl:tension}}, \mmr{\ref{A:chandelier.tension}}}
\item\label{A:fan.tension} You might also consider \ref{A:chandelier.tension}, which discusses the case of hanging a light fixture from the ceiling. If you have ever installed a fan in your house, then you will notice that you have to support the fan while the wires are connected.  Usually the fan has a shaft that connects to the ceiling at one end and the fan at the other and provides a mechanism for supporting the fan while you manage the wires, which pass through the shaft.  Since the fan houses the motor, it is usually reasonably heavy.   The nice property of use a metal shaft to support the fan is that it doesn't stretch or wiggle like a chain might.  The difficulty in this example is that it is more difficulty to notice the tension in the shaft.  If you are the person hanging the fan, then one thing you might be able to notice is that if you flick the metal with a finger when it is not supporting anything, it will have a slightly different ``ting'' than when it is supporting the fan.\dothis{Is this sufficiently noticeable?} \autoreturn{irl:tension}
\item\label{A:chandelier.tension} If you have ever installed a chandelier in your house, then you will notice that the light has to be supported between the joists of the ceiling.  There will be an electrical box with a screw to which you will attach the support for the chain that holds the chandelier.  The wires will run through the support chain.  The heavier the chandelier, the tauter the chain, much as described in \ref{A:swing.tension}. This tension is much easier to see than the tension in the shaft of the fan.  \return{A:fan.tension}
\end{AIQ}

\chapter{Adventures}

Throughout the book, there are examples and adventures.  The follow-up stories are contained below.
\begin{Story}
\item\label{a:parkandwalk} Just as planned, you pull over and park the car.  \studentB\index{\studentB} suggests one of you stays with the car, probably because \heB\ has physics homework to do.  If you decide to separate, read \ref{a:nogas}.  If you decide to journey together, read \ref{a:intosunset}.
\item\label{a:NIIIaction} As \studentC\index{\studentC} gets pushed, you notice that \heC\ was not aware of the pending doom.  \HeC\ is standing casually with \hisC\ feet set to support his own weight, but not to brace \himC\ against the sideways force.  When \heC\ gets pushed from the side, \hisC\ feet stay in place and \hisC\ torso topples, rotating \himC\ about \hisC\ center of mass\Foreshadow{The physics of why an object (or person) rotates when they fall over is discussed with \protect{\hyperref[c:torque]{torque}}.}{} as \heC\ falls to the ground. \studentD\ points to \hisD\ phone and says, ``I recorded the whole thing!''  If you respond, ``Awesome! Can I watch the part about how \studentZ\index{\studentZ} acts?'', please read \ref{a:NIIIreaction}.  If you respond, ``Awesome! Let's show the psychology and physics faculty our cool video!'', please read \ref{a:NIIIfaculty}. If you respond, ``Yeah, we probably should have intervened before this happened instead of just watching.  Let's go talk to Campus Security.'', please read \ref{a:NIIIsecurity}.
\item\label{a:coastindrive} As soon as you decide to do this, the gas runs out.  Thinking you can make it to the gas station, you take your foot off of the gas pedal.  You slow down fairly quickly and get nervous that you might get rear-ended.  You turn on the hazard-lights.  After about a minute you are travelling $30 \unit{mph}$ and you pass the time by working out \autoref{ex:slowcar}  (pg.~\pageref{ex:slowcar}).  People are honking at you as they try to pass.  \studentB\index{\studentB} turns to you and asks you why you are going so slow.  If you start a discussion about Newton's First Law, then go to \ref{a:NIdrive}.  If you get embarrassed and decide to pull over, then read \ref{a:parkandwalk2}.
\item\label{a:NIIIreaction} As \studentZ\index{\studentZ} pushes, you notice that because \heZ\ was being intentional, \heZ\ put one foot behind \himZ\ to brace \hisZ\ body during the push.  \HeZ\ leans into the push and stays standing.  You are intrigued.  If you decide to do a follow-up experiment by pushing \studentD\index{\studentD} over without bracing yourself, then read \ref{a:NIIIexperiment}.  If you decide to exercise self-restraint, then read \ref{a:NIIIrestraint}.
\item\label{a:coastinneutral} You speed up to $60 \unit{mph}$ before the gas runs out and then you quickly pop the car into neutral.  You slow down gradually and, in an effort to not get rear-ended, you cleverly turn on the hazard-lights.  After about a minute you are travelling $52 \unit{mph}$ and you pass the time by working out \autoref{ex:coasting}  (pg.~\pageref{ex:coasting}).  After $2\unit{min}$, you are travelling $43 \unit{mph}$ and people are getting impatient as they try to pass.  After $2.79\unit{min}$, you triumphantly coast into the gas station at a comfortable speed of $36.7\unit{mph}$.  \studentB\index{\studentB} is so happy, \heB\ buys you a full tank of gas and the two of you start a discussion about Newton's First Law while pumping the gas.  Please read \ref{a:NIresult}.
\item\label{a:NIIIconcern} Being the thoughtful and considerate person you are, you rush over and startle \studentC\index{\studentC} out of \hisC\ reverie.  \studentZ\index{\studentZ} is quite angry and now focuses \hisZ\ attention on you!  \HeZ\ rushes towards you and shoves as hard as he can.  You go \textit{flying} backwards and land on your tailbone while he just stands there laughing.  \studentC\ and \studentD\index{\studentD} both rush over to help you while \studentZ\ wanders off.  Surprisingly, \studentC\ has an icepack, which helps.  If you go speak to your faculty members about this, please read \ref{a:NIIIfaculty}. If you decide to talk to Campus Security, please read \ref{a:NIIIsecurity}.
\item\label{a:NIdrive} After some discussion, you and \studentB\index{\studentB} realize that when the car is in drive, the transmission (the part of the car that converts how-fast-the-engine-spins to how-fast-the-axel-and-wheels-turn) is connected to the axel, which means that the rolling wheels are trying to turn the engine parts as well as the wheels themselves.  The engine parts have grease and oil, but still take a lot of energy to turn.  This causes friction, which dissipates energy and, more importantly, exerts a backwards force on the spinning wheels.  Your car is not being described by Newton's First Law, which requires there to be no force applied.  Instead your car is being described by Newton's Second Law and the force is changing the velocity to cause you to go slower.  It only takes the car $2.2\unit{min}$ to stop and you still have to walk to the gas station.  \studentB\ laments ``If only there were a way to reduce the force on the axel\ldots'' If it occurs to you to speculate about putting the car in neutral when the gas ran out, then imagine reading \ref{a:coastinneutral}.  If you stop talking and walk to the gas station, then read \ref{a:intosunset2}.
\item\label{a:NIIIexperiment} You turn and push \studentD\index{\studentD} over.  Like \studentC\index{\studentC}, \heD\ did not expect it and was not braced, so \heD\ falls over. Similarly, you decided not to brace yourself and in pushing \studentD, you fall over backwards!  \studentD\ did not have to \textit{choose} to push on you.  The act of you deciding to push \himD\ necessarily and simultaneously produces a force on you, equal in magnitude and opposite in direction.  Unfortunately, \studentD\ doesn't think this was a useful exercise and shouts ``I have the whole thing on video!'' and storms off to Campus Security.  You are arrested for assault, miss your physics class for a couple of weeks and ultimately fail all of your classes. I certainly hope this was all happening in your head and not in real life!  You learned something about physics, but at what cost to your humanity?  \hyperref[cyoa:NIII]{The end!}
\item\label{a:nogas} You leave \studentB\index{\studentB} in the car and walk the 45 minutes to the gas station.  You buy a gas can, fill it up, and start to carry it back to the car.  It is very heavy and you notice vultures circling overhead.  You hope you survive this.  It might have been a better idea to bring \studentB\ with you to share the burden.  You stumble once, and then again.  You swear to be more cautious about estimating your gas consumption.  After walking for what seems like hours and stumbling back to the car, you find \studentB\ very excited. \HeB\ declares that \heB\ has invented a time machine so you can go back to the \hyperref[cyoa:NI]{adventure} and start over to learn something about Newton's First Law!
\item\label{a:NIIIrestraint} \studentD\index{\studentD} points to \hisD\ phone and says, ``I recorded the whole thing!''  If you respond, ``Awesome! Can I watch the part about how \studentC\index{\studentC} acts?'', please read \ref{a:NIIIaction}.  If you respond, ``Awesome! Let's show the psychology and physics faculty our cool video!'', please read \ref{a:NIIIfaculty}. If you respond, ``Yeah, we probably should have intervened before this happened instead of just watching.  Let's go talk to Campus Security'', please read \ref{a:NIIIsecurity}.
\item\label{a:parkandwalk2} You pull over and park the car.  \studentB\index{\studentB} suggests one of you stays with the car, probably because \heB\ has physics homework to do.  If you decide to separate, read \ref{a:nogas2}.  If you decide to journey together, read \ref{a:intosunset}.
\item\label{a:NIIIfaculty} The psychology faculty member speaks to you both about how to be good citizens and about the psychological effects of bullies both on the bully and on the recipient.  If you decide to learn more about this, please read \href{https://www.psychologytoday.com/basics/bullying}{Psychology Today}.  The physics faculty member points out that when one person pushes another, the person being pushed does not brace \himselfC, whereas the person doing the pushing does.  Furthermore one might imagine what would happen if you did not brace yourself when you pushed each other, such as in \ref{a:NIIIexperiment}.  You are asked to review both \autoref{ex:braced}  (pg.~\pageref{ex:braced}) and \autoref{ex:unbraced}  (pg.~\pageref{ex:unbraced}) before the next exam.  On your way out the door, you hear a voice suggest ``\ldots and you \textit{might} want to talk to \hyperref[a:NIIIsecurity]{Campus Security} about the incident\ldots''
\item\label{a:intosunset} Everything goes as planned.  You drive off into the sunset sadly ignorant of the physics you might have learned. \hyperref[cyoa:NI]{The end!}
\item\label{a:NIIIsecurity} You speak with Campus Security about the incident and \studentZ\index{\studentZ} gets taken in for assault.  The Dean thanks you for being brave enough to speak up. \hyperref[cyoa:NIII]{The end!}
\item\label{a:NIresult} During the discussion, you and \studentB\index{\studentB} realize that the rolling wheels and the spinning axel are still connected to the not-spinning frame of the car.  While this causes less friction than if the car were in drive, there is still some friction, which dissipates energy and, more importantly, exerts a backwards force on the spinning wheels.  Your car is not being described by Newton's First Law, which requires there to be no force applied.  Instead your car is being described by Newton's Second Law and the force is changing the velocity to cause you to go slower.  You finish getting gas and drive on to many happy adventures. \hyperref[cyoa:NI]{The end!}
\item\label{a:guilty} You feel guilty for letting \studentZ\index{\studentZ} push \studentC\index{\studentC} down despite your amazing score on the next physics test.  It wasn't worth it.  \hyperref[cyoa:NIII]{The end!}
\item\label{a:nogas2} You leave \studentB\index{\studentB} in the car and walk the 31 minutes to the gas station.  You buy a gas can, fill it up, and start to carry it back to the car.  It is very heavy and you notice vultures circling overhead.  You hope you survive this.  It might have been a better idea to bring \studentB\ with you to share the burden.  You stumble once, and then again.  You swear to be more cautious about estimating your gas consumption.  After walking for what seems like hours and stumbling back to the car, you find \studentB\ very excited. \HeB\ declares that \heB\ has invented a time machine so you can go back to the \hyperref[cyoa:NI]{adventure} and start over to learn something about Newton's First Law!
\item\label{a:intosunset2} You and \studentB\index{\studentB} happily walk the 25 minutes to the gas station, discussing and working out physics problems the whole way.  You buy a gas can, fill it up, and share the burden of carrying a heavy gas can.  You return to the car, add gas, and drive on to many happy adventures.  \hyperref[cyoa:NI]{The end!}
\end{Story}

%%%%%%%%%%%%%%%%%%%%%%%%%%%%%%%%%%%%%%%%%%%%%%%%%%%%%%%%%%%%%%%%%%%%%%%%%%%%%%%%%%%%%%%%%%%%%%%%%%%%%%

\chapter{Characters}

This textbook has five characters who follow you throughout the book.  They appear in the examples and some homework problems.  They also remember previous experiences.  I need to adjust the examples in \autoref{c:force} such that the people pushing boxes are helping the reader rearrange furniture.

The index lists\dothis{The index will recognize the people in two different formats.  One is by my name for them, which is $\backslash$studentX (where X is A, B, C, D, \ldots Z).  The other is by the name assigned to that variable.  So these show up in different places in the Index.}{} the pages that the characters appear.  The point of this chapter is to highlight some of the primary adventures of the characters according to their own perspectives.  \textbf{None of the links in this chapter will be given a corresponding return link.}  This chapter is for me to track relationships and will likely go away when the book is ready for publication.
%
I can, at the header of the code, define the name, gender, mass, and dimensions of each individual.\dothis[inline]{\href{http://malveyauthor.com/}{Madeline Alvey}, the author of  \protect{\href{http://escapepod.org/2017/03/09/ep566-honey-and-bone-artemis-rising-3/}{``Honey and Bone'' at EscapePod}} is a physics and English undergraduate student at UK in Lexington.  I might consider hiring(?) her to help storyboard the characters.}

\section{\studentA\index{\studentA!inside}}\index{\studentA!outside}\dothis{The index-call that is \textbf{outside} of the section title registers as $\backslash$studentA, which puts the name alphabetically under $\backslash$studentA, rather than \studentA.  The index-call that is \textbf{inside} of the section title registers as \studentA, which puts the name alphabetically under \studentA, rather than $\backslash$studentA.}
\index{\studentA|(} % Begin page-range
\begin{itemize}
\item In \autoref{s:forcewords}, \studentB{} gives \studentA{} a good-natured shove in the arm in order to get the language clarified and begin the conversation about the on-by notation.
\item In \ref{se:FBD-AB} \studentA{} helps \studentB{}\ldots
    \begin{itemize}
    \item (in the current version) push an object to make it accelerate and feel a reaction force causing \himA{} to accelerate backwards.
    \item (in the future version) will help the reader move into or out of their residence hall by pushing on heavier furniture.
    \item[NOTE:] This is all drawn in \autoref{f:firstFBD}, which is updated in \autoref{f:firstFBDupdate}.
    \end{itemize}
\item In \ref{se:weightA}, \studentA{} falls from a small height.  (maybe he is jumping off a short ledge while taking a short-cut to class?)
\item In \autoref{ex:baking}, \studentA{} decides to bake some bread for a party at \studentB's house, measuring the time it takes to warm his oven.
\end{itemize}

\index{\studentA|)} % end page-range
\section{\studentB\index{\studentB}}
\index{\studentB|(} % Begin page-range

\begin{itemize}
\item \studentB{} is a passenger in the reader's car in \autoref{ex:slowcar} when the reader runs out of gas and coasts to a stop.
\item \studentB{} is a passenger in the reader's car in \autoref{ex:coasting} and speculates about how fast to go before putting the car in neutral to coast to a stop.
\item \studentB{} joins the reader on a road trip in \autoref{cyoa:NI} and runs out of gas.  This results in multiple possible adventures:
\begin{itemize}
    \item \ref{a:parkandwalk}, which leads to either an end at \ref{a:nogas} or an end at \ref{a:intosunset}.
    \item \ref{a:coastindrive}, which leads to either \ref{a:NIdrive} (choose \ref{a:coastinneutral} or end with \ref{a:intosunset2}) or \ref{a:parkandwalk2} (choose \ref{a:intosunset} or end at \ref{a:nogas2})
    \item \ref{a:coastinneutral}, which leads to an end at \ref{a:NIresult}.
\end{itemize}
\item In \autoref{s:forcewords}, \studentB{} gives \studentA{} a good-natured shove in the arm in order to get the language clarified and begin the conversation about the on-by notation.
\item In \autoref{se:FBD-AB}, \studentB{} helps \studentA{}\ldots
    \begin{itemize}
    \item (in the current version) pull an object to make it accelerate and feel a reaction force causing \himB{} to accelerate backwards.
    \item (in the future version) will help the reader move into or out of their residence hall by pushing on heavier furniture.
    \item[NOTE:] This is all drawn in \autoref{f:firstFBD}, which is updated in \autoref{f:firstFBDupdate}.
    \end{itemize}
\item In \ref{se:FNB}, \studentB{} has a normal force supporting \himB.  (This touches \ref{A:floor}, \ref{A:second}, and \ref{A:third}.)
\item At some point, \studentB{} has a party, because in \autoref{ex:baking}, \studentA{} decides to bake some bread for a party at \studentB's house.
\end{itemize}

\index{\studentB|)} % end page-range
\section{\studentC}
\index{\studentC|(} % Begin page-range

\index{\studentC|)} % end page-range
\section{\studentD}
\index{\studentD|(} % Begin page-range

\index{\studentD|)} % end page-range
\section{\studentE}
\index{\studentE|(} % Begin page-range

\index{\studentE|)} % end page-range
\section{\studentF}
\index{\studentF|(} % Begin page-range

\index{\studentF|)} % end page-range
\section{\studentZ}
\index{\studentZ|(} % Begin page-range

\index{\studentZ|)} % end page-range



%%%%%%%%%%%%%%%%%%%%%%%%%%%%%%%%%%%%%%%%%%%%%%%%%%%%%%%%%%%%%%%%%%%%%%%%%%%%%%%%%%%%%%%%%%%%%%%%%%%%%%

\addcontentsline{toc}{chapter}{Index}
%\printindex
\documentclass[11pt,letter,openany,makeidx]{book}
\usepackage{amsmath}
\usepackage{macros}
\usepackage{comment}
\usepackage{graphicx}
\usepackage{microtype}
\usepackage{gfsdidot}
\usepackage[T1]{fontenc}
\usepackage{booktabs}
\usepackage{underscore,cancel}
\usepackage{caption}
\usepackage[within=chapter,chapterlistsgap=6pt]{newfloat}
\usepackage{tocloft}
\usepackage{xpicture}
\usepackage{xcolor}
%\usepackage[dvips]{xcolor}
%\GetGinDriver  % for xcolor to work well with hyperref
%\usepackage[\GinDriver]{hyperref}
\usepackage{ulem}%for \sout (done)
\usepackage[colorinlistoftodos]{todonotes}
%\usepackage[disable,colorinlistoftodos]{todonotes}
%\usepackage{layouts}
%\usepackage{showframe}
\usepackage{coordsys}
\usepackage{tikz}
\usepackage[pdftex]{hyperref}

%\usepackage{cellpage}

\hypersetup{colorlinks=true,bookmarks=true,pdftitle=Algebra-Based Introductory Physics,pdfauthor=J.Christensen,pdfdisplaydoctitle}
% If using a bibliography, then include "backref" in list of \hypersetup items
% linkcolor=color of internal links (red); anchorcolor = color of anchor text (black); citecolor = bibliographic citations (green); filecolor = color for local URL files (cyan); menucolor = Acrobat menu (red); urlcolor = external links (magenta); hidelinks = remove all color
% citebordercolor = color of box for citations (0 1 0); fileborder = links to files box (0 .5 .5); linkbordercolor = normal links (1 0 0); menuborder; urlborder; allbordercolors; pdfborder

\includecomment{ForMe}
\includecomment{ForReviewer}
\includecomment{ForPublic}

\makeindex

\newlistof{example}{loe}{List of Examples}
\DeclareFloatingEnvironment[fileext=loe,listname="List of Examples",name=Example]{example}
\setlength{\cftexamplenumwidth}{1cm}
\newcounter{sample}
\newcounter{carrysample}
\renewcommand{\thesample}{Simple Example \arabic{sample}}
\renewcommand{\thecarrysample}{Simple Example \arabic{carrysample}}
\newenvironment{sample}{\color{rgb:red,0;green,2;blue,1}\begin{list}{\textbf{\thesample}:}{\usecounter{sample} \setcounter{sample}{\value{carrysample}} \leftmargin 12pt}}{\end{list}\setcounter{carrysample}{\value{sample}}}
\newcommand{\THREE}[6]{\vspace{-3pt}\begin{flushright} Select one:  \mbox{#1 (\ref{#4})},  \mbox{#2 (\ref{#5})}, or \mbox{#3 (\ref{#6})}.\end{flushright}}
\newcommand{\TWO}[4]{\begin{flushright} Select one:  \mbox{#1 (\ref{#3})} or \mbox{#2 (\ref{#4})}.\end{flushright}}
\newcommand{\YN}[2]{\TWO{Yes}{No}{#1}{#2}}
\newcommand{\TF}[2]{\TWO{True}{False}{#1}{#2}}
\newcommand{\return}[1]{{} \hfill \mbox{Return to \ref{#1}.}}
\newcommand{\autoreturn}[1]{{} \hfill \mbox{Return to \autoref{#1}.}}
\newcommand{\linkreturn}[2][a related idea]{{}\hfill \mbox{Return to the discussion of \protect{\hyperlink{#2}{#1}}.}}
\newcommand{\mmr}[1]{\mbox{[\protect{#1}]}}
\newcommand{\multireturn}[1]{{}\hfill Return to one of the following locations: \newline #1.}
\newcounter{AtIQ}
\renewcommand{\theAtIQ}{Answer \arabic{AtIQ}}
\newenvironment{AIQ}{\begin{list}{\textbf{Interactive \theAtIQ}:}{\usecounter{AtIQ} \leftmargin 12pt}}{\end{list}}

% Related (return), but not part of...
\newcommand{\mreturn}[1]{\note{Return to \protect{\ref{#1}}.}}
\newcommand{\mlinkreturn}[2][a related idea]{\note{Return to the discussion of \protect{\hyperlink{#2}{#1}}.}}
\newcommand{\mautoreturn}[1]{\note{Return to \protect{\autoref{#1}}.}}
\newcommand{\mmultireturn}[1]{\note{Return to one of the following locations: \newline #1.}}


\newlistof{adventure}{loa}{List of Adventures}
\DeclareFloatingEnvironment[fileext=loa,listname="List of Adventures",name=Adventure]{adventure}
\setlength{\cftadventurenumwidth}{1cm}
\newcounter{CYOA}
\renewcommand{\theCYOA}{Plan \Alph{CYOA}}
\newenvironment{CYOA}{\begin{list}{\textbf{\theCYOA}:}{\usecounter{CYOA}}}{\end{list}}
\newcounter{storyline}
\renewcommand{\thestoryline}{Storyline \arabic{storyline}}
\newenvironment{Story}{\begin{list}{\textbf{\thestoryline}:}{\usecounter{storyline} \leftmargin 12pt}}{\end{list}}

\newlistof{reallife}{irl}{List of Real Life Patterns}
%\DeclareFloatingEnvironment[fileext=irl,listname="List of Real Life Patterns",chapterlistsgaps=off,name=Real Life Patterns]{reallife}
\DeclareFloatingEnvironment[fileext=irl,listname="List of Real Life Patterns",name=Real Life Patterns]{reallife}
\setlength{\cftreallifenumwidth}{1cm}
\newcounter{IRL}
%\renewcommand{\theIRL}{\arabic{IRL}}
\newenvironment{realtable}{%\renewcommand{\arraystretch}{2}
                           %\hspace{-.2in}
                            \begin{tabular}{@{}lll@{}} \toprule Do This & Notice This & Ask This  \\ }
                            {\bottomrule \end{tabular} }%\renewcommand{\arraystretch}{.5}}
\newcommand{\dna}[3]{\midrule \begin{minipage}{4cm}\raggedright #1 \end{minipage}
                   & \begin{minipage}{4cm}\raggedright #2 \end{minipage}
                   & \begin{minipage}{4cm}\raggedright #3 \end{minipage} \\ }
\newcommand{\multidna}[1]{\multicolumn{3}{|c|}{\begin{minipage}{13cm}\center #1 \end{minipage}} \\ \midrule }


\newlistof{story}{los}{The Stories of the Equations}
\DeclareFloatingEnvironment[fileext=los,listname="The Stories of the Equations",name=This Equation's Story]{story}
\setlength{\cftstorynumwidth}{1cm}
\newcommand{\thestoryof}[1]{\marginpar{\raggedright \footnotesize The story of \\ \fcolorbox{black}{yellow}{\begin{minipage}[c]{1.5in} \center $\deq #1$ \end{minipage}}}}
\newcommand{\EqStory}[2]{\left[ {\color{rgb:red,1;green,1;blue,4} \begin{minipage}{#1}\raggedright\begin{center} #2 \end{center}\end{minipage}} \right]}
\newcommand{\EqStoryOver}[3]{\overbrace{\EqStory{#1}{#2}}^{\displaystyle #3}}
\newcommand{\EqStoryUnder}[3]{\underbrace{\EqStory{#1}{#2}}_{\displaystyle #3}}
\newcommand{\EqStoryFrac}[5]{\frac{\overbrace{\EqStory{#1}{#2}}^{\displaystyle #3}}
                                 {\underbrace{\EqStory{#1}{#4}}_{\displaystyle #5}}}


%%%%%%%%%%%%%%%%%%%%%%%%%%%%%%%%%%%%%%%%%%%%%%%%%%%%%%%%%%%%
%
%\presetkeys{todonotes}{fancyline,color=blue!15}{}
\presetkeys{todonotes}{color=blue!15,linecolor=blue!75,size=\footnotesize}{}
%
\newcounter{todocounter}
\newcommand{\dothis}[2][]
{\stepcounter{todocounter}\todo[color=green!30, #1]{\thetodocounter: #2}}
\newcommand{\docaption}[3][]
{\stepcounter{todocounter}\todo[color=green!30, prepend, caption={\thetodocounter: \underline{#2}}, #1]{#3}}
\newcommand{\addlink}[2][]
{\stepcounter{todocounter}\todo[prepend, caption={\thetodocounter: \underline{Add Link}}, #1]{#2}}
\newcounter{todourgentcounter}
\newcommand{\urgent}[2][]
{\stepcounter{todourgentcounter}\todo[color=orange!50, #1]{\thetodourgentcounter: #2}}
\newcommand{\urgcap}[3][]
{\stepcounter{todourgentcounter}\todo[color=orange!50, prepend, caption={\thetodourgentcounter: \underline{#2}}, #1]{#3}}
\newcommand{\done}[2][]
{\todo[color=yellow!10, #1]{\sout{#2}}}
%
%\newcommand{\new}[2]{}%
\newcommand{\new}[2]{\marginpar{\raggedright \footnotesize New to #1 \\ \fcolorbox{blue}{yellow!10}{\begin{minipage}[c]{1.5in} \center {\color{blue} #2 } \end{minipage}}}}%
%%%%%%%%%%%%%%%%%%%%%%%%%%%%%%%%%%%%%%%%%%%%%%%%%%%%%%%%%%%%


%%%%%%%%%%%%%%%%%%%%%%%%%%%%%%%%%%%%%%%%%%%%%%%%%%%%%%%%%%%%
%
%\newcommand{\deq}{\displaystyle}
%\newcommand{\txtfrac}[2]{{}^{#1}\!/_{\!#2}}
%
%%%%%%%%%%%%%%%%%%%%%%%%%%%%%%%%%%%%%%%%%%%%%%%%%%%%%%%%%%%%



%%%%%%%%%%%%%%%%%%%%%%%%%%%%%%%%%%%%%%%%%%%%%%%%%%%%%%%%%%%%
%
% PEOPLE AND PRONOUNS
%
% According to https://www.cdc.gov/nchs/fastats/body-measurements.htm
% Measured average height, weight, and waist circumference for adults ages 20 years and over
% Men:
% Height (inches): 69.3                 = 1.760 m
% Weight (pounds): 195.5                = 88.86 kg
% Waist circumference (inches): 39.7    = 1.01 m
% Women:
% Height (inches): 63.8                 = 1.621 m
% Weight (pounds): 166.2                = 75.55 kg
% Waist circumference (inches): 37.5    = 0.9525 m
% Source: Anthropometric Reference Data for Children and Adults: United States, 2007-2010, tables 4, 6, 10, 12, 19, 20[PDF - 1.7 MB]
%  https://www.cdc.gov/nchs/data/series/sr_11/sr11_252.pdf
%
\newcommand{\studentA}{Abdul}       \newcommand{\massA}{\mbox{$85.0\unit{kg}$}}
\newcommand{\studentB}{Beth}        \newcommand{\massB}{\mbox{$75.0\unit{kg}$}}
\newcommand{\studentC}{Carl}        \newcommand{\massC}{\mbox{$90.0\unit{kg}$}}
\newcommand{\studentD}{Diane}       \newcommand{\massD}{\mbox{$80.0\unit{kg}$}}
\newcommand{\studentE}{Erik}        \newcommand{\massE}{\mbox{$95.0\unit{kg}$}}
\newcommand{\studentF}{Frances}       \newcommand{\massF}{\mbox{$85.0\unit{kg}$}}
\newcommand{\studentX}{Xerxes}       \newcommand{\massX}{\mbox{$62.5\unit{kg}$}}
\newcommand{\studentZ}{Zambert}     \newcommand{\massZ}{\mbox{$95.0\unit{kg}$}}
% Male
\newcommand{\heA}{he}\newcommand{\himA}{him}\newcommand{\hisA}{his}\newcommand{\himselfA}{himself}
\newcommand{\HeA}{He}\newcommand{\HimA}{Him}\newcommand{\HisA}{His}
\newcommand{\heC}{he}\newcommand{\himC}{him}\newcommand{\hisC}{his}\newcommand{\himselfC}{himself}
\newcommand{\HeC}{He}\newcommand{\HimC}{Him}\newcommand{\HisC}{His}
\newcommand{\heE}{he}\newcommand{\himE}{him}\newcommand{\hisE}{his}\newcommand{\himselfE}{himself}
\newcommand{\HeE}{He}\newcommand{\HimE}{Him}\newcommand{\HisE}{His}
\newcommand{\heZ}{he}\newcommand{\himZ}{him}\newcommand{\hisZ}{his}\newcommand{\himselfZ}{himself}
\newcommand{\HeZ}{He}\newcommand{\HimZ}{Him}\newcommand{\HisZ}{His}
% Female
\newcommand{\heB}{she}\newcommand{\himB}{her}\newcommand{\hisB}{her}\newcommand{\himselfB}{herself}
\newcommand{\HeB}{She}\newcommand{\HimB}{Her}\newcommand{\HisB}{Her}
\newcommand{\heD}{she}\newcommand{\himD}{her}\newcommand{\hisD}{her}\newcommand{\himselfD}{herself}
\newcommand{\HeD}{She}\newcommand{\HimD}{Her}\newcommand{\HisD}{Her}
\newcommand{\heF}{she}\newcommand{\himF}{her}\newcommand{\hisF}{her}\newcommand{\himselfF}{herself}
\newcommand{\HeF}{She}\newcommand{\HimF}{Her}\newcommand{\HisF}{Her}
%
\newcommand{\heX}{\studentX}\newcommand{\himX}{\studentX}\newcommand{\hisX}{\studentX's}\newcommand{\himselfX}{the person of \studentX}
\newcommand{\HeX}{\studentX}\newcommand{\HimX}{\studentX}\newcommand{\HisX}{\studentX's}
%%%%%%%%%%%%%%%%%%%%%%%%%%%%%%%%%%%%%%%%%%%%%%%%%%%%%%%%%%%%%


%%%%%%%%%%%%%%%%%%%%%%%%%%%%%%%%%%%%%%%%%%%%%%%%%%%%%%%%%%%%
%
% Book macros
%
\newcommand{\aside}[2]{\marginpar{\raggedright \footnotesize\textbf{#1}: #2}}
\newcommand{\important}[1]{\\ \fcolorbox{black}{yellow}{\begin{minipage}[c]{4.925in} \center #1 \end{minipage}}\\}
\newcommand{\inlife}{\marginpar[\scriptsize \raggedright How you might observe $\Rightarrow$ this in your life.]
                               {\scriptsize \raggedleft $\Leftarrow$ How you might observe this in your life.}}
\newcommand{\touchstone}{\marginpar[\scriptsize \raggedright Where have I seen this $\Rightarrow$ before?]
                                   {\scriptsize \raggedleft $\Leftarrow$ Where have I seen this before?}}
\newcommand{\foreshadow}{\marginpar[\scriptsize \raggedright When will I ever use this? $\Rightarrow$]
                                   {\scriptsize \raggedleft $\Leftarrow$ When will I ever use this?}}
\newcommand{\foreshadowR}{\reversemarginpar
                          \marginpar[\scriptsize \raggedright When will I ever use this? $\Rightarrow$]
                                    {\scriptsize \raggedleft $\Leftarrow$ When will I ever use this?}}
\newcommand{\Touchstone}[1]{\marginpar[\scriptsize \raggedright Where have I seen this $\Rightarrow$ \\ before? #1]
                                      {\scriptsize \raggedleft $\Leftarrow$ Where have I seen this before? #1}}
\newcommand{\Foreshadow}[1]{\marginpar[\scriptsize \raggedright When will I ever use this? $\Rightarrow$ \\ #1]
                                      {\scriptsize \raggedleft $\Leftarrow$ When will I ever use this? #1}}
%
%%%%%%%%%%%%%%%%%%%%%%%%%%%%%%%%%%%%%%%%%%%%%%%%%%%%%%%%%%%%


\begin{document}

%\title{Algebra-Based Introductory Physics}
%\author{J Christensen}
%\date{Jan 2017}
%\maketitle
%\pagestyle{cellpage}

\begin{titlepage}
	\centering
%	\includegraphics[width=0.15\textwidth]{example-image-1x1}\par\vspace{1cm}
	{\Huge\bfseries Physics Connected\par}
	\vspace{1cm}
	{\Large\bfseries An Algebra-Based Introductory Physics Textbook\par}
	\vspace{1cm}
	{\large Learn like you think: an interconnected view of physics\par}
	\vspace{2cm}
	{\Large\itshape by: J Christensen\par}
	\vfill
\begin{ForReviewer}
	Version 2.3\par
	{\footnotesize
    \begin{itemize}
    \item Ideas yet to implement:
        \begin{itemize}
        \item The examples are phrased as descriptions, not examples like the homework problems.  Need to consider rephrasing these, not calling them examples, or adding actual examples that better show how to respond to the way homework problems are written.
        \item Define a different page dimension that fits on a cell phone display.  (Enhance possible cell-phone reading.)
        \end{itemize}
    \item version 2.3: June 16-28, 2017
        \begin{itemize}
        \item Updated Section 81. $F=mg$ and Section 8.2 Normal Force
        \item Added specific list of Flame Challenges
        \item Rearranged some of the subsections in the ``Seeing Physics'', added references
        \item Equations of motion for constant acceleration (Need the Story Of)
        \item Added a section to Chap 5 (1-D motion) that gives examples of solutions that require multiple steps  (one equation is insufficient)
        \item Developed the weight and mass discussion and examples
        \item Ladder leaning example in torque, plus some homework problems
        \item Added some Conceptual Homework to weight/mass
        \item Added placeholders to the Gravity chapter
        \item Removed indicators of v1.7 changes
        \end{itemize}
    \item version 2.2: June 16, 2017
        \begin{itemize}
        \item Created conversation about $F=mg$ for Chapter on types of forces.  Caused modifications in lots of places
            \begin{itemize}
            \item Added freefall to the motion chapter
            \item Created IRL and Example dropping objects to see acceleration in $F=mg$, then moved to freefall section -- new Answers to interactive questions
            \item Commented on air resistance
            \item Comments about precision in language (need to do more with precision in mathematics)
            \item Started a couple of ideas about effective theories.  (need to decide where it goes)
            \item Added detail about SI, and specifically the pound-force, pound-mass, and kilogram. to sections \ref{s:SI-MKS} and \ref{ss:weightmass}
            \item Added NIST and BIPM references (found in Wikipedia and then searched further)
            \item conversation about weight and mass.  (required reference to the chapter on Fluids and density)
            \item Moved Google search about significant figures
            \end{itemize}
        \item Added comments about fundamental forces to the section on types of force
        \item Removed indicators of v1.5 and v1.6 changes
        \end{itemize}
    \item version 2.1: June 10, 2017
        \begin{itemize}
        \item Re-commented the $\backslash$new command
        \item Started the chapters on Seeing Physics [\autoref{c:physics}] and Deeper Dive [\autoref{c:revisted}] (These should be renamed)
        \item Moved some sections on fundamental interactions
        \end{itemize}
    \item version 2.0: April 10, 2017
        \begin{itemize}
        \item Re-enabled v1.8 hides
        \item Added a link to \textit{Spacepod}, \textit{Physics Footnotes}, and \textit{Sixty Symbols}
        \item Fixed a $\backslash$dothis that was inside an $\backslash$important, causing a compile error.
        \item Removed indicators of v1.4 changes
        \end{itemize}
    \item version 1.8: April 1, 2017
        \begin{itemize}
        \item Prepare for "the public": "Disabled" the To-Do items, "Hid" the $\backslash$new revision notes, Hid the List of Tables (have none yet)
        \end{itemize}
    \end{itemize}
    }
\end{ForReviewer}
\begin{ForPublic}
{\flushleft
\textbf{Note to the reviewers:}\new{v1.8}{Added the note}
My goal with this book is to create an electronically viewable book that makes use of the advantages of being electronic.  While current e-books have the advantage of being viewable on various devices with having to carry a physical book around, most e-textbooks do not take advantage of hyperlinked text.  With this book I hope to integrate links both forward and backward.  The forward links will be used to motivate curious students.  The backward links will be used to support students who lose track of previous topics.  The integration of these will also provide a convenient opportunity for students to browse through topics they are interested in.
\newpar

At this time, I am providing a single chapter to gauge the viability.  The chapter I am providing is on Newton's Laws.  However, as you read this document, you will find many, many more partially written chapters.  All of the partial chapters and sections are intended to be place-holders for the forward- and backward-links that \autoref{c:force} depends upon.
\newpar

I created this as a PDF that, I believe, can be easily viewed on a computer or tablet.  Since some of my students also seem to read on their phone, I verified that I am also able to view the text in a reasonable manner on my Samsung phone in landscape mode.  In each case, the links should be active and easily manageable.

}
\end{ForPublic}
	\vfill

% Bottom of the page
	{\large \today\par}
\end{titlepage}

\tableofcontents
\newpage
\begin{ForReviewer}
\listoftables
\vfill
\end{ForReviewer}
%\newpage
\listoffigures
\vfill
%\newpage
%\listofstorys
%\newpage
\listofexamples
\vfill
%\newpage
\listofadventures
\vfill
%\newpage
\listofreallifes
\vfill
\newpage

\listoftodos

\newpage

\chapter*{Preface}\new{v1.8}{Modified this for the public distribution}

The purpose of creating this book is to make better use of the technology that electronic texts allow for without losing the functionality of a print book.  While this text should be comparable to any other print text, when this is provided in the online format it will provide links back and forth between early and later topics.  Linking from later material to earlier material will allow students to refresh their memory of what was previously discussed.  Linking from earlier material to later material will inspire students to look ahead to how that topic will be used in more interesting scenarios.

Having these links will allow for some other interesting features that can be placed in the back of the book and accessed through links.  Examples of this might be:
\begin{enumerate}
\item ``Dig Deeper'' where some of the more tedious and some of the more interesting aspects can be investigated. For example in \autoref{c:motion} on the equations of motion, one might see how these equations are direct applications of calculus for those students who happen to have taken that course (which is common for biology and pre-medical students).
\item ``Every Equation Tells a Story'' which discusses how the description-in-English and the description-with-math interrelate to build intuition in both directions.
\item ``Examples'', with the difference from a traditional textbook being that students can interact with the example as: ``If you have this question, then go here. If you have that question, then go there.''
\item ``In the `Real World''' where students see how the concept lives in the messy real world and why physicists simplify or ignore complicating aspects.
\item ``Connections'', which might take one of three forms:
\begin{enumerate}
\item ``Where have I seen this before?'' (linking back to earlier material)
\item ``When will I ever use this?'' (linking ahead to later material)
\item ``Why is this interesting?'' (linking to popular or complex topics)
\end{enumerate}
\end{enumerate}
The goal of the book is to encourage curiosity in the reader. Since there is an expectation that students will explore the material on their own, advanced topics will explicitly note where the reader can look for supporting material and basic topics will be motivated with links to more advanced topics.  To help maintain the interest of the reader, recurring characters will be featured in the examples.  These characters will live a storyline\dothis{storyboard the characters and how they develop}{} and interact with each other.  It is possible to read the examples as a separate storyline for the N\urgent{Decide how many characters}{} interacting characters.

I am choosing the approach described above based on the assumption that students will prefer to develop their knowledge by building a world-view that connects to their current understanding, their interests, and their world-view. Providing the cross-referencing links without distracting students with all of the information at once will enable them to explore the information. Writing the text in a narrative style that helps students see the explanations for the world they live in will encourage them to explore ``what happens when I do this'' in their real life. Fostering this spirit of exploration will enable the instructors to bring their own active-learning techniques into the classroom.

This textbook is in several Parts\urgcap{book layout}{Here we should add information about Adventures, Examples, Equation-Stories, and IRLs.}:  \textbf{Part I} is for the preliminaries, including descriptions of science in general, physics in particular, and the use of math.  \textbf{Part II} is intended to introduce three fundamental and powerful concepts.  These concepts are motion, force, and energy.  I have found that if a student can understand these ideas sufficiently well, then they can quickly pick up any other idea that we introduce, even if the idea seems initially unfamiliar.  \textbf{Part III} develops the ideas in Part II by introducing momentum, circular motion, rotational motion, torque, and the Newtonian theory of gravitation.  \textbf{Parts IV} and \textbf{V} are oscillations and thermodynamics.  With the traditional organization of the two-semester introductory physics, these parts can be covered in either order and can be chosen to be put in either semester. \textbf{Part VI} covers electricity, magnetism, light, and optics.  This is traditionally the meat of the second semester. \textbf{Part VII} touches on the topics that are usually referred to as ``modern physics''.  The goal with including these chapters is to provide some inspiration for what some students see as the tedium of the standard material.  These chapters will be linked to throughout the book as examples of how the traditional material supports the material that may be in the news and is more talked about in popular science.  The last final part, \textbf{Part VIII}, holds the answers to the interactive examples mentioned above, the bulk of the adventures the reader can investigate in order to test their understanding of the material, and the story lines of each of the characters in the text.

\textbf{A note about viewing the PDF online:}  If you are viewing this as a PDF set to view ``single page,'' then the links will take you to the top of the relevant page, rather than to the specific topic.  If, on the other hand, you are viewing this in ``continuous view'' then you should go directly to the location of interest. If you are viewing this in ``two-page'' mode (whether continuous or not), it might not be immediately obvious to which page (left or right) you have jumped.  Most of the PDF viewers I have encountered allow you to follow links and to return to your previous location.  On most PCs, the way to return to your previous location is by holding the [ctrl] key and pressing the $[\leftarrow]$ button.  There are a few PDF viewers that do not allow you to ``go back'' to the location you linked from.  Whether or not you have that capability, I have placed ``return links'' in the margins so that you can get back to the place from which you linked.


\part{Prerequisites}

\chapter{The Story of Science}

Once upon a time\done{start the book}{} somebody saw the world around them and thought something equivalent to ``well, that's an interesting pattern\ldots'' and predictions were born.  Every human and many animals build their own world of expectations such as: objects will fall down, food will arrive at mealtime, or certain people will smile at me.  Scientists study the patterns in the world around us and do so in a fairly specific way.  Novelists, sociologists, historians, and cartoonists also look at the world around us in a very particular way.  The story of humanity is a story about observing the world around us.

Scientists, in general, observe patterns through careful, detailed measurements \ldots\dothis{Add description of science.}{}

Physicists, in particular, consider the patterns in the physical world around us.\dothis{Add description of physics}{}

\hypertarget{d:physicspatterns}{Some patterns} that you might experience help us take very different experiences and group them together.  For example\inlife, there are ways in which \hyperlink{d:freefall}{dropping your keys} and \hyperlink{d:ballistic}{throwing a dog toy} are very similar.  They both fall, even thought the fall along rather different paths.  There are also patterns that you experience that might look very similar but can be treated very differently.  For example\inlife, \hyperref[irl:nonparabolic]{the path of baseball pitch} is very different for a fast ball compared to a slider, a curve ball, or a knuckleball.

\begin{figure}[h]
\hrule\hrule
  \missingfigure{Photograph a park with tennis courts and basketball hoops in the background and falling car keys and a dog in the foreground. I think we could do that at Boone County park.}
\begin{ForPublic}
\centering
\fbox{\begin{minipage}{4in}
\vspace{1in}
This will be a photograph of a park with tennis courts and basketball hoops in the background and falling car keys and a dog in the foreground.
\vspace{1in}
\end{minipage}}
\end{ForPublic}
  \caption{\label{Fig:BoonePark} Life is full of examples of physics all around us. }
\hrule\hrule
\end{figure}

%\todo[due=2017-4-1]{this one has a due date}

\section{Careful, Detailed Observation}

[Discussion of ``\hypertarget{d:casual}{casual observer}\mautoreturn{ss:NI}'' as intuition versus ``scientific observing'' and mathematical modelling]\dothis{Consider the ``casual to the obvious observer'' joke}{}

\noindent
[Discussion of common student comment: ``in physics class it is this way, but in \textit{real life} it is that way.'']

\section{Theory versus Law}\label{s:law}\mautoreturn{s:Newton}


\chapter{Seeing Physics}\label{c:physics}\new{v2.1}{Filled in the details a little. This chapter should mirror \protect{\autoref{c:revisted}}.}

\section{The Flame Challenge and Other Brief Descriptions}\label{s:flame}

What you will find in this book is a series of chapters that, on the surface, feel like a list of isolated topics.  Each chapter will have examples that focus your attention on examples of that specific concept.  However, the really interesting aspect of physics is that these descriptions of the world around us come together in different ways to explain complex systems that might feel unrelated.  For example, the thermodynamics of making your refrigerator work on Earth comes from the same theories of thermodynamics that help us understand the heat flow of the sun.  Furthermore, in order to understand the sun, we also need to understand the gravitational interaction, which also describes how baseballs fly through the air.

This chapter will introduce a set of quick-overview explanations of phenomena to indicate how different ideas tie together in some complex systems.  The point  is specifically to over-simplify complex ideas in order to ``get the idea''.  You will also be pointed to the various chapters that go into the details of the relevant physics where you can learn more.  Then, at the end of the book in~\autoref{c:revisted}, we will revisit each of these ideas and go into the description in more depth assuming you have understood each of the relevant chapters, with reference back to the sections that provide the basis of our understanding.

\textbf{Caution}: Since this particular chapter is intended to be background introduction, rather than a place to study details, none of the links to other sections here will have return links in the rest of the text.  So, if you intend to use this as a jumping off point, you might want to create a bookmark here so that you can return after you read the details in other sections.

\subsection{The Flame Challenge}\label{ss:flame}\new{v2.3}{Added the questions.  These might be better in their respective sections.}
\href{http://www.aldakavlilearningcenter.org/practice/flame-challenge}{The Flame Challenge}

Useful?  \href{https://newsstand.google.com/articles/CAIiEBF_HbPTdq-9q-hjA0W51WYqFggEKg4IACoGCAow9vBNMK3UCDDq0Rc}{How Alan Alda Makes Science Understandable}

\href{http://www.aldakavlilearningcenter.org/practice/flame-challenge/what-is-a-flame}{2012: What is a flame?} \\
\href{http://www.aldakavlilearningcenter.org/flame-challenge/past-challenges/what-time}{2013: What is time?} \\
\href{http://www.aldakavlilearningcenter.org/practice/flame-challenge/past-challenges/what-is-color}{2014: What is color?} \\
\href{http://www.aldakavlilearningcenter.org/practice/flame-challenge/past-challenges/what-is-sleep}{2015: What is sleep?} \\
\href{http://www.aldakavlilearningcenter.org/practice/flame-challenge/past-challenges/what-is-sound}{2016: What is sound?} \\
\href{http://www.aldakavlilearningcenter.org/practice/flame-challenge/past-challenges/energy}{2017: What is energy?}


\subsection{The Forming of Matter in the Universe}\label{ss:matter}\new{v2.1}{Started this section to give a sense\ldots}

In the early ages of the universe, which is an entirely different story that could be told, there were a ridiculously large number of particles created and drifting around.  There were a variety of types (\autoref{ss:StandardModel}), some being positively charged (\autoref{s:Echarge}), some negatively charged, and some were neutral; but the larger ones tended to gradually decay (\autoref{ss:particledecay}) into smaller ones.  The smaller of the positively-charged baryons (\autoref{s:particle}), which we call protons, and the smallest of the negatively-charged leptons (\autoref{s:particle}), which we call electrons, also tended to stick together because of their electrical charges (\autoref{s:Echarge}), forming hydrogen atoms.  You may note that as this happens, sometimes the more ambitious of the particles form larger clumps of two protons and two neutrons, making helium atoms that are held together by the strong nuclear force ([need ref])\dothis{Stopped mid-stream.  This is a good place to jump back in when I am stuck someplace else.}{}

\subsection{Things in the Sky}\new{v2.3}{Rearranged sections}

\subsubsection{The Sun}\label{sss:sun}\new{v2.1}{This point of this will be to connect gravity-thermo-nuclear and to do it in 1-2 paragraphs (a la the flame challenge).}
The bright, shiny sun, which keeps us all alive, is a nice example of a rather complex system that allows us to verify our various theories of the world around us.  As an over-simplification of the process, we can consider the existence of a star in three phases: the ignition (some have said ``birth'') of a star, the shining (some would say ``life'') of the star, and the snuffing (``death''?) of the star.

\subsection{Things on the ground}

\subsubsection{Hot Tea and Iced Tea}\label{sss:tea}\mautoreturn{s:surface.tension}

On 28 April, 2017,\new{v2.3}{New source of info}
\href{http://www.cbc.ca/podcasting}{CBC Broadcasting} published a
\href{http://www.cbc.ca/podcasting/includes/quirks.xml}{\textit{Quirks and Quarks}} episode discussing why
\href{https://podcast-a.akamaihd.net/mp3/podcasts/quirks_20170429_19254.mp3}{hot water sounds different from cold water when they are poured}.
Spoiler Alert: It is due to surface tension, size of droplets when heated, and auditory perception.


%\subsection{Kitchen Appliances}
\subsubsection{Oven}
\subsubsection{Refrigerator}
\subsubsection{Microwave}
\subsubsection{Television}

\subsection{Automobile}
\subsubsection{Coolant and Antifreeze}
\subsubsection{Tires}
\subsubsection{Torque}

\subsection{Cool Ideas}
\subsubsection{Black Holes}\label{sss:blackhole1}
\subsubsection{Quantum Mechanics}
\subsubsection{Relativity}
\subsubsection{String Theory}
\subsubsection{Fusion}

On 28 April, 2017,\new{v2.3}{New source of info}
\href{http://www.cbc.ca/podcasting}{CBC Broadcasting} published a
\href{http://www.cbc.ca/podcasting/includes/quirks.xml}{\textit{Quirks and Quarks}} episode discussing a
\href{https://podcast-a.akamaihd.net/mp3/podcasts/quirks_20170429_51936.mp3}{documentary compares the massive scale ITER approach to fusion with the much smaller approach by a Canadian company}.
\textbf{I don't think I want to use this, but it might be helpful to listen again to the nice summary of fusion.}  Maybe get some resources on ``state of the art''.


\section{Effective Theory}\label{s:effective1}\dothis{Should this be here or in \protect{\autoref{s:effective2}}?}{}

All of our explanations are approximations.  This section will describe some physics in the world around us in one or two paragraphs with links to the sections in the book that provide the detailed understanding of that piece which connects to the mathematics and the underlying foundation.  Each topic will also link to a more detailed discussion at the end of the book with a longer conversation that gets into more nitty-gritty details which assume you have learned the details from the book.  In short, this section looks forward to what is possible to understand and that chapter looks back at how you do understand.  Each of these topics will also be accompanied by a five-minute podcast describing the topic.

The term ``effective theory'' is used in physics to describe a wide-reaching phenomenon which can be approximated by a simpler theory in a smaller circumstance.  So, for example, Einstein's theory of general relativity as a complex description of the gravitational interaction.  It would be unwieldy and impractical to use that to describe our day-to-day interactions with the gravitational interaction.  On the other hand, Newton's theory of the gravitational interaction is a special case of Einstein's general theory of relativity that works perfectly well so long as you behave yourself and do not try to travel at a significant fraction of the speed of light.  We can say that Newton's theory of gravity is an effective theory for Einstein's theory of gravity that accounts for acceleration at low speeds.  Likewise, Einstein's special theory of relativity is an effective theory of the general theory of relativity.  The special theory is relevant when you do not allow for acceleration, but do allow for faster speeds.  Once you reach beyond the limitations of the effective theory, the description ``breaks down''.


\chapter{Why so much math?}

\section{Every equation tells a story}\label{s:story}

Mathematics is its own language.  It is the language of patterns.  Humans are very adept at tracking patterns.  Physics is the study of patterns in the physical world.  It turns out that the language of physics provides a natural and concise mechanism for expressing patterns in a uniquely precise manner.  Equations allow us to connect physical reality to very specific predictions.  For example, the equation for thermal conductivity, \autoref{eq:thermalconductivity} in \autoref{ss:thermalconductivity}, allows \studentA\index{\studentA} to predict the time it takes for \hisA\ oven to warm up to a specific temperature because
$\displaystyle \frac{Q}{\Delta t} = \kappa A \, \frac{\Delta T}{\Delta x}$\todo{I would love for this to be a mouse-over in the equation}{} says that {the rate at which energy flows} {depends on} {how well air allows energy to flow,} {the size of the oven,} and {the amount the temperature needs to change} {across the height of the oven} as follows:\dothis{Consider ``chunking'' the ``story'' with colors to indicate the pieces.}{}
\[\begin{array}{ccccc}
\deq \frac{Q}{\Delta t} & = & \deq \kappa & \deq A & \deq \frac{\Delta T}{\Delta x} \\
\EqStoryOver{45pt}{the rate at which energy flows}{}
& \EqStoryOver{40pt}{depends on}{}
& \EqStoryOver{50pt}{how well air allows energy to flow,}{}
& \EqStoryOver{50pt}{the size of the oven,}{}
& \EqStoryFrac{75pt}{and the amount the temperature needs to change}{}
                    {across the height of the oven}{}
\end{array}\]
We will see this particular story in more detail with \autoref{ex:baking} (pg.~\pageref{ex:baking}) when \studentA\index{\studentA} prepares to bake some bread for \hisA\ friends.  Some of the more important equations are listed below.  By jumping between these narratives, you can get a better sense of how to think about physics in general.

\begin{ForPublic}
\begin{table}[h]
\centering
\begin{tabular}{ccc}
\hyperref[st:F=ma]{$\deq \vec F_\mathrm{net} = m \vec a$} & .......... & \pageref{st:F=ma}
\end{tabular}
\end{table}
\end{ForPublic}
\begin{ForMe}
\dothis{Decide if should use the ``public version'' or the ``me version'' (which uses $\backslash$listofstorys).}{}
\listofstorys
\vfill
\end{ForMe}


\section{The Metric System}\label{s:SI-MKS}\mautoreturn{ss:weightmass}\new{v2.2}{Added detail}

The International System of Units (SI)
% https://en.wikipedia.org/wiki/International_System_of_Units
was adopted in 1960 at the
\href{http://www.bipm.org/jsp/en/ListCGPMResolution.jsp?CGPM=11}{eleventh meeting}
of the
\href{http://www.bipm.org/en/about-us/}{International Bureau of Weights and Measures (BIPM)}.\footnote{In French this organization is the Bureau International des poids et mesures, so the acronym is BIPM.}
%  https://en.wikipedia.org/wiki/General_Conference_on_Weights_and_Measures

In 1901 at the
\href{http://www.bipm.org/jsp/en/ListCGPMResolution.jsp?CGPM=3}{third meeting}
of the BIPM, it
\href{http://www.bipm.org/en/CGPM/db/3/2/}{was declared} that
\begin{enumerate}
\item The kilogram is the unit of mass; it is equal to the mass of the international prototype of the kilogram;
\item The word ``weight'' denotes a quantity of the same nature as a ``force'': the weight of a body is the product of its mass and the acceleration due to gravity; in particular, the standard weight of a body is the product of its mass and the standard acceleration due to gravity;
\item The value adopted in the International Service of Weights and Measures for the standard acceleration due to gravity is $980.665 \unitfrac{cm}{s^2}$, value already stated in the laws of some countries.
\end{enumerate}
The 11th meeting (1960) redefined the meter in terms of wavelengths of light.
The 13th meeting (1967) redefined the second in terms of the frequency of radiation from $^{133}$Cs.
The 17th meeting (1983) redefined the meter in terms of the speed of light and seconds.
The 24th (2011) and 25th (2014) meeting discussed redefining the kilogram in terms of the Planck constant, with an expectation that it will be redefined at the 26th meeting (Nov, 2018).  See note in \autoref{ss:units}.

Note \href{https://www.nist.gov/sites/default/files/documents/2016/11/10/appb-17-hb44-final.pdf}{Handbook 44, page B-6} talks about SI.\new{v2.2}{References to NIST}

Note
\href{https://www.nist.gov/pml/weights-and-measures/publications/nist-handbooks/handbook-44}{Handbook 44 webpage}
still links to
\href{https://www.nist.gov/sites/default/files/documents/2016/11/10/appc-17-hb44-final.pdf}{the 2016 pdf}
instead of the
\href{https://www.nist.gov/sites/default/files/documents/2017/04/28/AppC-12-hb44-final.pdf}{the 2017 pdf}
even though it says it was updated in 2017.

There is also
\href{https://www.nist.gov/pml/special-publication-811-extended-contents}{a special publication} from NIST that summarizes the use and conversation between units in the SI.

\subsection{Units Quantify Dimensions}

\subsection{Conversion from English Units}\label{ss:convertunits}\mmultireturn{\mmr{\autoref{s:sigfig}}, \mmr{\autoref{ex:slowcar}}}

Note internet search comments in \autoref{ss:weightmass} regarding the ``conversion'' of kilograms-to-pounds, with special attention to \hyperref[s:sigfig]{significant digits}.\index{Significant Digits}\dothis{rephrase this.  I moved that discussion to \protect{\autoref{s:sigfig}}.}


\subsection{Fundamental Units versus Derived Units}\label{ss:units}\mmultireturn{\mmr{\autoref{sss:unit-N}}, \mmr{\autoref{ss:weightmass}}}

Note conversation in \autoref{sss:unit-N} about the Newton.

See\new{v2.2}{Possible redefinition of the kilogram.}\mautoreturn{s:SI-MKS}
\href{https://scitechdaily.com/researchers-to-redefine-the-kilogram-in-terms-of-plancks-constant/}{the 2012 article from SciTechDaily.com}
and
\href{https://www.nist.gov/physical-measurement-laboratory/plancks-constant}{the NIST explanation} about redefining the kilogram in terms of the Planck constant at the 26th meeting (Nov, 2018) of BIPM.


\section{A graph is worth a thousand pictures}

\subsection{Coordinate Systems}

\noindent
\begin{itemize}
\item Discussion of the choice of origin (possible reference to zero-value of the potential energy)
\item Discussion of the choice of the positive-direction (possible reference to falling objects and using positive-up versus positive-down)
\item \hypertarget{d:referenceframe}{Definition of a reference frame}\mmultireturn{\mmr{\autoref{ss:addvel}}, \mmr{\autoref{ss:noninertial}}, \mmr{\hyperlink{d:NewtonInertial}{Newton's Laws}}}
\begin{itemize}
    \item (different locations) The view from the roof versus from the ground
    \item (different speeds) The view from the sidewalk versus from a moving car  (See also \autoref{ss:noninertial}.)
    \item (different types of motion) The view from a park bench versus from a merry-go-round.  (See also \autoref{s:noninertial}.)
\end{itemize}
\end{itemize}

\subsection{The Vocabulary of Graphs}

[Quick review of parameters and variables of $y=mx+b$ and $y=ax^2+bx+c$.]

\begin{center}
\setlength{\unitlength}{1cm}
\begin{Picture}(-2.5,-5.5)(3.5,3.5)
\cartesiangrid(-2,-5)(3,3)
\pictcolor{blue}
%\qbezier(-1,-7.405)(0.306,9.322)(1.612,-7.405)
%\qbezier(-1,-10.405)(0.612,15.075)(2.223,-10.405)
\qbezier(-0.5,-3.726)(0.612,8.396)(1.723,-3.726)
\end{Picture}
\end{center}

\section{Trigonometry and Vectors}

\subsection{Trigonometry}
\subsection{Vectors}\label{ss:vectors}\mmultireturn{\mmr{\hyperlink{d:pushvector}{the direction of force}}, \mmr{\autoref{sss:netforce}}}
\begin{ForMe}
\begin{figure}
\hrule\hrule
  \centering
  \caption{\LaTeX\ lines and vectors.  This will be deleted, but is here for reference.}\label{f:lines}
\begin{picture}(300,500)(0,0)
\put(0,0){\line(1,0){300}} \put(301,-2){(1,0) $0^\circ$}
\put(0,0){\line(6,1){300}} \put(301,48){(6,1) $9.46^\circ$}
\put(0,0){\line(5,1){300}} \put(301,58){(5,1) $11.31^\circ$}
\put(0,0){\line(4,1){300}} \put(301,73){(4,1) $14.04^\circ$}
\put(0,0){\line(3,1){300}} \put(301,98){(3,1) $18.43^\circ$}
\put(0,0){\line(2,1){300}} \put(301,148){(2,1) $26.57^\circ$}
\put(0,0){\line(1,1){250}} \put(251,248){(1,1) $45^\circ$}
%\put(0,0){\line(6,2){300}} \put(301,98){(6,2) $18.43^\circ$}
\put(0,0){\line(5,2){300}} \put(301,118){(5,2) $21.8^\circ$}
%\put(0,0){\line(4,2){300}} \put(301,148){(4,2) $26.57^\circ$}
\put(0,0){\line(3,2){300}} \put(301,198){(3,2) $33.69^\circ$}
%\put(0,0){\line(2,2){250}} \put(251,248){(2,2) $45^\circ$}
\put(0,0){\line(1,2){171}} \put(172,340){(1,2) $63.43^\circ$}
%\put(0,0){\line(6,3){300}} \put(301,148){(6,3) $26.57^\circ$}
\put(0,0){\line(5,3){300}} \put(301,178){(5,3) $30.96^\circ$}
\put(0,0){\line(4,3){300}} \put(301,223){(4,3) $36.87^\circ$}
%\put(0,0){\line(3,3){250}} \put(251,248){(3,3) $45^\circ$}
\put(0,0){\line(2,3){204}} \put(205,304){(2,3) $56.31^\circ$}
\put(0,0){\line(1,3){128}} \put(129,382){(1,3) $71.57^\circ$}
%\put(0,0){\line(6,4){300}} \put(301,198){(6,4) $33.69^\circ$}
\put(0,0){\line(5,4){300}} \put(301,238){(5,4) $38.66^\circ$}
%\put(0,0){\line(4,4){250}} \put(251,248){(4,4) $45^\circ$}
\put(0,0){\line(3,4){218}} \put(219,288.666666666667){(3,4) $53.13^\circ$}
%\put(0,0){\line(2,4){171}} \put(172,340){(2,4) $63.43^\circ$}
\put(0,0){\line(1,4){101}} \put(102,402){(1,4) $75.96^\circ$}
\put(0,0){\line(6,5){300}} \put(301,248){(6,5) $39.81^\circ$}
%\put(0,0){\line(5,5){250}} \put(251,248){(5,5) $45^\circ$}
\put(0,0){\line(4,5){225}} \put(226,279.25){(4,5) $51.34^\circ$}
\put(0,0){\line(3,5){192}} \put(193,318){(3,5) $59.04^\circ$}
\put(0,0){\line(2,5){146}} \put(147,363){(2,5) $68.2^\circ$}
\put(0,0){\line(1,5){84}} \put(85,418){(1,5) $78.69^\circ$}
%\put(0,0){\line(6,6){250}} \put(251,248){(6,6) $45^\circ$}
\put(0,0){\line(5,6){230}} \put(231,264){(5,6) $50.19^\circ$}
%\put(0,0){\line(4,6){204}} \put(205,304){(4,6) $56.31^\circ$}
%\put(0,0){\line(3,6){171}} \put(172,340){(3,6) $63.43^\circ$}
%\put(0,0){\line(2,6){128}} \put(129,382){(2,6) $71.57^\circ$}
\put(0,0){\line(1,6){71}} \put(72,434){(1,6) $80.54^\circ$}
%
\put(0,0){\line(0,1){450}} \put(-5,451){(0,1) $90^\circ$}
%
%
%
\put(0,0){\vector(1,0){225}}
\put(0,0){\vector(6,1){221.9}}
\put(0,0){\vector(5,1){220.6}}
\put(0,0){\vector(4,1){218.3}}
\put(0,0){\vector(3,1){213.5}}
\put(0,0){\vector(5,2){208.9}}
\put(0,0){\vector(2,1){201.2}}
\put(0,0){\vector(5,3){192.9}}
\put(0,0){\vector(3,2){187.2}}
\put(0,0){\vector(4,3){180}}
\put(0,0){\vector(5,4){175.7}}
\put(0,0){\vector(6,5){172.8}}
\put(0,0){\vector(1,1){159.1}}
\put(0,0){\vector(5,6){144}}
\put(0,0){\vector(4,5){140.6}}
\put(0,0){\vector(3,4){135}}
\put(0,0){\vector(2,3){124.8}}
\put(0,0){\vector(3,5){115.8}}
\put(0,0){\vector(1,2){100.6}}
\put(0,0){\vector(2,5){83.6}}
\put(0,0){\vector(1,3){71.2}}
\put(0,0){\vector(1,4){54.6}}
\put(0,0){\vector(1,5){44.1}}
\put(0,0){\vector(1,6){37}}
\put(0,0){\vector(0,1){225}}
%
\end{picture}
%\hrule\hrule
\end{figure}
\begin{figure}
%\hrule\hrule
  \centering
  \caption{\LaTeX\ lines and vectors.  This will be deleted, but is here for reference.}\label{f:lines2}
\begin{tikzpicture}
\draw [<->, rounded corners, thick, gray] (10,0) -- (0,0) --(0,10);
\draw [lightgray] (0,6) arc [radius=6, start angle=90, end angle=0];  % start at the (+y) of the circle, end at the (+x) of the circle
\draw [lightgray] (9,1) arc [radius=17, start angle=-10, end angle=52];
\draw [->] (0,0) -- (5.92,0.99); \draw (0,0) -- (9.08,1.51);  \node [right] at (9.08,1.51) {(6,1) $9.46^\circ$};
\draw [->] (0,0) -- (5.88,1.18); \draw (0,0) -- (9.12,1.82);  \node [right] at (9.12,1.82) {(5,1) $11.31^\circ$};
\draw [->] (0,0) -- (5.82,1.46); \draw (0,0) -- (9.18,2.29);  \node [right] at (9.18,2.29) {(4,1) $14.04^\circ$};
\draw [->] (0,0) -- (5.69,1.9); \draw (0,0) -- (9.24,3.08);  \node [right] at (9.24,3.08) {(3,1) $18.43^\circ$};
% \draw [->] (0,0) -- (5.69,1.9); \draw (0,0) -- (9.24,3.08);  \node [right] at (9.24,3.08) {(6,2) $18.43^\circ$};
\draw [->] (0,0) -- (5.57,2.23); \draw (0,0) -- (9.26,3.7);  \node [right] at (9.26,3.7) {(5,2) $21.8^\circ$};
\draw [->] (0,0) -- (5.37,2.68); \draw (0,0) -- (9.25,4.62);  \node [right] at (9.25,4.62) {(2,1) $26.57^\circ$};
% \draw [->] (0,0) -- (5.37,2.68); \draw (0,0) -- (9.25,4.62);  \node [right] at (9.25,4.62) {(4,2) $26.57^\circ$};
% \draw [->] (0,0) -- (5.37,2.68); \draw (0,0) -- (9.25,4.62);  \node [right] at (9.25,4.62) {(6,3) $26.57^\circ$};
\draw [->] (0,0) -- (5.14,3.09); \draw (0,0) -- (9.19,5.51);  \node [right] at (9.19,5.51) {(5,3) $30.96^\circ$};
\draw [->] (0,0) -- (4.99,3.33); \draw (0,0) -- (9.12,6.08);  \node [right] at (9.12,6.08) {(3,2) $33.69^\circ$};
% \draw [->] (0,0) -- (4.99,3.33); \draw (0,0) -- (9.12,6.08);  \node [right] at (9.12,6.08) {(6,4) $33.69^\circ$};
\draw [->] (0,0) -- (4.8,3.6); \draw (0,0) -- (9.02,6.77);  \node [right] at (9.02,6.77) {(4,3) $36.87^\circ$};
\draw [->] (0,0) -- (4.69,3.75); \draw (0,0) -- (8.95,7.16);  \node [right] at (8.95,7.16) {(5,4) $38.66^\circ$};
\draw [->] (0,0) -- (4.61,3.84); \draw (0,0) -- (8.9,7.42);  \node [right] at (8.9,7.42) {(6,5) $39.81^\circ$};
\draw [->] (0,0) -- (4.24,4.24); \draw (0,0) -- (8.61,8.61);  \node [right] at (8.61,8.61) {(1,1) $45^\circ$};
% \draw [->] (0,0) -- (4.24,4.24); \draw (0,0) -- (8.61,8.61);  \node [right] at (8.61,8.61) {(2,2) $45^\circ$};
% \draw [->] (0,0) -- (4.24,4.24); \draw (0,0) -- (8.61,8.61);  \node [right] at (8.61,8.61) {(3,3) $45^\circ$};
% \draw [->] (0,0) -- (4.24,4.24); \draw (0,0) -- (8.61,8.61);  \node [right] at (8.61,8.61) {(4,4) $45^\circ$};
% \draw [->] (0,0) -- (4.24,4.24); \draw (0,0) -- (8.61,8.61);  \node [right] at (8.61,8.61) {(5,5) $45^\circ$};
% \draw [->] (0,0) -- (4.24,4.24); \draw (0,0) -- (8.61,8.61);  \node [right] at (8.61,8.61) {(6,6) $45^\circ$};
\draw [->] (0,0) -- (3.84,4.61); \draw (0,0) -- (8.2,9.85);  \node [right] at (8.2,9.85) {(5,6) $50.19^\circ$};
\draw [->] (0,0) -- (3.75,4.69); \draw (0,0) -- (8.1,10.12);  \node [right] at (8.1,10.12) {(4,5) $51.34^\circ$};
\draw [->] (0,0) -- (3.6,4.8); \draw (0,0) -- (7.92,10.56);  \node [right] at (7.92,10.56) {(3,4) $53.13^\circ$};
\draw [->] (0,0) -- (3.33,4.99); \draw (0,0) -- (7.57,11.35);  \node [right] at (7.57,11.35) {(2,3) $56.31^\circ$};
% \draw [->] (0,0) -- (3.33,4.99); \draw (0,0) -- (7.57,11.35);  \node [right] at (7.57,11.35) {(4,6) $56.31^\circ$};
\draw [->] (0,0) -- (3.09,5.14); \draw (0,0) -- (7.22,12.03);  \node [right] at (7.22,12.03) {(3,5) $59.04^\circ$};
\draw [->] (0,0) -- (2.68,5.37); \draw (0,0) -- (6.57,13.13);  \node [right] at (6.57,13.13) {(1,2) $63.43^\circ$};
% \draw [->] (0,0) -- (2.68,5.37); \draw (0,0) -- (6.57,13.13);  \node [right] at (6.57,13.13) {(2,4) $63.43^\circ$};
% \draw [->] (0,0) -- (2.68,5.37); \draw (0,0) -- (6.57,13.13);  \node [right] at (6.57,13.13) {(3,6) $63.43^\circ$};
\draw [->] (0,0) -- (2.23,5.57); \draw (0,0) -- (5.73,14.32);  \node [right] at (5.73,14.32) {(2,5) $68.2^\circ$};
\draw [->] (0,0) -- (1.9,5.69); \draw (0,0) -- (5.05,15.15);  \node [right] at (5.05,15.15) {(1,3) $71.57^\circ$};
% \draw [->] (0,0) -- (1.9,5.69); \draw (0,0) -- (5.05,15.15);  \node [right] at (5.05,15.15) {(2,6) $71.57^\circ$};
\draw [->] (0,0) -- (1.46,5.82); \draw (0,0) -- (4.05,16.2);  \node [right] at (4.05,16.2) {(1,4) $75.96^\circ$};
\draw [->] (0,0) -- (1.18,5.88); \draw (0,0) -- (3.36,16.82);  \node [right] at (3.36,16.82) {(1,5) $78.69^\circ$};
\draw [->] (0,0) -- (0.99,5.92); \draw (0,0) -- (2.87,17.23);  \node [right] at (2.87,17.23) {(1,6) $80.54^\circ$};
\end{tikzpicture}
%\hrule\hrule
\end{figure}
\end{ForMe}
\subsubsection{Scalar Quantities versus Vector Quantities} \label{sss:scalarvector}\mlinkreturn[the direction of forces]{d:pushvector}
\subsubsection{Vector Equations}\label{sss:vectorequations}\mmultireturn{\mmr{\hyperlink{d:2Dmotion}{the ballistic freefall}}, \mmr{\hyperlink{d:f=ma}{$F=ma$}}}
\ldots If $\vec A = 3 \vec B$, then this is true for each component.
\begin{eqnarray}
A_x & = & 3 B_x \\
A_y & = & 3 B_y \\
A_z & = & 3 B_z
\end{eqnarray}

This can also be written in two different ways:
\[ \vect{A_x}{+A_y}{+A_z} \ = \ 3 \left( \vect{B_x}{+B_y}{+B_z} \right) \ = \ \vect{(3B_x)}{+(3B_y)}{+(3B_z)} \]

This will be useful when\foreshadow{} we are discussing \hyperref[ss:ballistic]{ballistics} (2-dimensional motion), \hyperref[ss:NII]{Newton's second law} (combining multiple forces pushing on an object), \hyperref[s:2Dcollisions]{2-dimensional collisions}, and the calculation of \hyperref[ss:Efield]{electrical fields}.

\subsubsection{Multiplication, but Not Division}\label{sss:vectorproducts}

[define dot product]

\noindent
[define cross-product]

\noindent
Can do magnitude-equations like $F=ma$ or $m=F/a$.  But for vector equations, while you can do $\vec F=m\vec a$, you cannot do something like
\hypertarget{d:dividevectors}{$\deq m = \frac{3\ihat+4\jhat}{2\ihat-5\jhat}$}\mreturn{se:netF-m}; but, in that case, you can use the magnitudes as follows
\[ m = \frac{\sqrt{(3)^2+(4)^2}}{\sqrt{(2)^2+(-5)^2}} = \frac{\sqrt{25}}{\sqrt{29}} = \sqrt{\frac{25}{29}} = 0.\sig{9}{28}{} \]

\chapter{Estimating and Uncertainty}

\section{Precision and Accuracy}\label{s:precision}\index{Precision}\new{v2.2}{Added section, added some detail}

In this section, we will consider the benefits of being precise both in measurements and in our language.  Sometimes people confuse the words precise and accurate, but they mean different things.  It may help to remember that the opposite of precise is vague.  Being precise makes it easier to determine if a statement is accurate.  If we already know the answer, then we can know if a result is accurate.  However, the exciting aspect of science is to study that which we do not already know.  In this case, gauging accuracy can be tricky.  If we are do not already know an answer, then we can try to be consistent within our accepted precision.

Since physics has its roots in the natural philosophy of the ancient Greeks and developed mathematically with Galileo and Newton, it has been around long enough for the technical language to both evolve (Newton used the word ``action'' for what we refer to as ``force'') and to be absorbed into everyday (colloquial) language. Words like force and energy have taken on broader meanings in English.  In this text, we will try to be precise with the language.  Hopefully we can avoid using the dismissive phrase, ``Oh, you \textit{know} what I \textit{mean}.''

One example\mautoreturn{ss:weightmass} of not being careful with the language comes when people use the term ``massive'' to mean ``big.''  The word massive actually means ``has a large amount of material'' whereas big means ``takes up a large amount of space'' (which might be replaced by the word ``voluminous'' rather than ``massive''). These are related by \hyperref[s:density]{the density}\index{Density} but it is possible to be massive and not voluminous (see, for example, the discussion of \hyperref[s:blakhole2]{black holes}).  While it is \textit{technically} inaccurate to use massive to mean big, ``we'' know what ``we'' mean.

\section{Significant Figures}\label{s:sigfig}\mmultireturn{\mmr{\autoref{ss:convertunits}}, \mmr{\autoref{ss:weightmass}}}

Note the comments in \autoref{ss:weightmass} regarding an internet search on the ``\hyperref[ss:convertunits]{conversion}'' of kilograms-to-pounds.\index{Significant Digits}\dothis{Remove this sentence and make the next paragraph sensible.  It makes more sense here than in \protect{\autoref{ss:weightmass}}.}

A short Google\textsuperscript{tm} search by the author found that the conversation rate between pounds and kilograms was $1 \unit{kg} = 2.2046226218 \unit{lbs}$.  Several sites go on to list about 10 decimal places for all of the conversions.  First, you should recall our discussion about \hyperref[s:sigfig]{significant digits}\index{Significant Digits}\dothis{Refocus this paragraph as an \textit{example} about significant digits.}\new{v2.2}{moved this conversation here.}.  Second you should note that the unit of pounds is a measure of force (how much the Earth pulls on you)\index{Weight}, whereas the unit of kilogram is a measure of mass (how much ``stuff'' there is)\index{Mass}.  These are related in proportion to the strength of the gravitational field, which varies in the third digit (on the order of about 1\%\addlink{variation in $g$}) around the globe.  Some sites indicate that they are shortening their conversion factor to 3 digits for convenience, but this is not an issue of convenience, it is an issue of precision\index{Precision}.

\section{Scientific Notation}


\section{Effective Theories}\label{s:effective2}\mmultireturn{\mmr{\hyperref[ss:noninertial]{non-inertial reference frames}}, \mmr{air resistance \autoref{ss:airresistance}}, \mmr{air resistance \autoref{ss:ballisticairresistance}}, \mmr{\hyperref[ss:NI]{Newton's first law}}, \mmr{\hyperref[ss:NII]{Newton's second law}}, \mmr{\autoref{s:Fg}}, \mmr{\hyperlink{d:fundamental}{fundamental forces}}, \mmr{\autoref{s:FT}}}\new{v2.2}{Added section to indicate approximate truth}\dothis{Should this be here or in \protect{\autoref{s:effective1}}?}{}

Life is complicated.  One mechanism that scientists in general and physicists in particular use to simplify their descriptions of the world around us is to build an effective theory.  These are not intended to be true (accurate) to as many decimal places as can be calculated, but rather are intended to be good enough.  In this context, good enough is most likely to mean something like: true to a reasonable number of decimal places.

A colloquial example of this is when you wear a smile to give the impression of happiness even if you are not in the mood.  Most of your casual interactions will be the same as when you are in a good mood, but your friends who know you better will recognize the small discrepancies.

A technical example of this is that Newton's theory of gravity is very precise as long as none of the objects being described are travelling ``close'' to the speed of light.  How close counts as close depends on the level of precision the measurement needs to be.  If any of the objects are moving close to the speed of light, then we need Einstein's general theory of relativity.  One way to describe this is that Newton's theory is a special case of Einstein's Theory.  Another way is describe it is that Newton's theory is an effective theory for Einstein's theory, effective when the speed is low.  It is possible for us to measure the difference between Newton's theory and Einstein's theory, but it is often not worth the effort of using the more complex theory in the cases where the simpler one will do; it is effectively true (rather than actually true).

Another technical example is that Einstein's special theory of relativity is a special case of the general theory of relativity.  The aspect that makes it a special case is that the special theory only considers motion without acceleration.

One final case that should be mentioned up front is to notice that humans experience the Earth \textit{as if} it were stationary.\dothis{Decide if this should be filled out more or if it should reference the variety of places where the text fills out these kinds of details.}{}

\part{Introducing Motion, Force, and Energy}

The trio of topics in this part of the book are fundamental and powerful concepts\footnote{The idea of Fundamental and Powerful Concepts (F\&PC) is taken from Dr. Gerald Nosich, \textit{Learning to Think Things Through}, Prentice Hall, 2012.}.  These are fundamental in that most other topics in physics are built upon them.  They are powerful in that if they are well-understood, then one is empowered to use them to understand and develop an intuition for nearly any other topic that is experienced.  It has been my experience that with these topics, students can jump into a surprisingly wide variety of other, significantly more esoteric, topics and develop a reasonable grasp of the key concepts.  Furthermore, the development of understanding of these ideas introduce the language and thought processes of being a professional physicist such that it nicely bridges the language barrier that might otherwise exist due to the jargon of physics.

\chapter{One-Dimensional Motion}\label{c:motion}

\section{How Physicists Use the Words (Notation)}

\begin{itemize}
\item Position = where is it?  Also discuss location as a vector and giving directions as defining a coordinate system (locate a common origin and unit-vector, then give a series of magnitudes and directions).
\begin{itemize}
\item This chapter will distinguish location versus distance.
\item This chapter will distinguish distance traveled versus displacement.
\end{itemize}
\item Velocity = which way did it go?  Is its position changing?
\begin{itemize}
\item This chapter will distinguish speed and velocity.
\item Introduce the language of ``at rest''.
\end{itemize}
\item Acceleration = Is its velocity changing?
\begin{itemize}
\item This chapter will clarify acceleration, deceleration, and changing direction.
\item This chapter will distinguish distance traveled versus displacement.
\end{itemize}
\end{itemize}

\section{Connecting the Concepts: distance equals rate times time}

\subsection{Position}

Identifying the position requires identifying a common known position (which we could call ``the origin''), a distance from that known location (which we could call ``a magnitude''), and a direction from the origin in which to travel such a distance.  The common example\inlife{} that identifies the location as ``I am in my room'' references ``your room'' as the common, known origin.  If the author of this text were to tell you that he was in his room, then your next obvious question is: ``OK, but where is your room?''

Position can be seen to be a vector when you describe a meeting place or destination to a friend who has never been to that location:  ``Well, you know where the bookstore is, right?'' (establishes a common origin).  ``OK, so, if you face the sports gear shop\ldots'' (sets the coordinate axis and defines the position direction) ``\ldots turn left and walk a block'' (defines the magnitude and the direction).

\subsection{Speed versus Velocity}

When you are not moving, physicists will describe you as being \hypertarget{d:atrest}{``at rest''}\mlinkreturn[Newton's First Law]{d:atrestinmotion}.  When you drive to the store\inlife, your car ``starts from rest'' and then travels some distance in some time.  When you arrive at the store, your car ``ends at rest'' when you arrive at your destination.

When you are moving\ldots\dothis{Discussion of speed as $\txtfrac{\Delta x}{\Delta t}$}{}

To be moving, you must be moving in a particular direction.\dothis{Discussion of velocity as a vector}{}


\subsection{Adding Velocities}\label{ss:addvel}

Comment on inertial \hyperlink{d:referenceframe}{reference frames}.\index{Reference Frames!Inertial}

\section{Extending the Concepts: Changing How You Move}\label{s:acceleration}\mmultireturn{\mmr{\hyperlink{d:NewtonInertial}{non-inertial reference frames}}, \mmr{\hyperlink{d:f=ma}{$F=ma$}}}

\subsection{Moving versus Speeding Up}\label{ss:acceleration}

\begin{itemize}
\item Description of \hypertarget{d:motion}{``moving''} as \textit{moving at constant velocity}.
    \mmultireturn{\mmr{\hyperlink{d:objectinmotion}{objects in motion}}, \mmr{\hyperlink{d:atrestinmotion}{Newton's First Law}}}

\item Description of \hypertarget{d:acc}{\textit{accelerating}} as either ``accelerating'', ``decelerating'', or ``turning.''\mautoreturn{s:forcewords}

\end{itemize}

Discussion of \autoref{ex:slowcar}  (pg.~\pageref{ex:slowcar}) and \autoref{ex:coasting}  (pg.~\pageref{ex:coasting}).

\section{Connecting the English to the Math}\label{s:EOM}\mreturn{se:netF-a}

\hypertarget{d:EOM}{The equations of constant acceleration}\mautoreturn{ex:ceiling} can be summarized as\new{v2.3}{referenced.   Need the story of these equations}
\begin{eqnarray*}
x_f & = & x_i + v_i t + \frac{1}{2} a t^2 \\
v_f & = & v_i + a t \\
v_f^2 & = & v_i^2 + 2 a \, \Delta x
\end{eqnarray*}

\section{Examples}

\begin{example}[hbpt]
\fcolorbox{black}{yellow!10}{\begin{minipage}{4.925in}
\caption{\label{ex:slowcar} How far will you go?}
You and your friend, \studentB\index{\studentB}, are driving along at $55.0\unit{mph}$ and run out of gas $2.25\unit{mi}$ from a gas station.  You leave the car in gear and find that after $t_1=1.00\unit{min}$, you are travelling $v_1=30\unit{mph}$.  Will you make it to the gas station?

\color{blue}
The first thing we should do is notice what information is given to us and make sure that everything is in consistent units.  I will convert everything to \hyperref[ss:convertunits]{SI units}.
\begin{eqnarray*}
v_i & = & 55.0\unitfrac{mi}{hr} \convert{1609 \unit{ft}}{1.0000 \unit{mi}}_{4\unit{sig}} \convert{1\unit{hr}}{3600\unit{s}}_\mathrm{exact} = \sigfrac{24.5}{8}{m}{s} \\
\Delta x & = & 2.25\unit{mi} \convert{1609 \unit{ft}}{1.0000 \unit{mi}}_{4\unit{sig}} = \sig{362}{0.3}{m} = \sig{3.62}{0\ten{3}}{m} \\
\end{eqnarray*}
[This example is not done, but the work will result in the following numbers:
With $t_1$ and $v_1$, you can find $a=-1500\unitfrac{mi}{hr^2}$.  From that you can find, for $v_f=0\unitfrac ms$, that $t=2.2\unit{min}$ and $\Delta x = \sig{1.00}{8}{mi}$.]

You do not make it to the gas station.

\autoreturn{ss:acceleration}
\color{black}
\end{minipage}}
\end{example}
\begin{example}[hbpt]
\fcolorbox{black}{yellow!10}{\begin{minipage}{4.925in}
\caption{\label{ex:coasting} How fast should you start?}
You and your friend, \studentB\index{\studentB}, are driving along at $55.0\unit{mph}$ and run out of gas $2.25\unit{mi}$ from a gas station.  You put the car in neutral because you know that the car will slow down with an acceleration of $a=500\unitfrac{mi}{hr^2}$.  With what speed should you be going when you put your car into neutral in order to coast to a stop at the gas station?

\color{blue}
The first thing we should do is notice what information is given to us and make sure that everything is in consistent units.  I will convert everything to metric.
\begin{eqnarray*}
v_i & = & 55.0\unitfrac{mi}{hr} \convert{1609 \unit{ft}}{1.0000 \unit{mi}}_{4\unit{sig}} \convert{1\unit{hr}}{3600\unit{s}}_\mathrm{exact} = \sigfrac{24.5}{8}{m}{s} \\
\Delta x & = & 2.25\unit{mi} \convert{1609 \unit{ft}}{1.0000 \unit{mi}}_{4\unit{sig}} = \sig{362}{0.3}{m} = \sig{3.62}{0\ten{3}}{m} \\
\end{eqnarray*}
[This example is not done, but the work will result in the following numbers:
With $a=-500\unitfrac{mi}{hr^2}$.  You can find, for $v_f=0\unitfrac ms$, that $t=6.6\unit{min}$ and $\Delta x = 3.025\unit{mi}$.  You clearly make it to the gas station.  You can also find that for $\Delta x = 2.25\unit{mi}$, $t=\sig{3.25}{9}{min}$ and $v_f=\sigfrac{27.8}{388}{mi}{hr}$.]

So, if you start at $55.0\unitfrac{mi}{hr} - 27.8\unitfrac{mi}{hr} = \sigfrac{27.1}{6}{mi}{hr}$ you should make it exactly.

\autoreturn{ss:acceleration}
\color{black}
\end{minipage}}
\end{example}

\subsection{Freefall}\label{ss:freefall}\index{Freefall}\new{v2.2}{Adding detail}\mautoreturn{ex:ceiling}

Since acceleration is the change in velocity (magnitude and/or direction), it is possible to select your own rate of change while driving your car.  However, that acceleration is difficult to measure directly.  Your speedometer measures the speed and you have to compute your acceleration based on how quickly your speed changes.  It turns out that there is a convenient way to start from the acceleration and compute the expected velocity:  \hypertarget{d:freefall}{Drop a ball or your keys}\mlinkreturn[the description of physics]{d:physicspatterns}\inlife{}.  To convince yourself that objects do, in fact, accelerate when they fall, we can consider dropping items.  One of the complications during such an experiment will be discussed in \autoref{ss:airresistance}.  If we drop a sheet of paper, air resistance causes an obvious effect (fluttering).  For this section, I will assume that the mass-to-surface-area ratio is large enough that we can effectively\Touchstone{Recall \protect{\hyperref[s:effective2]{effective theories}}.}{} ignore the air resistance.

The \hypertarget{d:accgrav}{patterns} that you see when you drop objects is that objects fall faster than humans are used to paying attention to.  The green box of \autoref{irl:freefall} (on page~\pageref{irl:freefall})\footnote{\protect{\href{https://www.osha.gov/}{OSHA}} standard \protect{\href{https://www.osha.gov/pls/oshaweb/owadisp.show_document?p_table=standards&p_id=10839}{1926.1053(a)(3)(i)}} says ``Rungs \ldots of portable \ldots  and fixed ladders \ldots shall be spaced not less than 10 inches (25 cm) apart, nor more than 14 inches (36 cm) apart \ldots ''} shows you how you can pay close attention to the patterns that result from observing falling objects.
%
\begin{reallife}[bthp]
\hspace{-.2in}
\fcolorbox{black}{green!10}{\begin{minipage}{5.29in} \center
\caption{\label{irl:freefall}\index{Freefall!Real Life} The motion of dropped objects.}
\begin{minipage}{4.925in}
Because \studentC\index{\studentC} is a pitcher on the local baseball team, \heC\ decides to drop a ball and watch what happens.  You and \studentD\index{\studentD} decide to join him.  \studentD\ provides a few other objects that can also be dropped: a tennis ball, a hammer, a small Wonder Woman toy, and a broken cell phone.  Some of these are dropped at the same time.  \studentD\ notices that it is important to release the objects at exactly the same time.
\studentC\ notices that it is important to have the objects line up at the bottom so that if they travel at the same speed, then they hit at the same time.
\end{minipage}
\begin{realtable}
\dna{Drop \textit{any} two objects at the same time from the same height}
    {Are there any objects that always hit first or last? \ref{A:firstfall}}
    {If so, what are the properties of those objects? \ref{A:firstwhy}}
\dna{Drop one of these objects from about eye-level}
    {Observe the speed of the object as it falls}
    {Is the object moving at a constant speed? \ref{A:fallv}}
\dna{Climb a tall ladder, drop the ball from at least eight-feet high}
    {Observe the time it takes the object to pass four rungs near the top of the ladder and compare it to the time it takes the object to pass four rungs near the bottom of the ladder}
    {Is one set of four-rungs a shorter time or are they the same amount of time? \ref{A:falla}}
\end{realtable}
\begin{minipage}{4.925in}
You and your friends should get together to see if you can come up with a way to measure the acceleration due to the gravitational force.
\end{minipage}

\flushright
\multireturn{\mmr{\hyperlink{d:accgrav}{freefall}}, \mmr{\hyperlink{d:Fgrav}{the force of gravity}}}
\end{minipage}}
\end{reallife}
%
You should go do those experiments before reading further.  Go ahead.  I'll wait.

You did do them, right?  You're not just reading ahead?  Really?  OK.  Doing that experiment will help you see (1) that everything falls at the same rate and (2) that objects accelerate as they fall.
It turns out that, ignoring the effect of \hyperref[ss:airresistance]{air resistance}, all objects fall with the same acceleration (due to the gravity), $a_g = 9.81\unitfrac{m}{s^2}$ downwards.
In this book,\index{Freefall}
\important{``being in freefall'' will mean moving only under the influence of gravity and having an acceleration of $a_g = 9.81\unitfrac{m}{s^2}$ downwards.}
We will start to discuss the reason for this in \autoref{s:Fg} and then get into more detail in \autoref{c:gravity}.
For now, \autoref{ex:freefall} shows the type of experiment that can allow you to calculate the acceleration due to gravity.
%
\begin{example}[hbpt]
\fcolorbox{black}{yellow!10}{\begin{minipage}{4.925in}
\caption{\label{ex:freefall} How quickly does it fall?}
Your friend, \studentC\index{\studentC}, is a baseball player and is curious to learn about the rate that baseballs fly through the air.  You get on a $12\unit{ft}$ ladder and \heC\ lays on the ground below you aiming his radar gun (which measures speed) upwards.  Each rung is $1.0\unit{ft}$ apart and his gun is at the first rung.  When you drop the ball three rungs above the gun, he measures the final velocity to be $4.24\unitfrac ms$.  When you drop the ball six rungs above the gun, he measures the final velocity to be $6.00\unitfrac ms$.  When you drop the ball eleven  rungs above the gun, he measures the final velocity to be $8.11\unitfrac ms$.  Find the acceleration of the ball in each case.

\color{blue}
The first thing we should do is notice what information is given to us and make sure that everything is in consistent units.  I will convert everything to metric.
\begin{eqnarray*}
3\unit{rungs}  & = & 3.00\unit{ft} \convert{0.3048 \unit m}{1.00000\unit{ft}} = 0.\sig{914}{4}{m} \\
6\unit{rungs}  & = & 6.00\unit{ft} \convert{0.3048 \unit m}{1.00000\unit{ft}} = \sig{1.82}{9}{m} \\
11\unit{rungs}  & = & 11.00\unit{ft} \convert{0.3048 \unit m}{1.00000\unit{ft}} = \sig{3.35}{3}{m}
\end{eqnarray*}
To find the acceleration in each case, we can solve $v_f^2 = v_i^2 + 2 a \, \Delta x$ for the acceleration:
\begin{eqnarray*}
a_3 & = & \frac{(4.23\unitfrac ms)^2 - (0 \unitfrac ms)^2}{2(0.\sig{914}{4}{m})} \ = \ \sigfrac{9.83}{1}{m}{s^2} \\
a_6 & = & \frac{(6.00\unitfrac ms)^2 - (0 \unitfrac ms)^2}{2(\sig{1.82}{9}{m})} \ = \ \sigfrac{9.84}{1}{m}{s^2} \\
a_11 & = & \frac{(8.11\unitfrac ms)^2 - (0 \unitfrac ms)^2}{2(\sig{3.35}{3}{m})} \ = \ \sigfrac{9.80}{8}{m}{s^2}
\end{eqnarray*}
Notice that these have some variation due to the rounding.  It turns out that the variation in the value of acceleration depends on the composition of the earth in your location as well as your altitude above sea-level.  That will be discussed in detail in \autoref{c:gravity}, for simplicity we will assume that all objects accelerate at the rate of $9.81\unitfrac{m}{s^2}$ when they are solely under the influence of gravity.

\multireturn{\mmr{\hyperlink{d:accgrav}{freefall}}, \mmr{\autoref{d:Fgball}}}
\color{black}
\end{minipage}}
\end{example}
%
It also turns out that you can also see this acceleration where you throw an object straight up into the air.


\section{Complications}
\subsection{Non-Inertial Accelerated Reference Frames} \label{ss:noninertial}\mmultireturn{\mmr{\hyperlink{d:referenceframe}{Reference Frames}}, \mmr{\autoref{ss:NI}}}
\index{Reference Frames!Inertial}
\index{Reference Frames!Non-inertial}

[Discuss non-rotating linearly accelerating \hyperlink{d:referenceframe}{reference frames}. See also \autoref{s:noninertial} for a discussion on rotating reference frames.]

[Comment on the Earth as essentially stationary?  See \autoref{s:effective2} on effective theories.\new{v2.2}{Effective theories}{}]

\subsection{Air Resistance}\label{ss:airresistance}\mmultireturn{\mmr{\hyperlink{d:accgrav}{freefall}}, \mmr{\ref{A:falls}}, \mmr{\autoref{s:Fg}}}
Terminal velocity\ldots
When do we include air resistance and when can we ignore it?  \ldots
[Comment on air resistance being a small effect in some cases?  See \autoref{s:effective2} on effective theories.\new{v2.2}{Reference effective theories}{}]

\subsection{Multi-Step Solutions}\new{v2.3}{new section, new example}

\begin{example}[hbpt]
\fcolorbox{black}{yellow!10}{\begin{minipage}{4.925in}
\caption{\label{ex:ceiling} \studentC\ hits the ceiling!}
\studentC\index{\studentC} gets bored one day in physics class (what?!?) and tossed a baseball ($m_b = 0.145\unit{kg}$) at the ceiling\ldots a little too hard.  The initial velocity is $v_i = +5.00\unitfrac ms \jhat$ and it leaves \hisC\ hand $1.00\unit{m}$ below the ceiling.  The ball hits the ceiling and when it returns to \hisC\ hand, it is travelling $\vec v_f=-4.73\unitfrac ms \jhat$, slower than \heC\ expected.  (a) Assuming that the ball is in contact with the ceiling for $\Delta t = 0.142\unit{s}$, find the acceleration of the ball during the collision.  (b) On the other hand, if the ceiling had not been there, then how high would the ball have gone and how fast would it have been going when it returned to \studentC's hand?

\color{blue}
In order to solve part (a) for the acceleration, we need to recognize that (1) there are five stages to the motion of the baseball and that (2) \hyperlink{d:EOM}{the equations of constant acceleration} assume that the acceleration is constant.  The five stages of the motion are: the throw, the ball moving from \studentC's hand up to (but not yet hitting) the ceiling, the ball hitting the ceiling, the ball falling from the ceiling down to (but not yet touching) \studentC's hand, and the catching of the ball.  The acceleration is $\deq a = \frac{v_f-v_i}{\Delta t}$, but the story of this equation says that since the acceleration is only during the interaction with the ceiling, then the velocities in this equation are just-before the ball hits and just-after the ball hits (not the very beginning velocity and not the very final velocity).  Similarly, the $\Delta t$ in this equation is only the time during which it was interacting with the ceiling, not the entire flight.

During \underline{the first stage}, the ball is accelerating upwards and \studentC\ is interacting with the ball.  We are not going to consider this part of the motion at all because we are given the velocity that ends this stage (and begins the next stage).

\underline{The second stage} of the motion is while the ball moves from \studentC's hand up to the ceiling. During this stage only the gravitational force is acting on the ball.  Since it has left \studentC's hand, \heC\ is not interacting with it.  Since it has not yet hit the ceiling, the ceiling is not interacting with it.  We can therefore use \hyperlink{d:EOM}{the equations of constant acceleration} to describe the motion.  During this portion of the motion we know that the velocity at the bottom is $\vec v_\mathrm{bot} = +5.00\unitfrac ms \jhat$, that it travels $\Delta \vec x = +1.00\unit{m} \jhat$, and that (because it is in \href{ss:freefall}{freefall}) it is accelerating at $\vec a_g = -9.81\unitfrac{m}{s^2} \jhat$.

\color{black}
{}\hfill {\footnotesize \autoref*{ex:ceiling} continued on next page\ldots}
\end{minipage}}
\end{example}
\begin{example}[p]
\fcolorbox{black}{yellow!10}{\begin{minipage}{4.925in}\setlength{\parskip}{3pt}
{\footnotesize \autoref*{ex:ceiling} continued from previous page\ldots}
\color{blue}

We can find the time of flight (not useful) and the velocity when the ball reaches the ceiling:
\begin{eqnarray*}
v_\mathrm{top} & = & \sqrt{ v_\mathrm{bot}^2 + 2 a \, \Delta x} \\
v_\mathrm{top} & = & \sqrt{ (+5.00\unitfrac ms)^2 + 2 (-9.81\unitfrac{m}{s^2})(+1.00\unit{m})} \\
v_\mathrm{top} & = & +\sigfrac{2.31}{9}{m}{s}
\end{eqnarray*}
Note: When you take the square root, you have to decide if you should take the positive sign or the negative sign.  In this case, the ball is still moving upwards, so we \textit{choose} the positive sign.

\underline{The third stage} is while it is interacting with the ceiling.  In order to find the acceleration during this motion, we need to know the velocity immediately before hitting the ceiling (which we just found) and the velocity just after it finishes hitting the ceiling (which we have not yet found).  We will come back to this step.

\underline{The fourth stage}, like the second, is while the ball moves from the ceiling down to \studentC's hand.  During this portion of the motion we know that the velocity at the bottom (final) is $\vec v_\mathrm{bot} = -1.67\unitfrac ms \jhat$, that it travels $\Delta \vec x = -1.00\unit{m} \jhat$, and that (because it is in \href{ss:freefall}{freefall}) it is accelerating at $\vec a_g = -9.81\unitfrac{m}{s^2} \jhat$.  We can find the time of flight (not useful) and the velocity when the ball leaves the ceiling (initial), solving $v_\mathrm{bot}^2 = v_\mathrm{top}^2 + 2 a \, \Delta x$ for $v_\mathrm{top}$:
\begin{eqnarray*}
v_\mathrm{top} & = & \sqrt{ v_\mathrm{bot}^2 - 2 a \, \Delta x} \\
v_\mathrm{top} & = & \sqrt{ (-1.67\unitfrac ms)^2 - 2 (-9.81\unitfrac{m}{s^2})(+1.00\unit{m})} \\
v_\mathrm{top} & = & -\sigfrac{4.73}{4}{m}{s}
\end{eqnarray*}
Note: When you take the square root, you have to decide if you should take the positive sign or the negative sign.  In this case, the ball is now moving downwards, so we \textit{choose} the negative sign.

\color{black}
{}\hfill {\footnotesize \autoref*{ex:ceiling} continued on next page\ldots}
\end{minipage}}
\end{example}
\begin{example}[p]
\fcolorbox{black}{yellow!10}{\begin{minipage}{4.925in}\setlength{\parskip}{3pt}
{\footnotesize \autoref*{ex:ceiling} continued from previous page\ldots}
\color{blue}

During \underline{the fifth stage}, the ball is accelerating upwards while moving downwards and so \studentC\ is stopping the ball.  We are not going to consider this part of the motion at all. \\

Now that we have the velocities immediately before and after the collision with the ceiling, we can find the acceleration:
\[ a = \frac{v_f-v_i}{\Delta t} = \frac{(-\sigfrac{4.73}{4}{m}{s})-(+\sigfrac{2.31}{9}{m}{s})}{(0.142\unit s)} = -\sigfrac{28.0}{9}{m}{s^2} \,\jhat \]
Notice that the acceleration is negative because the ball went from going up to going down.

To solve part (b), we can just consider from after-thrown to before-caught.  During this motion, assuming there is no ceiling, the entire motion is in freefall, so we can use $v_f^2 = v_i^2 + 2 a \, \Delta x$ and solve for $\Delta x$.  However, we only want to consider from the lowest point to the highest point, not all the way back to \studentC's hand.

\centering{THIS NEEDS TO BE FINISHED}

\flushright
\multireturn{\mmr{\ref{se:ceiling}}, \mmr{\ref{se:throw-up}}}
\color{black}
\end{minipage}}
\end{example}\dothis{finish \protect{\autoref{ex:ceiling}}.  Maybe make it two examples, instead of one?}


\chapter{Two-Dimensional Motion}

\subsection{Ballistic Freefall}\label{ss:ballistic}\mautoreturn{sss:vectorequations}

\hypertarget{d:ballistic}{Discussion about throwing a ball\ldots}\mlinkreturn[the description of physics]{d:physicspatterns}\inlife.

\hypertarget{d:2Dmotion}{For 2-dimensional motion}\mlinkreturn[$F=ma$]{d:f=ma}, we will use \hyperref[sss:vectorequations]{vector equations} to describe the relationships.
When we write $\vec v_f = \vec v_i + \vec a t$, we mean that this relationship holds for the $x$-components and separately for the $y$-components:
\[ v_{fx} = v_{ix} + a_x t \hspace{1cm} v_{fy} = v_{iy} + a_y t \]

\section{Complications}
\subsection{Air Resistance}\label{ss:ballisticairresistance}
Terminal velocity\ldots non-parabolic paths \ldots \autoref{irl:nonparabolic} (pg.~\pageref{irl:nonparabolic})
\begin{reallife}[bhp]
\hspace{-.2in}
\fcolorbox{black}{green!10}{\begin{minipage}{5.29in} \center
\caption{\label{irl:nonparabolic} Baseball pitches are not usually parabolic.}
\begin{minipage}{4.925in}
\studentC\index{\studentC} is a pitcher on the local baseball team.  \HeC\ throws a fast ball, a slider, a curve ball, and a knuckleball.
\end{minipage}
\begin{realtable}
\dna{Go to a baseball game on a calm day.  Sit near third base.}
    {The path of fly balls to left field}
    {Are they parabolic? \ref{A:fly.balls}}
\dna{Go to a baseball game on a calm day.  Sit near third base.}
    {The path of pitch towards home plate.}
    {Are they parabolic? \ref{A:pitches.side}}
\dna{Go to a baseball game.  Sit up high behind home plate.}
    {The path of the baseball for various pitches.}
    {Do they all fly straight over the plate? \ref{A:pitches.top}}
\end{realtable}

\flushright
\multireturn{\mmr{\hyperlink{d:physicspatterns}{the description of physics}}, \mmr{\autoref{ss:ballisticairresistance}}}
\end{minipage}}
\end{reallife}

[Comment on air resistance being a small effect in some cases?  See \autoref{s:effective2} on effective theories.\new{v2.2}{Reference Effective theories}{}]



\chapter{Force}\label{c:force}

\section{How Physicists Use the Words (Notation)}\label{s:forcewords}

The technical term \textbf{force}\index{Force!description} refers to the general idea of pushing or pulling.  In the same way that\touchstone{} physicists use the word \hyperlink{d:acc}{acceleration} (technically \textit{changing the velocity}) to mean \textbf{either} \textit{speeding up} (colloquially ``acceleration'') \textbf{or} \textit{slowing down} (colloquially ``deceleration'') \textbf{or} \textit{changing the direction} (colloquially ``turning''), we will use \hypertarget{d:forcenoun}{\textbf{force} as a noun}\index{Force!noun}\mlinkreturn[heat as a verb]{d:heatverb} referring to the act of pushing or pulling.

\hypertarget{d:interaction}{You should note}\mmultireturn{\mmr{\autoref{ex:braced}}, \mmr{\hyperref[d:Fgball]{the falling ball}}} that you can't have a push or pull without \textbf{both} a thing that pushes or pulls \textbf{and} a thing that is pushed or pulled.\aside{Push or Pull}{By now you may have noticed that it is tedious to keep saying ``pushed or pulled,'' so I will only say ``push'' even when I am including the possibility of ``pushing or pulling''.}{}
\important{Forces are \underline{necessarily} an interaction\index{Interaction}\index{Force!Interaction} between two objects.}
Sometimes we care about the thing doing the pushing or pulling, sometimes we don't.  We always care about the thing being pushed or pulled.  We will \underline{distinguish these objects} by referring to the object that is pushing as the object ``causing the force'' or ``exerting the force'' and by referring to the object that is being pushed as the object ``feeling the force''.  We will \underline{distinguish these forces} as follows: Let's imagine that \studentB\index{\studentB} gives \studentA\index{\studentA} a good-natured shove in the arm.  The following are useful descriptions and are different ways of describing the same action.
\begin{itemize}\itemsep 1pt
\item \studentB\ exerted a force \underline{on} \studentA.
\item \studentA\ felt a force \underline{from} \studentB.
\item There was a force \underline{on} \studentA\ \underline{by} \studentB.
\end{itemize}
The notation for this will be $F_{A,B}$ where the first subscript is the person who felt the force (who the force is ``on'') and the second subscript is who exerted the force (who the force is ``by'').  In those instances when we only care about who is feeling the force and not who is exerting the force, we might just use one subscript $F_A$.  In some cases, there may be two forces acting on one person (or object).  In that case, it will be obvious who is feeling the force and we will use the subscript to distinguish which force it is, such as $F_1$ or $F_2$, rather than who feels the force.  This will be more relevant when we discuss in \autoref{c:forcetype} the types of forces that might be applied.

\hypertarget{d:Newtonahead}{Looking ahead} to \hyperref[s:Newton]{Newton's Laws}\index{Laws!Newton}\Touchstone{Recall the distinction between \protect{\hyperref[s:law]{theories and laws}}.}, you should be ready to notice that the \hyperref[ss:NI]{first law} is about objects that are not feeling a force, the \hyperref[ss:NII]{second law} is about a specific object that is feeling a force, and the \hyperref[ss:NIII]{third law} is about the interaction between the two objects.  In all three of these, we care about the object feeling the force.   It is only in the third law that we care about the object exerting the force.

\hypertarget{d:pushvector}{Looking back}, forces are \hyperref[sss:scalarvector]{vectors}:
\important{Pushing on something intrinsically involves a direction.}
You will use this property to show that multiple people pushing\inlife{} in the same direction increases the effect, whereas multiple people pushing in opposite directions reduces the effect.  One might say that people who push an object in opposite directions work\footnote{After you study \protect{\autoref{s:work}}, this play on words will be hilarious!} against each other.  Because the force is a vector, whenever you are answering a question about a force, you should always expect to give the strength of the force (the \hyperref[ss:vectors]{magnitude}) and \hyperref[ss:vectors]{the direction} of the force (relative to some specific axis, usually the positive $x$-axis).

\section{Connecting the Concepts: Newton's Laws}\label{s:Newton}\mlinkreturn[how to describe forces]{d:Newtonahead}

Newton's Laws (recall \hyperref[s:law]{Theory versus Law}) describe our observations about three questions\inlife:
\begin{enumerate}\itemsep 1pt
  \item What happens to an object when I \textit{don't} push on it?
  \item What happens \textit{to an object} when I do push on it?
  \item What happens \textit{to me} when I push on an object?
\end{enumerate}
The answer to these questions have a precise, concise, technical language and the point of the next three subsections are to translate that into (modern) English, into math, and into intuition.  The statement of these laws has slightly different versions in different texts to emphasize different points.  We will state them as follows\index{Newton!Laws}\index{Laws!Newton}:
\begin{enumerate}
    \item \hypertarget{sum:Newton'sLaws}{When} viewed from an inertial reference frame, an object with no forces acting on it will maintain its velocity, which may be zero.
    \item When viewed from an inertial reference frame, the vector-sum of all forces acting on an object will cause that object to accelerate in proportion to its mass: $\vec F_\mathrm{net} = m \vec a$.
    \item For every force acting (the ``action'') on one object by an other object, there is an equal-in-magnitude reaction-force acting on the other object in the opposite direction.
\end{enumerate}
\hypertarget{d:NewtonInertial}{There are a few terms} that should be clarified in these laws.  Being in an inertial \hyperlink{d:referenceframe}{reference frame}\index{Reference Frames!Inertial} essentially means being in a place in which you do not have to hold on in order to maintain your position.  If you are \hyperref[s:acceleration]{accelerating} (recall that this means \textit{speeding up}, \textit{slowing down}, or \textit{changing direction}) then you are not in an inertial reference frame, but rather are in a non-inertial reference frame.  In this case, you will misinterpret the forces acting.  This will be discussed in more detail in \autoref{s:noninertial} when we discuss centripetal force and in \autoref{ss:coriolis} when we discuss the Coriolis effect.

Sometimes Newton's first law is written to include the phrase ``an \hypertarget{d:objectinmotion}{object in motion}'', which I will be careful to link directly to \hyperlink{d:motion}{velocity}, as was done above.  However, it technically should reference the momentum\foreshadow, which is discussed in \autoref{c:momentum}.

The way Newton's third law is often written and referred to includes the words ``action'' and ``reaction''.  Newton was referring to forces with these words and to keep it clear in our discussion, we will use the word force, with the occasional clarification of the action-force or the reaction-force.

\subsection{Translating Newton's First Law: The Law of Inertia}\label{ss:NI}\mlinkreturn[how to describe forces]{d:Newtonahead}

%\important{} is not designed to start a new paragraph
\ \vspace{-12pt}
%\begin{quote}
\important{\textbf{Newton's First Law}:\index{Newton!First Law} When viewed from an inertial reference frame, an object with no forces acting on it will maintain its velocity, which may be zero.}
%\end{quote}
Let's take this apart and connect it to your daily experiences.  Looking ahead, we will discuss the surface of the Earth as a \hyperlink{d:noninertial}{non-inertial rotating reference frame}\Touchstone{You might also recall the discussion in \protect{\autoref{ss:noninertial}}.}; however, the effect is small enough that for most of what we \hyperlink{d:casual}{casually observe}, we can safely pretend that the Earth is stationary and that we are actually at rest while sitting on the curb watching the world go by.  This is so true that our human brains already interpret everything around us as though it were an inertial reference frame.  This psychological perspective is exactly the feature that both allows us to make fairly reliable predictions about the world around us \textit{and} causes us to make incorrect judgements when we encounter non-inertial reference frames.  That is to say, as long as we don't measure our world too closely\Touchstone{Recall \protect{\hyperref[s:effective2]{effective theories}}.}, we are viewing it from an essentially inertial reference frame.  This point is so implicit, that many books do not even include the portion of the statement referring to the reference frame.
\begin{ForMe}
\todo{Consider adding comment about Newton not including ``inertial reference frame'' to his laws.}
%\footnote{Newton himself did not include it in his original statement.}%\urgcap{Newton's ``inertial frames''}{check the implications of this}{}
\end{ForMe}

\hypertarget{d:atrestinmotion}{The rest of this statement} is often written a little differently (and less concisely) as ``an object at rest remains at rest unless acted on by an external force and an object in motion remains in motion unless acted on by an external force.''  Since being \hyperlink{d:atrest}{``at rest''} is a statement about the velocity ($\vec v=0$) and being \hyperlink{d:motion}{``in motion''} is also a statement about the velocity, each of these statements can be understood as saying that
\important{Forces are those things that cause a change in the velocity.}
In other words, Newton's first law says that without a force, the velocity will not change.  In the discussion of \hyperref[sss:equilibrium]{equilibrium}, we will note that this is often extended to say that without a \hyperref[sss:netforce]{net force} the velocity will not change, but that is a special case of \hyperref[ss:NII]{Newton's second law}.

\subsubsection{Inertia}\label{sss:inertia}\index{Inertia}

This law is often called the ``law of inertia''.  The concept of inertia can be described as \textit{the tendency of an object to maintain its velocity}.  This is describing how the object behaves when you don't do anything to it.  The inertia is not a quantity that physicists calculate, but physicists do refer to objects as having a lot of inertia, usually to indicate that it will take a large force to change the object's motion, or as having a small amount of inertia, usually to indicate that it should be relatively easy to change the object's motion.  However, the inertia does not actually refer to the force needed.  Instead, the inertia most often refers to the ``inertial mass'' of an object, which shows up in the second law.
\important{Inertia is not a force.}
Sometimes when physicists are not being careful with their language, they will appear to use the word inertia interchangeably with the term \hyperref[c:momentum]{momentum}, which we will discuss in more detail in \autoref{ss:inertia}.

\subsubsection{How the Laws Work Together}\label{sss:NItogether}

You should notice that Newton's First Law is about what happens when you are \textit{not pushing} on the object, which is to say, the tendency of an object to maintain its own motion without a force acting on it; this is the inertia of the object.  On the other hand, Newton's Second Law is about what happens \textit{to the object} when you \textit{do push} on an it.  This is what we will consider next.  After that, Newton's third law will describe what happens \textit{to the thing pushing} rather than just to the thing being pushed.  \hyperref[sss:NIItogether]{The section} at the end of \autoref{ss:NII} will explore these ideas further.



\subsection{Translating Newton's Second Law: The Equation Law}\label{ss:NII}\mmultireturn{\mmr{\autoref{sss:vectorequations}}, \mmr{\hyperlink{d:Newtonahead}{how to describe forces}}, \mmr{\hyperlink{d:atrestinmotion}{Newton's first law}}, \mmr{\autoref{d:Fgball}}}

%\important{} is not designed to start a new paragraph
\ \vspace{-12pt}
%\begin{quote}
\important{\textbf{Newton's Second Law}\index{Newton!Second Law}: When viewed from an inertial reference frame, the vector-sum of all forces acting on an object will cause that object to accelerate in proportion to its mass: $\vec F_\mathrm{net} = m \vec a$.}\docaption{Inline-math formatting}{Why does an equation that starts a new line get indented slightly?}
%\end{quote}
Let's take this apart and connect it to your daily experiences.  As with Newton's First Law, the \hyperlink{d:noninertial}{non-inertial rotating reference frame} of the surface of the Earth is a small enough effect that, as long as we don't measure our world too closely\Touchstone{Recall \protect{\hyperref[s:effective2]{effective theories}}.}, we can pretend that we are viewing it essentially from an inertial reference frame.

\hypertarget{d:f=ma}{For this law}, it is often sufficient to write down the equation and know that the words are there for back-up.  While most people have no trouble remembering $F=ma$, it is important to pay attention to two aspects:
\begin{itemize}
\item This is a vector-equation, which means\Touchstone{Recall (1) the generic explanation for \protect{\hyperref[sss:vectorequations]{vector equations}} and (2) the \protect{\hyperlink{d:2Dmotion}{vector-equations}} for two-dimensional motion} that
    \begin{itemize}
    \item the equation is true for each component separately, and
    \item the direction of $\vec F_\mathrm{net}$ is the same as the direction of $\vec a$ (which, of course, might be different than the direction of the velocity).
    \end{itemize}
\item The force in this equation is the \textit{net force}, which means that we must consider \underline{all} forces that are acting on this object and only those forces that are acting on \underline{this} object.
\end{itemize}
\addcontentsline{los}{story}{$\vec F_\mathrm{net} =m \vec a$}
\phantomsection\label{st:F=ma}\thestoryof{\vec F_\mathrm{net} = m \vec a}\mautoreturn{ex:2Dforce}
This equation is all about what happens to a specific object, $m$.  If the object, $m$, is accelerating in a particular direction, $\vec a$, then it is because the combination of forces, $\vec F_\mathrm{net}$, do not entirely cancel each other out.  This also can be expressed as: if the combination of forces, $\vec F_\mathrm{net}$, do not entirely cancel each other out, then our friend $m$ must be accelerating,~$\vec a$, in a particular direction.  Furthermore the resulting direction of the net force determines the direction of the acceleration.  \hypertarget{d:f=ma}{Connecting the English} and the math:
\[\begin{array}{cccc}
\deq \vec F_\mathrm{net} & = & \deq m & \deq \vec a \\
\EqStoryOver{65pt}{the combination of all forces acting on $m$}{}
& \EqStoryOver{33pt}{causes}{}
& \EqStoryOver{35pt}{that object}{}
& \EqStoryOver{40pt}{to change its velocity}{}
\end{array}\]
You should \hyperref[s:acceleration]{recall}\touchstone, that the direction of the acceleration does not determine the direction \textit{of the motion}, but rather determines the direction \textit{of the change} in motion.  That idea will be important\foreshadow{} when we discuss how a \hyperref[s:FT]{tension} acts as a \hyperref[s:centripetal]{centripetal force}, the relationship between velocity and acceleration in a \hyperref[s:springs]{spring} that \hyperref[c:SHMspring]{oscillates}, and objects that are propelled through either a \hyperref[s:Gfield]{gravitational} or an \hyperref[ss:Efield]{electrical} field.

\subsubsection{Units of Force}\label{sss:unit-N}\mautoreturn{ss:units}\index{Force!Units of}

Recall that the fundamental units of the \hyperref[ss:units]{SI-system} are meters, kilograms, and seconds (MKS)\addlink{maybe note the MKS-to-SI transition. maybe leave that in \protect{\autoref{ss:units}}}.  With our relationship connecting force to mass $\unit{(kg)}$ and acceleration $\left(\unitfrac{m}{s^2}\right)$, we can see that the units of force are ${}\unitfrac{kg \cdot m}{s^2}$.  This quantity is so common that we would like to have a shorthand for it.  Furthermore, Sir Isaac Newton did such ground-breaking work on the concept, that it was decided in 1948\footnote{According to: International Bureau of Weights and Measures (1977),The international system of units (330-331) (3rd ed.), U.S. Dept. of Commerce, National Bureau of Standards, \protect{\href{https://books.google.com/books?id=YvZNdSdeCnEC&pg=PA17\#v=onepage&q&f=false}{p. 17}},
%  ISBN 0745649742, Found at https://en.wikipedia.org/wiki/Newton_(unit)
which refers to
\protect{\href{http://www.bipm.org/jsp/en/ViewCGPMResolution.jsp?CGPM=9&RES=7}{the 7th resolution}} (Mar, 2017) of
\protect{\href{http://www.bipm.org/jsp/en/ListCGPMResolution.jsp?CGPM=9}{the 9th CGPM}} (Mar, 2017).} to name the unit the Newton, such that
\important{$1 \unit{N} = 1 \unitfrac{kg \cdot m}{s^2}$}

\subsubsection{Calculating the Net Force}\label{sss:netforce}\mlinkreturn[Newton's first law]{d:atrestinmotion}

The word ``net'' that goes with force is here to indicate the total, which is useful to think of as ``everything collected with the net.''\footnote{Although according to \protect{\href{http://www.etymonline.com/index.php?term=net&allowed_in_frame=0}{etymonline.com}} (Mar, 2017), it is actually from the Old French \textit{net} for ``neat'' or ``clean'', having the sense of trim and elegant.}  The intention here is that wherever there are multiple forces acting on a single object, we must combine them as \hyperref[ss:vectors]{vectors} as follows: \\
\begin{minipage}[c]{3.25in}
\begin{sample}
\item\label{se:netFadd} If there is a $5.0 \unit{N}$ force to the right and a $4.0 \unit{N}$ force to the right, then the net force is $9.0 \unit{N}$ to the right.
    \[ \vec F_\mathrm{net} = \vec F_1 + \vec F_2 = \left( 5.0\unit{N} \ihat\right) + \left( 4.0 \unit{N} \ihat\right) = +9.0\unit{N} \ihat \]
\end{sample}
\end{minipage}\mmultireturn{\mmr{\ref{se:netF-a}}, \mmr{\hyperref[sss:equilibrium]{Equilibrium}}, \mmr{\ref{se:FBD-AB}}, \mmr{\autoref{f:firstFBD}}, \mmr{\hyperlink{d:FBD-AB}{discussion about \ref*{se:FBD-AB}}}}
\hfill
\fbox{\begin{minipage}[c]{1.5in}
\begin{FBD}{10}{15}{15}{10}{object}
\twori{50}{$5\unit N$}{black}{40}{$4\unit N$}{black}
\end{FBD}
\end{minipage}}\dothis{Make this object a desk so that we can have \studentA\ and \studentB\ helping you rearrange your room in your residence hall.}{}
\begin{minipage}[c]{3.25in}
\begin{sample}
\item\label{se:netFsub} If there is a $5.0 \unit{N}$ force to the left and a $4.0 \unit{N}$ force to the right, then the net force is $1.0 \unit{N}$ to the left.
    \[ \vec F_\mathrm{net} = \vec F_1 + \vec F_2 = \left(-5.0\unit{N} \ihat\right) + \left( 4.0 \unit{N} \ihat\right) = -1.0\unit{N} \ihat \]
\end{sample}
\end{minipage}\mmultireturn{\mmr{\ref{se:netF-m}}, \mmr{\hyperref[sss:equilibrium]{Equilibrium}}}
\hfill
\fbox{\begin{minipage}[c]{1.5in}
\begin{FBD}{10}{15}{15}{10}{object}
\onele{50}{$5\unit N$}{black}
\oneri{40}{$4\unit N$}{black}
\end{FBD}
\end{minipage}}
\begin{minipage}[c]{3.25in}
\begin{sample}
\item\label{se:equi} If there is a $3.0 \unit{N}$ force to the right and a $3.0 \unit{N}$ force to the left, then the net force is $0.0 \unit{N}$.
    \[ \vec F_\mathrm{net} = \vec F_1 + \vec F_2 = \left( 3.0\unit{N} \ihat\right) + \left(-3.0 \unit{N} \ihat\right) = 0.0\unit{N} \ihat \]
    \mbox{In this case, the object is said to be ``\hyperref[sss:equilibrium]{in equilibrium}.''}
\end{sample}
\end{minipage}\mmultireturn{\mmr{\hyperref[sss:equilibrium]{Equilibrium}}, \mmr{\hyperlink{d:equi}{discussion about \ref*{se:equi}}}}
\hfill
\fbox{\begin{minipage}[b]{1.5in}
\begin{FBD}{10}{15}{15}{10}{object}
\onele{30}{$3\unit N$}{black}
\oneri{30}{$3\unit N$}{black}
\end{FBD}
\end{minipage}}
The images included in these examples will eventually\foreshadow{} be referred to as ``\hyperref[sss:FBD]{free-body diagrams}\index{Free-Body Diagrams!Images},'' but for now, you can just consider them images of the forces acting on the bodies.

\hypertarget{d:netforce}{Next}, we should do a couple of examples that show the math for situations with forces in two dimensions.   The first, \autoref{ex:2Dforce}  (pg.~\pageref{ex:2Dforce}), has one force in the $x$-direction and another in the $y$-direction.
%
\begin{example}[hb]
\fcolorbox{black}{yellow!10}{\begin{minipage}{4.925in}
\caption{\label{ex:2Dforce} An object is pushed by perpendicular forces.}
A $2.0\unit{kg}$ mass is being pushed north with $5.0\unit{N}$ and east with $4.0 \unit{N}$.  What is the net force?
\color{blue}

\vspace{9pt}
\begin{minipage}{3.25in}
Since we have multiple forces acting on a mass to cause an acceleration, it should be clear (recall the \hyperref[st:F=ma]{story}) that we need to use Newton's second law and find the net force in order to compute the acceleration.  We will, as usual, start with a free-body diagram (at right).  This example is made easier because the forces happen to be at right angles and so finding their $x$ and $y$ components is not difficult. By adding
\end{minipage}
\hfill
\fbox{\begin{minipage}{1.5in}
\begin{FBD}{10}{15}{15}{40}{object}
\oneup{50}{$5\unit N$}{black}
\oneri{40}{$4\unit N$}{black}
\end{FBD}
\end{minipage}}
\vspace{2pt}

\begin{minipage}[c]{1.95in}
\begin{forcetable}
\force{F_1}{ 0 \unit N}{+5 \unit N}
\force{F_2}{+4 \unit N}{ 0 \unit N} \hline\hline
\force{F_\mathrm{net}}{+4\unit N}{+5\unit{N}}
\end{forcetable}
\end{minipage}
\hfill
\begin{minipage}[c]{2.8in}
the $x$-components and separately adding the $y$-components, we have found the components of the net force.  From there, we can easily find the magnitude and direction of the net force.
\end{minipage}
\magdir{+4\unit N}{+5\unit{N}}{F_\mathrm{net}}{\sig{6.4}{0}{N}}{\theta}{\sig{51}{.3}{^\circ}}{N of E}
%
(The direction can be stated as $\theta = 51^\circ \textrm{N of E}$ or as $\phi = 39^\circ\textrm{E of N}$.)
\autoref{ex:2Dfa}  (pg.~\pageref{ex:2Dfa}) will use this calculation to find the acceleration.
\flushright
\multireturn{\mmr{the discussion of \hyperlink{d:netforce}{the net force}}, \mmr{\autoref{ex:2Dforce2}}}
\end{minipage}}
\end{example}
%
The second, \autoref{ex:2Dforce2}  (pg.~\pageref{ex:2Dforce2}), has one force in the $x$-direction and the other in the second quadrant.
%
\begin{example}[hb]
\fcolorbox{black}{yellow!10}{\begin{minipage}{4.925in}
\caption{\label{ex:2Dforce2} Three forces act on an object.}
A $2.0\unit{kg}$ mass is being pushed northwest with $5.0\unit{N}$ at an angle $63.4^\circ\textrm{N of W}$, southwest with $6.0\unit{N}$ at an angle of $21.8^\circ\textrm{S of W}$, and east with $4.0 \unit{N}$.  What is the net force?
\color{blue}

\vspace{9pt}
\begin{minipage}{3.25in}
This follows the same logic as \autoref{ex:2Dforce}  (pg.~\pageref{ex:2Dforce}), which I will not restate here. This example is slightly harder because the forces have to be split into their $x$ and $y$ components. By adding the $x$-components and separately adding the $y$-components, we have found the components of the net force.  From there, we can easily find the magnitude and direction of the net force.
\end{minipage}
\hfill
\fbox{\begin{minipage}{1.5in}
\begin{FBD}{10}{15}{25}{40}{object}
%\oneup{50}{$5\unit N$}{black}
%\put(24,56){\vector(-1,2){22.36}}
%\put(0,102){\color{black} \tiny $5\unit N$}
\oneul{22}{48}{-1}{2}{$5 \unit N$}{black}
\onedl{58}{22}{-5}{-2}{$6 \unit N$}{black}
\oneri{40}{$4\unit N$}{black}
\end{FBD}
\end{minipage}}
\vspace{2pt}

%\begin{minipage}[c]{1.95in}
\begin{forcetable}
\foTWO{F_1}{-(5.0\unit N)\cos(63.4^\circ) =}{-\sig{2.2}{4}{N}}{(5.0\unit N)\sin(63.4^\circ) =}{+\sig{4.4}{7}{N}}
\foTWO{F_2}{-(6.0\unit N)\cos(21.8^\circ) =}{-\sig{5.5}{7}{N}}{-(6.0\unit N)\sin(21.8^\circ) =}{-\sig{2.2}{3}{N}}
\force{F_3}{+4.0 \unit N}{ 0 \unit N} \hline\hline
\force{F_\mathrm{net}}{-\sig{3.8}{1}{N}}{+\sig{2.2}{4}{N}}
\end{forcetable}
%\end{minipage}
%
\magdir{-\sig{3.8}{1}{N}}{+\sig{2.2}{4}{N}}{F_\mathrm{net}}{\sig{4.4}{2}{N}}{\theta}{\sig{30}{.5}{^\circ}}{N of W}
%
(The direction can be stated as $\theta = 31^\circ \textrm{N of W}$ or as $\phi = 60^\circ\textrm{W of N}$.)
\autoref{ex:2Dfa2}  (pg.~\pageref{ex:2Dfa2}) will use this calculation to find the acceleration.

\linkreturn[the net force]{d:netforce}
\end{minipage}}
\end{example}


\subsubsection{Using the Net Force to Calculate Other Quantities}

Generally, the point of finding the net force is that it causes an object to change its velocity. Let's also consider a few simple examples of this calculation.
\begin{sample}
\item\label{se:netF-a}\mmultireturn{\mmr{\hyperlink{d:m=f/a}{finding $m$ from $F=ma$}}, \mmr{\ref{se:FBD-AB}}, \mmr{\autoref{f:firstFBD}},\mmr{\hyperlink{d:FBD-AB}{discussion about \ref*{se:FBD-AB}}}, \mmr{\ref{se:weightA}}, \mmr{\ref{A:netF-a}}, \mmr{\autoref{f:firstFBDupdate}}} If the forces in \ref{se:netFadd}\dothis{As before, make this object a desk so that we can have \studentA\ and \studentB\ helping you rearrange your room in your residence hall.  (Change the mass!)}{} are applied to an object with mass $2.0\unit{kg}$, then it will accelerate at the rate of
    \[ \vec a =\frac{\vec F_\mathrm{net}}{m} = \frac{+(9.0\unit N)\ihat}{2.0 \unit{kg}} = 4.5\unitfrac{N}{kg} \ihat = 4.5\unitfrac{kg \cdot m}{s^2 \cdot kg} \ihat \ = \ 4.5 \unitfrac{m}{s^2} \ihat \]
    which (recall \autoref{s:EOM}), after acting for $1.6\unit{s}$ on an object originally at rest, would result in a final speed of
    \[ v_f = (0\unitfrac{m}{s}) + (+4.5\unitfrac{m}{s^2})(1.6\unit{s}) = 7.2 \unitfrac{m}{s} \]
\end{sample}
We can do this same kind of procedure for the case when forces are in two dimensions.\dothis{Merge \autoref{ex:2Dfa} and \autoref{ex:2Dfa2}.  Also reference the example here.}
%
\begin{example}[p]
\fcolorbox{black}{yellow!10}{\begin{minipage}{4.925in} \setlength{\parsep}{3pt}
\caption{\label{ex:2Dfa} Moving a pushed box.}
A $2.0\unit{kg}$ mass is being pushed north with $5.0\unit{N}$ and east with $4.0 \unit{N}$.  What is the acceleration?
\color{blue}

\vspace{9pt}
\begin{minipage}{3.25in}
\autoref{ex:2Dforce}  (pg.~\pageref{ex:2Dforce}) already found the net force to be
\[ \vec F_\mathrm{net} = 4.0\unit{N} \ihat + 5.0\unit{N} \jhat \]
which is $F_\mathrm{net}=6.4\unit N$ at $51^\circ$ N of E.
What remains is to find the acceleration.  Since this is a vector, we can either find the components from the components of the net force
\end{minipage}
\hfill
\fbox{\begin{minipage}{1.5in}
\begin{FBD}{10}{15}{15}{40}{object}
\oneup{50}{$5\unit N$}{lightgray}
\oneri{40}{$4\unit N$}{lightgray}
\oneur{38}{48}{4}{5}{$F_\mathrm{net}$}{black}
\end{FBD}
\end{minipage}}
\vspace{2pt}
%
\[ \vec a =\frac{\vec F_\mathrm{net}}{m} = \frac{4.0\unit{N} \ihat + 5.0\unit{N} \jhat}{2.0\unit{kg}}  = 2.0\unitfrac{m}{s^2} \ihat + 2.5\unitfrac{m}{s^2} \jhat \]
or we can use the magnitude of the net force to find the magnitude of the acceleration
\[ a = \frac{6.4\unit N}{2.0\unit{kg}} = 3.2 \unitfrac{m}{s^2} \]
and know that the direction of the acceleration is the same as the acceleration of the net force: $51^\circ$ N of E.

You should notice that you can also use the components of the acceleration to find the magnitude and direction of the acceleration.
\magdir{+2.0\unitfrac{m}{s^2}}{+2.5\unitfrac{m}{s^2}}{a}{\sig{3.2}{0}{N}}{\theta}{\sig{51}{}{^\circ}}{N of E}

\color{black}
\end{minipage}}
\end{example}
%
\begin{example}[p]
\fcolorbox{black}{yellow!10}{\begin{minipage}{4.925in} \setlength{\parsep}{3pt}
\caption{\label{ex:2Dfa2} Moving a box pushed by three forces.}
A $2.0\unit{kg}$ mass is being pushed north with $5.0\unit{N}$ and east with $4.0 \unit{N}$.  What is the acceleration?
\color{blue}

\vspace{9pt}
\begin{minipage}{3.25in}
\autoref{ex:2Dforce2}  (pg.~\pageref{ex:2Dforce2}) already found the net force to be
\[ \vec F_\mathrm{net} = 3.8\unit{N} \ihat + 2.2\unit{N} \jhat \]
which is $F_\mathrm{net}=4.4\unit N$ at $31^\circ$ N of W.
What remains is to find the acceleration.  Since this is a vector, we can either find the components from the components of the net force
\end{minipage}
\hfill
\fbox{\begin{minipage}{1.5in}
\begin{FBD}{10}{15}{15}{40}{object}
%\oneup{50}{$5\unit N$}{black}
%\put(24,56){\vector(-1,2){22.36}}
%\put(0,102){\color{black} \tiny $5\unit N$}
\oneul{22}{48}{-1}{2}{$5 \unit N$}{lightgray}
\onedl{58}{22}{-5}{-2}{$6 \unit N$}{lightgray}
\oneri{40}{$4\unit N$}{lightgray}
\oneul{53}{22}{-5}{3}{$F_\mathrm{net}$}{black}
\end{FBD}
\end{minipage}}
\vspace{2pt}
%
\[ \vec a =\frac{\vec F_\mathrm{net}}{m} = \frac{3.8\unit{N} \ihat + 2.2\unit{N} \jhat}{2.0\unit{kg}}  = 1.9\unitfrac{m}{s^2} \ihat + 1.1\unitfrac{m}{s^2} \jhat \]
or we can use the magnitude of the net force to find the magnitude of the acceleration
\[ a = \frac{4.4\unit N}{2.0\unit{kg}} = 2.2 \unitfrac{m}{s^2} \]
and know that the direction of the acceleration is the same as the acceleration of the net force: $31^\circ$ N of W.

You should notice that you can also use the components of the acceleration to find the magnitude and direction of the acceleration.
\magdir{+1.9\unitfrac{m}{s^2}}{+1.1\unitfrac{m}{s^2}}{a}{\sigfrac{2.2}{}{m}{s^2}}{\theta}{\sig{31}{}{^\circ}}{N of W}

\color{black}
\end{minipage}}
\end{example}
%

\hypertarget{d:m=f/a}{In} \ref{se:netF-a}, we used the forces to find the acceleration.  It is also possible to use the forces to find the mass of an object, as follows:
\begin{sample}
\item\label{se:netF-m} If the forces in \ref{se:netFsub} are applied to an object with unknown mass and produce an acceleration of $3.2\unitfrac{m}{s^2}$, then what is the mass of the object?

    Naively, one might consider $\deq m = \frac{\vec F_\mathrm{net}}{\vec a}$, but it does not make mathematical sense to \hyperlink{d:dividevectors}{divide vectors}.  In this case, you \textit{must} consider the magnitudes of force and acceleration, knowing that their directions are the same. (We are \textit{not} ``cancelling'' the directions.)
    \[ m =\frac{F_\mathrm{net}}{a} = \frac{9.0\unit N}{3.2 \unitfrac{m}{s^2}} = \sigfrac{2.8}{1}{N \cdot s^2}{m} = 2.8\unitfrac{kg \cdot m \cdot s^2}{s^2 \cdot m} \ = \ 2.8 \unit{kg} \]
\end{sample}
\hypertarget{d:usesofF=ma}{Yet another example}\foreshadow{} of using this equation can be seen in many bathrooms.  The scale that people stand on uses a spring (introduced in \autoref{ss:scales} and discussed in detail in \autoref{s:springs}) to adjust the force provided until your acceleration is zero (placing you in equilibrium) and then tells you the force it needed to balance your weight.

It will be easier to visualize these ideas when we introduce the tool of a free-body diagram in \autoref{sss:FBD}.

\subsubsection{Equilibrium}\label{sss:equilibrium}\index{Equilibrium}\mmultireturn{\mmr{\ref{se:equi}}, \mmr{\hyperlink{d:atrestinmotion}{Newton's first law}}, \mmr{\autoref{d:Fgball}}}

This word can be traced back to Latin and Old English with the prefix \textit{equi-} for \textbf{equal} and the root \textit{libra} referring to a \textbf{pair of scales, as in a balance}, such as those depicted in images of the astronomical constellation Libra.  When the scales are equal, they are in equilibrium.  Since the second law asks us to calculate the sum of the forces acting on an object, one of the primary questions is to determine if those forces balance each other.  In \ref{se:netFadd} and \ref{se:netFsub}, the forces are not balanced, the object ``is not in equilibrium'', and it will be accelerated in a particular direction.  In the \ref{se:equi}, the forces are balanced, the object ``is in equilibrium'', and it will \textit{not change} its velocity (in accord with the first law).
\important{An object in equilibrium has $\vec F_\mathrm{net} = 0\unit N$ and $\vec a =0$.}

\subsubsection{How the Laws Work Together}\label{sss:NIItogether}\mautoreturn{sss:NItogether}

When forces act on an object, Newton's second law applies, so we usually start with the second law.  If those forces combine to give a net force of zero, such that the object is in equilibrium, then Newton's first law applies.  If we also care about the person or thing pushing, then the third law also applies.

To better understand how the first and second laws work together, \autoref{irl:NI} (pg.~\pageref{irl:NI}) provides some activities that you can do or consider in order to think about the patterns you can see when you are or aren't pushing on objects.  \autoref{cyoa:NI} (pg.~\pageref{cyoa:NI})
%
\begin{adventure}[bhpt]\fcolorbox{black}{blue!10}{\begin{minipage}{4.925in}\caption{\label{cyoa:NI} Out of gas}
On a long road trip with your friend \studentB\index{\studentB}, your car starts to sputter as it runs out of gas shortly before arriving in a new town.  You see a sign for a gas station in the distance and have to decide what to do.  You and \studentB\ can think of three options.
\begin{CYOA}
\item\label{c:parkandwalk} Pull over, park the car, walk to the gas station, buy a gas can, fill it up, carry it back to the car, and drive on!   If you follow this plan, then read \ref{a:parkandwalk}.
\item\label{c:coastindrive} Leave the car in drive, continue holding the gas-pedal down until there is absolutely no gas, and hope against all hope that you get the car to the gas station so that nobody needs to carry a heavy gas can. If you follow this plan, then read \ref{a:coastindrive}.
\item\label{c:coastinneutral} Speed up to just over the speed limit, put the car in neutral, turn on your blinking hazard-lights, coast as far as you can possibly coast, and hope against all hope that you get the car to the gas station so that nobody needs to carry a heavy gas can. If you follow this plan, then read \ref{a:coastinneutral}.
\end{CYOA}
\autoreturn{sss:NIItogether}
\end{minipage}}
\end{adventure}
%
will help you think through some of the consequences of the first and second law.  When you are ready to solve some problems, you can jump to \autoref{s:NewtonExamples}, but some of those examples will also reference Newton's third law.

\subsection{Translating Newton's Third Law: Action \& Reaction}\label{ss:NIII}\mmultireturn{\mmr{\hyperlink{d:Newtonahead}{how to describe forces}}, \mmr{\autoref{ex:braced}}, \mmr{\autoref{ex:unbraced}}}

%\important{} is not designed to start a new paragraph
\ \vspace{-12pt}
%\begin{quote}
\important{\textbf{Newton's Third Law}\index{Newton!Third Law}: $\overbrace{\mbox{For every force acting}}^{\mbox{\scriptsize ``For every action''}}$ \textit{on} one object \textit{by} an \underline{other object}, there is an equal-in-magnitude reaction-force acting \textit{on} the \underline{other object} in the opposite direction.}
%\end{quote}
This law is often shortened to ``For every action, there is an equal and opposite reaction.''  The statement given above is meant to emphasize several points:
\begin{itemize}\itemsep 0pt
\item These ``actions'' are specifically forces.
\item Forces are an interaction in which the acting force is \underline{on one object} by another and necessitates that there is a reaction force on the other object by the one.  That is to say, an object cannot feel a force without also exerting a force back on the other object.
\end{itemize}
Another way to say this is that all forces come in action/reaction pairs that necessarily have equal magnitude and opposite direction and necessarily act on different objects.
This law is the force-version of\Foreshadow{\protect{\hyperref[s:conservemom]{Conservation of momentum}}}{} the statement of the conservation of momentum, which will be discussed in \autoref{c:momentum}.

\hypertarget{d:NIIIbracing}{Let's take this apart} and connect it to your daily experiences.  Students of physics will often see the terms action and reaction and connect it to the way humans react to the actions of their friends.  However, this implies a causal response that is not true for Newton's forces.  That is to say, this is not a ``revenge law'' whereby if you push on me, then I will choose to push you back.  Instead, it is expressing that forces are intrinsically interactions between a pair of objects.  When you push on me, I am -- independent of my choosing to do so -- necessarily pushing back on you.  But, you might say, ``if that were true, then why am I able to sneak up on you and push you over without falling over myself?''  Well, you can think about how that works by reading \autoref{cyoa:NIII} (pg.~\pageref{cyoa:NIII}). After we introduce the tool of a free-body diagram in the next subsection, you can also explore this idea by comparing \autoref{ex:braced}  (pg.~\pageref{ex:braced}) to \autoref{ex:unbraced}  (pg.~\pageref{ex:unbraced}).

\subsubsection{The Free-Body Diagram (FBD)}\label{sss:FBD}\mmultireturn{\mmr{\ref{se:equi}}, \mmr{\hyperlink{d:usesofF=ma}{uses of $F=ma$}}}\index{Free-Body Diagrams}

In the more interesting situations where there are several forces acting, it can be easy to lose track of what is pushing whom where.  In order to better organize our information and direct our attention, we can make use of \textit{free-body diagrams}.  The basic idea is to make a diagram for each individual object that we care about in a given situation, and \textit{free} from the overall picture.  This allows us to identify the forces acting on a single object (relevant for Newton's second law) and more easily pair them with third-law pairs that act on different objects.

To see how this works, the next simple example will build on the previous simple examples to help us consider not only the (2nd law) forces on the object, but also the (3rd law) forces on the people doing the pushing and pulling.
\begin{sample}
\item\label{se:FBD-AB}\mautoreturn{f:firstFBD} As you revisit \ref{se:netFadd}\dothis{As before, make this object a desk so that we can have \studentA\ and \studentB\ helping you rearrange your room in your residence hall.  (Change the mass!  But, it is still nice to be moving a small mass so that it has a large acc, and the people have small acc.)}, imagine that \studentA\index{\studentA} is exerting $\vec F_1 = +5.0\unit{N}\ihat$ and \studentB\index{\studentB} is exerting $\vec F_2 = +4.0\unit{N}\ihat$.  \ref{se:netF-a} showed how the object moved (because Newton's second law focuses on the object to which the forces are applied).  Newton's third law tells us about the interaction between objects and, from this, we can say the following. In order to better describe the situation, let's assume \studentA\ is to the left of the object, pushing it to the right, and \studentB\ is to the right of the object, pulling it to the right.  See \autoref{f:firstFBD} on page~\pageref{f:firstFBD}.
    \begin{enumerate}
    \item Since \studentA\ exerts a force of $5.0\unit{N}$ to the right $(+\ihat)$ on the object, the third law reminds us that the object exerts a force of $5.0\unit{N}$ to the left  $(-\ihat)$ on \studentA.  Since \studentA\ has a mass of \massA, the second law reminds us that \heA\ is accelerated at the rate of
        \[ \vec a_1 = \frac{-5.0\unit{N} \ihat}{\massA} = -0.0\sigfrac{58}{8}{m}{s^2}\ihat = -5.9\ten{-2}\unitfrac{m}{s^2} \ihat\]
        which is to the left with a small enough value that it is easy for \himA\ to brace against.   Even if \heA\ doesn't brace, if \heA\ starts from rest, \heA\ will only be moving \\
        $v_{1f} = (0\unitfrac{m}{s})+(-0.0\sigfrac{58}{8}{m}{s^2})(1.6\unit s) = -0.0\sigfrac{94}{1}{m}{s}$.
    \item Since \studentB\ exerts a force of $4.0\unit{N}$ to the right on the object, the third law reminds us that the object exerts a force of $4.0\unit{N}$ to the left on \studentB. Since \studentB\ has a mass of \massB, the second law reminds us that \heB\ is accelerated at the rate of
        \[ \vec a_2 = \frac{-4.0\unit{N} \ihat}{\massB} = -0.0\sigfrac{53}{3}{m}{s^2}\ihat = -5.3\ten{-2}\unitfrac{m}{s^2}\]
        which is also to the left with a small enough value that it is easy for \himB\ to brace against.  Even if \heB\ doesn't brace, if \heB\ starts from rest, \heB\ will only be moving \\
        $v_{2f} = (0\unitfrac{m}{s})+(-0.0\sigfrac{53}{3}{m}{s^2})(1.6\unit s) = -0.0\sigfrac{85}{3}{m}{s}$.
    \end{enumerate}
\end{sample}
There are \hypertarget{d:FBD-AB}{a couple of important aspects} to take away from \ref{se:FBD-AB}:, especially as it builds on \ref{se:netFadd} (which told us about the forces on the object) and \ref{se:netF-a} (which told us
\begin{minipage}{3.25in}
about how the object moved).  The earlier examples were relevant to the 2nd law and only affected the object itself.  This information is captured in the middle free-body diagram of \autoref{f:firstFBD} and reproduced here.
\end{minipage}
\hfill
\fbox{\begin{minipage}{1.5in}
\begin{FBD}{10}{15}{15}{10}{object}
\twori{50}{$5\unit N$}{black}{40}{$4\unit N$}{black}
\end{FBD}
\end{minipage}}

As indicated above, the free-body diagram also helps us visualize the third-law (action/reaction) force pairs.  By reproducing the following image\index{Free-Body Diagrams!Images} from \autoref{f:firstFBD} here and adding color, we can see that the {\color{green} green forces} form an action-reaction pair and separately the {\color{blue} blue forces} form
\newline
\begin{minipage}{\textwidth}
\fbox{\begin{minipage}{1.5in}
\begin{FBD}{10}{25}{15}{10}{\studentA}
\onele{50}{$5\unit N$}{green}
\end{FBD}
\end{minipage}}
\hfill
\fbox{\begin{minipage}{1.5in}
\begin{FBD}{10}{15}{15}{10}{object}
\twori{50}{$5\unit N$}{green}{40}{$4\unit N$}{blue}
\end{FBD}
\end{minipage}}
\hfill
\fbox{\begin{minipage}{1.5in}
\begin{FBD}{10}{20}{15}{10}{\studentB}
\onele{40}{$4\unit N$}{blue}
\end{FBD}
\end{minipage}}
\end{minipage}
an action-reaction pair.  None of these three objects are in equilibrium.

If we now \hypertarget{d:equi}{do the same thing}\index{Free-Body Diagrams!Images} with the \ref{se:equi}, but let \studentC\index{\studentC} be the person on the left {\color{green} pushing to the right} and \studentD\index{\studentD} be the person on the right {\color{blue} pushing to the left}, then we see that while the {\color{green} green forces} form an action-reaction pair showing the third-law interaction between \studentC\ and the object and while the {\color{blue} blue forces} form an action-reaction pair showing the third-law interaction between \studentD\ and the object, it
%\newline
\begin{minipage}{\textwidth}
\fbox{\begin{minipage}{1.5in}
\begin{FBD}{10}{15}{15}{10}{\studentC}
\onele{30}{$3\unit N$}{green}
\end{FBD}
\end{minipage}}
\hfill
\fbox{\begin{minipage}{1.5in}
\begin{FBD}{10}{15}{15}{10}{object}
\oneri{30}{$3\unit N$}{green}
\onele{30}{$3\unit N$}{blue}
\end{FBD}
\end{minipage}}
\hfill
\fbox{\begin{minipage}{1.5in}
\begin{FBD}{10}{15}{15}{10}{\studentD}
\oneri{30}{$3\unit N$}{blue}
\end{FBD}
\end{minipage}}
\end{minipage}
is the combination of the {\color{blue} blue} and the {\color{green} green} forces, which only act on the object itself, that coincidentally cancel to leave the object in (second law) equilibrium.  In these images, \studentC\ and \studentD\ are \textit{not} in equilibrium.

To say this more specifically, the forces within one free-body diagram are described by Newton's second law. They do get added together to form the net-force (which is to say that we can add the object's green force to the object's blue force). They are able to cancel each other if they \textit{happen to} be equal in magnitude and opposite in direction. Finally, they will determine how that specific object accelerates.  On the other hand, Newton's third law describes any specific pair of forces that interact between free-body diagrams (each colored pair); they \textit{will necessarily be} equal in magnitude and opposite in direction, but they cannot be canceled because they cannot be added because they are on different objects.

\begin{figure}
\hrule\hrule
\caption{\label{f:firstFBD} A couple of people push a box.}\index{Free-Body Diagrams!Images}
First we will draw a picture of the situation described by \ref{se:FBD-AB}, which builds on \ref{se:netFadd}.  \studentA\index{\studentA} stands to the left of the object and exerts a force (pushes) to the right.  \studentB\index{\studentB} stands to the right of the object and exerts a force (pulls) to the right.  Both of these forces are \textit{on} the object and, by Newton's second law cause it to accelerate (as described in \ref{se:netF-a}).  By Newton's third law, we can learn about the forces on \studentA\ and \studentB.

\noindent
{}\hfill
\begin{minipage}{3.5in}
\begin{picture}(200,100)(-30,-25)
% Dimensions and offset: (width,height)(x offset,y offset)
% Insert picture commands (\line,\circle, etc...) here:
\put(0,0){\line(1,0){200}}
\put(60,2){\line(1,0){60}}
\drawbox{30}{1}{20}{50} %\studentA
\drawbox{50}{25}{18}{5} %\studentA's arms
\drawbox{70}{3}{20}{30} % object
\drawbox{150}{1}{20}{40} %\studentB
\drawbox{134}{25}{16}{5} %\studentB's arms
\put(90,27.5){\oval(2,2)[r]}
\put(91,27.5){\line(1,0){43}}
\put(30,53){\scriptsize \studentA}
\put(70,35){\scriptsize object}
\put(150,43){\scriptsize \studentB}
\put(60,-12){\begin{minipage}{60pt}
\scriptsize The object is on a sheet of ice.
\end{minipage}}
\end{picture}
\end{minipage}
\hfill {}

\noindent
Now we will draw a free-body diagram for each individual.  Notice that each \textit{free-body} diagram is on its own, free from the rest of the picture.  These diagrams will be discussed further in the \protect{\hyperlink{d:FBD-AB}{discussion about \ref*{se:FBD-AB}}}.

\noindent % \textwidth default is 5in for a book
\fbox{\begin{minipage}{1.5in}
\begin{FBD}{10}{25}{15}{10}{\studentA}
\onele{50}{$5\unit N$}{black}
\end{FBD}
\raggedright
Because \studentA\ pushes on the object to the right, \studentA\ feels a force \textit{on} \himA\ \textit{by} the object towards the left.
\end{minipage}}
\hfill
\fbox{\begin{minipage}{1.5in}
\begin{FBD}{10}{15}{15}{10}{object}
\twori{50}{$5\unit N$}{black}{40}{$4\unit N$}{black}
\end{FBD}
\raggedright
The object is in the middle.  Recall from \protect{\ref{se:netFadd}} that the $F_\mathrm{net} = 9.0 \unit N$ on this object.
\end{minipage}}
\hfill
\fbox{\begin{minipage}{1.5in}
\begin{FBD}{10}{20}{15}{10}{\studentB}
\onele{40}{$4\unit N$}{black}
\end{FBD}
\raggedright
Because \studentB\ pulls on the object to the right, \studentB\ feels a force \textit{on} \himB\ \textit{by} the object towards the left.
\end{minipage}}

\noindent
It turns out that this is a little oversimplified.  When we get to \protect{\autoref{s:Fg}} and \protect{\autoref{s:FN}}, we will see that we have to include a downwards force of gravity and an upwards support force.  This will be explained in \protect{\autoref{f:firstFBDupdate}} on page~\protect{\pageref{f:firstFBDupdate}}.
\flushright
\multireturn{\mmr{\autoref{ex:braced}}, \mmr{\autoref{ex:unbraced}}, \mmr{\hyperlink{d:rope.net}{rope-tension}}}
\hrule\hrule
\end{figure}

\section{Examples} \label{s:NewtonExamples}\mautoreturn{sss:NIItogether}

Next, we can consider a simple interactive example that is intended to help you think about how you know a force is acting.
%
\begin{sample}
\item\label{IQ:holdbook} You hold a book a little above your desk.  When you let go, it falls and then hits your desk.
    \begin{enumerate}
    \item While you are holding it, it has no acceleration.  Are there forces acting on it?  \YN{A:hbf}{A:hbnof}
    \item While you are holding it, is it in equilibrium?  \YN{A:true1}{A:false1}
    \item After you let go and while the book falls, it accelerates downwards.  Are there forces acting on it?  \YN{A:falls}{A:falls}
    \item While it is hitting the desk, is it accelerating?  \YN{A:hitY}{A:hitN}
    \item After it has landed and is sitting on the desk, is it in equilibrium? \YN{A:landedY}{A:landedN}
    \item After it has landed and is sitting on the desk, how many forces are acting on it? \THREE{Zero}{One}{Two}{A:zero}{A:one}{A:two}
    \end{enumerate}
\end{sample}
%
Next, we can \hypertarget{d:irlNI}{consider} pushing an object across the floor in \autoref{irl:NI} (pg.~\pageref{irl:NI}) to get a different sense of observations we can make that help us recognize patterns that are due to forces we might not have thought to look for.
%
\begin{reallife}[hp]
\hspace{-.2in}
\fcolorbox{black}{green!10}{\begin{minipage}{5.29in} \center
\caption{\label{irl:NI} Pushing an Object Across the Floor}
\begin{realtable}
\dna{Push a chair across a carpet floor}
    {When you stop pushing, it stops moving.}
    {Does force cause motion? \ref{A:chair1}}
\dna{Push a chair across a tile floor}
    {When you stop pushing, it probably stops moving.}
    {Does force cause motion? \ref{A:chair2}}
\dna{Push a chair \textit{with wheels} across a tile floor, with some strength, then let it go.}
    {What happens when you stop pushing? \ref{A:chair3}}
    {If force causes motion, why does the chair move after you stop touching it?  \ref{A:chair4}}
\dna{Push a chair \textit{with wheels} across a tile floor, change your behavior after you let it go.}
    {Do your actions when you are not touching the chair have \textit{any} impact on the chair? \ref{A:chair5}}
    {Is it possible that there is a ``residual effect'' that you have on the chair after letting it go? \ref{A:chair6}}
\multidna{Newton's First Law says that if you give the chair a velocity, it should keep that velocity.}
\dna{Repeat the first three suggestions}
    {Correlate the interaction-with-the-ground to the motion-after-you-push-and-release}
    {Is there a force that the chair feels after you release it?  \ref{A:chair7}}
\multidna{Newton's Second Law says that a net force will change the velocity.}
\dna{Push a chair gently across the floor}
    {A constant force (balanced by the force of friction) will move at a constant speed}
    {What if there were no friction? \ref{A:chair8}}
\dna{Push a chair forcefully across the floor}
    {A constant force (stronger than the force of friction) will accelerate the object away from your push}
    {Can you list surfaces that are essentially frictionless?}
\end{realtable}
\begin{minipage}{4.925in}
Notice in each case that you are not the only thing interacting with the chair.  The floor is also interacting with the chair.  The floor exerts a \hyperref[s:Ff]{force of friction} on the chair.  So, when you interpret how your force causes the chair to move, you \textit{must} also account for the interaction with the floor in your expectations.  We can minimize the effect of friction, by modifying the floor surface.  If you have ever driven on ice and felt out of control, you might have begun to develop your Newtonian intuition.
\flushright\vspace{-12pt}
\multireturn{\mmr{\autoref{sss:NIItogether}}, \mmr{\hyperlink{d:irlNI}{\autoref*{s:NewtonExamples} reference to \autoref*{irl:NI}}}}
\end{minipage}
\end{minipage}}
\end{reallife}
%
Building on that, it is useful to also consider how human beings behave when they are pushing or getting pushed.  Because people have \textit{intention} in their actions, we subconsciously balance ourselves and we don't always recognize that we are \hypertarget{d:cyoaNIII}{doing it}.  \autoref{cyoa:NIII}  (pg.~\pageref{cyoa:NIII}) provides an interactive storyline that starts to show some of the patterns that can lead to a recognition of how we balance ourselves.
%
\begin{adventure}[bpht]
\fcolorbox{black}{blue!10}{\begin{minipage}{4.925in}
\caption{\label{cyoa:NIII} The Town Bully}
\studentZ\index{\studentZ} is the town bully.  One day, he spies a biology student, \studentC\index{\studentC}, minding \hisC\ own business studying an interesting ecological phenomenon.  At the same time, you are standing across the street chatting with your friend \studentD\index{\studentD}, who happens to be taking a psychology class.  \studentD\ has been quite fascinated lately with watching the way others interact and points out the way \studentZ\ is menacingly approaching the unsuspecting \studentC.  You both predict that \studentZ\ is going to push \studentC\ over.  \studentD\ is mesmerized by the psychological effects and you, having just learned about Newton's laws, are excited to see if this action does indeed produce a reaction.
\begin{CYOA}
\item\label{c:one} If you watch the way \studentC\ is standing before, during, and after \studentZ\ pushes \himC, then read \ref{a:NIIIaction}.
\item\label{c:two} If you watch the way \studentZ\ is standing before, during, and after \heZ\ pushes \studentC, then read \ref{a:NIIIreaction}.
\item\label{c:three} If, on the other hand, you shout a warning to \studentC\ and a criticism to \studentZ, trying to keep the incident from becoming violent, then read \ref{a:NIIIconcern}.
\end{CYOA}
\flushright
\multireturn{\mmr{\hyperlink{d:NIIIbracing}{the discussion of action-reaction forces}}, \mmr{\hyperlink{d:cyoaNIII}{\autoref*{s:NewtonExamples} reference to \autoref*{cyoa:NIII}}}, \mmr{\autoref{ex:braced}}, \mmr{\autoref{ex:unbraced}}}
\end{minipage}}
\end{adventure}
%

\begin{example}[p]
\fcolorbox{black}{yellow!10}{\begin{minipage}{4.925in}\setlength{\parskip}{3pt}
\caption{\label{ex:braced} \studentZ\index{\studentZ} intentionally braces when pushing \studentC\index{\studentC}.}
(To better understand \hyperref[ss:NIII]{Newton's third law}, you should compare this example to \autoref{ex:unbraced}  [pg.~\pageref{ex:unbraced}].)
\begin{quote}
\studentZ\index{\studentZ}, the \hyperref[cyoa:NIII]{town bully} (with $m_Z=\massZ$), decides to vent \hisZ\ frustration on \studentC\index{\studentC}\ for all the times that \studentC\ makes \studentZ\ look bad in class.  While \studentC\ ($m_C=\massC$) has \hisC\ back turned, \studentZ\ walks up, leans in, and shoves \studentC\ with a force of $\vec F_{C,Z} = 215\unit{N}\ihat$.  How does \underline{\studentZ}{} accelerate during this exchange?
\end{quote}
%
%\begin{quote}
%Aside: Newton's second law tells us how this affects \studentC. See \ref{se:netF-a} and homework problem \ref{hmwk:pushbrace}.
%\end{quote}
%\noindent
\textbf{What do we know?}  As usual, it is convenient to start with a picture to help decide on the appropriate coordinate system.
We can also list
\\[2pt]
\begin{minipage}{2.6in}
the information that we know.
We know $m_Z$, which is useful for relating $F_{Z,\mathrm{net}}$ to $a_Z$.
We know $m_C$, which is useful for relating $F_{C,\mathrm{net}}$ to $a_C$.  (This is not asked for, but is asked in homework problem \ref{hmwk:pushbrace}.)
We know $F_{C,Z}$, how hard \studentZ\ pushes on \studentC.
\end{minipage}
\hfill
\begin{minipage}{150pt}
\begin{picture}(150,90)(-30,-25)
% Dimensions and offset: (width,height)(x offset,y offset)
% Insert picture commands (\line,\circle, etc...) here:
\drawbox{-10}{-20}{120}{20}  % Earth
\drawbox{25}{1}{20}{50} %\studentZ
\drawbox{45}{35}{18}{5} %\studentZ's arms
 %\studentZ's legs
    \put(24,19){\line(0,-1){6}}
    \put(24,19){\line(-1,-2){9}}
    \put(15,1){\line(1,0){4}}
    \put(19,1){\line(1,2){6}}
\drawbox{65}{1}{20}{45} %\studentC
\put(25,53){\scriptsize \studentZ}
\put(65,48){\scriptsize \studentC}
\put(40,-15){\scriptsize Earth}
\put(-40,24){\begin{minipage}{58pt}
\color{blue} \scriptsize \studentZ\ braces \himselfZ. \hfill $\searrow$
\end{minipage}}
\end{picture}
\end{minipage}
%\hfill {}
We also know that \studentC\ is not bracing \himselfC\ (because \heC\ ``has \hisC\ back turned'') so he only feels one force, and that \studentZ\ is bracing \himselfZ\ (because the problem states that \heZ\ ``leans in'') so he exerts multiple forces.

\textbf{What do we want to know?}  We want to know about the forces acting on \studentZ, in order to find  $F_{Z,\textrm{net}}$ and therefore $a_Z$.

\textbf{How are these related?}  First, since \studentZ\ exerts a force on \studentC, Newton's third law tells us that \studentZ\ feels a force of \mbox{$F_{Z,C}=-215 \unit{N}\ihat$}.
Second, because \studentZ\ \textit{knew} \heZ\ was going to feel this reaction force, \heZ\ compensates by bracing \himselfZ.  This means \heZ\ chooses to exert a force of $215\unit{N}$ on the Earth in the $-\ihat$ direction, probably by putting one leg behind \himselfZ\ and pushing the ground backwards with \hisZ\ foot.  Newton's third law then tells us that \studentZ\ feels a force of \mbox{$F_{Z,C}=+215 \unit{N}\ihat$} from the ground.

{}\hfill {\footnotesize \autoref*{ex:braced} continued on next page\ldots}
\end{minipage}}
\end{example}
\begin{example}[p]
\fcolorbox{black}{yellow!10}{\begin{minipage}{4.925in}\setlength{\parskip}{3pt}
{\footnotesize \autoref*{ex:braced} continued from previous page\ldots}

\textbf{Free-Body Diagrams:}  We are told of the force on \studentC.  We are told that \studentZ\ braces \himselfZ, which implies the force on the Earth. Newton's third law then helps us recognize the forces on \studentZ.  (Recall the \hyperlink{d:interaction}{``on-by'' notation}.)

\noindent % \textwidth default is 5in for a book
\fbox{\begin{minipage}{2.25in}
\begin{FBD}{10}{25}{15}{10}{\studentZ}
\onele{50}{$F_{Z,C}=215\unit N$}{black}
\oneri{50}{$F_{Z,E}=215\unit N$}{blue}
\end{FBD}
\vspace{-10pt}
\raggedright
\studentZ\ is pushed by \studentC\ to the left.  \studentZ\ is pushed to the right by the Earth.
\end{minipage}}
\hfill
\fbox{\begin{minipage}{2.25in}
\begin{FBD}{10}{23}{15}{10}{\studentC}
\oneri{50}{$F_{C,Z}=215\unit N$}{black}
\end{FBD}
\vspace{-10pt}
\raggedright
\studentC\ feels \studentZ\ push to the right.
\end{minipage}}
% \\
\fbox{\begin{minipage}{4.75in}
\begin{FBD}{60}{10}{15}{10}{Earth}
\onele{50}{$F_{E,Z}=215\unit N$}{blue}
\end{FBD}
\vspace{-10pt}
\raggedright
Earth feels a force by \studentZ\ to the left.
\end{minipage}}

\textbf{Concepts to Consider:}  Newton's third law guarantees that the action-reaction force pairs, such as $F_{Z,C}$ and $F_{C,Z}$ or $F_{Z,E}$ and $F_{E,Z}$, are equal and opposite.  There is no such guarantee on $F_{Z,C}$ and $F_{Z,E}$.  These are equal because \studentZ\ chose to make $F_{C,Z}$ and $F_{E,Z}$ equal.  \HeZ\ pushed on the two others in equal amounts so that the reaction forces that act on \himZ\ will balance for Newton's \textit{second} law so that \hisZ\ acceleration would be zero.

\textbf{Solution to the example:}  After using Newton's third law to find the forces on \studentZ, we can use Newton's second law to find \hisZ\ acceleration:
\[ a_Z = \frac{F_{Z,\mathrm{net}}}{m_Z} = \frac{\left[ \vec F_{Z,C} + \vec F_{Z,E} \right]}{\massZ} = \frac{\left[ \left( -215\unit N \ihat \right) + \left( +215\unit N \ihat \right) \right]}{\massZ} = 0 \unitfrac{m}{s^2}  \]

%\begin{quote}
\textbf{Aside:} This example only considers the left-right forces that act in order to make a point about our intuition regarding forces we intend to apply.  Please consider how \protect{\autoref{f:firstFBDupdate}} updates \autoref{f:firstFBD} to make yourself aware of the other forces that are acting here, but are being ignored.
%\end{quote}

\linkreturn[action-reaction]{d:NIIIbracing}
\end{minipage}}
\end{example}

\begin{example}[p]
\fcolorbox{black}{yellow!10}{\begin{minipage}{4.925in}\setlength{\parskip}{3pt}
\caption{\label{ex:unbraced} \studentD\index{\studentD} does not brace \himselfD\ when pushing \studentC\index{\studentC}.}
(To better understand \hyperref[ss:NIII]{Newton's third law}, you should compare this example to \autoref{ex:braced}  [pg.~\pageref{ex:braced}].)
\begin{quote}
In the lab room one day, while waiting for the instructor, \studentD\index{\studentD} (who has a mass of $m_D=\massD$) decides to try a physics experiment to test Newton's third law.  \HeD\ politely asks \hisD\ lab partner, \studentC\index{\studentC} ($m_C=\massC$), to turn \hisC\ back while \heD\ squares his feet underneath \himselfD\ and pushes with a force of $\vec F_{C,D} = 215\unit{N}\ihat$.  Despite the experience of \autoref{ex:braced} (as told in \autoref{cyoa:NIII}  [pg.~\pageref{cyoa:NIII}]), \studentC\ reluctantly agrees.  How does \underline{\studentD}{} accelerate during this exchange?
\end{quote}
%
%\begin{quote}
%Aside: Newton's second law tells us how this affects \studentC. See \ref{se:netF-a} and homework problem \ref{hmwk:pushbrace}.
%\end{quote}
%\noindent
\textbf{What do we know?}  As usual, it is convenient to start with a picture to help decide on the appropriate coordinate system.
We can also list
\\[2pt]
\begin{minipage}{2.6in}
the information that we know.
We know $m_D$, which is useful for relating $F_{D,\mathrm{net}}$ to $a_D$.
We know $m_C$, which is useful for relating $F_{C,\mathrm{net}}$ to $a_C$.  (This is not asked for, but is asked in homework problem \ref{hmwk:pushbrace}.)
We know $F_{C,D}$, how hard \studentD\ pushes on \studentC.
\end{minipage}
\hfill
\begin{minipage}{150pt}
\begin{picture}(150,90)(-30,-25)
% Dimensions and offset: (width,height)(x offset,y offset)
% Insert picture commands (\line,\circle, etc...) here:
\drawbox{-10}{-20}{120}{20}  % Earth
\drawbox{25}{1}{20}{40} %\studentD
\drawbox{45}{25}{18}{5} %\studentD's arms
\drawbox{65}{1}{20}{45} %\studentC
\put(25,43){\scriptsize \studentD}
\put(65,48){\scriptsize \studentC}
\put(40,-15){\scriptsize Earth}
\put(-40,24){\begin{minipage}{58pt}
\color{blue} \raggedright \scriptsize \studentD\ does not brace \himselfD. \\\hfill $\searrow$
\end{minipage}}
\end{picture}
\end{minipage}
%\hfill {}
\\[2pt]
We also know that neither person is bracing for the push. So, both \studentC\ and \studentD\ each only feel one force.

\textbf{What do we want to know?}  We want to know about the forces acting on \studentD, in order to find  $F_{D,\textrm{net}}$ and therefore $a_D$.

\textbf{How are these related?}  First, since \studentD\ exerts a force on \studentC, Newton's third law tells us that \studentD\ feels a force of \mbox{$F_{D,C}=-215 \unit{N}\ihat$}.
Second, unlike \studentZ\ in \autoref{ex:braced}  (pg.~\pageref{ex:braced}), \studentD\ chooses not to exert a force on the Earth in the $-\ihat$ direction.

\textbf{Free-Body Diagrams:}  We again draw free-body diagrams:

\noindent % \textwidth default is 5in for a book
\fbox{\begin{minipage}{2.25in}
\begin{FBD}{10}{20}{15}{10}{\studentD}
\onele{50}{$F_{D,C}=215\unit N$}{black}
\end{FBD}
\vspace{-10pt}
\raggedright
\studentD\ is pushed by \studentC\ to the left.
\end{minipage}}
\hfill
\fbox{\begin{minipage}{2.25in}
\begin{FBD}{10}{23}{15}{10}{\studentC}
\oneri{50}{$F_{C,D}=215\unit N$}{black}
\end{FBD}
\vspace{-10pt}
\raggedright
\studentC\ feels \studentD\ push to the right.
\end{minipage}}

{}\hfill {\footnotesize\autoref*{ex:unbraced} continued on next page\ldots}
\end{minipage}}
\end{example}
\begin{example}[p]
\fcolorbox{black}{yellow!10}{\begin{minipage}{4.925in}\setlength{\parskip}{3pt}
{\footnotesize \autoref*{ex:unbraced} continued from previous page\ldots}

\textbf{Concepts to Consider:}  Newton's third law guarantees that the action-reaction force pairs, $F_{D,C}$ and $F_{C,D}$, are equal and opposite.  Because these forces are not on the same person, we cannot add these forces.  Newton's second law will then indicate how each person accelerates.

\textbf{Solution to the example:}  After using Newton's third law to find the forces on \studentD, we can use Newton's second law to find \hisD\ acceleration:
\[ a_D = \frac{F_{D,\mathrm{net}}}{m_D} = \frac{\left[ \vec F_{D,C} \right]}{\massD} = \frac{\left[ \left( -215\unit N \ihat \right) \right]}{\massD} = -\sigfrac{2.68}{75}{m}{s^2} \ihat \]

%\begin{quote}
\textbf{Aside:} This example only considers the left-right forces that act in order to make a point about our intuition regarding forces we intend to apply.  Please consider how \protect{\autoref{f:firstFBDupdate}} updates \autoref{f:firstFBD} to make yourself aware of the other forces that are acting here, but are being ignored.
%\end{quote}
\flushright
\multireturn{\mmr{\hyperlink{d:NIIIbracing}{the discussion of action-reaction forces}}, \mmr{\autoref{ex:braced}}}
\end{minipage}}
\end{example}

\section{Summary and Homework}

\subsection{Summary of Concepts and Equations}

This chapter introduced the way physicists describe forces.  The concept of force encodes how objects interact.
After reading this chapter, you should be comfortable responding to the following questions or comments.
Unlike the other links in this book, if you follow the links in this summary section, there is no link to return to this page.  (This is on purpose to encourage you to answer these points without following these links.)
\begin{itemize}
\item State Newton's Laws. \hyperlink{sum:Newton'sLaws}{(Answer)}
\item How is the unit of Newton related to the fundamental units of the SI system?  \hyperref[sss:unit-N]{(Answer)}
\item How do you know when a system is in equilibrium? \hyperref[sss:equilibrium]{(Answer)}
\item You should know how to draw a free-body diagram.  \hyperref[f:firstFBD]{(Example)}
\end{itemize}

\subsection*{Conceptual Questions}\dothis{Add more conceptual questions}
%\vspace{-24pt}
\begin{enumerate}
\item In order to climb a tree, you reach up and grab a branch and pull.  Most people refer to this as ``pulling yourself up.'' In terms of Newton's third law, describe what is happening in more technical terms.
\item Some cars have a ``cruise-control'' feature that keeps your speed constant as you drive down the highway.  (a) If you are driving due north with the cruise-control on, are you in equilibrium?  (b) If, instead, you have the cruise-control set while you are following the road around a gradual curve of the road as it follows the shore of a lake, then are you in equilibrium?  (c) In both cases, how can you tell if you are in equilibrium?
\end{enumerate}
\subsection*{Problems}\dothis{Add more variety of problems.}
%\vspace{-24pt}
\begin{enumerate}
 \item\label{hmwk:pushbrace} If \studentZ, with $m_Z=\massZ$, braces \himselfZ\ (so that he does not accelerate) and pushes \studentC\ ($m_C=\massC$) with a force of $\vec F_{C,Z} = 215\unit{N}\ihat$, find the following:
\begin{enumerate}
    \item What is the acceleration of \studentC?  \answer{\mbox{$\deq\vec a_C = \frac{215 \unit{N}\ihat}{\massC} = \sigfrac{2.38}{9}{m}{s^2} \ihat$.}}
    \item What net force does \studentZ\ feel? \answer{$F_{Z,\mathrm{net}}=0\unit N$}
    \item If \studentZ\ braces \himselfZ\ against the Earth, then what must that bracing force be?  \answer{$\vec F_{E,Z} = -215\unit{N}\ihat$}
    \item What are the individual forces that \studentZ\ feels? \answer{$F_{Z,C}=-215\unit N \ihat$ and $F_{Z,E}=215\unit N \ihat$}
    \item What is the acceleration of the Earth?  \answer{\mbox{$\deq\vec a_E = \frac{-215 \unit{N}\ihat}{5.97\ten{24}\unit{kg}} = -\sigfrac{3.60}{1\ten{-23}}{m}{s^2} \ihat$.}}
    \item Which of Newton's laws allows you to answer each of these questions?
\end{enumerate}
\item If you apply a force of $4.65\unit N$ to a mass of $2.18\unit{kg}$, then how much will it accelerate?
\item How much force must you apply to cause a mass of $80.0\unit{kg}$ to accelerate at $a=0.795\unitfrac{m}{s^2}$?
\item You arrive home to find a box that came in the mail.  You find that you have to exert $54.3\unit N$ to cause it to accelerate $a=1.25\unitfrac{m}{s^2}$.  (a) What is its mass?  (b) Is that a heavy box or a light box?  (c) Is it likely that this box would fit in a mailbox?
\item Your $2538 \unit{kg}$ car has run out of gas.  So you ask your friend, \studentB{} who has a mass of $\massB$, to put it in neutral, sit inside, and steer while you push.  If you apply enough force to cause a net forward force of magnitude $37.5\unit N$, how much time will it take for the car to move faster than you can walk?  Assume your walking speed is $3.0\unitfrac{mi}{hr}$.  How far will the car have travelled in that time?
\item Find the components of the net force on a large crate if three forces are applied: $\vec F_1 = -3.0\unit N \ihat + 2.5 \unit N \jhat$, $\vec F_2 = -6.25\unit N \jhat$, and $\vec F_3 = 4.5\unit N \ihat + 1.63 \unit{N} \jhat$.
\item Find the components of the net force on a large crate if three forces are applied: $F_1 = 3.61 \unit N $ at $71.6^\circ$ north of east, $F_2 = 4.61\unit N$ due west, and $F_3 = 8.13\unit N$ at $21.8^\circ$ south of east.
\item Find the magnitude and direction of the net force on a large crate if three forces are applied: $\vec F_1 = 4.25\unit N \ihat - 4.66 \unit N \jhat$, $\vec F_2 = -2.65\unit N \jhat$, and $\vec F_3 = -5.4\unit N \ihat + 2.93 \unit{N} \jhat$.
\item Find the magnitude and direction of the net force on a large crate if three forces are applied: $F_1 = 2.65 \unit N $ at $26.6^\circ$ north of west, $F_2 = 2.22\unit N$ at $56.31^\circ$ south of west, and $F_3 = 7.12\unit N$ at $28.4^\circ$ north of east.
\end{enumerate}



\chapter{The Many Types of Force}\label{c:forcetype}\mlinkreturn[subscript notation of forces]{d:interaction}

\section{Gravity at the Surface of the Earth}\label{s:Fg}\mmultireturn{\mmr{\hyperlink{d:accgrav}{freefall}}, \mmr{\autoref{f:firstFBD}}}\new{v2.2}{Adding detail}

Perhaps the force that is the most obvious to humanity is the one that helps us fall when we stumble: the gravitational force\index{Gravity!Surface of Earth}.  This is one of the fundamental forces discussed in \autoref{s:fundamental}.  In addition, the details about how the planets, moon, and the sun experience this force will be discussed in \autoref{c:gravity}.  For now, we can consider how this interaction manifests itself on our daily lives.  In this section, we will start with how objects move when the gravitational force is the only force acting.  Subsections~\ref{ss:weightmass} and~\ref{ss:equivmm} will clarify some subtleties and then we'll jump into the examples in \autoref{ss:local.mg}.

We can investigate what happens when the gravitational force is the only force acting on an object by holding it in the air and dropping it\index{Freefall}.  One of the complications during such an experiment was discussed in \autoref{ss:airresistance}.  If we drop a sheet of paper, there is air resistance in addition to the gravitational force.  For this section, I will assume that the mass-to-surface-area ratio is large enough that we can effectively\Touchstone{Recall \protect{\hyperref[s:effective2]{effective theories}}.}{} ignore the air resistance.

Since objects fall faster than humans are used to paying attention to, the \hypertarget{d:Fgrav}{patterns} are difficult to see.  The green box of \autoref{irl:freefall} (on page~\pageref{irl:freefall}) shows you how you can pay close attention to the patterns that result from observing falling objects.
You should go do those experiments before reading further.  Go ahead.  I'll wait.

You did do them, right?  You're not just reading ahead?  Really?  OK.  Doing that experiment will help you see (1) that everything falls at the same rate and (2) that objects accelerate as they fall\phantomsection\label{d:Fgball}.  This first point is a bit less intuitive and will be discussed further in \autoref{ss:equivmm}.  This second point should be exactly what you expect, when you consider \hyperref[ss:NII]{Newton's second Law}: If there is only one force (the gravitational force), then the object cannot be in \hyperref[sss:equilibrium]{equilibrium} and it must be accelerating.  (You should notice that this is the language of \hyperref[st:F=ma]{the story of Newton's second law}.)

In order to evaluate this further, let's consider a specific object, like a baseball.  Our baseball has a mass of $m_b = 0.145\unit{kg}$.  If the only force acting \textit{on} the ball is the gravitational force \textit{by} the Earth, then the net force is the gravitational force: $\vec F_\mathrm{net} = \vec F_{bEg}$\Touchstone{\hyperlink{d:interaction}{the on-by notation}}.  Here the subscripts are $b$ (because the force is on the \underline{b}all), $E$ (because the force is exerted by the \underline{E}arth), and $g$ (because it is a \underline{g}ravitational force).  Since the acceleration is due to the gravitational force, I will use either $a_g$ (usually when the object is in \hyperref[ss:freefall]{freefall} and therefore accelerating at this rate) or $g$ (usually when the object is not actually accelerating at that rate).  With this notation, Newton's second law becomes:  \[ \vec F_{bEg} = m_b \vec a_g \]
At this point, we know the mass, but we don't know the force or the acceleration.  However, we have conveniently already done the experiment (recall \autoref{ex:freefall}) that will tell us the acceleration is $a_g = 9.81\unitfrac{m}{s^2}$ downwards.  (Recall that ``downwards'' is the direction of the vector, which can be expressed as $-\jhat$.)  If we know the mass and the acceleration, then we can compute the force.
\begin{sample}
\item\label{se:weightball} If a baseball with mass $m_b = 0.145\unit{kg}$ is dropped (allowed to \hyperref[ss:freefall]{fall freely}) so that it accelerates at $a_g = 9.81\unitfrac{m}{s^2}$ downwards, then while it falls it feels the gravitational force:
    \[ \vec F_g = m \vec g = (0.145\unit{kg}) [-(9.81\unitfrac{m}{s^2})\,\jhat] = -\sig{1.42}{24}{N} \jhat = -1.42 \unit N \jhat \]
\end{sample}
This is the force of the gravitational force on the baseball.  Although we computed the force while the ball was falling, the gravitational force does not magically vanish when the ball is sitting on the floor.  So, we can say that (as long as the ball is close to the surface of the Earth, as noted in \autoref{c:gravity}) the force always has this value.  Rather than continuing to say ``the force of gravity'' we call this force the weight\index{Weight}.
\important{The weight of an object is computed as its mass times the acceleration due to gravity, even when the object is not actually accelerating at that rate:  $\mathbf{F_g \equiv mg}$.}

\subsection{Weight versus Mass}\label{ss:weightmass}\mmultireturn{\mmr{\autoref{ss:convertunits}}, \mmr{\autoref{s:sigfig}}}\index{Weight}\new{v2.2}{Added detail.  Moved the previous version to \protect{\autoref{s:sigfig}} to smooth the transition to \protect{\autoref{ss:equivmm}}.}

Since all objects have the same acceleration due to gravity at the surface of the Earth, the weight of an object and the mass of an object are very closely correlated, but they are not the same quantity.  This tends to cause some confusion when the discussion is not explicitly technical.  Recall the discussion about \hyperref[s:precision]{being precise in our language}.  One complication for people in the United States is that there are two definitions of the pound; one is a unit of mass\footnote{There are also multiple versions of the pound-mass.  You can find these explained on the internet, but most of these are considered obsolete.  The one I will use is the ``avoirdupois-pound'', which is defined in the NIST publication
% found in https://en.wikipedia.org/wiki/Pound_(mass)
\protect{\href{https://www.nist.gov/sites/default/files/documents/2017/04/28/AppC-12-hb44-final.pdf}{Handbook 44}}, page C-19, as exactly $453.592 37\unit{g}$.}
and the other is a unit of force.  Since the pound-force\footnote{There is also a unit of force called the kilogram-force.} is defined as the standard unit of mass times the standard unit for the acceleration due to gravity,
% https://en.wikipedia.org/wiki/Pound_(force)
as discussed in \autoref{s:SI-MKS}\dothis{Update \protect{\autoref{s:SI-MKS}} with this information.}, the conversion directly from pound-force to Newtons will \underline{not} match the longer, but more appropriate, conversion from pound-mass to kilogram that gets multiplied by the local acceleration due to gravity (as opposed to the standard $g$) into Newtons.  It may also be useful to review the comments about unit-conversion in the section on \hyperref[s:sigfig]{significant digits}\index{Significant Digits}.

In the discussion about \hyperref[s:precision]{being precise in our language}, we distinguished ``massive'' (the amount) from ``voluminous'' (the size).  Now that we understand \hyperref[ss:NII]{Newton's second law}, we can distinguish ``massive''
%(an amount of material)
from ``weighty.'' %
(a strength needed to lift).
The concept that goes with
\important{mass is the amount of material,}
whereas, the concept that goes with
\important{weight is how strongly the gravitational force pulls on the object.}
Having mass affects both the inertia (ease of moving) and the weight (force of gravity).
Having weight expresses the gravitational force due to whichever large object (moon, planet, sun, etc.) you happen to be on or near.  Noticing that the \hyperref[s:SI-MKS]{SI-unit} is different for different types of quantities, such as a kilogram (a \hyperref[ss:units]{fundamental unit}) for mass and a Newton (a \hyperref[ss:units]{derived unit}) for weight, may help you remember that these are different kinds of quantities.

The interesting aspect of this relationship is that while having more mass makes an object harder to move (the same force produces less acceleration for more massive objects), when objects fall under the influence of the gravitational force, they accelerate at the \textit{same} rate.  This reveals that the gravitational force must be stronger for more massive objects \textit{by the exact amount} needed to compensate for that larger mass.  This is called the equivalence principle and is discussed in \autoref{ss:equivmm}.


\subsection{Calculating the weight}\label{ss:local.mg}\new{v2.2}{renamed this section and added detail}

When calculating the forces acting on a person or an object, we will often need to account for the force of gravity, while other forces may also be at work.  As mentioned above, the weight is found by multiplying the mass times the local acceleration due to gravity, even if the object is not actually accelerating at that rate.  Chapter~\ref{c:gravity} will clarify why it is true\footnote{The short answer is that the altitude (distance from the surface of the Earth) and local geology affect the strength of the gravitational field.  Since the Earth is slightly oblate (bulges at the equator), the altitude at different latitudes corresponds to a different distance from the center of the Earth.  In addition, while the spin of the Earth does not affect the strength of the gravitational field, it does affect how objects accelerate. The \protect{\href{http://www2.csr.utexas.edu/grace/gallery/animations/ggm01/ggm01_gif-200.html}{GRACE project}} has measured the variations across the globe.}, but for now please note that the acceleration due to gravity is (1) different according to where we are and also (2) the same for all objects at that location.\dothis{Gather values of $g$ at various locations.  Wiki has a list, but need to find the source.  Wolfram has numbers, but they seem to be calculated off a formula, not measurements.  \protect{\href{http://www.physics.montana.edu/demonstrations/video/1_mechanics/demos/localgravitychart.html}{U Montana}} has values but no reference.
\protect{\href{http://www.calpoly.edu/~gthorncr/ME302/documents/AccuracyofGravity.pdf}{Glen Thorncroft at Cal Poly}} has a formula and lists the level of each effect.}\index{Acceleration!Gravity}\index{Gravity!Acceleration}\done{Add a table of measured values of $g$ at various locations.  Compute the weight of a specific person at various locations.}

\hypertarget{d:weightmass}{Because} of the peculiarities in the definition of pound (\autoref{ss:weightmass}) it will be useful to build some intuition about masses in terms of kilograms and Newtons.  \autoref{t:weightmass} lists the mass of some common objects and, using the standard value for $g$, their corresponding weights.
%
\begin{table}[bhtp]
\hrule\hrule
\begin{center}
\caption[Comparison of masses and weights of common objects]{\label{t:weightmass} The list of objects is intended to give a sense of scale so that the reader can better estimate the value of the mass of an object.  You might notice that (except for the apple) each of these is between 4 and 4.5 times heavier than the previous object.  Note that these are rough estimates; for example, while the author weighs about $200\unit{lbs}$ this is not typical, nor average.
%\linkreturn[weight and mass]{d:weightmass}
% reference weight of an apple:  \url{http://www.applejournal.com/ref.htm}
}
\begin{tabular}{lrrr}
Object & pounds & mass (kg) & weight (N) \\ \hline
apple & 0.33 & 0.15 & 1.5 \\
lean, healthy cat & 10 & 4.6 & 45 \\
medium-sized dog & 44 & 20 & 196 \\
human & 200 & 91 & 890 \\
horse & 1000 & 362 & $3.56\ten{3}$ \\
large pick-up truck & 4000 & $1.81\ten{3}$ & $1.78\ten{4}$
\end{tabular}
\end{center}
\hrule\hrule
\end{table}
%
\hyperref[c:weightmass]{Conceptual Problem \ref{c:weightmass}} asks you to estimate the mass of some other common objects.  \hyperref[c:massweight]{Conceptual Problem \ref{c:massweight}} asks you to think of common objects with a specified mass.

Now let's do some calculations\ldots
\begin{sample}
\item \studentA\index{\studentA} notices that \heA\ needs to exert $F=1.5\unit{N}$ to support the apple listed in \autoref{t:weightmass}. \HeA\ then drops it  and notices its acceleration of $9.81\unitfrac{m}{s^2}$.  \HeA\ computes the mass to be
    \[ m = \frac{F_g}{a_a} \ = \ \frac{1.5\unit{N}}{9.81\unitfrac{m}{s^2}} \ = \ \frac{1.5\unitfrac{kg \cdot m}{s^2}}{9.81\unitfrac{m}{s^2}} \ = \ 0.\sig{15}{3}{kg} \]
    (If you know the weight, you can compute the mass, even if the mass is not actually in freefall.)
\item\label{se:weightA} \studentA\index{\studentA}\new{v2.3}{modified and supplemented}, who knows \hisA\ own mass ($\massA$), then imagines\mmultireturn{\mmr{\autoref{f:firstFBDupdate}}, \mmr{\autoref{f:firstFBDangle}}} dropping \himselfA\ (!) from a (small) height.  While \heA\ falls, \heA\ recognizes the gravitational force on \himA, which is computed to be
    \[ \vec F_g = m \vec g = (\massA) [-(9.81\unitfrac{m}{s^2})\,\jhat] = -\sig{833}{.85}{N} \jhat = -834 \unit N \jhat \]
    Since \heA\ is in freefall and there is only one force is acting on \himA, the net force is easy to compute:  $\vec F_\mathrm{net} = -834 \unit N$.
    However, if you know the mass something, you can compute the weight even if that object is not in freefall.
    You should repeat this calculation for the mass in \ref{se:netF-a}.  (\ref{A:netF-a})
\end{sample}
You should note that
\important{$F_\mathrm{net} \ (=ma)$ is always related to the actual acceleration of the object, \\ $F_g\ (=mg)$ is always related to the local acceleration due to gravity.}
You should also note that
\important{the actual acceleration is only equal to the local acceleration due to gravity if the object is in freefall.}
\begin{sample}
\item\label{se:FNB} If\mmultireturn{\mmr{\autoref{f:firstFBDupdate}}, \mmr{\autoref{f:firstFBDangle}}} \studentB\index{\studentB} is not falling, but rather standing safely on the floor, then the gravitational force is still acting.  It can be computed as
    \[ \vec F_g = m \vec g = (\massB) [-(9.81\unitfrac{m}{s^2})\,\jhat] = -\sig{735}{.75}{N} \jhat = -736 \unit N \jhat \]
    However, since we can see that \hisB\ acceleration is zero, the $\vec F_\mathrm{net}$ \textit{must be zero}.  The only way that can happen, though is if there is another force acting upwards on \studentB.  What could possibly be pushing up on \himB?  \ref{A:floor}.  Whatever it is pushing up on \himB, it is supplying a support force, which can be calculated since $\vec F_\mathrm{net} = \vec F_g + \vec F_\mathrm{support}$ and we can solve for
    \[ \vec F_\mathrm{support} = \vec F_\mathrm{net} - \vec F_g = m\left(0\unitfrac{m}{s^2}\right) - \left[ -(\sig{735}{.8}{N}) \jhat\right] = +736\unit N \jhat \]
    Because it is in the direction opposite to $\vec F_g$, it is upwards $(+\jhat)$.

    Can you identify \textit{why} the support force is equal in magnitude and opposite in direction to the gravitational force?
    \TWO{Newton's second law}{Newton's third law}{A:second}{A:third}
\end{sample}
As was mentioned earlier, the value of the acceleration due to gravity also varies across the surface, although this is less than about a percent or so (see~\autoref{t:gworld}).
Nonetheless, this means that your weight can change even when your mass remains the same.
\begin{sample}
\item\label{se:gworld} While talking to your friend \studentB\index{\studentB}, you learn that \hisB\ parents, \studentE\index{\studentE} and \studentF\index{\studentF}, grew up in Norway, visited Puerto Rico, and climbed Mount Everest before settling in the United States.  Using \autoref{t:gworld}, compute \studentE's weight are each location, assuming \hisE\ mass is \massE.
\begin{enumerate}
\item[Norway] $F_g = mg = (\massE)(9.825\unitfrac{m}{s^2}) \ = \ \sig{933}{.4}{N}$
\item[Puerto Rico] $F_g = mg = (\massE)(9.782\unitfrac{m}{s^2}) \ = \ \sig{929}{.3}{N}$
\item[Mount Everest] $F_g = mg = (\massE)(9.763\unitfrac{m}{s^2}) \ = \ \sig{927}{.5}{N}$
\end{enumerate}
\end{sample}
%
Because the variation is small, throughout this text when we are considering situations ``at the surface of the Earth'', we will assume that
\important{the acceleration due to gravity is $9.81\unitfrac{m}{s^2}$ to three significant figures.}
%
\begin{table}[bhtp]
\hrule\hrule
\begin{center}
\caption[Comparison of $g$ at a few places on Earth]{\label{t:gworld} Comparison of $g$ at a few places on Earth.  {\color{gray} [While both the latitude-longitude and the local value of $g$ were found using the
\href{https://www.wolframalpha.com/}{WolframAlpha$^R$ computational knowledge engine},
these $g$ values do not necessarily correspond to these coordinates.  The $g$ values are based on a theoretical model of the Earth.]}
You should look for a pattern as the latitude increases.  (\ref{A:gworld})
You might notice the values for  Mount Everest and Denver; Can you explain any peculiarity?  (\ref{A:gpeaks})
\return{se:gworld}
}
\begin{tabular}{lccr}
Location & latitude & longitude & local $g (\!\!\unitfrac{m}{s^2})$ \\ \hline
San Juan, Puerto Rico & $18^\circ 26' 24'' \unit{N}$  & $66^\circ 7' 48'' \unit W$ & $9.782 \unitfrac{m}{s^2}$ \\
Brownsville, TX & $26^\circ 1' 6'' \unit{N}$  & $97^\circ 27' 14'' \unit W$ & $9.788 \unitfrac{m}{s^2}$ \\
Mount Everest & $27^\circ 59' 17'' \unit{N}$  & $86^\circ 55' 31'' \unit E$ & $9.763\unitfrac{m}{s^2}$ \\
Cincinnati, OH & $39^\circ 8' 24'' \unit{N}$  & $84^\circ 30' 23'' \unit W$ & $9.801\unitfrac{m}{s^2}$ \\
Denver, CO & $39^\circ 45' 43'' \unit{N}$  & $104^\circ 52' 50'' \unit W$ & $9.798\unitfrac{m}{s^2}$ \\
Paris, France & $48^\circ 51' 36'' \unit{N}$  & $2^\circ 20' 24'' \unit E$ & $9.813\unitfrac{m}{s^2}$ \\
Oslo, Norway & $59^\circ 54' 36'' \unit{N}$  & $10^\circ 45' \phantom{24''} \unit E$ & $9.825\unitfrac{m}{s^2}$ \\
Anchorage, AK & $61^\circ 10' 39'' \unit{N}$  & $149^\circ 16' 28'' \unit E$ & $9.826\unitfrac{m}{s^2}$
\end{tabular}
\end{center}
\hrule\hrule
\end{table}
%



\section{Fundamental Forces}\label{s:fundamental}\index{Force!Fundamental}\new{v2.1}{Started the section on fundamental interactions.  Link ahead, rather than detailling here.}

The previous section describes our (macroscopic) experience of the gravitational interaction when standing on the surface of the Earth.  This is essentially the same across the surface, but does change with altitude and the difference can be measured on mountain tops and in caves.  In fact, one can use the differences from one location to another to predict where we might find a a pocket of oil.\new{v2.2}{Filled out the detail.  Changed the approach.}

In later \hypertarget{d:fundamental}{sections}, we will consider this and other interactions that depend on the physical properties, such as mass and charge.  All particles with the property of mass (which we will start to call gravitational charge) will interact according to the gravitational force; however, this description is better described by the mathematics in \autoref{c:gravity}.  All particles with the property of electrical charge will interact according to the electrical force.  The basic theory will be discussed in \autoref{c:electric}.  A more complicated version that incorporates quantum mechanics is called quantum electrodynamics (QED) and this will be touched on in \autoref{ss:QED}.  Particles like protons and neutrons (hadrons) are actually made up of other particles (quarks) that are held together by an interaction that is sometimes called the strong nuclear force (\autoref{ss:strong}) and is described by the theory of quantum chromodynamics (QCD); this will be touched on in \autoref{ss:QCD}.  Finally, in \autoref{ss:weak} another fundamental force, called the weak nuclear force, will be discussed.

For the most part, these theories describe the interaction between microscopic particles, so we will not discuss them in detail here.  However, the gravitational interaction is exception in a variety of ways.  In particular, the gravitational interaction does affect macroscopic objects.  These fundamental forces have a particular description that allows us to pretend (recall \hyperref[s:effective2]{effective theories}) that they are action-at-a-distance interactions.  All other forces (introduced next) will require physical contact in order to exert the force.

\section{Normal Force}\label{s:FN}\mmultireturn{\mmr{\autoref{f:firstFBD}}, \mmr{\ref{A:floor}}, \mmr{\autoref{s:FT}}}\new{v2.2}{Added detail}\index{Force!Normal}

The word ``normal'' \href{http://etymonline.com/index.php?term=normal}{originates}\index{Normal} with the idea of conformity to the pattern.  While in everyday life this the typical state of being, the origins actually refer to a carpenter's square, which put corners into a right angle.  In math and physics, the word is used to mean perpendicular.  In the context of forces,
\important{the normal force is the force that a surface exerts to keep objects from passing through them.  The direction of this force is always in the outward direction, normal (perpendicular) to the surface.}

Let's consider some specific situations\ldots\inlife{} In \ref{se:FNB}\dothis{DO we need to repeat the example here? no?}, \studentB\ felt the downwards gravitational force even while \heB\ was standing on the ground.  We noticed that \heB\ was not falling (and so not accelerating).  Colloquially, we say that the ground is supporting \studentB.  This support force is keeping \studentB\ from passing through the floor; this is a normal force.  The normal force from the floor is acting upwards, which is normal (perpendicular) to the surface of the floor.  \autoref{f:firstFBDupdate} updates the free-body diagrams of \autoref{f:firstFBD} to show how the gravitational and normal forces impact that calculation.
%
\begin{figure}
\hrule\hrule
\caption{\label{f:firstFBDupdate} An updated version of \protect{\autoref{f:firstFBD}}, people pushing a box.}\index{Free-Body Diagrams!Images}
Again, we can start by drawing a picture of the situation.  The description is the same as it was for \autoref{f:firstFBD}.  In addition to those forces, each of the three bodies has a downwards gravitational force.  This analogous to the calculation in \ref{se:weightA}, which was only for \studentA\index{\studentA}; but you can calculate the weight for the mass in \ref{se:netF-a} and \studentB\index{\studentB}'s weight was computed in \ref{se:FNB}.  In addition to the downward gravitational force (the weight), Newton's second law and the fact that nothing is accelerating up or down together tells us that

\noindent
\begin{minipage}[b]{150pt}
there must also be a normal force on each body.  This is analogous to the calculation in \ref{se:FNB}, which was only for \studentB; but you can deduce it for the object and for \studentA.
\end{minipage}
\hfill\begin{minipage}[b]{220pt}
\begin{picture}(220,85)(-10,-25)
\put(0,0){\line(1,0){200}}
\put(60,2){\line(1,0){60}}
\drawbox{30}{1}{20}{50} %\studentA
\drawbox{50}{25}{18}{5} %\studentA's arms
\put(30,53){\scriptsize \studentA}
\drawbox{70}{3}{20}{30} % object
\put(70,35){\scriptsize object}
\drawbox{150}{1}{20}{40} %\studentB
\drawbox{134}{25}{16}{5} %\studentB's arms
\put(150,43){\scriptsize \studentB}
\put(90,27.5){\oval(2,2)[r]}
\put(91,27.5){\line(1,0){43}}
\put(60,-12){\begin{minipage}{60pt}
\scriptsize The object is on a sheet of ice.
\end{minipage}}
\end{picture}
\end{minipage}


Now, as in \autoref{f:firstFBD}, we will draw a free-body diagram for each individual separately.  However, this time we will use \ref{se:weightA} and \ref{se:FNB} to include the gravitational force (the weight) and the normal force.

\noindent % \textwidth default is 5in for a book
\fbox{\begin{minipage}{1.5in}
\begin{FBD}{10}{25}{15}{80}{\studentA}
\onele{20}{$5\unit N$}{black}
\onedo{100}{$834\unit N$}{black}
\oneup{100}{$834\unit N$}{black}
\end{FBD}
\raggedright
Even with the vertical forces, \studentA\ still has a $\vec F_\mathrm{net} = -5.0\unit N \ihat$.
\end{minipage}}
\hfill
\fbox{\begin{minipage}{1.5in}
\begin{FBD}{10}{15}{15}{25}{object}
\twori{20}{$5\unit N$}{black}{16}{$4\unit N$}{black}
\onedo{35}{$20\unit N$}{black}
\oneup{35}{$20\unit N$}{black}
\end{FBD}
\raggedright
Even with the vertical forces, the object still has a $\vec F_\mathrm{net} = +9.0\unit N \ihat$.
\end{minipage}}
\hfill
\fbox{\begin{minipage}{1.5in}
\begin{FBD}{10}{20}{15}{75}{\studentB}
\onele{16}{$4\unit N$}{black}
\onedo{88}{$736\unit N$}{black}
\oneup{88}{$736\unit N$}{black}
\end{FBD}
\raggedright
Even with the vertical forces, \studentB\ still has a $\vec F_\mathrm{net} = -4.0\unit N \ihat$.
\end{minipage}}
\flushright
\multireturn{\mmr{\autoref{ex:braced}}, \mmr{\autoref{ex:unbraced}}, \mmr{\autoref{s:FN}}, \mmr{\hyperlink{d:rope.net}{rope-tension}}, \mmr{\autoref{f:firstFBDangle}}}
\hrule\hrule
\end{figure}

Let's consider some other specific situations\ldots If you decide to lean against a wall, the wall will provide a normal force that pushes horizontally, keeping you from moving through the wall.\new{v2.3}{Answered \protect{\ref{se:ladderN}} and its related problems.}
%
\begin{sample}
\item\label{se:ladderN} \studentC\ leans a $22.7\unit{kg}$ ladder against a wall at an angle of $75.5^\circ$, consistent with \protect{\href{https://www.osha.gov/}{OSHA}} standard \protect{\href{https://www.osha.gov/pls/oshaweb/owadisp.show_document?p_table=standards&p_id=10839}{1926.1053(a)(1)(ii)}}, so that about $\txtfrac{1}{8}$ of the weight is leaning into the wall.  \begin{enumerate}
\item Find the magnitude and direction of the normal force exerted by the wall on the ladder.
\item Find the magnitude and direction of the normal force exerted by the wall on the ladder.
    \end{enumerate}

Since the weight is $F_g = mg = (22.7\unit{kg})(9.81\unitfrac{m}{s^2}) = \sig{222}{.69}{N}$, an eighth of this is $\sig{27.8}{36}{N}$.  This force is pressing into the wall (horizontally, which I will choose as the $+\ihat$ direction).  By \hyperref[ss:NIII]{Newton's third law} if the ladder presses into the wall with $\sig{27.8}{36}{N}$ in the $+\ihat$ direction (this is also a normal force), then the wall pushes the ladder with a normal force of $\sig{27.8}{36}{N}$ in the $-\ihat$ direction.  \textbf{Notice that this is normal (perpendicular) to the surface of the wall.}

Since the full weight of the ladder, $F_g = \sig{222}{.69}{N}$, is still pressing downwards $(-\jhat)$ into the floor (as a normal force), \hyperref[ss:NIII]{Newton's third law} says that the floor pushes the ladder upwards $(+\jhat)$ with a normal force of $\sig{222}{.69}{N}$.  \textbf{Notice that this is normal (perpendicular) to the surface of the floor.}

\autoref{ex:ladder2} goes into the full details of how one calculates the necessary values.
\end{sample}
%
If you lose control of your car and run into a tree, the tree also provides a normal force pushing the car away from the tree; this normal force will stop the car.
%
\begin{sample}
\item\label{se:tree} \studentZ\index{\studentZ} is driving home after a late night of studying at the library.  \HeZ\ is kind of tired and drifts off during the drive.  While traveling $\vec v_i = 13.0\unitfrac ms \ihat$, \studentZ\ runs into a tree, bringing \hisZ\ car $(m=2.1\ten{3}\unit{kg})$ to a halt in $\Delta t = 0.243\unit s$.  (\studentZ\ remains unharmed because \heZ\ was awake enough to wear \hisZ\ seatbelt and
\noindent
\begin{minipage}[b]{240pt}
had a car with a functioning airbag.  Whew.)  Find the normal force by the tree on the car. \\

To be clear about what is happening, I will draw the picture. In order to find the force, we will first need to find the acceleration.
\end{minipage}
\hfill\begin{minipage}[b]{130pt}
\begin{picture}(120,80)(-10,-5)
\put(0,0){\line(1,0){100}}
\drawbox{70}{1}{20}{50} %\studentA
\drawbox{10}{5}{30}{20} % object
\put(15,3){\circle{5}}
\put(35,3){\circle{5}}
\put(0,40){\scriptsize $v=13.0\unitfrac ms$}
\put (10,35){\vector(1,0){30}}
\put(72,33){\scriptsize Tree}
\put(15,15){\scriptsize car}
\end{picture}
\end{minipage}
\[ \vec a = \frac{\vec v_f-\vec v_i}{\Delta t} = \frac{(0\unitfrac ms)-(13.0\unitfrac ms \ihat)}{0.243\unit s} = -\sigfrac{53.4}{9}{m}{s^2}\ihat \]
That the acceleration is in the direction opposite the velocity corresponds to the object slowing down.  Now we can find the \underline{net force} from Newton's second law:
\[ \vec F_\mathrm{net} = m \vec a = (2.1\ten{3}\unit{kg})(-\sigfrac{53.4}{9}{m}{s^2}\ihat) = -\sig{1.1}{2\ten{5}}{N} \ihat \]
There are three forces acting on the car, as can be seen in the free-body diagrams of \autoref{f:firstFBDupdate}.  So, we can draw a free-body diagram here as well.  The gravitational force on the car is
\[ \vec F_g = m\vec g = (2.1\ten{3}\unit{kg})(-9.81\unitfrac{m}{s^2}\jhat) = -\sig{2.0}{6\ten{4}}{N}\jhat \]
Since this in the vertical direction and the net force is in the horizontal direction, there must be an upwards normal force from the ground
\begin{minipage}[b]{240pt}
$ F_{N,\mathrm{ground}} = \sig{2.0}{6\ten{4}}{N}\jhat$.
\textbf{This is normal (perpendicular) to the surface of the ground.} \\

The remaining horizontal force is the normal force from the tree,
$ \deq F_{N,\mathrm{tree}} = -\sig{1.1}{2\ten{5}}{N} \ihat$.
\end{minipage}
\hfill\begin{minipage}[b]{130pt}
\fbox{\begin{minipage}[b]{100pt}
\begin{FBD}{15}{10}{15}{25}{car}
\onele{40}{$F_{N,\mathrm{tree}}$}{rgb:red,0;green,2;blue,1}
\onedo{30}{$F_g$}{rgb:red,0;green,2;blue,1}
\oneup{30}{$F_{N,\mathrm{ground}}$}{rgb:red,0;green,2;blue,1}
\end{FBD}
\end{minipage}}
\end{minipage}
\textbf{This is normal (perpendicular) to the surface of the tree.}
\end{sample}%
(Notice that \ref{se:tree} also shows why it is not always necessary to consider the vertical forces when we ``know'' that they cancel.)  If you throw a ball at the ceiling, the ceiling will provide a normal force downwards, keeping the ball from moving through the surface.
%
\begin{sample}
\item\label{se:ceiling} \studentC\index{\studentC} recalls that one time \heC\ got bored one day in physics class (what?!?) and tossed a baseball ($m_b = 0.145\unit{kg}$) at the ceiling\ldots a little too hard \ldots as recounted in \autoref{ex:ceiling}.  The acceleration during that collision with the ceiling was $\vec a = - \sigfrac{28.0}{9}{m}{s^2} \jhat$.  Find the normal force by the ceiling on the ball.

There are five stages to the motion: (a) throwing, (b) falling up, (c) hitting the ceiling, (d) falling down, and (e) catching show the forces involved. \\
\color{lightgray}
\fbox{\begin{minipage}[b]{55pt}
\begin{picture}(50,100)(0,0)
\put(25,25){\circle{10}}
\put(25,26){\vector(0,1){25}}
\put(25,24){\vector(0,-1){15}}
\put(28,35){$F_\mathrm{throw}$}
\put(28,10){$F_g$}
\end{picture}
\centering{(a) throwing}
\end{minipage}}
\hfill
\fbox{\begin{minipage}[b]{55pt}
\begin{picture}(50,100)(0,0)
\put(25,50){\circle{10}}
\put(25,50){\vector(0,-1){15}}
\put(28,35){$F_g$}
\end{picture}
\centering{(b) falling up}
\end{minipage}}
\hfill
\color{rgb:red,0;green,2;blue,1}
\fbox{\begin{minipage}[b]{55pt}
\begin{picture}(50,100)(0,0)
\put(25,95){\circle{10}}
\put(26,95){\vector(0,-1){25}}
\put(24,95){\vector(0,-1){15}}
\put(28,75){$F_N$}
\put(10,75){$F_g$}
\end{picture}
\centering{(c) \\ hitting}
\end{minipage}}
\hfill
\color{lightgray}
\fbox{\begin{minipage}[b]{55pt}
\begin{picture}(50,100)(0,0)
\put(25,50){\circle{10}}
\put(25,50){\vector(0,-1){15}}
\put(28,35){$F_g$}
\end{picture}
\centering{(d) falling down}
\end{minipage}}
\hfill
\fbox{\begin{minipage}[b]{55pt}
\begin{picture}(50,100)(0,0)
\put(25,25){\circle{10}}
\put(25,26){\vector(0,1){25}}
\put(25,24){\vector(0,-1){15}}
\put(28,35){$F_\mathrm{catch}$}
\put(28,10){$F_g$}
\end{picture}
\centering{(e) catching}
\end{minipage}}
\color{rgb:red,0;green,2;blue,1}
\\
In this particular problem, we are only concerned with step (c) when the ball hits the ceiling, because that is the only part where the normal force acts. \ref{se:throw-up} will describe what happens during steps (a) and (e).

During step (c), we have the actual acceleration, which tells us about the net force.  We will also need to know the weight of the baseball, because gravity is still acting during the collision.
\begin{eqnarray*}
\vec F_N + \vec F_g & = &  \vec F_\mathrm{net} \ = \ m \vec a \\
\vec F_N + m \vec g & = &  m \vec a \\
\vec F_N  & = &  m \vec a - m \vec g \\
\vec F_N  & = &  \left[ (0.145\unit{kg})(-\sigfrac{28.0}{9}{m}{s^2}\jhat) \right] - \left[ (0.145\unit{kg})(-9.81\unitfrac{m}{s^2}\jhat) \right] \\
\vec F_N  & = &  \left[ -\sig{4.07}{3}{N} \jhat \right] - \left[  - \sig{1.42}{2}{N} \jhat \right] \ = \ -\sig{2.65}{1}{N} \jhat
\end{eqnarray*}
You can see that the downward normal force $(\sig{2.65}{1}{N})$ combined with the downward gravitational force $(\sig{1.42}{2}{N})$ together create the downward net force $(\sig{4.07}{3}{N})$.
\end{sample}
%
If you make a \hypertarget{d:bank-shot}{``bank shot''} with either a basketball off the backboard or a pool ball\footnote{Resources for \protect{\href{http://wpapool.com/equipment-specifications/\#Balls-and-Ball-Rack}{specifications}} and
\protect{\href{http://c.ymcdn.com/sites/bca-pool.com/resource/resmgr/imported/BCAEquipmentSpecifications_2008.pdf}{a PDF version}}.
These provide:
    weight ($5.5\unit{oz}=0.\sig{155}{92}{kg}$ and $6.0\unit{oz}=0.\sig{170}{097}{kg}$ cue),
    diameter ($2.250\pm 0.005\unit{in}=\sig{5.71}{5}{cm}\pm 0.0127\unit{cm}$),
    rail height ($63.5 \%$ of the ball height, $= \sig{3.62}{9}{cm}$),  and
    dimension limits on the cue stick:
        $L_\mathrm{min}=40.00\unit{in}=1.016\unit{m}$,
        $m_\mathrm{max} = 25.0\unit{oz} = 0.\sig{708}{75}{kg}$, and
        tip-width $w_\mathrm{max}=1.4\unit{cm}$.
You might also consider the information and calculations at
\protect{\href{http://billiards.colostate.edu/technical_proofs/index.html}{Dr.~Dave's site}},
which gives
    slow ($1\unit{mph}$), medium ($3\unit{mph}$), and fast ($7\unit{mph}$);
    coefficient of friction ball-to-ball $\mu=0.06$; and
    ball-ball collision times as $250\unit{\mu s}$-$300\unit{\mu s}$.
}
off the bumper, then the surface provides a normal force that is perpendicular to the surface, in this case redirecting the ball rather than stopping it.  Unfortunately, the actual mechanism is somewhat more complicated than we are ready for; these are considered a little bit in the \autoref{irl:poolcushion} (pg.~\pageref{irl:poolcushion})\dothis{\protect{\autoref{irl:poolcushion}} should be moved to a section that has more about friction and angular momentum.  It is too complex for this section.}.
%
\begin{reallife}[bthp]
\hspace{-.2in}
\fcolorbox{black}{green!10}{\begin{minipage}{5.29in} \center
\caption{\label{irl:poolcushion}\index{Pool!Real Life} Pool balls and bumpers / cushions.}
\begin{minipage}{4.925in}
\studentD\index{\studentD} is relaxing with the local physics club, playing pool.  \HeD\ shoots a bank-shot and the ricochet reminds all of you about the normal force from the bumper on the ball.
\end{minipage}
\begin{realtable}
\dna{Find a billiards table}
    {Notice the felt, the bumpers (cushion), and the dimensions of the table}
    {Does the ball roll as far on felt as it does on hardwood?  \ref{A:pool.roll} \\
     How soft is the bumper? \ref{A:pool.bumper}}
\dna{Find a set of pool balls}
    {Compare the solid-colored balls, the striped balls, and the cue ball}
    {Are there differences in size of weight? \ref{A:noncue}}
\dna{Hit the cue-ball off of a bumper in the manner intended for
\protect{\href{http://c.ymcdn.com/sites/bca-pool.com/resource/resmgr/imported/BCAEquipmentSpecifications_2008.pdf}{testing cushions}}.}
    {Compare the angle it leaves the bumper (reflected angle) match the angle at which it came in (incident angle)}
    {Does the spin of the ball matter? \ref{A:pool.spin}}
\dna{Place a pool ball against the bumper and ricochet the cue ball off the pool ball instead of the bumper itself.}
    {Notice how the pool ball reacts. \ref{A:pool.later}}
    {Why does the pool ball jump off the bumper? \\
     Does the pool ball move along the wall? \\
     Where did you hit the pool ball?}
\end{realtable}
\begin{minipage}{4.925in}
Billiard tables have a lot of interesting physics, which can help us see a wide variety of physics, for example:
\hyperref[irl:poolfriction]{friction}, \hyperref[irl:poolelastic]{elastic versus inelastic collisions}, \hyperref[irl:poolrotmot]{rotational motion}, and \hyperref[irl:poolangmom]{angular momentum}.
\end{minipage}

\flushright
\linkreturn[pool]{d:bank-shot}
\end{minipage}}
\end{reallife}
%

\subsection{Bathroom Scales Measure the Normal Force}\label{ss:scales}\mlinkreturn[uses of $F=ma$]{d:usesofF=ma}

To get a good sense of what how the normal force works, it helps to consider the way a bathroom scale works.  Consider the concepts presented in the \autoref{irl:scale} (pg.~\pageref{irl:scale}).
%
\begin{reallife}[bthp]
\hspace{-.2in}
\fcolorbox{black}{green!10}{\begin{minipage}{5.29in} \center
\caption{\label{irl:scale}\index{Force!Normal} Playing with a scale.}
\begin{minipage}{4.925in}
While speaking to your friend, \studentB\index{\studentB} about \hisB\ recent accomplishment of losing $45\unit{N}$, you mention that your scale always gives a different number than the one in the doctor's office.  You suggest \heB\ gets on your scale to verify the calibration.  \studentB\ currently has a mass of $\massB$.
\end{minipage}
\begin{realtable}
\dna{Try to lose $45\unit{N}$.}
    {Compare this to your weight}
    {Is this a lot of weight to lose? \ref{A:weight.loss}}
\dna{Place your toe on the scale while \studentB\ weighs \himselfB}
    {This increases the value the scale reads}
    {Does \studentB\ weigh more? \ref{A:weight.gain}}
\dna{With your hands, press down on \studentB's shoulders while \heB\ stands on the scale}
    {Control the value read by the scale.  Increase the reading by $20\unit{N}$, $30\unit{N}$, etc.}
    {Does \studentB's weight change?  \ref{A:weight.gain} Are you adding weight to the scale? \ref{A:scale.increase}}
\dna{Have \studentB\ lean on a nearby table or counter while \heB\ stands on the scale}
    {Control the value read by the scale.  Decrease the reading by $20\unit{N}$, $30\unit{N}$, etc.}
    {Does \studentB's weight change?  \ref{A:weight.gain} }
\dna{Hold the scale against the wall and press into it.}
    {Control the value read by the scale.  Increase the reading by $20\unit{N}$, $30\unit{N}$, etc.}
    {What is the scale measuring? \ref{A:scale.measure}}
\dna{Imagine placing a scale on a ramp that can be laid flat or raised to any angle up to a vertical (making it a wall)}
    {Imagine standing on the scale on the ramp while it is lifted from horizontal (like a floor) to vertical (like a wall)}
    {Does the scale always read the same value while it is raised to different angles? \ref{A:scale.ramp}}
\end{realtable}
%\begin{minipage}{4.925in}
%If you can control the value read by the scale while at the same time not changing your actual mass, does the scale literally measure the weight of the object on the scale?  \ref{A:scale.measure}
%\end{minipage}

\flushright
\autoreturn{ss:scales}
\end{minipage}}
\end{reallife}
%
Some digital scales are inconvenient for understanding how they work because they don't display the value until it has come to something close to equilibrium.  If you have access to an analog scale, then you can watch the value change as it settles down and it might be easier to build your intuition.

As you consider the values that you read on the scale, you should consider what happens if you jump off of or land upon a scale.  \textbf{Note that actually doing this can decalibrate your scale, if not break it entirely.  Scales are not meant to be handled this way.} While you are jumping from your scale, it must provide not only the force necessary to support your weight, but also the upwards force require to accelerate you upwards.  While you are landing on the scale, it musty provide not only the force necessary to support your weight, but also the upwards force necessary to decelerate you.

Bathroom scales use leverage (i.e., \hyperref[s:leverarm]{torque}) and a \hyperref[s:springs]{spring}-system to balance the force pressing into them.  The mechanism can be seen at \href{http://home.howstuffworks.com/inside-scale.htm}{How Stuff Works}.

\section{Tension}\label{s:FT}\mlinkreturn[$F=ma$]{d:f=ma}\index{Force!Tension}

Where the \hyperref[s:FN]{normal force} is appropriate for pushing against surfaces,
\important{tension is the pulling force that is transmitted through materials \\ such as cable, chain, or rope.}
Tension is closely related to the compression force experienced by support beams.  One can simplistically think of tension as pulling\dothis{add a link to (and the section itself) to a section on the modulus and stress/strain.}{} and compression as pushing\dothis{Maybe add an IRL about a house settling and the compression forces.  Loading a pick-up truck and watching the bed sag as weight is added.  Hammock as an example of adding weight and increasing the tension force.}{} on the intermediate object that transmits force between the objects at either end.\footnote{It doesn't usually make sense to talk about the compression of a rope or chain.}
When engineers design the skeleton of bridges and buildings, one of the primary considerations is the tension and compression of the steel beams.  You can build your intuition by considering the \autoref{irl:tension} (pg.~\pageref{irl:tension}).\dothis{Still need to update the \protect{\autoref{irl:tension}}.}
%
\begin{reallife}[bthp]
\hspace{-.2in}
\fcolorbox{black}{green!10}{\begin{minipage}{5.29in} \center
\caption{\label{irl:tension}\index{Force!Tension} Pull my finger.}
\begin{minipage}{4.925in}
We talk about tension and stress in our daily lives.  This is an analogy to the physical version of tension, stress, and strain.  While \protect{\href{http://etymonline.com/index.php?allowed_in_frame=0&search=stress}{stress}} and \protect{\href{http://etymonline.com/index.php?search=strain&searchmode=&p=0&allowed_in_frame=0}{strain}} come from the the concept of tightening, tension \protect{\href{http://etymonline.com/index.php?allowed_in_frame=0&search=tension}{comes from}} the concept of stretching.
\end{minipage}
\begin{realtable}
\dna{Sit on a swing }
    {Notice the tightness of the support ropes/chains}
    {How tight are the supports when the swing is empty? When a small child is in the swing? When a full-sized adult is in the swing? \ref{A:swing.tension}}
\dna{Install a fan or light fixture that hangs from the ceiling}
    {You don't want the fan to be supported by the electrical wires, but rather by the metal shaft}
    {How is the fan supported? \ref{A:fan.tension}}
\dna{Pull on a doorknob}
    {Imagine replacing the knobs (inside and outside) with large knots on a rope that runs through the hole the doorknob used to occupy.}
    {What if the doorknob were replaced with a rope, knotted on either side of the door? [Answer]}
\dna{Take a dog for a walk on a leash}
    {Try to pay attention to Newton's second and third law when the dog changes its level of enthusiasm for pulling on the leash.}
    {If the dog pulls very hard on the leash and you balance that force without allowing the dog to move away from you, then describe the way the force connects you to the dog. [Answer]}
\end{realtable}
%\begin{minipage}{4.925in}
%If you can control the value read by the scale while at the same time not changing your actual mass, does the scale literally measure the weight of the object on the scale?  \ref{A:scale.measure}
%\end{minipage}

\flushright
\autoreturn{s:FT}
\end{minipage}}
\end{reallife}
%

When considering the tension in the rope, the context is generally that the rope is connecting two objects that are trying to pull on each other.  It is convenient to recognize that each object only ``sees'' the rope, not the object at the far side.  This can be seen in a couple of contexts.\new{v2.4}{Modified}

We will start with the \hyperref[s:effective2]{simplistic approximation} of ropes that only transmit the force.  As your understanding improves, we will add some examples where the tension in the rope also affects the rope itself.  In that more complicated situation, the tension will change across the rope\dothis{maybe add links}{} and the rope may stretch\dothis{maybe add links}{}.  Since ropes and cables are twisted strands while chains are links, ropes and cables can also introduce a \hyperref[s:torsion]{torsion}\foreshadow{} that tend not to occur in chains.

\subsection{Tension as a Support Force}\label{ss:tension.support}

Ropes and chains (and beams) can use tension to support (from above) dangling objects.
\begin{sample}
\item\label{se:purse} \studentD\index{\studentD} hangs her purse $(m=1.36\unit{kg})$ on a hook.  How much tension is in the shoulder strap to keep it from falling?

The strap connects the hook to the purse.  We can consider the interaction between the hook and the strap or between the purse and the strap.  We will consider the latter since we don't know anything about the hook.  Considering the forces on the purse, we know that there is a downwards gravitational force of $\deq F_g = (1.36\unit{kg})(9.81\unitfrac{m}{s^2}) = \sig{13.3}{4}{N}$ and that the net force must zero (because the purse is not accelerating). So, the strap must provide an upwards (tension) force.
\begin{eqnarray*}
\vec F_T + \vec F_g & = & m \cancelto{0}{\vec a} \\
\vec F_T + (-\sig{13.3}{4}{N} \jhat) & = & 0\unit N \\
\vec F_T & = & +\sig{13.3}{4}{N} \jhat
\end{eqnarray*}
This is the upwards force that the strap applies to the purse; however, the tension strap is doing two jobs: It is pulling up on the purse (as indicated above) \textbf{and} it is pulling down on the hook.
\end{sample}
The important thing to take away from \ref{se:purse} is not that we can compute the value (although that is, of course, a useful skill), but rather that
\important{the tension is conveying the force between the two objects.}  In the same way that the \hyperref[s:FN]{normal force} on a scale does not measure your weight, but rather the amount you press into the scale, the tension passes force on to the attached object.  The hook doesn't feel the weight of the purse, but does feel the tension required to support the purse.

In \hyperref[sss:multiple.mass]{an upcoming section}, we will consider what happens when multiple masses are hung from the rope.

\subsubsection{How Physicists Use the Words (Vocabulary)}

You can probably think of several examples of objects dangling: a purse on a hook, a flag on a pole, a shop sign attached to a post, a pendulum,\dothis{Add an image of an immovable surface to that section}{} \\
\begin{minipage}{4.25in}
a swing set, etc.  Since these are all similar in some ways (although different in other ways), \textbf{we can treat all of them as a mass at the end of a rope}.  Typically, because we do not want to deal with the complications that come from sagging supports, we will use the \hyperref[s:effective2]{approximation} of an ``\textbf{immovable support}.''  This will be indicated by hashing the surface.
\end{minipage}
\hfill
\begin{minipage}{30pt}
\begin{picture}(35,80)
\put(0,70){\line(1,0){25}}
\multiput(5,70)(5,0){4}{\line(1,1){5}}
\put(12.5,70){\line(0,-1){50}}
\put(7.5,20){\line(1,0){10}}
\put(7.5,20){\line(0,-1){10}}
\put(17.5,10){\line(-1,0){10}}
\put(17.5,10){\line(0,1){10}}
\end{picture}
\end{minipage}

\subsection{Tension as Dragging Force}\label{ss:tension.drag}

We can also consider the \hypertarget{d:rope.net}{tension} in a rope used to drag an object across the floor.  You may recall that in \autoref{f:firstFBD} (and the updated version, \autoref{f:firstFBDupdate}) \studentB\index{\studentB} pulled a box across a sheet of ice.  It is possible that  \studentB\ was grabbing the object itself, but it is more likely that \heB\ was pulling on a rope that was attached to the object.  In that case, the tension in the rope was $4.0 \unit N$.  This tension is what pulled \studentB\ leftwards \textbf{and} what pulled the object rightwards.

We can further update this by considering the case where \studentB\ pulls the rope up at an angle.  In that case, some of the tension is used to drag the box and some is used to reduce the normal force.  In \autoref{f:firstFBDangle}, we will have \studentA\ continue to push with $5.0\unit{N}$ horizontally and have \studentB\ pull with $4.0\unit{N}$ at a $14^\circ$ angle above the horizontal.
%
\begin{figure}
\hrule\hrule
\caption{\label{f:firstFBDangle} An updated version of \protect{\autoref{f:firstFBDupdate}}, people pushing a box.}\index{Free-Body Diagrams!Images}
Again, we can start by drawing a picture of the situation.  The description is the same as it was for \autoref{f:firstFBDupdate} except that \studentB\ pulls at a slight

\noindent
\begin{minipage}[b]{150pt}
angle upwards.  We will again need the gravitational force for \studentA\index{\studentA} (\ref{se:weightA}) and \studentB\index{\studentB} (\ref{se:FNB}).  As before, since nothing is accelerating up or down together, there must also be a normal force on each body.
\end{minipage}
\hfill\begin{minipage}[b]{220pt}
\begin{picture}(220,85)(-10,-25)
\put(0,0){\line(1,0){200}}
\put(60,2){\line(1,0){60}}
\drawbox{30}{1}{20}{50} %\studentA
\drawbox{50}{25}{18}{5} %\studentA's arms
\drawbox{70}{3}{20}{30} % object
\drawbox{150}{1}{20}{40} %\studentB
\drawbox{134}{25}{16}{5} %\studentB's arms
\put(90,16.5){\oval(2,2)[r]}
\put(91,16.5){\line(4,1){44}}
\put(30,53){\scriptsize \studentA}
\put(70,35){\scriptsize object}
\put(150,43){\scriptsize \studentB}
\put(60,-12){\begin{minipage}{60pt}
\scriptsize The object is on a sheet of ice.
\end{minipage}}
\end{picture}
\end{minipage}


Now, as in \autoref{f:firstFBDupdate}, we will draw a free-body diagram for each individual separately.  However, this time we will put the tension of the rope at the appropriate angle.  We will need to do a small calculation to find the value of the normal forces.

\noindent % \textwidth default is 5in for a book
\fbox{\begin{minipage}{1.5in}
\begin{FBD}{10}{25}{15}{80}{\studentA}
\onele{20}{$5\unit N$}{black}
\onedo{100}{$834\unit N$}{black}
\oneup{100}{$834\unit N$}{black}
\end{FBD}
\raggedright
The forces on \studentA\ have not changed.
\end{minipage}}
\hfill
\begin{minipage}{1.5in}
\fbox{\begin{minipage}{1.5in}
\begin{FBD}{10}{15}{15}{25}{object}
\oneri{20}{}{black}\put(43,30){\color{black}\tiny  $5\unit N$}
\onedo{35}{$20\unit N$}{black}
\oneup{35}{$F_N$}{black}
\put(26,41){\color{black} \vector(4,1){20}}
\put(43,46){\color{black} \tiny $4 \unit{N}$}
\end{FBD}
\raggedright
The forces on the object \textit{have} changed.
\end{minipage}}
\begin{picture}(100,60)
\put(0,10){\line(4,1){80}}
\put(0,10){\line(1,0){80}}
\put(80,10){\line(0,1){20}}
\put(15,10){\oval(5,8)[rt]}
\put(25,11){\tiny $14^\circ$}
\put(35,28){\tiny $F_T=4.0 \unit{N}$}
\put(82,30){\tiny $F_{Ty}=$}
\put(82,20){\tiny $= F_T\,\sin 14^\circ$}
\put(82,10){\tiny $ = 0.\sig{96}{8}{N}$}
\put(5,0){\tiny $F_{Tx}=F_T \, \cos 14^\circ = \sig{3.8}{8}{N}$}
\end{picture}
\end{minipage}
\hfill
\fbox{\begin{minipage}{1.5in}
\begin{FBD}{10}{20}{15}{75}{\studentB}
%\onele{16}{$4\unit N$}{black}
\onedo{88}{$736\unit N$}{black}
\oneup{88}{$F_N$}{black}
\put(24,94){\color{black} \vector(-4,-1){20}}
\put(0,92){\color{black} \tiny $4\unit{N}$}
\end{FBD}
\raggedright
The forces on \studentB\ \textit{have} changed.
\end{minipage}}

\noindent
\textbf{For the object}: Since the y-component of the net force is zero, we can find the normal force to be $F_N = -[(-20\unit{N})+(+0.\sig{96}{8}{N})] = 19\unit{N}$.
The x-component of the net force is $F_{\mathrm{net},x}=(5.0\unit N)+(\sig{3.8}{8}{N}) = \sig{8.8}{8}{N}$.

\textbf{For \studentB}: Since the y-component of the net force is zero, we can find the normal force to be $F_N = -[(-736\unit{N})+(-0.\sig{96}{8}{N}) = 737\unit{N}$.
The x-component of the net force is $F_{\mathrm{net},x}=(-\sig{3.8}{8}{N})$.

\flushright
\linkreturn[rope-tension]{d:rope.net}
\hrule\hrule
\end{figure}
%
You should note that since the tension on the object is pulling up, helping the normal force, this allows the normal force (what a scale would read) to be a little smaller.
You should also note that since the tension on \studentB\ is pulling down, counter-acting the normal force, this requires the normal force (what a scale would read) to be a little larger.

\subsection{Pulleys}

While the flexibility of ropes makes them inconvenient for pushing, their flexibility makes them \textit{very useful} for changing the direction of the pull.  The mechanism for changing the direction is the pulley.  Furthermore, by allowing us to change the direction of the pull, we are also able to double, triple, or further improve the strength of the pull.  The term for this is ``the mechanical advantage'' of a pulley-system.

First we will consider three simple cases of redirecting the force.  In each of these cases, I will \hyperref[s:effective2]{assume} that the pulley and rope have no mass and that there is no friction in the turning of the pulley (assume it is trivially easy to spin).  If we do not make this assumption, then the problem gets significantly more complicated.\dothis{add a reference to the section (problem?) where this is considered.}{}

\begin{minipage}[c]{3.25in}
\begin{sample}
\item\studentA\index{\studentA} decides to hold a box that weighs $20\unit N$ using a pulley-system.  What is the tension in the rope?

Since the mass is in equilibrium, the net force is zero and the tension must balance the weight.  This tells us that the tension in the rope is $20 \unit N$.

If the pulley were difficult to turn (had friction) that stickiness could help support the mass and the tension on \studentA's side might be less than $20\unit N$; but since we assumed the pulley to be frictionless, \studentA\ must provide the full $20\unit N$ of tension to the rope.
\end{sample}
\end{minipage}
\hfill
\begin{minipage}{1in}
\begin{picture}(100,120)(0,7)
\put(31,105){\oval(36,36)[t]}
\put(31,105){\circle{33}}
\put(31,106){\line(0,1){29}}
\put(49,105){\line(0,-1){62}}
\put(13,105){\line(0,-1){70}}
\put(-30,7){\line(1,0){100}} % floor
\put(0,135){\line(1,0){62}} % ceiling
\multiput(5,135)(10,0){6}{\line(1,1){5}} % immovable
%
\drawbox{-26}{8}{20}{50} %\studentA
\drawbox{-6}{32}{18}{5} %\studentA's arms
\put(-26,60){\scriptsize \studentA}
%
%\drawbox{5}{19}{16}{16}
%\put(6,25){\small $m_1$}
%
\drawbox{41}{19}{16}{24}
\put(42,25){\small $m$}
%
%\put(49,-1){\vector(0,1){20}}
%\put(49,-1){\vector(0,-1){20}}
%\put(51,-1){\tiny $12\unit m$}
\end{picture}
\end{minipage}
%

\noindent
The interesting aspect is that \studentA\ must pull \textit{down} in order to produce the \textit{upward} tension on the box.  This means that both \studentA\ and the mass are pulling down.  Since the rope is draped over the pulley, the pulley feels $40\unit N$ downwards, $20\unit N$ from the tension supporting the mass and $20\unit N$ from \studentA\ who is creating the tension that supports the mass.  This means that the second rope that is connecting the pulley to the ceiling must be supporting the full $40\unit N$ in order to keep the pulley in equilibrium.



\subsection{Interesting Complications}

\subsubsection{What is the net force on the rope itself?}
The answer to this depends on how complicated you want the answer to be (recall the discussion about effective theories in \autoref{s:effective2}).  Some reasonable answers are:
\begin{itemize}
\item If the rope (and the attachments) are static, then the net force on the rope must be zero even while it maintains the tension.  It is also possible that the rope is accelerating, in which case the net force on the rope while it transfers the forces between the objects at each end is whatever is necessary to produce the acceleration $\vec F_\mathrm{net} = m_\mathrm{rope} \vec a_\mathrm{rope}$.
\item A different answer is to assume that the mass of the rope is small enough that whether it is in equilibrium or accelerating, it does not require a net force and it merely passes its tension through to the object at the other end.
\end{itemize}

\subsubsection{Multiple Masses}\label{sss:multiple.mass}\mautoreturn{ss:tension.support}

Now that we have a few examples of tension under our belts, we can consider some more interesting examples.

\autoref{ex:multiweight.tension} considers the case of hanging multiple masses, which extends the ideas of \autoref{ss:tension.support}.
%
\begin{example}[hbpt]
\fcolorbox{black}{yellow!10}{\begin{minipage}{4.925in}
\caption{\label{ex:multiweight.tension} How many weights?}
While preparing to hang some ornament on a tree, you chain them from a hook on the wall.  You hang ornament 1 from ornament 2 from ornament 3.  What is the tension in each subsequent string?

\color{blue}
The first thing we should do is notice what information is given to us and make sure that everything is in consistent units.  I will convert everything to \hyperref[ss:convertunits]{SI units}.

\color{black}
\autoreturn{sss:multiple.mass}
\end{minipage}}
\end{example}
%
\autoref{ex:multidrag.tension} considers the case of dragging multiple masses, which extends the ideas of \autoref{ss:tension.drag}.
%
\begin{example}[hbpt]
\fcolorbox{black}{yellow!10}{\begin{minipage}{4.925in}
\caption{\label{ex:multidrag.tension} Caravan}
While pulling a sled on which your son sits, your son's sled is tied to a sled on which your dog sits.  Your dog's sled is then connected to a sled with provisions for the day.  What is the tension in each subsequent string?

\color{blue}
The first thing we should do is notice what information is given to us and make sure that everything is in consistent units.  I will convert everything to \hyperref[ss:convertunits]{SI units}.

\color{black}
\autoreturn{sss:multiple.mass}
\end{minipage}}
\end{example}
%
You should note that these examples are essentially expressing the same idea in two different contexts.

\subsubsection{Atwood's Machine}\label{sss:Atwood}

The\dothis{imported a homework problem from Giordano.  Need to modify it to fit my purposes.}{} two crates in the figure (p. 114) hang over a pulley (in what is called an ``Atwood's machine'').  I will select $m_1=35\unit{kg}$ (because it looks smaller) and $m_2=85\unit{kg}$ (because it looks bigger).  We will assume that the pulley is massless and frictionless (so that the tension is the same throughout the rope).  Find the acceleration and the time it takes $m_2$ to accelerate down for the $12\unit m$ to the floor.

\begin{minipage}{1in}
\begin{picture}(100,150)(0,-50)
\put(31,80){\oval(36,36)[t]}
\put(31,80){\circle{33}}
\put(31,81){\line(0,1){29}}
\put(49,80){\line(0,-1){62}}
\put(13,80){\line(0,-1){70}}
%
\put(5,-6){\line(0,1){16}}
\put(5,-6){\line(1,0){16}}
\put(21,10){\line(0,-1){16}}
\put(21,10){\line(-1,0){16}}
\put(6,0){\small $m_1$}
%
\put(41,-6){\line(0,1){24}}
\put(41,-6){\line(1,0){16}}
\put(57,18){\line(0,-1){24}}
\put(57,18){\line(-1,0){16}}
\put(42,0){\small $m_2$}
%
\put(49,-26){\vector(0,1){20}}
\put(49,-26){\vector(0,-1){20}}
\put(51,-26){\tiny $12\unit m$}
\end{picture}
\end{minipage}
\hfill
\begin{minipage}{4.5in}
The easy way to do this is to say that $m_1$ pulls down on the left with $F_{g1} = (35\unit{kg})(9.81\unitfrac{m}{s^2})=\sig{34}{3.4}{N}$ and $m_2$ pulls down on the right with $F_{g2}=(85\unit{kg})(9.81\unitfrac{m}{s^2})=\sig{83}{3.5}{N}$ for a difference of $F_{net} = \sig{49}{0}{N}$ down to the right.  Since this has to move both $m_1$ and $m_2$, the acceleration is
\[ a = \frac{F_{\rm net}}{m_1+m_2} = \frac{\sig{49}{0}{N}}{(35\unit{kg})+(85\unit{kg})} = \frac{\sig{49}{0}{N}}{\sig{120}{}{kg}} = \sigfrac{4.0}{87}{m}{s^2} \]
This acceleration then causes $m_2$ to drop and the time it takes is found from the equation that include distance and time, \\
$y_f \ = \  y_i + v_i \, t + \frac{1}{2} a \,  t^2 $
\[ (0\unit m) \ = \ (12\unit m) + (0\unitfrac ms) \, t + \frac{1}{2} (-\sigfrac{4.0}{9}{m}{s^2}) \,  t^2 \]
\end{minipage}
which we can solve for time:
\[ t \ = \ \sqrt{ \frac{-(12\unit m)}{\frac{1}{2} (-\sigfrac{4.0}{9}{m}{s^2})} } \ = \  \sqrt{ \sig{5.8}{7}{s^2}} \ = \ \sig{2.4}{2}{s} \]

\footnoterule
\small
However, this does not show what the tension is, and many students make a mistake with the tension.  So, I will also answer the question about the tension. We can draw three free-body diagrams. The equation for $m_1$ is as follows, where I am putting the sign in by
\newpar

\begin{minipage}{4.5in}
hand to indicate the direction: \hfill
$\displaystyle (-F_{g1}) + (+F_T) = m_1 (+a) $ \\
The equation for $m_2$ is as follows: \hfill
$\displaystyle (-F_{g2}) + (+F_T) = m_2 (-a) $ \\
Since we know the weights and the masses, these two equations and two unknowns can be written as
\begin{eqnarray*}
(-\sig{34}{3}{N}) + (+F_T) & = & (35\unit{kg}) (+a) \\
(-\sig{83}{3}{N}) + (+F_T) & = & (85\unit{kg}) (-a)
\end{eqnarray*}
There are many ways to solve two equations and two unknowns.
If we subtract the second equation from the first, then we get the equation on the left.
But, if we solve the first equation for $a$ and plug it into the second, then we get the equation on the right
\end{minipage}
\hfill
\begin{minipage}{1in}
\begin{picture}(100,150)(0,-50)
%\put(31,80){\oval(36,36)[t]}
\put(31,80){\circle{33}}
\put(31,81){\vector(0,1){50}}
\put(47.5,80){\vector(0,-1){30}}
\put(14.5,80){\vector(0,-1){30}}
\put(50,55){\tiny $F_T$}
\put(15,55){\tiny $F_T$}
%
\put(5,-6){\line(0,1){16}}
\put(5,-6){\line(1,0){16}}
\put(21,10){\line(0,-1){16}}
\put(21,10){\line(-1,0){16}}
\put(13,4){\vector(0,1){30}}
\put(13,0){\vector(0,-1){20}}
\put(14,20){\tiny $F_T$}
\put(14,-15){\tiny $F_{g1}$}
%
\put(49,8){\vector(0,1){30}}
\put(49,4){\vector(0,-1){40}}
\put(41,-6){\line(0,1){24}}
\put(41,-6){\line(1,0){16}}
\put(57,18){\line(0,-1){24}}
\put(57,18){\line(-1,0){16}}
\put(51,30){\tiny $F_T$}
\put(51,-15){\tiny $F_{g1}$}
\end{picture}
\end{minipage}

\[ \begin{array}{ccc}
\deq
(-\sig{34}{3}{N}) - (-\sig{83}{3}{N}) \ = \ \left[ (35\unit{kg}) + (85\unit{kg}) \right] (a) &&
\deq
(-\sig{83}{3}{N}) + (F_T) \ = \  - (85\unit{kg}) \left[ \frac{(-\sig{34}{3}{N}) + (F_T)}{(35\unit{kg})} \right] \\
\deq
a \ = \ \frac{\sig{49}{0}{N}}{(35\unit{kg})+(85\unit{kg})} = \sigfrac{4.0}{87}{m}{s^2} &&
\deq
F_T \ = \ \frac{-(35\unit{kg})(-\sig{83}{3}{N})-(85\unit{kg})(-\sig{34}{3}{N})}{[(35\unit{kg})+(85\unit{kg})]} \ = \ \sig{48}{6}{N}
\end{array} \]
The acceleration is as above.  The tension is not enough to support $m_2$ (so it falls) and more than enough to lift $m_1$ (so it rises).
You should note that
$\left[(\sig{48}{6}{N}-\sig{34}{3}{N})/(35\unit{kg})=\sigfrac{4.0}{9}{m}{s^2}\right]$
\hfill and \hfill
$\left[(\sig{83}{3}{N}-\sig{48}{6}{N})/(85\unit{kg})=\sigfrac{4.0}{9}{m}{s^2}\right]$.

\normalsize

\subsubsection{Surface Tension}

As a \hypertarget{d:surf.tension}{final note}, \hyperref[s:surface.tension]{surface tension} is something else entirely.  See \autoref{sss:tea} for a comment on the contribution to hot versus cold spoon noises.

\section{Frictional Force}\label{s:Ff}\mmultireturn{\mmr{\ref{A:chair2}}, \mmr{\ref{A:chair6}}, \mmr{\ref{A:chair7}}, \mmr{\autoref{A:fly.balls}}}

%
\begin{reallife}[bthp]
\hspace{-.2in}
\fcolorbox{black}{green!10}{\begin{minipage}{5.29in} \center
\caption{\label{irl:poolfriction}\index{Pool!Real Life} Rolling pool balls and friction.}
\begin{minipage}{4.925in}
\studentD\index{\studentD} is relaxing with the local physics club, playing pool.  \HeD\ hits the cue ball and counts the number of walls \heD\ can hit in one shot.
\end{minipage}
\begin{realtable}
\dna{Hit the cue-ball off of a bumper in the manner intended for
\protect{\href{http://c.ymcdn.com/sites/bca-pool.com/resource/resmgr/imported/BCAEquipmentSpecifications_2008.pdf}{testing cushions}}.}
    {Compare the strength of the hit to the distance travelled}
    {How much is the total distance affected by the number of bumpers hit? \\
     Does it matter if you shoot along the length of the table versus the width of the table?  \\
     Why does friction slow the ball down instead of just make it turn $v=\omega r$ (no slip)}
\end{realtable}
\begin{minipage}{4.925in}
Billiard tables have a lot of interesting physics, which can help us see a wide variety of physics, for example:
\hyperref[irl:poolnormal]{normal force}, \hyperref[irl:poolelastic]{elastic versus inelastic collisions}, \hyperref[irl:poolrotmot]{rotational motion}, and \hyperref[irl:poolangmom]{angular momentum}.
\end{minipage}

%\flushright
%\linkreturn[pool]{d:bank-shot}
\end{minipage}}
\end{reallife}
%

\section{Spring Force}\label{s:springs}\mmultireturn{\mmr{\hyperlink{d:f=ma}{$F=ma$}}, \mmr{\hyperlink{d:usesofF=ma}{uses of $F=ma$}}, \mmr{\autoref{ss:scales}}}

\section{Applied Force}

The term ``an applied force'' is used to describe any force applied by any object when there isn't really a formula to find it.  So this is kind of a ``any other force'' category.  I will use this type of force to describe forces exerted by people.  We have seen some examples where a person throws an object.  We can now revisit those examples and consider the force exerted (applied) by the person who threw the object.
\begin{sample}
\item\label{se:throw-up} \studentC\index{\studentC} recalls that one time \heC\ got bored one day in physics class (what?!?) and tossed a baseball ($m_b = 0.145\unit{kg}$) at the ceiling\ldots a little too hard \ldots as recounted in \autoref{ex:ceiling}.  Recall that \ref{se:ceiling} found the normal force by the ceiling on the ball.  Please now find the force \studentC\ applied while throwing and catching the ball assuming that the throw took $0.200\unit{s}$ to gain the speed of $5.00\unitfrac ms$ and the catch took $0.250\unit s$ to slow the ball from $4.73\unitfrac ms$ to rest.

There are five stages to the motion: (a) throwing, (b) falling up, (c) hitting the ceiling, (d) falling down, and (e) catching show the forces involved. \\
\fbox{\begin{minipage}[b]{55pt}
\begin{picture}(50,100)(0,0)
\put(25,25){\circle{10}}
\put(25,26){\vector(0,1){25}}
\put(25,24){\vector(0,-1){15}}
\put(28,35){$F_\mathrm{throw}$}
\put(28,10){$F_g$}
\end{picture}
\centering{(a) throwing}
\end{minipage}}
\hfill
\color{lightgray}
\fbox{\begin{minipage}[b]{55pt}
\begin{picture}(50,100)(0,0)
\put(25,50){\circle{10}}
\put(25,50){\vector(0,-1){15}}
\put(28,35){$F_g$}
\end{picture}
\centering{(b) falling up}
\end{minipage}}
\hfill
\fbox{\begin{minipage}[b]{55pt}
\begin{picture}(50,100)(0,0)
\put(25,95){\circle{10}}
\put(26,95){\vector(0,-1){25}}
\put(24,95){\vector(0,-1){15}}
\put(28,75){$F_N$}
\put(10,75){$F_g$}
\end{picture}
\centering{(c) \\ hitting}
\end{minipage}}
\hfill
\fbox{\begin{minipage}[b]{55pt}
\begin{picture}(50,100)(0,0)
\put(25,50){\circle{10}}
\put(25,50){\vector(0,-1){15}}
\put(28,35){$F_g$}
\end{picture}
\centering{(d) falling down}
\end{minipage}}
\hfill
\color{rgb:red,0;green,2;blue,1}
\fbox{\begin{minipage}[b]{55pt}
\begin{picture}(50,100)(0,0)
\put(25,25){\circle{10}}
\put(25,26){\vector(0,1){25}}
\put(25,24){\vector(0,-1){15}}
\put(28,35){$F_\mathrm{catch}$}
\put(28,10){$F_g$}
\end{picture}
\centering{(e) catching}
\end{minipage}}
\\
In this particular problem, we are only concerned with steps (a) and (e) because that's where \studentC\ throws and catches the ball. In each case, we need the acceleration: \\
\begin{minipage}[b]{150pt}
\begin{eqnarray*}
\vec a_\mathrm{throw} & = & \frac{(+5.00\unitfrac ms \jhat)-(0\unitfrac ms \jhat)}{0.200\unit s} \\
& = & +\sigfrac{25.0}{0}{m}{s^2} \jhat
\end{eqnarray*}
\end{minipage}
\hfill
\begin{minipage}[b]{150pt}
\begin{eqnarray*}
\vec a_\mathrm{catch} & = & \frac{(0\unitfrac ms \jhat)-(-4.73\unitfrac ms \jhat)}{0.250\unit s} \\
& = & +\sigfrac{18.9}{2}{m}{s^2} \jhat
\end{eqnarray*}
\end{minipage}

During each step, we have the actual acceleration, which tells us about the net force.  We will also need to know the weight of the baseball $F_g=\sig{1.42}{2}{N}$, because gravity is still acting during the collision.  Let's consider the throwing part first.
\begin{eqnarray*}
\vec F_N + \vec F_g & = &  \vec F_\mathrm{net} \ = \ m \vec a \\
\vec F_A  & = &  m \vec a - \vec F_g \\
\vec F_A  & = &  \left[ (0.145\unit{kg})(+\sigfrac{25.0}{0}{m}{s^2}\jhat) \right] - \left[  - \sig{1.42}{2}{N} \jhat \right] \\
\vec F_A  & = &  \left[ +\sig{3.62}{5}{N} \jhat \right] - \left[  - \sig{1.42}{2}{N} \jhat \right] \ = \ +\sig{5.04}{7}{N} \jhat
\end{eqnarray*}
You can see that the upward applied force $(\sig{5.04}{7}{N})$ has to be large enough so that when it is combined with the downward gravitational force $(\sig{1.42}{2}{N})$ they can together result in the necessary (but smaller) upward net force $(\sig{3.62}{5}{N})$ to get it going upwards.

For the catching part, the ball is moving downwards and needs to be stopped, so the catching applied force must be upwards.
\begin{eqnarray*}
\vec F_A + \vec F_g & = &  \vec F_\mathrm{net} \ = \ m \vec a \\
\vec F_A  & = &  m \vec a - \vec F_g \\
\vec F_A  & = &  \left[ (0.145\unit{kg})(+\sigfrac{18.9}{2}{m}{s^2}\jhat) \right] - \left[  - \sig{1.42}{2}{N} \jhat \right] \\
\vec F_A  & = &  \left[ +\sig{2.74}{3}{N} \jhat \right] - \left[  - \sig{1.42}{2}{N} \jhat \right] \ = \ +\sig{4.16}{5}{N} \jhat
\end{eqnarray*}
You can see that the upward applied force $(\sig{4.16}{5}{N})$ has to be large enough so that when it is combined with the downward gravitational force $(\sig{1.42}{2}{N})$ they can together result in the necessary upward net force $(\sig{2.74}{3}{N})$ to stop it from continuing downwards.
\end{sample}

\section{Putting it Together, $F_\mathrm{net}$}\label{s:Fnet}

\subsection{Translational Equilibrium}

blah blah blah
\phantomsection\label{ss:transeq} Translational equilibrium: $F_\mathrm{net} = m \cancelto{0}{a}$.  blah blah blah

\subsection{Static Equilibrium}

\subsection{Dynamic Equilibrium}


\section{Summary and Homework}

\subsection{Summary of Concepts and Equations}\new{v2.3}{Created this section}

\ldots

\subsection*{Conceptual Questions}\new{v2.3}{Added two conceptual problems.}
%\vspace{-24pt}
\begin{enumerate}
\item\label{c:weightmass} Estimate, preferably without using the internet, the mass of the following: (a) a four-door sedan, (b) dishwasher, (c) a pair of glasses, (d) a cell phone.  You should be able to estimate to within one significant digit.
\item\label{c:massweight} List at least one object, preferably without using the internet, that has the following mass: (a) $2500\unit{kg}$ (b) $41\unit{kg}$, (c) $3\unit{kg}$, (d) $50\unit{g}$.
\end{enumerate}
\subsection*{Problems}\new{v2.3}{Created section.}\dothis{Add more problems.}
%\vspace{-24pt}
\begin{enumerate}
 \item\ldots
\end{enumerate}


\chapter{Energy and the Transfer of Energy}

\hypertarget{d:energynoun}{Energy is a noun}\index{Energy!noun}; objects can \textit{have} energy.  \hypertarget{d:workverb}{Work is a verb}\index{Work!verb}\mlinkreturn[heat as a verb]{d:heatverb}; doing work is the process of \textit{exchanging} energy.

\section{Objects Can Have Energy}

\section{A Force Can Transfer Energy} \label{s:work}\mlinkreturn[the direction of forces]{d:pushvector}

\section{Dissipating Energy} \label{s:Wfr}

pool balls on cushion/bumper

\section{Conserving Energy} \label{s:PE}

%
\begin{reallife}[bthp]
\hspace{-.2in}
\fcolorbox{black}{green!10}{\begin{minipage}{5.29in} \center
\caption{\label{irl:poolelastic}\index{Pool!Real Life} 1-D elastic collisions of pool balls.  inelastic collisions off the bumper.}
\begin{minipage}{4.925in}
\studentD\index{\studentD} is relaxing with the local physics club, playing pool.  \HeD\ hits the cue ball and counts the number of walls \heD\ can hit in one shot.
\end{minipage}
\begin{realtable}
\dna{collide balls.}
    {where does it hit}
    {$90^\circ$ output}
\end{realtable}
\begin{minipage}{4.925in}
Billiard tables have a lot of interesting physics, which can help us see a wide variety of physics, for example:
\hyperref[irl:poolnormal]{normal force}, \hyperref[irl:poolelastic]{elastic versus inelastic collisions}, \hyperref[irl:poolrotmot]{rotational motion}, and \hyperref[irl:poolangmom]{angular momentum}.
\end{minipage}

%\flushright
%\linkreturn[pool]{d:bank-shot}
\end{minipage}}
\end{reallife}
%

\subsection{Gravitational Potential Energy}\label{ss:PEg}\mautoreturn{s:PEG}
See also \ref{s:PEG}
\subsection{Spring Potential Energy}\label{ss:PEs}
\subsection{Conservative Forces in General}

\part{Interesting Uses of Motion, Force, and Energy}

\chapter{Momentum: A Better Way to Describe Force}\label{c:momentum}\mmultireturn{\mmr{\hyperlink{d:objectinmotion}{objects in motion}}, \mmr{\autoref{sss:inertia}}, \mmr{\autoref{ss:NIII}}, \mmr{\ref{A:chair6}}}

Useful to include?
\href{https://www.wired.com/2017/06/physics-bullets-versus-wonder-womans-bracelets/}{The Physics of Bullets Vs. Wonder Woman's Bracelets}

\section{Revising Newton's First and Second Laws}

\subsection{Inertia and Momentum}\label{ss:inertia}\mautoreturn{sss:inertia}
Recall \autoref{sss:inertia}.

\section{Revising Newton's Third Law: Conservation of Momentum}\label{s:conservemom}\mautoreturn{ss:NIII}

\section{Two-Dimensional Collisions}\label{s:2Dcollisions}\mautoreturn{sss:vectorequations}

pool balls?  What about rolling?
%
\begin{reallife}[bthp]
\hspace{-.2in}
\fcolorbox{black}{green!10}{\begin{minipage}{5.29in} \center
\caption{\label{irl:pool2Dcollision}\index{Pool!Real Life} 2-D collisions of pool balls.}
\begin{minipage}{4.925in}
\studentD\index{\studentD} is relaxing with the local physics club, playing pool.  \HeD\ hits the cue ball and counts the number of walls \heD\ can hit in one shot.
\end{minipage}
\begin{realtable}
\dna{collide balls.}
    {where does it hit}
    {$90^\circ$ output}
\end{realtable}
\begin{minipage}{4.925in}
Billiard tables have a lot of interesting physics, which can help us see a wide variety of physics, for example:
\hyperref[irl:poolnormal]{normal force}, \hyperref[irl:poolelastic]{elastic versus inelastic collisions}, \hyperref[irl:poolrotmot]{rotational motion}, and \hyperref[irl:poolangmom]{angular momentum}.
\end{minipage}

%\flushright
%\linkreturn[pool]{d:bank-shot}
\end{minipage}}
\end{reallife}
%


\chapter{Rotational Motion}

\section{The Equations of Rotational Motion}

%
\begin{reallife}[bthp]
\hspace{-.2in}
\fcolorbox{black}{green!10}{\begin{minipage}{5.29in} \center
\caption{\label{irl:poolrotmot}\index{Pool!Real Life} Rolling pool balls.}
\begin{minipage}{4.925in}
\studentD\index{\studentD} is relaxing with the local physics club, playing pool.  \HeD\ hits the cue ball and counts the number of walls \heD\ can hit in one shot.
\end{minipage}
\begin{realtable}
\dna{Roll a striped ball along the table.}
    {Use the stripe to notice the rate of rotation}
    {How does the rotation compare to the translation?}
\dna{Roll a striped ball along the table.}
    {Notice the distance the ball travels}
    {Why does friction slow the ball down instead of just make it turn $v=\omega r$ (no slip)}
\end{realtable}
\begin{minipage}{4.925in}
Billiard tables have a lot of interesting physics, which can help us see a wide variety of physics, for example:
\hyperref[irl:poolnormal]{normal force}, \hyperref[irl:poolelastic]{elastic versus inelastic collisions}, \hyperref[irl:poolrotmot]{rotational motion}, and \hyperref[irl:poolangmom]{angular momentum}.
\end{minipage}

%\flushright
%\linkreturn[pool]{d:bank-shot}
\end{minipage}}
\end{reallife}
%

\section{Angular Momentum}

%
\begin{reallife}[bthp]
\hspace{-.2in}
\fcolorbox{black}{green!10}{\begin{minipage}{5.29in} \center
\caption{\label{irl:poolangmom}\index{Pool!Real Life} Rolling pool balls.}
\begin{minipage}{4.925in}
\studentD\index{\studentD} is relaxing with the local physics club, playing pool.  \HeD\ hits the cue ball and counts the number of walls \heD\ can hit in one shot.
\end{minipage}
\begin{realtable}
\dna{Roll a striped ball along the table.}
    {Use the stripe to notice the rate of rotation}
    {How does the rotation compare to the translation?}
\end{realtable}
\begin{minipage}{4.925in}
Billiard tables have a lot of interesting physics, which can help us see a wide variety of physics, for example:
\hyperref[irl:poolnormal]{normal force}, \hyperref[irl:poolelastic]{elastic versus inelastic collisions}, \hyperref[irl:poolrotmot]{rotational motion}, and \hyperref[irl:poolangmom]{angular momentum}.
\end{minipage}

%\flushright
%\linkreturn[pool]{d:bank-shot}
\end{minipage}}
\end{reallife}
%


\section{Non-inertial Rotational Reference Frames} \label{s:noninertial}\mmultireturn{\mmr{\autoref{ss:noninertial}}, \mmr{\hyperlink{d:NewtonInertial}{non-inertial reference frames}}, \mmr{\autoref{ss:NI}}}
\index{Reference Frames!Inertial}
\index{Reference Frames!Non-inertial}

Because the Earth \hypertarget{d:noninertial}{rotates}\mautoreturn{ss:NII}, we are actually in a non-inertial reference frame.  In fact, we can prove that the Earth rotates by observing the effects, such as the \hyperlink{d:coriolis}{Coriolis effect}, that in our non-inertial frame seem to require unexplainable forces but which, in a non-rotating frame, follow the expected laws of physics.

\subsection{The Coriolis Effect}\label{ss:coriolis}\mmultireturn{\mmr{\hyperlink{d:NewtonInertial}{non-inertial reference frames}}, \mmr{\hyperlink{d:noninertial}{Non-inertial Rotational Reference Frames}}}

\hypertarget{d:coriolis}{weather, etc}
\newpar

In her podcast\new{v2.0}{\textit{Spacepod}}, \textit{Spacepod}\footnote{Nugent, Carrie (Producer, Host). \textit{Spacepod} [Audio podcast], episode 89 (19 May, 2017).  Retrieved from \hyperref{http://spacepod.libsyn.com/}{T4LTFdOxHD5WWzdD}{99}{\nolinkurl{http://spacepod.libsyn.com/}}
on 9 Apr. 2017.} Dr. Carrie Nugent interviews Dr. Andy Thompson about ``underwater flying objects'' that investigate the ocean.  He notes that ocean waters, because they are such a large-scale system, can see the effect of the rotation of the Earth.

\subsection{The Foucault Pendulum}\label{ss:Foucault}

See \href{https://www.youtube.com/watch?v=sWDi-Xk3rgw}{youtube video} by \href{http://sixtysymbols.com/}{Sixty Symbols}.\new{v2.0}{Foucault video}




\chapter{Circular Motion and Centripetal Force}

\section{Circular Motion}
\section{Centripetal Force}\label{s:centripetal}\mlinkreturn[$F=ma$]{d:f=ma}




\chapter{Torque and the $F=ma$ of Rotations}\label{c:torque}\mreturn{a:NIIIaction}\new{v2.3}{Added an example that is computable here, but helps introduce normal force in \protect{\autoref{s:FN}}.}

\section{Leverage}\label{s:leverarm}\mautoreturn{ss:scales}

\section{Putting it all together, $\tau_\mathrm{net}$}

\subsection{Rotational Equilibrium}

blah blah blah
\phantomsection\label{ss:roteq} Rotational equilibrium: $\tau_\mathrm{net} = I \cancelto{0}{\alpha}$.  blah blah blah

\subsection{Static (Rotational) Equilibrium}

\subsection{Dynamic (Rotational) Equilibrium}

\new{v2.3}{Answered \protect{\autoref{ex:ladder2}} and its related problems.}
\begin{example}[p]
\fcolorbox{black}{yellow!10}{\begin{minipage}{4.925in}\setlength{\parskip}{3pt}
\caption{\label{ex:ladder2} \studentC\index{\studentC} uses a ladder}
\begin{quote}
\studentC\ leans a $22.7\unit{kg}$ ladder against a wall at an angle of $75.5^\circ$, consistent with \protect{\href{https://www.osha.gov/}{OSHA}} standard \protect{\href{https://www.osha.gov/pls/oshaweb/owadisp.show_document?p_table=standards&p_id=10839}{1926.1053(a)(1)(ii)}}.
The coefficient of friction between the ladder and the floor is $\mu_f=0.31$.
The coefficient of friction between the ladder and the wall is $\mu_w=0.19$.
Use the rotational and translational equilibrium to determine if the ladder slides.
\end{quote}

Since we are asked to distinguish between two cases that cannot both be true, we should assume one (the easier one to calculate is that the ladder does not slip) and then verify that the result is consistent with that assumption.

\textbf{What do we know?}
We know that the floor has a normal force $(F_{Nf})$ upwards and a frictional force $(F_{ff})$ to the left.
We know that the wall
\\[2pt]
\begin{minipage}{3.2in}
has a normal force $(F_{Nw})$ to the right and a frictional force $(F_{fw})$ up (keeping the ladder from sliding down).
We know the weight is
\[ F_g = mg = (22.7\unit{kg})(9.81\unitfrac{m}{s^2}) = \sig{222}{.69}{N} \]
\textbf{What do we want to know?}  We want to know about the the magnitudes of both normal
\end{minipage}
\hfill
\begin{minipage}{100pt}
\begin{picture}(100,100)(-10,-5)
% Dimensions and offset: (width,height)(x offset,y offset)
% Insert picture commands (\line,\circle, etc...) here:
\put(0,0){\line(0,1){100}}
\put(0,0){\line(1,0){75}}
\put(20,0){\color{blue}\line(-1,4){20}}     % ladder
\put(10,40){\color{red}\vector(0,-1){30}}   % Fg
\put(20,1){\color{red}\vector(0,1){25}}     % FNf
\put(19,1){\color{red}\vector(-1,0){12}}
\put(1,80){\color{red}\vector(1,0){12}}
\put(1,81){\color{red}\vector(0,1){8}}
\end{picture}
\end{minipage}
%\hfill {}
\\[3pt]
forces and both frictional forces.
Can we easily deduce the magnitude of $F_{Nf}$? \ref{A:ladderNf}.

\textbf{How are these related?}  The forces acting on any body are related by static \hyperref[ss:transeq]{translational equilibrium}
\begin{eqnarray*}
x: \hspace{.5cm} 0 & = & \cancelto{0}{F_{gx}} + \cancelto{0}{F_{Nfx}} + F_{ffx} + F_{Nwx} + \cancelto{0}{F_{fwx}} \\
y: \hspace{.5cm} 0 & = & F_{gy} + F_{Nfy} + \cancelto{0}{F_{ffy}} + \cancelto{0}{F_{Nwy}} + F_{fwy}
\end{eqnarray*}
and static \hyperref[ss:roteq]{rotational equilibrium}, assuming the pivot point is at the ground, and using the relationship $F_f=\mu F_N$, we find
\begin{eqnarray*}
0 & = & \tau_{g} + \cancelto{0}{\tau_{Nf}} + \cancelto{0}{\tau_{ff}} + \tau_{Nw} + \tau_{fw} \\
0 & = & \left[ F_g \frac{l}{2} \sin 14.5^\circ \right] + \left[ - F_{Nw} l \sin(75.5^\circ) \right] + \left[ - F_{fw} l \sin(14.5^\circ) \right] \\
F_{Nw} & = & \left[ F_g \frac{l}{2} \sin 14.5^\circ \right] / \left[  l \sin(75.5^\circ) + \mu_w l \sin(14.5^\circ) \right]
\end{eqnarray*}
%\textbf{Free-Body Diagrams:}  Since the picture is so simple, we will not draw the free-body diagram.


{}\hfill {\footnotesize\autoref*{ex:ladder2} continued on next page\ldots}
\end{minipage}}
\end{example}
\begin{example}[p]
\fcolorbox{black}{yellow!10}{\begin{minipage}{4.925in}\setlength{\parskip}{3pt}
{\footnotesize \autoref*{ex:ladder2} continued from previous page\ldots}

\textbf{Concepts to Consider:}  First, the length of the ladder cancels from the expression; what matters is the angle at which it is propped.

Second, every force value will be linearly dependent on the mass of the ladder.  So once we solve this problem, we can easily scale the answers to any mass.

Third, the friction with the wall is, by far, the smallest effect and it might be interesting to approximate all of this with $\mu_w=0$.  You can check your calculation against \ref{A:nowall}.

\textbf{Solution to the example:}  When we worry about significant figures,
\begin{eqnarray*}
F_{Nw} & = & \frac{\left[ (\sig{222}{.7}{N})(\txtfrac{1}{2}) (0.\sig{250}{4}{}) \right]}{\left[  (0.\sig{968}{2}{}) + (0.19) (0.\sig{250}{4}{}) \right]}
\ = \ \frac{\left[ (\sig{27.8}{8}{N})\right]}{\left[  (0.\sig{968}{2}{}) + (0.0\sig{47}{6}{}) \right]} \\
F_{Nw} & = & \frac{\left[ (\sig{27.8}{8}{N})\right]}{\left[  (\sig{1.015}{7}{}) \right]}
\ = \ \sig{27.4}{4}{N} \\
F_{fw,\mathrm{max}} & = & (0.19)(\sig{27.4}{4}{N}) \ = \ \sig{5.2}{15}{N} \\
F_{Nf} & = & F_g - F_{fw} = (\sig{222}{.7}{N})-(\sig{5.2}{15}{N}) \ = \ \sig{217}{.5}{N} \\
F_{ff,\mathrm{max}} & = & (0.31)(\sig{217}{.5}{N}) \ = \ \sig{672}{.4}{N}
\end{eqnarray*}
Since $F_{ff} >F_{Nw}$, the friction is sufficient to hold the ladder in place, as assumed.

%\begin{quote}
\textbf{Aside:} Since $F_{ff}$ only needs to be $\sig{27.4}{4}{N}$ to hold the ladder in place, it is possible for the ladder to not slide on a floor that only has
$\mu_\mathrm{min} = (\sig{27.4}{4}{N})/(\sig{217}{.5}{N}) = 0.\sig{126}{2}{}$; but that would not allow a person to climb the ladder.

\textbf{Homework:} Homework problem~\ref{h:ladderC} asks you to determine if the ladder slides when \studentC\ climbs to different locations on the ladder.
%\end{quote}
\flushright
\multireturn{\mmr{\ref{se:ladderN}}, \mmr{\autoref{ss:roteq}}}
\end{minipage}}
\end{example}

\section{Torsion}\label{s:torsion}\mautoreturn{s:FT}\new{v2.4}{Created this section}

\section{Summary and Homework}

\subsection{Summary of Concepts and Equations}\new{v2.3}{Created this section}

\ldots

\subsection*{Conceptual Questions}\dothis{Add conceptual problems.}
%\vspace{-24pt}
\begin{enumerate}
\item\ldots
\end{enumerate}
\subsection*{Problems}\new{v2.3}{Added problems.}\dothis{Add more problems.}
%\vspace{-24pt}
\begin{enumerate}
 \item\label{h:ladderC} \studentC\ leans a $22.7\unit{kg}$ ladder against a wall at an angle of $75.5^\circ$, consistent with \protect{\href{https://www.osha.gov/}{OSHA}} standard \protect{\href{https://www.osha.gov/pls/oshaweb/owadisp.show_document?p_table=standards&p_id=10839}{1926.1053(a)(1)(ii)}}.\new{v2.3}{Answered \protect{\ref{h:ladderC}} and its related problems.}
The coefficient of friction between the ladder and the floor is $\mu_f=0.31$.
The coefficient of friction between the ladder and the wall is $\mu_w=0.19$.
Use the rotational and translational equilibrium to determine if the ladder slides when \studentC\ ($\massC$) climbs to
\begin{enumerate}
\item the third-rung from the top of the ladder, so that he is $1.53\unit m$ from the bottom of the ladder.
    (See \ref{A:nowallC} for that answers if $\mu_w = 0$.)
\begin{ForMe}
\color{blue} Answers:
\begin{eqnarray*}
F_{Nw} & = & \sig{163}{.9}{N} \\
F_{fw,\mathrm{max}} & = & (0.19)(\sig{163}{.9}{N}) \ = \ \sig{31}{.14}{N} \\
F_{Nf} & = & \sig{1074}{.4}{N} \\
F_{ff,\mathrm{max}} & = & \sig{333}{.0}{N} < \sig{163}{.9}{N}
\end{eqnarray*}
$\mu_\mathrm{min} = 0.\sig{152}{56}{}$
\color{black}
\end{ForMe}
\item the third-rung from the bottom of the ladder, so that he is $0.914\unit m$ from the bottom of the ladder.
\begin{ForMe}
\color{blue}
Answers:
\begin{eqnarray*}
F_{Nw} & = & \sig{108}{.97}{N} \\
F_{fw,\mathrm{max}} & = & (0.19)(\sig{108}{.97}{N}) \ = \ \sig{20}{.70}{N} \\
F_{Nf} & = & \sig{1084}{.9}{N} \\
F_{ff,\mathrm{max}} & = & \sig{336}{.3}{N} < \sig{108}{.97}{N}
\end{eqnarray*}
$\mu_\mathrm{min} = 0.\sig{100}{45}{}$

If $\mu_w = 0$.
\begin{eqnarray*}
F_{Nw} & = & \sig{114}{.3}{N} \\
F_{fw,\mathrm{max}} & = & 0 \unit N \\
F_{Nf} & = & \sig{1105}{.6}{N} \\
F_{ff,\mathrm{max}} & = & \sig{342}{.7}{N} < \sig{114}{.3}{N}
\end{eqnarray*}
$\mu_\mathrm{min} = 0.\sig{103}{4}{}$
\color{black}
\end{ForMe}
\end{enumerate}
\end{enumerate}


\chapter{Energy of Rotating Objects}
\section{Rotational Kinetic Energy}
pool balls

\chapter{The Gravitational Force on a Large Scale}\label{c:gravity}\mmultireturn{\mmr{\hyperlink{d:accgrav}{freefall}}, \mmr{\hyperlink{d:fundamental}{fundamental forces}}}

\section{Gravitational Force and Field}\label{s:Gfield}\mlinkreturn[$F=ma$]{d:f=ma}\new{v2.3}{Added some placeholders}

The value of the acceleration due to gravity  varies according to the mass and size of any celestial body.\dothis{Reference a table of $g$ on other planets and compute the weight of a space craft at each planet.}
This means that, as was seen in \ref{se:gworld}, your weight can change even when your mass remains the same.
\begin{sample}
\item\label{se:gplanets} In conversation with a visiting alien, \studentX\index{\studentX}, you find that \studentX\ has been to the moon and several planets both within and outside of our solar system.  In addition to the Earth, \studentX\ has visited our moon, Mars, Pluto, and Planet X.  Using \autoref{t:gplanets}, compute \studentX's weight are each location, assuming \hisX\ mass is \massX.
\begin{enumerate}
\item[Earth] $F_g = (\massX)\left[ \frac{ G M_E}{R_E^2} \right] = (\massX)(9.825\unitfrac{m}{s^2}) \ = \ \sig{933}{.4}{N}$
\item[moon] $F_g = (\massX)\left[ \frac{ G M_m}{R_m^2} \right] = (\massX)(9.782\unitfrac{m}{s^2}) \ = \ \sig{929}{.3}{N}$
\item[Mars] $F_g = (\massX)\left[ \frac{ G M_M}{R_M^2} \right] = (\massX)(9.763\unitfrac{m}{s^2}) \ = \ \sig{927}{.5}{N}$
\item[Pluto] $F_g = (\massX)\left[ \frac{ G M_P}{R_P^2} \right] = (\massX)(9.763\unitfrac{m}{s^2}) \ = \ \sig{927}{.5}{N}$
\item[Planet X] $F_g = (\massX)\left[ \frac{ G M_X}{R_X^2} \right] = (\massX)(9.763\unitfrac{m}{s^2}) \ = \ \sig{927}{.5}{N}$
\end{enumerate}
\end{sample}
%
\begin{table}[bhtp]
\hrule\hrule
\begin{center}
\caption[Properties of various celestial bodies]{\label{t:gplanets} Properties of various celestial bodies.
\return{se:gplanets}
}
\begin{tabular}{lccr}
Planet & Mass (kg) & Mean Radius (m) & $g (\unitfrac{m}{s^2})$ \\
\end{tabular}
\end{center}
\hrule\hrule
\end{table}
%


\subsection{Inertial Mass versus Gravitational Mass}\label{ss:equivmm}\mautoreturn{ss:weightmass}\new{v2.2}{Moved this here, might need to move it back.}

\section{Gravitational Potential Energy} \label{s:PEG}\mautoreturn{ss:PEg}

Recall \ref{ss:PEg}

\section{Making Connections}\label{s:Gconnection}\mautoreturn{s:Econnection}

\[ \begin{array}{ccccc}
& & \vec F = m \vec g & & \\
& \deq F = G \frac{m_1 m_2}{R^2} & \leftrightarrow & \deq g = G \frac{m}{R^2} & \\
\Delta \PE = -\vec F \cdot \Delta\vec x & \updownarrow & & \updownarrow & \mbox{\scriptsize [for later]} \\
& \deq \PE = G \frac{m_1 m_2}{R} & \leftrightarrow & \mbox{[for later]} & \\
& & \mbox{\scriptsize [for later]} & &
\end{array} \]
(Look ahead to the parallel with the electrical interaction in \autoref{s:Econnection}.)

\section{Orbits}


\part{Making Waves}

\chapter{Fluids}\new{v2.2}{Placeholder}
\section{Density}\label{s:density}\index{Density}\mautoreturn{ss:weightmass}

\section{Surface Tension}\label{s:surface.tension}\mlinkreturn{d:surf.tension}



\chapter{Oscillations}\label{c:SHM}
\section{Oscillating Springs}\label{c:SHMspring}\mlinkreturn[$F=ma$]{d:f=ma}
\section{Oscillating Pendulums}\label{c:SHMpend}

\section{Other Examples of Oscillations}\label{s:SHMother}

On 13 April, 2017,\new{v2.3}{New source of info}
\href{http://www.cbc.ca/podcasting}{CBC Broadcasting} published a
\href{http://www.cbc.ca/podcasting/includes/quirks.xml}{\textit{Quirks and Quarks}} episode discussing how we can find
\href{https://podcast-a.akamaihd.net/mp3/podcasts/quirks_20170415_12100.mp3}{solutions to health issues caused by swaying office towers and vibrating floors}.

\chapter{Sound}
\subsection{Musical Instruments}\label{ss:stringed.instruments} \mautoreturn{A:swing.tension}



\part{Is It Hot in Here?}

\chapter{The flow of thermal energy}

\phantomsection\label{find:heatwarm}
Energy is a noun\index{Energy!noun}; objects can \textit{have} energy.  \hypertarget{d:heatverb}{Heat is a verb}\index{Heat!verb}; heating is a process of \textit{exchanging} energy.  Recall our \hyperlink{d:forcenoun}{discussions of force}\index{Force!noun} and \hyperlink{d:workverb}{work}\index{Work!verb}.

\section{Specific Heat Capacity}\label{s:specificheat}

\hypertarget{d:heatwarm}{Heating (positive $Q$)} can warm (positive $\Delta T$) a material.
\begin{equation}\label{eq:Q=mcDT}
Q = m c \, \Delta T
\end{equation}
but \autoref{eq:Q=mL} (as one example) shows that it is possible to heat (positive $Q$) a material without warming it (constant $T$). When we get to \autoref{s:PV} we will see other examples of ``isothermal processes'' that have a non-zero $Q$ (heat the system or heat the surroundings) without warming or cooling the system.

\section{Latent Heat}

Heating might also change the phase of a material.\mlinkreturn[heating versus warming]{d:heatwarm}
\begin{equation}\label{eq:Q=mL}
Q = \pm mL
\end{equation}

\section{The Flow of Thermal Energy}

\subsection{Thermal Conductivity}\label{ss:thermalconductivity}\mautoreturn{s:story}

\begin{equation}\label{eq:thermalconductivity}
\frac{Q}{\Delta t} = \kappa A \, \frac{\Delta T}{\Delta x}
\end{equation}

\begin{example}
\fcolorbox{black}{yellow!10}{\begin{minipage}{4.925in}
\caption{\label{ex:baking}\studentA\protect{\index{\studentA}} warms \hisA\ oven.}
\studentA\protect{\index{\studentA}} decides to bake some bread for the dinner party at \studentB\protect{\index{\studentB}}'s house, but \heA\ is on a tight schedule.  In order to set \hisA\ schedule, \heA\ needs to know how long it will take \hisA\ oven to \hyperref[find:heatwarm]{warm up}.

\autoreturn{s:story}
\end{minipage}}
\end{example}

\subsection{Convection}
\subsection{Radiation}

\chapter{Ideal Gas Law}
\section{$P$-$V$ Diagrams}\label{s:PV}\mlinkreturn[heating versus warming]{d:heatwarm}

\part{Let There be Light!}

\chapter{The Electrical Interaction}\label{c:electric}\mlinkreturn[fundamental forces]{d:fundamental}
\section{Electrical Charge}\label{s:Echarge}\new{v2.1}{Decide where this should go.}

\section{The Big Picture}

\subsection{Electric Forces and Fields}\label{ss:Efield}\mmultireturn{\mmr{\autoref{sss:vectorequations}}, \mmr{\hyperlink{d:f=ma}{$F=ma$}}}

pst-electricfield

\subsubsection{Examples}

\subsection{Electric Forces, Fields, and Potential Energy}

\subsection{Electric Fields, Potential Energy, and Potential}

\section{Making Connections}\label{s:Econnection}\mautoreturn{s:Gconnection}

\[ \begin{array}{ccccc}
& & \vec F = q \vec E & & \\
& \deq F = k \frac{q_1 q_2}{r^2} & \leftrightarrow & \deq E = k \frac{q}{r^2} & \\
\Delta \PE = -\vec F \cdot \Delta\vec x & \updownarrow & & \updownarrow & \Delta V = -\vec E \cdot \Delta\vec x  \\
& \deq \PE = k \frac{q_1 q_2}{r} & \leftrightarrow & \deq V = k \frac{q}{r} & \\
& & \Delta \PE = q \Delta V & &
\end{array} \]
(Recall the parallel with the gravitational interaction in \autoref{s:Gconnection}.)

\chapter{Electricity}

\chapter{The Magnetic Interaction}

pst-magneticfield

\chapter{``Magnicity?''}

\chapter{Light}

\chapter{Optics}

\part{What Have You Done for Me Lately?}

\chapter{Relativity}
\chapter{Quantum Mechanics}\new{v2.1}{Decide if these subsections should be chapters in and of themselves.  These are now labeled.}
\section{Atomic Physics} \subsection{The Periodic Table and Quantum Numbers}
\section{Nuclear Physics} \subsection{Nuclear Decay}\label{ss:nucleardecay}
\subsection{The Strong Nuclear Force}\label{ss:strong}\mlinkreturn[fundamental forces]{d:fundamental}
\subsection{The Weak Nuclear Force}\label{ss:weak}\mlinkreturn[fundamental forces]{d:fundamental}
\section{Particle Physics}\label{s:particle}
\subsection{Field Theory}
\subsection{Quantum Electrodynamics}\label{ss:QED}\mlinkreturn[fundamental forces]{d:fundamental}
\subsection{Quantum Chromodynamics}\label{ss:QCD}\mlinkreturn[fundamental forces]{d:fundamental}
\subsection{The Standard Model}\label{ss:StandardModel}
\subsection{Particle Decay}\label{ss:particledecay}
\chapter{Condensed Matter}
\chapter{Astronomy}
\chapter{Cosmology}

\part{Supplements}

\chapter{Deeper Dive}\label{c:revisted}\new{v2.1}{This chapter should mirror \protect{\autoref{c:physics}}.}

This is where I will put the fuller explanations.

\subsection{The Sun}\label{sss:sun}
The bright, shiny sun, which keeps us all alive, is a nice example of a rather complex system that allows us to verify our various theories of the world around us.  We can consider the existence of a star in three phases: the birth of a star, the life of the star, and the death of the star.

\subsubsection{The Birth of a Star}
\subsubsection{The Life of a Star}
\subsubsection{The Death of a Star}


\subsection{Kitchen Appliances}
\subsubsection{Oven}
\subsubsection{Refrigerator}
\subsubsection{Microwave}
\subsubsection{Television}

\subsection{Automobile}
\subsubsection{Coolant and Antifreeze}
\subsubsection{Tires}
\subsubsection{Torque}

\subsection{Cool Ideas}
\subsubsection{Black Holes}\label{sss:blackhole2}\mautoreturn{ss:weightmass}

On 7 April, 2017,\new{v2.3}{New source of info}
\href{http://www.cbc.ca/podcasting}{CBC Broadcasting} published a
\href{http://www.cbc.ca/podcasting/includes/quirks.xml}{\textit{Quirks and Quarks}} episode discussing how we can
\href{https://podcast-a.akamaihd.net/mp3/podcasts/quirks_20170408_51226.mp3}{turn our planet into a giant telescope to get a photo of a black hole}.
The results should be available by the early 2018.\dothis{Follow-up in 2018 to find the results.}

\subsubsection{Quantum Mechanics}
\subsubsection{Relativity}
\subsubsection{String Theory}



\chapter{Podcasts and Videos}\label{c:videos}\label{c:podcasts}

\section{Podcasts}\label{s:podcasts}
\hyperref{http://spacepod.libsyn.com/}{T4LTFdOxHD5WWzdD}{99}{Spacepod with Carrie Nugent} \\
\href{http://www.sciencefriday.com/}{Science Friday with Ira Flatow}

\section{Videos}\label{s:videos}
\href{http://physicsfootnotes.com/}{Physics Footnotes} \\
\href{http://sixtysymbols.com/}{Sixty Symbols}

\section{Websites}\label{s:websites}
\href{http://www.aldakavlilearningcenter.org/practice/flame-challenge}{The Flame Challenge}

\chapter{Answers to Interactive Questions}

\begin{AIQ}
\item\label{A:hbf} There are forces acting on it.  You should be able to tell this because you are exerting one of the forces. While it is true that there are forces on it, it is also true that there is no \textit{net force}.  If you are exerting an upward force on the book, can you guess (\ref{A:gravity}) what the downward force is?   \return{IQ:holdbook}
\item\label{A:chair1} If we refer to ``motion'' as describing the velocity, then no. Force causes a \textit{change in} velocity. When you stop pushing, the chair stops because there is a force from the carpet acting to oppose the force you apply while you push the chair. \autoreturn{irl:NI}
\item\label{A:chair2} This is essentially the same as \ref{A:chair1}, but the carpet exerts more force than the tile.  In either case, force causes a \textit{change in} velocity. You are trying to speed the chair up and the floor is trying to slow the chair down.  (Both are trying to change the velocity, but cancel to result in a constant velocity.)  When you stop pushing, the chair stops moving because there is a force from the tile acting to oppose the force you apply while you push the chair; when you let go, this force slows the chair until the chair stops and then the force stops acting. (See \autoref{s:Ff} for more details.) \autoreturn{irl:NI}
\item\label{A:weight.loss} Since \studentB\index{\studentB} weighs $(\massB)(9.81\unitfrac{m}{s^2})=736\unit{N}$, $45\unit{N}$ is about $6\%$ of her weight.  This is fairly substantial.  You should compute how much $6\%$ of your weight is and convert that to kilograms and Newtons.  \autoreturn{irl:scale}
\item \label{A:ladderNf} Since the full weight of the ladder, $F_g = \sig{222}{.69}{N}$, is still pressing downwards into the floor (as a normal force), it is tempting to say that \hyperref[ss:NIII]{Newton's third law} implies that the floor pushes the ladder upwards with a normal force of $\sig{222}{.69}{N}$ but this would not account for the frictional force on the wall, $F_{fw}$.  If there were no friction between the ladder and the wall, then we could deduce $F_{Nf}$, but at this point, we cannot. \autoreturn{ex:ladder2}
\item\label{A:hbnof}  It is true that while you hold the book, there is no \textit{net force}, but that does not mean that there is no force acting.  If there were no forces on the book, then your hand would not need to be there.  In fact, if you remove the force your hand provides, then the book falls. This shows that there is an upward force (by your hand on the book) and a downward force (of gravity by the Earth on the book).  \return{IQ:holdbook}
\item\label{A:netF-a} Since the object in \ref{se:netF-a} has a mass of $2.0\unit{kg}$, we can find the weight by
    \[ \vec F_g = m \vec g = (2.0\unit{kg}) [-(9.81\unitfrac{m}{s^2})\,\jhat] = -\sig{19}{.62}{N} \jhat = -20 \unit N \jhat \]
    \return{se:weightA}
\item\label{A:chair3} For a chair with wheels being pushed across a tile floor, when you stop pushing it probably continues to move across the floor for at least a short distance.  \autoreturn{irl:NI}
\item\label{A:weight.gain} When one person stands on the scale, the scale provides just enough of an upwards normal force to keep that person in equilibrium\dothis{link equilibrium?}.  In that case, the upwards force is balancing the weight of the person.  This gives the impression that the scale is telling you your weight; however, when you press down or help support whomever is standing on the scale, the scale adjusts the amount it must provide.  The scale is not trying to tell you your weight.  Rather the scale is trying to create equilibrium by balancing whatever force(s) are pressing into it.  Your weight is determined by the gravitational force\dothis{link the gravitational force?}{} and does not change when you press harder or lighter onto the scale.  \autoreturn{irl:scale}
\item\label{A:nowall} If we consider $\mu_w\rightarrow 0$, then $F_{fw}=0\unit N$,  $\vec F_{Nf} = -\vec F_g = \sig{222}{.7}{N} \jhat$, and $\vec F_{Nw} = - \vec F_{ff} = \sig{28.7}{9}{N} \ihat$.  In this case, $\mu_f$ could be as small as $0.\sig{129}{3}{}$ and still hold the ladder in place, unless \studentC\index{\studentC} climbs the ladder, in which case see \ref{A:nowallC}.  \autoreturn{ex:ladder2}
\item\label{A:true1} It is in equilibrium.  When the acceleration is zero, then the net force must be zero and those properties are what define equilibrium. \return{IQ:holdbook}
\item\label{A:chair4} The chair continues to move for the same reason that the chair without wheels and the chair on carpet \textit{all} continued to move when you let go.  The reason is that this is \textit{how all objects behave; they maintain their velocity when allowed to act without interference.}  (This is why Newton's first law says what it does.)   Because the chair with wheels has much less friction there is a smaller force trying to interfere with the motion and so it continues to move for a noticeable distance. The other chairs slowed to a stop almost immediately.  The wheel-less chair on tile might have continued for a short distance if it was moving fast enough that it required a long enough time to change its velocity to zero.  \autoreturn{irl:NI}
\item\label{A:scale.increase} When you press down on \studentB's shoulders, you are not adding weight.  Weight has a specific definition: it is specifically the value that the gravitational force\dothis{link the gravitational force}{} pulls on any object.  Pushing the person does not change their weight; it does, however, change the amount that they press into the Earth.  That is to say, it increases their downwards normal force, but not their weight.  \autoreturn{irl:scale}
\item\label{A:nowallC} If we consider $\mu_w\rightarrow 0$ with \studentC\index{\studentC} ($m=\massC$) at the third-rung-from-the-top of the ladder, ($1.53\unit m$ up the ladder), then $F_{fw}=0\unit N$,  $\vec F_{Nf} = \sig{1105}{.6}{N} \jhat$, and $\vec F_{Nw} = - \vec F_{ff} = \sig{171}{.97}{N} \ihat$.  In this case, $\mu_f$ could be as small as $0.\sig{155}{5}{}$ and still hold the ladder in place. \multireturn{\mmr{\ref{A:nowall}}, \mmr{\autoref{ex:ladder2}}}
\item\label{A:gworld} Because the Earth was spinning as it cooled (forming the crust), it formed an oblate spheroid\footnote{The equator is slightly further from the center than the poles are.}.  Since the strength of the gravitational interaction depends (among other things) on how far you are from the center (slightly weaker further away), the acceleration due to gravity is smaller when you are at smaller latitudes (closer to the equator).  \multireturn{\mmr{\ref{A:gpeaks}}, \mmr{\autoref{t:gworld}}}
\item\label{A:false1} The definition of equilibrium is that the forces balance.  The result of this is that the net force must be zero and the acceleration is then zero.  You can tell this is true because the velocity is \textit{not changing}.  It is not important that the velocity is zero, what is important is that the velocity \textit{stays} zero.  While you hold it, the book is in equilibrium. \return{IQ:holdbook}
\item\label{A:chair5} No. But if it does not matter what you do after you let go of the chair, then why do coaches (in basketball free-throws, tennis serves and swings, baseball pitches, and all manner of arm and leg propulsion) tell you to pay attention to your ``follow through''? \TWO{They have been fooled; follow-through doesn't matter}{they are right; follow-through does matter!}{A:noFT}{A:FT} \autoreturn{irl:NI}
\item\label{A:scale.measure} Since your weight is a force pulling downwards, having the scale on the wall shows that the scale cannot be balancing weight.  Since you are pushing into the wall, you are exerting a normal force into the scale and the scale is exerting a normal force back at you.  Both of these forces are horizontal (assuming the wall is plumb).  \autoreturn{irl:scale}
\item\label{A:gpeaks} In addition to being an oblate spheroid (\ref{A:gworld}), the Earth has mountains and valleys.  Since the strength of the gravitational interaction depends (among other things) on how far you are from the center (slightly weaker further away), the acceleration due to gravity is smaller when you are at at high altitudes, such as Denver, CO and Mount Everest.  \autoreturn{t:gworld}
\item\label{A:falls}  Both ``Yes'' and ``No'' bring you to this answer.  Yes, there is \underline{a force} on the book while it falls (the force of gravity), but no, there are not force\underline{s} (plural).  There is only one force.  ``But, wait!'' you say, ``What about the force of air resistance?''  Aha!   You are correct; there is a force of air resistance, but in this case, it is negligible and we will not consider it.  Please read \autoref{ss:airresistance} for more information about deciding when to use or ignore this phenomenon.  \return{IQ:holdbook}
\item\label{A:chair6} No.  When you throw a ball very high into the air, you can dance a jig or do any manner of things and it will obviously not affect the ball.  The force you exerted on the chair goes away the instant you stop touching the chair.  It is, however, true that your force gave the chair some velocity (actually \hyperref[c:momentum]{momentum}) and Newton's first law (inertia) says that the chair would prefer to keep that velocity.  Unfortunately, the friction with the ground slows it down.  The careful way to describe the situation is that your force gave the chair some velocity (actually \hyperref[c:momentum]{momentum}) and its characteristic inertia made it difficult for the \hyperref[s:Ff]{frictional force} to slow it down rapidly. \autoreturn{irl:NI}
\item\label{A:fly.balls} If you watch them carefully, you will notice that long fly balls are not parabolic.  It turns out that the air resistance is fairly complicated, but in the case of baseballs, the part that is relevant is that air resistance is strong when the ball is moving faster and weak when the ball is moving slower.  (This is different than the surface friction you will see in \autoref{s:Ff}.)  The effect of this is that the ball (usually) looks like it travels up into the air on a fairly straight path with a slight bend, which would produce a very wide parabola.  As it slows, the horizontal motion decreases, which tightens the parabola.  By the time the ball gets to its highest point, it is often travelling fairly slowly and has mostly all vertical motion by the time it drops into the outfielder's glove. \autoreturn{irl:nonparabolic}
\item\label{A:hitY} The book is accelerating.  The velocity \underline{is changing} from ``moving downwards'' to ``stopped''.  The book is not in equilibrium.  \return{IQ:holdbook}
\item\label{A:noFT} They haven't been fooled, but follow-through matters in a different way.  What does matter is not literally how you move \textit{after} the release, but rather how you move \textit{before} you release the ball. By paying attention to your follow-through, you are also changing the way you move before you release or impact the ball.  You want a smooth flow throughout the motion and a sloppy follow-through often implies a sloppy initiation of the motion.  \return{A:chair5}
\item\label{A:scale.ramp} When the scale is on the flat, horizontal floor, it balances your full weight.  When the scale is on the vertical wall it does not carry any of your weight.  At any angle in between those values, it carries some fraction of your weight while friction keeps you from sliding down the ramp\dothis{link to the section on friction and ramps}.  It will turn out that since the cosine function\dothis{link to the trig section}{} behaves in just the right way, we can use\dothis{link ``can use'' to the section on ramps}{} the cosine to find the component of the weight that the normal force from the scale has to support.   \autoreturn{irl:scale}
\item\label{A:pitches.side} The way a pitch travels is highly dependant on the way the pitcher releases the ball.  As the ball rolls out of the pitcher's hand, a spin is (usually) given to the ball and this spin interacts with the air to modify the direction that the air presses on the ball during the flight.  This will slightly affect the flight of the ball during the time it takes for the ball to get from the pitcher's mound to home plate.  In addition, fast balls have less time for the gravitational force to pull the ball down, so they will curve downwards less than a slower pitch.  This makes following the path of the ball somewhat difficult, but with some practice and careful attention, you should be able to see it.  All balls will drop somewhat, but the effect of the air resistance is exactly the mechanism for making a pitch unpredictable, so it is unlikely that you see the ball drop in a clean parabolic path.  \autoreturn{irl:nonparabolic}
\item\label{A:hitN} While it is hitting the desk, the velocity is changing from ``moving downwards'' to ``stopped''.  Since the velocity is changing, \underline{the book is accelerating}.  Since it is accelerating, the book is not in equilibrium.  \return{IQ:holdbook}
\item\label{A:chair7} The force that the chair feels after you release it is \hyperref[s:Ff]{friction}.  For the carpet, there is a lot of friction and the chair slows down very quickly (essentially instantaneously).  For the wheel-less chair on the tile floor, the chair slows rapidly although it may leave your hand.  The wheels provide the least amount of friction and that chair goes the furthest.  You may note that the friction slowing the chair-with-wheels is primarily between the rolling wheel and its axel (where it connects to the non-rolling chair leg) rather than between the wheel and the floor (although the friction between the wheel and the floor also plays a role).  This is discussed in more detail in \autoref{s:Ff}. \autoreturn{irl:NI}
\item\label{A:pitches.top} The way a pitch travels is highly dependant on the way the pitcher releases the ball.  As the ball rolls out of the pitcher's hand, a spin is (usually) given to the ball and this spin interacts with the air to modify the direction that the air presses on the ball during the flight.  In many cases, this will affect the flight of the ball (especially to the right or to the left) during the time it takes for the ball to get from the pitcher's mound to home plate.  If you watch from behind home plate, this sideways motion should be fairly clear.  \autoreturn{irl:nonparabolic}
\item\label{A:landedY} It is in equilibrium.  The book is at rest and \textit{continues to be} at rest on the desk. There are forces acting, but they cancel each other, resulting in no net force.  \return{IQ:holdbook}
\item\label{A:gravity}  It is the force of gravity. \return{A:hbf}
\item\label{A:chair8} If there were no friction, then you could start the chair and it would move on its own at a constant speed; you wouldn't need to continue pushing to keep it moving.  On the other hand, if you did continue to push, then the chair would continue to speed up and you would have to run faster and faster to keep up with it. On the other hand, if the chair were not experiencing friction, then you probably wouldn't either and you couldn't get enough traction to keep up with the chair, so it would sail away almost immediately, being then described by Newton's first law! \autoreturn{irl:NI}
\item\label{A:pool.roll} First, you should not roll a pool ball across just any floor; there is felt on the pool table for a reason.  However, if you have a clean, smooth surface and are able to reproduce your rolling speed, you will find that the pool ball rolls further on the stiff, nonyielding surface than it will on the felt.  The reason for this is beyond the scope of this textbook, but you can read more from \href{http://stacks.iop.org/0031-9120/30/i=3/a=009}{``Sliding and rolling: the physics of a rolling ball,'' J. Hierrezuelo and C. Carnero, Physics Education, Volume 30, Number 3} (unofficially at \href{http://billiards.colostate.edu/physics/Hierrezuelo_PhysEd_95_article.pdf}{this PDF}). \autoreturn{irl:poolcushion}
\item\label{A:landedN} After it has landed, the book stops moving.  Once the book comes to rest on the desk, it \textit{continues to stay at rest}.  This says that the velocity is not changing, so the book is not accelerating.  That means that the book is in equilibrium. There are forces acting, but they cancel each other, resulting in no net force. \return{IQ:holdbook}
\item\label{A:FT} What does matter is not literally how you move \textit{after} the release, but rather how you move \textit{before} you release the ball. By paying attention to your follow-through, you are also changing the way you move before you release or impact the ball.  You want a smooth flow throughout the motion and a sloppy follow-through often implies a sloppy initiation of the motion.  \return{A:chair5}
\item\label{A:pool.bumper} The cushion (sometimes called a bumper) is pretty still to the touch, but it is made of a springy rubber that allows the balls to bounce reasonably well.  The \protect{\href{http://c.ymcdn.com/sites/bca-pool.com/resource/resmgr/imported/BCAEquipmentSpecifications_2008.pdf}{document}} indicates that you should be able to firmly strike a ball at some angle to the far wall and have it bounce around the table four to four-and-a-half times.  If the bumpers were perfectly \hyperref[s:elastic]{elastic}, then the normal force would be normal to the restign surface; but since the bumper has some flexibility, when the ball hits the bumper with a glancing blow, then bumper bends inwards and the normal force is directed in a way that depends on the shape of the dent.  \autoreturn{irl:poolcushion}.
\item\label{A:zero} Recall the situation when you were holding the book.  Gravity is still pulling the book down and the desk is holding the book up.  There are two forces acting on the book while it is at rest on the desk. \return{IQ:holdbook}
\item\label{A:firstfall} As long as you are careful about releasing at the same time, you should not see any object consistently land first or consistently land last.  It is true that a piece of paper  will consistently land last, but this is because of the air resistance that we previously agreed to avoid.  If you crumple the paper into a tight ball (yes, it has to be a tight ball), then this will minimize the effect of air resistance and you might still be able to make the comparisons.   It is possible that some of the objects you are dropping (such as those in \ref{A:firstwhy}) have a shape that makes air resistance relevant.  \autoreturn{irl:freefall}
\item\label{A:floor}  I hope you guessed the floor.  That is the only thing pushing up on \studentB\index{\studentB}.  One useful way to think about it is that the floor is the thing keeping \himB\ from falling.  The direction of this force is \textit{normal} (perpendicular) to the horizontal floor, so it is in the vertical direction.  This will be discussed in more detail in \autoref{s:FN}. \return{se:FNB}
\item\label{A:firstwhy} As long as you are careful about releasing at the same time, it is unlikely that you will find anything consistently falling faster or slower than the others.  If you do notice a pattern, then the likely culprit is that air resistance is having an effect.  If you have something flat, like a computer (!) or a book that is falling more slowly than something else, like a hammer, then try dropping the flat object in different orientations to see if that affects the air resistance.  If you have something somewhat cylindrical, like a wine bottle (!) or a pencil that is falling more quickly than something else, like a hammer, then try dropping the cylindrical object in different orientations to see if that affects the air resistance. Remember that we are trying to eliminate differences due to air resistance so that we can study the effect of the gravitational force. \textbf{The effect you should notice is that so long as air resistance does not affect one object differently than the other, all objects fall at the same rate.}  \autoreturn{irl:freefall}
\item\label{A:noncue} The \href{http://wpapool.com/equipment-specifications/\#Balls-and-Ball-Rack}{specifications} show that there is no difference between the solids and stripes, but the cue ball weighs $9\%$ more that the other balls ($6.0\unit{oz}$ versus $5.5\unit{oz}$).  The colored balls and the cue ball are otherwise identical.  \autoreturn{irl:poolcushion}
\item\label{A:one} If there were only one force on the book, it could not be a balanced force, so the book could not be in equilibrium and the book would be accelerating.  The book is not accelerating, so there are either two forces (\ref{A:two}) or no forces (\ref{A:zero}).  \return{IQ:holdbook}
\item\label{A:fallv} When you drop something from eye-level, it takes less than a half-second for it to hit the ground.  Due to the limited need to gauge speed, it is very difficult for most humans to distinguish constant speed from accelerated motion in this small of a time interval.  Athletes can often tell is an object is moving fast or slow, but even then it is difficult to gauge acceleration.  Practice measuring the time-of-flight by counting out loud: ``one-one-thousand\ldots two-one-thousand\ldots''.  For this fall, you will likely only get to ``one-one-thou''. \autoreturn{irl:freefall}
\item\label{A:pool.spin} Because the bumper is covered in felt, it has a small grip on the ball.  Because the bumper has some give to it, it dents in when hit and provide more surface area, which increases the grip.  Both of these mean that the spin of the ball gets transferred to the pool table somewhat and change the way a spinning ball exits from the bumper collision.  \autoreturn{irl:poolcushion}
\item\label{A:second} You can tell that it is Newton's second law $(\vec F_\mathrm{net} = m \vec a)$ because the forces we are considering are acting on the \textit{same} object.  In this case, the gravitational force is caused by the Earth and the normal force is caused by the floor by they are both felt by \studentB\index{\studentB}.  These forces happen to be equal and opposite because \heB\ happens to be in equilibrium. \HeB\ does not \textit{have to be} in equilibrium, such as when \heB\ jumps, in which case the forces would not be equal and might not be opposite. \return{se:FNB}
\item\label{A:falla} Measure the time-of-flight by counting out loud: ``one-one-thousand\ldots two-one-thousand\ldots''.  For the four rungs near the top, you will likely only get to ``one-one-thou''.  For the four rungs near the bottom, you will likely only get to ``one-wa''.  Since those two distances are the same, it should be clear that the object is going faster at the bottom of the ladder. \textbf{Objects speed up (accelerate) while they fall.}  \autoreturn{irl:freefall}
\item\label{A:pool.later} This answer is getting \important{too complex for the section} it is in.  I need to move the IRL before I finish considering how to answer this question. \autoreturn{irl:poolcushion}
\item\label{A:two} There are two forces acting on the book while it is at rest on the desk. Similar to the situation when you were holding the book, gravity is pulling the book down and the desk is holding the book up.  \return{IQ:holdbook}
\item\label{A:third} If it were Newton's third law, then the two forces we were discussing would be acting on different objects and would be unrelated to the fact that the object (in this case, \studentB\index{\studentB}) is in equilibrium.  The gravitational force and the normal force in this case are both acting on \studentB, so although they happen to be equal and opposite, this is not due to Newton's third law.

    You should, however, note that the force that is reaction-paired to the gravitational force on \studentB\index{\studentB} by the Earth is a gravitational force on the Earth by \studentB.  Similarly, the reaction-paired force to the normal force on \studentB\ by the floor is a normal force on the floor by \studentB.  (Please note the ``on'' and ``by'' in each case.) \return{se:FNB}

\item\label{A:swing.tension} To make this comparison, let's consider a swing that is supported by chains.  If you are sitting in the swing and take hold of the chains at about shoulder height, you should be able to shake them in (towards your chest) and out (away from you, towards your neighbor swings).  You can do this same motion while standing next to the swing.  If you do this when the swing is empty, it is very easy to do this.  If you ask a series of successively larger people to sit in the swing, you will notice that it gets progressively more difficult to extend them very far.  The chains are increasing in tension; they are pulled more taut.  Your ability to move the chain in this way is exactly analogous to the way a bow draws across a violin or the way your fingers pluck a guitar, as described in \autoref{ss:stringed.instruments}. \multireturn{\mmr{\autoref{irl:tension}}, \mmr{\ref{A:chandelier.tension}}}
\item\label{A:fan.tension} You might also consider \ref{A:chandelier.tension}, which discusses the case of hanging a light fixture from the ceiling. If you have ever installed a fan in your house, then you will notice that you have to support the fan while the wires are connected.  Usually the fan has a shaft that connects to the ceiling at one end and the fan at the other and provides a mechanism for supporting the fan while you manage the wires, which pass through the shaft.  Since the fan houses the motor, it is usually reasonably heavy.   The nice property of use a metal shaft to support the fan is that it doesn't stretch or wiggle like a chain might.  The difficulty in this example is that it is more difficulty to notice the tension in the shaft.  If you are the person hanging the fan, then one thing you might be able to notice is that if you flick the metal with a finger when it is not supporting anything, it will have a slightly different ``ting'' than when it is supporting the fan.\dothis{Is this sufficiently noticeable?} \autoreturn{irl:tension}
\item\label{A:chandelier.tension} If you have ever installed a chandelier in your house, then you will notice that the light has to be supported between the joists of the ceiling.  There will be an electrical box with a screw to which you will attach the support for the chain that holds the chandelier.  The wires will run through the support chain.  The heavier the chandelier, the tauter the chain, much as described in \ref{A:swing.tension}. This tension is much easier to see than the tension in the shaft of the fan.  \return{A:fan.tension}
\end{AIQ}

\chapter{Adventures}

Throughout the book, there are examples and adventures.  The follow-up stories are contained below.
\begin{Story}
\item\label{a:parkandwalk} Just as planned, you pull over and park the car.  \studentB\index{\studentB} suggests one of you stays with the car, probably because \heB\ has physics homework to do.  If you decide to separate, read \ref{a:nogas}.  If you decide to journey together, read \ref{a:intosunset}.
\item\label{a:NIIIaction} As \studentC\index{\studentC} gets pushed, you notice that \heC\ was not aware of the pending doom.  \HeC\ is standing casually with \hisC\ feet set to support his own weight, but not to brace \himC\ against the sideways force.  When \heC\ gets pushed from the side, \hisC\ feet stay in place and \hisC\ torso topples, rotating \himC\ about \hisC\ center of mass\Foreshadow{The physics of why an object (or person) rotates when they fall over is discussed with \protect{\hyperref[c:torque]{torque}}.}{} as \heC\ falls to the ground. \studentD\ points to \hisD\ phone and says, ``I recorded the whole thing!''  If you respond, ``Awesome! Can I watch the part about how \studentZ\index{\studentZ} acts?'', please read \ref{a:NIIIreaction}.  If you respond, ``Awesome! Let's show the psychology and physics faculty our cool video!'', please read \ref{a:NIIIfaculty}. If you respond, ``Yeah, we probably should have intervened before this happened instead of just watching.  Let's go talk to Campus Security.'', please read \ref{a:NIIIsecurity}.
\item\label{a:coastindrive} As soon as you decide to do this, the gas runs out.  Thinking you can make it to the gas station, you take your foot off of the gas pedal.  You slow down fairly quickly and get nervous that you might get rear-ended.  You turn on the hazard-lights.  After about a minute you are travelling $30 \unit{mph}$ and you pass the time by working out \autoref{ex:slowcar}  (pg.~\pageref{ex:slowcar}).  People are honking at you as they try to pass.  \studentB\index{\studentB} turns to you and asks you why you are going so slow.  If you start a discussion about Newton's First Law, then go to \ref{a:NIdrive}.  If you get embarrassed and decide to pull over, then read \ref{a:parkandwalk2}.
\item\label{a:NIIIreaction} As \studentZ\index{\studentZ} pushes, you notice that because \heZ\ was being intentional, \heZ\ put one foot behind \himZ\ to brace \hisZ\ body during the push.  \HeZ\ leans into the push and stays standing.  You are intrigued.  If you decide to do a follow-up experiment by pushing \studentD\index{\studentD} over without bracing yourself, then read \ref{a:NIIIexperiment}.  If you decide to exercise self-restraint, then read \ref{a:NIIIrestraint}.
\item\label{a:coastinneutral} You speed up to $60 \unit{mph}$ before the gas runs out and then you quickly pop the car into neutral.  You slow down gradually and, in an effort to not get rear-ended, you cleverly turn on the hazard-lights.  After about a minute you are travelling $52 \unit{mph}$ and you pass the time by working out \autoref{ex:coasting}  (pg.~\pageref{ex:coasting}).  After $2\unit{min}$, you are travelling $43 \unit{mph}$ and people are getting impatient as they try to pass.  After $2.79\unit{min}$, you triumphantly coast into the gas station at a comfortable speed of $36.7\unit{mph}$.  \studentB\index{\studentB} is so happy, \heB\ buys you a full tank of gas and the two of you start a discussion about Newton's First Law while pumping the gas.  Please read \ref{a:NIresult}.
\item\label{a:NIIIconcern} Being the thoughtful and considerate person you are, you rush over and startle \studentC\index{\studentC} out of \hisC\ reverie.  \studentZ\index{\studentZ} is quite angry and now focuses \hisZ\ attention on you!  \HeZ\ rushes towards you and shoves as hard as he can.  You go \textit{flying} backwards and land on your tailbone while he just stands there laughing.  \studentC\ and \studentD\index{\studentD} both rush over to help you while \studentZ\ wanders off.  Surprisingly, \studentC\ has an icepack, which helps.  If you go speak to your faculty members about this, please read \ref{a:NIIIfaculty}. If you decide to talk to Campus Security, please read \ref{a:NIIIsecurity}.
\item\label{a:NIdrive} After some discussion, you and \studentB\index{\studentB} realize that when the car is in drive, the transmission (the part of the car that converts how-fast-the-engine-spins to how-fast-the-axel-and-wheels-turn) is connected to the axel, which means that the rolling wheels are trying to turn the engine parts as well as the wheels themselves.  The engine parts have grease and oil, but still take a lot of energy to turn.  This causes friction, which dissipates energy and, more importantly, exerts a backwards force on the spinning wheels.  Your car is not being described by Newton's First Law, which requires there to be no force applied.  Instead your car is being described by Newton's Second Law and the force is changing the velocity to cause you to go slower.  It only takes the car $2.2\unit{min}$ to stop and you still have to walk to the gas station.  \studentB\ laments ``If only there were a way to reduce the force on the axel\ldots'' If it occurs to you to speculate about putting the car in neutral when the gas ran out, then imagine reading \ref{a:coastinneutral}.  If you stop talking and walk to the gas station, then read \ref{a:intosunset2}.
\item\label{a:NIIIexperiment} You turn and push \studentD\index{\studentD} over.  Like \studentC\index{\studentC}, \heD\ did not expect it and was not braced, so \heD\ falls over. Similarly, you decided not to brace yourself and in pushing \studentD, you fall over backwards!  \studentD\ did not have to \textit{choose} to push on you.  The act of you deciding to push \himD\ necessarily and simultaneously produces a force on you, equal in magnitude and opposite in direction.  Unfortunately, \studentD\ doesn't think this was a useful exercise and shouts ``I have the whole thing on video!'' and storms off to Campus Security.  You are arrested for assault, miss your physics class for a couple of weeks and ultimately fail all of your classes. I certainly hope this was all happening in your head and not in real life!  You learned something about physics, but at what cost to your humanity?  \hyperref[cyoa:NIII]{The end!}
\item\label{a:nogas} You leave \studentB\index{\studentB} in the car and walk the 45 minutes to the gas station.  You buy a gas can, fill it up, and start to carry it back to the car.  It is very heavy and you notice vultures circling overhead.  You hope you survive this.  It might have been a better idea to bring \studentB\ with you to share the burden.  You stumble once, and then again.  You swear to be more cautious about estimating your gas consumption.  After walking for what seems like hours and stumbling back to the car, you find \studentB\ very excited. \HeB\ declares that \heB\ has invented a time machine so you can go back to the \hyperref[cyoa:NI]{adventure} and start over to learn something about Newton's First Law!
\item\label{a:NIIIrestraint} \studentD\index{\studentD} points to \hisD\ phone and says, ``I recorded the whole thing!''  If you respond, ``Awesome! Can I watch the part about how \studentC\index{\studentC} acts?'', please read \ref{a:NIIIaction}.  If you respond, ``Awesome! Let's show the psychology and physics faculty our cool video!'', please read \ref{a:NIIIfaculty}. If you respond, ``Yeah, we probably should have intervened before this happened instead of just watching.  Let's go talk to Campus Security'', please read \ref{a:NIIIsecurity}.
\item\label{a:parkandwalk2} You pull over and park the car.  \studentB\index{\studentB} suggests one of you stays with the car, probably because \heB\ has physics homework to do.  If you decide to separate, read \ref{a:nogas2}.  If you decide to journey together, read \ref{a:intosunset}.
\item\label{a:NIIIfaculty} The psychology faculty member speaks to you both about how to be good citizens and about the psychological effects of bullies both on the bully and on the recipient.  If you decide to learn more about this, please read \href{https://www.psychologytoday.com/basics/bullying}{Psychology Today}.  The physics faculty member points out that when one person pushes another, the person being pushed does not brace \himselfC, whereas the person doing the pushing does.  Furthermore one might imagine what would happen if you did not brace yourself when you pushed each other, such as in \ref{a:NIIIexperiment}.  You are asked to review both \autoref{ex:braced}  (pg.~\pageref{ex:braced}) and \autoref{ex:unbraced}  (pg.~\pageref{ex:unbraced}) before the next exam.  On your way out the door, you hear a voice suggest ``\ldots and you \textit{might} want to talk to \hyperref[a:NIIIsecurity]{Campus Security} about the incident\ldots''
\item\label{a:intosunset} Everything goes as planned.  You drive off into the sunset sadly ignorant of the physics you might have learned. \hyperref[cyoa:NI]{The end!}
\item\label{a:NIIIsecurity} You speak with Campus Security about the incident and \studentZ\index{\studentZ} gets taken in for assault.  The Dean thanks you for being brave enough to speak up. \hyperref[cyoa:NIII]{The end!}
\item\label{a:NIresult} During the discussion, you and \studentB\index{\studentB} realize that the rolling wheels and the spinning axel are still connected to the not-spinning frame of the car.  While this causes less friction than if the car were in drive, there is still some friction, which dissipates energy and, more importantly, exerts a backwards force on the spinning wheels.  Your car is not being described by Newton's First Law, which requires there to be no force applied.  Instead your car is being described by Newton's Second Law and the force is changing the velocity to cause you to go slower.  You finish getting gas and drive on to many happy adventures. \hyperref[cyoa:NI]{The end!}
\item\label{a:guilty} You feel guilty for letting \studentZ\index{\studentZ} push \studentC\index{\studentC} down despite your amazing score on the next physics test.  It wasn't worth it.  \hyperref[cyoa:NIII]{The end!}
\item\label{a:nogas2} You leave \studentB\index{\studentB} in the car and walk the 31 minutes to the gas station.  You buy a gas can, fill it up, and start to carry it back to the car.  It is very heavy and you notice vultures circling overhead.  You hope you survive this.  It might have been a better idea to bring \studentB\ with you to share the burden.  You stumble once, and then again.  You swear to be more cautious about estimating your gas consumption.  After walking for what seems like hours and stumbling back to the car, you find \studentB\ very excited. \HeB\ declares that \heB\ has invented a time machine so you can go back to the \hyperref[cyoa:NI]{adventure} and start over to learn something about Newton's First Law!
\item\label{a:intosunset2} You and \studentB\index{\studentB} happily walk the 25 minutes to the gas station, discussing and working out physics problems the whole way.  You buy a gas can, fill it up, and share the burden of carrying a heavy gas can.  You return to the car, add gas, and drive on to many happy adventures.  \hyperref[cyoa:NI]{The end!}
\end{Story}

%%%%%%%%%%%%%%%%%%%%%%%%%%%%%%%%%%%%%%%%%%%%%%%%%%%%%%%%%%%%%%%%%%%%%%%%%%%%%%%%%%%%%%%%%%%%%%%%%%%%%%

\chapter{Characters}

This textbook has five characters who follow you throughout the book.  They appear in the examples and some homework problems.  They also remember previous experiences.  I need to adjust the examples in \autoref{c:force} such that the people pushing boxes are helping the reader rearrange furniture.

The index lists\dothis{The index will recognize the people in two different formats.  One is by my name for them, which is $\backslash$studentX (where X is A, B, C, D, \ldots Z).  The other is by the name assigned to that variable.  So these show up in different places in the Index.}{} the pages that the characters appear.  The point of this chapter is to highlight some of the primary adventures of the characters according to their own perspectives.  \textbf{None of the links in this chapter will be given a corresponding return link.}  This chapter is for me to track relationships and will likely go away when the book is ready for publication.
%
I can, at the header of the code, define the name, gender, mass, and dimensions of each individual.\dothis[inline]{\href{http://malveyauthor.com/}{Madeline Alvey}, the author of  \protect{\href{http://escapepod.org/2017/03/09/ep566-honey-and-bone-artemis-rising-3/}{``Honey and Bone'' at EscapePod}} is a physics and English undergraduate student at UK in Lexington.  I might consider hiring(?) her to help storyboard the characters.}

\section{\studentA\index{\studentA!inside}}\index{\studentA!outside}\dothis{The index-call that is \textbf{outside} of the section title registers as $\backslash$studentA, which puts the name alphabetically under $\backslash$studentA, rather than \studentA.  The index-call that is \textbf{inside} of the section title registers as \studentA, which puts the name alphabetically under \studentA, rather than $\backslash$studentA.}
\index{\studentA|(} % Begin page-range
\begin{itemize}
\item In \autoref{s:forcewords}, \studentB{} gives \studentA{} a good-natured shove in the arm in order to get the language clarified and begin the conversation about the on-by notation.
\item In \ref{se:FBD-AB} \studentA{} helps \studentB{}\ldots
    \begin{itemize}
    \item (in the current version) push an object to make it accelerate and feel a reaction force causing \himA{} to accelerate backwards.
    \item (in the future version) will help the reader move into or out of their residence hall by pushing on heavier furniture.
    \item[NOTE:] This is all drawn in \autoref{f:firstFBD}, which is updated in \autoref{f:firstFBDupdate}.
    \end{itemize}
\item In \ref{se:weightA}, \studentA{} falls from a small height.  (maybe he is jumping off a short ledge while taking a short-cut to class?)
\item In \autoref{ex:baking}, \studentA{} decides to bake some bread for a party at \studentB's house, measuring the time it takes to warm his oven.
\end{itemize}

\index{\studentA|)} % end page-range
\section{\studentB\index{\studentB}}
\index{\studentB|(} % Begin page-range

\begin{itemize}
\item \studentB{} is a passenger in the reader's car in \autoref{ex:slowcar} when the reader runs out of gas and coasts to a stop.
\item \studentB{} is a passenger in the reader's car in \autoref{ex:coasting} and speculates about how fast to go before putting the car in neutral to coast to a stop.
\item \studentB{} joins the reader on a road trip in \autoref{cyoa:NI} and runs out of gas.  This results in multiple possible adventures:
\begin{itemize}
    \item \ref{a:parkandwalk}, which leads to either an end at \ref{a:nogas} or an end at \ref{a:intosunset}.
    \item \ref{a:coastindrive}, which leads to either \ref{a:NIdrive} (choose \ref{a:coastinneutral} or end with \ref{a:intosunset2}) or \ref{a:parkandwalk2} (choose \ref{a:intosunset} or end at \ref{a:nogas2})
    \item \ref{a:coastinneutral}, which leads to an end at \ref{a:NIresult}.
\end{itemize}
\item In \autoref{s:forcewords}, \studentB{} gives \studentA{} a good-natured shove in the arm in order to get the language clarified and begin the conversation about the on-by notation.
\item In \autoref{se:FBD-AB}, \studentB{} helps \studentA{}\ldots
    \begin{itemize}
    \item (in the current version) pull an object to make it accelerate and feel a reaction force causing \himB{} to accelerate backwards.
    \item (in the future version) will help the reader move into or out of their residence hall by pushing on heavier furniture.
    \item[NOTE:] This is all drawn in \autoref{f:firstFBD}, which is updated in \autoref{f:firstFBDupdate}.
    \end{itemize}
\item In \ref{se:FNB}, \studentB{} has a normal force supporting \himB.  (This touches \ref{A:floor}, \ref{A:second}, and \ref{A:third}.)
\item At some point, \studentB{} has a party, because in \autoref{ex:baking}, \studentA{} decides to bake some bread for a party at \studentB's house.
\end{itemize}

\index{\studentB|)} % end page-range
\section{\studentC}
\index{\studentC|(} % Begin page-range

\index{\studentC|)} % end page-range
\section{\studentD}
\index{\studentD|(} % Begin page-range

\index{\studentD|)} % end page-range
\section{\studentE}
\index{\studentE|(} % Begin page-range

\index{\studentE|)} % end page-range
\section{\studentF}
\index{\studentF|(} % Begin page-range

\index{\studentF|)} % end page-range
\section{\studentZ}
\index{\studentZ|(} % Begin page-range

\index{\studentZ|)} % end page-range



%%%%%%%%%%%%%%%%%%%%%%%%%%%%%%%%%%%%%%%%%%%%%%%%%%%%%%%%%%%%%%%%%%%%%%%%%%%%%%%%%%%%%%%%%%%%%%%%%%%%%%

\addcontentsline{toc}{chapter}{Index}
%\printindex
\documentclass[11pt,letter,openany,makeidx]{book}
\usepackage{amsmath}
\usepackage{macros}
\usepackage{comment}
\usepackage{graphicx}
\usepackage{microtype}
\usepackage{gfsdidot}
\usepackage[T1]{fontenc}
\usepackage{booktabs}
\usepackage{underscore,cancel}
\usepackage{caption}
\usepackage[within=chapter,chapterlistsgap=6pt]{newfloat}
\usepackage{tocloft}
\usepackage{xpicture}
\usepackage{xcolor}
%\usepackage[dvips]{xcolor}
%\GetGinDriver  % for xcolor to work well with hyperref
%\usepackage[\GinDriver]{hyperref}
\usepackage{ulem}%for \sout (done)
\usepackage[colorinlistoftodos]{todonotes}
%\usepackage[disable,colorinlistoftodos]{todonotes}
%\usepackage{layouts}
%\usepackage{showframe}
\usepackage{coordsys}
\usepackage{tikz}
\usepackage[pdftex]{hyperref}

%\usepackage{cellpage}

\hypersetup{colorlinks=true,bookmarks=true,pdftitle=Algebra-Based Introductory Physics,pdfauthor=J.Christensen,pdfdisplaydoctitle}
% If using a bibliography, then include "backref" in list of \hypersetup items
% linkcolor=color of internal links (red); anchorcolor = color of anchor text (black); citecolor = bibliographic citations (green); filecolor = color for local URL files (cyan); menucolor = Acrobat menu (red); urlcolor = external links (magenta); hidelinks = remove all color
% citebordercolor = color of box for citations (0 1 0); fileborder = links to files box (0 .5 .5); linkbordercolor = normal links (1 0 0); menuborder; urlborder; allbordercolors; pdfborder

\includecomment{ForMe}
\includecomment{ForReviewer}
\includecomment{ForPublic}

\makeindex

\newlistof{example}{loe}{List of Examples}
\DeclareFloatingEnvironment[fileext=loe,listname="List of Examples",name=Example]{example}
\setlength{\cftexamplenumwidth}{1cm}
\newcounter{sample}
\newcounter{carrysample}
\renewcommand{\thesample}{Simple Example \arabic{sample}}
\renewcommand{\thecarrysample}{Simple Example \arabic{carrysample}}
\newenvironment{sample}{\color{rgb:red,0;green,2;blue,1}\begin{list}{\textbf{\thesample}:}{\usecounter{sample} \setcounter{sample}{\value{carrysample}} \leftmargin 12pt}}{\end{list}\setcounter{carrysample}{\value{sample}}}
\newcommand{\THREE}[6]{\vspace{-3pt}\begin{flushright} Select one:  \mbox{#1 (\ref{#4})},  \mbox{#2 (\ref{#5})}, or \mbox{#3 (\ref{#6})}.\end{flushright}}
\newcommand{\TWO}[4]{\begin{flushright} Select one:  \mbox{#1 (\ref{#3})} or \mbox{#2 (\ref{#4})}.\end{flushright}}
\newcommand{\YN}[2]{\TWO{Yes}{No}{#1}{#2}}
\newcommand{\TF}[2]{\TWO{True}{False}{#1}{#2}}
\newcommand{\return}[1]{{} \hfill \mbox{Return to \ref{#1}.}}
\newcommand{\autoreturn}[1]{{} \hfill \mbox{Return to \autoref{#1}.}}
\newcommand{\linkreturn}[2][a related idea]{{}\hfill \mbox{Return to the discussion of \protect{\hyperlink{#2}{#1}}.}}
\newcommand{\mmr}[1]{\mbox{[\protect{#1}]}}
\newcommand{\multireturn}[1]{{}\hfill Return to one of the following locations: \newline #1.}
\newcounter{AtIQ}
\renewcommand{\theAtIQ}{Answer \arabic{AtIQ}}
\newenvironment{AIQ}{\begin{list}{\textbf{Interactive \theAtIQ}:}{\usecounter{AtIQ} \leftmargin 12pt}}{\end{list}}

% Related (return), but not part of...
\newcommand{\mreturn}[1]{\note{Return to \protect{\ref{#1}}.}}
\newcommand{\mlinkreturn}[2][a related idea]{\note{Return to the discussion of \protect{\hyperlink{#2}{#1}}.}}
\newcommand{\mautoreturn}[1]{\note{Return to \protect{\autoref{#1}}.}}
\newcommand{\mmultireturn}[1]{\note{Return to one of the following locations: \newline #1.}}


\newlistof{adventure}{loa}{List of Adventures}
\DeclareFloatingEnvironment[fileext=loa,listname="List of Adventures",name=Adventure]{adventure}
\setlength{\cftadventurenumwidth}{1cm}
\newcounter{CYOA}
\renewcommand{\theCYOA}{Plan \Alph{CYOA}}
\newenvironment{CYOA}{\begin{list}{\textbf{\theCYOA}:}{\usecounter{CYOA}}}{\end{list}}
\newcounter{storyline}
\renewcommand{\thestoryline}{Storyline \arabic{storyline}}
\newenvironment{Story}{\begin{list}{\textbf{\thestoryline}:}{\usecounter{storyline} \leftmargin 12pt}}{\end{list}}

\newlistof{reallife}{irl}{List of Real Life Patterns}
%\DeclareFloatingEnvironment[fileext=irl,listname="List of Real Life Patterns",chapterlistsgaps=off,name=Real Life Patterns]{reallife}
\DeclareFloatingEnvironment[fileext=irl,listname="List of Real Life Patterns",name=Real Life Patterns]{reallife}
\setlength{\cftreallifenumwidth}{1cm}
\newcounter{IRL}
%\renewcommand{\theIRL}{\arabic{IRL}}
\newenvironment{realtable}{%\renewcommand{\arraystretch}{2}
                           %\hspace{-.2in}
                            \begin{tabular}{@{}lll@{}} \toprule Do This & Notice This & Ask This  \\ }
                            {\bottomrule \end{tabular} }%\renewcommand{\arraystretch}{.5}}
\newcommand{\dna}[3]{\midrule \begin{minipage}{4cm}\raggedright #1 \end{minipage}
                   & \begin{minipage}{4cm}\raggedright #2 \end{minipage}
                   & \begin{minipage}{4cm}\raggedright #3 \end{minipage} \\ }
\newcommand{\multidna}[1]{\multicolumn{3}{|c|}{\begin{minipage}{13cm}\center #1 \end{minipage}} \\ \midrule }


\newlistof{story}{los}{The Stories of the Equations}
\DeclareFloatingEnvironment[fileext=los,listname="The Stories of the Equations",name=This Equation's Story]{story}
\setlength{\cftstorynumwidth}{1cm}
\newcommand{\thestoryof}[1]{\marginpar{\raggedright \footnotesize The story of \\ \fcolorbox{black}{yellow}{\begin{minipage}[c]{1.5in} \center $\deq #1$ \end{minipage}}}}
\newcommand{\EqStory}[2]{\left[ {\color{rgb:red,1;green,1;blue,4} \begin{minipage}{#1}\raggedright\begin{center} #2 \end{center}\end{minipage}} \right]}
\newcommand{\EqStoryOver}[3]{\overbrace{\EqStory{#1}{#2}}^{\displaystyle #3}}
\newcommand{\EqStoryUnder}[3]{\underbrace{\EqStory{#1}{#2}}_{\displaystyle #3}}
\newcommand{\EqStoryFrac}[5]{\frac{\overbrace{\EqStory{#1}{#2}}^{\displaystyle #3}}
                                 {\underbrace{\EqStory{#1}{#4}}_{\displaystyle #5}}}


%%%%%%%%%%%%%%%%%%%%%%%%%%%%%%%%%%%%%%%%%%%%%%%%%%%%%%%%%%%%
%
%\presetkeys{todonotes}{fancyline,color=blue!15}{}
\presetkeys{todonotes}{color=blue!15,linecolor=blue!75,size=\footnotesize}{}
%
\newcounter{todocounter}
\newcommand{\dothis}[2][]
{\stepcounter{todocounter}\todo[color=green!30, #1]{\thetodocounter: #2}}
\newcommand{\docaption}[3][]
{\stepcounter{todocounter}\todo[color=green!30, prepend, caption={\thetodocounter: \underline{#2}}, #1]{#3}}
\newcommand{\addlink}[2][]
{\stepcounter{todocounter}\todo[prepend, caption={\thetodocounter: \underline{Add Link}}, #1]{#2}}
\newcounter{todourgentcounter}
\newcommand{\urgent}[2][]
{\stepcounter{todourgentcounter}\todo[color=orange!50, #1]{\thetodourgentcounter: #2}}
\newcommand{\urgcap}[3][]
{\stepcounter{todourgentcounter}\todo[color=orange!50, prepend, caption={\thetodourgentcounter: \underline{#2}}, #1]{#3}}
\newcommand{\done}[2][]
{\todo[color=yellow!10, #1]{\sout{#2}}}
%
%\newcommand{\new}[2]{}%
\newcommand{\new}[2]{\marginpar{\raggedright \footnotesize New to #1 \\ \fcolorbox{blue}{yellow!10}{\begin{minipage}[c]{1.5in} \center {\color{blue} #2 } \end{minipage}}}}%
%%%%%%%%%%%%%%%%%%%%%%%%%%%%%%%%%%%%%%%%%%%%%%%%%%%%%%%%%%%%


%%%%%%%%%%%%%%%%%%%%%%%%%%%%%%%%%%%%%%%%%%%%%%%%%%%%%%%%%%%%
%
%\newcommand{\deq}{\displaystyle}
%\newcommand{\txtfrac}[2]{{}^{#1}\!/_{\!#2}}
%
%%%%%%%%%%%%%%%%%%%%%%%%%%%%%%%%%%%%%%%%%%%%%%%%%%%%%%%%%%%%



%%%%%%%%%%%%%%%%%%%%%%%%%%%%%%%%%%%%%%%%%%%%%%%%%%%%%%%%%%%%
%
% PEOPLE AND PRONOUNS
%
% According to https://www.cdc.gov/nchs/fastats/body-measurements.htm
% Measured average height, weight, and waist circumference for adults ages 20 years and over
% Men:
% Height (inches): 69.3                 = 1.760 m
% Weight (pounds): 195.5                = 88.86 kg
% Waist circumference (inches): 39.7    = 1.01 m
% Women:
% Height (inches): 63.8                 = 1.621 m
% Weight (pounds): 166.2                = 75.55 kg
% Waist circumference (inches): 37.5    = 0.9525 m
% Source: Anthropometric Reference Data for Children and Adults: United States, 2007-2010, tables 4, 6, 10, 12, 19, 20[PDF - 1.7 MB]
%  https://www.cdc.gov/nchs/data/series/sr_11/sr11_252.pdf
%
\newcommand{\studentA}{Abdul}       \newcommand{\massA}{\mbox{$85.0\unit{kg}$}}
\newcommand{\studentB}{Beth}        \newcommand{\massB}{\mbox{$75.0\unit{kg}$}}
\newcommand{\studentC}{Carl}        \newcommand{\massC}{\mbox{$90.0\unit{kg}$}}
\newcommand{\studentD}{Diane}       \newcommand{\massD}{\mbox{$80.0\unit{kg}$}}
\newcommand{\studentE}{Erik}        \newcommand{\massE}{\mbox{$95.0\unit{kg}$}}
\newcommand{\studentF}{Frances}       \newcommand{\massF}{\mbox{$85.0\unit{kg}$}}
\newcommand{\studentX}{Xerxes}       \newcommand{\massX}{\mbox{$62.5\unit{kg}$}}
\newcommand{\studentZ}{Zambert}     \newcommand{\massZ}{\mbox{$95.0\unit{kg}$}}
% Male
\newcommand{\heA}{he}\newcommand{\himA}{him}\newcommand{\hisA}{his}\newcommand{\himselfA}{himself}
\newcommand{\HeA}{He}\newcommand{\HimA}{Him}\newcommand{\HisA}{His}
\newcommand{\heC}{he}\newcommand{\himC}{him}\newcommand{\hisC}{his}\newcommand{\himselfC}{himself}
\newcommand{\HeC}{He}\newcommand{\HimC}{Him}\newcommand{\HisC}{His}
\newcommand{\heE}{he}\newcommand{\himE}{him}\newcommand{\hisE}{his}\newcommand{\himselfE}{himself}
\newcommand{\HeE}{He}\newcommand{\HimE}{Him}\newcommand{\HisE}{His}
\newcommand{\heZ}{he}\newcommand{\himZ}{him}\newcommand{\hisZ}{his}\newcommand{\himselfZ}{himself}
\newcommand{\HeZ}{He}\newcommand{\HimZ}{Him}\newcommand{\HisZ}{His}
% Female
\newcommand{\heB}{she}\newcommand{\himB}{her}\newcommand{\hisB}{her}\newcommand{\himselfB}{herself}
\newcommand{\HeB}{She}\newcommand{\HimB}{Her}\newcommand{\HisB}{Her}
\newcommand{\heD}{she}\newcommand{\himD}{her}\newcommand{\hisD}{her}\newcommand{\himselfD}{herself}
\newcommand{\HeD}{She}\newcommand{\HimD}{Her}\newcommand{\HisD}{Her}
\newcommand{\heF}{she}\newcommand{\himF}{her}\newcommand{\hisF}{her}\newcommand{\himselfF}{herself}
\newcommand{\HeF}{She}\newcommand{\HimF}{Her}\newcommand{\HisF}{Her}
%
\newcommand{\heX}{\studentX}\newcommand{\himX}{\studentX}\newcommand{\hisX}{\studentX's}\newcommand{\himselfX}{the person of \studentX}
\newcommand{\HeX}{\studentX}\newcommand{\HimX}{\studentX}\newcommand{\HisX}{\studentX's}
%%%%%%%%%%%%%%%%%%%%%%%%%%%%%%%%%%%%%%%%%%%%%%%%%%%%%%%%%%%%%


%%%%%%%%%%%%%%%%%%%%%%%%%%%%%%%%%%%%%%%%%%%%%%%%%%%%%%%%%%%%
%
% Book macros
%
\newcommand{\aside}[2]{\marginpar{\raggedright \footnotesize\textbf{#1}: #2}}
\newcommand{\important}[1]{\\ \fcolorbox{black}{yellow}{\begin{minipage}[c]{4.925in} \center #1 \end{minipage}}\\}
\newcommand{\inlife}{\marginpar[\scriptsize \raggedright How you might observe $\Rightarrow$ this in your life.]
                               {\scriptsize \raggedleft $\Leftarrow$ How you might observe this in your life.}}
\newcommand{\touchstone}{\marginpar[\scriptsize \raggedright Where have I seen this $\Rightarrow$ before?]
                                   {\scriptsize \raggedleft $\Leftarrow$ Where have I seen this before?}}
\newcommand{\foreshadow}{\marginpar[\scriptsize \raggedright When will I ever use this? $\Rightarrow$]
                                   {\scriptsize \raggedleft $\Leftarrow$ When will I ever use this?}}
\newcommand{\foreshadowR}{\reversemarginpar
                          \marginpar[\scriptsize \raggedright When will I ever use this? $\Rightarrow$]
                                    {\scriptsize \raggedleft $\Leftarrow$ When will I ever use this?}}
\newcommand{\Touchstone}[1]{\marginpar[\scriptsize \raggedright Where have I seen this $\Rightarrow$ \\ before? #1]
                                      {\scriptsize \raggedleft $\Leftarrow$ Where have I seen this before? #1}}
\newcommand{\Foreshadow}[1]{\marginpar[\scriptsize \raggedright When will I ever use this? $\Rightarrow$ \\ #1]
                                      {\scriptsize \raggedleft $\Leftarrow$ When will I ever use this? #1}}
%
%%%%%%%%%%%%%%%%%%%%%%%%%%%%%%%%%%%%%%%%%%%%%%%%%%%%%%%%%%%%


\begin{document}

%\title{Algebra-Based Introductory Physics}
%\author{J Christensen}
%\date{Jan 2017}
%\maketitle
%\pagestyle{cellpage}

\begin{titlepage}
	\centering
%	\includegraphics[width=0.15\textwidth]{example-image-1x1}\par\vspace{1cm}
	{\Huge\bfseries Physics Connected\par}
	\vspace{1cm}
	{\Large\bfseries An Algebra-Based Introductory Physics Textbook\par}
	\vspace{1cm}
	{\large Learn like you think: an interconnected view of physics\par}
	\vspace{2cm}
	{\Large\itshape by: J Christensen\par}
	\vfill
\begin{ForReviewer}
	Version 2.3\par
	{\footnotesize
    \begin{itemize}
    \item Ideas yet to implement:
        \begin{itemize}
        \item The examples are phrased as descriptions, not examples like the homework problems.  Need to consider rephrasing these, not calling them examples, or adding actual examples that better show how to respond to the way homework problems are written.
        \item Define a different page dimension that fits on a cell phone display.  (Enhance possible cell-phone reading.)
        \end{itemize}
    \item version 2.3: June 16-28, 2017
        \begin{itemize}
        \item Updated Section 81. $F=mg$ and Section 8.2 Normal Force
        \item Added specific list of Flame Challenges
        \item Rearranged some of the subsections in the ``Seeing Physics'', added references
        \item Equations of motion for constant acceleration (Need the Story Of)
        \item Added a section to Chap 5 (1-D motion) that gives examples of solutions that require multiple steps  (one equation is insufficient)
        \item Developed the weight and mass discussion and examples
        \item Ladder leaning example in torque, plus some homework problems
        \item Added some Conceptual Homework to weight/mass
        \item Added placeholders to the Gravity chapter
        \item Removed indicators of v1.7 changes
        \end{itemize}
    \item version 2.2: June 16, 2017
        \begin{itemize}
        \item Created conversation about $F=mg$ for Chapter on types of forces.  Caused modifications in lots of places
            \begin{itemize}
            \item Added freefall to the motion chapter
            \item Created IRL and Example dropping objects to see acceleration in $F=mg$, then moved to freefall section -- new Answers to interactive questions
            \item Commented on air resistance
            \item Comments about precision in language (need to do more with precision in mathematics)
            \item Started a couple of ideas about effective theories.  (need to decide where it goes)
            \item Added detail about SI, and specifically the pound-force, pound-mass, and kilogram. to sections \ref{s:SI-MKS} and \ref{ss:weightmass}
            \item Added NIST and BIPM references (found in Wikipedia and then searched further)
            \item conversation about weight and mass.  (required reference to the chapter on Fluids and density)
            \item Moved Google search about significant figures
            \end{itemize}
        \item Added comments about fundamental forces to the section on types of force
        \item Removed indicators of v1.5 and v1.6 changes
        \end{itemize}
    \item version 2.1: June 10, 2017
        \begin{itemize}
        \item Re-commented the $\backslash$new command
        \item Started the chapters on Seeing Physics [\autoref{c:physics}] and Deeper Dive [\autoref{c:revisted}] (These should be renamed)
        \item Moved some sections on fundamental interactions
        \end{itemize}
    \item version 2.0: April 10, 2017
        \begin{itemize}
        \item Re-enabled v1.8 hides
        \item Added a link to \textit{Spacepod}, \textit{Physics Footnotes}, and \textit{Sixty Symbols}
        \item Fixed a $\backslash$dothis that was inside an $\backslash$important, causing a compile error.
        \item Removed indicators of v1.4 changes
        \end{itemize}
    \item version 1.8: April 1, 2017
        \begin{itemize}
        \item Prepare for "the public": "Disabled" the To-Do items, "Hid" the $\backslash$new revision notes, Hid the List of Tables (have none yet)
        \end{itemize}
    \end{itemize}
    }
\end{ForReviewer}
\begin{ForPublic}
{\flushleft
\textbf{Note to the reviewers:}\new{v1.8}{Added the note}
My goal with this book is to create an electronically viewable book that makes use of the advantages of being electronic.  While current e-books have the advantage of being viewable on various devices with having to carry a physical book around, most e-textbooks do not take advantage of hyperlinked text.  With this book I hope to integrate links both forward and backward.  The forward links will be used to motivate curious students.  The backward links will be used to support students who lose track of previous topics.  The integration of these will also provide a convenient opportunity for students to browse through topics they are interested in.
\newpar

At this time, I am providing a single chapter to gauge the viability.  The chapter I am providing is on Newton's Laws.  However, as you read this document, you will find many, many more partially written chapters.  All of the partial chapters and sections are intended to be place-holders for the forward- and backward-links that \autoref{c:force} depends upon.
\newpar

I created this as a PDF that, I believe, can be easily viewed on a computer or tablet.  Since some of my students also seem to read on their phone, I verified that I am also able to view the text in a reasonable manner on my Samsung phone in landscape mode.  In each case, the links should be active and easily manageable.

}
\end{ForPublic}
	\vfill

% Bottom of the page
	{\large \today\par}
\end{titlepage}

\tableofcontents
\newpage
\begin{ForReviewer}
\listoftables
\vfill
\end{ForReviewer}
%\newpage
\listoffigures
\vfill
%\newpage
%\listofstorys
%\newpage
\listofexamples
\vfill
%\newpage
\listofadventures
\vfill
%\newpage
\listofreallifes
\vfill
\newpage

\listoftodos

\newpage

\chapter*{Preface}\new{v1.8}{Modified this for the public distribution}

The purpose of creating this book is to make better use of the technology that electronic texts allow for without losing the functionality of a print book.  While this text should be comparable to any other print text, when this is provided in the online format it will provide links back and forth between early and later topics.  Linking from later material to earlier material will allow students to refresh their memory of what was previously discussed.  Linking from earlier material to later material will inspire students to look ahead to how that topic will be used in more interesting scenarios.

Having these links will allow for some other interesting features that can be placed in the back of the book and accessed through links.  Examples of this might be:
\begin{enumerate}
\item ``Dig Deeper'' where some of the more tedious and some of the more interesting aspects can be investigated. For example in \autoref{c:motion} on the equations of motion, one might see how these equations are direct applications of calculus for those students who happen to have taken that course (which is common for biology and pre-medical students).
\item ``Every Equation Tells a Story'' which discusses how the description-in-English and the description-with-math interrelate to build intuition in both directions.
\item ``Examples'', with the difference from a traditional textbook being that students can interact with the example as: ``If you have this question, then go here. If you have that question, then go there.''
\item ``In the `Real World''' where students see how the concept lives in the messy real world and why physicists simplify or ignore complicating aspects.
\item ``Connections'', which might take one of three forms:
\begin{enumerate}
\item ``Where have I seen this before?'' (linking back to earlier material)
\item ``When will I ever use this?'' (linking ahead to later material)
\item ``Why is this interesting?'' (linking to popular or complex topics)
\end{enumerate}
\end{enumerate}
The goal of the book is to encourage curiosity in the reader. Since there is an expectation that students will explore the material on their own, advanced topics will explicitly note where the reader can look for supporting material and basic topics will be motivated with links to more advanced topics.  To help maintain the interest of the reader, recurring characters will be featured in the examples.  These characters will live a storyline\dothis{storyboard the characters and how they develop}{} and interact with each other.  It is possible to read the examples as a separate storyline for the N\urgent{Decide how many characters}{} interacting characters.

I am choosing the approach described above based on the assumption that students will prefer to develop their knowledge by building a world-view that connects to their current understanding, their interests, and their world-view. Providing the cross-referencing links without distracting students with all of the information at once will enable them to explore the information. Writing the text in a narrative style that helps students see the explanations for the world they live in will encourage them to explore ``what happens when I do this'' in their real life. Fostering this spirit of exploration will enable the instructors to bring their own active-learning techniques into the classroom.

This textbook is in several Parts\urgcap{book layout}{Here we should add information about Adventures, Examples, Equation-Stories, and IRLs.}:  \textbf{Part I} is for the preliminaries, including descriptions of science in general, physics in particular, and the use of math.  \textbf{Part II} is intended to introduce three fundamental and powerful concepts.  These concepts are motion, force, and energy.  I have found that if a student can understand these ideas sufficiently well, then they can quickly pick up any other idea that we introduce, even if the idea seems initially unfamiliar.  \textbf{Part III} develops the ideas in Part II by introducing momentum, circular motion, rotational motion, torque, and the Newtonian theory of gravitation.  \textbf{Parts IV} and \textbf{V} are oscillations and thermodynamics.  With the traditional organization of the two-semester introductory physics, these parts can be covered in either order and can be chosen to be put in either semester. \textbf{Part VI} covers electricity, magnetism, light, and optics.  This is traditionally the meat of the second semester. \textbf{Part VII} touches on the topics that are usually referred to as ``modern physics''.  The goal with including these chapters is to provide some inspiration for what some students see as the tedium of the standard material.  These chapters will be linked to throughout the book as examples of how the traditional material supports the material that may be in the news and is more talked about in popular science.  The last final part, \textbf{Part VIII}, holds the answers to the interactive examples mentioned above, the bulk of the adventures the reader can investigate in order to test their understanding of the material, and the story lines of each of the characters in the text.

\textbf{A note about viewing the PDF online:}  If you are viewing this as a PDF set to view ``single page,'' then the links will take you to the top of the relevant page, rather than to the specific topic.  If, on the other hand, you are viewing this in ``continuous view'' then you should go directly to the location of interest. If you are viewing this in ``two-page'' mode (whether continuous or not), it might not be immediately obvious to which page (left or right) you have jumped.  Most of the PDF viewers I have encountered allow you to follow links and to return to your previous location.  On most PCs, the way to return to your previous location is by holding the [ctrl] key and pressing the $[\leftarrow]$ button.  There are a few PDF viewers that do not allow you to ``go back'' to the location you linked from.  Whether or not you have that capability, I have placed ``return links'' in the margins so that you can get back to the place from which you linked.


\part{Prerequisites}

\chapter{The Story of Science}

Once upon a time\done{start the book}{} somebody saw the world around them and thought something equivalent to ``well, that's an interesting pattern\ldots'' and predictions were born.  Every human and many animals build their own world of expectations such as: objects will fall down, food will arrive at mealtime, or certain people will smile at me.  Scientists study the patterns in the world around us and do so in a fairly specific way.  Novelists, sociologists, historians, and cartoonists also look at the world around us in a very particular way.  The story of humanity is a story about observing the world around us.

Scientists, in general, observe patterns through careful, detailed measurements \ldots\dothis{Add description of science.}{}

Physicists, in particular, consider the patterns in the physical world around us.\dothis{Add description of physics}{}

\hypertarget{d:physicspatterns}{Some patterns} that you might experience help us take very different experiences and group them together.  For example\inlife, there are ways in which \hyperlink{d:freefall}{dropping your keys} and \hyperlink{d:ballistic}{throwing a dog toy} are very similar.  They both fall, even thought the fall along rather different paths.  There are also patterns that you experience that might look very similar but can be treated very differently.  For example\inlife, \hyperref[irl:nonparabolic]{the path of baseball pitch} is very different for a fast ball compared to a slider, a curve ball, or a knuckleball.

\begin{figure}[h]
\hrule\hrule
  \missingfigure{Photograph a park with tennis courts and basketball hoops in the background and falling car keys and a dog in the foreground. I think we could do that at Boone County park.}
\begin{ForPublic}
\centering
\fbox{\begin{minipage}{4in}
\vspace{1in}
This will be a photograph of a park with tennis courts and basketball hoops in the background and falling car keys and a dog in the foreground.
\vspace{1in}
\end{minipage}}
\end{ForPublic}
  \caption{\label{Fig:BoonePark} Life is full of examples of physics all around us. }
\hrule\hrule
\end{figure}

%\todo[due=2017-4-1]{this one has a due date}

\section{Careful, Detailed Observation}

[Discussion of ``\hypertarget{d:casual}{casual observer}\mautoreturn{ss:NI}'' as intuition versus ``scientific observing'' and mathematical modelling]\dothis{Consider the ``casual to the obvious observer'' joke}{}

\noindent
[Discussion of common student comment: ``in physics class it is this way, but in \textit{real life} it is that way.'']

\section{Theory versus Law}\label{s:law}\mautoreturn{s:Newton}


\chapter{Seeing Physics}\label{c:physics}\new{v2.1}{Filled in the details a little. This chapter should mirror \protect{\autoref{c:revisted}}.}

\section{The Flame Challenge and Other Brief Descriptions}\label{s:flame}

What you will find in this book is a series of chapters that, on the surface, feel like a list of isolated topics.  Each chapter will have examples that focus your attention on examples of that specific concept.  However, the really interesting aspect of physics is that these descriptions of the world around us come together in different ways to explain complex systems that might feel unrelated.  For example, the thermodynamics of making your refrigerator work on Earth comes from the same theories of thermodynamics that help us understand the heat flow of the sun.  Furthermore, in order to understand the sun, we also need to understand the gravitational interaction, which also describes how baseballs fly through the air.

This chapter will introduce a set of quick-overview explanations of phenomena to indicate how different ideas tie together in some complex systems.  The point  is specifically to over-simplify complex ideas in order to ``get the idea''.  You will also be pointed to the various chapters that go into the details of the relevant physics where you can learn more.  Then, at the end of the book in~\autoref{c:revisted}, we will revisit each of these ideas and go into the description in more depth assuming you have understood each of the relevant chapters, with reference back to the sections that provide the basis of our understanding.

\textbf{Caution}: Since this particular chapter is intended to be background introduction, rather than a place to study details, none of the links to other sections here will have return links in the rest of the text.  So, if you intend to use this as a jumping off point, you might want to create a bookmark here so that you can return after you read the details in other sections.

\subsection{The Flame Challenge}\label{ss:flame}\new{v2.3}{Added the questions.  These might be better in their respective sections.}
\href{http://www.aldakavlilearningcenter.org/practice/flame-challenge}{The Flame Challenge}

Useful?  \href{https://newsstand.google.com/articles/CAIiEBF_HbPTdq-9q-hjA0W51WYqFggEKg4IACoGCAow9vBNMK3UCDDq0Rc}{How Alan Alda Makes Science Understandable}

\href{http://www.aldakavlilearningcenter.org/practice/flame-challenge/what-is-a-flame}{2012: What is a flame?} \\
\href{http://www.aldakavlilearningcenter.org/flame-challenge/past-challenges/what-time}{2013: What is time?} \\
\href{http://www.aldakavlilearningcenter.org/practice/flame-challenge/past-challenges/what-is-color}{2014: What is color?} \\
\href{http://www.aldakavlilearningcenter.org/practice/flame-challenge/past-challenges/what-is-sleep}{2015: What is sleep?} \\
\href{http://www.aldakavlilearningcenter.org/practice/flame-challenge/past-challenges/what-is-sound}{2016: What is sound?} \\
\href{http://www.aldakavlilearningcenter.org/practice/flame-challenge/past-challenges/energy}{2017: What is energy?}


\subsection{The Forming of Matter in the Universe}\label{ss:matter}\new{v2.1}{Started this section to give a sense\ldots}

In the early ages of the universe, which is an entirely different story that could be told, there were a ridiculously large number of particles created and drifting around.  There were a variety of types (\autoref{ss:StandardModel}), some being positively charged (\autoref{s:Echarge}), some negatively charged, and some were neutral; but the larger ones tended to gradually decay (\autoref{ss:particledecay}) into smaller ones.  The smaller of the positively-charged baryons (\autoref{s:particle}), which we call protons, and the smallest of the negatively-charged leptons (\autoref{s:particle}), which we call electrons, also tended to stick together because of their electrical charges (\autoref{s:Echarge}), forming hydrogen atoms.  You may note that as this happens, sometimes the more ambitious of the particles form larger clumps of two protons and two neutrons, making helium atoms that are held together by the strong nuclear force ([need ref])\dothis{Stopped mid-stream.  This is a good place to jump back in when I am stuck someplace else.}{}

\subsection{Things in the Sky}\new{v2.3}{Rearranged sections}

\subsubsection{The Sun}\label{sss:sun}\new{v2.1}{This point of this will be to connect gravity-thermo-nuclear and to do it in 1-2 paragraphs (a la the flame challenge).}
The bright, shiny sun, which keeps us all alive, is a nice example of a rather complex system that allows us to verify our various theories of the world around us.  As an over-simplification of the process, we can consider the existence of a star in three phases: the ignition (some have said ``birth'') of a star, the shining (some would say ``life'') of the star, and the snuffing (``death''?) of the star.

\subsection{Things on the ground}

\subsubsection{Hot Tea and Iced Tea}\label{sss:tea}\mautoreturn{s:surface.tension}

On 28 April, 2017,\new{v2.3}{New source of info}
\href{http://www.cbc.ca/podcasting}{CBC Broadcasting} published a
\href{http://www.cbc.ca/podcasting/includes/quirks.xml}{\textit{Quirks and Quarks}} episode discussing why
\href{https://podcast-a.akamaihd.net/mp3/podcasts/quirks_20170429_19254.mp3}{hot water sounds different from cold water when they are poured}.
Spoiler Alert: It is due to surface tension, size of droplets when heated, and auditory perception.


%\subsection{Kitchen Appliances}
\subsubsection{Oven}
\subsubsection{Refrigerator}
\subsubsection{Microwave}
\subsubsection{Television}

\subsection{Automobile}
\subsubsection{Coolant and Antifreeze}
\subsubsection{Tires}
\subsubsection{Torque}

\subsection{Cool Ideas}
\subsubsection{Black Holes}\label{sss:blackhole1}
\subsubsection{Quantum Mechanics}
\subsubsection{Relativity}
\subsubsection{String Theory}
\subsubsection{Fusion}

On 28 April, 2017,\new{v2.3}{New source of info}
\href{http://www.cbc.ca/podcasting}{CBC Broadcasting} published a
\href{http://www.cbc.ca/podcasting/includes/quirks.xml}{\textit{Quirks and Quarks}} episode discussing a
\href{https://podcast-a.akamaihd.net/mp3/podcasts/quirks_20170429_51936.mp3}{documentary compares the massive scale ITER approach to fusion with the much smaller approach by a Canadian company}.
\textbf{I don't think I want to use this, but it might be helpful to listen again to the nice summary of fusion.}  Maybe get some resources on ``state of the art''.


\section{Effective Theory}\label{s:effective1}\dothis{Should this be here or in \protect{\autoref{s:effective2}}?}{}

All of our explanations are approximations.  This section will describe some physics in the world around us in one or two paragraphs with links to the sections in the book that provide the detailed understanding of that piece which connects to the mathematics and the underlying foundation.  Each topic will also link to a more detailed discussion at the end of the book with a longer conversation that gets into more nitty-gritty details which assume you have learned the details from the book.  In short, this section looks forward to what is possible to understand and that chapter looks back at how you do understand.  Each of these topics will also be accompanied by a five-minute podcast describing the topic.

The term ``effective theory'' is used in physics to describe a wide-reaching phenomenon which can be approximated by a simpler theory in a smaller circumstance.  So, for example, Einstein's theory of general relativity as a complex description of the gravitational interaction.  It would be unwieldy and impractical to use that to describe our day-to-day interactions with the gravitational interaction.  On the other hand, Newton's theory of the gravitational interaction is a special case of Einstein's general theory of relativity that works perfectly well so long as you behave yourself and do not try to travel at a significant fraction of the speed of light.  We can say that Newton's theory of gravity is an effective theory for Einstein's theory of gravity that accounts for acceleration at low speeds.  Likewise, Einstein's special theory of relativity is an effective theory of the general theory of relativity.  The special theory is relevant when you do not allow for acceleration, but do allow for faster speeds.  Once you reach beyond the limitations of the effective theory, the description ``breaks down''.


\chapter{Why so much math?}

\section{Every equation tells a story}\label{s:story}

Mathematics is its own language.  It is the language of patterns.  Humans are very adept at tracking patterns.  Physics is the study of patterns in the physical world.  It turns out that the language of physics provides a natural and concise mechanism for expressing patterns in a uniquely precise manner.  Equations allow us to connect physical reality to very specific predictions.  For example, the equation for thermal conductivity, \autoref{eq:thermalconductivity} in \autoref{ss:thermalconductivity}, allows \studentA\index{\studentA} to predict the time it takes for \hisA\ oven to warm up to a specific temperature because
$\displaystyle \frac{Q}{\Delta t} = \kappa A \, \frac{\Delta T}{\Delta x}$\todo{I would love for this to be a mouse-over in the equation}{} says that {the rate at which energy flows} {depends on} {how well air allows energy to flow,} {the size of the oven,} and {the amount the temperature needs to change} {across the height of the oven} as follows:\dothis{Consider ``chunking'' the ``story'' with colors to indicate the pieces.}{}
\[\begin{array}{ccccc}
\deq \frac{Q}{\Delta t} & = & \deq \kappa & \deq A & \deq \frac{\Delta T}{\Delta x} \\
\EqStoryOver{45pt}{the rate at which energy flows}{}
& \EqStoryOver{40pt}{depends on}{}
& \EqStoryOver{50pt}{how well air allows energy to flow,}{}
& \EqStoryOver{50pt}{the size of the oven,}{}
& \EqStoryFrac{75pt}{and the amount the temperature needs to change}{}
                    {across the height of the oven}{}
\end{array}\]
We will see this particular story in more detail with \autoref{ex:baking} (pg.~\pageref{ex:baking}) when \studentA\index{\studentA} prepares to bake some bread for \hisA\ friends.  Some of the more important equations are listed below.  By jumping between these narratives, you can get a better sense of how to think about physics in general.

\begin{ForPublic}
\begin{table}[h]
\centering
\begin{tabular}{ccc}
\hyperref[st:F=ma]{$\deq \vec F_\mathrm{net} = m \vec a$} & .......... & \pageref{st:F=ma}
\end{tabular}
\end{table}
\end{ForPublic}
\begin{ForMe}
\dothis{Decide if should use the ``public version'' or the ``me version'' (which uses $\backslash$listofstorys).}{}
\listofstorys
\vfill
\end{ForMe}


\section{The Metric System}\label{s:SI-MKS}\mautoreturn{ss:weightmass}\new{v2.2}{Added detail}

The International System of Units (SI)
% https://en.wikipedia.org/wiki/International_System_of_Units
was adopted in 1960 at the
\href{http://www.bipm.org/jsp/en/ListCGPMResolution.jsp?CGPM=11}{eleventh meeting}
of the
\href{http://www.bipm.org/en/about-us/}{International Bureau of Weights and Measures (BIPM)}.\footnote{In French this organization is the Bureau International des poids et mesures, so the acronym is BIPM.}
%  https://en.wikipedia.org/wiki/General_Conference_on_Weights_and_Measures

In 1901 at the
\href{http://www.bipm.org/jsp/en/ListCGPMResolution.jsp?CGPM=3}{third meeting}
of the BIPM, it
\href{http://www.bipm.org/en/CGPM/db/3/2/}{was declared} that
\begin{enumerate}
\item The kilogram is the unit of mass; it is equal to the mass of the international prototype of the kilogram;
\item The word ``weight'' denotes a quantity of the same nature as a ``force'': the weight of a body is the product of its mass and the acceleration due to gravity; in particular, the standard weight of a body is the product of its mass and the standard acceleration due to gravity;
\item The value adopted in the International Service of Weights and Measures for the standard acceleration due to gravity is $980.665 \unitfrac{cm}{s^2}$, value already stated in the laws of some countries.
\end{enumerate}
The 11th meeting (1960) redefined the meter in terms of wavelengths of light.
The 13th meeting (1967) redefined the second in terms of the frequency of radiation from $^{133}$Cs.
The 17th meeting (1983) redefined the meter in terms of the speed of light and seconds.
The 24th (2011) and 25th (2014) meeting discussed redefining the kilogram in terms of the Planck constant, with an expectation that it will be redefined at the 26th meeting (Nov, 2018).  See note in \autoref{ss:units}.

Note \href{https://www.nist.gov/sites/default/files/documents/2016/11/10/appb-17-hb44-final.pdf}{Handbook 44, page B-6} talks about SI.\new{v2.2}{References to NIST}

Note
\href{https://www.nist.gov/pml/weights-and-measures/publications/nist-handbooks/handbook-44}{Handbook 44 webpage}
still links to
\href{https://www.nist.gov/sites/default/files/documents/2016/11/10/appc-17-hb44-final.pdf}{the 2016 pdf}
instead of the
\href{https://www.nist.gov/sites/default/files/documents/2017/04/28/AppC-12-hb44-final.pdf}{the 2017 pdf}
even though it says it was updated in 2017.

There is also
\href{https://www.nist.gov/pml/special-publication-811-extended-contents}{a special publication} from NIST that summarizes the use and conversation between units in the SI.

\subsection{Units Quantify Dimensions}

\subsection{Conversion from English Units}\label{ss:convertunits}\mmultireturn{\mmr{\autoref{s:sigfig}}, \mmr{\autoref{ex:slowcar}}}

Note internet search comments in \autoref{ss:weightmass} regarding the ``conversion'' of kilograms-to-pounds, with special attention to \hyperref[s:sigfig]{significant digits}.\index{Significant Digits}\dothis{rephrase this.  I moved that discussion to \protect{\autoref{s:sigfig}}.}


\subsection{Fundamental Units versus Derived Units}\label{ss:units}\mmultireturn{\mmr{\autoref{sss:unit-N}}, \mmr{\autoref{ss:weightmass}}}

Note conversation in \autoref{sss:unit-N} about the Newton.

See\new{v2.2}{Possible redefinition of the kilogram.}\mautoreturn{s:SI-MKS}
\href{https://scitechdaily.com/researchers-to-redefine-the-kilogram-in-terms-of-plancks-constant/}{the 2012 article from SciTechDaily.com}
and
\href{https://www.nist.gov/physical-measurement-laboratory/plancks-constant}{the NIST explanation} about redefining the kilogram in terms of the Planck constant at the 26th meeting (Nov, 2018) of BIPM.


\section{A graph is worth a thousand pictures}

\subsection{Coordinate Systems}

\noindent
\begin{itemize}
\item Discussion of the choice of origin (possible reference to zero-value of the potential energy)
\item Discussion of the choice of the positive-direction (possible reference to falling objects and using positive-up versus positive-down)
\item \hypertarget{d:referenceframe}{Definition of a reference frame}\mmultireturn{\mmr{\autoref{ss:addvel}}, \mmr{\autoref{ss:noninertial}}, \mmr{\hyperlink{d:NewtonInertial}{Newton's Laws}}}
\begin{itemize}
    \item (different locations) The view from the roof versus from the ground
    \item (different speeds) The view from the sidewalk versus from a moving car  (See also \autoref{ss:noninertial}.)
    \item (different types of motion) The view from a park bench versus from a merry-go-round.  (See also \autoref{s:noninertial}.)
\end{itemize}
\end{itemize}

\subsection{The Vocabulary of Graphs}

[Quick review of parameters and variables of $y=mx+b$ and $y=ax^2+bx+c$.]

\begin{center}
\setlength{\unitlength}{1cm}
\begin{Picture}(-2.5,-5.5)(3.5,3.5)
\cartesiangrid(-2,-5)(3,3)
\pictcolor{blue}
%\qbezier(-1,-7.405)(0.306,9.322)(1.612,-7.405)
%\qbezier(-1,-10.405)(0.612,15.075)(2.223,-10.405)
\qbezier(-0.5,-3.726)(0.612,8.396)(1.723,-3.726)
\end{Picture}
\end{center}

\section{Trigonometry and Vectors}

\subsection{Trigonometry}
\subsection{Vectors}\label{ss:vectors}\mmultireturn{\mmr{\hyperlink{d:pushvector}{the direction of force}}, \mmr{\autoref{sss:netforce}}}
\begin{ForMe}
\begin{figure}
\hrule\hrule
  \centering
  \caption{\LaTeX\ lines and vectors.  This will be deleted, but is here for reference.}\label{f:lines}
\begin{picture}(300,500)(0,0)
\put(0,0){\line(1,0){300}} \put(301,-2){(1,0) $0^\circ$}
\put(0,0){\line(6,1){300}} \put(301,48){(6,1) $9.46^\circ$}
\put(0,0){\line(5,1){300}} \put(301,58){(5,1) $11.31^\circ$}
\put(0,0){\line(4,1){300}} \put(301,73){(4,1) $14.04^\circ$}
\put(0,0){\line(3,1){300}} \put(301,98){(3,1) $18.43^\circ$}
\put(0,0){\line(2,1){300}} \put(301,148){(2,1) $26.57^\circ$}
\put(0,0){\line(1,1){250}} \put(251,248){(1,1) $45^\circ$}
%\put(0,0){\line(6,2){300}} \put(301,98){(6,2) $18.43^\circ$}
\put(0,0){\line(5,2){300}} \put(301,118){(5,2) $21.8^\circ$}
%\put(0,0){\line(4,2){300}} \put(301,148){(4,2) $26.57^\circ$}
\put(0,0){\line(3,2){300}} \put(301,198){(3,2) $33.69^\circ$}
%\put(0,0){\line(2,2){250}} \put(251,248){(2,2) $45^\circ$}
\put(0,0){\line(1,2){171}} \put(172,340){(1,2) $63.43^\circ$}
%\put(0,0){\line(6,3){300}} \put(301,148){(6,3) $26.57^\circ$}
\put(0,0){\line(5,3){300}} \put(301,178){(5,3) $30.96^\circ$}
\put(0,0){\line(4,3){300}} \put(301,223){(4,3) $36.87^\circ$}
%\put(0,0){\line(3,3){250}} \put(251,248){(3,3) $45^\circ$}
\put(0,0){\line(2,3){204}} \put(205,304){(2,3) $56.31^\circ$}
\put(0,0){\line(1,3){128}} \put(129,382){(1,3) $71.57^\circ$}
%\put(0,0){\line(6,4){300}} \put(301,198){(6,4) $33.69^\circ$}
\put(0,0){\line(5,4){300}} \put(301,238){(5,4) $38.66^\circ$}
%\put(0,0){\line(4,4){250}} \put(251,248){(4,4) $45^\circ$}
\put(0,0){\line(3,4){218}} \put(219,288.666666666667){(3,4) $53.13^\circ$}
%\put(0,0){\line(2,4){171}} \put(172,340){(2,4) $63.43^\circ$}
\put(0,0){\line(1,4){101}} \put(102,402){(1,4) $75.96^\circ$}
\put(0,0){\line(6,5){300}} \put(301,248){(6,5) $39.81^\circ$}
%\put(0,0){\line(5,5){250}} \put(251,248){(5,5) $45^\circ$}
\put(0,0){\line(4,5){225}} \put(226,279.25){(4,5) $51.34^\circ$}
\put(0,0){\line(3,5){192}} \put(193,318){(3,5) $59.04^\circ$}
\put(0,0){\line(2,5){146}} \put(147,363){(2,5) $68.2^\circ$}
\put(0,0){\line(1,5){84}} \put(85,418){(1,5) $78.69^\circ$}
%\put(0,0){\line(6,6){250}} \put(251,248){(6,6) $45^\circ$}
\put(0,0){\line(5,6){230}} \put(231,264){(5,6) $50.19^\circ$}
%\put(0,0){\line(4,6){204}} \put(205,304){(4,6) $56.31^\circ$}
%\put(0,0){\line(3,6){171}} \put(172,340){(3,6) $63.43^\circ$}
%\put(0,0){\line(2,6){128}} \put(129,382){(2,6) $71.57^\circ$}
\put(0,0){\line(1,6){71}} \put(72,434){(1,6) $80.54^\circ$}
%
\put(0,0){\line(0,1){450}} \put(-5,451){(0,1) $90^\circ$}
%
%
%
\put(0,0){\vector(1,0){225}}
\put(0,0){\vector(6,1){221.9}}
\put(0,0){\vector(5,1){220.6}}
\put(0,0){\vector(4,1){218.3}}
\put(0,0){\vector(3,1){213.5}}
\put(0,0){\vector(5,2){208.9}}
\put(0,0){\vector(2,1){201.2}}
\put(0,0){\vector(5,3){192.9}}
\put(0,0){\vector(3,2){187.2}}
\put(0,0){\vector(4,3){180}}
\put(0,0){\vector(5,4){175.7}}
\put(0,0){\vector(6,5){172.8}}
\put(0,0){\vector(1,1){159.1}}
\put(0,0){\vector(5,6){144}}
\put(0,0){\vector(4,5){140.6}}
\put(0,0){\vector(3,4){135}}
\put(0,0){\vector(2,3){124.8}}
\put(0,0){\vector(3,5){115.8}}
\put(0,0){\vector(1,2){100.6}}
\put(0,0){\vector(2,5){83.6}}
\put(0,0){\vector(1,3){71.2}}
\put(0,0){\vector(1,4){54.6}}
\put(0,0){\vector(1,5){44.1}}
\put(0,0){\vector(1,6){37}}
\put(0,0){\vector(0,1){225}}
%
\end{picture}
%\hrule\hrule
\end{figure}
\begin{figure}
%\hrule\hrule
  \centering
  \caption{\LaTeX\ lines and vectors.  This will be deleted, but is here for reference.}\label{f:lines2}
\begin{tikzpicture}
\draw [<->, rounded corners, thick, gray] (10,0) -- (0,0) --(0,10);
\draw [lightgray] (0,6) arc [radius=6, start angle=90, end angle=0];  % start at the (+y) of the circle, end at the (+x) of the circle
\draw [lightgray] (9,1) arc [radius=17, start angle=-10, end angle=52];
\draw [->] (0,0) -- (5.92,0.99); \draw (0,0) -- (9.08,1.51);  \node [right] at (9.08,1.51) {(6,1) $9.46^\circ$};
\draw [->] (0,0) -- (5.88,1.18); \draw (0,0) -- (9.12,1.82);  \node [right] at (9.12,1.82) {(5,1) $11.31^\circ$};
\draw [->] (0,0) -- (5.82,1.46); \draw (0,0) -- (9.18,2.29);  \node [right] at (9.18,2.29) {(4,1) $14.04^\circ$};
\draw [->] (0,0) -- (5.69,1.9); \draw (0,0) -- (9.24,3.08);  \node [right] at (9.24,3.08) {(3,1) $18.43^\circ$};
% \draw [->] (0,0) -- (5.69,1.9); \draw (0,0) -- (9.24,3.08);  \node [right] at (9.24,3.08) {(6,2) $18.43^\circ$};
\draw [->] (0,0) -- (5.57,2.23); \draw (0,0) -- (9.26,3.7);  \node [right] at (9.26,3.7) {(5,2) $21.8^\circ$};
\draw [->] (0,0) -- (5.37,2.68); \draw (0,0) -- (9.25,4.62);  \node [right] at (9.25,4.62) {(2,1) $26.57^\circ$};
% \draw [->] (0,0) -- (5.37,2.68); \draw (0,0) -- (9.25,4.62);  \node [right] at (9.25,4.62) {(4,2) $26.57^\circ$};
% \draw [->] (0,0) -- (5.37,2.68); \draw (0,0) -- (9.25,4.62);  \node [right] at (9.25,4.62) {(6,3) $26.57^\circ$};
\draw [->] (0,0) -- (5.14,3.09); \draw (0,0) -- (9.19,5.51);  \node [right] at (9.19,5.51) {(5,3) $30.96^\circ$};
\draw [->] (0,0) -- (4.99,3.33); \draw (0,0) -- (9.12,6.08);  \node [right] at (9.12,6.08) {(3,2) $33.69^\circ$};
% \draw [->] (0,0) -- (4.99,3.33); \draw (0,0) -- (9.12,6.08);  \node [right] at (9.12,6.08) {(6,4) $33.69^\circ$};
\draw [->] (0,0) -- (4.8,3.6); \draw (0,0) -- (9.02,6.77);  \node [right] at (9.02,6.77) {(4,3) $36.87^\circ$};
\draw [->] (0,0) -- (4.69,3.75); \draw (0,0) -- (8.95,7.16);  \node [right] at (8.95,7.16) {(5,4) $38.66^\circ$};
\draw [->] (0,0) -- (4.61,3.84); \draw (0,0) -- (8.9,7.42);  \node [right] at (8.9,7.42) {(6,5) $39.81^\circ$};
\draw [->] (0,0) -- (4.24,4.24); \draw (0,0) -- (8.61,8.61);  \node [right] at (8.61,8.61) {(1,1) $45^\circ$};
% \draw [->] (0,0) -- (4.24,4.24); \draw (0,0) -- (8.61,8.61);  \node [right] at (8.61,8.61) {(2,2) $45^\circ$};
% \draw [->] (0,0) -- (4.24,4.24); \draw (0,0) -- (8.61,8.61);  \node [right] at (8.61,8.61) {(3,3) $45^\circ$};
% \draw [->] (0,0) -- (4.24,4.24); \draw (0,0) -- (8.61,8.61);  \node [right] at (8.61,8.61) {(4,4) $45^\circ$};
% \draw [->] (0,0) -- (4.24,4.24); \draw (0,0) -- (8.61,8.61);  \node [right] at (8.61,8.61) {(5,5) $45^\circ$};
% \draw [->] (0,0) -- (4.24,4.24); \draw (0,0) -- (8.61,8.61);  \node [right] at (8.61,8.61) {(6,6) $45^\circ$};
\draw [->] (0,0) -- (3.84,4.61); \draw (0,0) -- (8.2,9.85);  \node [right] at (8.2,9.85) {(5,6) $50.19^\circ$};
\draw [->] (0,0) -- (3.75,4.69); \draw (0,0) -- (8.1,10.12);  \node [right] at (8.1,10.12) {(4,5) $51.34^\circ$};
\draw [->] (0,0) -- (3.6,4.8); \draw (0,0) -- (7.92,10.56);  \node [right] at (7.92,10.56) {(3,4) $53.13^\circ$};
\draw [->] (0,0) -- (3.33,4.99); \draw (0,0) -- (7.57,11.35);  \node [right] at (7.57,11.35) {(2,3) $56.31^\circ$};
% \draw [->] (0,0) -- (3.33,4.99); \draw (0,0) -- (7.57,11.35);  \node [right] at (7.57,11.35) {(4,6) $56.31^\circ$};
\draw [->] (0,0) -- (3.09,5.14); \draw (0,0) -- (7.22,12.03);  \node [right] at (7.22,12.03) {(3,5) $59.04^\circ$};
\draw [->] (0,0) -- (2.68,5.37); \draw (0,0) -- (6.57,13.13);  \node [right] at (6.57,13.13) {(1,2) $63.43^\circ$};
% \draw [->] (0,0) -- (2.68,5.37); \draw (0,0) -- (6.57,13.13);  \node [right] at (6.57,13.13) {(2,4) $63.43^\circ$};
% \draw [->] (0,0) -- (2.68,5.37); \draw (0,0) -- (6.57,13.13);  \node [right] at (6.57,13.13) {(3,6) $63.43^\circ$};
\draw [->] (0,0) -- (2.23,5.57); \draw (0,0) -- (5.73,14.32);  \node [right] at (5.73,14.32) {(2,5) $68.2^\circ$};
\draw [->] (0,0) -- (1.9,5.69); \draw (0,0) -- (5.05,15.15);  \node [right] at (5.05,15.15) {(1,3) $71.57^\circ$};
% \draw [->] (0,0) -- (1.9,5.69); \draw (0,0) -- (5.05,15.15);  \node [right] at (5.05,15.15) {(2,6) $71.57^\circ$};
\draw [->] (0,0) -- (1.46,5.82); \draw (0,0) -- (4.05,16.2);  \node [right] at (4.05,16.2) {(1,4) $75.96^\circ$};
\draw [->] (0,0) -- (1.18,5.88); \draw (0,0) -- (3.36,16.82);  \node [right] at (3.36,16.82) {(1,5) $78.69^\circ$};
\draw [->] (0,0) -- (0.99,5.92); \draw (0,0) -- (2.87,17.23);  \node [right] at (2.87,17.23) {(1,6) $80.54^\circ$};
\end{tikzpicture}
%\hrule\hrule
\end{figure}
\end{ForMe}
\subsubsection{Scalar Quantities versus Vector Quantities} \label{sss:scalarvector}\mlinkreturn[the direction of forces]{d:pushvector}
\subsubsection{Vector Equations}\label{sss:vectorequations}\mmultireturn{\mmr{\hyperlink{d:2Dmotion}{the ballistic freefall}}, \mmr{\hyperlink{d:f=ma}{$F=ma$}}}
\ldots If $\vec A = 3 \vec B$, then this is true for each component.
\begin{eqnarray}
A_x & = & 3 B_x \\
A_y & = & 3 B_y \\
A_z & = & 3 B_z
\end{eqnarray}

This can also be written in two different ways:
\[ \vect{A_x}{+A_y}{+A_z} \ = \ 3 \left( \vect{B_x}{+B_y}{+B_z} \right) \ = \ \vect{(3B_x)}{+(3B_y)}{+(3B_z)} \]

This will be useful when\foreshadow{} we are discussing \hyperref[ss:ballistic]{ballistics} (2-dimensional motion), \hyperref[ss:NII]{Newton's second law} (combining multiple forces pushing on an object), \hyperref[s:2Dcollisions]{2-dimensional collisions}, and the calculation of \hyperref[ss:Efield]{electrical fields}.

\subsubsection{Multiplication, but Not Division}\label{sss:vectorproducts}

[define dot product]

\noindent
[define cross-product]

\noindent
Can do magnitude-equations like $F=ma$ or $m=F/a$.  But for vector equations, while you can do $\vec F=m\vec a$, you cannot do something like
\hypertarget{d:dividevectors}{$\deq m = \frac{3\ihat+4\jhat}{2\ihat-5\jhat}$}\mreturn{se:netF-m}; but, in that case, you can use the magnitudes as follows
\[ m = \frac{\sqrt{(3)^2+(4)^2}}{\sqrt{(2)^2+(-5)^2}} = \frac{\sqrt{25}}{\sqrt{29}} = \sqrt{\frac{25}{29}} = 0.\sig{9}{28}{} \]

\chapter{Estimating and Uncertainty}

\section{Precision and Accuracy}\label{s:precision}\index{Precision}\new{v2.2}{Added section, added some detail}

In this section, we will consider the benefits of being precise both in measurements and in our language.  Sometimes people confuse the words precise and accurate, but they mean different things.  It may help to remember that the opposite of precise is vague.  Being precise makes it easier to determine if a statement is accurate.  If we already know the answer, then we can know if a result is accurate.  However, the exciting aspect of science is to study that which we do not already know.  In this case, gauging accuracy can be tricky.  If we are do not already know an answer, then we can try to be consistent within our accepted precision.

Since physics has its roots in the natural philosophy of the ancient Greeks and developed mathematically with Galileo and Newton, it has been around long enough for the technical language to both evolve (Newton used the word ``action'' for what we refer to as ``force'') and to be absorbed into everyday (colloquial) language. Words like force and energy have taken on broader meanings in English.  In this text, we will try to be precise with the language.  Hopefully we can avoid using the dismissive phrase, ``Oh, you \textit{know} what I \textit{mean}.''

One example\mautoreturn{ss:weightmass} of not being careful with the language comes when people use the term ``massive'' to mean ``big.''  The word massive actually means ``has a large amount of material'' whereas big means ``takes up a large amount of space'' (which might be replaced by the word ``voluminous'' rather than ``massive''). These are related by \hyperref[s:density]{the density}\index{Density} but it is possible to be massive and not voluminous (see, for example, the discussion of \hyperref[s:blakhole2]{black holes}).  While it is \textit{technically} inaccurate to use massive to mean big, ``we'' know what ``we'' mean.

\section{Significant Figures}\label{s:sigfig}\mmultireturn{\mmr{\autoref{ss:convertunits}}, \mmr{\autoref{ss:weightmass}}}

Note the comments in \autoref{ss:weightmass} regarding an internet search on the ``\hyperref[ss:convertunits]{conversion}'' of kilograms-to-pounds.\index{Significant Digits}\dothis{Remove this sentence and make the next paragraph sensible.  It makes more sense here than in \protect{\autoref{ss:weightmass}}.}

A short Google\textsuperscript{tm} search by the author found that the conversation rate between pounds and kilograms was $1 \unit{kg} = 2.2046226218 \unit{lbs}$.  Several sites go on to list about 10 decimal places for all of the conversions.  First, you should recall our discussion about \hyperref[s:sigfig]{significant digits}\index{Significant Digits}\dothis{Refocus this paragraph as an \textit{example} about significant digits.}\new{v2.2}{moved this conversation here.}.  Second you should note that the unit of pounds is a measure of force (how much the Earth pulls on you)\index{Weight}, whereas the unit of kilogram is a measure of mass (how much ``stuff'' there is)\index{Mass}.  These are related in proportion to the strength of the gravitational field, which varies in the third digit (on the order of about 1\%\addlink{variation in $g$}) around the globe.  Some sites indicate that they are shortening their conversion factor to 3 digits for convenience, but this is not an issue of convenience, it is an issue of precision\index{Precision}.

\section{Scientific Notation}


\section{Effective Theories}\label{s:effective2}\mmultireturn{\mmr{\hyperref[ss:noninertial]{non-inertial reference frames}}, \mmr{air resistance \autoref{ss:airresistance}}, \mmr{air resistance \autoref{ss:ballisticairresistance}}, \mmr{\hyperref[ss:NI]{Newton's first law}}, \mmr{\hyperref[ss:NII]{Newton's second law}}, \mmr{\autoref{s:Fg}}, \mmr{\hyperlink{d:fundamental}{fundamental forces}}, \mmr{\autoref{s:FT}}}\new{v2.2}{Added section to indicate approximate truth}\dothis{Should this be here or in \protect{\autoref{s:effective1}}?}{}

Life is complicated.  One mechanism that scientists in general and physicists in particular use to simplify their descriptions of the world around us is to build an effective theory.  These are not intended to be true (accurate) to as many decimal places as can be calculated, but rather are intended to be good enough.  In this context, good enough is most likely to mean something like: true to a reasonable number of decimal places.

A colloquial example of this is when you wear a smile to give the impression of happiness even if you are not in the mood.  Most of your casual interactions will be the same as when you are in a good mood, but your friends who know you better will recognize the small discrepancies.

A technical example of this is that Newton's theory of gravity is very precise as long as none of the objects being described are travelling ``close'' to the speed of light.  How close counts as close depends on the level of precision the measurement needs to be.  If any of the objects are moving close to the speed of light, then we need Einstein's general theory of relativity.  One way to describe this is that Newton's theory is a special case of Einstein's Theory.  Another way is describe it is that Newton's theory is an effective theory for Einstein's theory, effective when the speed is low.  It is possible for us to measure the difference between Newton's theory and Einstein's theory, but it is often not worth the effort of using the more complex theory in the cases where the simpler one will do; it is effectively true (rather than actually true).

Another technical example is that Einstein's special theory of relativity is a special case of the general theory of relativity.  The aspect that makes it a special case is that the special theory only considers motion without acceleration.

One final case that should be mentioned up front is to notice that humans experience the Earth \textit{as if} it were stationary.\dothis{Decide if this should be filled out more or if it should reference the variety of places where the text fills out these kinds of details.}{}

\part{Introducing Motion, Force, and Energy}

The trio of topics in this part of the book are fundamental and powerful concepts\footnote{The idea of Fundamental and Powerful Concepts (F\&PC) is taken from Dr. Gerald Nosich, \textit{Learning to Think Things Through}, Prentice Hall, 2012.}.  These are fundamental in that most other topics in physics are built upon them.  They are powerful in that if they are well-understood, then one is empowered to use them to understand and develop an intuition for nearly any other topic that is experienced.  It has been my experience that with these topics, students can jump into a surprisingly wide variety of other, significantly more esoteric, topics and develop a reasonable grasp of the key concepts.  Furthermore, the development of understanding of these ideas introduce the language and thought processes of being a professional physicist such that it nicely bridges the language barrier that might otherwise exist due to the jargon of physics.

\chapter{One-Dimensional Motion}\label{c:motion}

\section{How Physicists Use the Words (Notation)}

\begin{itemize}
\item Position = where is it?  Also discuss location as a vector and giving directions as defining a coordinate system (locate a common origin and unit-vector, then give a series of magnitudes and directions).
\begin{itemize}
\item This chapter will distinguish location versus distance.
\item This chapter will distinguish distance traveled versus displacement.
\end{itemize}
\item Velocity = which way did it go?  Is its position changing?
\begin{itemize}
\item This chapter will distinguish speed and velocity.
\item Introduce the language of ``at rest''.
\end{itemize}
\item Acceleration = Is its velocity changing?
\begin{itemize}
\item This chapter will clarify acceleration, deceleration, and changing direction.
\item This chapter will distinguish distance traveled versus displacement.
\end{itemize}
\end{itemize}

\section{Connecting the Concepts: distance equals rate times time}

\subsection{Position}

Identifying the position requires identifying a common known position (which we could call ``the origin''), a distance from that known location (which we could call ``a magnitude''), and a direction from the origin in which to travel such a distance.  The common example\inlife{} that identifies the location as ``I am in my room'' references ``your room'' as the common, known origin.  If the author of this text were to tell you that he was in his room, then your next obvious question is: ``OK, but where is your room?''

Position can be seen to be a vector when you describe a meeting place or destination to a friend who has never been to that location:  ``Well, you know where the bookstore is, right?'' (establishes a common origin).  ``OK, so, if you face the sports gear shop\ldots'' (sets the coordinate axis and defines the position direction) ``\ldots turn left and walk a block'' (defines the magnitude and the direction).

\subsection{Speed versus Velocity}

When you are not moving, physicists will describe you as being \hypertarget{d:atrest}{``at rest''}\mlinkreturn[Newton's First Law]{d:atrestinmotion}.  When you drive to the store\inlife, your car ``starts from rest'' and then travels some distance in some time.  When you arrive at the store, your car ``ends at rest'' when you arrive at your destination.

When you are moving\ldots\dothis{Discussion of speed as $\txtfrac{\Delta x}{\Delta t}$}{}

To be moving, you must be moving in a particular direction.\dothis{Discussion of velocity as a vector}{}


\subsection{Adding Velocities}\label{ss:addvel}

Comment on inertial \hyperlink{d:referenceframe}{reference frames}.\index{Reference Frames!Inertial}

\section{Extending the Concepts: Changing How You Move}\label{s:acceleration}\mmultireturn{\mmr{\hyperlink{d:NewtonInertial}{non-inertial reference frames}}, \mmr{\hyperlink{d:f=ma}{$F=ma$}}}

\subsection{Moving versus Speeding Up}\label{ss:acceleration}

\begin{itemize}
\item Description of \hypertarget{d:motion}{``moving''} as \textit{moving at constant velocity}.
    \mmultireturn{\mmr{\hyperlink{d:objectinmotion}{objects in motion}}, \mmr{\hyperlink{d:atrestinmotion}{Newton's First Law}}}

\item Description of \hypertarget{d:acc}{\textit{accelerating}} as either ``accelerating'', ``decelerating'', or ``turning.''\mautoreturn{s:forcewords}

\end{itemize}

Discussion of \autoref{ex:slowcar}  (pg.~\pageref{ex:slowcar}) and \autoref{ex:coasting}  (pg.~\pageref{ex:coasting}).

\section{Connecting the English to the Math}\label{s:EOM}\mreturn{se:netF-a}

\hypertarget{d:EOM}{The equations of constant acceleration}\mautoreturn{ex:ceiling} can be summarized as\new{v2.3}{referenced.   Need the story of these equations}
\begin{eqnarray*}
x_f & = & x_i + v_i t + \frac{1}{2} a t^2 \\
v_f & = & v_i + a t \\
v_f^2 & = & v_i^2 + 2 a \, \Delta x
\end{eqnarray*}

\section{Examples}

\begin{example}[hbpt]
\fcolorbox{black}{yellow!10}{\begin{minipage}{4.925in}
\caption{\label{ex:slowcar} How far will you go?}
You and your friend, \studentB\index{\studentB}, are driving along at $55.0\unit{mph}$ and run out of gas $2.25\unit{mi}$ from a gas station.  You leave the car in gear and find that after $t_1=1.00\unit{min}$, you are travelling $v_1=30\unit{mph}$.  Will you make it to the gas station?

\color{blue}
The first thing we should do is notice what information is given to us and make sure that everything is in consistent units.  I will convert everything to \hyperref[ss:convertunits]{SI units}.
\begin{eqnarray*}
v_i & = & 55.0\unitfrac{mi}{hr} \convert{1609 \unit{ft}}{1.0000 \unit{mi}}_{4\unit{sig}} \convert{1\unit{hr}}{3600\unit{s}}_\mathrm{exact} = \sigfrac{24.5}{8}{m}{s} \\
\Delta x & = & 2.25\unit{mi} \convert{1609 \unit{ft}}{1.0000 \unit{mi}}_{4\unit{sig}} = \sig{362}{0.3}{m} = \sig{3.62}{0\ten{3}}{m} \\
\end{eqnarray*}
[This example is not done, but the work will result in the following numbers:
With $t_1$ and $v_1$, you can find $a=-1500\unitfrac{mi}{hr^2}$.  From that you can find, for $v_f=0\unitfrac ms$, that $t=2.2\unit{min}$ and $\Delta x = \sig{1.00}{8}{mi}$.]

You do not make it to the gas station.

\autoreturn{ss:acceleration}
\color{black}
\end{minipage}}
\end{example}
\begin{example}[hbpt]
\fcolorbox{black}{yellow!10}{\begin{minipage}{4.925in}
\caption{\label{ex:coasting} How fast should you start?}
You and your friend, \studentB\index{\studentB}, are driving along at $55.0\unit{mph}$ and run out of gas $2.25\unit{mi}$ from a gas station.  You put the car in neutral because you know that the car will slow down with an acceleration of $a=500\unitfrac{mi}{hr^2}$.  With what speed should you be going when you put your car into neutral in order to coast to a stop at the gas station?

\color{blue}
The first thing we should do is notice what information is given to us and make sure that everything is in consistent units.  I will convert everything to metric.
\begin{eqnarray*}
v_i & = & 55.0\unitfrac{mi}{hr} \convert{1609 \unit{ft}}{1.0000 \unit{mi}}_{4\unit{sig}} \convert{1\unit{hr}}{3600\unit{s}}_\mathrm{exact} = \sigfrac{24.5}{8}{m}{s} \\
\Delta x & = & 2.25\unit{mi} \convert{1609 \unit{ft}}{1.0000 \unit{mi}}_{4\unit{sig}} = \sig{362}{0.3}{m} = \sig{3.62}{0\ten{3}}{m} \\
\end{eqnarray*}
[This example is not done, but the work will result in the following numbers:
With $a=-500\unitfrac{mi}{hr^2}$.  You can find, for $v_f=0\unitfrac ms$, that $t=6.6\unit{min}$ and $\Delta x = 3.025\unit{mi}$.  You clearly make it to the gas station.  You can also find that for $\Delta x = 2.25\unit{mi}$, $t=\sig{3.25}{9}{min}$ and $v_f=\sigfrac{27.8}{388}{mi}{hr}$.]

So, if you start at $55.0\unitfrac{mi}{hr} - 27.8\unitfrac{mi}{hr} = \sigfrac{27.1}{6}{mi}{hr}$ you should make it exactly.

\autoreturn{ss:acceleration}
\color{black}
\end{minipage}}
\end{example}

\subsection{Freefall}\label{ss:freefall}\index{Freefall}\new{v2.2}{Adding detail}\mautoreturn{ex:ceiling}

Since acceleration is the change in velocity (magnitude and/or direction), it is possible to select your own rate of change while driving your car.  However, that acceleration is difficult to measure directly.  Your speedometer measures the speed and you have to compute your acceleration based on how quickly your speed changes.  It turns out that there is a convenient way to start from the acceleration and compute the expected velocity:  \hypertarget{d:freefall}{Drop a ball or your keys}\mlinkreturn[the description of physics]{d:physicspatterns}\inlife{}.  To convince yourself that objects do, in fact, accelerate when they fall, we can consider dropping items.  One of the complications during such an experiment will be discussed in \autoref{ss:airresistance}.  If we drop a sheet of paper, air resistance causes an obvious effect (fluttering).  For this section, I will assume that the mass-to-surface-area ratio is large enough that we can effectively\Touchstone{Recall \protect{\hyperref[s:effective2]{effective theories}}.}{} ignore the air resistance.

The \hypertarget{d:accgrav}{patterns} that you see when you drop objects is that objects fall faster than humans are used to paying attention to.  The green box of \autoref{irl:freefall} (on page~\pageref{irl:freefall})\footnote{\protect{\href{https://www.osha.gov/}{OSHA}} standard \protect{\href{https://www.osha.gov/pls/oshaweb/owadisp.show_document?p_table=standards&p_id=10839}{1926.1053(a)(3)(i)}} says ``Rungs \ldots of portable \ldots  and fixed ladders \ldots shall be spaced not less than 10 inches (25 cm) apart, nor more than 14 inches (36 cm) apart \ldots ''} shows you how you can pay close attention to the patterns that result from observing falling objects.
%
\begin{reallife}[bthp]
\hspace{-.2in}
\fcolorbox{black}{green!10}{\begin{minipage}{5.29in} \center
\caption{\label{irl:freefall}\index{Freefall!Real Life} The motion of dropped objects.}
\begin{minipage}{4.925in}
Because \studentC\index{\studentC} is a pitcher on the local baseball team, \heC\ decides to drop a ball and watch what happens.  You and \studentD\index{\studentD} decide to join him.  \studentD\ provides a few other objects that can also be dropped: a tennis ball, a hammer, a small Wonder Woman toy, and a broken cell phone.  Some of these are dropped at the same time.  \studentD\ notices that it is important to release the objects at exactly the same time.
\studentC\ notices that it is important to have the objects line up at the bottom so that if they travel at the same speed, then they hit at the same time.
\end{minipage}
\begin{realtable}
\dna{Drop \textit{any} two objects at the same time from the same height}
    {Are there any objects that always hit first or last? \ref{A:firstfall}}
    {If so, what are the properties of those objects? \ref{A:firstwhy}}
\dna{Drop one of these objects from about eye-level}
    {Observe the speed of the object as it falls}
    {Is the object moving at a constant speed? \ref{A:fallv}}
\dna{Climb a tall ladder, drop the ball from at least eight-feet high}
    {Observe the time it takes the object to pass four rungs near the top of the ladder and compare it to the time it takes the object to pass four rungs near the bottom of the ladder}
    {Is one set of four-rungs a shorter time or are they the same amount of time? \ref{A:falla}}
\end{realtable}
\begin{minipage}{4.925in}
You and your friends should get together to see if you can come up with a way to measure the acceleration due to the gravitational force.
\end{minipage}

\flushright
\multireturn{\mmr{\hyperlink{d:accgrav}{freefall}}, \mmr{\hyperlink{d:Fgrav}{the force of gravity}}}
\end{minipage}}
\end{reallife}
%
You should go do those experiments before reading further.  Go ahead.  I'll wait.

You did do them, right?  You're not just reading ahead?  Really?  OK.  Doing that experiment will help you see (1) that everything falls at the same rate and (2) that objects accelerate as they fall.
It turns out that, ignoring the effect of \hyperref[ss:airresistance]{air resistance}, all objects fall with the same acceleration (due to the gravity), $a_g = 9.81\unitfrac{m}{s^2}$ downwards.
In this book,\index{Freefall}
\important{``being in freefall'' will mean moving only under the influence of gravity and having an acceleration of $a_g = 9.81\unitfrac{m}{s^2}$ downwards.}
We will start to discuss the reason for this in \autoref{s:Fg} and then get into more detail in \autoref{c:gravity}.
For now, \autoref{ex:freefall} shows the type of experiment that can allow you to calculate the acceleration due to gravity.
%
\begin{example}[hbpt]
\fcolorbox{black}{yellow!10}{\begin{minipage}{4.925in}
\caption{\label{ex:freefall} How quickly does it fall?}
Your friend, \studentC\index{\studentC}, is a baseball player and is curious to learn about the rate that baseballs fly through the air.  You get on a $12\unit{ft}$ ladder and \heC\ lays on the ground below you aiming his radar gun (which measures speed) upwards.  Each rung is $1.0\unit{ft}$ apart and his gun is at the first rung.  When you drop the ball three rungs above the gun, he measures the final velocity to be $4.24\unitfrac ms$.  When you drop the ball six rungs above the gun, he measures the final velocity to be $6.00\unitfrac ms$.  When you drop the ball eleven  rungs above the gun, he measures the final velocity to be $8.11\unitfrac ms$.  Find the acceleration of the ball in each case.

\color{blue}
The first thing we should do is notice what information is given to us and make sure that everything is in consistent units.  I will convert everything to metric.
\begin{eqnarray*}
3\unit{rungs}  & = & 3.00\unit{ft} \convert{0.3048 \unit m}{1.00000\unit{ft}} = 0.\sig{914}{4}{m} \\
6\unit{rungs}  & = & 6.00\unit{ft} \convert{0.3048 \unit m}{1.00000\unit{ft}} = \sig{1.82}{9}{m} \\
11\unit{rungs}  & = & 11.00\unit{ft} \convert{0.3048 \unit m}{1.00000\unit{ft}} = \sig{3.35}{3}{m}
\end{eqnarray*}
To find the acceleration in each case, we can solve $v_f^2 = v_i^2 + 2 a \, \Delta x$ for the acceleration:
\begin{eqnarray*}
a_3 & = & \frac{(4.23\unitfrac ms)^2 - (0 \unitfrac ms)^2}{2(0.\sig{914}{4}{m})} \ = \ \sigfrac{9.83}{1}{m}{s^2} \\
a_6 & = & \frac{(6.00\unitfrac ms)^2 - (0 \unitfrac ms)^2}{2(\sig{1.82}{9}{m})} \ = \ \sigfrac{9.84}{1}{m}{s^2} \\
a_11 & = & \frac{(8.11\unitfrac ms)^2 - (0 \unitfrac ms)^2}{2(\sig{3.35}{3}{m})} \ = \ \sigfrac{9.80}{8}{m}{s^2}
\end{eqnarray*}
Notice that these have some variation due to the rounding.  It turns out that the variation in the value of acceleration depends on the composition of the earth in your location as well as your altitude above sea-level.  That will be discussed in detail in \autoref{c:gravity}, for simplicity we will assume that all objects accelerate at the rate of $9.81\unitfrac{m}{s^2}$ when they are solely under the influence of gravity.

\multireturn{\mmr{\hyperlink{d:accgrav}{freefall}}, \mmr{\autoref{d:Fgball}}}
\color{black}
\end{minipage}}
\end{example}
%
It also turns out that you can also see this acceleration where you throw an object straight up into the air.


\section{Complications}
\subsection{Non-Inertial Accelerated Reference Frames} \label{ss:noninertial}\mmultireturn{\mmr{\hyperlink{d:referenceframe}{Reference Frames}}, \mmr{\autoref{ss:NI}}}
\index{Reference Frames!Inertial}
\index{Reference Frames!Non-inertial}

[Discuss non-rotating linearly accelerating \hyperlink{d:referenceframe}{reference frames}. See also \autoref{s:noninertial} for a discussion on rotating reference frames.]

[Comment on the Earth as essentially stationary?  See \autoref{s:effective2} on effective theories.\new{v2.2}{Effective theories}{}]

\subsection{Air Resistance}\label{ss:airresistance}\mmultireturn{\mmr{\hyperlink{d:accgrav}{freefall}}, \mmr{\ref{A:falls}}, \mmr{\autoref{s:Fg}}}
Terminal velocity\ldots
When do we include air resistance and when can we ignore it?  \ldots
[Comment on air resistance being a small effect in some cases?  See \autoref{s:effective2} on effective theories.\new{v2.2}{Reference effective theories}{}]

\subsection{Multi-Step Solutions}\new{v2.3}{new section, new example}

\begin{example}[hbpt]
\fcolorbox{black}{yellow!10}{\begin{minipage}{4.925in}
\caption{\label{ex:ceiling} \studentC\ hits the ceiling!}
\studentC\index{\studentC} gets bored one day in physics class (what?!?) and tossed a baseball ($m_b = 0.145\unit{kg}$) at the ceiling\ldots a little too hard.  The initial velocity is $v_i = +5.00\unitfrac ms \jhat$ and it leaves \hisC\ hand $1.00\unit{m}$ below the ceiling.  The ball hits the ceiling and when it returns to \hisC\ hand, it is travelling $\vec v_f=-4.73\unitfrac ms \jhat$, slower than \heC\ expected.  (a) Assuming that the ball is in contact with the ceiling for $\Delta t = 0.142\unit{s}$, find the acceleration of the ball during the collision.  (b) On the other hand, if the ceiling had not been there, then how high would the ball have gone and how fast would it have been going when it returned to \studentC's hand?

\color{blue}
In order to solve part (a) for the acceleration, we need to recognize that (1) there are five stages to the motion of the baseball and that (2) \hyperlink{d:EOM}{the equations of constant acceleration} assume that the acceleration is constant.  The five stages of the motion are: the throw, the ball moving from \studentC's hand up to (but not yet hitting) the ceiling, the ball hitting the ceiling, the ball falling from the ceiling down to (but not yet touching) \studentC's hand, and the catching of the ball.  The acceleration is $\deq a = \frac{v_f-v_i}{\Delta t}$, but the story of this equation says that since the acceleration is only during the interaction with the ceiling, then the velocities in this equation are just-before the ball hits and just-after the ball hits (not the very beginning velocity and not the very final velocity).  Similarly, the $\Delta t$ in this equation is only the time during which it was interacting with the ceiling, not the entire flight.

During \underline{the first stage}, the ball is accelerating upwards and \studentC\ is interacting with the ball.  We are not going to consider this part of the motion at all because we are given the velocity that ends this stage (and begins the next stage).

\underline{The second stage} of the motion is while the ball moves from \studentC's hand up to the ceiling. During this stage only the gravitational force is acting on the ball.  Since it has left \studentC's hand, \heC\ is not interacting with it.  Since it has not yet hit the ceiling, the ceiling is not interacting with it.  We can therefore use \hyperlink{d:EOM}{the equations of constant acceleration} to describe the motion.  During this portion of the motion we know that the velocity at the bottom is $\vec v_\mathrm{bot} = +5.00\unitfrac ms \jhat$, that it travels $\Delta \vec x = +1.00\unit{m} \jhat$, and that (because it is in \href{ss:freefall}{freefall}) it is accelerating at $\vec a_g = -9.81\unitfrac{m}{s^2} \jhat$.

\color{black}
{}\hfill {\footnotesize \autoref*{ex:ceiling} continued on next page\ldots}
\end{minipage}}
\end{example}
\begin{example}[p]
\fcolorbox{black}{yellow!10}{\begin{minipage}{4.925in}\setlength{\parskip}{3pt}
{\footnotesize \autoref*{ex:ceiling} continued from previous page\ldots}
\color{blue}

We can find the time of flight (not useful) and the velocity when the ball reaches the ceiling:
\begin{eqnarray*}
v_\mathrm{top} & = & \sqrt{ v_\mathrm{bot}^2 + 2 a \, \Delta x} \\
v_\mathrm{top} & = & \sqrt{ (+5.00\unitfrac ms)^2 + 2 (-9.81\unitfrac{m}{s^2})(+1.00\unit{m})} \\
v_\mathrm{top} & = & +\sigfrac{2.31}{9}{m}{s}
\end{eqnarray*}
Note: When you take the square root, you have to decide if you should take the positive sign or the negative sign.  In this case, the ball is still moving upwards, so we \textit{choose} the positive sign.

\underline{The third stage} is while it is interacting with the ceiling.  In order to find the acceleration during this motion, we need to know the velocity immediately before hitting the ceiling (which we just found) and the velocity just after it finishes hitting the ceiling (which we have not yet found).  We will come back to this step.

\underline{The fourth stage}, like the second, is while the ball moves from the ceiling down to \studentC's hand.  During this portion of the motion we know that the velocity at the bottom (final) is $\vec v_\mathrm{bot} = -1.67\unitfrac ms \jhat$, that it travels $\Delta \vec x = -1.00\unit{m} \jhat$, and that (because it is in \href{ss:freefall}{freefall}) it is accelerating at $\vec a_g = -9.81\unitfrac{m}{s^2} \jhat$.  We can find the time of flight (not useful) and the velocity when the ball leaves the ceiling (initial), solving $v_\mathrm{bot}^2 = v_\mathrm{top}^2 + 2 a \, \Delta x$ for $v_\mathrm{top}$:
\begin{eqnarray*}
v_\mathrm{top} & = & \sqrt{ v_\mathrm{bot}^2 - 2 a \, \Delta x} \\
v_\mathrm{top} & = & \sqrt{ (-1.67\unitfrac ms)^2 - 2 (-9.81\unitfrac{m}{s^2})(+1.00\unit{m})} \\
v_\mathrm{top} & = & -\sigfrac{4.73}{4}{m}{s}
\end{eqnarray*}
Note: When you take the square root, you have to decide if you should take the positive sign or the negative sign.  In this case, the ball is now moving downwards, so we \textit{choose} the negative sign.

\color{black}
{}\hfill {\footnotesize \autoref*{ex:ceiling} continued on next page\ldots}
\end{minipage}}
\end{example}
\begin{example}[p]
\fcolorbox{black}{yellow!10}{\begin{minipage}{4.925in}\setlength{\parskip}{3pt}
{\footnotesize \autoref*{ex:ceiling} continued from previous page\ldots}
\color{blue}

During \underline{the fifth stage}, the ball is accelerating upwards while moving downwards and so \studentC\ is stopping the ball.  We are not going to consider this part of the motion at all. \\

Now that we have the velocities immediately before and after the collision with the ceiling, we can find the acceleration:
\[ a = \frac{v_f-v_i}{\Delta t} = \frac{(-\sigfrac{4.73}{4}{m}{s})-(+\sigfrac{2.31}{9}{m}{s})}{(0.142\unit s)} = -\sigfrac{28.0}{9}{m}{s^2} \,\jhat \]
Notice that the acceleration is negative because the ball went from going up to going down.

To solve part (b), we can just consider from after-thrown to before-caught.  During this motion, assuming there is no ceiling, the entire motion is in freefall, so we can use $v_f^2 = v_i^2 + 2 a \, \Delta x$ and solve for $\Delta x$.  However, we only want to consider from the lowest point to the highest point, not all the way back to \studentC's hand.

\centering{THIS NEEDS TO BE FINISHED}

\flushright
\multireturn{\mmr{\ref{se:ceiling}}, \mmr{\ref{se:throw-up}}}
\color{black}
\end{minipage}}
\end{example}\dothis{finish \protect{\autoref{ex:ceiling}}.  Maybe make it two examples, instead of one?}


\chapter{Two-Dimensional Motion}

\subsection{Ballistic Freefall}\label{ss:ballistic}\mautoreturn{sss:vectorequations}

\hypertarget{d:ballistic}{Discussion about throwing a ball\ldots}\mlinkreturn[the description of physics]{d:physicspatterns}\inlife.

\hypertarget{d:2Dmotion}{For 2-dimensional motion}\mlinkreturn[$F=ma$]{d:f=ma}, we will use \hyperref[sss:vectorequations]{vector equations} to describe the relationships.
When we write $\vec v_f = \vec v_i + \vec a t$, we mean that this relationship holds for the $x$-components and separately for the $y$-components:
\[ v_{fx} = v_{ix} + a_x t \hspace{1cm} v_{fy} = v_{iy} + a_y t \]

\section{Complications}
\subsection{Air Resistance}\label{ss:ballisticairresistance}
Terminal velocity\ldots non-parabolic paths \ldots \autoref{irl:nonparabolic} (pg.~\pageref{irl:nonparabolic})
\begin{reallife}[bhp]
\hspace{-.2in}
\fcolorbox{black}{green!10}{\begin{minipage}{5.29in} \center
\caption{\label{irl:nonparabolic} Baseball pitches are not usually parabolic.}
\begin{minipage}{4.925in}
\studentC\index{\studentC} is a pitcher on the local baseball team.  \HeC\ throws a fast ball, a slider, a curve ball, and a knuckleball.
\end{minipage}
\begin{realtable}
\dna{Go to a baseball game on a calm day.  Sit near third base.}
    {The path of fly balls to left field}
    {Are they parabolic? \ref{A:fly.balls}}
\dna{Go to a baseball game on a calm day.  Sit near third base.}
    {The path of pitch towards home plate.}
    {Are they parabolic? \ref{A:pitches.side}}
\dna{Go to a baseball game.  Sit up high behind home plate.}
    {The path of the baseball for various pitches.}
    {Do they all fly straight over the plate? \ref{A:pitches.top}}
\end{realtable}

\flushright
\multireturn{\mmr{\hyperlink{d:physicspatterns}{the description of physics}}, \mmr{\autoref{ss:ballisticairresistance}}}
\end{minipage}}
\end{reallife}

[Comment on air resistance being a small effect in some cases?  See \autoref{s:effective2} on effective theories.\new{v2.2}{Reference Effective theories}{}]



\chapter{Force}\label{c:force}

\section{How Physicists Use the Words (Notation)}\label{s:forcewords}

The technical term \textbf{force}\index{Force!description} refers to the general idea of pushing or pulling.  In the same way that\touchstone{} physicists use the word \hyperlink{d:acc}{acceleration} (technically \textit{changing the velocity}) to mean \textbf{either} \textit{speeding up} (colloquially ``acceleration'') \textbf{or} \textit{slowing down} (colloquially ``deceleration'') \textbf{or} \textit{changing the direction} (colloquially ``turning''), we will use \hypertarget{d:forcenoun}{\textbf{force} as a noun}\index{Force!noun}\mlinkreturn[heat as a verb]{d:heatverb} referring to the act of pushing or pulling.

\hypertarget{d:interaction}{You should note}\mmultireturn{\mmr{\autoref{ex:braced}}, \mmr{\hyperref[d:Fgball]{the falling ball}}} that you can't have a push or pull without \textbf{both} a thing that pushes or pulls \textbf{and} a thing that is pushed or pulled.\aside{Push or Pull}{By now you may have noticed that it is tedious to keep saying ``pushed or pulled,'' so I will only say ``push'' even when I am including the possibility of ``pushing or pulling''.}{}
\important{Forces are \underline{necessarily} an interaction\index{Interaction}\index{Force!Interaction} between two objects.}
Sometimes we care about the thing doing the pushing or pulling, sometimes we don't.  We always care about the thing being pushed or pulled.  We will \underline{distinguish these objects} by referring to the object that is pushing as the object ``causing the force'' or ``exerting the force'' and by referring to the object that is being pushed as the object ``feeling the force''.  We will \underline{distinguish these forces} as follows: Let's imagine that \studentB\index{\studentB} gives \studentA\index{\studentA} a good-natured shove in the arm.  The following are useful descriptions and are different ways of describing the same action.
\begin{itemize}\itemsep 1pt
\item \studentB\ exerted a force \underline{on} \studentA.
\item \studentA\ felt a force \underline{from} \studentB.
\item There was a force \underline{on} \studentA\ \underline{by} \studentB.
\end{itemize}
The notation for this will be $F_{A,B}$ where the first subscript is the person who felt the force (who the force is ``on'') and the second subscript is who exerted the force (who the force is ``by'').  In those instances when we only care about who is feeling the force and not who is exerting the force, we might just use one subscript $F_A$.  In some cases, there may be two forces acting on one person (or object).  In that case, it will be obvious who is feeling the force and we will use the subscript to distinguish which force it is, such as $F_1$ or $F_2$, rather than who feels the force.  This will be more relevant when we discuss in \autoref{c:forcetype} the types of forces that might be applied.

\hypertarget{d:Newtonahead}{Looking ahead} to \hyperref[s:Newton]{Newton's Laws}\index{Laws!Newton}\Touchstone{Recall the distinction between \protect{\hyperref[s:law]{theories and laws}}.}, you should be ready to notice that the \hyperref[ss:NI]{first law} is about objects that are not feeling a force, the \hyperref[ss:NII]{second law} is about a specific object that is feeling a force, and the \hyperref[ss:NIII]{third law} is about the interaction between the two objects.  In all three of these, we care about the object feeling the force.   It is only in the third law that we care about the object exerting the force.

\hypertarget{d:pushvector}{Looking back}, forces are \hyperref[sss:scalarvector]{vectors}:
\important{Pushing on something intrinsically involves a direction.}
You will use this property to show that multiple people pushing\inlife{} in the same direction increases the effect, whereas multiple people pushing in opposite directions reduces the effect.  One might say that people who push an object in opposite directions work\footnote{After you study \protect{\autoref{s:work}}, this play on words will be hilarious!} against each other.  Because the force is a vector, whenever you are answering a question about a force, you should always expect to give the strength of the force (the \hyperref[ss:vectors]{magnitude}) and \hyperref[ss:vectors]{the direction} of the force (relative to some specific axis, usually the positive $x$-axis).

\section{Connecting the Concepts: Newton's Laws}\label{s:Newton}\mlinkreturn[how to describe forces]{d:Newtonahead}

Newton's Laws (recall \hyperref[s:law]{Theory versus Law}) describe our observations about three questions\inlife:
\begin{enumerate}\itemsep 1pt
  \item What happens to an object when I \textit{don't} push on it?
  \item What happens \textit{to an object} when I do push on it?
  \item What happens \textit{to me} when I push on an object?
\end{enumerate}
The answer to these questions have a precise, concise, technical language and the point of the next three subsections are to translate that into (modern) English, into math, and into intuition.  The statement of these laws has slightly different versions in different texts to emphasize different points.  We will state them as follows\index{Newton!Laws}\index{Laws!Newton}:
\begin{enumerate}
    \item \hypertarget{sum:Newton'sLaws}{When} viewed from an inertial reference frame, an object with no forces acting on it will maintain its velocity, which may be zero.
    \item When viewed from an inertial reference frame, the vector-sum of all forces acting on an object will cause that object to accelerate in proportion to its mass: $\vec F_\mathrm{net} = m \vec a$.
    \item For every force acting (the ``action'') on one object by an other object, there is an equal-in-magnitude reaction-force acting on the other object in the opposite direction.
\end{enumerate}
\hypertarget{d:NewtonInertial}{There are a few terms} that should be clarified in these laws.  Being in an inertial \hyperlink{d:referenceframe}{reference frame}\index{Reference Frames!Inertial} essentially means being in a place in which you do not have to hold on in order to maintain your position.  If you are \hyperref[s:acceleration]{accelerating} (recall that this means \textit{speeding up}, \textit{slowing down}, or \textit{changing direction}) then you are not in an inertial reference frame, but rather are in a non-inertial reference frame.  In this case, you will misinterpret the forces acting.  This will be discussed in more detail in \autoref{s:noninertial} when we discuss centripetal force and in \autoref{ss:coriolis} when we discuss the Coriolis effect.

Sometimes Newton's first law is written to include the phrase ``an \hypertarget{d:objectinmotion}{object in motion}'', which I will be careful to link directly to \hyperlink{d:motion}{velocity}, as was done above.  However, it technically should reference the momentum\foreshadow, which is discussed in \autoref{c:momentum}.

The way Newton's third law is often written and referred to includes the words ``action'' and ``reaction''.  Newton was referring to forces with these words and to keep it clear in our discussion, we will use the word force, with the occasional clarification of the action-force or the reaction-force.

\subsection{Translating Newton's First Law: The Law of Inertia}\label{ss:NI}\mlinkreturn[how to describe forces]{d:Newtonahead}

%\important{} is not designed to start a new paragraph
\ \vspace{-12pt}
%\begin{quote}
\important{\textbf{Newton's First Law}:\index{Newton!First Law} When viewed from an inertial reference frame, an object with no forces acting on it will maintain its velocity, which may be zero.}
%\end{quote}
Let's take this apart and connect it to your daily experiences.  Looking ahead, we will discuss the surface of the Earth as a \hyperlink{d:noninertial}{non-inertial rotating reference frame}\Touchstone{You might also recall the discussion in \protect{\autoref{ss:noninertial}}.}; however, the effect is small enough that for most of what we \hyperlink{d:casual}{casually observe}, we can safely pretend that the Earth is stationary and that we are actually at rest while sitting on the curb watching the world go by.  This is so true that our human brains already interpret everything around us as though it were an inertial reference frame.  This psychological perspective is exactly the feature that both allows us to make fairly reliable predictions about the world around us \textit{and} causes us to make incorrect judgements when we encounter non-inertial reference frames.  That is to say, as long as we don't measure our world too closely\Touchstone{Recall \protect{\hyperref[s:effective2]{effective theories}}.}, we are viewing it from an essentially inertial reference frame.  This point is so implicit, that many books do not even include the portion of the statement referring to the reference frame.
\begin{ForMe}
\todo{Consider adding comment about Newton not including ``inertial reference frame'' to his laws.}
%\footnote{Newton himself did not include it in his original statement.}%\urgcap{Newton's ``inertial frames''}{check the implications of this}{}
\end{ForMe}

\hypertarget{d:atrestinmotion}{The rest of this statement} is often written a little differently (and less concisely) as ``an object at rest remains at rest unless acted on by an external force and an object in motion remains in motion unless acted on by an external force.''  Since being \hyperlink{d:atrest}{``at rest''} is a statement about the velocity ($\vec v=0$) and being \hyperlink{d:motion}{``in motion''} is also a statement about the velocity, each of these statements can be understood as saying that
\important{Forces are those things that cause a change in the velocity.}
In other words, Newton's first law says that without a force, the velocity will not change.  In the discussion of \hyperref[sss:equilibrium]{equilibrium}, we will note that this is often extended to say that without a \hyperref[sss:netforce]{net force} the velocity will not change, but that is a special case of \hyperref[ss:NII]{Newton's second law}.

\subsubsection{Inertia}\label{sss:inertia}\index{Inertia}

This law is often called the ``law of inertia''.  The concept of inertia can be described as \textit{the tendency of an object to maintain its velocity}.  This is describing how the object behaves when you don't do anything to it.  The inertia is not a quantity that physicists calculate, but physicists do refer to objects as having a lot of inertia, usually to indicate that it will take a large force to change the object's motion, or as having a small amount of inertia, usually to indicate that it should be relatively easy to change the object's motion.  However, the inertia does not actually refer to the force needed.  Instead, the inertia most often refers to the ``inertial mass'' of an object, which shows up in the second law.
\important{Inertia is not a force.}
Sometimes when physicists are not being careful with their language, they will appear to use the word inertia interchangeably with the term \hyperref[c:momentum]{momentum}, which we will discuss in more detail in \autoref{ss:inertia}.

\subsubsection{How the Laws Work Together}\label{sss:NItogether}

You should notice that Newton's First Law is about what happens when you are \textit{not pushing} on the object, which is to say, the tendency of an object to maintain its own motion without a force acting on it; this is the inertia of the object.  On the other hand, Newton's Second Law is about what happens \textit{to the object} when you \textit{do push} on an it.  This is what we will consider next.  After that, Newton's third law will describe what happens \textit{to the thing pushing} rather than just to the thing being pushed.  \hyperref[sss:NIItogether]{The section} at the end of \autoref{ss:NII} will explore these ideas further.



\subsection{Translating Newton's Second Law: The Equation Law}\label{ss:NII}\mmultireturn{\mmr{\autoref{sss:vectorequations}}, \mmr{\hyperlink{d:Newtonahead}{how to describe forces}}, \mmr{\hyperlink{d:atrestinmotion}{Newton's first law}}, \mmr{\autoref{d:Fgball}}}

%\important{} is not designed to start a new paragraph
\ \vspace{-12pt}
%\begin{quote}
\important{\textbf{Newton's Second Law}\index{Newton!Second Law}: When viewed from an inertial reference frame, the vector-sum of all forces acting on an object will cause that object to accelerate in proportion to its mass: $\vec F_\mathrm{net} = m \vec a$.}\docaption{Inline-math formatting}{Why does an equation that starts a new line get indented slightly?}
%\end{quote}
Let's take this apart and connect it to your daily experiences.  As with Newton's First Law, the \hyperlink{d:noninertial}{non-inertial rotating reference frame} of the surface of the Earth is a small enough effect that, as long as we don't measure our world too closely\Touchstone{Recall \protect{\hyperref[s:effective2]{effective theories}}.}, we can pretend that we are viewing it essentially from an inertial reference frame.

\hypertarget{d:f=ma}{For this law}, it is often sufficient to write down the equation and know that the words are there for back-up.  While most people have no trouble remembering $F=ma$, it is important to pay attention to two aspects:
\begin{itemize}
\item This is a vector-equation, which means\Touchstone{Recall (1) the generic explanation for \protect{\hyperref[sss:vectorequations]{vector equations}} and (2) the \protect{\hyperlink{d:2Dmotion}{vector-equations}} for two-dimensional motion} that
    \begin{itemize}
    \item the equation is true for each component separately, and
    \item the direction of $\vec F_\mathrm{net}$ is the same as the direction of $\vec a$ (which, of course, might be different than the direction of the velocity).
    \end{itemize}
\item The force in this equation is the \textit{net force}, which means that we must consider \underline{all} forces that are acting on this object and only those forces that are acting on \underline{this} object.
\end{itemize}
\addcontentsline{los}{story}{$\vec F_\mathrm{net} =m \vec a$}
\phantomsection\label{st:F=ma}\thestoryof{\vec F_\mathrm{net} = m \vec a}\mautoreturn{ex:2Dforce}
This equation is all about what happens to a specific object, $m$.  If the object, $m$, is accelerating in a particular direction, $\vec a$, then it is because the combination of forces, $\vec F_\mathrm{net}$, do not entirely cancel each other out.  This also can be expressed as: if the combination of forces, $\vec F_\mathrm{net}$, do not entirely cancel each other out, then our friend $m$ must be accelerating,~$\vec a$, in a particular direction.  Furthermore the resulting direction of the net force determines the direction of the acceleration.  \hypertarget{d:f=ma}{Connecting the English} and the math:
\[\begin{array}{cccc}
\deq \vec F_\mathrm{net} & = & \deq m & \deq \vec a \\
\EqStoryOver{65pt}{the combination of all forces acting on $m$}{}
& \EqStoryOver{33pt}{causes}{}
& \EqStoryOver{35pt}{that object}{}
& \EqStoryOver{40pt}{to change its velocity}{}
\end{array}\]
You should \hyperref[s:acceleration]{recall}\touchstone, that the direction of the acceleration does not determine the direction \textit{of the motion}, but rather determines the direction \textit{of the change} in motion.  That idea will be important\foreshadow{} when we discuss how a \hyperref[s:FT]{tension} acts as a \hyperref[s:centripetal]{centripetal force}, the relationship between velocity and acceleration in a \hyperref[s:springs]{spring} that \hyperref[c:SHMspring]{oscillates}, and objects that are propelled through either a \hyperref[s:Gfield]{gravitational} or an \hyperref[ss:Efield]{electrical} field.

\subsubsection{Units of Force}\label{sss:unit-N}\mautoreturn{ss:units}\index{Force!Units of}

Recall that the fundamental units of the \hyperref[ss:units]{SI-system} are meters, kilograms, and seconds (MKS)\addlink{maybe note the MKS-to-SI transition. maybe leave that in \protect{\autoref{ss:units}}}.  With our relationship connecting force to mass $\unit{(kg)}$ and acceleration $\left(\unitfrac{m}{s^2}\right)$, we can see that the units of force are ${}\unitfrac{kg \cdot m}{s^2}$.  This quantity is so common that we would like to have a shorthand for it.  Furthermore, Sir Isaac Newton did such ground-breaking work on the concept, that it was decided in 1948\footnote{According to: International Bureau of Weights and Measures (1977),The international system of units (330-331) (3rd ed.), U.S. Dept. of Commerce, National Bureau of Standards, \protect{\href{https://books.google.com/books?id=YvZNdSdeCnEC&pg=PA17\#v=onepage&q&f=false}{p. 17}},
%  ISBN 0745649742, Found at https://en.wikipedia.org/wiki/Newton_(unit)
which refers to
\protect{\href{http://www.bipm.org/jsp/en/ViewCGPMResolution.jsp?CGPM=9&RES=7}{the 7th resolution}} (Mar, 2017) of
\protect{\href{http://www.bipm.org/jsp/en/ListCGPMResolution.jsp?CGPM=9}{the 9th CGPM}} (Mar, 2017).} to name the unit the Newton, such that
\important{$1 \unit{N} = 1 \unitfrac{kg \cdot m}{s^2}$}

\subsubsection{Calculating the Net Force}\label{sss:netforce}\mlinkreturn[Newton's first law]{d:atrestinmotion}

The word ``net'' that goes with force is here to indicate the total, which is useful to think of as ``everything collected with the net.''\footnote{Although according to \protect{\href{http://www.etymonline.com/index.php?term=net&allowed_in_frame=0}{etymonline.com}} (Mar, 2017), it is actually from the Old French \textit{net} for ``neat'' or ``clean'', having the sense of trim and elegant.}  The intention here is that wherever there are multiple forces acting on a single object, we must combine them as \hyperref[ss:vectors]{vectors} as follows: \\
\begin{minipage}[c]{3.25in}
\begin{sample}
\item\label{se:netFadd} If there is a $5.0 \unit{N}$ force to the right and a $4.0 \unit{N}$ force to the right, then the net force is $9.0 \unit{N}$ to the right.
    \[ \vec F_\mathrm{net} = \vec F_1 + \vec F_2 = \left( 5.0\unit{N} \ihat\right) + \left( 4.0 \unit{N} \ihat\right) = +9.0\unit{N} \ihat \]
\end{sample}
\end{minipage}\mmultireturn{\mmr{\ref{se:netF-a}}, \mmr{\hyperref[sss:equilibrium]{Equilibrium}}, \mmr{\ref{se:FBD-AB}}, \mmr{\autoref{f:firstFBD}}, \mmr{\hyperlink{d:FBD-AB}{discussion about \ref*{se:FBD-AB}}}}
\hfill
\fbox{\begin{minipage}[c]{1.5in}
\begin{FBD}{10}{15}{15}{10}{object}
\twori{50}{$5\unit N$}{black}{40}{$4\unit N$}{black}
\end{FBD}
\end{minipage}}\dothis{Make this object a desk so that we can have \studentA\ and \studentB\ helping you rearrange your room in your residence hall.}{}
\begin{minipage}[c]{3.25in}
\begin{sample}
\item\label{se:netFsub} If there is a $5.0 \unit{N}$ force to the left and a $4.0 \unit{N}$ force to the right, then the net force is $1.0 \unit{N}$ to the left.
    \[ \vec F_\mathrm{net} = \vec F_1 + \vec F_2 = \left(-5.0\unit{N} \ihat\right) + \left( 4.0 \unit{N} \ihat\right) = -1.0\unit{N} \ihat \]
\end{sample}
\end{minipage}\mmultireturn{\mmr{\ref{se:netF-m}}, \mmr{\hyperref[sss:equilibrium]{Equilibrium}}}
\hfill
\fbox{\begin{minipage}[c]{1.5in}
\begin{FBD}{10}{15}{15}{10}{object}
\onele{50}{$5\unit N$}{black}
\oneri{40}{$4\unit N$}{black}
\end{FBD}
\end{minipage}}
\begin{minipage}[c]{3.25in}
\begin{sample}
\item\label{se:equi} If there is a $3.0 \unit{N}$ force to the right and a $3.0 \unit{N}$ force to the left, then the net force is $0.0 \unit{N}$.
    \[ \vec F_\mathrm{net} = \vec F_1 + \vec F_2 = \left( 3.0\unit{N} \ihat\right) + \left(-3.0 \unit{N} \ihat\right) = 0.0\unit{N} \ihat \]
    \mbox{In this case, the object is said to be ``\hyperref[sss:equilibrium]{in equilibrium}.''}
\end{sample}
\end{minipage}\mmultireturn{\mmr{\hyperref[sss:equilibrium]{Equilibrium}}, \mmr{\hyperlink{d:equi}{discussion about \ref*{se:equi}}}}
\hfill
\fbox{\begin{minipage}[b]{1.5in}
\begin{FBD}{10}{15}{15}{10}{object}
\onele{30}{$3\unit N$}{black}
\oneri{30}{$3\unit N$}{black}
\end{FBD}
\end{minipage}}
The images included in these examples will eventually\foreshadow{} be referred to as ``\hyperref[sss:FBD]{free-body diagrams}\index{Free-Body Diagrams!Images},'' but for now, you can just consider them images of the forces acting on the bodies.

\hypertarget{d:netforce}{Next}, we should do a couple of examples that show the math for situations with forces in two dimensions.   The first, \autoref{ex:2Dforce}  (pg.~\pageref{ex:2Dforce}), has one force in the $x$-direction and another in the $y$-direction.
%
\begin{example}[hb]
\fcolorbox{black}{yellow!10}{\begin{minipage}{4.925in}
\caption{\label{ex:2Dforce} An object is pushed by perpendicular forces.}
A $2.0\unit{kg}$ mass is being pushed north with $5.0\unit{N}$ and east with $4.0 \unit{N}$.  What is the net force?
\color{blue}

\vspace{9pt}
\begin{minipage}{3.25in}
Since we have multiple forces acting on a mass to cause an acceleration, it should be clear (recall the \hyperref[st:F=ma]{story}) that we need to use Newton's second law and find the net force in order to compute the acceleration.  We will, as usual, start with a free-body diagram (at right).  This example is made easier because the forces happen to be at right angles and so finding their $x$ and $y$ components is not difficult. By adding
\end{minipage}
\hfill
\fbox{\begin{minipage}{1.5in}
\begin{FBD}{10}{15}{15}{40}{object}
\oneup{50}{$5\unit N$}{black}
\oneri{40}{$4\unit N$}{black}
\end{FBD}
\end{minipage}}
\vspace{2pt}

\begin{minipage}[c]{1.95in}
\begin{forcetable}
\force{F_1}{ 0 \unit N}{+5 \unit N}
\force{F_2}{+4 \unit N}{ 0 \unit N} \hline\hline
\force{F_\mathrm{net}}{+4\unit N}{+5\unit{N}}
\end{forcetable}
\end{minipage}
\hfill
\begin{minipage}[c]{2.8in}
the $x$-components and separately adding the $y$-components, we have found the components of the net force.  From there, we can easily find the magnitude and direction of the net force.
\end{minipage}
\magdir{+4\unit N}{+5\unit{N}}{F_\mathrm{net}}{\sig{6.4}{0}{N}}{\theta}{\sig{51}{.3}{^\circ}}{N of E}
%
(The direction can be stated as $\theta = 51^\circ \textrm{N of E}$ or as $\phi = 39^\circ\textrm{E of N}$.)
\autoref{ex:2Dfa}  (pg.~\pageref{ex:2Dfa}) will use this calculation to find the acceleration.
\flushright
\multireturn{\mmr{the discussion of \hyperlink{d:netforce}{the net force}}, \mmr{\autoref{ex:2Dforce2}}}
\end{minipage}}
\end{example}
%
The second, \autoref{ex:2Dforce2}  (pg.~\pageref{ex:2Dforce2}), has one force in the $x$-direction and the other in the second quadrant.
%
\begin{example}[hb]
\fcolorbox{black}{yellow!10}{\begin{minipage}{4.925in}
\caption{\label{ex:2Dforce2} Three forces act on an object.}
A $2.0\unit{kg}$ mass is being pushed northwest with $5.0\unit{N}$ at an angle $63.4^\circ\textrm{N of W}$, southwest with $6.0\unit{N}$ at an angle of $21.8^\circ\textrm{S of W}$, and east with $4.0 \unit{N}$.  What is the net force?
\color{blue}

\vspace{9pt}
\begin{minipage}{3.25in}
This follows the same logic as \autoref{ex:2Dforce}  (pg.~\pageref{ex:2Dforce}), which I will not restate here. This example is slightly harder because the forces have to be split into their $x$ and $y$ components. By adding the $x$-components and separately adding the $y$-components, we have found the components of the net force.  From there, we can easily find the magnitude and direction of the net force.
\end{minipage}
\hfill
\fbox{\begin{minipage}{1.5in}
\begin{FBD}{10}{15}{25}{40}{object}
%\oneup{50}{$5\unit N$}{black}
%\put(24,56){\vector(-1,2){22.36}}
%\put(0,102){\color{black} \tiny $5\unit N$}
\oneul{22}{48}{-1}{2}{$5 \unit N$}{black}
\onedl{58}{22}{-5}{-2}{$6 \unit N$}{black}
\oneri{40}{$4\unit N$}{black}
\end{FBD}
\end{minipage}}
\vspace{2pt}

%\begin{minipage}[c]{1.95in}
\begin{forcetable}
\foTWO{F_1}{-(5.0\unit N)\cos(63.4^\circ) =}{-\sig{2.2}{4}{N}}{(5.0\unit N)\sin(63.4^\circ) =}{+\sig{4.4}{7}{N}}
\foTWO{F_2}{-(6.0\unit N)\cos(21.8^\circ) =}{-\sig{5.5}{7}{N}}{-(6.0\unit N)\sin(21.8^\circ) =}{-\sig{2.2}{3}{N}}
\force{F_3}{+4.0 \unit N}{ 0 \unit N} \hline\hline
\force{F_\mathrm{net}}{-\sig{3.8}{1}{N}}{+\sig{2.2}{4}{N}}
\end{forcetable}
%\end{minipage}
%
\magdir{-\sig{3.8}{1}{N}}{+\sig{2.2}{4}{N}}{F_\mathrm{net}}{\sig{4.4}{2}{N}}{\theta}{\sig{30}{.5}{^\circ}}{N of W}
%
(The direction can be stated as $\theta = 31^\circ \textrm{N of W}$ or as $\phi = 60^\circ\textrm{W of N}$.)
\autoref{ex:2Dfa2}  (pg.~\pageref{ex:2Dfa2}) will use this calculation to find the acceleration.

\linkreturn[the net force]{d:netforce}
\end{minipage}}
\end{example}


\subsubsection{Using the Net Force to Calculate Other Quantities}

Generally, the point of finding the net force is that it causes an object to change its velocity. Let's also consider a few simple examples of this calculation.
\begin{sample}
\item\label{se:netF-a}\mmultireturn{\mmr{\hyperlink{d:m=f/a}{finding $m$ from $F=ma$}}, \mmr{\ref{se:FBD-AB}}, \mmr{\autoref{f:firstFBD}},\mmr{\hyperlink{d:FBD-AB}{discussion about \ref*{se:FBD-AB}}}, \mmr{\ref{se:weightA}}, \mmr{\ref{A:netF-a}}, \mmr{\autoref{f:firstFBDupdate}}} If the forces in \ref{se:netFadd}\dothis{As before, make this object a desk so that we can have \studentA\ and \studentB\ helping you rearrange your room in your residence hall.  (Change the mass!)}{} are applied to an object with mass $2.0\unit{kg}$, then it will accelerate at the rate of
    \[ \vec a =\frac{\vec F_\mathrm{net}}{m} = \frac{+(9.0\unit N)\ihat}{2.0 \unit{kg}} = 4.5\unitfrac{N}{kg} \ihat = 4.5\unitfrac{kg \cdot m}{s^2 \cdot kg} \ihat \ = \ 4.5 \unitfrac{m}{s^2} \ihat \]
    which (recall \autoref{s:EOM}), after acting for $1.6\unit{s}$ on an object originally at rest, would result in a final speed of
    \[ v_f = (0\unitfrac{m}{s}) + (+4.5\unitfrac{m}{s^2})(1.6\unit{s}) = 7.2 \unitfrac{m}{s} \]
\end{sample}
We can do this same kind of procedure for the case when forces are in two dimensions.\dothis{Merge \autoref{ex:2Dfa} and \autoref{ex:2Dfa2}.  Also reference the example here.}
%
\begin{example}[p]
\fcolorbox{black}{yellow!10}{\begin{minipage}{4.925in} \setlength{\parsep}{3pt}
\caption{\label{ex:2Dfa} Moving a pushed box.}
A $2.0\unit{kg}$ mass is being pushed north with $5.0\unit{N}$ and east with $4.0 \unit{N}$.  What is the acceleration?
\color{blue}

\vspace{9pt}
\begin{minipage}{3.25in}
\autoref{ex:2Dforce}  (pg.~\pageref{ex:2Dforce}) already found the net force to be
\[ \vec F_\mathrm{net} = 4.0\unit{N} \ihat + 5.0\unit{N} \jhat \]
which is $F_\mathrm{net}=6.4\unit N$ at $51^\circ$ N of E.
What remains is to find the acceleration.  Since this is a vector, we can either find the components from the components of the net force
\end{minipage}
\hfill
\fbox{\begin{minipage}{1.5in}
\begin{FBD}{10}{15}{15}{40}{object}
\oneup{50}{$5\unit N$}{lightgray}
\oneri{40}{$4\unit N$}{lightgray}
\oneur{38}{48}{4}{5}{$F_\mathrm{net}$}{black}
\end{FBD}
\end{minipage}}
\vspace{2pt}
%
\[ \vec a =\frac{\vec F_\mathrm{net}}{m} = \frac{4.0\unit{N} \ihat + 5.0\unit{N} \jhat}{2.0\unit{kg}}  = 2.0\unitfrac{m}{s^2} \ihat + 2.5\unitfrac{m}{s^2} \jhat \]
or we can use the magnitude of the net force to find the magnitude of the acceleration
\[ a = \frac{6.4\unit N}{2.0\unit{kg}} = 3.2 \unitfrac{m}{s^2} \]
and know that the direction of the acceleration is the same as the acceleration of the net force: $51^\circ$ N of E.

You should notice that you can also use the components of the acceleration to find the magnitude and direction of the acceleration.
\magdir{+2.0\unitfrac{m}{s^2}}{+2.5\unitfrac{m}{s^2}}{a}{\sig{3.2}{0}{N}}{\theta}{\sig{51}{}{^\circ}}{N of E}

\color{black}
\end{minipage}}
\end{example}
%
\begin{example}[p]
\fcolorbox{black}{yellow!10}{\begin{minipage}{4.925in} \setlength{\parsep}{3pt}
\caption{\label{ex:2Dfa2} Moving a box pushed by three forces.}
A $2.0\unit{kg}$ mass is being pushed north with $5.0\unit{N}$ and east with $4.0 \unit{N}$.  What is the acceleration?
\color{blue}

\vspace{9pt}
\begin{minipage}{3.25in}
\autoref{ex:2Dforce2}  (pg.~\pageref{ex:2Dforce2}) already found the net force to be
\[ \vec F_\mathrm{net} = 3.8\unit{N} \ihat + 2.2\unit{N} \jhat \]
which is $F_\mathrm{net}=4.4\unit N$ at $31^\circ$ N of W.
What remains is to find the acceleration.  Since this is a vector, we can either find the components from the components of the net force
\end{minipage}
\hfill
\fbox{\begin{minipage}{1.5in}
\begin{FBD}{10}{15}{15}{40}{object}
%\oneup{50}{$5\unit N$}{black}
%\put(24,56){\vector(-1,2){22.36}}
%\put(0,102){\color{black} \tiny $5\unit N$}
\oneul{22}{48}{-1}{2}{$5 \unit N$}{lightgray}
\onedl{58}{22}{-5}{-2}{$6 \unit N$}{lightgray}
\oneri{40}{$4\unit N$}{lightgray}
\oneul{53}{22}{-5}{3}{$F_\mathrm{net}$}{black}
\end{FBD}
\end{minipage}}
\vspace{2pt}
%
\[ \vec a =\frac{\vec F_\mathrm{net}}{m} = \frac{3.8\unit{N} \ihat + 2.2\unit{N} \jhat}{2.0\unit{kg}}  = 1.9\unitfrac{m}{s^2} \ihat + 1.1\unitfrac{m}{s^2} \jhat \]
or we can use the magnitude of the net force to find the magnitude of the acceleration
\[ a = \frac{4.4\unit N}{2.0\unit{kg}} = 2.2 \unitfrac{m}{s^2} \]
and know that the direction of the acceleration is the same as the acceleration of the net force: $31^\circ$ N of W.

You should notice that you can also use the components of the acceleration to find the magnitude and direction of the acceleration.
\magdir{+1.9\unitfrac{m}{s^2}}{+1.1\unitfrac{m}{s^2}}{a}{\sigfrac{2.2}{}{m}{s^2}}{\theta}{\sig{31}{}{^\circ}}{N of W}

\color{black}
\end{minipage}}
\end{example}
%

\hypertarget{d:m=f/a}{In} \ref{se:netF-a}, we used the forces to find the acceleration.  It is also possible to use the forces to find the mass of an object, as follows:
\begin{sample}
\item\label{se:netF-m} If the forces in \ref{se:netFsub} are applied to an object with unknown mass and produce an acceleration of $3.2\unitfrac{m}{s^2}$, then what is the mass of the object?

    Naively, one might consider $\deq m = \frac{\vec F_\mathrm{net}}{\vec a}$, but it does not make mathematical sense to \hyperlink{d:dividevectors}{divide vectors}.  In this case, you \textit{must} consider the magnitudes of force and acceleration, knowing that their directions are the same. (We are \textit{not} ``cancelling'' the directions.)
    \[ m =\frac{F_\mathrm{net}}{a} = \frac{9.0\unit N}{3.2 \unitfrac{m}{s^2}} = \sigfrac{2.8}{1}{N \cdot s^2}{m} = 2.8\unitfrac{kg \cdot m \cdot s^2}{s^2 \cdot m} \ = \ 2.8 \unit{kg} \]
\end{sample}
\hypertarget{d:usesofF=ma}{Yet another example}\foreshadow{} of using this equation can be seen in many bathrooms.  The scale that people stand on uses a spring (introduced in \autoref{ss:scales} and discussed in detail in \autoref{s:springs}) to adjust the force provided until your acceleration is zero (placing you in equilibrium) and then tells you the force it needed to balance your weight.

It will be easier to visualize these ideas when we introduce the tool of a free-body diagram in \autoref{sss:FBD}.

\subsubsection{Equilibrium}\label{sss:equilibrium}\index{Equilibrium}\mmultireturn{\mmr{\ref{se:equi}}, \mmr{\hyperlink{d:atrestinmotion}{Newton's first law}}, \mmr{\autoref{d:Fgball}}}

This word can be traced back to Latin and Old English with the prefix \textit{equi-} for \textbf{equal} and the root \textit{libra} referring to a \textbf{pair of scales, as in a balance}, such as those depicted in images of the astronomical constellation Libra.  When the scales are equal, they are in equilibrium.  Since the second law asks us to calculate the sum of the forces acting on an object, one of the primary questions is to determine if those forces balance each other.  In \ref{se:netFadd} and \ref{se:netFsub}, the forces are not balanced, the object ``is not in equilibrium'', and it will be accelerated in a particular direction.  In the \ref{se:equi}, the forces are balanced, the object ``is in equilibrium'', and it will \textit{not change} its velocity (in accord with the first law).
\important{An object in equilibrium has $\vec F_\mathrm{net} = 0\unit N$ and $\vec a =0$.}

\subsubsection{How the Laws Work Together}\label{sss:NIItogether}\mautoreturn{sss:NItogether}

When forces act on an object, Newton's second law applies, so we usually start with the second law.  If those forces combine to give a net force of zero, such that the object is in equilibrium, then Newton's first law applies.  If we also care about the person or thing pushing, then the third law also applies.

To better understand how the first and second laws work together, \autoref{irl:NI} (pg.~\pageref{irl:NI}) provides some activities that you can do or consider in order to think about the patterns you can see when you are or aren't pushing on objects.  \autoref{cyoa:NI} (pg.~\pageref{cyoa:NI})
%
\begin{adventure}[bhpt]\fcolorbox{black}{blue!10}{\begin{minipage}{4.925in}\caption{\label{cyoa:NI} Out of gas}
On a long road trip with your friend \studentB\index{\studentB}, your car starts to sputter as it runs out of gas shortly before arriving in a new town.  You see a sign for a gas station in the distance and have to decide what to do.  You and \studentB\ can think of three options.
\begin{CYOA}
\item\label{c:parkandwalk} Pull over, park the car, walk to the gas station, buy a gas can, fill it up, carry it back to the car, and drive on!   If you follow this plan, then read \ref{a:parkandwalk}.
\item\label{c:coastindrive} Leave the car in drive, continue holding the gas-pedal down until there is absolutely no gas, and hope against all hope that you get the car to the gas station so that nobody needs to carry a heavy gas can. If you follow this plan, then read \ref{a:coastindrive}.
\item\label{c:coastinneutral} Speed up to just over the speed limit, put the car in neutral, turn on your blinking hazard-lights, coast as far as you can possibly coast, and hope against all hope that you get the car to the gas station so that nobody needs to carry a heavy gas can. If you follow this plan, then read \ref{a:coastinneutral}.
\end{CYOA}
\autoreturn{sss:NIItogether}
\end{minipage}}
\end{adventure}
%
will help you think through some of the consequences of the first and second law.  When you are ready to solve some problems, you can jump to \autoref{s:NewtonExamples}, but some of those examples will also reference Newton's third law.

\subsection{Translating Newton's Third Law: Action \& Reaction}\label{ss:NIII}\mmultireturn{\mmr{\hyperlink{d:Newtonahead}{how to describe forces}}, \mmr{\autoref{ex:braced}}, \mmr{\autoref{ex:unbraced}}}

%\important{} is not designed to start a new paragraph
\ \vspace{-12pt}
%\begin{quote}
\important{\textbf{Newton's Third Law}\index{Newton!Third Law}: $\overbrace{\mbox{For every force acting}}^{\mbox{\scriptsize ``For every action''}}$ \textit{on} one object \textit{by} an \underline{other object}, there is an equal-in-magnitude reaction-force acting \textit{on} the \underline{other object} in the opposite direction.}
%\end{quote}
This law is often shortened to ``For every action, there is an equal and opposite reaction.''  The statement given above is meant to emphasize several points:
\begin{itemize}\itemsep 0pt
\item These ``actions'' are specifically forces.
\item Forces are an interaction in which the acting force is \underline{on one object} by another and necessitates that there is a reaction force on the other object by the one.  That is to say, an object cannot feel a force without also exerting a force back on the other object.
\end{itemize}
Another way to say this is that all forces come in action/reaction pairs that necessarily have equal magnitude and opposite direction and necessarily act on different objects.
This law is the force-version of\Foreshadow{\protect{\hyperref[s:conservemom]{Conservation of momentum}}}{} the statement of the conservation of momentum, which will be discussed in \autoref{c:momentum}.

\hypertarget{d:NIIIbracing}{Let's take this apart} and connect it to your daily experiences.  Students of physics will often see the terms action and reaction and connect it to the way humans react to the actions of their friends.  However, this implies a causal response that is not true for Newton's forces.  That is to say, this is not a ``revenge law'' whereby if you push on me, then I will choose to push you back.  Instead, it is expressing that forces are intrinsically interactions between a pair of objects.  When you push on me, I am -- independent of my choosing to do so -- necessarily pushing back on you.  But, you might say, ``if that were true, then why am I able to sneak up on you and push you over without falling over myself?''  Well, you can think about how that works by reading \autoref{cyoa:NIII} (pg.~\pageref{cyoa:NIII}). After we introduce the tool of a free-body diagram in the next subsection, you can also explore this idea by comparing \autoref{ex:braced}  (pg.~\pageref{ex:braced}) to \autoref{ex:unbraced}  (pg.~\pageref{ex:unbraced}).

\subsubsection{The Free-Body Diagram (FBD)}\label{sss:FBD}\mmultireturn{\mmr{\ref{se:equi}}, \mmr{\hyperlink{d:usesofF=ma}{uses of $F=ma$}}}\index{Free-Body Diagrams}

In the more interesting situations where there are several forces acting, it can be easy to lose track of what is pushing whom where.  In order to better organize our information and direct our attention, we can make use of \textit{free-body diagrams}.  The basic idea is to make a diagram for each individual object that we care about in a given situation, and \textit{free} from the overall picture.  This allows us to identify the forces acting on a single object (relevant for Newton's second law) and more easily pair them with third-law pairs that act on different objects.

To see how this works, the next simple example will build on the previous simple examples to help us consider not only the (2nd law) forces on the object, but also the (3rd law) forces on the people doing the pushing and pulling.
\begin{sample}
\item\label{se:FBD-AB}\mautoreturn{f:firstFBD} As you revisit \ref{se:netFadd}\dothis{As before, make this object a desk so that we can have \studentA\ and \studentB\ helping you rearrange your room in your residence hall.  (Change the mass!  But, it is still nice to be moving a small mass so that it has a large acc, and the people have small acc.)}, imagine that \studentA\index{\studentA} is exerting $\vec F_1 = +5.0\unit{N}\ihat$ and \studentB\index{\studentB} is exerting $\vec F_2 = +4.0\unit{N}\ihat$.  \ref{se:netF-a} showed how the object moved (because Newton's second law focuses on the object to which the forces are applied).  Newton's third law tells us about the interaction between objects and, from this, we can say the following. In order to better describe the situation, let's assume \studentA\ is to the left of the object, pushing it to the right, and \studentB\ is to the right of the object, pulling it to the right.  See \autoref{f:firstFBD} on page~\pageref{f:firstFBD}.
    \begin{enumerate}
    \item Since \studentA\ exerts a force of $5.0\unit{N}$ to the right $(+\ihat)$ on the object, the third law reminds us that the object exerts a force of $5.0\unit{N}$ to the left  $(-\ihat)$ on \studentA.  Since \studentA\ has a mass of \massA, the second law reminds us that \heA\ is accelerated at the rate of
        \[ \vec a_1 = \frac{-5.0\unit{N} \ihat}{\massA} = -0.0\sigfrac{58}{8}{m}{s^2}\ihat = -5.9\ten{-2}\unitfrac{m}{s^2} \ihat\]
        which is to the left with a small enough value that it is easy for \himA\ to brace against.   Even if \heA\ doesn't brace, if \heA\ starts from rest, \heA\ will only be moving \\
        $v_{1f} = (0\unitfrac{m}{s})+(-0.0\sigfrac{58}{8}{m}{s^2})(1.6\unit s) = -0.0\sigfrac{94}{1}{m}{s}$.
    \item Since \studentB\ exerts a force of $4.0\unit{N}$ to the right on the object, the third law reminds us that the object exerts a force of $4.0\unit{N}$ to the left on \studentB. Since \studentB\ has a mass of \massB, the second law reminds us that \heB\ is accelerated at the rate of
        \[ \vec a_2 = \frac{-4.0\unit{N} \ihat}{\massB} = -0.0\sigfrac{53}{3}{m}{s^2}\ihat = -5.3\ten{-2}\unitfrac{m}{s^2}\]
        which is also to the left with a small enough value that it is easy for \himB\ to brace against.  Even if \heB\ doesn't brace, if \heB\ starts from rest, \heB\ will only be moving \\
        $v_{2f} = (0\unitfrac{m}{s})+(-0.0\sigfrac{53}{3}{m}{s^2})(1.6\unit s) = -0.0\sigfrac{85}{3}{m}{s}$.
    \end{enumerate}
\end{sample}
There are \hypertarget{d:FBD-AB}{a couple of important aspects} to take away from \ref{se:FBD-AB}:, especially as it builds on \ref{se:netFadd} (which told us about the forces on the object) and \ref{se:netF-a} (which told us
\begin{minipage}{3.25in}
about how the object moved).  The earlier examples were relevant to the 2nd law and only affected the object itself.  This information is captured in the middle free-body diagram of \autoref{f:firstFBD} and reproduced here.
\end{minipage}
\hfill
\fbox{\begin{minipage}{1.5in}
\begin{FBD}{10}{15}{15}{10}{object}
\twori{50}{$5\unit N$}{black}{40}{$4\unit N$}{black}
\end{FBD}
\end{minipage}}

As indicated above, the free-body diagram also helps us visualize the third-law (action/reaction) force pairs.  By reproducing the following image\index{Free-Body Diagrams!Images} from \autoref{f:firstFBD} here and adding color, we can see that the {\color{green} green forces} form an action-reaction pair and separately the {\color{blue} blue forces} form
\newline
\begin{minipage}{\textwidth}
\fbox{\begin{minipage}{1.5in}
\begin{FBD}{10}{25}{15}{10}{\studentA}
\onele{50}{$5\unit N$}{green}
\end{FBD}
\end{minipage}}
\hfill
\fbox{\begin{minipage}{1.5in}
\begin{FBD}{10}{15}{15}{10}{object}
\twori{50}{$5\unit N$}{green}{40}{$4\unit N$}{blue}
\end{FBD}
\end{minipage}}
\hfill
\fbox{\begin{minipage}{1.5in}
\begin{FBD}{10}{20}{15}{10}{\studentB}
\onele{40}{$4\unit N$}{blue}
\end{FBD}
\end{minipage}}
\end{minipage}
an action-reaction pair.  None of these three objects are in equilibrium.

If we now \hypertarget{d:equi}{do the same thing}\index{Free-Body Diagrams!Images} with the \ref{se:equi}, but let \studentC\index{\studentC} be the person on the left {\color{green} pushing to the right} and \studentD\index{\studentD} be the person on the right {\color{blue} pushing to the left}, then we see that while the {\color{green} green forces} form an action-reaction pair showing the third-law interaction between \studentC\ and the object and while the {\color{blue} blue forces} form an action-reaction pair showing the third-law interaction between \studentD\ and the object, it
%\newline
\begin{minipage}{\textwidth}
\fbox{\begin{minipage}{1.5in}
\begin{FBD}{10}{15}{15}{10}{\studentC}
\onele{30}{$3\unit N$}{green}
\end{FBD}
\end{minipage}}
\hfill
\fbox{\begin{minipage}{1.5in}
\begin{FBD}{10}{15}{15}{10}{object}
\oneri{30}{$3\unit N$}{green}
\onele{30}{$3\unit N$}{blue}
\end{FBD}
\end{minipage}}
\hfill
\fbox{\begin{minipage}{1.5in}
\begin{FBD}{10}{15}{15}{10}{\studentD}
\oneri{30}{$3\unit N$}{blue}
\end{FBD}
\end{minipage}}
\end{minipage}
is the combination of the {\color{blue} blue} and the {\color{green} green} forces, which only act on the object itself, that coincidentally cancel to leave the object in (second law) equilibrium.  In these images, \studentC\ and \studentD\ are \textit{not} in equilibrium.

To say this more specifically, the forces within one free-body diagram are described by Newton's second law. They do get added together to form the net-force (which is to say that we can add the object's green force to the object's blue force). They are able to cancel each other if they \textit{happen to} be equal in magnitude and opposite in direction. Finally, they will determine how that specific object accelerates.  On the other hand, Newton's third law describes any specific pair of forces that interact between free-body diagrams (each colored pair); they \textit{will necessarily be} equal in magnitude and opposite in direction, but they cannot be canceled because they cannot be added because they are on different objects.

\begin{figure}
\hrule\hrule
\caption{\label{f:firstFBD} A couple of people push a box.}\index{Free-Body Diagrams!Images}
First we will draw a picture of the situation described by \ref{se:FBD-AB}, which builds on \ref{se:netFadd}.  \studentA\index{\studentA} stands to the left of the object and exerts a force (pushes) to the right.  \studentB\index{\studentB} stands to the right of the object and exerts a force (pulls) to the right.  Both of these forces are \textit{on} the object and, by Newton's second law cause it to accelerate (as described in \ref{se:netF-a}).  By Newton's third law, we can learn about the forces on \studentA\ and \studentB.

\noindent
{}\hfill
\begin{minipage}{3.5in}
\begin{picture}(200,100)(-30,-25)
% Dimensions and offset: (width,height)(x offset,y offset)
% Insert picture commands (\line,\circle, etc...) here:
\put(0,0){\line(1,0){200}}
\put(60,2){\line(1,0){60}}
\drawbox{30}{1}{20}{50} %\studentA
\drawbox{50}{25}{18}{5} %\studentA's arms
\drawbox{70}{3}{20}{30} % object
\drawbox{150}{1}{20}{40} %\studentB
\drawbox{134}{25}{16}{5} %\studentB's arms
\put(90,27.5){\oval(2,2)[r]}
\put(91,27.5){\line(1,0){43}}
\put(30,53){\scriptsize \studentA}
\put(70,35){\scriptsize object}
\put(150,43){\scriptsize \studentB}
\put(60,-12){\begin{minipage}{60pt}
\scriptsize The object is on a sheet of ice.
\end{minipage}}
\end{picture}
\end{minipage}
\hfill {}

\noindent
Now we will draw a free-body diagram for each individual.  Notice that each \textit{free-body} diagram is on its own, free from the rest of the picture.  These diagrams will be discussed further in the \protect{\hyperlink{d:FBD-AB}{discussion about \ref*{se:FBD-AB}}}.

\noindent % \textwidth default is 5in for a book
\fbox{\begin{minipage}{1.5in}
\begin{FBD}{10}{25}{15}{10}{\studentA}
\onele{50}{$5\unit N$}{black}
\end{FBD}
\raggedright
Because \studentA\ pushes on the object to the right, \studentA\ feels a force \textit{on} \himA\ \textit{by} the object towards the left.
\end{minipage}}
\hfill
\fbox{\begin{minipage}{1.5in}
\begin{FBD}{10}{15}{15}{10}{object}
\twori{50}{$5\unit N$}{black}{40}{$4\unit N$}{black}
\end{FBD}
\raggedright
The object is in the middle.  Recall from \protect{\ref{se:netFadd}} that the $F_\mathrm{net} = 9.0 \unit N$ on this object.
\end{minipage}}
\hfill
\fbox{\begin{minipage}{1.5in}
\begin{FBD}{10}{20}{15}{10}{\studentB}
\onele{40}{$4\unit N$}{black}
\end{FBD}
\raggedright
Because \studentB\ pulls on the object to the right, \studentB\ feels a force \textit{on} \himB\ \textit{by} the object towards the left.
\end{minipage}}

\noindent
It turns out that this is a little oversimplified.  When we get to \protect{\autoref{s:Fg}} and \protect{\autoref{s:FN}}, we will see that we have to include a downwards force of gravity and an upwards support force.  This will be explained in \protect{\autoref{f:firstFBDupdate}} on page~\protect{\pageref{f:firstFBDupdate}}.
\flushright
\multireturn{\mmr{\autoref{ex:braced}}, \mmr{\autoref{ex:unbraced}}, \mmr{\hyperlink{d:rope.net}{rope-tension}}}
\hrule\hrule
\end{figure}

\section{Examples} \label{s:NewtonExamples}\mautoreturn{sss:NIItogether}

Next, we can consider a simple interactive example that is intended to help you think about how you know a force is acting.
%
\begin{sample}
\item\label{IQ:holdbook} You hold a book a little above your desk.  When you let go, it falls and then hits your desk.
    \begin{enumerate}
    \item While you are holding it, it has no acceleration.  Are there forces acting on it?  \YN{A:hbf}{A:hbnof}
    \item While you are holding it, is it in equilibrium?  \YN{A:true1}{A:false1}
    \item After you let go and while the book falls, it accelerates downwards.  Are there forces acting on it?  \YN{A:falls}{A:falls}
    \item While it is hitting the desk, is it accelerating?  \YN{A:hitY}{A:hitN}
    \item After it has landed and is sitting on the desk, is it in equilibrium? \YN{A:landedY}{A:landedN}
    \item After it has landed and is sitting on the desk, how many forces are acting on it? \THREE{Zero}{One}{Two}{A:zero}{A:one}{A:two}
    \end{enumerate}
\end{sample}
%
Next, we can \hypertarget{d:irlNI}{consider} pushing an object across the floor in \autoref{irl:NI} (pg.~\pageref{irl:NI}) to get a different sense of observations we can make that help us recognize patterns that are due to forces we might not have thought to look for.
%
\begin{reallife}[hp]
\hspace{-.2in}
\fcolorbox{black}{green!10}{\begin{minipage}{5.29in} \center
\caption{\label{irl:NI} Pushing an Object Across the Floor}
\begin{realtable}
\dna{Push a chair across a carpet floor}
    {When you stop pushing, it stops moving.}
    {Does force cause motion? \ref{A:chair1}}
\dna{Push a chair across a tile floor}
    {When you stop pushing, it probably stops moving.}
    {Does force cause motion? \ref{A:chair2}}
\dna{Push a chair \textit{with wheels} across a tile floor, with some strength, then let it go.}
    {What happens when you stop pushing? \ref{A:chair3}}
    {If force causes motion, why does the chair move after you stop touching it?  \ref{A:chair4}}
\dna{Push a chair \textit{with wheels} across a tile floor, change your behavior after you let it go.}
    {Do your actions when you are not touching the chair have \textit{any} impact on the chair? \ref{A:chair5}}
    {Is it possible that there is a ``residual effect'' that you have on the chair after letting it go? \ref{A:chair6}}
\multidna{Newton's First Law says that if you give the chair a velocity, it should keep that velocity.}
\dna{Repeat the first three suggestions}
    {Correlate the interaction-with-the-ground to the motion-after-you-push-and-release}
    {Is there a force that the chair feels after you release it?  \ref{A:chair7}}
\multidna{Newton's Second Law says that a net force will change the velocity.}
\dna{Push a chair gently across the floor}
    {A constant force (balanced by the force of friction) will move at a constant speed}
    {What if there were no friction? \ref{A:chair8}}
\dna{Push a chair forcefully across the floor}
    {A constant force (stronger than the force of friction) will accelerate the object away from your push}
    {Can you list surfaces that are essentially frictionless?}
\end{realtable}
\begin{minipage}{4.925in}
Notice in each case that you are not the only thing interacting with the chair.  The floor is also interacting with the chair.  The floor exerts a \hyperref[s:Ff]{force of friction} on the chair.  So, when you interpret how your force causes the chair to move, you \textit{must} also account for the interaction with the floor in your expectations.  We can minimize the effect of friction, by modifying the floor surface.  If you have ever driven on ice and felt out of control, you might have begun to develop your Newtonian intuition.
\flushright\vspace{-12pt}
\multireturn{\mmr{\autoref{sss:NIItogether}}, \mmr{\hyperlink{d:irlNI}{\autoref*{s:NewtonExamples} reference to \autoref*{irl:NI}}}}
\end{minipage}
\end{minipage}}
\end{reallife}
%
Building on that, it is useful to also consider how human beings behave when they are pushing or getting pushed.  Because people have \textit{intention} in their actions, we subconsciously balance ourselves and we don't always recognize that we are \hypertarget{d:cyoaNIII}{doing it}.  \autoref{cyoa:NIII}  (pg.~\pageref{cyoa:NIII}) provides an interactive storyline that starts to show some of the patterns that can lead to a recognition of how we balance ourselves.
%
\begin{adventure}[bpht]
\fcolorbox{black}{blue!10}{\begin{minipage}{4.925in}
\caption{\label{cyoa:NIII} The Town Bully}
\studentZ\index{\studentZ} is the town bully.  One day, he spies a biology student, \studentC\index{\studentC}, minding \hisC\ own business studying an interesting ecological phenomenon.  At the same time, you are standing across the street chatting with your friend \studentD\index{\studentD}, who happens to be taking a psychology class.  \studentD\ has been quite fascinated lately with watching the way others interact and points out the way \studentZ\ is menacingly approaching the unsuspecting \studentC.  You both predict that \studentZ\ is going to push \studentC\ over.  \studentD\ is mesmerized by the psychological effects and you, having just learned about Newton's laws, are excited to see if this action does indeed produce a reaction.
\begin{CYOA}
\item\label{c:one} If you watch the way \studentC\ is standing before, during, and after \studentZ\ pushes \himC, then read \ref{a:NIIIaction}.
\item\label{c:two} If you watch the way \studentZ\ is standing before, during, and after \heZ\ pushes \studentC, then read \ref{a:NIIIreaction}.
\item\label{c:three} If, on the other hand, you shout a warning to \studentC\ and a criticism to \studentZ, trying to keep the incident from becoming violent, then read \ref{a:NIIIconcern}.
\end{CYOA}
\flushright
\multireturn{\mmr{\hyperlink{d:NIIIbracing}{the discussion of action-reaction forces}}, \mmr{\hyperlink{d:cyoaNIII}{\autoref*{s:NewtonExamples} reference to \autoref*{cyoa:NIII}}}, \mmr{\autoref{ex:braced}}, \mmr{\autoref{ex:unbraced}}}
\end{minipage}}
\end{adventure}
%

\begin{example}[p]
\fcolorbox{black}{yellow!10}{\begin{minipage}{4.925in}\setlength{\parskip}{3pt}
\caption{\label{ex:braced} \studentZ\index{\studentZ} intentionally braces when pushing \studentC\index{\studentC}.}
(To better understand \hyperref[ss:NIII]{Newton's third law}, you should compare this example to \autoref{ex:unbraced}  [pg.~\pageref{ex:unbraced}].)
\begin{quote}
\studentZ\index{\studentZ}, the \hyperref[cyoa:NIII]{town bully} (with $m_Z=\massZ$), decides to vent \hisZ\ frustration on \studentC\index{\studentC}\ for all the times that \studentC\ makes \studentZ\ look bad in class.  While \studentC\ ($m_C=\massC$) has \hisC\ back turned, \studentZ\ walks up, leans in, and shoves \studentC\ with a force of $\vec F_{C,Z} = 215\unit{N}\ihat$.  How does \underline{\studentZ}{} accelerate during this exchange?
\end{quote}
%
%\begin{quote}
%Aside: Newton's second law tells us how this affects \studentC. See \ref{se:netF-a} and homework problem \ref{hmwk:pushbrace}.
%\end{quote}
%\noindent
\textbf{What do we know?}  As usual, it is convenient to start with a picture to help decide on the appropriate coordinate system.
We can also list
\\[2pt]
\begin{minipage}{2.6in}
the information that we know.
We know $m_Z$, which is useful for relating $F_{Z,\mathrm{net}}$ to $a_Z$.
We know $m_C$, which is useful for relating $F_{C,\mathrm{net}}$ to $a_C$.  (This is not asked for, but is asked in homework problem \ref{hmwk:pushbrace}.)
We know $F_{C,Z}$, how hard \studentZ\ pushes on \studentC.
\end{minipage}
\hfill
\begin{minipage}{150pt}
\begin{picture}(150,90)(-30,-25)
% Dimensions and offset: (width,height)(x offset,y offset)
% Insert picture commands (\line,\circle, etc...) here:
\drawbox{-10}{-20}{120}{20}  % Earth
\drawbox{25}{1}{20}{50} %\studentZ
\drawbox{45}{35}{18}{5} %\studentZ's arms
 %\studentZ's legs
    \put(24,19){\line(0,-1){6}}
    \put(24,19){\line(-1,-2){9}}
    \put(15,1){\line(1,0){4}}
    \put(19,1){\line(1,2){6}}
\drawbox{65}{1}{20}{45} %\studentC
\put(25,53){\scriptsize \studentZ}
\put(65,48){\scriptsize \studentC}
\put(40,-15){\scriptsize Earth}
\put(-40,24){\begin{minipage}{58pt}
\color{blue} \scriptsize \studentZ\ braces \himselfZ. \hfill $\searrow$
\end{minipage}}
\end{picture}
\end{minipage}
%\hfill {}
We also know that \studentC\ is not bracing \himselfC\ (because \heC\ ``has \hisC\ back turned'') so he only feels one force, and that \studentZ\ is bracing \himselfZ\ (because the problem states that \heZ\ ``leans in'') so he exerts multiple forces.

\textbf{What do we want to know?}  We want to know about the forces acting on \studentZ, in order to find  $F_{Z,\textrm{net}}$ and therefore $a_Z$.

\textbf{How are these related?}  First, since \studentZ\ exerts a force on \studentC, Newton's third law tells us that \studentZ\ feels a force of \mbox{$F_{Z,C}=-215 \unit{N}\ihat$}.
Second, because \studentZ\ \textit{knew} \heZ\ was going to feel this reaction force, \heZ\ compensates by bracing \himselfZ.  This means \heZ\ chooses to exert a force of $215\unit{N}$ on the Earth in the $-\ihat$ direction, probably by putting one leg behind \himselfZ\ and pushing the ground backwards with \hisZ\ foot.  Newton's third law then tells us that \studentZ\ feels a force of \mbox{$F_{Z,C}=+215 \unit{N}\ihat$} from the ground.

{}\hfill {\footnotesize \autoref*{ex:braced} continued on next page\ldots}
\end{minipage}}
\end{example}
\begin{example}[p]
\fcolorbox{black}{yellow!10}{\begin{minipage}{4.925in}\setlength{\parskip}{3pt}
{\footnotesize \autoref*{ex:braced} continued from previous page\ldots}

\textbf{Free-Body Diagrams:}  We are told of the force on \studentC.  We are told that \studentZ\ braces \himselfZ, which implies the force on the Earth. Newton's third law then helps us recognize the forces on \studentZ.  (Recall the \hyperlink{d:interaction}{``on-by'' notation}.)

\noindent % \textwidth default is 5in for a book
\fbox{\begin{minipage}{2.25in}
\begin{FBD}{10}{25}{15}{10}{\studentZ}
\onele{50}{$F_{Z,C}=215\unit N$}{black}
\oneri{50}{$F_{Z,E}=215\unit N$}{blue}
\end{FBD}
\vspace{-10pt}
\raggedright
\studentZ\ is pushed by \studentC\ to the left.  \studentZ\ is pushed to the right by the Earth.
\end{minipage}}
\hfill
\fbox{\begin{minipage}{2.25in}
\begin{FBD}{10}{23}{15}{10}{\studentC}
\oneri{50}{$F_{C,Z}=215\unit N$}{black}
\end{FBD}
\vspace{-10pt}
\raggedright
\studentC\ feels \studentZ\ push to the right.
\end{minipage}}
% \\
\fbox{\begin{minipage}{4.75in}
\begin{FBD}{60}{10}{15}{10}{Earth}
\onele{50}{$F_{E,Z}=215\unit N$}{blue}
\end{FBD}
\vspace{-10pt}
\raggedright
Earth feels a force by \studentZ\ to the left.
\end{minipage}}

\textbf{Concepts to Consider:}  Newton's third law guarantees that the action-reaction force pairs, such as $F_{Z,C}$ and $F_{C,Z}$ or $F_{Z,E}$ and $F_{E,Z}$, are equal and opposite.  There is no such guarantee on $F_{Z,C}$ and $F_{Z,E}$.  These are equal because \studentZ\ chose to make $F_{C,Z}$ and $F_{E,Z}$ equal.  \HeZ\ pushed on the two others in equal amounts so that the reaction forces that act on \himZ\ will balance for Newton's \textit{second} law so that \hisZ\ acceleration would be zero.

\textbf{Solution to the example:}  After using Newton's third law to find the forces on \studentZ, we can use Newton's second law to find \hisZ\ acceleration:
\[ a_Z = \frac{F_{Z,\mathrm{net}}}{m_Z} = \frac{\left[ \vec F_{Z,C} + \vec F_{Z,E} \right]}{\massZ} = \frac{\left[ \left( -215\unit N \ihat \right) + \left( +215\unit N \ihat \right) \right]}{\massZ} = 0 \unitfrac{m}{s^2}  \]

%\begin{quote}
\textbf{Aside:} This example only considers the left-right forces that act in order to make a point about our intuition regarding forces we intend to apply.  Please consider how \protect{\autoref{f:firstFBDupdate}} updates \autoref{f:firstFBD} to make yourself aware of the other forces that are acting here, but are being ignored.
%\end{quote}

\linkreturn[action-reaction]{d:NIIIbracing}
\end{minipage}}
\end{example}

\begin{example}[p]
\fcolorbox{black}{yellow!10}{\begin{minipage}{4.925in}\setlength{\parskip}{3pt}
\caption{\label{ex:unbraced} \studentD\index{\studentD} does not brace \himselfD\ when pushing \studentC\index{\studentC}.}
(To better understand \hyperref[ss:NIII]{Newton's third law}, you should compare this example to \autoref{ex:braced}  [pg.~\pageref{ex:braced}].)
\begin{quote}
In the lab room one day, while waiting for the instructor, \studentD\index{\studentD} (who has a mass of $m_D=\massD$) decides to try a physics experiment to test Newton's third law.  \HeD\ politely asks \hisD\ lab partner, \studentC\index{\studentC} ($m_C=\massC$), to turn \hisC\ back while \heD\ squares his feet underneath \himselfD\ and pushes with a force of $\vec F_{C,D} = 215\unit{N}\ihat$.  Despite the experience of \autoref{ex:braced} (as told in \autoref{cyoa:NIII}  [pg.~\pageref{cyoa:NIII}]), \studentC\ reluctantly agrees.  How does \underline{\studentD}{} accelerate during this exchange?
\end{quote}
%
%\begin{quote}
%Aside: Newton's second law tells us how this affects \studentC. See \ref{se:netF-a} and homework problem \ref{hmwk:pushbrace}.
%\end{quote}
%\noindent
\textbf{What do we know?}  As usual, it is convenient to start with a picture to help decide on the appropriate coordinate system.
We can also list
\\[2pt]
\begin{minipage}{2.6in}
the information that we know.
We know $m_D$, which is useful for relating $F_{D,\mathrm{net}}$ to $a_D$.
We know $m_C$, which is useful for relating $F_{C,\mathrm{net}}$ to $a_C$.  (This is not asked for, but is asked in homework problem \ref{hmwk:pushbrace}.)
We know $F_{C,D}$, how hard \studentD\ pushes on \studentC.
\end{minipage}
\hfill
\begin{minipage}{150pt}
\begin{picture}(150,90)(-30,-25)
% Dimensions and offset: (width,height)(x offset,y offset)
% Insert picture commands (\line,\circle, etc...) here:
\drawbox{-10}{-20}{120}{20}  % Earth
\drawbox{25}{1}{20}{40} %\studentD
\drawbox{45}{25}{18}{5} %\studentD's arms
\drawbox{65}{1}{20}{45} %\studentC
\put(25,43){\scriptsize \studentD}
\put(65,48){\scriptsize \studentC}
\put(40,-15){\scriptsize Earth}
\put(-40,24){\begin{minipage}{58pt}
\color{blue} \raggedright \scriptsize \studentD\ does not brace \himselfD. \\\hfill $\searrow$
\end{minipage}}
\end{picture}
\end{minipage}
%\hfill {}
\\[2pt]
We also know that neither person is bracing for the push. So, both \studentC\ and \studentD\ each only feel one force.

\textbf{What do we want to know?}  We want to know about the forces acting on \studentD, in order to find  $F_{D,\textrm{net}}$ and therefore $a_D$.

\textbf{How are these related?}  First, since \studentD\ exerts a force on \studentC, Newton's third law tells us that \studentD\ feels a force of \mbox{$F_{D,C}=-215 \unit{N}\ihat$}.
Second, unlike \studentZ\ in \autoref{ex:braced}  (pg.~\pageref{ex:braced}), \studentD\ chooses not to exert a force on the Earth in the $-\ihat$ direction.

\textbf{Free-Body Diagrams:}  We again draw free-body diagrams:

\noindent % \textwidth default is 5in for a book
\fbox{\begin{minipage}{2.25in}
\begin{FBD}{10}{20}{15}{10}{\studentD}
\onele{50}{$F_{D,C}=215\unit N$}{black}
\end{FBD}
\vspace{-10pt}
\raggedright
\studentD\ is pushed by \studentC\ to the left.
\end{minipage}}
\hfill
\fbox{\begin{minipage}{2.25in}
\begin{FBD}{10}{23}{15}{10}{\studentC}
\oneri{50}{$F_{C,D}=215\unit N$}{black}
\end{FBD}
\vspace{-10pt}
\raggedright
\studentC\ feels \studentD\ push to the right.
\end{minipage}}

{}\hfill {\footnotesize\autoref*{ex:unbraced} continued on next page\ldots}
\end{minipage}}
\end{example}
\begin{example}[p]
\fcolorbox{black}{yellow!10}{\begin{minipage}{4.925in}\setlength{\parskip}{3pt}
{\footnotesize \autoref*{ex:unbraced} continued from previous page\ldots}

\textbf{Concepts to Consider:}  Newton's third law guarantees that the action-reaction force pairs, $F_{D,C}$ and $F_{C,D}$, are equal and opposite.  Because these forces are not on the same person, we cannot add these forces.  Newton's second law will then indicate how each person accelerates.

\textbf{Solution to the example:}  After using Newton's third law to find the forces on \studentD, we can use Newton's second law to find \hisD\ acceleration:
\[ a_D = \frac{F_{D,\mathrm{net}}}{m_D} = \frac{\left[ \vec F_{D,C} \right]}{\massD} = \frac{\left[ \left( -215\unit N \ihat \right) \right]}{\massD} = -\sigfrac{2.68}{75}{m}{s^2} \ihat \]

%\begin{quote}
\textbf{Aside:} This example only considers the left-right forces that act in order to make a point about our intuition regarding forces we intend to apply.  Please consider how \protect{\autoref{f:firstFBDupdate}} updates \autoref{f:firstFBD} to make yourself aware of the other forces that are acting here, but are being ignored.
%\end{quote}
\flushright
\multireturn{\mmr{\hyperlink{d:NIIIbracing}{the discussion of action-reaction forces}}, \mmr{\autoref{ex:braced}}}
\end{minipage}}
\end{example}

\section{Summary and Homework}

\subsection{Summary of Concepts and Equations}

This chapter introduced the way physicists describe forces.  The concept of force encodes how objects interact.
After reading this chapter, you should be comfortable responding to the following questions or comments.
Unlike the other links in this book, if you follow the links in this summary section, there is no link to return to this page.  (This is on purpose to encourage you to answer these points without following these links.)
\begin{itemize}
\item State Newton's Laws. \hyperlink{sum:Newton'sLaws}{(Answer)}
\item How is the unit of Newton related to the fundamental units of the SI system?  \hyperref[sss:unit-N]{(Answer)}
\item How do you know when a system is in equilibrium? \hyperref[sss:equilibrium]{(Answer)}
\item You should know how to draw a free-body diagram.  \hyperref[f:firstFBD]{(Example)}
\end{itemize}

\subsection*{Conceptual Questions}\dothis{Add more conceptual questions}
%\vspace{-24pt}
\begin{enumerate}
\item In order to climb a tree, you reach up and grab a branch and pull.  Most people refer to this as ``pulling yourself up.'' In terms of Newton's third law, describe what is happening in more technical terms.
\item Some cars have a ``cruise-control'' feature that keeps your speed constant as you drive down the highway.  (a) If you are driving due north with the cruise-control on, are you in equilibrium?  (b) If, instead, you have the cruise-control set while you are following the road around a gradual curve of the road as it follows the shore of a lake, then are you in equilibrium?  (c) In both cases, how can you tell if you are in equilibrium?
\end{enumerate}
\subsection*{Problems}\dothis{Add more variety of problems.}
%\vspace{-24pt}
\begin{enumerate}
 \item\label{hmwk:pushbrace} If \studentZ, with $m_Z=\massZ$, braces \himselfZ\ (so that he does not accelerate) and pushes \studentC\ ($m_C=\massC$) with a force of $\vec F_{C,Z} = 215\unit{N}\ihat$, find the following:
\begin{enumerate}
    \item What is the acceleration of \studentC?  \answer{\mbox{$\deq\vec a_C = \frac{215 \unit{N}\ihat}{\massC} = \sigfrac{2.38}{9}{m}{s^2} \ihat$.}}
    \item What net force does \studentZ\ feel? \answer{$F_{Z,\mathrm{net}}=0\unit N$}
    \item If \studentZ\ braces \himselfZ\ against the Earth, then what must that bracing force be?  \answer{$\vec F_{E,Z} = -215\unit{N}\ihat$}
    \item What are the individual forces that \studentZ\ feels? \answer{$F_{Z,C}=-215\unit N \ihat$ and $F_{Z,E}=215\unit N \ihat$}
    \item What is the acceleration of the Earth?  \answer{\mbox{$\deq\vec a_E = \frac{-215 \unit{N}\ihat}{5.97\ten{24}\unit{kg}} = -\sigfrac{3.60}{1\ten{-23}}{m}{s^2} \ihat$.}}
    \item Which of Newton's laws allows you to answer each of these questions?
\end{enumerate}
\item If you apply a force of $4.65\unit N$ to a mass of $2.18\unit{kg}$, then how much will it accelerate?
\item How much force must you apply to cause a mass of $80.0\unit{kg}$ to accelerate at $a=0.795\unitfrac{m}{s^2}$?
\item You arrive home to find a box that came in the mail.  You find that you have to exert $54.3\unit N$ to cause it to accelerate $a=1.25\unitfrac{m}{s^2}$.  (a) What is its mass?  (b) Is that a heavy box or a light box?  (c) Is it likely that this box would fit in a mailbox?
\item Your $2538 \unit{kg}$ car has run out of gas.  So you ask your friend, \studentB{} who has a mass of $\massB$, to put it in neutral, sit inside, and steer while you push.  If you apply enough force to cause a net forward force of magnitude $37.5\unit N$, how much time will it take for the car to move faster than you can walk?  Assume your walking speed is $3.0\unitfrac{mi}{hr}$.  How far will the car have travelled in that time?
\item Find the components of the net force on a large crate if three forces are applied: $\vec F_1 = -3.0\unit N \ihat + 2.5 \unit N \jhat$, $\vec F_2 = -6.25\unit N \jhat$, and $\vec F_3 = 4.5\unit N \ihat + 1.63 \unit{N} \jhat$.
\item Find the components of the net force on a large crate if three forces are applied: $F_1 = 3.61 \unit N $ at $71.6^\circ$ north of east, $F_2 = 4.61\unit N$ due west, and $F_3 = 8.13\unit N$ at $21.8^\circ$ south of east.
\item Find the magnitude and direction of the net force on a large crate if three forces are applied: $\vec F_1 = 4.25\unit N \ihat - 4.66 \unit N \jhat$, $\vec F_2 = -2.65\unit N \jhat$, and $\vec F_3 = -5.4\unit N \ihat + 2.93 \unit{N} \jhat$.
\item Find the magnitude and direction of the net force on a large crate if three forces are applied: $F_1 = 2.65 \unit N $ at $26.6^\circ$ north of west, $F_2 = 2.22\unit N$ at $56.31^\circ$ south of west, and $F_3 = 7.12\unit N$ at $28.4^\circ$ north of east.
\end{enumerate}



\chapter{The Many Types of Force}\label{c:forcetype}\mlinkreturn[subscript notation of forces]{d:interaction}

\section{Gravity at the Surface of the Earth}\label{s:Fg}\mmultireturn{\mmr{\hyperlink{d:accgrav}{freefall}}, \mmr{\autoref{f:firstFBD}}}\new{v2.2}{Adding detail}

Perhaps the force that is the most obvious to humanity is the one that helps us fall when we stumble: the gravitational force\index{Gravity!Surface of Earth}.  This is one of the fundamental forces discussed in \autoref{s:fundamental}.  In addition, the details about how the planets, moon, and the sun experience this force will be discussed in \autoref{c:gravity}.  For now, we can consider how this interaction manifests itself on our daily lives.  In this section, we will start with how objects move when the gravitational force is the only force acting.  Subsections~\ref{ss:weightmass} and~\ref{ss:equivmm} will clarify some subtleties and then we'll jump into the examples in \autoref{ss:local.mg}.

We can investigate what happens when the gravitational force is the only force acting on an object by holding it in the air and dropping it\index{Freefall}.  One of the complications during such an experiment was discussed in \autoref{ss:airresistance}.  If we drop a sheet of paper, there is air resistance in addition to the gravitational force.  For this section, I will assume that the mass-to-surface-area ratio is large enough that we can effectively\Touchstone{Recall \protect{\hyperref[s:effective2]{effective theories}}.}{} ignore the air resistance.

Since objects fall faster than humans are used to paying attention to, the \hypertarget{d:Fgrav}{patterns} are difficult to see.  The green box of \autoref{irl:freefall} (on page~\pageref{irl:freefall}) shows you how you can pay close attention to the patterns that result from observing falling objects.
You should go do those experiments before reading further.  Go ahead.  I'll wait.

You did do them, right?  You're not just reading ahead?  Really?  OK.  Doing that experiment will help you see (1) that everything falls at the same rate and (2) that objects accelerate as they fall\phantomsection\label{d:Fgball}.  This first point is a bit less intuitive and will be discussed further in \autoref{ss:equivmm}.  This second point should be exactly what you expect, when you consider \hyperref[ss:NII]{Newton's second Law}: If there is only one force (the gravitational force), then the object cannot be in \hyperref[sss:equilibrium]{equilibrium} and it must be accelerating.  (You should notice that this is the language of \hyperref[st:F=ma]{the story of Newton's second law}.)

In order to evaluate this further, let's consider a specific object, like a baseball.  Our baseball has a mass of $m_b = 0.145\unit{kg}$.  If the only force acting \textit{on} the ball is the gravitational force \textit{by} the Earth, then the net force is the gravitational force: $\vec F_\mathrm{net} = \vec F_{bEg}$\Touchstone{\hyperlink{d:interaction}{the on-by notation}}.  Here the subscripts are $b$ (because the force is on the \underline{b}all), $E$ (because the force is exerted by the \underline{E}arth), and $g$ (because it is a \underline{g}ravitational force).  Since the acceleration is due to the gravitational force, I will use either $a_g$ (usually when the object is in \hyperref[ss:freefall]{freefall} and therefore accelerating at this rate) or $g$ (usually when the object is not actually accelerating at that rate).  With this notation, Newton's second law becomes:  \[ \vec F_{bEg} = m_b \vec a_g \]
At this point, we know the mass, but we don't know the force or the acceleration.  However, we have conveniently already done the experiment (recall \autoref{ex:freefall}) that will tell us the acceleration is $a_g = 9.81\unitfrac{m}{s^2}$ downwards.  (Recall that ``downwards'' is the direction of the vector, which can be expressed as $-\jhat$.)  If we know the mass and the acceleration, then we can compute the force.
\begin{sample}
\item\label{se:weightball} If a baseball with mass $m_b = 0.145\unit{kg}$ is dropped (allowed to \hyperref[ss:freefall]{fall freely}) so that it accelerates at $a_g = 9.81\unitfrac{m}{s^2}$ downwards, then while it falls it feels the gravitational force:
    \[ \vec F_g = m \vec g = (0.145\unit{kg}) [-(9.81\unitfrac{m}{s^2})\,\jhat] = -\sig{1.42}{24}{N} \jhat = -1.42 \unit N \jhat \]
\end{sample}
This is the force of the gravitational force on the baseball.  Although we computed the force while the ball was falling, the gravitational force does not magically vanish when the ball is sitting on the floor.  So, we can say that (as long as the ball is close to the surface of the Earth, as noted in \autoref{c:gravity}) the force always has this value.  Rather than continuing to say ``the force of gravity'' we call this force the weight\index{Weight}.
\important{The weight of an object is computed as its mass times the acceleration due to gravity, even when the object is not actually accelerating at that rate:  $\mathbf{F_g \equiv mg}$.}

\subsection{Weight versus Mass}\label{ss:weightmass}\mmultireturn{\mmr{\autoref{ss:convertunits}}, \mmr{\autoref{s:sigfig}}}\index{Weight}\new{v2.2}{Added detail.  Moved the previous version to \protect{\autoref{s:sigfig}} to smooth the transition to \protect{\autoref{ss:equivmm}}.}

Since all objects have the same acceleration due to gravity at the surface of the Earth, the weight of an object and the mass of an object are very closely correlated, but they are not the same quantity.  This tends to cause some confusion when the discussion is not explicitly technical.  Recall the discussion about \hyperref[s:precision]{being precise in our language}.  One complication for people in the United States is that there are two definitions of the pound; one is a unit of mass\footnote{There are also multiple versions of the pound-mass.  You can find these explained on the internet, but most of these are considered obsolete.  The one I will use is the ``avoirdupois-pound'', which is defined in the NIST publication
% found in https://en.wikipedia.org/wiki/Pound_(mass)
\protect{\href{https://www.nist.gov/sites/default/files/documents/2017/04/28/AppC-12-hb44-final.pdf}{Handbook 44}}, page C-19, as exactly $453.592 37\unit{g}$.}
and the other is a unit of force.  Since the pound-force\footnote{There is also a unit of force called the kilogram-force.} is defined as the standard unit of mass times the standard unit for the acceleration due to gravity,
% https://en.wikipedia.org/wiki/Pound_(force)
as discussed in \autoref{s:SI-MKS}\dothis{Update \protect{\autoref{s:SI-MKS}} with this information.}, the conversion directly from pound-force to Newtons will \underline{not} match the longer, but more appropriate, conversion from pound-mass to kilogram that gets multiplied by the local acceleration due to gravity (as opposed to the standard $g$) into Newtons.  It may also be useful to review the comments about unit-conversion in the section on \hyperref[s:sigfig]{significant digits}\index{Significant Digits}.

In the discussion about \hyperref[s:precision]{being precise in our language}, we distinguished ``massive'' (the amount) from ``voluminous'' (the size).  Now that we understand \hyperref[ss:NII]{Newton's second law}, we can distinguish ``massive''
%(an amount of material)
from ``weighty.'' %
(a strength needed to lift).
The concept that goes with
\important{mass is the amount of material,}
whereas, the concept that goes with
\important{weight is how strongly the gravitational force pulls on the object.}
Having mass affects both the inertia (ease of moving) and the weight (force of gravity).
Having weight expresses the gravitational force due to whichever large object (moon, planet, sun, etc.) you happen to be on or near.  Noticing that the \hyperref[s:SI-MKS]{SI-unit} is different for different types of quantities, such as a kilogram (a \hyperref[ss:units]{fundamental unit}) for mass and a Newton (a \hyperref[ss:units]{derived unit}) for weight, may help you remember that these are different kinds of quantities.

The interesting aspect of this relationship is that while having more mass makes an object harder to move (the same force produces less acceleration for more massive objects), when objects fall under the influence of the gravitational force, they accelerate at the \textit{same} rate.  This reveals that the gravitational force must be stronger for more massive objects \textit{by the exact amount} needed to compensate for that larger mass.  This is called the equivalence principle and is discussed in \autoref{ss:equivmm}.


\subsection{Calculating the weight}\label{ss:local.mg}\new{v2.2}{renamed this section and added detail}

When calculating the forces acting on a person or an object, we will often need to account for the force of gravity, while other forces may also be at work.  As mentioned above, the weight is found by multiplying the mass times the local acceleration due to gravity, even if the object is not actually accelerating at that rate.  Chapter~\ref{c:gravity} will clarify why it is true\footnote{The short answer is that the altitude (distance from the surface of the Earth) and local geology affect the strength of the gravitational field.  Since the Earth is slightly oblate (bulges at the equator), the altitude at different latitudes corresponds to a different distance from the center of the Earth.  In addition, while the spin of the Earth does not affect the strength of the gravitational field, it does affect how objects accelerate. The \protect{\href{http://www2.csr.utexas.edu/grace/gallery/animations/ggm01/ggm01_gif-200.html}{GRACE project}} has measured the variations across the globe.}, but for now please note that the acceleration due to gravity is (1) different according to where we are and also (2) the same for all objects at that location.\dothis{Gather values of $g$ at various locations.  Wiki has a list, but need to find the source.  Wolfram has numbers, but they seem to be calculated off a formula, not measurements.  \protect{\href{http://www.physics.montana.edu/demonstrations/video/1_mechanics/demos/localgravitychart.html}{U Montana}} has values but no reference.
\protect{\href{http://www.calpoly.edu/~gthorncr/ME302/documents/AccuracyofGravity.pdf}{Glen Thorncroft at Cal Poly}} has a formula and lists the level of each effect.}\index{Acceleration!Gravity}\index{Gravity!Acceleration}\done{Add a table of measured values of $g$ at various locations.  Compute the weight of a specific person at various locations.}

\hypertarget{d:weightmass}{Because} of the peculiarities in the definition of pound (\autoref{ss:weightmass}) it will be useful to build some intuition about masses in terms of kilograms and Newtons.  \autoref{t:weightmass} lists the mass of some common objects and, using the standard value for $g$, their corresponding weights.
%
\begin{table}[bhtp]
\hrule\hrule
\begin{center}
\caption[Comparison of masses and weights of common objects]{\label{t:weightmass} The list of objects is intended to give a sense of scale so that the reader can better estimate the value of the mass of an object.  You might notice that (except for the apple) each of these is between 4 and 4.5 times heavier than the previous object.  Note that these are rough estimates; for example, while the author weighs about $200\unit{lbs}$ this is not typical, nor average.
%\linkreturn[weight and mass]{d:weightmass}
% reference weight of an apple:  \url{http://www.applejournal.com/ref.htm}
}
\begin{tabular}{lrrr}
Object & pounds & mass (kg) & weight (N) \\ \hline
apple & 0.33 & 0.15 & 1.5 \\
lean, healthy cat & 10 & 4.6 & 45 \\
medium-sized dog & 44 & 20 & 196 \\
human & 200 & 91 & 890 \\
horse & 1000 & 362 & $3.56\ten{3}$ \\
large pick-up truck & 4000 & $1.81\ten{3}$ & $1.78\ten{4}$
\end{tabular}
\end{center}
\hrule\hrule
\end{table}
%
\hyperref[c:weightmass]{Conceptual Problem \ref{c:weightmass}} asks you to estimate the mass of some other common objects.  \hyperref[c:massweight]{Conceptual Problem \ref{c:massweight}} asks you to think of common objects with a specified mass.

Now let's do some calculations\ldots
\begin{sample}
\item \studentA\index{\studentA} notices that \heA\ needs to exert $F=1.5\unit{N}$ to support the apple listed in \autoref{t:weightmass}. \HeA\ then drops it  and notices its acceleration of $9.81\unitfrac{m}{s^2}$.  \HeA\ computes the mass to be
    \[ m = \frac{F_g}{a_a} \ = \ \frac{1.5\unit{N}}{9.81\unitfrac{m}{s^2}} \ = \ \frac{1.5\unitfrac{kg \cdot m}{s^2}}{9.81\unitfrac{m}{s^2}} \ = \ 0.\sig{15}{3}{kg} \]
    (If you know the weight, you can compute the mass, even if the mass is not actually in freefall.)
\item\label{se:weightA} \studentA\index{\studentA}\new{v2.3}{modified and supplemented}, who knows \hisA\ own mass ($\massA$), then imagines\mmultireturn{\mmr{\autoref{f:firstFBDupdate}}, \mmr{\autoref{f:firstFBDangle}}} dropping \himselfA\ (!) from a (small) height.  While \heA\ falls, \heA\ recognizes the gravitational force on \himA, which is computed to be
    \[ \vec F_g = m \vec g = (\massA) [-(9.81\unitfrac{m}{s^2})\,\jhat] = -\sig{833}{.85}{N} \jhat = -834 \unit N \jhat \]
    Since \heA\ is in freefall and there is only one force is acting on \himA, the net force is easy to compute:  $\vec F_\mathrm{net} = -834 \unit N$.
    However, if you know the mass something, you can compute the weight even if that object is not in freefall.
    You should repeat this calculation for the mass in \ref{se:netF-a}.  (\ref{A:netF-a})
\end{sample}
You should note that
\important{$F_\mathrm{net} \ (=ma)$ is always related to the actual acceleration of the object, \\ $F_g\ (=mg)$ is always related to the local acceleration due to gravity.}
You should also note that
\important{the actual acceleration is only equal to the local acceleration due to gravity if the object is in freefall.}
\begin{sample}
\item\label{se:FNB} If\mmultireturn{\mmr{\autoref{f:firstFBDupdate}}, \mmr{\autoref{f:firstFBDangle}}} \studentB\index{\studentB} is not falling, but rather standing safely on the floor, then the gravitational force is still acting.  It can be computed as
    \[ \vec F_g = m \vec g = (\massB) [-(9.81\unitfrac{m}{s^2})\,\jhat] = -\sig{735}{.75}{N} \jhat = -736 \unit N \jhat \]
    However, since we can see that \hisB\ acceleration is zero, the $\vec F_\mathrm{net}$ \textit{must be zero}.  The only way that can happen, though is if there is another force acting upwards on \studentB.  What could possibly be pushing up on \himB?  \ref{A:floor}.  Whatever it is pushing up on \himB, it is supplying a support force, which can be calculated since $\vec F_\mathrm{net} = \vec F_g + \vec F_\mathrm{support}$ and we can solve for
    \[ \vec F_\mathrm{support} = \vec F_\mathrm{net} - \vec F_g = m\left(0\unitfrac{m}{s^2}\right) - \left[ -(\sig{735}{.8}{N}) \jhat\right] = +736\unit N \jhat \]
    Because it is in the direction opposite to $\vec F_g$, it is upwards $(+\jhat)$.

    Can you identify \textit{why} the support force is equal in magnitude and opposite in direction to the gravitational force?
    \TWO{Newton's second law}{Newton's third law}{A:second}{A:third}
\end{sample}
As was mentioned earlier, the value of the acceleration due to gravity also varies across the surface, although this is less than about a percent or so (see~\autoref{t:gworld}).
Nonetheless, this means that your weight can change even when your mass remains the same.
\begin{sample}
\item\label{se:gworld} While talking to your friend \studentB\index{\studentB}, you learn that \hisB\ parents, \studentE\index{\studentE} and \studentF\index{\studentF}, grew up in Norway, visited Puerto Rico, and climbed Mount Everest before settling in the United States.  Using \autoref{t:gworld}, compute \studentE's weight are each location, assuming \hisE\ mass is \massE.
\begin{enumerate}
\item[Norway] $F_g = mg = (\massE)(9.825\unitfrac{m}{s^2}) \ = \ \sig{933}{.4}{N}$
\item[Puerto Rico] $F_g = mg = (\massE)(9.782\unitfrac{m}{s^2}) \ = \ \sig{929}{.3}{N}$
\item[Mount Everest] $F_g = mg = (\massE)(9.763\unitfrac{m}{s^2}) \ = \ \sig{927}{.5}{N}$
\end{enumerate}
\end{sample}
%
Because the variation is small, throughout this text when we are considering situations ``at the surface of the Earth'', we will assume that
\important{the acceleration due to gravity is $9.81\unitfrac{m}{s^2}$ to three significant figures.}
%
\begin{table}[bhtp]
\hrule\hrule
\begin{center}
\caption[Comparison of $g$ at a few places on Earth]{\label{t:gworld} Comparison of $g$ at a few places on Earth.  {\color{gray} [While both the latitude-longitude and the local value of $g$ were found using the
\href{https://www.wolframalpha.com/}{WolframAlpha$^R$ computational knowledge engine},
these $g$ values do not necessarily correspond to these coordinates.  The $g$ values are based on a theoretical model of the Earth.]}
You should look for a pattern as the latitude increases.  (\ref{A:gworld})
You might notice the values for  Mount Everest and Denver; Can you explain any peculiarity?  (\ref{A:gpeaks})
\return{se:gworld}
}
\begin{tabular}{lccr}
Location & latitude & longitude & local $g (\!\!\unitfrac{m}{s^2})$ \\ \hline
San Juan, Puerto Rico & $18^\circ 26' 24'' \unit{N}$  & $66^\circ 7' 48'' \unit W$ & $9.782 \unitfrac{m}{s^2}$ \\
Brownsville, TX & $26^\circ 1' 6'' \unit{N}$  & $97^\circ 27' 14'' \unit W$ & $9.788 \unitfrac{m}{s^2}$ \\
Mount Everest & $27^\circ 59' 17'' \unit{N}$  & $86^\circ 55' 31'' \unit E$ & $9.763\unitfrac{m}{s^2}$ \\
Cincinnati, OH & $39^\circ 8' 24'' \unit{N}$  & $84^\circ 30' 23'' \unit W$ & $9.801\unitfrac{m}{s^2}$ \\
Denver, CO & $39^\circ 45' 43'' \unit{N}$  & $104^\circ 52' 50'' \unit W$ & $9.798\unitfrac{m}{s^2}$ \\
Paris, France & $48^\circ 51' 36'' \unit{N}$  & $2^\circ 20' 24'' \unit E$ & $9.813\unitfrac{m}{s^2}$ \\
Oslo, Norway & $59^\circ 54' 36'' \unit{N}$  & $10^\circ 45' \phantom{24''} \unit E$ & $9.825\unitfrac{m}{s^2}$ \\
Anchorage, AK & $61^\circ 10' 39'' \unit{N}$  & $149^\circ 16' 28'' \unit E$ & $9.826\unitfrac{m}{s^2}$
\end{tabular}
\end{center}
\hrule\hrule
\end{table}
%



\section{Fundamental Forces}\label{s:fundamental}\index{Force!Fundamental}\new{v2.1}{Started the section on fundamental interactions.  Link ahead, rather than detailling here.}

The previous section describes our (macroscopic) experience of the gravitational interaction when standing on the surface of the Earth.  This is essentially the same across the surface, but does change with altitude and the difference can be measured on mountain tops and in caves.  In fact, one can use the differences from one location to another to predict where we might find a a pocket of oil.\new{v2.2}{Filled out the detail.  Changed the approach.}

In later \hypertarget{d:fundamental}{sections}, we will consider this and other interactions that depend on the physical properties, such as mass and charge.  All particles with the property of mass (which we will start to call gravitational charge) will interact according to the gravitational force; however, this description is better described by the mathematics in \autoref{c:gravity}.  All particles with the property of electrical charge will interact according to the electrical force.  The basic theory will be discussed in \autoref{c:electric}.  A more complicated version that incorporates quantum mechanics is called quantum electrodynamics (QED) and this will be touched on in \autoref{ss:QED}.  Particles like protons and neutrons (hadrons) are actually made up of other particles (quarks) that are held together by an interaction that is sometimes called the strong nuclear force (\autoref{ss:strong}) and is described by the theory of quantum chromodynamics (QCD); this will be touched on in \autoref{ss:QCD}.  Finally, in \autoref{ss:weak} another fundamental force, called the weak nuclear force, will be discussed.

For the most part, these theories describe the interaction between microscopic particles, so we will not discuss them in detail here.  However, the gravitational interaction is exception in a variety of ways.  In particular, the gravitational interaction does affect macroscopic objects.  These fundamental forces have a particular description that allows us to pretend (recall \hyperref[s:effective2]{effective theories}) that they are action-at-a-distance interactions.  All other forces (introduced next) will require physical contact in order to exert the force.

\section{Normal Force}\label{s:FN}\mmultireturn{\mmr{\autoref{f:firstFBD}}, \mmr{\ref{A:floor}}, \mmr{\autoref{s:FT}}}\new{v2.2}{Added detail}\index{Force!Normal}

The word ``normal'' \href{http://etymonline.com/index.php?term=normal}{originates}\index{Normal} with the idea of conformity to the pattern.  While in everyday life this the typical state of being, the origins actually refer to a carpenter's square, which put corners into a right angle.  In math and physics, the word is used to mean perpendicular.  In the context of forces,
\important{the normal force is the force that a surface exerts to keep objects from passing through them.  The direction of this force is always in the outward direction, normal (perpendicular) to the surface.}

Let's consider some specific situations\ldots\inlife{} In \ref{se:FNB}\dothis{DO we need to repeat the example here? no?}, \studentB\ felt the downwards gravitational force even while \heB\ was standing on the ground.  We noticed that \heB\ was not falling (and so not accelerating).  Colloquially, we say that the ground is supporting \studentB.  This support force is keeping \studentB\ from passing through the floor; this is a normal force.  The normal force from the floor is acting upwards, which is normal (perpendicular) to the surface of the floor.  \autoref{f:firstFBDupdate} updates the free-body diagrams of \autoref{f:firstFBD} to show how the gravitational and normal forces impact that calculation.
%
\begin{figure}
\hrule\hrule
\caption{\label{f:firstFBDupdate} An updated version of \protect{\autoref{f:firstFBD}}, people pushing a box.}\index{Free-Body Diagrams!Images}
Again, we can start by drawing a picture of the situation.  The description is the same as it was for \autoref{f:firstFBD}.  In addition to those forces, each of the three bodies has a downwards gravitational force.  This analogous to the calculation in \ref{se:weightA}, which was only for \studentA\index{\studentA}; but you can calculate the weight for the mass in \ref{se:netF-a} and \studentB\index{\studentB}'s weight was computed in \ref{se:FNB}.  In addition to the downward gravitational force (the weight), Newton's second law and the fact that nothing is accelerating up or down together tells us that

\noindent
\begin{minipage}[b]{150pt}
there must also be a normal force on each body.  This is analogous to the calculation in \ref{se:FNB}, which was only for \studentB; but you can deduce it for the object and for \studentA.
\end{minipage}
\hfill\begin{minipage}[b]{220pt}
\begin{picture}(220,85)(-10,-25)
\put(0,0){\line(1,0){200}}
\put(60,2){\line(1,0){60}}
\drawbox{30}{1}{20}{50} %\studentA
\drawbox{50}{25}{18}{5} %\studentA's arms
\put(30,53){\scriptsize \studentA}
\drawbox{70}{3}{20}{30} % object
\put(70,35){\scriptsize object}
\drawbox{150}{1}{20}{40} %\studentB
\drawbox{134}{25}{16}{5} %\studentB's arms
\put(150,43){\scriptsize \studentB}
\put(90,27.5){\oval(2,2)[r]}
\put(91,27.5){\line(1,0){43}}
\put(60,-12){\begin{minipage}{60pt}
\scriptsize The object is on a sheet of ice.
\end{minipage}}
\end{picture}
\end{minipage}


Now, as in \autoref{f:firstFBD}, we will draw a free-body diagram for each individual separately.  However, this time we will use \ref{se:weightA} and \ref{se:FNB} to include the gravitational force (the weight) and the normal force.

\noindent % \textwidth default is 5in for a book
\fbox{\begin{minipage}{1.5in}
\begin{FBD}{10}{25}{15}{80}{\studentA}
\onele{20}{$5\unit N$}{black}
\onedo{100}{$834\unit N$}{black}
\oneup{100}{$834\unit N$}{black}
\end{FBD}
\raggedright
Even with the vertical forces, \studentA\ still has a $\vec F_\mathrm{net} = -5.0\unit N \ihat$.
\end{minipage}}
\hfill
\fbox{\begin{minipage}{1.5in}
\begin{FBD}{10}{15}{15}{25}{object}
\twori{20}{$5\unit N$}{black}{16}{$4\unit N$}{black}
\onedo{35}{$20\unit N$}{black}
\oneup{35}{$20\unit N$}{black}
\end{FBD}
\raggedright
Even with the vertical forces, the object still has a $\vec F_\mathrm{net} = +9.0\unit N \ihat$.
\end{minipage}}
\hfill
\fbox{\begin{minipage}{1.5in}
\begin{FBD}{10}{20}{15}{75}{\studentB}
\onele{16}{$4\unit N$}{black}
\onedo{88}{$736\unit N$}{black}
\oneup{88}{$736\unit N$}{black}
\end{FBD}
\raggedright
Even with the vertical forces, \studentB\ still has a $\vec F_\mathrm{net} = -4.0\unit N \ihat$.
\end{minipage}}
\flushright
\multireturn{\mmr{\autoref{ex:braced}}, \mmr{\autoref{ex:unbraced}}, \mmr{\autoref{s:FN}}, \mmr{\hyperlink{d:rope.net}{rope-tension}}, \mmr{\autoref{f:firstFBDangle}}}
\hrule\hrule
\end{figure}

Let's consider some other specific situations\ldots If you decide to lean against a wall, the wall will provide a normal force that pushes horizontally, keeping you from moving through the wall.\new{v2.3}{Answered \protect{\ref{se:ladderN}} and its related problems.}
%
\begin{sample}
\item\label{se:ladderN} \studentC\ leans a $22.7\unit{kg}$ ladder against a wall at an angle of $75.5^\circ$, consistent with \protect{\href{https://www.osha.gov/}{OSHA}} standard \protect{\href{https://www.osha.gov/pls/oshaweb/owadisp.show_document?p_table=standards&p_id=10839}{1926.1053(a)(1)(ii)}}, so that about $\txtfrac{1}{8}$ of the weight is leaning into the wall.  \begin{enumerate}
\item Find the magnitude and direction of the normal force exerted by the wall on the ladder.
\item Find the magnitude and direction of the normal force exerted by the wall on the ladder.
    \end{enumerate}

Since the weight is $F_g = mg = (22.7\unit{kg})(9.81\unitfrac{m}{s^2}) = \sig{222}{.69}{N}$, an eighth of this is $\sig{27.8}{36}{N}$.  This force is pressing into the wall (horizontally, which I will choose as the $+\ihat$ direction).  By \hyperref[ss:NIII]{Newton's third law} if the ladder presses into the wall with $\sig{27.8}{36}{N}$ in the $+\ihat$ direction (this is also a normal force), then the wall pushes the ladder with a normal force of $\sig{27.8}{36}{N}$ in the $-\ihat$ direction.  \textbf{Notice that this is normal (perpendicular) to the surface of the wall.}

Since the full weight of the ladder, $F_g = \sig{222}{.69}{N}$, is still pressing downwards $(-\jhat)$ into the floor (as a normal force), \hyperref[ss:NIII]{Newton's third law} says that the floor pushes the ladder upwards $(+\jhat)$ with a normal force of $\sig{222}{.69}{N}$.  \textbf{Notice that this is normal (perpendicular) to the surface of the floor.}

\autoref{ex:ladder2} goes into the full details of how one calculates the necessary values.
\end{sample}
%
If you lose control of your car and run into a tree, the tree also provides a normal force pushing the car away from the tree; this normal force will stop the car.
%
\begin{sample}
\item\label{se:tree} \studentZ\index{\studentZ} is driving home after a late night of studying at the library.  \HeZ\ is kind of tired and drifts off during the drive.  While traveling $\vec v_i = 13.0\unitfrac ms \ihat$, \studentZ\ runs into a tree, bringing \hisZ\ car $(m=2.1\ten{3}\unit{kg})$ to a halt in $\Delta t = 0.243\unit s$.  (\studentZ\ remains unharmed because \heZ\ was awake enough to wear \hisZ\ seatbelt and
\noindent
\begin{minipage}[b]{240pt}
had a car with a functioning airbag.  Whew.)  Find the normal force by the tree on the car. \\

To be clear about what is happening, I will draw the picture. In order to find the force, we will first need to find the acceleration.
\end{minipage}
\hfill\begin{minipage}[b]{130pt}
\begin{picture}(120,80)(-10,-5)
\put(0,0){\line(1,0){100}}
\drawbox{70}{1}{20}{50} %\studentA
\drawbox{10}{5}{30}{20} % object
\put(15,3){\circle{5}}
\put(35,3){\circle{5}}
\put(0,40){\scriptsize $v=13.0\unitfrac ms$}
\put (10,35){\vector(1,0){30}}
\put(72,33){\scriptsize Tree}
\put(15,15){\scriptsize car}
\end{picture}
\end{minipage}
\[ \vec a = \frac{\vec v_f-\vec v_i}{\Delta t} = \frac{(0\unitfrac ms)-(13.0\unitfrac ms \ihat)}{0.243\unit s} = -\sigfrac{53.4}{9}{m}{s^2}\ihat \]
That the acceleration is in the direction opposite the velocity corresponds to the object slowing down.  Now we can find the \underline{net force} from Newton's second law:
\[ \vec F_\mathrm{net} = m \vec a = (2.1\ten{3}\unit{kg})(-\sigfrac{53.4}{9}{m}{s^2}\ihat) = -\sig{1.1}{2\ten{5}}{N} \ihat \]
There are three forces acting on the car, as can be seen in the free-body diagrams of \autoref{f:firstFBDupdate}.  So, we can draw a free-body diagram here as well.  The gravitational force on the car is
\[ \vec F_g = m\vec g = (2.1\ten{3}\unit{kg})(-9.81\unitfrac{m}{s^2}\jhat) = -\sig{2.0}{6\ten{4}}{N}\jhat \]
Since this in the vertical direction and the net force is in the horizontal direction, there must be an upwards normal force from the ground
\begin{minipage}[b]{240pt}
$ F_{N,\mathrm{ground}} = \sig{2.0}{6\ten{4}}{N}\jhat$.
\textbf{This is normal (perpendicular) to the surface of the ground.} \\

The remaining horizontal force is the normal force from the tree,
$ \deq F_{N,\mathrm{tree}} = -\sig{1.1}{2\ten{5}}{N} \ihat$.
\end{minipage}
\hfill\begin{minipage}[b]{130pt}
\fbox{\begin{minipage}[b]{100pt}
\begin{FBD}{15}{10}{15}{25}{car}
\onele{40}{$F_{N,\mathrm{tree}}$}{rgb:red,0;green,2;blue,1}
\onedo{30}{$F_g$}{rgb:red,0;green,2;blue,1}
\oneup{30}{$F_{N,\mathrm{ground}}$}{rgb:red,0;green,2;blue,1}
\end{FBD}
\end{minipage}}
\end{minipage}
\textbf{This is normal (perpendicular) to the surface of the tree.}
\end{sample}%
(Notice that \ref{se:tree} also shows why it is not always necessary to consider the vertical forces when we ``know'' that they cancel.)  If you throw a ball at the ceiling, the ceiling will provide a normal force downwards, keeping the ball from moving through the surface.
%
\begin{sample}
\item\label{se:ceiling} \studentC\index{\studentC} recalls that one time \heC\ got bored one day in physics class (what?!?) and tossed a baseball ($m_b = 0.145\unit{kg}$) at the ceiling\ldots a little too hard \ldots as recounted in \autoref{ex:ceiling}.  The acceleration during that collision with the ceiling was $\vec a = - \sigfrac{28.0}{9}{m}{s^2} \jhat$.  Find the normal force by the ceiling on the ball.

There are five stages to the motion: (a) throwing, (b) falling up, (c) hitting the ceiling, (d) falling down, and (e) catching show the forces involved. \\
\color{lightgray}
\fbox{\begin{minipage}[b]{55pt}
\begin{picture}(50,100)(0,0)
\put(25,25){\circle{10}}
\put(25,26){\vector(0,1){25}}
\put(25,24){\vector(0,-1){15}}
\put(28,35){$F_\mathrm{throw}$}
\put(28,10){$F_g$}
\end{picture}
\centering{(a) throwing}
\end{minipage}}
\hfill
\fbox{\begin{minipage}[b]{55pt}
\begin{picture}(50,100)(0,0)
\put(25,50){\circle{10}}
\put(25,50){\vector(0,-1){15}}
\put(28,35){$F_g$}
\end{picture}
\centering{(b) falling up}
\end{minipage}}
\hfill
\color{rgb:red,0;green,2;blue,1}
\fbox{\begin{minipage}[b]{55pt}
\begin{picture}(50,100)(0,0)
\put(25,95){\circle{10}}
\put(26,95){\vector(0,-1){25}}
\put(24,95){\vector(0,-1){15}}
\put(28,75){$F_N$}
\put(10,75){$F_g$}
\end{picture}
\centering{(c) \\ hitting}
\end{minipage}}
\hfill
\color{lightgray}
\fbox{\begin{minipage}[b]{55pt}
\begin{picture}(50,100)(0,0)
\put(25,50){\circle{10}}
\put(25,50){\vector(0,-1){15}}
\put(28,35){$F_g$}
\end{picture}
\centering{(d) falling down}
\end{minipage}}
\hfill
\fbox{\begin{minipage}[b]{55pt}
\begin{picture}(50,100)(0,0)
\put(25,25){\circle{10}}
\put(25,26){\vector(0,1){25}}
\put(25,24){\vector(0,-1){15}}
\put(28,35){$F_\mathrm{catch}$}
\put(28,10){$F_g$}
\end{picture}
\centering{(e) catching}
\end{minipage}}
\color{rgb:red,0;green,2;blue,1}
\\
In this particular problem, we are only concerned with step (c) when the ball hits the ceiling, because that is the only part where the normal force acts. \ref{se:throw-up} will describe what happens during steps (a) and (e).

During step (c), we have the actual acceleration, which tells us about the net force.  We will also need to know the weight of the baseball, because gravity is still acting during the collision.
\begin{eqnarray*}
\vec F_N + \vec F_g & = &  \vec F_\mathrm{net} \ = \ m \vec a \\
\vec F_N + m \vec g & = &  m \vec a \\
\vec F_N  & = &  m \vec a - m \vec g \\
\vec F_N  & = &  \left[ (0.145\unit{kg})(-\sigfrac{28.0}{9}{m}{s^2}\jhat) \right] - \left[ (0.145\unit{kg})(-9.81\unitfrac{m}{s^2}\jhat) \right] \\
\vec F_N  & = &  \left[ -\sig{4.07}{3}{N} \jhat \right] - \left[  - \sig{1.42}{2}{N} \jhat \right] \ = \ -\sig{2.65}{1}{N} \jhat
\end{eqnarray*}
You can see that the downward normal force $(\sig{2.65}{1}{N})$ combined with the downward gravitational force $(\sig{1.42}{2}{N})$ together create the downward net force $(\sig{4.07}{3}{N})$.
\end{sample}
%
If you make a \hypertarget{d:bank-shot}{``bank shot''} with either a basketball off the backboard or a pool ball\footnote{Resources for \protect{\href{http://wpapool.com/equipment-specifications/\#Balls-and-Ball-Rack}{specifications}} and
\protect{\href{http://c.ymcdn.com/sites/bca-pool.com/resource/resmgr/imported/BCAEquipmentSpecifications_2008.pdf}{a PDF version}}.
These provide:
    weight ($5.5\unit{oz}=0.\sig{155}{92}{kg}$ and $6.0\unit{oz}=0.\sig{170}{097}{kg}$ cue),
    diameter ($2.250\pm 0.005\unit{in}=\sig{5.71}{5}{cm}\pm 0.0127\unit{cm}$),
    rail height ($63.5 \%$ of the ball height, $= \sig{3.62}{9}{cm}$),  and
    dimension limits on the cue stick:
        $L_\mathrm{min}=40.00\unit{in}=1.016\unit{m}$,
        $m_\mathrm{max} = 25.0\unit{oz} = 0.\sig{708}{75}{kg}$, and
        tip-width $w_\mathrm{max}=1.4\unit{cm}$.
You might also consider the information and calculations at
\protect{\href{http://billiards.colostate.edu/technical_proofs/index.html}{Dr.~Dave's site}},
which gives
    slow ($1\unit{mph}$), medium ($3\unit{mph}$), and fast ($7\unit{mph}$);
    coefficient of friction ball-to-ball $\mu=0.06$; and
    ball-ball collision times as $250\unit{\mu s}$-$300\unit{\mu s}$.
}
off the bumper, then the surface provides a normal force that is perpendicular to the surface, in this case redirecting the ball rather than stopping it.  Unfortunately, the actual mechanism is somewhat more complicated than we are ready for; these are considered a little bit in the \autoref{irl:poolcushion} (pg.~\pageref{irl:poolcushion})\dothis{\protect{\autoref{irl:poolcushion}} should be moved to a section that has more about friction and angular momentum.  It is too complex for this section.}.
%
\begin{reallife}[bthp]
\hspace{-.2in}
\fcolorbox{black}{green!10}{\begin{minipage}{5.29in} \center
\caption{\label{irl:poolcushion}\index{Pool!Real Life} Pool balls and bumpers / cushions.}
\begin{minipage}{4.925in}
\studentD\index{\studentD} is relaxing with the local physics club, playing pool.  \HeD\ shoots a bank-shot and the ricochet reminds all of you about the normal force from the bumper on the ball.
\end{minipage}
\begin{realtable}
\dna{Find a billiards table}
    {Notice the felt, the bumpers (cushion), and the dimensions of the table}
    {Does the ball roll as far on felt as it does on hardwood?  \ref{A:pool.roll} \\
     How soft is the bumper? \ref{A:pool.bumper}}
\dna{Find a set of pool balls}
    {Compare the solid-colored balls, the striped balls, and the cue ball}
    {Are there differences in size of weight? \ref{A:noncue}}
\dna{Hit the cue-ball off of a bumper in the manner intended for
\protect{\href{http://c.ymcdn.com/sites/bca-pool.com/resource/resmgr/imported/BCAEquipmentSpecifications_2008.pdf}{testing cushions}}.}
    {Compare the angle it leaves the bumper (reflected angle) match the angle at which it came in (incident angle)}
    {Does the spin of the ball matter? \ref{A:pool.spin}}
\dna{Place a pool ball against the bumper and ricochet the cue ball off the pool ball instead of the bumper itself.}
    {Notice how the pool ball reacts. \ref{A:pool.later}}
    {Why does the pool ball jump off the bumper? \\
     Does the pool ball move along the wall? \\
     Where did you hit the pool ball?}
\end{realtable}
\begin{minipage}{4.925in}
Billiard tables have a lot of interesting physics, which can help us see a wide variety of physics, for example:
\hyperref[irl:poolfriction]{friction}, \hyperref[irl:poolelastic]{elastic versus inelastic collisions}, \hyperref[irl:poolrotmot]{rotational motion}, and \hyperref[irl:poolangmom]{angular momentum}.
\end{minipage}

\flushright
\linkreturn[pool]{d:bank-shot}
\end{minipage}}
\end{reallife}
%

\subsection{Bathroom Scales Measure the Normal Force}\label{ss:scales}\mlinkreturn[uses of $F=ma$]{d:usesofF=ma}

To get a good sense of what how the normal force works, it helps to consider the way a bathroom scale works.  Consider the concepts presented in the \autoref{irl:scale} (pg.~\pageref{irl:scale}).
%
\begin{reallife}[bthp]
\hspace{-.2in}
\fcolorbox{black}{green!10}{\begin{minipage}{5.29in} \center
\caption{\label{irl:scale}\index{Force!Normal} Playing with a scale.}
\begin{minipage}{4.925in}
While speaking to your friend, \studentB\index{\studentB} about \hisB\ recent accomplishment of losing $45\unit{N}$, you mention that your scale always gives a different number than the one in the doctor's office.  You suggest \heB\ gets on your scale to verify the calibration.  \studentB\ currently has a mass of $\massB$.
\end{minipage}
\begin{realtable}
\dna{Try to lose $45\unit{N}$.}
    {Compare this to your weight}
    {Is this a lot of weight to lose? \ref{A:weight.loss}}
\dna{Place your toe on the scale while \studentB\ weighs \himselfB}
    {This increases the value the scale reads}
    {Does \studentB\ weigh more? \ref{A:weight.gain}}
\dna{With your hands, press down on \studentB's shoulders while \heB\ stands on the scale}
    {Control the value read by the scale.  Increase the reading by $20\unit{N}$, $30\unit{N}$, etc.}
    {Does \studentB's weight change?  \ref{A:weight.gain} Are you adding weight to the scale? \ref{A:scale.increase}}
\dna{Have \studentB\ lean on a nearby table or counter while \heB\ stands on the scale}
    {Control the value read by the scale.  Decrease the reading by $20\unit{N}$, $30\unit{N}$, etc.}
    {Does \studentB's weight change?  \ref{A:weight.gain} }
\dna{Hold the scale against the wall and press into it.}
    {Control the value read by the scale.  Increase the reading by $20\unit{N}$, $30\unit{N}$, etc.}
    {What is the scale measuring? \ref{A:scale.measure}}
\dna{Imagine placing a scale on a ramp that can be laid flat or raised to any angle up to a vertical (making it a wall)}
    {Imagine standing on the scale on the ramp while it is lifted from horizontal (like a floor) to vertical (like a wall)}
    {Does the scale always read the same value while it is raised to different angles? \ref{A:scale.ramp}}
\end{realtable}
%\begin{minipage}{4.925in}
%If you can control the value read by the scale while at the same time not changing your actual mass, does the scale literally measure the weight of the object on the scale?  \ref{A:scale.measure}
%\end{minipage}

\flushright
\autoreturn{ss:scales}
\end{minipage}}
\end{reallife}
%
Some digital scales are inconvenient for understanding how they work because they don't display the value until it has come to something close to equilibrium.  If you have access to an analog scale, then you can watch the value change as it settles down and it might be easier to build your intuition.

As you consider the values that you read on the scale, you should consider what happens if you jump off of or land upon a scale.  \textbf{Note that actually doing this can decalibrate your scale, if not break it entirely.  Scales are not meant to be handled this way.} While you are jumping from your scale, it must provide not only the force necessary to support your weight, but also the upwards force require to accelerate you upwards.  While you are landing on the scale, it musty provide not only the force necessary to support your weight, but also the upwards force necessary to decelerate you.

Bathroom scales use leverage (i.e., \hyperref[s:leverarm]{torque}) and a \hyperref[s:springs]{spring}-system to balance the force pressing into them.  The mechanism can be seen at \href{http://home.howstuffworks.com/inside-scale.htm}{How Stuff Works}.

\section{Tension}\label{s:FT}\mlinkreturn[$F=ma$]{d:f=ma}\index{Force!Tension}

Where the \hyperref[s:FN]{normal force} is appropriate for pushing against surfaces,
\important{tension is the pulling force that is transmitted through materials \\ such as cable, chain, or rope.}
Tension is closely related to the compression force experienced by support beams.  One can simplistically think of tension as pulling\dothis{add a link to (and the section itself) to a section on the modulus and stress/strain.}{} and compression as pushing\dothis{Maybe add an IRL about a house settling and the compression forces.  Loading a pick-up truck and watching the bed sag as weight is added.  Hammock as an example of adding weight and increasing the tension force.}{} on the intermediate object that transmits force between the objects at either end.\footnote{It doesn't usually make sense to talk about the compression of a rope or chain.}
When engineers design the skeleton of bridges and buildings, one of the primary considerations is the tension and compression of the steel beams.  You can build your intuition by considering the \autoref{irl:tension} (pg.~\pageref{irl:tension}).\dothis{Still need to update the \protect{\autoref{irl:tension}}.}
%
\begin{reallife}[bthp]
\hspace{-.2in}
\fcolorbox{black}{green!10}{\begin{minipage}{5.29in} \center
\caption{\label{irl:tension}\index{Force!Tension} Pull my finger.}
\begin{minipage}{4.925in}
We talk about tension and stress in our daily lives.  This is an analogy to the physical version of tension, stress, and strain.  While \protect{\href{http://etymonline.com/index.php?allowed_in_frame=0&search=stress}{stress}} and \protect{\href{http://etymonline.com/index.php?search=strain&searchmode=&p=0&allowed_in_frame=0}{strain}} come from the the concept of tightening, tension \protect{\href{http://etymonline.com/index.php?allowed_in_frame=0&search=tension}{comes from}} the concept of stretching.
\end{minipage}
\begin{realtable}
\dna{Sit on a swing }
    {Notice the tightness of the support ropes/chains}
    {How tight are the supports when the swing is empty? When a small child is in the swing? When a full-sized adult is in the swing? \ref{A:swing.tension}}
\dna{Install a fan or light fixture that hangs from the ceiling}
    {You don't want the fan to be supported by the electrical wires, but rather by the metal shaft}
    {How is the fan supported? \ref{A:fan.tension}}
\dna{Pull on a doorknob}
    {Imagine replacing the knobs (inside and outside) with large knots on a rope that runs through the hole the doorknob used to occupy.}
    {What if the doorknob were replaced with a rope, knotted on either side of the door? [Answer]}
\dna{Take a dog for a walk on a leash}
    {Try to pay attention to Newton's second and third law when the dog changes its level of enthusiasm for pulling on the leash.}
    {If the dog pulls very hard on the leash and you balance that force without allowing the dog to move away from you, then describe the way the force connects you to the dog. [Answer]}
\end{realtable}
%\begin{minipage}{4.925in}
%If you can control the value read by the scale while at the same time not changing your actual mass, does the scale literally measure the weight of the object on the scale?  \ref{A:scale.measure}
%\end{minipage}

\flushright
\autoreturn{s:FT}
\end{minipage}}
\end{reallife}
%

When considering the tension in the rope, the context is generally that the rope is connecting two objects that are trying to pull on each other.  It is convenient to recognize that each object only ``sees'' the rope, not the object at the far side.  This can be seen in a couple of contexts.\new{v2.4}{Modified}

We will start with the \hyperref[s:effective2]{simplistic approximation} of ropes that only transmit the force.  As your understanding improves, we will add some examples where the tension in the rope also affects the rope itself.  In that more complicated situation, the tension will change across the rope\dothis{maybe add links}{} and the rope may stretch\dothis{maybe add links}{}.  Since ropes and cables are twisted strands while chains are links, ropes and cables can also introduce a \hyperref[s:torsion]{torsion}\foreshadow{} that tend not to occur in chains.

\subsection{Tension as a Support Force}\label{ss:tension.support}

Ropes and chains (and beams) can use tension to support (from above) dangling objects.
\begin{sample}
\item\label{se:purse} \studentD\index{\studentD} hangs her purse $(m=1.36\unit{kg})$ on a hook.  How much tension is in the shoulder strap to keep it from falling?

The strap connects the hook to the purse.  We can consider the interaction between the hook and the strap or between the purse and the strap.  We will consider the latter since we don't know anything about the hook.  Considering the forces on the purse, we know that there is a downwards gravitational force of $\deq F_g = (1.36\unit{kg})(9.81\unitfrac{m}{s^2}) = \sig{13.3}{4}{N}$ and that the net force must zero (because the purse is not accelerating). So, the strap must provide an upwards (tension) force.
\begin{eqnarray*}
\vec F_T + \vec F_g & = & m \cancelto{0}{\vec a} \\
\vec F_T + (-\sig{13.3}{4}{N} \jhat) & = & 0\unit N \\
\vec F_T & = & +\sig{13.3}{4}{N} \jhat
\end{eqnarray*}
This is the upwards force that the strap applies to the purse; however, the tension strap is doing two jobs: It is pulling up on the purse (as indicated above) \textbf{and} it is pulling down on the hook.
\end{sample}
The important thing to take away from \ref{se:purse} is not that we can compute the value (although that is, of course, a useful skill), but rather that
\important{the tension is conveying the force between the two objects.}  In the same way that the \hyperref[s:FN]{normal force} on a scale does not measure your weight, but rather the amount you press into the scale, the tension passes force on to the attached object.  The hook doesn't feel the weight of the purse, but does feel the tension required to support the purse.

In \hyperref[sss:multiple.mass]{an upcoming section}, we will consider what happens when multiple masses are hung from the rope.

\subsubsection{How Physicists Use the Words (Vocabulary)}

You can probably think of several examples of objects dangling: a purse on a hook, a flag on a pole, a shop sign attached to a post, a pendulum,\dothis{Add an image of an immovable surface to that section}{} \\
\begin{minipage}{4.25in}
a swing set, etc.  Since these are all similar in some ways (although different in other ways), \textbf{we can treat all of them as a mass at the end of a rope}.  Typically, because we do not want to deal with the complications that come from sagging supports, we will use the \hyperref[s:effective2]{approximation} of an ``\textbf{immovable support}.''  This will be indicated by hashing the surface.
\end{minipage}
\hfill
\begin{minipage}{30pt}
\begin{picture}(35,80)
\put(0,70){\line(1,0){25}}
\multiput(5,70)(5,0){4}{\line(1,1){5}}
\put(12.5,70){\line(0,-1){50}}
\put(7.5,20){\line(1,0){10}}
\put(7.5,20){\line(0,-1){10}}
\put(17.5,10){\line(-1,0){10}}
\put(17.5,10){\line(0,1){10}}
\end{picture}
\end{minipage}

\subsection{Tension as Dragging Force}\label{ss:tension.drag}

We can also consider the \hypertarget{d:rope.net}{tension} in a rope used to drag an object across the floor.  You may recall that in \autoref{f:firstFBD} (and the updated version, \autoref{f:firstFBDupdate}) \studentB\index{\studentB} pulled a box across a sheet of ice.  It is possible that  \studentB\ was grabbing the object itself, but it is more likely that \heB\ was pulling on a rope that was attached to the object.  In that case, the tension in the rope was $4.0 \unit N$.  This tension is what pulled \studentB\ leftwards \textbf{and} what pulled the object rightwards.

We can further update this by considering the case where \studentB\ pulls the rope up at an angle.  In that case, some of the tension is used to drag the box and some is used to reduce the normal force.  In \autoref{f:firstFBDangle}, we will have \studentA\ continue to push with $5.0\unit{N}$ horizontally and have \studentB\ pull with $4.0\unit{N}$ at a $14^\circ$ angle above the horizontal.
%
\begin{figure}
\hrule\hrule
\caption{\label{f:firstFBDangle} An updated version of \protect{\autoref{f:firstFBDupdate}}, people pushing a box.}\index{Free-Body Diagrams!Images}
Again, we can start by drawing a picture of the situation.  The description is the same as it was for \autoref{f:firstFBDupdate} except that \studentB\ pulls at a slight

\noindent
\begin{minipage}[b]{150pt}
angle upwards.  We will again need the gravitational force for \studentA\index{\studentA} (\ref{se:weightA}) and \studentB\index{\studentB} (\ref{se:FNB}).  As before, since nothing is accelerating up or down together, there must also be a normal force on each body.
\end{minipage}
\hfill\begin{minipage}[b]{220pt}
\begin{picture}(220,85)(-10,-25)
\put(0,0){\line(1,0){200}}
\put(60,2){\line(1,0){60}}
\drawbox{30}{1}{20}{50} %\studentA
\drawbox{50}{25}{18}{5} %\studentA's arms
\drawbox{70}{3}{20}{30} % object
\drawbox{150}{1}{20}{40} %\studentB
\drawbox{134}{25}{16}{5} %\studentB's arms
\put(90,16.5){\oval(2,2)[r]}
\put(91,16.5){\line(4,1){44}}
\put(30,53){\scriptsize \studentA}
\put(70,35){\scriptsize object}
\put(150,43){\scriptsize \studentB}
\put(60,-12){\begin{minipage}{60pt}
\scriptsize The object is on a sheet of ice.
\end{minipage}}
\end{picture}
\end{minipage}


Now, as in \autoref{f:firstFBDupdate}, we will draw a free-body diagram for each individual separately.  However, this time we will put the tension of the rope at the appropriate angle.  We will need to do a small calculation to find the value of the normal forces.

\noindent % \textwidth default is 5in for a book
\fbox{\begin{minipage}{1.5in}
\begin{FBD}{10}{25}{15}{80}{\studentA}
\onele{20}{$5\unit N$}{black}
\onedo{100}{$834\unit N$}{black}
\oneup{100}{$834\unit N$}{black}
\end{FBD}
\raggedright
The forces on \studentA\ have not changed.
\end{minipage}}
\hfill
\begin{minipage}{1.5in}
\fbox{\begin{minipage}{1.5in}
\begin{FBD}{10}{15}{15}{25}{object}
\oneri{20}{}{black}\put(43,30){\color{black}\tiny  $5\unit N$}
\onedo{35}{$20\unit N$}{black}
\oneup{35}{$F_N$}{black}
\put(26,41){\color{black} \vector(4,1){20}}
\put(43,46){\color{black} \tiny $4 \unit{N}$}
\end{FBD}
\raggedright
The forces on the object \textit{have} changed.
\end{minipage}}
\begin{picture}(100,60)
\put(0,10){\line(4,1){80}}
\put(0,10){\line(1,0){80}}
\put(80,10){\line(0,1){20}}
\put(15,10){\oval(5,8)[rt]}
\put(25,11){\tiny $14^\circ$}
\put(35,28){\tiny $F_T=4.0 \unit{N}$}
\put(82,30){\tiny $F_{Ty}=$}
\put(82,20){\tiny $= F_T\,\sin 14^\circ$}
\put(82,10){\tiny $ = 0.\sig{96}{8}{N}$}
\put(5,0){\tiny $F_{Tx}=F_T \, \cos 14^\circ = \sig{3.8}{8}{N}$}
\end{picture}
\end{minipage}
\hfill
\fbox{\begin{minipage}{1.5in}
\begin{FBD}{10}{20}{15}{75}{\studentB}
%\onele{16}{$4\unit N$}{black}
\onedo{88}{$736\unit N$}{black}
\oneup{88}{$F_N$}{black}
\put(24,94){\color{black} \vector(-4,-1){20}}
\put(0,92){\color{black} \tiny $4\unit{N}$}
\end{FBD}
\raggedright
The forces on \studentB\ \textit{have} changed.
\end{minipage}}

\noindent
\textbf{For the object}: Since the y-component of the net force is zero, we can find the normal force to be $F_N = -[(-20\unit{N})+(+0.\sig{96}{8}{N})] = 19\unit{N}$.
The x-component of the net force is $F_{\mathrm{net},x}=(5.0\unit N)+(\sig{3.8}{8}{N}) = \sig{8.8}{8}{N}$.

\textbf{For \studentB}: Since the y-component of the net force is zero, we can find the normal force to be $F_N = -[(-736\unit{N})+(-0.\sig{96}{8}{N}) = 737\unit{N}$.
The x-component of the net force is $F_{\mathrm{net},x}=(-\sig{3.8}{8}{N})$.

\flushright
\linkreturn[rope-tension]{d:rope.net}
\hrule\hrule
\end{figure}
%
You should note that since the tension on the object is pulling up, helping the normal force, this allows the normal force (what a scale would read) to be a little smaller.
You should also note that since the tension on \studentB\ is pulling down, counter-acting the normal force, this requires the normal force (what a scale would read) to be a little larger.

\subsection{Pulleys}

While the flexibility of ropes makes them inconvenient for pushing, their flexibility makes them \textit{very useful} for changing the direction of the pull.  The mechanism for changing the direction is the pulley.  Furthermore, by allowing us to change the direction of the pull, we are also able to double, triple, or further improve the strength of the pull.  The term for this is ``the mechanical advantage'' of a pulley-system.

First we will consider three simple cases of redirecting the force.  In each of these cases, I will \hyperref[s:effective2]{assume} that the pulley and rope have no mass and that there is no friction in the turning of the pulley (assume it is trivially easy to spin).  If we do not make this assumption, then the problem gets significantly more complicated.\dothis{add a reference to the section (problem?) where this is considered.}{}

\begin{minipage}[c]{3.25in}
\begin{sample}
\item\studentA\index{\studentA} decides to hold a box that weighs $20\unit N$ using a pulley-system.  What is the tension in the rope?

Since the mass is in equilibrium, the net force is zero and the tension must balance the weight.  This tells us that the tension in the rope is $20 \unit N$.

If the pulley were difficult to turn (had friction) that stickiness could help support the mass and the tension on \studentA's side might be less than $20\unit N$; but since we assumed the pulley to be frictionless, \studentA\ must provide the full $20\unit N$ of tension to the rope.
\end{sample}
\end{minipage}
\hfill
\begin{minipage}{1in}
\begin{picture}(100,120)(0,7)
\put(31,105){\oval(36,36)[t]}
\put(31,105){\circle{33}}
\put(31,106){\line(0,1){29}}
\put(49,105){\line(0,-1){62}}
\put(13,105){\line(0,-1){70}}
\put(-30,7){\line(1,0){100}} % floor
\put(0,135){\line(1,0){62}} % ceiling
\multiput(5,135)(10,0){6}{\line(1,1){5}} % immovable
%
\drawbox{-26}{8}{20}{50} %\studentA
\drawbox{-6}{32}{18}{5} %\studentA's arms
\put(-26,60){\scriptsize \studentA}
%
%\drawbox{5}{19}{16}{16}
%\put(6,25){\small $m_1$}
%
\drawbox{41}{19}{16}{24}
\put(42,25){\small $m$}
%
%\put(49,-1){\vector(0,1){20}}
%\put(49,-1){\vector(0,-1){20}}
%\put(51,-1){\tiny $12\unit m$}
\end{picture}
\end{minipage}
%

\noindent
The interesting aspect is that \studentA\ must pull \textit{down} in order to produce the \textit{upward} tension on the box.  This means that both \studentA\ and the mass are pulling down.  Since the rope is draped over the pulley, the pulley feels $40\unit N$ downwards, $20\unit N$ from the tension supporting the mass and $20\unit N$ from \studentA\ who is creating the tension that supports the mass.  This means that the second rope that is connecting the pulley to the ceiling must be supporting the full $40\unit N$ in order to keep the pulley in equilibrium.



\subsection{Interesting Complications}

\subsubsection{What is the net force on the rope itself?}
The answer to this depends on how complicated you want the answer to be (recall the discussion about effective theories in \autoref{s:effective2}).  Some reasonable answers are:
\begin{itemize}
\item If the rope (and the attachments) are static, then the net force on the rope must be zero even while it maintains the tension.  It is also possible that the rope is accelerating, in which case the net force on the rope while it transfers the forces between the objects at each end is whatever is necessary to produce the acceleration $\vec F_\mathrm{net} = m_\mathrm{rope} \vec a_\mathrm{rope}$.
\item A different answer is to assume that the mass of the rope is small enough that whether it is in equilibrium or accelerating, it does not require a net force and it merely passes its tension through to the object at the other end.
\end{itemize}

\subsubsection{Multiple Masses}\label{sss:multiple.mass}\mautoreturn{ss:tension.support}

Now that we have a few examples of tension under our belts, we can consider some more interesting examples.

\autoref{ex:multiweight.tension} considers the case of hanging multiple masses, which extends the ideas of \autoref{ss:tension.support}.
%
\begin{example}[hbpt]
\fcolorbox{black}{yellow!10}{\begin{minipage}{4.925in}
\caption{\label{ex:multiweight.tension} How many weights?}
While preparing to hang some ornament on a tree, you chain them from a hook on the wall.  You hang ornament 1 from ornament 2 from ornament 3.  What is the tension in each subsequent string?

\color{blue}
The first thing we should do is notice what information is given to us and make sure that everything is in consistent units.  I will convert everything to \hyperref[ss:convertunits]{SI units}.

\color{black}
\autoreturn{sss:multiple.mass}
\end{minipage}}
\end{example}
%
\autoref{ex:multidrag.tension} considers the case of dragging multiple masses, which extends the ideas of \autoref{ss:tension.drag}.
%
\begin{example}[hbpt]
\fcolorbox{black}{yellow!10}{\begin{minipage}{4.925in}
\caption{\label{ex:multidrag.tension} Caravan}
While pulling a sled on which your son sits, your son's sled is tied to a sled on which your dog sits.  Your dog's sled is then connected to a sled with provisions for the day.  What is the tension in each subsequent string?

\color{blue}
The first thing we should do is notice what information is given to us and make sure that everything is in consistent units.  I will convert everything to \hyperref[ss:convertunits]{SI units}.

\color{black}
\autoreturn{sss:multiple.mass}
\end{minipage}}
\end{example}
%
You should note that these examples are essentially expressing the same idea in two different contexts.

\subsubsection{Atwood's Machine}\label{sss:Atwood}

The\dothis{imported a homework problem from Giordano.  Need to modify it to fit my purposes.}{} two crates in the figure (p. 114) hang over a pulley (in what is called an ``Atwood's machine'').  I will select $m_1=35\unit{kg}$ (because it looks smaller) and $m_2=85\unit{kg}$ (because it looks bigger).  We will assume that the pulley is massless and frictionless (so that the tension is the same throughout the rope).  Find the acceleration and the time it takes $m_2$ to accelerate down for the $12\unit m$ to the floor.

\begin{minipage}{1in}
\begin{picture}(100,150)(0,-50)
\put(31,80){\oval(36,36)[t]}
\put(31,80){\circle{33}}
\put(31,81){\line(0,1){29}}
\put(49,80){\line(0,-1){62}}
\put(13,80){\line(0,-1){70}}
%
\put(5,-6){\line(0,1){16}}
\put(5,-6){\line(1,0){16}}
\put(21,10){\line(0,-1){16}}
\put(21,10){\line(-1,0){16}}
\put(6,0){\small $m_1$}
%
\put(41,-6){\line(0,1){24}}
\put(41,-6){\line(1,0){16}}
\put(57,18){\line(0,-1){24}}
\put(57,18){\line(-1,0){16}}
\put(42,0){\small $m_2$}
%
\put(49,-26){\vector(0,1){20}}
\put(49,-26){\vector(0,-1){20}}
\put(51,-26){\tiny $12\unit m$}
\end{picture}
\end{minipage}
\hfill
\begin{minipage}{4.5in}
The easy way to do this is to say that $m_1$ pulls down on the left with $F_{g1} = (35\unit{kg})(9.81\unitfrac{m}{s^2})=\sig{34}{3.4}{N}$ and $m_2$ pulls down on the right with $F_{g2}=(85\unit{kg})(9.81\unitfrac{m}{s^2})=\sig{83}{3.5}{N}$ for a difference of $F_{net} = \sig{49}{0}{N}$ down to the right.  Since this has to move both $m_1$ and $m_2$, the acceleration is
\[ a = \frac{F_{\rm net}}{m_1+m_2} = \frac{\sig{49}{0}{N}}{(35\unit{kg})+(85\unit{kg})} = \frac{\sig{49}{0}{N}}{\sig{120}{}{kg}} = \sigfrac{4.0}{87}{m}{s^2} \]
This acceleration then causes $m_2$ to drop and the time it takes is found from the equation that include distance and time, \\
$y_f \ = \  y_i + v_i \, t + \frac{1}{2} a \,  t^2 $
\[ (0\unit m) \ = \ (12\unit m) + (0\unitfrac ms) \, t + \frac{1}{2} (-\sigfrac{4.0}{9}{m}{s^2}) \,  t^2 \]
\end{minipage}
which we can solve for time:
\[ t \ = \ \sqrt{ \frac{-(12\unit m)}{\frac{1}{2} (-\sigfrac{4.0}{9}{m}{s^2})} } \ = \  \sqrt{ \sig{5.8}{7}{s^2}} \ = \ \sig{2.4}{2}{s} \]

\footnoterule
\small
However, this does not show what the tension is, and many students make a mistake with the tension.  So, I will also answer the question about the tension. We can draw three free-body diagrams. The equation for $m_1$ is as follows, where I am putting the sign in by
\newpar

\begin{minipage}{4.5in}
hand to indicate the direction: \hfill
$\displaystyle (-F_{g1}) + (+F_T) = m_1 (+a) $ \\
The equation for $m_2$ is as follows: \hfill
$\displaystyle (-F_{g2}) + (+F_T) = m_2 (-a) $ \\
Since we know the weights and the masses, these two equations and two unknowns can be written as
\begin{eqnarray*}
(-\sig{34}{3}{N}) + (+F_T) & = & (35\unit{kg}) (+a) \\
(-\sig{83}{3}{N}) + (+F_T) & = & (85\unit{kg}) (-a)
\end{eqnarray*}
There are many ways to solve two equations and two unknowns.
If we subtract the second equation from the first, then we get the equation on the left.
But, if we solve the first equation for $a$ and plug it into the second, then we get the equation on the right
\end{minipage}
\hfill
\begin{minipage}{1in}
\begin{picture}(100,150)(0,-50)
%\put(31,80){\oval(36,36)[t]}
\put(31,80){\circle{33}}
\put(31,81){\vector(0,1){50}}
\put(47.5,80){\vector(0,-1){30}}
\put(14.5,80){\vector(0,-1){30}}
\put(50,55){\tiny $F_T$}
\put(15,55){\tiny $F_T$}
%
\put(5,-6){\line(0,1){16}}
\put(5,-6){\line(1,0){16}}
\put(21,10){\line(0,-1){16}}
\put(21,10){\line(-1,0){16}}
\put(13,4){\vector(0,1){30}}
\put(13,0){\vector(0,-1){20}}
\put(14,20){\tiny $F_T$}
\put(14,-15){\tiny $F_{g1}$}
%
\put(49,8){\vector(0,1){30}}
\put(49,4){\vector(0,-1){40}}
\put(41,-6){\line(0,1){24}}
\put(41,-6){\line(1,0){16}}
\put(57,18){\line(0,-1){24}}
\put(57,18){\line(-1,0){16}}
\put(51,30){\tiny $F_T$}
\put(51,-15){\tiny $F_{g1}$}
\end{picture}
\end{minipage}

\[ \begin{array}{ccc}
\deq
(-\sig{34}{3}{N}) - (-\sig{83}{3}{N}) \ = \ \left[ (35\unit{kg}) + (85\unit{kg}) \right] (a) &&
\deq
(-\sig{83}{3}{N}) + (F_T) \ = \  - (85\unit{kg}) \left[ \frac{(-\sig{34}{3}{N}) + (F_T)}{(35\unit{kg})} \right] \\
\deq
a \ = \ \frac{\sig{49}{0}{N}}{(35\unit{kg})+(85\unit{kg})} = \sigfrac{4.0}{87}{m}{s^2} &&
\deq
F_T \ = \ \frac{-(35\unit{kg})(-\sig{83}{3}{N})-(85\unit{kg})(-\sig{34}{3}{N})}{[(35\unit{kg})+(85\unit{kg})]} \ = \ \sig{48}{6}{N}
\end{array} \]
The acceleration is as above.  The tension is not enough to support $m_2$ (so it falls) and more than enough to lift $m_1$ (so it rises).
You should note that
$\left[(\sig{48}{6}{N}-\sig{34}{3}{N})/(35\unit{kg})=\sigfrac{4.0}{9}{m}{s^2}\right]$
\hfill and \hfill
$\left[(\sig{83}{3}{N}-\sig{48}{6}{N})/(85\unit{kg})=\sigfrac{4.0}{9}{m}{s^2}\right]$.

\normalsize

\subsubsection{Surface Tension}

As a \hypertarget{d:surf.tension}{final note}, \hyperref[s:surface.tension]{surface tension} is something else entirely.  See \autoref{sss:tea} for a comment on the contribution to hot versus cold spoon noises.

\section{Frictional Force}\label{s:Ff}\mmultireturn{\mmr{\ref{A:chair2}}, \mmr{\ref{A:chair6}}, \mmr{\ref{A:chair7}}, \mmr{\autoref{A:fly.balls}}}

%
\begin{reallife}[bthp]
\hspace{-.2in}
\fcolorbox{black}{green!10}{\begin{minipage}{5.29in} \center
\caption{\label{irl:poolfriction}\index{Pool!Real Life} Rolling pool balls and friction.}
\begin{minipage}{4.925in}
\studentD\index{\studentD} is relaxing with the local physics club, playing pool.  \HeD\ hits the cue ball and counts the number of walls \heD\ can hit in one shot.
\end{minipage}
\begin{realtable}
\dna{Hit the cue-ball off of a bumper in the manner intended for
\protect{\href{http://c.ymcdn.com/sites/bca-pool.com/resource/resmgr/imported/BCAEquipmentSpecifications_2008.pdf}{testing cushions}}.}
    {Compare the strength of the hit to the distance travelled}
    {How much is the total distance affected by the number of bumpers hit? \\
     Does it matter if you shoot along the length of the table versus the width of the table?  \\
     Why does friction slow the ball down instead of just make it turn $v=\omega r$ (no slip)}
\end{realtable}
\begin{minipage}{4.925in}
Billiard tables have a lot of interesting physics, which can help us see a wide variety of physics, for example:
\hyperref[irl:poolnormal]{normal force}, \hyperref[irl:poolelastic]{elastic versus inelastic collisions}, \hyperref[irl:poolrotmot]{rotational motion}, and \hyperref[irl:poolangmom]{angular momentum}.
\end{minipage}

%\flushright
%\linkreturn[pool]{d:bank-shot}
\end{minipage}}
\end{reallife}
%

\section{Spring Force}\label{s:springs}\mmultireturn{\mmr{\hyperlink{d:f=ma}{$F=ma$}}, \mmr{\hyperlink{d:usesofF=ma}{uses of $F=ma$}}, \mmr{\autoref{ss:scales}}}

\section{Applied Force}

The term ``an applied force'' is used to describe any force applied by any object when there isn't really a formula to find it.  So this is kind of a ``any other force'' category.  I will use this type of force to describe forces exerted by people.  We have seen some examples where a person throws an object.  We can now revisit those examples and consider the force exerted (applied) by the person who threw the object.
\begin{sample}
\item\label{se:throw-up} \studentC\index{\studentC} recalls that one time \heC\ got bored one day in physics class (what?!?) and tossed a baseball ($m_b = 0.145\unit{kg}$) at the ceiling\ldots a little too hard \ldots as recounted in \autoref{ex:ceiling}.  Recall that \ref{se:ceiling} found the normal force by the ceiling on the ball.  Please now find the force \studentC\ applied while throwing and catching the ball assuming that the throw took $0.200\unit{s}$ to gain the speed of $5.00\unitfrac ms$ and the catch took $0.250\unit s$ to slow the ball from $4.73\unitfrac ms$ to rest.

There are five stages to the motion: (a) throwing, (b) falling up, (c) hitting the ceiling, (d) falling down, and (e) catching show the forces involved. \\
\fbox{\begin{minipage}[b]{55pt}
\begin{picture}(50,100)(0,0)
\put(25,25){\circle{10}}
\put(25,26){\vector(0,1){25}}
\put(25,24){\vector(0,-1){15}}
\put(28,35){$F_\mathrm{throw}$}
\put(28,10){$F_g$}
\end{picture}
\centering{(a) throwing}
\end{minipage}}
\hfill
\color{lightgray}
\fbox{\begin{minipage}[b]{55pt}
\begin{picture}(50,100)(0,0)
\put(25,50){\circle{10}}
\put(25,50){\vector(0,-1){15}}
\put(28,35){$F_g$}
\end{picture}
\centering{(b) falling up}
\end{minipage}}
\hfill
\fbox{\begin{minipage}[b]{55pt}
\begin{picture}(50,100)(0,0)
\put(25,95){\circle{10}}
\put(26,95){\vector(0,-1){25}}
\put(24,95){\vector(0,-1){15}}
\put(28,75){$F_N$}
\put(10,75){$F_g$}
\end{picture}
\centering{(c) \\ hitting}
\end{minipage}}
\hfill
\fbox{\begin{minipage}[b]{55pt}
\begin{picture}(50,100)(0,0)
\put(25,50){\circle{10}}
\put(25,50){\vector(0,-1){15}}
\put(28,35){$F_g$}
\end{picture}
\centering{(d) falling down}
\end{minipage}}
\hfill
\color{rgb:red,0;green,2;blue,1}
\fbox{\begin{minipage}[b]{55pt}
\begin{picture}(50,100)(0,0)
\put(25,25){\circle{10}}
\put(25,26){\vector(0,1){25}}
\put(25,24){\vector(0,-1){15}}
\put(28,35){$F_\mathrm{catch}$}
\put(28,10){$F_g$}
\end{picture}
\centering{(e) catching}
\end{minipage}}
\\
In this particular problem, we are only concerned with steps (a) and (e) because that's where \studentC\ throws and catches the ball. In each case, we need the acceleration: \\
\begin{minipage}[b]{150pt}
\begin{eqnarray*}
\vec a_\mathrm{throw} & = & \frac{(+5.00\unitfrac ms \jhat)-(0\unitfrac ms \jhat)}{0.200\unit s} \\
& = & +\sigfrac{25.0}{0}{m}{s^2} \jhat
\end{eqnarray*}
\end{minipage}
\hfill
\begin{minipage}[b]{150pt}
\begin{eqnarray*}
\vec a_\mathrm{catch} & = & \frac{(0\unitfrac ms \jhat)-(-4.73\unitfrac ms \jhat)}{0.250\unit s} \\
& = & +\sigfrac{18.9}{2}{m}{s^2} \jhat
\end{eqnarray*}
\end{minipage}

During each step, we have the actual acceleration, which tells us about the net force.  We will also need to know the weight of the baseball $F_g=\sig{1.42}{2}{N}$, because gravity is still acting during the collision.  Let's consider the throwing part first.
\begin{eqnarray*}
\vec F_N + \vec F_g & = &  \vec F_\mathrm{net} \ = \ m \vec a \\
\vec F_A  & = &  m \vec a - \vec F_g \\
\vec F_A  & = &  \left[ (0.145\unit{kg})(+\sigfrac{25.0}{0}{m}{s^2}\jhat) \right] - \left[  - \sig{1.42}{2}{N} \jhat \right] \\
\vec F_A  & = &  \left[ +\sig{3.62}{5}{N} \jhat \right] - \left[  - \sig{1.42}{2}{N} \jhat \right] \ = \ +\sig{5.04}{7}{N} \jhat
\end{eqnarray*}
You can see that the upward applied force $(\sig{5.04}{7}{N})$ has to be large enough so that when it is combined with the downward gravitational force $(\sig{1.42}{2}{N})$ they can together result in the necessary (but smaller) upward net force $(\sig{3.62}{5}{N})$ to get it going upwards.

For the catching part, the ball is moving downwards and needs to be stopped, so the catching applied force must be upwards.
\begin{eqnarray*}
\vec F_A + \vec F_g & = &  \vec F_\mathrm{net} \ = \ m \vec a \\
\vec F_A  & = &  m \vec a - \vec F_g \\
\vec F_A  & = &  \left[ (0.145\unit{kg})(+\sigfrac{18.9}{2}{m}{s^2}\jhat) \right] - \left[  - \sig{1.42}{2}{N} \jhat \right] \\
\vec F_A  & = &  \left[ +\sig{2.74}{3}{N} \jhat \right] - \left[  - \sig{1.42}{2}{N} \jhat \right] \ = \ +\sig{4.16}{5}{N} \jhat
\end{eqnarray*}
You can see that the upward applied force $(\sig{4.16}{5}{N})$ has to be large enough so that when it is combined with the downward gravitational force $(\sig{1.42}{2}{N})$ they can together result in the necessary upward net force $(\sig{2.74}{3}{N})$ to stop it from continuing downwards.
\end{sample}

\section{Putting it Together, $F_\mathrm{net}$}\label{s:Fnet}

\subsection{Translational Equilibrium}

blah blah blah
\phantomsection\label{ss:transeq} Translational equilibrium: $F_\mathrm{net} = m \cancelto{0}{a}$.  blah blah blah

\subsection{Static Equilibrium}

\subsection{Dynamic Equilibrium}


\section{Summary and Homework}

\subsection{Summary of Concepts and Equations}\new{v2.3}{Created this section}

\ldots

\subsection*{Conceptual Questions}\new{v2.3}{Added two conceptual problems.}
%\vspace{-24pt}
\begin{enumerate}
\item\label{c:weightmass} Estimate, preferably without using the internet, the mass of the following: (a) a four-door sedan, (b) dishwasher, (c) a pair of glasses, (d) a cell phone.  You should be able to estimate to within one significant digit.
\item\label{c:massweight} List at least one object, preferably without using the internet, that has the following mass: (a) $2500\unit{kg}$ (b) $41\unit{kg}$, (c) $3\unit{kg}$, (d) $50\unit{g}$.
\end{enumerate}
\subsection*{Problems}\new{v2.3}{Created section.}\dothis{Add more problems.}
%\vspace{-24pt}
\begin{enumerate}
 \item\ldots
\end{enumerate}


\chapter{Energy and the Transfer of Energy}

\hypertarget{d:energynoun}{Energy is a noun}\index{Energy!noun}; objects can \textit{have} energy.  \hypertarget{d:workverb}{Work is a verb}\index{Work!verb}\mlinkreturn[heat as a verb]{d:heatverb}; doing work is the process of \textit{exchanging} energy.

\section{Objects Can Have Energy}

\section{A Force Can Transfer Energy} \label{s:work}\mlinkreturn[the direction of forces]{d:pushvector}

\section{Dissipating Energy} \label{s:Wfr}

pool balls on cushion/bumper

\section{Conserving Energy} \label{s:PE}

%
\begin{reallife}[bthp]
\hspace{-.2in}
\fcolorbox{black}{green!10}{\begin{minipage}{5.29in} \center
\caption{\label{irl:poolelastic}\index{Pool!Real Life} 1-D elastic collisions of pool balls.  inelastic collisions off the bumper.}
\begin{minipage}{4.925in}
\studentD\index{\studentD} is relaxing with the local physics club, playing pool.  \HeD\ hits the cue ball and counts the number of walls \heD\ can hit in one shot.
\end{minipage}
\begin{realtable}
\dna{collide balls.}
    {where does it hit}
    {$90^\circ$ output}
\end{realtable}
\begin{minipage}{4.925in}
Billiard tables have a lot of interesting physics, which can help us see a wide variety of physics, for example:
\hyperref[irl:poolnormal]{normal force}, \hyperref[irl:poolelastic]{elastic versus inelastic collisions}, \hyperref[irl:poolrotmot]{rotational motion}, and \hyperref[irl:poolangmom]{angular momentum}.
\end{minipage}

%\flushright
%\linkreturn[pool]{d:bank-shot}
\end{minipage}}
\end{reallife}
%

\subsection{Gravitational Potential Energy}\label{ss:PEg}\mautoreturn{s:PEG}
See also \ref{s:PEG}
\subsection{Spring Potential Energy}\label{ss:PEs}
\subsection{Conservative Forces in General}

\part{Interesting Uses of Motion, Force, and Energy}

\chapter{Momentum: A Better Way to Describe Force}\label{c:momentum}\mmultireturn{\mmr{\hyperlink{d:objectinmotion}{objects in motion}}, \mmr{\autoref{sss:inertia}}, \mmr{\autoref{ss:NIII}}, \mmr{\ref{A:chair6}}}

Useful to include?
\href{https://www.wired.com/2017/06/physics-bullets-versus-wonder-womans-bracelets/}{The Physics of Bullets Vs. Wonder Woman's Bracelets}

\section{Revising Newton's First and Second Laws}

\subsection{Inertia and Momentum}\label{ss:inertia}\mautoreturn{sss:inertia}
Recall \autoref{sss:inertia}.

\section{Revising Newton's Third Law: Conservation of Momentum}\label{s:conservemom}\mautoreturn{ss:NIII}

\section{Two-Dimensional Collisions}\label{s:2Dcollisions}\mautoreturn{sss:vectorequations}

pool balls?  What about rolling?
%
\begin{reallife}[bthp]
\hspace{-.2in}
\fcolorbox{black}{green!10}{\begin{minipage}{5.29in} \center
\caption{\label{irl:pool2Dcollision}\index{Pool!Real Life} 2-D collisions of pool balls.}
\begin{minipage}{4.925in}
\studentD\index{\studentD} is relaxing with the local physics club, playing pool.  \HeD\ hits the cue ball and counts the number of walls \heD\ can hit in one shot.
\end{minipage}
\begin{realtable}
\dna{collide balls.}
    {where does it hit}
    {$90^\circ$ output}
\end{realtable}
\begin{minipage}{4.925in}
Billiard tables have a lot of interesting physics, which can help us see a wide variety of physics, for example:
\hyperref[irl:poolnormal]{normal force}, \hyperref[irl:poolelastic]{elastic versus inelastic collisions}, \hyperref[irl:poolrotmot]{rotational motion}, and \hyperref[irl:poolangmom]{angular momentum}.
\end{minipage}

%\flushright
%\linkreturn[pool]{d:bank-shot}
\end{minipage}}
\end{reallife}
%


\chapter{Rotational Motion}

\section{The Equations of Rotational Motion}

%
\begin{reallife}[bthp]
\hspace{-.2in}
\fcolorbox{black}{green!10}{\begin{minipage}{5.29in} \center
\caption{\label{irl:poolrotmot}\index{Pool!Real Life} Rolling pool balls.}
\begin{minipage}{4.925in}
\studentD\index{\studentD} is relaxing with the local physics club, playing pool.  \HeD\ hits the cue ball and counts the number of walls \heD\ can hit in one shot.
\end{minipage}
\begin{realtable}
\dna{Roll a striped ball along the table.}
    {Use the stripe to notice the rate of rotation}
    {How does the rotation compare to the translation?}
\dna{Roll a striped ball along the table.}
    {Notice the distance the ball travels}
    {Why does friction slow the ball down instead of just make it turn $v=\omega r$ (no slip)}
\end{realtable}
\begin{minipage}{4.925in}
Billiard tables have a lot of interesting physics, which can help us see a wide variety of physics, for example:
\hyperref[irl:poolnormal]{normal force}, \hyperref[irl:poolelastic]{elastic versus inelastic collisions}, \hyperref[irl:poolrotmot]{rotational motion}, and \hyperref[irl:poolangmom]{angular momentum}.
\end{minipage}

%\flushright
%\linkreturn[pool]{d:bank-shot}
\end{minipage}}
\end{reallife}
%

\section{Angular Momentum}

%
\begin{reallife}[bthp]
\hspace{-.2in}
\fcolorbox{black}{green!10}{\begin{minipage}{5.29in} \center
\caption{\label{irl:poolangmom}\index{Pool!Real Life} Rolling pool balls.}
\begin{minipage}{4.925in}
\studentD\index{\studentD} is relaxing with the local physics club, playing pool.  \HeD\ hits the cue ball and counts the number of walls \heD\ can hit in one shot.
\end{minipage}
\begin{realtable}
\dna{Roll a striped ball along the table.}
    {Use the stripe to notice the rate of rotation}
    {How does the rotation compare to the translation?}
\end{realtable}
\begin{minipage}{4.925in}
Billiard tables have a lot of interesting physics, which can help us see a wide variety of physics, for example:
\hyperref[irl:poolnormal]{normal force}, \hyperref[irl:poolelastic]{elastic versus inelastic collisions}, \hyperref[irl:poolrotmot]{rotational motion}, and \hyperref[irl:poolangmom]{angular momentum}.
\end{minipage}

%\flushright
%\linkreturn[pool]{d:bank-shot}
\end{minipage}}
\end{reallife}
%


\section{Non-inertial Rotational Reference Frames} \label{s:noninertial}\mmultireturn{\mmr{\autoref{ss:noninertial}}, \mmr{\hyperlink{d:NewtonInertial}{non-inertial reference frames}}, \mmr{\autoref{ss:NI}}}
\index{Reference Frames!Inertial}
\index{Reference Frames!Non-inertial}

Because the Earth \hypertarget{d:noninertial}{rotates}\mautoreturn{ss:NII}, we are actually in a non-inertial reference frame.  In fact, we can prove that the Earth rotates by observing the effects, such as the \hyperlink{d:coriolis}{Coriolis effect}, that in our non-inertial frame seem to require unexplainable forces but which, in a non-rotating frame, follow the expected laws of physics.

\subsection{The Coriolis Effect}\label{ss:coriolis}\mmultireturn{\mmr{\hyperlink{d:NewtonInertial}{non-inertial reference frames}}, \mmr{\hyperlink{d:noninertial}{Non-inertial Rotational Reference Frames}}}

\hypertarget{d:coriolis}{weather, etc}
\newpar

In her podcast\new{v2.0}{\textit{Spacepod}}, \textit{Spacepod}\footnote{Nugent, Carrie (Producer, Host). \textit{Spacepod} [Audio podcast], episode 89 (19 May, 2017).  Retrieved from \hyperref{http://spacepod.libsyn.com/}{T4LTFdOxHD5WWzdD}{99}{\nolinkurl{http://spacepod.libsyn.com/}}
on 9 Apr. 2017.} Dr. Carrie Nugent interviews Dr. Andy Thompson about ``underwater flying objects'' that investigate the ocean.  He notes that ocean waters, because they are such a large-scale system, can see the effect of the rotation of the Earth.

\subsection{The Foucault Pendulum}\label{ss:Foucault}

See \href{https://www.youtube.com/watch?v=sWDi-Xk3rgw}{youtube video} by \href{http://sixtysymbols.com/}{Sixty Symbols}.\new{v2.0}{Foucault video}




\chapter{Circular Motion and Centripetal Force}

\section{Circular Motion}
\section{Centripetal Force}\label{s:centripetal}\mlinkreturn[$F=ma$]{d:f=ma}




\chapter{Torque and the $F=ma$ of Rotations}\label{c:torque}\mreturn{a:NIIIaction}\new{v2.3}{Added an example that is computable here, but helps introduce normal force in \protect{\autoref{s:FN}}.}

\section{Leverage}\label{s:leverarm}\mautoreturn{ss:scales}

\section{Putting it all together, $\tau_\mathrm{net}$}

\subsection{Rotational Equilibrium}

blah blah blah
\phantomsection\label{ss:roteq} Rotational equilibrium: $\tau_\mathrm{net} = I \cancelto{0}{\alpha}$.  blah blah blah

\subsection{Static (Rotational) Equilibrium}

\subsection{Dynamic (Rotational) Equilibrium}

\new{v2.3}{Answered \protect{\autoref{ex:ladder2}} and its related problems.}
\begin{example}[p]
\fcolorbox{black}{yellow!10}{\begin{minipage}{4.925in}\setlength{\parskip}{3pt}
\caption{\label{ex:ladder2} \studentC\index{\studentC} uses a ladder}
\begin{quote}
\studentC\ leans a $22.7\unit{kg}$ ladder against a wall at an angle of $75.5^\circ$, consistent with \protect{\href{https://www.osha.gov/}{OSHA}} standard \protect{\href{https://www.osha.gov/pls/oshaweb/owadisp.show_document?p_table=standards&p_id=10839}{1926.1053(a)(1)(ii)}}.
The coefficient of friction between the ladder and the floor is $\mu_f=0.31$.
The coefficient of friction between the ladder and the wall is $\mu_w=0.19$.
Use the rotational and translational equilibrium to determine if the ladder slides.
\end{quote}

Since we are asked to distinguish between two cases that cannot both be true, we should assume one (the easier one to calculate is that the ladder does not slip) and then verify that the result is consistent with that assumption.

\textbf{What do we know?}
We know that the floor has a normal force $(F_{Nf})$ upwards and a frictional force $(F_{ff})$ to the left.
We know that the wall
\\[2pt]
\begin{minipage}{3.2in}
has a normal force $(F_{Nw})$ to the right and a frictional force $(F_{fw})$ up (keeping the ladder from sliding down).
We know the weight is
\[ F_g = mg = (22.7\unit{kg})(9.81\unitfrac{m}{s^2}) = \sig{222}{.69}{N} \]
\textbf{What do we want to know?}  We want to know about the the magnitudes of both normal
\end{minipage}
\hfill
\begin{minipage}{100pt}
\begin{picture}(100,100)(-10,-5)
% Dimensions and offset: (width,height)(x offset,y offset)
% Insert picture commands (\line,\circle, etc...) here:
\put(0,0){\line(0,1){100}}
\put(0,0){\line(1,0){75}}
\put(20,0){\color{blue}\line(-1,4){20}}     % ladder
\put(10,40){\color{red}\vector(0,-1){30}}   % Fg
\put(20,1){\color{red}\vector(0,1){25}}     % FNf
\put(19,1){\color{red}\vector(-1,0){12}}
\put(1,80){\color{red}\vector(1,0){12}}
\put(1,81){\color{red}\vector(0,1){8}}
\end{picture}
\end{minipage}
%\hfill {}
\\[3pt]
forces and both frictional forces.
Can we easily deduce the magnitude of $F_{Nf}$? \ref{A:ladderNf}.

\textbf{How are these related?}  The forces acting on any body are related by static \hyperref[ss:transeq]{translational equilibrium}
\begin{eqnarray*}
x: \hspace{.5cm} 0 & = & \cancelto{0}{F_{gx}} + \cancelto{0}{F_{Nfx}} + F_{ffx} + F_{Nwx} + \cancelto{0}{F_{fwx}} \\
y: \hspace{.5cm} 0 & = & F_{gy} + F_{Nfy} + \cancelto{0}{F_{ffy}} + \cancelto{0}{F_{Nwy}} + F_{fwy}
\end{eqnarray*}
and static \hyperref[ss:roteq]{rotational equilibrium}, assuming the pivot point is at the ground, and using the relationship $F_f=\mu F_N$, we find
\begin{eqnarray*}
0 & = & \tau_{g} + \cancelto{0}{\tau_{Nf}} + \cancelto{0}{\tau_{ff}} + \tau_{Nw} + \tau_{fw} \\
0 & = & \left[ F_g \frac{l}{2} \sin 14.5^\circ \right] + \left[ - F_{Nw} l \sin(75.5^\circ) \right] + \left[ - F_{fw} l \sin(14.5^\circ) \right] \\
F_{Nw} & = & \left[ F_g \frac{l}{2} \sin 14.5^\circ \right] / \left[  l \sin(75.5^\circ) + \mu_w l \sin(14.5^\circ) \right]
\end{eqnarray*}
%\textbf{Free-Body Diagrams:}  Since the picture is so simple, we will not draw the free-body diagram.


{}\hfill {\footnotesize\autoref*{ex:ladder2} continued on next page\ldots}
\end{minipage}}
\end{example}
\begin{example}[p]
\fcolorbox{black}{yellow!10}{\begin{minipage}{4.925in}\setlength{\parskip}{3pt}
{\footnotesize \autoref*{ex:ladder2} continued from previous page\ldots}

\textbf{Concepts to Consider:}  First, the length of the ladder cancels from the expression; what matters is the angle at which it is propped.

Second, every force value will be linearly dependent on the mass of the ladder.  So once we solve this problem, we can easily scale the answers to any mass.

Third, the friction with the wall is, by far, the smallest effect and it might be interesting to approximate all of this with $\mu_w=0$.  You can check your calculation against \ref{A:nowall}.

\textbf{Solution to the example:}  When we worry about significant figures,
\begin{eqnarray*}
F_{Nw} & = & \frac{\left[ (\sig{222}{.7}{N})(\txtfrac{1}{2}) (0.\sig{250}{4}{}) \right]}{\left[  (0.\sig{968}{2}{}) + (0.19) (0.\sig{250}{4}{}) \right]}
\ = \ \frac{\left[ (\sig{27.8}{8}{N})\right]}{\left[  (0.\sig{968}{2}{}) + (0.0\sig{47}{6}{}) \right]} \\
F_{Nw} & = & \frac{\left[ (\sig{27.8}{8}{N})\right]}{\left[  (\sig{1.015}{7}{}) \right]}
\ = \ \sig{27.4}{4}{N} \\
F_{fw,\mathrm{max}} & = & (0.19)(\sig{27.4}{4}{N}) \ = \ \sig{5.2}{15}{N} \\
F_{Nf} & = & F_g - F_{fw} = (\sig{222}{.7}{N})-(\sig{5.2}{15}{N}) \ = \ \sig{217}{.5}{N} \\
F_{ff,\mathrm{max}} & = & (0.31)(\sig{217}{.5}{N}) \ = \ \sig{672}{.4}{N}
\end{eqnarray*}
Since $F_{ff} >F_{Nw}$, the friction is sufficient to hold the ladder in place, as assumed.

%\begin{quote}
\textbf{Aside:} Since $F_{ff}$ only needs to be $\sig{27.4}{4}{N}$ to hold the ladder in place, it is possible for the ladder to not slide on a floor that only has
$\mu_\mathrm{min} = (\sig{27.4}{4}{N})/(\sig{217}{.5}{N}) = 0.\sig{126}{2}{}$; but that would not allow a person to climb the ladder.

\textbf{Homework:} Homework problem~\ref{h:ladderC} asks you to determine if the ladder slides when \studentC\ climbs to different locations on the ladder.
%\end{quote}
\flushright
\multireturn{\mmr{\ref{se:ladderN}}, \mmr{\autoref{ss:roteq}}}
\end{minipage}}
\end{example}

\section{Torsion}\label{s:torsion}\mautoreturn{s:FT}\new{v2.4}{Created this section}

\section{Summary and Homework}

\subsection{Summary of Concepts and Equations}\new{v2.3}{Created this section}

\ldots

\subsection*{Conceptual Questions}\dothis{Add conceptual problems.}
%\vspace{-24pt}
\begin{enumerate}
\item\ldots
\end{enumerate}
\subsection*{Problems}\new{v2.3}{Added problems.}\dothis{Add more problems.}
%\vspace{-24pt}
\begin{enumerate}
 \item\label{h:ladderC} \studentC\ leans a $22.7\unit{kg}$ ladder against a wall at an angle of $75.5^\circ$, consistent with \protect{\href{https://www.osha.gov/}{OSHA}} standard \protect{\href{https://www.osha.gov/pls/oshaweb/owadisp.show_document?p_table=standards&p_id=10839}{1926.1053(a)(1)(ii)}}.\new{v2.3}{Answered \protect{\ref{h:ladderC}} and its related problems.}
The coefficient of friction between the ladder and the floor is $\mu_f=0.31$.
The coefficient of friction between the ladder and the wall is $\mu_w=0.19$.
Use the rotational and translational equilibrium to determine if the ladder slides when \studentC\ ($\massC$) climbs to
\begin{enumerate}
\item the third-rung from the top of the ladder, so that he is $1.53\unit m$ from the bottom of the ladder.
    (See \ref{A:nowallC} for that answers if $\mu_w = 0$.)
\begin{ForMe}
\color{blue} Answers:
\begin{eqnarray*}
F_{Nw} & = & \sig{163}{.9}{N} \\
F_{fw,\mathrm{max}} & = & (0.19)(\sig{163}{.9}{N}) \ = \ \sig{31}{.14}{N} \\
F_{Nf} & = & \sig{1074}{.4}{N} \\
F_{ff,\mathrm{max}} & = & \sig{333}{.0}{N} < \sig{163}{.9}{N}
\end{eqnarray*}
$\mu_\mathrm{min} = 0.\sig{152}{56}{}$
\color{black}
\end{ForMe}
\item the third-rung from the bottom of the ladder, so that he is $0.914\unit m$ from the bottom of the ladder.
\begin{ForMe}
\color{blue}
Answers:
\begin{eqnarray*}
F_{Nw} & = & \sig{108}{.97}{N} \\
F_{fw,\mathrm{max}} & = & (0.19)(\sig{108}{.97}{N}) \ = \ \sig{20}{.70}{N} \\
F_{Nf} & = & \sig{1084}{.9}{N} \\
F_{ff,\mathrm{max}} & = & \sig{336}{.3}{N} < \sig{108}{.97}{N}
\end{eqnarray*}
$\mu_\mathrm{min} = 0.\sig{100}{45}{}$

If $\mu_w = 0$.
\begin{eqnarray*}
F_{Nw} & = & \sig{114}{.3}{N} \\
F_{fw,\mathrm{max}} & = & 0 \unit N \\
F_{Nf} & = & \sig{1105}{.6}{N} \\
F_{ff,\mathrm{max}} & = & \sig{342}{.7}{N} < \sig{114}{.3}{N}
\end{eqnarray*}
$\mu_\mathrm{min} = 0.\sig{103}{4}{}$
\color{black}
\end{ForMe}
\end{enumerate}
\end{enumerate}


\chapter{Energy of Rotating Objects}
\section{Rotational Kinetic Energy}
pool balls

\chapter{The Gravitational Force on a Large Scale}\label{c:gravity}\mmultireturn{\mmr{\hyperlink{d:accgrav}{freefall}}, \mmr{\hyperlink{d:fundamental}{fundamental forces}}}

\section{Gravitational Force and Field}\label{s:Gfield}\mlinkreturn[$F=ma$]{d:f=ma}\new{v2.3}{Added some placeholders}

The value of the acceleration due to gravity  varies according to the mass and size of any celestial body.\dothis{Reference a table of $g$ on other planets and compute the weight of a space craft at each planet.}
This means that, as was seen in \ref{se:gworld}, your weight can change even when your mass remains the same.
\begin{sample}
\item\label{se:gplanets} In conversation with a visiting alien, \studentX\index{\studentX}, you find that \studentX\ has been to the moon and several planets both within and outside of our solar system.  In addition to the Earth, \studentX\ has visited our moon, Mars, Pluto, and Planet X.  Using \autoref{t:gplanets}, compute \studentX's weight are each location, assuming \hisX\ mass is \massX.
\begin{enumerate}
\item[Earth] $F_g = (\massX)\left[ \frac{ G M_E}{R_E^2} \right] = (\massX)(9.825\unitfrac{m}{s^2}) \ = \ \sig{933}{.4}{N}$
\item[moon] $F_g = (\massX)\left[ \frac{ G M_m}{R_m^2} \right] = (\massX)(9.782\unitfrac{m}{s^2}) \ = \ \sig{929}{.3}{N}$
\item[Mars] $F_g = (\massX)\left[ \frac{ G M_M}{R_M^2} \right] = (\massX)(9.763\unitfrac{m}{s^2}) \ = \ \sig{927}{.5}{N}$
\item[Pluto] $F_g = (\massX)\left[ \frac{ G M_P}{R_P^2} \right] = (\massX)(9.763\unitfrac{m}{s^2}) \ = \ \sig{927}{.5}{N}$
\item[Planet X] $F_g = (\massX)\left[ \frac{ G M_X}{R_X^2} \right] = (\massX)(9.763\unitfrac{m}{s^2}) \ = \ \sig{927}{.5}{N}$
\end{enumerate}
\end{sample}
%
\begin{table}[bhtp]
\hrule\hrule
\begin{center}
\caption[Properties of various celestial bodies]{\label{t:gplanets} Properties of various celestial bodies.
\return{se:gplanets}
}
\begin{tabular}{lccr}
Planet & Mass (kg) & Mean Radius (m) & $g (\unitfrac{m}{s^2})$ \\
\end{tabular}
\end{center}
\hrule\hrule
\end{table}
%


\subsection{Inertial Mass versus Gravitational Mass}\label{ss:equivmm}\mautoreturn{ss:weightmass}\new{v2.2}{Moved this here, might need to move it back.}

\section{Gravitational Potential Energy} \label{s:PEG}\mautoreturn{ss:PEg}

Recall \ref{ss:PEg}

\section{Making Connections}\label{s:Gconnection}\mautoreturn{s:Econnection}

\[ \begin{array}{ccccc}
& & \vec F = m \vec g & & \\
& \deq F = G \frac{m_1 m_2}{R^2} & \leftrightarrow & \deq g = G \frac{m}{R^2} & \\
\Delta \PE = -\vec F \cdot \Delta\vec x & \updownarrow & & \updownarrow & \mbox{\scriptsize [for later]} \\
& \deq \PE = G \frac{m_1 m_2}{R} & \leftrightarrow & \mbox{[for later]} & \\
& & \mbox{\scriptsize [for later]} & &
\end{array} \]
(Look ahead to the parallel with the electrical interaction in \autoref{s:Econnection}.)

\section{Orbits}


\part{Making Waves}

\chapter{Fluids}\new{v2.2}{Placeholder}
\section{Density}\label{s:density}\index{Density}\mautoreturn{ss:weightmass}

\section{Surface Tension}\label{s:surface.tension}\mlinkreturn{d:surf.tension}



\chapter{Oscillations}\label{c:SHM}
\section{Oscillating Springs}\label{c:SHMspring}\mlinkreturn[$F=ma$]{d:f=ma}
\section{Oscillating Pendulums}\label{c:SHMpend}

\section{Other Examples of Oscillations}\label{s:SHMother}

On 13 April, 2017,\new{v2.3}{New source of info}
\href{http://www.cbc.ca/podcasting}{CBC Broadcasting} published a
\href{http://www.cbc.ca/podcasting/includes/quirks.xml}{\textit{Quirks and Quarks}} episode discussing how we can find
\href{https://podcast-a.akamaihd.net/mp3/podcasts/quirks_20170415_12100.mp3}{solutions to health issues caused by swaying office towers and vibrating floors}.

\chapter{Sound}
\subsection{Musical Instruments}\label{ss:stringed.instruments} \mautoreturn{A:swing.tension}



\part{Is It Hot in Here?}

\chapter{The flow of thermal energy}

\phantomsection\label{find:heatwarm}
Energy is a noun\index{Energy!noun}; objects can \textit{have} energy.  \hypertarget{d:heatverb}{Heat is a verb}\index{Heat!verb}; heating is a process of \textit{exchanging} energy.  Recall our \hyperlink{d:forcenoun}{discussions of force}\index{Force!noun} and \hyperlink{d:workverb}{work}\index{Work!verb}.

\section{Specific Heat Capacity}\label{s:specificheat}

\hypertarget{d:heatwarm}{Heating (positive $Q$)} can warm (positive $\Delta T$) a material.
\begin{equation}\label{eq:Q=mcDT}
Q = m c \, \Delta T
\end{equation}
but \autoref{eq:Q=mL} (as one example) shows that it is possible to heat (positive $Q$) a material without warming it (constant $T$). When we get to \autoref{s:PV} we will see other examples of ``isothermal processes'' that have a non-zero $Q$ (heat the system or heat the surroundings) without warming or cooling the system.

\section{Latent Heat}

Heating might also change the phase of a material.\mlinkreturn[heating versus warming]{d:heatwarm}
\begin{equation}\label{eq:Q=mL}
Q = \pm mL
\end{equation}

\section{The Flow of Thermal Energy}

\subsection{Thermal Conductivity}\label{ss:thermalconductivity}\mautoreturn{s:story}

\begin{equation}\label{eq:thermalconductivity}
\frac{Q}{\Delta t} = \kappa A \, \frac{\Delta T}{\Delta x}
\end{equation}

\begin{example}
\fcolorbox{black}{yellow!10}{\begin{minipage}{4.925in}
\caption{\label{ex:baking}\studentA\protect{\index{\studentA}} warms \hisA\ oven.}
\studentA\protect{\index{\studentA}} decides to bake some bread for the dinner party at \studentB\protect{\index{\studentB}}'s house, but \heA\ is on a tight schedule.  In order to set \hisA\ schedule, \heA\ needs to know how long it will take \hisA\ oven to \hyperref[find:heatwarm]{warm up}.

\autoreturn{s:story}
\end{minipage}}
\end{example}

\subsection{Convection}
\subsection{Radiation}

\chapter{Ideal Gas Law}
\section{$P$-$V$ Diagrams}\label{s:PV}\mlinkreturn[heating versus warming]{d:heatwarm}

\part{Let There be Light!}

\chapter{The Electrical Interaction}\label{c:electric}\mlinkreturn[fundamental forces]{d:fundamental}
\section{Electrical Charge}\label{s:Echarge}\new{v2.1}{Decide where this should go.}

\section{The Big Picture}

\subsection{Electric Forces and Fields}\label{ss:Efield}\mmultireturn{\mmr{\autoref{sss:vectorequations}}, \mmr{\hyperlink{d:f=ma}{$F=ma$}}}

pst-electricfield

\subsubsection{Examples}

\subsection{Electric Forces, Fields, and Potential Energy}

\subsection{Electric Fields, Potential Energy, and Potential}

\section{Making Connections}\label{s:Econnection}\mautoreturn{s:Gconnection}

\[ \begin{array}{ccccc}
& & \vec F = q \vec E & & \\
& \deq F = k \frac{q_1 q_2}{r^2} & \leftrightarrow & \deq E = k \frac{q}{r^2} & \\
\Delta \PE = -\vec F \cdot \Delta\vec x & \updownarrow & & \updownarrow & \Delta V = -\vec E \cdot \Delta\vec x  \\
& \deq \PE = k \frac{q_1 q_2}{r} & \leftrightarrow & \deq V = k \frac{q}{r} & \\
& & \Delta \PE = q \Delta V & &
\end{array} \]
(Recall the parallel with the gravitational interaction in \autoref{s:Gconnection}.)

\chapter{Electricity}

\chapter{The Magnetic Interaction}

pst-magneticfield

\chapter{``Magnicity?''}

\chapter{Light}

\chapter{Optics}

\part{What Have You Done for Me Lately?}

\chapter{Relativity}
\chapter{Quantum Mechanics}\new{v2.1}{Decide if these subsections should be chapters in and of themselves.  These are now labeled.}
\section{Atomic Physics} \subsection{The Periodic Table and Quantum Numbers}
\section{Nuclear Physics} \subsection{Nuclear Decay}\label{ss:nucleardecay}
\subsection{The Strong Nuclear Force}\label{ss:strong}\mlinkreturn[fundamental forces]{d:fundamental}
\subsection{The Weak Nuclear Force}\label{ss:weak}\mlinkreturn[fundamental forces]{d:fundamental}
\section{Particle Physics}\label{s:particle}
\subsection{Field Theory}
\subsection{Quantum Electrodynamics}\label{ss:QED}\mlinkreturn[fundamental forces]{d:fundamental}
\subsection{Quantum Chromodynamics}\label{ss:QCD}\mlinkreturn[fundamental forces]{d:fundamental}
\subsection{The Standard Model}\label{ss:StandardModel}
\subsection{Particle Decay}\label{ss:particledecay}
\chapter{Condensed Matter}
\chapter{Astronomy}
\chapter{Cosmology}

\part{Supplements}

\chapter{Deeper Dive}\label{c:revisted}\new{v2.1}{This chapter should mirror \protect{\autoref{c:physics}}.}

This is where I will put the fuller explanations.

\subsection{The Sun}\label{sss:sun}
The bright, shiny sun, which keeps us all alive, is a nice example of a rather complex system that allows us to verify our various theories of the world around us.  We can consider the existence of a star in three phases: the birth of a star, the life of the star, and the death of the star.

\subsubsection{The Birth of a Star}
\subsubsection{The Life of a Star}
\subsubsection{The Death of a Star}


\subsection{Kitchen Appliances}
\subsubsection{Oven}
\subsubsection{Refrigerator}
\subsubsection{Microwave}
\subsubsection{Television}

\subsection{Automobile}
\subsubsection{Coolant and Antifreeze}
\subsubsection{Tires}
\subsubsection{Torque}

\subsection{Cool Ideas}
\subsubsection{Black Holes}\label{sss:blackhole2}\mautoreturn{ss:weightmass}

On 7 April, 2017,\new{v2.3}{New source of info}
\href{http://www.cbc.ca/podcasting}{CBC Broadcasting} published a
\href{http://www.cbc.ca/podcasting/includes/quirks.xml}{\textit{Quirks and Quarks}} episode discussing how we can
\href{https://podcast-a.akamaihd.net/mp3/podcasts/quirks_20170408_51226.mp3}{turn our planet into a giant telescope to get a photo of a black hole}.
The results should be available by the early 2018.\dothis{Follow-up in 2018 to find the results.}

\subsubsection{Quantum Mechanics}
\subsubsection{Relativity}
\subsubsection{String Theory}



\chapter{Podcasts and Videos}\label{c:videos}\label{c:podcasts}

\section{Podcasts}\label{s:podcasts}
\hyperref{http://spacepod.libsyn.com/}{T4LTFdOxHD5WWzdD}{99}{Spacepod with Carrie Nugent} \\
\href{http://www.sciencefriday.com/}{Science Friday with Ira Flatow}

\section{Videos}\label{s:videos}
\href{http://physicsfootnotes.com/}{Physics Footnotes} \\
\href{http://sixtysymbols.com/}{Sixty Symbols}

\section{Websites}\label{s:websites}
\href{http://www.aldakavlilearningcenter.org/practice/flame-challenge}{The Flame Challenge}

\chapter{Answers to Interactive Questions}

\begin{AIQ}
\item\label{A:hbf} There are forces acting on it.  You should be able to tell this because you are exerting one of the forces. While it is true that there are forces on it, it is also true that there is no \textit{net force}.  If you are exerting an upward force on the book, can you guess (\ref{A:gravity}) what the downward force is?   \return{IQ:holdbook}
\item\label{A:chair1} If we refer to ``motion'' as describing the velocity, then no. Force causes a \textit{change in} velocity. When you stop pushing, the chair stops because there is a force from the carpet acting to oppose the force you apply while you push the chair. \autoreturn{irl:NI}
\item\label{A:chair2} This is essentially the same as \ref{A:chair1}, but the carpet exerts more force than the tile.  In either case, force causes a \textit{change in} velocity. You are trying to speed the chair up and the floor is trying to slow the chair down.  (Both are trying to change the velocity, but cancel to result in a constant velocity.)  When you stop pushing, the chair stops moving because there is a force from the tile acting to oppose the force you apply while you push the chair; when you let go, this force slows the chair until the chair stops and then the force stops acting. (See \autoref{s:Ff} for more details.) \autoreturn{irl:NI}
\item\label{A:weight.loss} Since \studentB\index{\studentB} weighs $(\massB)(9.81\unitfrac{m}{s^2})=736\unit{N}$, $45\unit{N}$ is about $6\%$ of her weight.  This is fairly substantial.  You should compute how much $6\%$ of your weight is and convert that to kilograms and Newtons.  \autoreturn{irl:scale}
\item \label{A:ladderNf} Since the full weight of the ladder, $F_g = \sig{222}{.69}{N}$, is still pressing downwards into the floor (as a normal force), it is tempting to say that \hyperref[ss:NIII]{Newton's third law} implies that the floor pushes the ladder upwards with a normal force of $\sig{222}{.69}{N}$ but this would not account for the frictional force on the wall, $F_{fw}$.  If there were no friction between the ladder and the wall, then we could deduce $F_{Nf}$, but at this point, we cannot. \autoreturn{ex:ladder2}
\item\label{A:hbnof}  It is true that while you hold the book, there is no \textit{net force}, but that does not mean that there is no force acting.  If there were no forces on the book, then your hand would not need to be there.  In fact, if you remove the force your hand provides, then the book falls. This shows that there is an upward force (by your hand on the book) and a downward force (of gravity by the Earth on the book).  \return{IQ:holdbook}
\item\label{A:netF-a} Since the object in \ref{se:netF-a} has a mass of $2.0\unit{kg}$, we can find the weight by
    \[ \vec F_g = m \vec g = (2.0\unit{kg}) [-(9.81\unitfrac{m}{s^2})\,\jhat] = -\sig{19}{.62}{N} \jhat = -20 \unit N \jhat \]
    \return{se:weightA}
\item\label{A:chair3} For a chair with wheels being pushed across a tile floor, when you stop pushing it probably continues to move across the floor for at least a short distance.  \autoreturn{irl:NI}
\item\label{A:weight.gain} When one person stands on the scale, the scale provides just enough of an upwards normal force to keep that person in equilibrium\dothis{link equilibrium?}.  In that case, the upwards force is balancing the weight of the person.  This gives the impression that the scale is telling you your weight; however, when you press down or help support whomever is standing on the scale, the scale adjusts the amount it must provide.  The scale is not trying to tell you your weight.  Rather the scale is trying to create equilibrium by balancing whatever force(s) are pressing into it.  Your weight is determined by the gravitational force\dothis{link the gravitational force?}{} and does not change when you press harder or lighter onto the scale.  \autoreturn{irl:scale}
\item\label{A:nowall} If we consider $\mu_w\rightarrow 0$, then $F_{fw}=0\unit N$,  $\vec F_{Nf} = -\vec F_g = \sig{222}{.7}{N} \jhat$, and $\vec F_{Nw} = - \vec F_{ff} = \sig{28.7}{9}{N} \ihat$.  In this case, $\mu_f$ could be as small as $0.\sig{129}{3}{}$ and still hold the ladder in place, unless \studentC\index{\studentC} climbs the ladder, in which case see \ref{A:nowallC}.  \autoreturn{ex:ladder2}
\item\label{A:true1} It is in equilibrium.  When the acceleration is zero, then the net force must be zero and those properties are what define equilibrium. \return{IQ:holdbook}
\item\label{A:chair4} The chair continues to move for the same reason that the chair without wheels and the chair on carpet \textit{all} continued to move when you let go.  The reason is that this is \textit{how all objects behave; they maintain their velocity when allowed to act without interference.}  (This is why Newton's first law says what it does.)   Because the chair with wheels has much less friction there is a smaller force trying to interfere with the motion and so it continues to move for a noticeable distance. The other chairs slowed to a stop almost immediately.  The wheel-less chair on tile might have continued for a short distance if it was moving fast enough that it required a long enough time to change its velocity to zero.  \autoreturn{irl:NI}
\item\label{A:scale.increase} When you press down on \studentB's shoulders, you are not adding weight.  Weight has a specific definition: it is specifically the value that the gravitational force\dothis{link the gravitational force}{} pulls on any object.  Pushing the person does not change their weight; it does, however, change the amount that they press into the Earth.  That is to say, it increases their downwards normal force, but not their weight.  \autoreturn{irl:scale}
\item\label{A:nowallC} If we consider $\mu_w\rightarrow 0$ with \studentC\index{\studentC} ($m=\massC$) at the third-rung-from-the-top of the ladder, ($1.53\unit m$ up the ladder), then $F_{fw}=0\unit N$,  $\vec F_{Nf} = \sig{1105}{.6}{N} \jhat$, and $\vec F_{Nw} = - \vec F_{ff} = \sig{171}{.97}{N} \ihat$.  In this case, $\mu_f$ could be as small as $0.\sig{155}{5}{}$ and still hold the ladder in place. \multireturn{\mmr{\ref{A:nowall}}, \mmr{\autoref{ex:ladder2}}}
\item\label{A:gworld} Because the Earth was spinning as it cooled (forming the crust), it formed an oblate spheroid\footnote{The equator is slightly further from the center than the poles are.}.  Since the strength of the gravitational interaction depends (among other things) on how far you are from the center (slightly weaker further away), the acceleration due to gravity is smaller when you are at smaller latitudes (closer to the equator).  \multireturn{\mmr{\ref{A:gpeaks}}, \mmr{\autoref{t:gworld}}}
\item\label{A:false1} The definition of equilibrium is that the forces balance.  The result of this is that the net force must be zero and the acceleration is then zero.  You can tell this is true because the velocity is \textit{not changing}.  It is not important that the velocity is zero, what is important is that the velocity \textit{stays} zero.  While you hold it, the book is in equilibrium. \return{IQ:holdbook}
\item\label{A:chair5} No. But if it does not matter what you do after you let go of the chair, then why do coaches (in basketball free-throws, tennis serves and swings, baseball pitches, and all manner of arm and leg propulsion) tell you to pay attention to your ``follow through''? \TWO{They have been fooled; follow-through doesn't matter}{they are right; follow-through does matter!}{A:noFT}{A:FT} \autoreturn{irl:NI}
\item\label{A:scale.measure} Since your weight is a force pulling downwards, having the scale on the wall shows that the scale cannot be balancing weight.  Since you are pushing into the wall, you are exerting a normal force into the scale and the scale is exerting a normal force back at you.  Both of these forces are horizontal (assuming the wall is plumb).  \autoreturn{irl:scale}
\item\label{A:gpeaks} In addition to being an oblate spheroid (\ref{A:gworld}), the Earth has mountains and valleys.  Since the strength of the gravitational interaction depends (among other things) on how far you are from the center (slightly weaker further away), the acceleration due to gravity is smaller when you are at at high altitudes, such as Denver, CO and Mount Everest.  \autoreturn{t:gworld}
\item\label{A:falls}  Both ``Yes'' and ``No'' bring you to this answer.  Yes, there is \underline{a force} on the book while it falls (the force of gravity), but no, there are not force\underline{s} (plural).  There is only one force.  ``But, wait!'' you say, ``What about the force of air resistance?''  Aha!   You are correct; there is a force of air resistance, but in this case, it is negligible and we will not consider it.  Please read \autoref{ss:airresistance} for more information about deciding when to use or ignore this phenomenon.  \return{IQ:holdbook}
\item\label{A:chair6} No.  When you throw a ball very high into the air, you can dance a jig or do any manner of things and it will obviously not affect the ball.  The force you exerted on the chair goes away the instant you stop touching the chair.  It is, however, true that your force gave the chair some velocity (actually \hyperref[c:momentum]{momentum}) and Newton's first law (inertia) says that the chair would prefer to keep that velocity.  Unfortunately, the friction with the ground slows it down.  The careful way to describe the situation is that your force gave the chair some velocity (actually \hyperref[c:momentum]{momentum}) and its characteristic inertia made it difficult for the \hyperref[s:Ff]{frictional force} to slow it down rapidly. \autoreturn{irl:NI}
\item\label{A:fly.balls} If you watch them carefully, you will notice that long fly balls are not parabolic.  It turns out that the air resistance is fairly complicated, but in the case of baseballs, the part that is relevant is that air resistance is strong when the ball is moving faster and weak when the ball is moving slower.  (This is different than the surface friction you will see in \autoref{s:Ff}.)  The effect of this is that the ball (usually) looks like it travels up into the air on a fairly straight path with a slight bend, which would produce a very wide parabola.  As it slows, the horizontal motion decreases, which tightens the parabola.  By the time the ball gets to its highest point, it is often travelling fairly slowly and has mostly all vertical motion by the time it drops into the outfielder's glove. \autoreturn{irl:nonparabolic}
\item\label{A:hitY} The book is accelerating.  The velocity \underline{is changing} from ``moving downwards'' to ``stopped''.  The book is not in equilibrium.  \return{IQ:holdbook}
\item\label{A:noFT} They haven't been fooled, but follow-through matters in a different way.  What does matter is not literally how you move \textit{after} the release, but rather how you move \textit{before} you release the ball. By paying attention to your follow-through, you are also changing the way you move before you release or impact the ball.  You want a smooth flow throughout the motion and a sloppy follow-through often implies a sloppy initiation of the motion.  \return{A:chair5}
\item\label{A:scale.ramp} When the scale is on the flat, horizontal floor, it balances your full weight.  When the scale is on the vertical wall it does not carry any of your weight.  At any angle in between those values, it carries some fraction of your weight while friction keeps you from sliding down the ramp\dothis{link to the section on friction and ramps}.  It will turn out that since the cosine function\dothis{link to the trig section}{} behaves in just the right way, we can use\dothis{link ``can use'' to the section on ramps}{} the cosine to find the component of the weight that the normal force from the scale has to support.   \autoreturn{irl:scale}
\item\label{A:pitches.side} The way a pitch travels is highly dependant on the way the pitcher releases the ball.  As the ball rolls out of the pitcher's hand, a spin is (usually) given to the ball and this spin interacts with the air to modify the direction that the air presses on the ball during the flight.  This will slightly affect the flight of the ball during the time it takes for the ball to get from the pitcher's mound to home plate.  In addition, fast balls have less time for the gravitational force to pull the ball down, so they will curve downwards less than a slower pitch.  This makes following the path of the ball somewhat difficult, but with some practice and careful attention, you should be able to see it.  All balls will drop somewhat, but the effect of the air resistance is exactly the mechanism for making a pitch unpredictable, so it is unlikely that you see the ball drop in a clean parabolic path.  \autoreturn{irl:nonparabolic}
\item\label{A:hitN} While it is hitting the desk, the velocity is changing from ``moving downwards'' to ``stopped''.  Since the velocity is changing, \underline{the book is accelerating}.  Since it is accelerating, the book is not in equilibrium.  \return{IQ:holdbook}
\item\label{A:chair7} The force that the chair feels after you release it is \hyperref[s:Ff]{friction}.  For the carpet, there is a lot of friction and the chair slows down very quickly (essentially instantaneously).  For the wheel-less chair on the tile floor, the chair slows rapidly although it may leave your hand.  The wheels provide the least amount of friction and that chair goes the furthest.  You may note that the friction slowing the chair-with-wheels is primarily between the rolling wheel and its axel (where it connects to the non-rolling chair leg) rather than between the wheel and the floor (although the friction between the wheel and the floor also plays a role).  This is discussed in more detail in \autoref{s:Ff}. \autoreturn{irl:NI}
\item\label{A:pitches.top} The way a pitch travels is highly dependant on the way the pitcher releases the ball.  As the ball rolls out of the pitcher's hand, a spin is (usually) given to the ball and this spin interacts with the air to modify the direction that the air presses on the ball during the flight.  In many cases, this will affect the flight of the ball (especially to the right or to the left) during the time it takes for the ball to get from the pitcher's mound to home plate.  If you watch from behind home plate, this sideways motion should be fairly clear.  \autoreturn{irl:nonparabolic}
\item\label{A:landedY} It is in equilibrium.  The book is at rest and \textit{continues to be} at rest on the desk. There are forces acting, but they cancel each other, resulting in no net force.  \return{IQ:holdbook}
\item\label{A:gravity}  It is the force of gravity. \return{A:hbf}
\item\label{A:chair8} If there were no friction, then you could start the chair and it would move on its own at a constant speed; you wouldn't need to continue pushing to keep it moving.  On the other hand, if you did continue to push, then the chair would continue to speed up and you would have to run faster and faster to keep up with it. On the other hand, if the chair were not experiencing friction, then you probably wouldn't either and you couldn't get enough traction to keep up with the chair, so it would sail away almost immediately, being then described by Newton's first law! \autoreturn{irl:NI}
\item\label{A:pool.roll} First, you should not roll a pool ball across just any floor; there is felt on the pool table for a reason.  However, if you have a clean, smooth surface and are able to reproduce your rolling speed, you will find that the pool ball rolls further on the stiff, nonyielding surface than it will on the felt.  The reason for this is beyond the scope of this textbook, but you can read more from \href{http://stacks.iop.org/0031-9120/30/i=3/a=009}{``Sliding and rolling: the physics of a rolling ball,'' J. Hierrezuelo and C. Carnero, Physics Education, Volume 30, Number 3} (unofficially at \href{http://billiards.colostate.edu/physics/Hierrezuelo_PhysEd_95_article.pdf}{this PDF}). \autoreturn{irl:poolcushion}
\item\label{A:landedN} After it has landed, the book stops moving.  Once the book comes to rest on the desk, it \textit{continues to stay at rest}.  This says that the velocity is not changing, so the book is not accelerating.  That means that the book is in equilibrium. There are forces acting, but they cancel each other, resulting in no net force. \return{IQ:holdbook}
\item\label{A:FT} What does matter is not literally how you move \textit{after} the release, but rather how you move \textit{before} you release the ball. By paying attention to your follow-through, you are also changing the way you move before you release or impact the ball.  You want a smooth flow throughout the motion and a sloppy follow-through often implies a sloppy initiation of the motion.  \return{A:chair5}
\item\label{A:pool.bumper} The cushion (sometimes called a bumper) is pretty still to the touch, but it is made of a springy rubber that allows the balls to bounce reasonably well.  The \protect{\href{http://c.ymcdn.com/sites/bca-pool.com/resource/resmgr/imported/BCAEquipmentSpecifications_2008.pdf}{document}} indicates that you should be able to firmly strike a ball at some angle to the far wall and have it bounce around the table four to four-and-a-half times.  If the bumpers were perfectly \hyperref[s:elastic]{elastic}, then the normal force would be normal to the restign surface; but since the bumper has some flexibility, when the ball hits the bumper with a glancing blow, then bumper bends inwards and the normal force is directed in a way that depends on the shape of the dent.  \autoreturn{irl:poolcushion}.
\item\label{A:zero} Recall the situation when you were holding the book.  Gravity is still pulling the book down and the desk is holding the book up.  There are two forces acting on the book while it is at rest on the desk. \return{IQ:holdbook}
\item\label{A:firstfall} As long as you are careful about releasing at the same time, you should not see any object consistently land first or consistently land last.  It is true that a piece of paper  will consistently land last, but this is because of the air resistance that we previously agreed to avoid.  If you crumple the paper into a tight ball (yes, it has to be a tight ball), then this will minimize the effect of air resistance and you might still be able to make the comparisons.   It is possible that some of the objects you are dropping (such as those in \ref{A:firstwhy}) have a shape that makes air resistance relevant.  \autoreturn{irl:freefall}
\item\label{A:floor}  I hope you guessed the floor.  That is the only thing pushing up on \studentB\index{\studentB}.  One useful way to think about it is that the floor is the thing keeping \himB\ from falling.  The direction of this force is \textit{normal} (perpendicular) to the horizontal floor, so it is in the vertical direction.  This will be discussed in more detail in \autoref{s:FN}. \return{se:FNB}
\item\label{A:firstwhy} As long as you are careful about releasing at the same time, it is unlikely that you will find anything consistently falling faster or slower than the others.  If you do notice a pattern, then the likely culprit is that air resistance is having an effect.  If you have something flat, like a computer (!) or a book that is falling more slowly than something else, like a hammer, then try dropping the flat object in different orientations to see if that affects the air resistance.  If you have something somewhat cylindrical, like a wine bottle (!) or a pencil that is falling more quickly than something else, like a hammer, then try dropping the cylindrical object in different orientations to see if that affects the air resistance. Remember that we are trying to eliminate differences due to air resistance so that we can study the effect of the gravitational force. \textbf{The effect you should notice is that so long as air resistance does not affect one object differently than the other, all objects fall at the same rate.}  \autoreturn{irl:freefall}
\item\label{A:noncue} The \href{http://wpapool.com/equipment-specifications/\#Balls-and-Ball-Rack}{specifications} show that there is no difference between the solids and stripes, but the cue ball weighs $9\%$ more that the other balls ($6.0\unit{oz}$ versus $5.5\unit{oz}$).  The colored balls and the cue ball are otherwise identical.  \autoreturn{irl:poolcushion}
\item\label{A:one} If there were only one force on the book, it could not be a balanced force, so the book could not be in equilibrium and the book would be accelerating.  The book is not accelerating, so there are either two forces (\ref{A:two}) or no forces (\ref{A:zero}).  \return{IQ:holdbook}
\item\label{A:fallv} When you drop something from eye-level, it takes less than a half-second for it to hit the ground.  Due to the limited need to gauge speed, it is very difficult for most humans to distinguish constant speed from accelerated motion in this small of a time interval.  Athletes can often tell is an object is moving fast or slow, but even then it is difficult to gauge acceleration.  Practice measuring the time-of-flight by counting out loud: ``one-one-thousand\ldots two-one-thousand\ldots''.  For this fall, you will likely only get to ``one-one-thou''. \autoreturn{irl:freefall}
\item\label{A:pool.spin} Because the bumper is covered in felt, it has a small grip on the ball.  Because the bumper has some give to it, it dents in when hit and provide more surface area, which increases the grip.  Both of these mean that the spin of the ball gets transferred to the pool table somewhat and change the way a spinning ball exits from the bumper collision.  \autoreturn{irl:poolcushion}
\item\label{A:second} You can tell that it is Newton's second law $(\vec F_\mathrm{net} = m \vec a)$ because the forces we are considering are acting on the \textit{same} object.  In this case, the gravitational force is caused by the Earth and the normal force is caused by the floor by they are both felt by \studentB\index{\studentB}.  These forces happen to be equal and opposite because \heB\ happens to be in equilibrium. \HeB\ does not \textit{have to be} in equilibrium, such as when \heB\ jumps, in which case the forces would not be equal and might not be opposite. \return{se:FNB}
\item\label{A:falla} Measure the time-of-flight by counting out loud: ``one-one-thousand\ldots two-one-thousand\ldots''.  For the four rungs near the top, you will likely only get to ``one-one-thou''.  For the four rungs near the bottom, you will likely only get to ``one-wa''.  Since those two distances are the same, it should be clear that the object is going faster at the bottom of the ladder. \textbf{Objects speed up (accelerate) while they fall.}  \autoreturn{irl:freefall}
\item\label{A:pool.later} This answer is getting \important{too complex for the section} it is in.  I need to move the IRL before I finish considering how to answer this question. \autoreturn{irl:poolcushion}
\item\label{A:two} There are two forces acting on the book while it is at rest on the desk. Similar to the situation when you were holding the book, gravity is pulling the book down and the desk is holding the book up.  \return{IQ:holdbook}
\item\label{A:third} If it were Newton's third law, then the two forces we were discussing would be acting on different objects and would be unrelated to the fact that the object (in this case, \studentB\index{\studentB}) is in equilibrium.  The gravitational force and the normal force in this case are both acting on \studentB, so although they happen to be equal and opposite, this is not due to Newton's third law.

    You should, however, note that the force that is reaction-paired to the gravitational force on \studentB\index{\studentB} by the Earth is a gravitational force on the Earth by \studentB.  Similarly, the reaction-paired force to the normal force on \studentB\ by the floor is a normal force on the floor by \studentB.  (Please note the ``on'' and ``by'' in each case.) \return{se:FNB}

\item\label{A:swing.tension} To make this comparison, let's consider a swing that is supported by chains.  If you are sitting in the swing and take hold of the chains at about shoulder height, you should be able to shake them in (towards your chest) and out (away from you, towards your neighbor swings).  You can do this same motion while standing next to the swing.  If you do this when the swing is empty, it is very easy to do this.  If you ask a series of successively larger people to sit in the swing, you will notice that it gets progressively more difficult to extend them very far.  The chains are increasing in tension; they are pulled more taut.  Your ability to move the chain in this way is exactly analogous to the way a bow draws across a violin or the way your fingers pluck a guitar, as described in \autoref{ss:stringed.instruments}. \multireturn{\mmr{\autoref{irl:tension}}, \mmr{\ref{A:chandelier.tension}}}
\item\label{A:fan.tension} You might also consider \ref{A:chandelier.tension}, which discusses the case of hanging a light fixture from the ceiling. If you have ever installed a fan in your house, then you will notice that you have to support the fan while the wires are connected.  Usually the fan has a shaft that connects to the ceiling at one end and the fan at the other and provides a mechanism for supporting the fan while you manage the wires, which pass through the shaft.  Since the fan houses the motor, it is usually reasonably heavy.   The nice property of use a metal shaft to support the fan is that it doesn't stretch or wiggle like a chain might.  The difficulty in this example is that it is more difficulty to notice the tension in the shaft.  If you are the person hanging the fan, then one thing you might be able to notice is that if you flick the metal with a finger when it is not supporting anything, it will have a slightly different ``ting'' than when it is supporting the fan.\dothis{Is this sufficiently noticeable?} \autoreturn{irl:tension}
\item\label{A:chandelier.tension} If you have ever installed a chandelier in your house, then you will notice that the light has to be supported between the joists of the ceiling.  There will be an electrical box with a screw to which you will attach the support for the chain that holds the chandelier.  The wires will run through the support chain.  The heavier the chandelier, the tauter the chain, much as described in \ref{A:swing.tension}. This tension is much easier to see than the tension in the shaft of the fan.  \return{A:fan.tension}
\end{AIQ}

\chapter{Adventures}

Throughout the book, there are examples and adventures.  The follow-up stories are contained below.
\begin{Story}
\item\label{a:parkandwalk} Just as planned, you pull over and park the car.  \studentB\index{\studentB} suggests one of you stays with the car, probably because \heB\ has physics homework to do.  If you decide to separate, read \ref{a:nogas}.  If you decide to journey together, read \ref{a:intosunset}.
\item\label{a:NIIIaction} As \studentC\index{\studentC} gets pushed, you notice that \heC\ was not aware of the pending doom.  \HeC\ is standing casually with \hisC\ feet set to support his own weight, but not to brace \himC\ against the sideways force.  When \heC\ gets pushed from the side, \hisC\ feet stay in place and \hisC\ torso topples, rotating \himC\ about \hisC\ center of mass\Foreshadow{The physics of why an object (or person) rotates when they fall over is discussed with \protect{\hyperref[c:torque]{torque}}.}{} as \heC\ falls to the ground. \studentD\ points to \hisD\ phone and says, ``I recorded the whole thing!''  If you respond, ``Awesome! Can I watch the part about how \studentZ\index{\studentZ} acts?'', please read \ref{a:NIIIreaction}.  If you respond, ``Awesome! Let's show the psychology and physics faculty our cool video!'', please read \ref{a:NIIIfaculty}. If you respond, ``Yeah, we probably should have intervened before this happened instead of just watching.  Let's go talk to Campus Security.'', please read \ref{a:NIIIsecurity}.
\item\label{a:coastindrive} As soon as you decide to do this, the gas runs out.  Thinking you can make it to the gas station, you take your foot off of the gas pedal.  You slow down fairly quickly and get nervous that you might get rear-ended.  You turn on the hazard-lights.  After about a minute you are travelling $30 \unit{mph}$ and you pass the time by working out \autoref{ex:slowcar}  (pg.~\pageref{ex:slowcar}).  People are honking at you as they try to pass.  \studentB\index{\studentB} turns to you and asks you why you are going so slow.  If you start a discussion about Newton's First Law, then go to \ref{a:NIdrive}.  If you get embarrassed and decide to pull over, then read \ref{a:parkandwalk2}.
\item\label{a:NIIIreaction} As \studentZ\index{\studentZ} pushes, you notice that because \heZ\ was being intentional, \heZ\ put one foot behind \himZ\ to brace \hisZ\ body during the push.  \HeZ\ leans into the push and stays standing.  You are intrigued.  If you decide to do a follow-up experiment by pushing \studentD\index{\studentD} over without bracing yourself, then read \ref{a:NIIIexperiment}.  If you decide to exercise self-restraint, then read \ref{a:NIIIrestraint}.
\item\label{a:coastinneutral} You speed up to $60 \unit{mph}$ before the gas runs out and then you quickly pop the car into neutral.  You slow down gradually and, in an effort to not get rear-ended, you cleverly turn on the hazard-lights.  After about a minute you are travelling $52 \unit{mph}$ and you pass the time by working out \autoref{ex:coasting}  (pg.~\pageref{ex:coasting}).  After $2\unit{min}$, you are travelling $43 \unit{mph}$ and people are getting impatient as they try to pass.  After $2.79\unit{min}$, you triumphantly coast into the gas station at a comfortable speed of $36.7\unit{mph}$.  \studentB\index{\studentB} is so happy, \heB\ buys you a full tank of gas and the two of you start a discussion about Newton's First Law while pumping the gas.  Please read \ref{a:NIresult}.
\item\label{a:NIIIconcern} Being the thoughtful and considerate person you are, you rush over and startle \studentC\index{\studentC} out of \hisC\ reverie.  \studentZ\index{\studentZ} is quite angry and now focuses \hisZ\ attention on you!  \HeZ\ rushes towards you and shoves as hard as he can.  You go \textit{flying} backwards and land on your tailbone while he just stands there laughing.  \studentC\ and \studentD\index{\studentD} both rush over to help you while \studentZ\ wanders off.  Surprisingly, \studentC\ has an icepack, which helps.  If you go speak to your faculty members about this, please read \ref{a:NIIIfaculty}. If you decide to talk to Campus Security, please read \ref{a:NIIIsecurity}.
\item\label{a:NIdrive} After some discussion, you and \studentB\index{\studentB} realize that when the car is in drive, the transmission (the part of the car that converts how-fast-the-engine-spins to how-fast-the-axel-and-wheels-turn) is connected to the axel, which means that the rolling wheels are trying to turn the engine parts as well as the wheels themselves.  The engine parts have grease and oil, but still take a lot of energy to turn.  This causes friction, which dissipates energy and, more importantly, exerts a backwards force on the spinning wheels.  Your car is not being described by Newton's First Law, which requires there to be no force applied.  Instead your car is being described by Newton's Second Law and the force is changing the velocity to cause you to go slower.  It only takes the car $2.2\unit{min}$ to stop and you still have to walk to the gas station.  \studentB\ laments ``If only there were a way to reduce the force on the axel\ldots'' If it occurs to you to speculate about putting the car in neutral when the gas ran out, then imagine reading \ref{a:coastinneutral}.  If you stop talking and walk to the gas station, then read \ref{a:intosunset2}.
\item\label{a:NIIIexperiment} You turn and push \studentD\index{\studentD} over.  Like \studentC\index{\studentC}, \heD\ did not expect it and was not braced, so \heD\ falls over. Similarly, you decided not to brace yourself and in pushing \studentD, you fall over backwards!  \studentD\ did not have to \textit{choose} to push on you.  The act of you deciding to push \himD\ necessarily and simultaneously produces a force on you, equal in magnitude and opposite in direction.  Unfortunately, \studentD\ doesn't think this was a useful exercise and shouts ``I have the whole thing on video!'' and storms off to Campus Security.  You are arrested for assault, miss your physics class for a couple of weeks and ultimately fail all of your classes. I certainly hope this was all happening in your head and not in real life!  You learned something about physics, but at what cost to your humanity?  \hyperref[cyoa:NIII]{The end!}
\item\label{a:nogas} You leave \studentB\index{\studentB} in the car and walk the 45 minutes to the gas station.  You buy a gas can, fill it up, and start to carry it back to the car.  It is very heavy and you notice vultures circling overhead.  You hope you survive this.  It might have been a better idea to bring \studentB\ with you to share the burden.  You stumble once, and then again.  You swear to be more cautious about estimating your gas consumption.  After walking for what seems like hours and stumbling back to the car, you find \studentB\ very excited. \HeB\ declares that \heB\ has invented a time machine so you can go back to the \hyperref[cyoa:NI]{adventure} and start over to learn something about Newton's First Law!
\item\label{a:NIIIrestraint} \studentD\index{\studentD} points to \hisD\ phone and says, ``I recorded the whole thing!''  If you respond, ``Awesome! Can I watch the part about how \studentC\index{\studentC} acts?'', please read \ref{a:NIIIaction}.  If you respond, ``Awesome! Let's show the psychology and physics faculty our cool video!'', please read \ref{a:NIIIfaculty}. If you respond, ``Yeah, we probably should have intervened before this happened instead of just watching.  Let's go talk to Campus Security'', please read \ref{a:NIIIsecurity}.
\item\label{a:parkandwalk2} You pull over and park the car.  \studentB\index{\studentB} suggests one of you stays with the car, probably because \heB\ has physics homework to do.  If you decide to separate, read \ref{a:nogas2}.  If you decide to journey together, read \ref{a:intosunset}.
\item\label{a:NIIIfaculty} The psychology faculty member speaks to you both about how to be good citizens and about the psychological effects of bullies both on the bully and on the recipient.  If you decide to learn more about this, please read \href{https://www.psychologytoday.com/basics/bullying}{Psychology Today}.  The physics faculty member points out that when one person pushes another, the person being pushed does not brace \himselfC, whereas the person doing the pushing does.  Furthermore one might imagine what would happen if you did not brace yourself when you pushed each other, such as in \ref{a:NIIIexperiment}.  You are asked to review both \autoref{ex:braced}  (pg.~\pageref{ex:braced}) and \autoref{ex:unbraced}  (pg.~\pageref{ex:unbraced}) before the next exam.  On your way out the door, you hear a voice suggest ``\ldots and you \textit{might} want to talk to \hyperref[a:NIIIsecurity]{Campus Security} about the incident\ldots''
\item\label{a:intosunset} Everything goes as planned.  You drive off into the sunset sadly ignorant of the physics you might have learned. \hyperref[cyoa:NI]{The end!}
\item\label{a:NIIIsecurity} You speak with Campus Security about the incident and \studentZ\index{\studentZ} gets taken in for assault.  The Dean thanks you for being brave enough to speak up. \hyperref[cyoa:NIII]{The end!}
\item\label{a:NIresult} During the discussion, you and \studentB\index{\studentB} realize that the rolling wheels and the spinning axel are still connected to the not-spinning frame of the car.  While this causes less friction than if the car were in drive, there is still some friction, which dissipates energy and, more importantly, exerts a backwards force on the spinning wheels.  Your car is not being described by Newton's First Law, which requires there to be no force applied.  Instead your car is being described by Newton's Second Law and the force is changing the velocity to cause you to go slower.  You finish getting gas and drive on to many happy adventures. \hyperref[cyoa:NI]{The end!}
\item\label{a:guilty} You feel guilty for letting \studentZ\index{\studentZ} push \studentC\index{\studentC} down despite your amazing score on the next physics test.  It wasn't worth it.  \hyperref[cyoa:NIII]{The end!}
\item\label{a:nogas2} You leave \studentB\index{\studentB} in the car and walk the 31 minutes to the gas station.  You buy a gas can, fill it up, and start to carry it back to the car.  It is very heavy and you notice vultures circling overhead.  You hope you survive this.  It might have been a better idea to bring \studentB\ with you to share the burden.  You stumble once, and then again.  You swear to be more cautious about estimating your gas consumption.  After walking for what seems like hours and stumbling back to the car, you find \studentB\ very excited. \HeB\ declares that \heB\ has invented a time machine so you can go back to the \hyperref[cyoa:NI]{adventure} and start over to learn something about Newton's First Law!
\item\label{a:intosunset2} You and \studentB\index{\studentB} happily walk the 25 minutes to the gas station, discussing and working out physics problems the whole way.  You buy a gas can, fill it up, and share the burden of carrying a heavy gas can.  You return to the car, add gas, and drive on to many happy adventures.  \hyperref[cyoa:NI]{The end!}
\end{Story}

%%%%%%%%%%%%%%%%%%%%%%%%%%%%%%%%%%%%%%%%%%%%%%%%%%%%%%%%%%%%%%%%%%%%%%%%%%%%%%%%%%%%%%%%%%%%%%%%%%%%%%

\chapter{Characters}

This textbook has five characters who follow you throughout the book.  They appear in the examples and some homework problems.  They also remember previous experiences.  I need to adjust the examples in \autoref{c:force} such that the people pushing boxes are helping the reader rearrange furniture.

The index lists\dothis{The index will recognize the people in two different formats.  One is by my name for them, which is $\backslash$studentX (where X is A, B, C, D, \ldots Z).  The other is by the name assigned to that variable.  So these show up in different places in the Index.}{} the pages that the characters appear.  The point of this chapter is to highlight some of the primary adventures of the characters according to their own perspectives.  \textbf{None of the links in this chapter will be given a corresponding return link.}  This chapter is for me to track relationships and will likely go away when the book is ready for publication.
%
I can, at the header of the code, define the name, gender, mass, and dimensions of each individual.\dothis[inline]{\href{http://malveyauthor.com/}{Madeline Alvey}, the author of  \protect{\href{http://escapepod.org/2017/03/09/ep566-honey-and-bone-artemis-rising-3/}{``Honey and Bone'' at EscapePod}} is a physics and English undergraduate student at UK in Lexington.  I might consider hiring(?) her to help storyboard the characters.}

\section{\studentA\index{\studentA!inside}}\index{\studentA!outside}\dothis{The index-call that is \textbf{outside} of the section title registers as $\backslash$studentA, which puts the name alphabetically under $\backslash$studentA, rather than \studentA.  The index-call that is \textbf{inside} of the section title registers as \studentA, which puts the name alphabetically under \studentA, rather than $\backslash$studentA.}
\index{\studentA|(} % Begin page-range
\begin{itemize}
\item In \autoref{s:forcewords}, \studentB{} gives \studentA{} a good-natured shove in the arm in order to get the language clarified and begin the conversation about the on-by notation.
\item In \ref{se:FBD-AB} \studentA{} helps \studentB{}\ldots
    \begin{itemize}
    \item (in the current version) push an object to make it accelerate and feel a reaction force causing \himA{} to accelerate backwards.
    \item (in the future version) will help the reader move into or out of their residence hall by pushing on heavier furniture.
    \item[NOTE:] This is all drawn in \autoref{f:firstFBD}, which is updated in \autoref{f:firstFBDupdate}.
    \end{itemize}
\item In \ref{se:weightA}, \studentA{} falls from a small height.  (maybe he is jumping off a short ledge while taking a short-cut to class?)
\item In \autoref{ex:baking}, \studentA{} decides to bake some bread for a party at \studentB's house, measuring the time it takes to warm his oven.
\end{itemize}

\index{\studentA|)} % end page-range
\section{\studentB\index{\studentB}}
\index{\studentB|(} % Begin page-range

\begin{itemize}
\item \studentB{} is a passenger in the reader's car in \autoref{ex:slowcar} when the reader runs out of gas and coasts to a stop.
\item \studentB{} is a passenger in the reader's car in \autoref{ex:coasting} and speculates about how fast to go before putting the car in neutral to coast to a stop.
\item \studentB{} joins the reader on a road trip in \autoref{cyoa:NI} and runs out of gas.  This results in multiple possible adventures:
\begin{itemize}
    \item \ref{a:parkandwalk}, which leads to either an end at \ref{a:nogas} or an end at \ref{a:intosunset}.
    \item \ref{a:coastindrive}, which leads to either \ref{a:NIdrive} (choose \ref{a:coastinneutral} or end with \ref{a:intosunset2}) or \ref{a:parkandwalk2} (choose \ref{a:intosunset} or end at \ref{a:nogas2})
    \item \ref{a:coastinneutral}, which leads to an end at \ref{a:NIresult}.
\end{itemize}
\item In \autoref{s:forcewords}, \studentB{} gives \studentA{} a good-natured shove in the arm in order to get the language clarified and begin the conversation about the on-by notation.
\item In \autoref{se:FBD-AB}, \studentB{} helps \studentA{}\ldots
    \begin{itemize}
    \item (in the current version) pull an object to make it accelerate and feel a reaction force causing \himB{} to accelerate backwards.
    \item (in the future version) will help the reader move into or out of their residence hall by pushing on heavier furniture.
    \item[NOTE:] This is all drawn in \autoref{f:firstFBD}, which is updated in \autoref{f:firstFBDupdate}.
    \end{itemize}
\item In \ref{se:FNB}, \studentB{} has a normal force supporting \himB.  (This touches \ref{A:floor}, \ref{A:second}, and \ref{A:third}.)
\item At some point, \studentB{} has a party, because in \autoref{ex:baking}, \studentA{} decides to bake some bread for a party at \studentB's house.
\end{itemize}

\index{\studentB|)} % end page-range
\section{\studentC}
\index{\studentC|(} % Begin page-range

\index{\studentC|)} % end page-range
\section{\studentD}
\index{\studentD|(} % Begin page-range

\index{\studentD|)} % end page-range
\section{\studentE}
\index{\studentE|(} % Begin page-range

\index{\studentE|)} % end page-range
\section{\studentF}
\index{\studentF|(} % Begin page-range

\index{\studentF|)} % end page-range
\section{\studentZ}
\index{\studentZ|(} % Begin page-range

\index{\studentZ|)} % end page-range



%%%%%%%%%%%%%%%%%%%%%%%%%%%%%%%%%%%%%%%%%%%%%%%%%%%%%%%%%%%%%%%%%%%%%%%%%%%%%%%%%%%%%%%%%%%%%%%%%%%%%%

\addcontentsline{toc}{chapter}{Index}
%\printindex
\input{ABIP.ind}

\newpage
Note: You can do some more fancy indexing with the formatting found at
\url{https://en.wikibooks.org/wiki/LaTeX/Indexing}

\end{document}

\newpage

%\verb[\marginparwidth]:
\printinunitsof{in}\prntlen{\marginparwidth}

%\verb[\marginparwidth]:
\printinunitsof{mm}\prntlen{\marginparwidth}

%\verb[\marginparwidth]:
\printinunitsof{pt}\prntlen{\marginparwidth}

\pagediagram




\newpage
Note: You can do some more fancy indexing with the formatting found at
\url{https://en.wikibooks.org/wiki/LaTeX/Indexing}

\end{document}

\newpage

%\verb[\marginparwidth]:
\printinunitsof{in}\prntlen{\marginparwidth}

%\verb[\marginparwidth]:
\printinunitsof{mm}\prntlen{\marginparwidth}

%\verb[\marginparwidth]:
\printinunitsof{pt}\prntlen{\marginparwidth}

\pagediagram




\newpage
Note: You can do some more fancy indexing with the formatting found at
\url{https://en.wikibooks.org/wiki/LaTeX/Indexing}

\end{document}

\newpage

%\verb[\marginparwidth]:
\printinunitsof{in}\prntlen{\marginparwidth}

%\verb[\marginparwidth]:
\printinunitsof{mm}\prntlen{\marginparwidth}

%\verb[\marginparwidth]:
\printinunitsof{pt}\prntlen{\marginparwidth}

\pagediagram




<!-- -->\newpage
Note: You can do some more fancy indexing with the formatting found at
\url{https://en.wikibooks.org/wiki/LaTeX/Indexing}

\end{document}

<!-- -->\newpage

%\verb[\marginparwidth]:
\printinunitsof{in}\prntlen{\marginparwidth}

%\verb[\marginparwidth]:
\printinunitsof{mm}\prntlen{\marginparwidth}

%\verb[\marginparwidth]:
\printinunitsof{pt}\prntlen{\marginparwidth}

\pagediagram


