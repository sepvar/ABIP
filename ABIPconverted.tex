\documentclass[11pt,letter,openany,makeidx]{book}
\usepackage{amsmath}
\usepackage{macros}
\usepackage{comment}
\usepackage{graphicx}
\usepackage{microtype}
\usepackage{gfsdidot}
\usepackage[T1]{fontenc}
\usepackage{booktabs}
\usepackage{underscore}
\usepackage{caption}
\usepackage[within=chapter,chapterlistsgap=6pt]{newfloat}
\usepackage{tocloft}
\usepackage{xpicture}
\usepackage{xcolor}
%\usepackage[dvips]{xcolor}
%\GetGinDriver  % for xcolor to work well with hyperref
%\usepackage[\GinDriver]{hyperref}
\usepackage{ulem}%for \sout (done)
\usepackage[colorinlistoftodos]{todonotes}
%\usepackage[disable,colorinlistoftodos]{todonotes}
%\usepackage{layouts}
%\usepackage{showframe}
\usepackage{coordsys}
\usepackage[pdftex]{hyperref}

%\usepackage{cellpage}

\hypersetup{colorlinks=true,bookmarks=true,pdftitle=Algebra-Based Introductory Physics,pdfauthor=J.Christensen,pdfdisplaydoctitle}
% If using a bibliography, then include "backref" in list of \hypersetup items
% linkcolor=color of internal links (red); anchorcolor = color of anchor text (black); citecolor = bibliographic citations (green); filecolor = color for local URL files (cyan); menucolor = Acrobat menu (red); urlcolor = external links (magenta); hidelinks = remove all color
% citebordercolor = color of box for citations (0 1 0); fileborder = links to files box (0 .5 .5); linkbordercolor = normal links (1 0 0); menuborder; urlborder; allbordercolors; pdfborder

\includecomment{ForMe}
\includecomment{ForReviewer}
\includecomment{ForPublic}

\makeindex

\newlistof{example}{loe}{List of Examples}
\DeclareFloatingEnvironment[fileext=loe,listname="List of Examples",name=Example]{example}
\setlength{\cftexamplenumwidth}{1cm}
\newcounter{sample}
\newcounter{carrysample}
\renewcommand{\thesample}{Simple Example \arabic{sample}}
\renewcommand{\thecarrysample}{Simple Example \arabic{carrysample}}
\newenvironment{sample}{\color{rgb:red,0;green,2;blue,1}\begin{list}{\textbf{\thesample}:}{\usecounter{sample} \setcounter{sample}{\value{carrysample}} \leftmargin 12pt}}{\end{list}\setcounter{carrysample}{\value{sample}}}
\newcommand{\THREE}[6]{\vspace{-3pt}\begin{flushright} Select one:  \mbox{#1 (\ref{#4})},  \mbox{#2 (\ref{#5})}, or \mbox{#3 (\ref{#6})}.\end{flushright}}
\newcommand{\TWO}[4]{\begin{flushright} Select one:  \mbox{#1 (\ref{#3})} or \mbox{#2 (\ref{#4})}.\end{flushright}}
\newcommand{\YN}[2]{\TWO{Yes}{No}{#1}{#2}}
\newcommand{\TF}[2]{\TWO{True}{False}{#1}{#2}}
\newcommand{\return}[1]{{} \hfill \mbox{Return to \ref{#1}.}}
\newcommand{\autoreturn}[1]{{} \hfill \mbox{Return to \autoref{#1}.}}
\newcommand{\linkreturn}[2][a related idea]{{}\hfill \mbox{Return to the discussion of \protect{\hyperlink{#2}{#1}}.}}
\newcommand{\mmr}[1]{\mbox{[\protect{#1}]}}
\newcommand{\multireturn}[1]{{}\hfill Return to one of the following locations: \newline #1.}
\newcounter{AtIQ}
\renewcommand{\theAtIQ}{Answer \arabic{AtIQ}}
\newenvironment{AIQ}{\begin{list}{\textbf{Interactive \theAtIQ}:}{\usecounter{AtIQ} \leftmargin 12pt}}{\end{list}}

% Related (return), but not part of...
\newcommand{\mreturn}[1]{\note{Return to \protect{\ref{#1}}.}}
\newcommand{\mlinkreturn}[2][a related idea]{\note{Return to the discussion of \protect{\hyperlink{#2}{#1}}.}}
\newcommand{\mautoreturn}[1]{\note{Return to \protect{\autoref{#1}}.}}
\newcommand{\mmultireturn}[1]{\note{Return to one of the following locations: \newline #1.}}


\newlistof{adventure}{loa}{List of Adventures}
\DeclareFloatingEnvironment[fileext=loa,listname="List of Adventures",name=Adventure]{adventure}
\setlength{\cftadventurenumwidth}{1cm}
\newcounter{CYOA}
\renewcommand{\theCYOA}{Plan \Alph{CYOA}}
\newenvironment{CYOA}{\begin{list}{\textbf{\theCYOA}:}{\usecounter{CYOA}}}{\end{list}}
\newcounter{storyline}
\renewcommand{\thestoryline}{Storyline \arabic{storyline}}
\newenvironment{Story}{\begin{list}{\textbf{\thestoryline}:}{\usecounter{storyline} \leftmargin 12pt}}{\end{list}}

\newlistof{reallife}{irl}{List of Real Life Patterns}
%\DeclareFloatingEnvironment[fileext=irl,listname="List of Real Life Patterns",chapterlistsgaps=off,name=Real Life Patterns]{reallife}
\DeclareFloatingEnvironment[fileext=irl,listname="List of Real Life Patterns",name=Real Life Patterns]{reallife}
\setlength{\cftreallifenumwidth}{1cm}
\newcounter{IRL}
%\renewcommand{\theIRL}{\arabic{IRL}}
\newenvironment{realtable}{%\renewcommand{\arraystretch}{2}
                           %\hspace{-.2in}
                            \begin{tabular}{@{}lll@{}} \toprule Do This & Notice This & Ask This  \\ }
                            {\bottomrule \end{tabular} }%\renewcommand{\arraystretch}{.5}}
\newcommand{\dna}[3]{\midrule \begin{minipage}{4cm}\raggedright #1 \end{minipage}
                   & \begin{minipage}{4cm}\raggedright #2 \end{minipage}
                   & \begin{minipage}{4cm}\raggedright #3 \end{minipage} \\ }
\newcommand{\multidna}[1]{\multicolumn{3}{|c|}{\begin{minipage}{13cm}\center #1 \end{minipage}} \\ \midrule }


\newlistof{story}{los}{The Stories of the Equations}
\DeclareFloatingEnvironment[fileext=los,listname="The Stories of the Equations",name=This Equation's Story]{story}
\setlength{\cftstorynumwidth}{1cm}
\newcommand{\thestoryof}[1]{\marginpar{\raggedright \footnotesize The story of \\ \fcolorbox{black}{yellow}{\begin{minipage}[c]{1.5in} \center $\deq #1$ \end{minipage}}}}
\newcommand{\EqStory}[2]{\left[ {\color{rgb:red,1;green,1;blue,4} \begin{minipage}{#1}\raggedright\begin{center} #2 \end{center}\end{minipage}} \right]}
\newcommand{\EqStoryOver}[3]{\overbrace{\EqStory{#1}{#2}}^{\displaystyle #3}}
\newcommand{\EqStoryUnder}[3]{\underbrace{\EqStory{#1}{#2}}_{\displaystyle #3}}
\newcommand{\EqStoryFrac}[5]{\frac{\overbrace{\EqStory{#1}{#2}}^{\displaystyle #3}}
                                 {\underbrace{\EqStory{#1}{#4}}_{\displaystyle #5}}}


%%%%%%%%%%%%%%%%%%%%%%%%%%%%%%%%%%%%%%%%%%%%%%%%%%%%%%%%%%%%
%
%\presetkeys{todonotes}{fancyline,color=blue!15}{}
\presetkeys{todonotes}{color=blue!15,linecolor=blue!75,size=\footnotesize}{}
%
\newcounter{todocounter}
\newcommand{\dothis}[2][]
{\stepcounter{todocounter}\todo[color=green!30, #1]{\thetodocounter: #2}}
\newcommand{\docaption}[3][]
{\stepcounter{todocounter}\todo[color=green!30, prepend, caption={\thetodocounter: \underline{#2}}, #1]{#3}}
\newcommand{\addlink}[2][]
{\stepcounter{todocounter}\todo[prepend, caption={\thetodocounter: \underline{Add Link}}, #1]{#2}}
\newcounter{todourgentcounter}
\newcommand{\urgent}[2][]
{\stepcounter{todourgentcounter}\todo[color=orange!50, #1]{\thetodourgentcounter: #2}}
\newcommand{\urgcap}[3][]
{\stepcounter{todourgentcounter}\todo[color=orange!50, prepend, caption={\thetodourgentcounter: \underline{#2}}, #1]{#3}}
\newcommand{\done}[2][]
{\todo[color=yellow!10, #1]{\sout{#2}}}
%
%\newcommand{\new}[2]{}%
\newcommand{\new}[2]{\marginpar{\raggedright \footnotesize New to #1 \\ \fcolorbox{blue}{yellow!10}{\begin{minipage}[c]{1.5in} \center {\color{blue} #2 } \end{minipage}}}}%
%%%%%%%%%%%%%%%%%%%%%%%%%%%%%%%%%%%%%%%%%%%%%%%%%%%%%%%%%%%%


%%%%%%%%%%%%%%%%%%%%%%%%%%%%%%%%%%%%%%%%%%%%%%%%%%%%%%%%%%%%
%
%\newcommand{\deq}{\displaystyle}
%\newcommand{\txtfrac}[2]{{}^{#1}\!/_{\!#2}}
%
%%%%%%%%%%%%%%%%%%%%%%%%%%%%%%%%%%%%%%%%%%%%%%%%%%%%%%%%%%%%



%%%%%%%%%%%%%%%%%%%%%%%%%%%%%%%%%%%%%%%%%%%%%%%%%%%%%%%%%%%%
%
% PEOPLE AND PRONOUNS
%
% According to https://www.cdc.gov/nchs/fastats/body-measurements.htm
% Measured average height, weight, and waist circumference for adults ages 20 years and over
% Men:
% Height (inches): 69.3                 = 1.760 m
% Weight (pounds): 195.5                = 88.86 kg
% Waist circumference (inches): 39.7    = 1.01 m
% Women:
% Height (inches): 63.8                 = 1.621 m
% Weight (pounds): 166.2                = 75.55 kg
% Waist circumference (inches): 37.5    = 0.9525 m
% Source: Anthropometric Reference Data for Children and Adults: United States, 2007-2010, tables 4, 6, 10, 12, 19, 20[PDF - 1.7 MB]
%  https://www.cdc.gov/nchs/data/series/sr_11/sr11_252.pdf
%
\newcommand{\studentA}{Abdul}       \newcommand{\massA}{\mbox{$85.0\unit{kg}$}}
\newcommand{\studentB}{Beth}        \newcommand{\massB}{\mbox{$75.0\unit{kg}$}}
\newcommand{\studentC}{Carl}        \newcommand{\massC}{\mbox{$90.0\unit{kg}$}}
\newcommand{\studentD}{Diane}       \newcommand{\massD}{\mbox{$80.0\unit{kg}$}}
\newcommand{\studentE}{Erik}        \newcommand{\massE}{\mbox{$95.0\unit{kg}$}}
\newcommand{\studentF}{Frances}       \newcommand{\massF}{\mbox{$85.0\unit{kg}$}}
\newcommand{\studentX}{Xerxes}       \newcommand{\massX}{\mbox{$62.5\unit{kg}$}}
\newcommand{\studentZ}{Zambert}     \newcommand{\massZ}{\mbox{$95.0\unit{kg}$}}
% Male
\newcommand{\heA}{he}\newcommand{\himA}{him}\newcommand{\hisA}{his}\newcommand{\himselfA}{himself}
\newcommand{\HeA}{He}\newcommand{\HimA}{Him}\newcommand{\HisA}{His}
\newcommand{\heC}{he}\newcommand{\himC}{him}\newcommand{\hisC}{his}\newcommand{\himselfC}{himself}
\newcommand{\HeC}{He}\newcommand{\HimC}{Him}\newcommand{\HisC}{His}
\newcommand{\heE}{he}\newcommand{\himE}{him}\newcommand{\hisE}{his}\newcommand{\himselfE}{himself}
\newcommand{\HeE}{He}\newcommand{\HimE}{Him}\newcommand{\HisE}{His}
\newcommand{\heZ}{he}\newcommand{\himZ}{him}\newcommand{\hisZ}{his}\newcommand{\himselfZ}{himself}
\newcommand{\HeZ}{He}\newcommand{\HimZ}{Him}\newcommand{\HisZ}{His}
% Female
\newcommand{\heB}{she}\newcommand{\himB}{her}\newcommand{\hisB}{her}\newcommand{\himselfB}{herself}
\newcommand{\HeB}{She}\newcommand{\HimB}{Her}\newcommand{\HisB}{Her}
\newcommand{\heD}{she}\newcommand{\himD}{her}\newcommand{\hisD}{her}\newcommand{\himselfD}{herself}
\newcommand{\HeD}{She}\newcommand{\HimD}{Her}\newcommand{\HisD}{Her}
\newcommand{\heF}{she}\newcommand{\himF}{her}\newcommand{\hisF}{her}\newcommand{\himselfF}{herself}
\newcommand{\HeF}{She}\newcommand{\HimF}{Her}\newcommand{\HisF}{Her}
%
\newcommand{\heX}{\studentX}\newcommand{\himX}{\studentX}\newcommand{\hisX}{\studentX's}\newcommand{\himselfX}{the person of \studentX}
\newcommand{\HeX}{\studentX}\newcommand{\HimX}{\studentX}\newcommand{\HisX}{\studentX's}
%%%%%%%%%%%%%%%%%%%%%%%%%%%%%%%%%%%%%%%%%%%%%%%%%%%%%%%%%%%%%


%%%%%%%%%%%%%%%%%%%%%%%%%%%%%%%%%%%%%%%%%%%%%%%%%%%%%%%%%%%%
%
% Book macros
%
\newcommand{\aside}[2]{\marginpar{\raggedright \footnotesize\textbf{#1}: #2}}
\newcommand{\important}[1]{\\ \fcolorbox{black}{yellow}{\begin{minipage}[c]{4.925in} \center #1 \end{minipage}}\\}
\newcommand{\inlife}{\marginpar[\scriptsize \raggedright How you might observe $\Rightarrow$ this in your life.]
                               {\scriptsize \raggedleft $\Leftarrow$ How you might observe this in your life.}}
\newcommand{\touchstone}{\marginpar[\scriptsize \raggedright Where have I seen this $\Rightarrow$ before?]
                                   {\scriptsize \raggedleft $\Leftarrow$ Where have I seen this before?}}
\newcommand{\foreshadow}{\marginpar[\scriptsize \raggedright When will I ever use this? $\Rightarrow$]
                                   {\scriptsize \raggedleft $\Leftarrow$ When will I ever use this?}}
\newcommand{\foreshadowR}{\reversemarginpar
                          \marginpar[\scriptsize \raggedright When will I ever use this? $\Rightarrow$]
                                    {\scriptsize \raggedleft $\Leftarrow$ When will I ever use this?}}
\newcommand{\Touchstone}[1]{\marginpar[\scriptsize \raggedright Where have I seen this $\Rightarrow$ \\ before? #1]
                                      {\scriptsize \raggedleft $\Leftarrow$ Where have I seen this before? #1}}
\newcommand{\Foreshadow}[1]{\marginpar[\scriptsize \raggedright When will I ever use this? $\Rightarrow$ \\ #1]
                                      {\scriptsize \raggedleft $\Leftarrow$ When will I ever use this? #1}}
%
%%%%%%%%%%%%%%%%%%%%%%%%%%%%%%%%%%%%%%%%%%%%%%%%%%%%%%%%%%%%


\begin{document}

%\title{Algebra-Based Introductory Physics}
%\author{J Christensen}
%\date{Jan 2017}
%\maketitle
%\pagestyle{cellpage}

\begin{titlepage}
	\centering
%	\includegraphics[width=0.15\textwidth]{example-image-1x1}\par\vspace{1cm}
	{\Huge\bfseries Physics Connected\par}
	\vspace{1cm}
	{\Large\bfseries An Algebra-Based Introductory Physics Textbook\par}
	\vspace{1cm}
	{\large Learn like you think: an interconnected view of physics\par}
	\vspace{2cm}
	{\Large\itshape by: J Christensen\par}
	\vfill
\begin{ForReviewer}
	Version 2.3\par
	{\footnotesize
    \begin{itemize}
    \item Ideas yet to implement:
        \begin{itemize}
        \item The examples are phrased as descriptions, not examples like the homework problems.  Need to consider rephrasing these, not calling them examples, or adding actual examples that better show how to respond to the way homework problems are written.
        \item Define a different page dimension that fits on a cell phone display.  (Enhance possible cell-phone reading.)
        \end{itemize}
    \item version 2.3: June 16-28, 2017
        \begin{itemize}
        \item Updated Section 81. $F=mg$ and Section 8.2 Normal Force
        \item Added specific list of Flame Challenges
        \item Rearranged some of the subsections in the ``Seeing Physics'', added references
        \item Equations of motion for constant acceleration (Need the Story Of)
        \item Added a section to Chap 5 (1-D motion) that gives examples of solutions that require multiple steps  (one equation is insufficient)
        \item Developed the weight and mass discussion and examples
        \item Ladder leaning example in torque, plus some homework problems
        \item Added some Conceptual Homework to weight/mass
        \item Added placeholders to the Gravity chapter
        \item Removed indicators of v1.7 changes
        \end{itemize}
    \item version 2.2: June 16, 2017
        \begin{itemize}
        \item Created conversation about $F=mg$ for Chapter on types of forces.  Caused modifications in lots of places
            \begin{itemize}
            \item Added freefall to the motion chapter
            \item Created IRL and Example dropping objects to see acceleration in $F=mg$, then moved to freefall section -- new Answers to interactive questions
            \item Commented on air resistance
            \item Comments about precision in language (need to do more with precision in mathematics)
            \item Started a couple of ideas about effective theories.  (need to decide where it goes)
            \item Added detail about SI, and specifically the pound-force, pound-mass, and kilogram. to sections \ref{s:SI-MKS} and \ref{ss:weightmass}
            \item Added NIST and BIPM references (found in Wikipedia and then searched further)
            \item conversation about weight and mass.  (required reference to the chapter on Fluids and density)
            \item Moved Google search about significant figures
            \end{itemize}
        \item Added comments about fundamental forces to the section on types of force
        \item Removed indicators of v1.5 and v1.6 changes
        \end{itemize}
    \item version 2.1: June 10, 2017
        \begin{itemize}
        \item Re-commented the $\backslash$new command
        \item Started the chapters on Seeing Physics [\autoref{c:physics}] and Deeper Dive [\autoref{c:revisted}] (These should be renamed)
        \item Moved some sections on fundamental interactions
        \end{itemize}
    \item version 2.0: April 10, 2017
        \begin{itemize}
        \item Re-enabled v1.8 hides
        \item Added a link to \textit{Spacepod}, \textit{Physics Footnotes}, and \textit{Sixty Symbols}
        \item Fixed a $\backslash$dothis that was inside an $\backslash$important, causing a compile error.
        \item Removed indicators of v1.4 changes
        \end{itemize}
    \item version 1.8: April 1, 2017
        \begin{itemize}
        \item Prepare for "the public": "Disabled" the To-Do items, "Hid" the $\backslash$new revision notes, Hid the List of Tables (have none yet)
        \end{itemize}
    \end{itemize}
    }
\end{ForReviewer}
\begin{ForPublic}
{\flushleft
\textbf{Note to the reviewers:}\new{v1.8}{Added the note}
My goal with this book is to create an electronically viewable book that makes use of the advantages of being electronic.  While current e-books have the advantage of being viewable on various devices with having to carry a physical book around, most e-textbooks do not take advantage of hyperlinked text.  With this book I hope to integrate links both forward and backward.  The forward links will be used to motivate curious students.  The backward links will be used to support students who lose track of previous topics.  The integration of these will also provide a convenient opportunity for students to browse through topics they are interested in.
\newpar

At this time, I am providing a single chapter to gauge the viability.  The chapter I am providing is on Newton's Laws.  However, as you read this document, you will find many, many more partially written chapters.  All of the partial chapters and sections are intended to be place-holders for the forward- and backward-links that \autoref{c:force} depends upon.
\newpar

I created this as a PDF that, I believe, can be easily viewed on a computer or tablet.  Since some of my students also seem to read on their phone, I verified that I am also able to view the text in a reasonable manner on my Samsung phone in landscape mode.  In each case, the links should be active and easily manageable.

}
\end{ForPublic}
	\vfill

% Bottom of the page
	{\large \today\par}
\end{titlepage}

\tableofcontents
\newpage
\begin{ForReviewer}
\listoftables
\vfill
\end{ForReviewer}
%\newpage
\listoffigures
\vfill
%\newpage
%\listofstorys
%\newpage
\listofexamples
\vfill
%\newpage
\listofadventures
\vfill
%\newpage
\listofreallifes
\vfill
\newpage

\listoftodos

\newpage



\part{Prerequisites}

<chapter><title>The Story of Science</title>

</chapter><chapter><title></title>Seeing Physics}\label{c-physics}

</chapter><chapter><title></title>Why so much math?}

</chapter><chapter><title></title>Estimating and Uncertainty}

\part{Introducing Motion, Force, and Energy}

</chapter><chapter><title></title>One-Dimensional Motion}\label{c-motion}

</chapter><chapter><title></title>Two-Dimensional Motion}

</chapter><chapter><title></title>Force}\label{c-force}

</section><section><title></title>How Physicists Use the Words (Notation)}\label{s-forcewords}

</section><section><title></title>Connecting the Concepts: Newton's Laws}\label{s-Newton}<aside><title>Referenced by</title> <p>Discussion of <xref provisional=""></xref></p></aside>[how to describe forces]{d-Newtonahead}

    </subsection><subsection><title></title>Translating Newton's First Law: The Law of Inertia}\label{ss-NI}<aside><title>Referenced by</title> <p>Discussion of <xref provisional=""></xref></p></aside>[how to describe forces]{d-Newtonahead}

    </subsection><subsection><title></title>Translating Newton's Second Law: The Equation Law}\label{ss-NII}<aside><title>Referenced by</title> <p></p></aside>{\mmr{<xref ref="" />sss-vectorequations}}, \mmr{<xref ref=""></xref>{d-Newtonahead}{how to describe forces}}, \mmr{<xref ref=""></xref>{d-atrestinmotion}{Newton's first law}}, \mmr{<xref ref="" />d-Fgball}}}

    </subsection><subsection><title></title>Translating Newton's Third Law: Action \& Reaction}\label{ss-NIII}<aside><title>Referenced by</title> <p></p></aside>{\mmr{<xref ref=""></xref>{d-Newtonahead}{how to describe forces}}, \mmr{<xref ref="" />ex-braced}}, \mmr{<xref ref="" />ex-unbraced}}}

</section><section><title></title>Examples} \label{s-NewtonExamples}<aside><title>Referenced by</title> <p><xref provisional="" /></p></aside>{sss-NIItogether}


% YOU ARE HERE


<section><title></title>Examples} \label{s-NewtonExamples}<aside><title>Referenced by</title> <p><xref provisional="" /></p></aside>{sss-NIItogether}

Next, we can consider a simple interactive example that is intended to help you think about how you know a force is acting.
%
\begin{sample}
</p></li><li><p>\label{IQ-holdbook} You hold a book a little above your desk.  When you let go, it falls and then hits your desk.
    <ol>
    </p></li><li><p> While you are holding it, it has no acceleration.  Are there forces acting on it?  \YN{A-hbf}{A-hbnof}
    </p></li><li><p> While you are holding it, is it in equilibrium?  \YN{A-true1}{A-false1}
    </p></li><li><p> After you let go and while the book falls, it accelerates downwards.  Are there forces acting on it?  \YN{A-falls}{A-falls}
    </p></li><li><p> While it is hitting the desk, is it accelerating?  \YN{A-hitY}{A-hitN}
    </p></li><li><p> After it has landed and is sitting on the desk, is it in equilibrium? \YN{A-landedY}{A-landedN}
    </p></li><li><p> After it has landed and is sitting on the desk, how many forces are acting on it? \THREE{Zero}{One}{Two}{A-zero}{A-one}{A-two}
    </ol>
\end{sample}
%
Next, we can <p xml:id=""></p>d-irlNI}{consider} pushing an object across the floor in <xref ref="" />irl-NI} (pg.~\pageref{irl-NI}) to get a different sense of observations we can make that help us recognize patterns that are due to forces we might not have thought to look for.
%
\begin{reallife}[hp]
\hspace{-.2in}
\fcolorbox{black}{green!10}{\begin{minipage}{5.29in} \center
\caption{\label{irl-NI} Pushing an Object Across the Floor}
\begin{realtable}
\dna{Push a chair across a carpet floor}
    {When you stop pushing, it stops moving.}
    {Does force cause motion? <xref ref="" text="type-global" />A-chair1}}
\dna{Push a chair across a tile floor}
    {When you stop pushing, it probably stops moving.}
    {Does force cause motion? <xref ref="" text="type-global" />A-chair2}}
\dna{Push a chair <em></em>with wheels} across a tile floor, with some strength, then let it go.}
    {What happens when you stop pushing? <xref ref="" text="type-global" />A-chair3}}
    {If force causes motion, why does the chair move after you stop touching it?  <xref ref="" text="type-global" />A-chair4}}
\dna{Push a chair <em></em>with wheels} across a tile floor, change your behavior after you let it go.}
    {Do your actions when you are not touching the chair have <em></em>any} impact on the chair? <xref ref="" text="type-global" />A-chair5}}
    {Is it possible that there is a <q>residual effect</q> that you have on the chair after letting it go? <xref ref="" text="type-global" />A-chair6}}
\multidna{Newton's First Law says that if you give the chair a velocity, it should keep that velocity.}
\dna{Repeat the first three suggestions}
    {Correlate the interaction-with-the-ground to the motion-after-you-push-and-release}
    {Is there a force that the chair feels after you release it?  <xref ref="" text="type-global" />A-chair7}}
\multidna{Newton's Second Law says that a net force will change the velocity.}
\dna{Push a chair gently across the floor}
    {A constant force (balanced by the force of friction) will move at a constant speed}
    {What if there were no friction? <xref ref="" text="type-global" />A-chair8}}
\dna{Push a chair forcefully across the floor}
    {A constant force (stronger than the force of friction) will accelerate the object away from your push}
    {Can you list surfaces that are essentially frictionless?}
\end{realtable}
\begin{minipage}{4.925in}
Notice in each case that you are not the only thing interacting with the chair.  The floor is also interacting with the chair.  The floor exerts a <xref ref="" text="title"></xref>[s-Ff]{force of friction} on the chair.  So, when you interpret how your force causes the chair to move, you <em></em>must} also account for the interaction with the floor in your expectations.  We can minimize the effect of friction, by modifying the floor surface.  If you have ever driven on ice and felt out of control, you might have begun to develop your Newtonian intuition.
\flushright\vspace{-12pt}
<assemblage>Return to: </assemblage>{\mmr{<xref ref="" />sss-NIItogether}}, \mmr{<xref ref=""></xref>{d-irlNI}{\autoref*{s-NewtonExamples} reference to \autoref*{irl-NI}}}}
\end{minipage}
\end{minipage}}
\end{reallife}
%
Building on that, it is useful to also consider how human beings behave when they are pushing or getting pushed.  Because people have <em></em>intention} in their actions, we subconsciously balance ourselves and we don't always recognize that we are <p xml:id=""></p>d-cyoaNIII}{doing it}.  <xref ref="" />cyoa-NIII}  (pg.~\pageref{cyoa-NIII}) provides an interactive storyline that starts to show some of the patterns that can lead to a recognition of how we balance ourselves.
%
\begin{adventure}[bpht]
\fcolorbox{black}{blue!10}{\begin{minipage}{4.925in}
\caption{\label{cyoa-NIII} The Town Bully}
\studentZ<idx><h sortby="\studentZ">\studentZ</h></idx> is the town bully.  One day, he spies a biology student, \studentC<idx><h sortby="\studentC">\studentC</h></idx>, minding \hisC\ own business studying an interesting ecological phenomenon.  At the same time, you are standing across the street chatting with your friend \studentD<idx><h sortby="\studentD">\studentD</h></idx>, who happens to be taking a psychology class.  \studentD\ has been quite fascinated lately with watching the way others interact and points out the way \studentZ\ is menacingly approaching the unsuspecting \studentC.  You both predict that \studentZ\ is going to push \studentC\ over.  \studentD\ is mesmerized by the psychological effects and you, having just learned about Newton's laws, are excited to see if this action does indeed produce a reaction.
\begin{CYOA}
</p></li><li><p>\label{c-one} If you watch the way \studentC\ is standing before, during, and after \studentZ\ pushes \himC, then read <xref ref="" text="type-global" />a-NIIIaction}.
</p></li><li><p>\label{c-two} If you watch the way \studentZ\ is standing before, during, and after \heZ\ pushes \studentC, then read <xref ref="" text="type-global" />a-NIIIreaction}.
</p></li><li><p>\label{c-three} If, on the other hand, you shout a warning to \studentC\ and a criticism to \studentZ, trying to keep the incident from becoming violent, then read <xref ref="" text="type-global" />a-NIIIconcern}.
\end{CYOA}
\flushright
<assemblage>Return to: </assemblage>{\mmr{<xref ref=""></xref>{d-NIIIbracing}{the discussion of action-reaction forces}}, \mmr{<xref ref=""></xref>{d-cyoaNIII}{\autoref*{s-NewtonExamples} reference to \autoref*{cyoa-NIII}}}, \mmr{<xref ref="" />ex-braced}}, \mmr{<xref ref="" />ex-unbraced}}}
\end{minipage}}
\end{adventure}
%

\begin{example}[p]
\fcolorbox{black}{yellow!10}{\begin{minipage}{4.925in}\setlength{\parskip}{3pt}
\caption{\label{ex-braced} \studentZ<idx><h sortby="\studentZ">\studentZ</h></idx> intentionally braces when pushing \studentC<idx><h sortby="\studentC">\studentC</h></idx>.}
(To better understand <xref ref="" text="title"></xref>[ss-NIII]{Newton's third law}, you should compare this example to <xref ref="" />ex-unbraced}  [pg.~\pageref{ex-unbraced}].)
\begin{quote}
\studentZ<idx><h sortby="\studentZ">\studentZ</h></idx>, the <xref ref="" text="title"></xref>[cyoa-NIII]{town bully} (with <m>m_Z=\massZ</m>), decides to vent \hisZ\ frustration on \studentC<idx><h sortby="\studentC">\studentC</h></idx>\ for all the times that \studentC\ makes \studentZ\ look bad in class.  While \studentC\ (<m>m_C=\massC</m>) has \hisC\ back turned, \studentZ\ walks up, leans in, and shoves \studentC\ with a force of <m>\vec F_{C,Z} = 215\unit{N}\ihat</m>.  How does \underline{\studentZ}{} accelerate during this exchange?
\end{quote}
%
%\begin{quote}
%Aside: Newton's second law tells us how this affects \studentC. See <xref ref="" text="type-global" />se-netF-a} and homework problem <xref ref="" text="type-global" />hmwk-pushbrace}.
%\end{quote}
%\noindent
<em></em>What do we know?}  As usual, it is convenient to start with a picture to help decide on the appropriate coordinate system.
We can also list
\\[2pt]
\begin{minipage}{2.6in}
the information that we know.
We know <m>m_Z</m>, which is useful for relating <m>F_{Z,\mathrm{net}}</m> to <m>a_Z</m>.
We know <m>m_C</m>, which is useful for relating <m>F_{C,\mathrm{net}}</m> to <m>a_C</m>.  (This is not asked for, but is asked in homework problem <xref ref="" text="type-global" />hmwk-pushbrace}.)
We know <m>F_{C,Z}</m>, how hard \studentZ\ pushes on \studentC.
\end{minipage}
\hfill
\begin{minipage}{150pt}
\begin{picture}(150,90)(-30,-25)
% Dimensions and offset: (width,height)(x offset,y offset)
% Insert picture commands (\line,\circle, etc...) here:
\drawbox{-10}{-20}{120}{20}  % Earth
\drawbox{25}{1}{20}{50} %\studentZ
\drawbox{45}{35}{18}{5} %\studentZ's arms
 %\studentZ's legs
    \put(24,19){\line(0,-1){6}}
    \put(24,19){\line(-1,-2){9}}
    \put(15,1){\line(1,0){4}}
    \put(19,1){\line(1,2){6}}
\drawbox{65}{1}{20}{45} %\studentC
\put(25,53){\scriptsize \studentZ}
\put(65,48){\scriptsize \studentC}
\put(40,-15){\scriptsize Earth}
\put(-40,24){\begin{minipage}{58pt}
\color{blue} \scriptsize \studentZ\ braces \himselfZ. \hfill <m>\searrow</m>
\end{minipage}}
\end{picture}
\end{minipage}
%\hfill {}
We also know that \studentC\ is not bracing \himselfC\ (because \heC\ <q>has \hisC\ back turned</q>) so he only feels one force, and that \studentZ\ is bracing \himselfZ\ (because the problem states that \heZ\ <q>leans in</q>) so he exerts multiple forces.

<em></em>What do we want to know?}  We want to know about the forces acting on \studentZ, in order to find  <m>F_{Z,\textrm{net}}</m> and therefore <m>a_Z</m>.

<em></em>How are these related?}  First, since \studentZ\ exerts a force on \studentC, Newton's third law tells us that \studentZ\ feels a force of \mbox{<m>F_{Z,C}=-215 \unit{N}\ihat</m>}.
Second, because \studentZ\ <em></em>knew} \heZ\ was going to feel this reaction force, \heZ\ compensates by bracing \himselfZ.  This means \heZ\ chooses to exert a force of <m>215\unit{N}</m> on the Earth in the <m>-\ihat</m> direction, probably by putting one leg behind \himselfZ\ and pushing the ground backwards with \hisZ\ foot.  Newton's third law then tells us that \studentZ\ feels a force of \mbox{<m>F_{Z,C}=+215 \unit{N}\ihat</m>} from the ground.

{}\hfill {\footnotesize \autoref*{ex-braced} continued on next page<ellipsis />}
\end{minipage}}
\end{example}
\begin{example}[p]
\fcolorbox{black}{yellow!10}{\begin{minipage}{4.925in}\setlength{\parskip}{3pt}
{\footnotesize \autoref*{ex-braced} continued from previous page<ellipsis />}

<em></em>Free-Body Diagrams:}  We are told of the force on \studentC.  We are told that \studentZ\ braces \himselfZ, which implies the force on the Earth. Newton's third law then helps us recognize the forces on \studentZ.  (Recall the <xref ref=""></xref>{d-interaction}{<q>on-by</q> notation}.)

\noindent % \textwidth default is 5in for a book
\fbox{\begin{minipage}{2.25in}
\begin{FBD}{10}{25}{15}{10}{\studentZ}
\onele{50}{<m>F_{Z,C}=215\unit N</m>}{black}
\oneri{50}{<m>F_{Z,E}=215\unit N</m>}{blue}
\end{FBD}
\vspace{-10pt}
\raggedright
\studentZ\ is pushed by \studentC\ to the left.  \studentZ\ is pushed to the right by the Earth.
\end{minipage}}
\hfill
\fbox{\begin{minipage}{2.25in}
\begin{FBD}{10}{23}{15}{10}{\studentC}
\oneri{50}{<m>F_{C,Z}=215\unit N</m>}{black}
\end{FBD}
\vspace{-10pt}
\raggedright
\studentC\ feels \studentZ\ push to the right.
\end{minipage}}
% \\
\fbox{\begin{minipage}{4.75in}
\begin{FBD}{60}{10}{15}{10}{Earth}
\onele{50}{<m>F_{E,Z}=215\unit N</m>}{blue}
\end{FBD}
\vspace{-10pt}
\raggedright
Earth feels a force by \studentZ\ to the left.
\end{minipage}}

<em></em>Concepts to Consider:}  Newton's third law guarantees that the action-reaction force pairs, such as <m>F_{Z,C}</m> and <m>F_{C,Z}</m> or <m>F_{Z,E}</m> and <m>F_{E,Z}</m>, are equal and opposite.  There is no such guarantee on <m>F_{Z,C}</m> and <m>F_{Z,E}</m>.  These are equal because \studentZ\ chose to make <m>F_{C,Z}</m> and <m>F_{E,Z}</m> equal.  \HeZ\ pushed on the two others in equal amounts so that the reaction forces that act on \himZ\ will balance for Newton's <em></em>second} law so that \hisZ\ acceleration would be zero.

<em></em>Solution to the example:}  After using Newton's third law to find the forces on \studentZ, we can use Newton's second law to find \hisZ\ acceleration:
<me> a_Z = \frac{F_{Z,\mathrm{net}}}{m_Z} = \frac{\left[ \vec F_{Z,C} + \vec F_{Z,E} \right]}{\massZ} = \frac{\left[ \left( -215\unit N \ihat \right) + \left( +215\unit N \ihat \right) \right]}{\massZ} = 0 \unitfrac{m}{s^2}  </me>

%\begin{quote}
<em></em>Aside:} This example only considers the left-right forces that act in order to make a point about our intuition regarding forces we intend to apply.  Please consider how \protect{<xref ref="" />f-firstFBDupdate}} updates <xref ref="" />f-firstFBD} to make yourself aware of the other forces that are acting here, but are being ignored.
%\end{quote}

<assemblage>Return to: </assemblage> <xref provisional=""></xref>[action-reaction]{d-NIIIbracing}
\end{minipage}}
\end{example}

\begin{example}[p]
\fcolorbox{black}{yellow!10}{\begin{minipage}{4.925in}\setlength{\parskip}{3pt}
\caption{\label{ex-unbraced} \studentD<idx><h sortby="\studentD">\studentD</h></idx> does not brace \himselfD\ when pushing \studentC<idx><h sortby="\studentC">\studentC</h></idx>.}
(To better understand <xref ref="" text="title"></xref>[ss-NIII]{Newton's third law}, you should compare this example to <xref ref="" />ex-braced}  [pg.~\pageref{ex-braced}].)
\begin{quote}
In the lab room one day, while waiting for the instructor, \studentD<idx><h sortby="\studentD">\studentD</h></idx> (who has a mass of <m>m_D=\massD</m>) decides to try a physics experiment to test Newton's third law.  \HeD\ politely asks \hisD\ lab partner, \studentC<idx><h sortby="\studentC">\studentC</h></idx> (<m>m_C=\massC</m>), to turn \hisC\ back while \heD\ squares his feet underneath \himselfD\ and pushes with a force of <m>\vec F_{C,D} = 215\unit{N}\ihat</m>.  Despite the experience of <xref ref="" />ex-braced} (as told in <xref ref="" />cyoa-NIII}  [pg.~\pageref{cyoa-NIII}]), \studentC\ reluctantly agrees.  How does \underline{\studentD}{} accelerate during this exchange?
\end{quote}
%
%\begin{quote}
%Aside: Newton's second law tells us how this affects \studentC. See <xref ref="" text="type-global" />se-netF-a} and homework problem <xref ref="" text="type-global" />hmwk-pushbrace}.
%\end{quote}
%\noindent
<em></em>What do we know?}  As usual, it is convenient to start with a picture to help decide on the appropriate coordinate system.
We can also list
\\[2pt]
\begin{minipage}{2.6in}
the information that we know.
We know <m>m_D</m>, which is useful for relating <m>F_{D,\mathrm{net}}</m> to <m>a_D</m>.
We know <m>m_C</m>, which is useful for relating <m>F_{C,\mathrm{net}}</m> to <m>a_C</m>.  (This is not asked for, but is asked in homework problem <xref ref="" text="type-global" />hmwk-pushbrace}.)
We know <m>F_{C,D}</m>, how hard \studentD\ pushes on \studentC.
\end{minipage}
\hfill
\begin{minipage}{150pt}
\begin{picture}(150,90)(-30,-25)
% Dimensions and offset: (width,height)(x offset,y offset)
% Insert picture commands (\line,\circle, etc...) here:
\drawbox{-10}{-20}{120}{20}  % Earth
\drawbox{25}{1}{20}{40} %\studentD
\drawbox{45}{25}{18}{5} %\studentD's arms
\drawbox{65}{1}{20}{45} %\studentC
\put(25,43){\scriptsize \studentD}
\put(65,48){\scriptsize \studentC}
\put(40,-15){\scriptsize Earth}
\put(-40,24){\begin{minipage}{58pt}
\color{blue} \raggedright \scriptsize \studentD\ does not brace \himselfD. \\\hfill <m>\searrow</m>
\end{minipage}}
\end{picture}
\end{minipage}
%\hfill {}
\\[2pt]
We also know that neither person is bracing for the push. So, both \studentC\ and \studentD\ each only feel one force.

<em></em>What do we want to know?}  We want to know about the forces acting on \studentD, in order to find  <m>F_{D,\textrm{net}}</m> and therefore <m>a_D</m>.

<em></em>How are these related?}  First, since \studentD\ exerts a force on \studentC, Newton's third law tells us that \studentD\ feels a force of \mbox{<m>F_{D,C}=-215 \unit{N}\ihat</m>}.
Second, unlike \studentZ\ in <xref ref="" />ex-braced}  (pg.~\pageref{ex-braced}), \studentD\ chooses not to exert a force on the Earth in the <m>-\ihat</m> direction.

<em></em>Free-Body Diagrams:}  We again draw free-body diagrams:

\noindent % \textwidth default is 5in for a book
\fbox{\begin{minipage}{2.25in}
\begin{FBD}{10}{20}{15}{10}{\studentD}
\onele{50}{<m>F_{D,C}=215\unit N</m>}{black}
\end{FBD}
\vspace{-10pt}
\raggedright
\studentD\ is pushed by \studentC\ to the left.
\end{minipage}}
\hfill
\fbox{\begin{minipage}{2.25in}
\begin{FBD}{10}{23}{15}{10}{\studentC}
\oneri{50}{<m>F_{C,D}=215\unit N</m>}{black}
\end{FBD}
\vspace{-10pt}
\raggedright
\studentC\ feels \studentD\ push to the right.
\end{minipage}}

{}\hfill {\footnotesize\autoref*{ex-unbraced} continued on next page<ellipsis />}
\end{minipage}}
\end{example}
\begin{example}[p]
\fcolorbox{black}{yellow!10}{\begin{minipage}{4.925in}\setlength{\parskip}{3pt}
{\footnotesize \autoref*{ex-unbraced} continued from previous page<ellipsis />}

<em></em>Concepts to Consider:}  Newton's third law guarantees that the action-reaction force pairs, <m>F_{D,C}</m> and <m>F_{C,D}</m>, are equal and opposite.  Because these forces are not on the same person, we cannot add these forces.  Newton's second law will then indicate how each person accelerates.

<em></em>Solution to the example:}  After using Newton's third law to find the forces on \studentD, we can use Newton's second law to find \hisD\ acceleration:
<me> a_D = \frac{F_{D,\mathrm{net}}}{m_D} = \frac{\left[ \vec F_{D,C} \right]}{\massD} = \frac{\left[ \left( -215\unit N \ihat \right) \right]}{\massD} = -\sigfrac{2.68}{75}{m}{s^2} \ihat </me>

%\begin{quote}
<em></em>Aside:} This example only considers the left-right forces that act in order to make a point about our intuition regarding forces we intend to apply.  Please consider how \protect{<xref ref="" />f-firstFBDupdate}} updates <xref ref="" />f-firstFBD} to make yourself aware of the other forces that are acting here, but are being ignored.
%\end{quote}
\flushright
<assemblage>Return to: </assemblage>{\mmr{<xref ref=""></xref>{d-NIIIbracing}{the discussion of action-reaction forces}}, \mmr{<xref ref="" />ex-braced}}}
\end{minipage}}
\end{example}

        </section>

</section><section><title></title>Summary and Homework}

</subsection><subsection><title></title>Summary of Concepts and Equations}

This chapter introduced the way physicists describe forces.  The concept of force encodes how objects interact.
After reading this chapter, you should be comfortable responding to the following questions or comments.
Unlike the other links in this book, if you follow the links in this summary section, there is no link to return to this page.  (This is on purpose to encourage you to answer these points without following these links.)
<ul>
</p></li><li><p> State Newton's Laws. <xref ref=""></xref>{sum-NewtonsLaws}{(Answer)}
</p></li><li><p> How is the unit of Newton related to the fundamental units of the SI system?  <xref ref="" text="title"></xref>[sss-unit-N]{(Answer)}
</p></li><li><p> How do you know when a system is in equilibrium? <xref ref="" text="title"></xref>[sss-equilibrium]{(Answer)}
</p></li><li><p> You should know how to draw a free-body diagram.  <xref ref="" text="title"></xref>[f-firstFBD]{(Example)}
</ul>

</subsection><subsection><title></title>Conceptual Questions}<todo>Add more conceptual questions</todo>
%\vspace{-24pt}
<ol>
</p></li><li><p> In order to climb a tree, you reach up and grab a branch and pull.  Most people refer to this as <q>pulling yourself up.</q> In terms of Newton's third law, describe what is happening in more technical terms.
</p></li><li><p> Some cars have a <q>cruise-control</q> feature that keeps your speed constant as you drive down the highway.  (a) If you are driving due north with the cruise-control on, are you in equilibrium?  (b) If, instead, you have the cruise-control set while you are following the road around a gradual curve of the road as it follows the shore of a lake, then are you in equilibrium?  (c) In both cases, how can you tell if you are in equilibrium?
</ol>
</subsection><subsection><title></title>Problems}<todo>Add more variety of problems.</todo>
%\vspace{-24pt}
<ol>
 </p></li><li><p>\label{hmwk-pushbrace} If \studentZ, with <m>m_Z=\massZ</m>, braces \himselfZ\ (so that he does not accelerate) and pushes \studentC\ (<m>m_C=\massC</m>) with a force of <m>\vec F_{C,Z} = 215\unit{N}\ihat</m>, find the following:
<ol>
    </p></li><li><p> What is the acceleration of \studentC?  \answer{\mbox{<m>\deq\vec a_C = \frac{215 \unit{N}\ihat}{\massC} = \sigfrac{2.38}{9}{m}{s^2} \ihat</m>.}}
    </p></li><li><p> What net force does \studentZ\ feel? \answer{<m>F_{Z,\mathrm{net}}=0\unit N</m>}
    </p></li><li><p> If \studentZ\ braces \himselfZ\ against the Earth, then what must that bracing force be?  \answer{<m>\vec F_{E,Z} = -215\unit{N}\ihat</m>}
    </p></li><li><p> What are the individual forces that \studentZ\ feels? \answer{<m>F_{Z,C}=-215\unit N \ihat</m> and <m>F_{Z,E}=215\unit N \ihat</m>}
    </p></li><li><p> What is the acceleration of the Earth?  \answer{\mbox{<m>\deq\vec a_E = \frac{-215 \unit{N}\ihat}{5.97\ten{24}\unit{kg}} = -\sigfrac{3.60}{1\ten{-23}}{m}{s^2} \ihat</m>.}}
    </p></li><li><p> Which of Newton's laws allows you to answer each of these questions?
</ol>
</p></li><li><p> If you apply a force of <m>4.65\unit N</m> to a mass of <m>2.18\unit{kg}</m>, then how much will it accelerate?
</p></li><li><p> How much force must you apply to cause a mass of <m>80.0\unit{kg}</m> to accelerate at <m>a=0.795\unitfrac{m}{s^2}</m>?
</p></li><li><p> You arrive home to find a box that came in the mail.  You find that you have to exert <m>54.3\unit N</m> to cause it to accelerate <m>a=1.25\unitfrac{m}{s^2}</m>.  (a) What is its mass?  (b) Is that a heavy box or a light box?  (c) Is it likely that this box would fit in a mailbox?
</p></li><li><p> Your <m>2538 \unit{kg}</m> car has run out of gas.  So you ask your friend, \studentB{} who has a mass of <m>\massB</m>, to put it in neutral, sit inside, and steer while you push.  If you apply enough force to cause a net forward force of magnitude <m>37.5\unit N</m>, how much time will it take for the car to move faster than you can walk?  Assume your walking speed is <m>3.0\unitfrac{mi}{hr}</m>.  How far will the car have travelled in that time?
</p></li><li><p> Find the components of the net force on a large crate if three forces are applied: <m>\vec F_1 = -3.0\unit N \ihat + 2.5 \unit N \jhat</m>, <m>\vec F_2 = -6.25\unit N \jhat</m>, and <m>\vec F_3 = 4.5\unit N \ihat + 1.63 \unit{N} \jhat</m>.
</p></li><li><p> Find the components of the net force on a large crate if three forces are applied: <m>F_1 = 3.61 \unit N </m> at <m>71.6^\circ</m> north of east, <m>F_2 = 4.61\unit N</m> due west, and <m>F_3 = 8.13\unit N</m> at <m>21.8^\circ</m> south of east.
</p></li><li><p> Find the magnitude and direction of the net force on a large crate if three forces are applied: <m>\vec F_1 = 4.25\unit N \ihat - 4.66 \unit N \jhat</m>, <m>\vec F_2 = -2.65\unit N \jhat</m>, and <m>\vec F_3 = -5.4\unit N \ihat + 2.93 \unit{N} \jhat</m>.
</p></li><li><p> Find the magnitude and direction of the net force on a large crate if three forces are applied: <m>F_1 = 2.65 \unit N </m> at <m>26.6^\circ</m> north of west, <m>F_2 = 2.22\unit N</m> at <m>56.31^\circ</m> south of west, and <m>F_3 = 7.12\unit N</m> at <m>28.4^\circ</m> north of east.
</ol>



</chapter><chapter><title></title>The Many Types of Force}\label{c-forcetype}<aside><title>Referenced by</title> <p>Discussion of <xref provisional=""></xref></p></aside>[subscript notation of forces]{d-interaction}

</section><section><title></title>Gravity at the Surface of the Earth}\label{s-Fg}<aside><title>Referenced by</title> <p></p></aside>{\mmr{<xref ref=""></xref>{d-accgrav}{freefall}}, \mmr{<xref ref="" />f-firstFBD}}}<!-- -->\new{v2.2}{Adding detail}

Perhaps the force that is the most obvious to humanity is the one that helps us fall when we stumble: the gravitational force<idx></idx>{Gravity!Surface of Earth}.  This is one of the fundamental forces discussed in <xref ref="" />s-fundamental}.  In addition, the details about how the planets, moon, and the sun experience this force will be discussed in <xref ref="" />c-gravity}.  For now, we can consider how this interaction manifests itself on our daily lives.  In this section, we will start with how objects move when the gravitational force is the only force acting.  Subsections~<xref ref="" text="type-global" />ss-weightmass} and~<xref ref="" text="type-global" />ss-equivmm} will clarify some subtleties and then we'll jump into the examples in <xref ref="" />ss-local.mg}.

We can investigate what happens when the gravitational force is the only force acting on an object by holding it in the air and dropping it<idx></idx>{Freefall}.  One of the complications during such an experiment was discussed in <xref ref="" />ss-airresistance}.  If we drop a sheet of paper, there is air resistance in addition to the gravitational force.  For this section, I will assume that the mass-to-surface-area ratio is large enough that we can effectively\Touchstone{Recall \protect{<xref ref="" text="title"></xref>[s-effective2]{effective theories}}.}{} ignore the air resistance.

Since objects fall faster than humans are used to paying attention to, the <p xml:id=""></p>d-Fgrav}{patterns} are difficult to see.  The green box of <xref ref="" />irl-freefall} (on page~\pageref{irl-freefall}) shows you how you can pay close attention to the patterns that result from observing falling objects.
You should go do those experiments before reading further.  Go ahead.  I'll wait.

You did do them, right?  You're not just reading ahead?  Really?  OK.  Doing that experiment will help you see (1) that everything falls at the same rate and (2) that objects accelerate as they fall\phantomsection\label{d-Fgball}.  This first point is a bit less intuitive and will be discussed further in <xref ref="" />ss-equivmm}.  This second point should be exactly what you expect, when you consider <xref ref="" text="title"></xref>[ss-NII]{Newton's second Law}: If there is only one force (the gravitational force), then the object cannot be in <xref ref="" text="title"></xref>[sss-equilibrium]{equilibrium} and it must be accelerating.  (You should notice that this is the language of <xref ref="" text="title"></xref>[st-Fma]{the story of Newton's second law}.)

In order to evaluate this further, let's consider a specific object, like a baseball.  Our baseball has a mass of <m>m_b = 0.145\unit{kg}</m>.  If the only force acting <em></em>on} the ball is the gravitational force <em></em>by} the Earth, then the net force is the gravitational force: <m>\vec F_\mathrm{net} = \vec F_{bEg}</m>\Touchstone{<xref ref=""></xref>{d-interaction}{the on-by notation}}.  Here the subscripts are <m>b</m> (because the force is on the \underline{b}all), <m>E</m> (because the force is exerted by the \underline{E}arth), and <m>g</m> (because it is a \underline{g}ravitational force).  Since the acceleration is due to the gravitational force, I will use either <m>a_g</m> (usually when the object is in <xref ref="" text="title"></xref>[ss-freefall]{freefall} and therefore accelerating at this rate) or <m>g</m> (usually when the object is not actually accelerating at that rate).  With this notation, Newton's second law becomes:  <m> \vec F_{bEg} = m_b \vec a_g </m>
At this point, we know the mass, but we don't know the force or the acceleration.  However, we have conveniently already done the experiment (recall <xref ref="" />ex-freefall}) that will tell us the acceleration is <m>a_g = 9.81\unitfrac{m}{s^2}</m> downwards.  (Recall that <q>downwards</q> is the direction of the vector, which can be expressed as <m>-\jhat</m>.)  If we know the mass and the acceleration, then we can compute the force.
\begin{sample}
</p></li><li><p>\label{se-weightball} If a baseball with mass <m>m_b = 0.145\unit{kg}</m> is dropped (allowed to <xref ref="" text="title"></xref>[ss-freefall]{fall freely}) so that it accelerates at <m>a_g = 9.81\unitfrac{m}{s^2}</m> downwards, then while it falls it feels the gravitational force:
    <m> \vec F_g = m \vec g = (0.145\unit{kg}) [-(9.81\unitfrac{m}{s^2})\,\jhat] = -\sig{1.42}{24}{N} \jhat = -1.42 \unit N \jhat </m>
\end{sample}
This is the force of the gravitational force on the baseball.  Although we computed the force while the ball was falling, the gravitational force does not magically vanish when the ball is sitting on the floor.  So, we can say that (as long as the ball is close to the surface of the Earth, as noted in <xref ref="" />c-gravity}) the force always has this value.  Rather than continuing to say <q>the force of gravity</q> we call this force the weight<idx></idx>{Weight}.
\important{The weight of an object is computed as its mass times the acceleration due to gravity, even when the object is not actually accelerating at that rate:  <m>\mathbf{F_g \equiv mg}</m>.}

</subsection><subsection><title></title>Weight versus Mass}\label{ss-weightmass}<aside><title>Referenced by</title> <p></p></aside>{\mmr{<xref ref="" />ss-convertunits}}, \mmr{<xref ref="" />s-sigfig}}}<idx></idx>{Weight}<!-- -->\new{v2.2}{Added detail.  Moved the previous version to \protect{<xref ref="" />s-sigfig}} to smooth the transition to \protect{<xref ref="" />ss-equivmm}}.}

Since all objects have the same acceleration due to gravity at the surface of the Earth, the weight of an object and the mass of an object are very closely correlated, but they are not the same quantity.  This tends to cause some confusion when the discussion is not explicitly technical.  Recall the discussion about <xref ref="" text="title"></xref>[s-precision]{being precise in our language}.  One complication for people in the United States is that there are two definitions of the pound; one is a unit of mass<fn xml:id=""></fn>{There are also multiple versions of the pound-mass.  You can find these explained on the internet, but most of these are considered obsolete.  The one I will use is the <q>avoirdupois-pound</q>, which is defined in the NIST publication
% found in https://en.wikipedia.org/wiki/Pound_(mass)
\protect{<url href=""></url>{https://www.nist.gov/sites/default/files/documents/2017/04/28/AppC-12-hb44-final.pdf}{Handbook 44}}, page C-19, as exactly <m>453.592 37\unit{g}</m>.}
and the other is a unit of force.  Since the pound-force<fn xml:id=""></fn>{There is also a unit of force called the kilogram-force.} is defined as the standard unit of mass times the standard unit for the acceleration due to gravity,
% https://en.wikipedia.org/wiki/Pound_(force)
as discussed in <xref ref="" />s-SI-MKS}<todo>Update \protect{<xref ref="" />s-SI-MKS}} with this information.</todo>, the conversion directly from pound-force to Newtons will \underline{not} match the longer, but more appropriate, conversion from pound-mass to kilogram that gets multiplied by the local acceleration due to gravity (as opposed to the standard <m>g</m>) into Newtons.  It may also be useful to review the comments about unit-conversion in the section on <xref ref="" text="title"></xref>[s-sigfig]{significant digits}<idx></idx>{Significant Digits}.

In the discussion about <xref ref="" text="title"></xref>[s-precision]{being precise in our language}, we distinguished <q>massive</q> (the amount) from <q>voluminous</q> (the size).  Now that we understand <xref ref="" text="title"></xref>[ss-NII]{Newton's second law}, we can distinguish <q>massive</q>
%(an amount of material)
from <q>weighty.</q> %
(a strength needed to lift).
The concept that goes with
\important{mass is the amount of material,}
whereas, the concept that goes with
\important{weight is how strongly the gravitational force pulls on the object.}
Having mass affects both the inertia (ease of moving) and the weight (force of gravity).
Having weight expresses the gravitational force due to whichever large object (moon, planet, sun, etc.) you happen to be on or near.  Noticing that the <xref ref="" text="title"></xref>[s-SI-MKS]{SI-unit} is different for different types of quantities, such as a kilogram (a <xref ref="" text="title"></xref>[ss-units]{fundamental unit}) for mass and a Newton (a <xref ref="" text="title"></xref>[ss-units]{derived unit}) for weight, may help you remember that these are different kinds of quantities.

The interesting aspect of this relationship is that while having more mass makes an object harder to move (the same force produces less acceleration for more massive objects), when objects fall under the influence of the gravitational force, they accelerate at the <em></em>same} rate.  This reveals that the gravitational force must be stronger for more massive objects <em></em>by the exact amount} needed to compensate for that larger mass.  This is called the equivalence principle and is discussed in <xref ref="" />ss-equivmm}.


</subsection><subsection><title></title>Calculating the weight}\label{ss-local.mg}<!-- -->\new{v2.2}{renamed this section and added detail}

When calculating the forces acting on a person or an object, we will often need to account for the force of gravity, while other forces may also be at work.  As mentioned above, the weight is found by multiplying the mass times the local acceleration due to gravity, even if the object is not actually accelerating at that rate.  Chapter~<xref ref="" text="type-global" />c-gravity} will clarify why it is true<fn xml:id=""></fn>{The short answer is that the altitude (distance from the surface of the Earth) and local geology affect the strength of the gravitational field.  Since the Earth is slightly oblate (bulges at the equator), the altitude at different latitudes corresponds to a different distance from the center of the Earth.  In addition, while the spin of the Earth does not affect the strength of the gravitational field, it does affect how objects accelerate. The \protect{<url href=""></url>{http://www2.csr.utexas.edu/grace/gallery/animations/ggm01/ggm01_gif-200.html}{GRACE project}} has measured the variations across the globe.}, but for now please note that the acceleration due to gravity is (1) different according to where we are and also (2) the same for all objects at that location.<todo>Gather values of <m>g</m> at various locations.  Wiki has a list, but need to find the source.  Wolfram has numbers, but they seem to be calculated off a formula, not measurements.  \protect{<url href=""></url>{http://www.physics.montana.edu/demonstrations/video/1_mechanics/demos/localgravitychart.html}{U Montana}} has values but no reference.
\protect{<url href=""></url>{http://www.calpoly.edu/~gthorncr/ME302/documents/AccuracyofGravity.pdf}{Glen Thorncroft at Cal Poly}} has a formula and lists the level of each effect.</todo><idx></idx>{Acceleration!Gravity}<idx></idx>{Gravity!Acceleration}

<p xml:id=""></p>d-weightmass}{Because} of the peculiarities in the definition of pound (<xref ref="" />ss-weightmass}) it will be useful to build some intuition about masses in terms of kilograms and Newtons.  <xref ref="" />t-weightmass} lists the mass of some common objects and, using the standard value for <m>g</m>, their corresponding weights.
%
\begin{table}[bhtp]
\hrule\hrule
\begin{center}
\caption[Comparison of masses and weights of common objects]{\label{t-weightmass} The list of objects is intended to give a sense of scale so that the reader can better estimate the value of the mass of an object.  You might notice that (except for the apple) each of these is between 4 and 4.5 times heavier than the previous object.  Note that these are rough estimates; for example, while the author weighs about <m>200\unit{lbs}</m> this is not typical, nor average.
%<assemblage>Return to: </assemblage> <xref provisional=""></xref>[weight and mass]{d-weightmass}
% reference weight of an apple:  \url{http://www.applejournal.com/ref.htm}
}
\begin{tabular}{lrrr}
Object & pounds & mass (kg) & weight (N) \\ \hline
apple & 0.33 & 0.15 & 1.5 \\
lean, healthy cat & 10 & 4.6 & 45 \\
medium-sized dog & 44 & 20 & 196 \\
human & 200 & 91 & 890 \\
horse & 1000 & 362 & <m>3.56\ten{3}</m> \\
large pick-up truck & 4000 & <m>1.81\ten{3}</m> & <m>1.78\ten{4}</m>
\end{tabular}
\end{center}
\hrule\hrule
\end{table}
%
<xref ref="" text="title"></xref>[c-weightmass]{Conceptual Problem <xref ref="" text="type-global" />c-weightmass}} asks you to estimate the mass of some other common objects.  <xref ref="" text="title"></xref>[c-massweight]{Conceptual Problem <xref ref="" text="type-global" />c-massweight}} asks you to think of common objects with a specified mass.

Now let's do some calculations<ellipsis />
\begin{sample}
</p></li><li><p> \studentA<idx><h sortby="\studentA">\studentA</h></idx> notices that \heA\ needs to exert <m>F=1.5\unit{N}</m> to support the apple listed in <xref ref="" />t-weightmass}. \HeA\ then drops it  and notices its acceleration of <m>9.81\unitfrac{m}{s^2}</m>.  \HeA\ computes the mass to be
    <me> m = \frac{F_g}{a_a} \ = \ \frac{1.5\unit{N}}{9.81\unitfrac{m}{s^2}} \ = \ \frac{1.5\unitfrac{kg \cdot m}{s^2}}{9.81\unitfrac{m}{s^2}} \ = \ 0.\sig{15}{3}{kg} </me>
    (If you know the weight, you can compute the mass, even if the mass is not actually in freefall.)
</p></li><li><p>\label{se-weightA} \studentA<idx><h sortby="\studentA">\studentA</h></idx><!-- -->\new{v2.3}{modified and supplemented}, who knows \hisA\ own mass (<m>\massA</m>), then imagines<aside><title>Referenced by</title> <p></p></aside>{\mmr{<xref ref="" />f-firstFBDupdate}}, \mmr{<xref ref="" />f-firstFBDangle}}} dropping \himselfA\ (!) from a (small) height.  While \heA\ falls, \heA\ recognizes the gravitational force on \himA, which is computed to be
    <me> \vec F_g = m \vec g = (\massA) [-(9.81\unitfrac{m}{s^2})\,\jhat] = -\sig{833}{.85}{N} \jhat = -834 \unit N \jhat </me>
    Since \heA\ is in freefall and there is only one force is acting on \himA, the net force is easy to compute:  <m>\vec F_\mathrm{net} = -834 \unit N</m>.
    However, if you know the mass something, you can compute the weight even if that object is not in freefall.
    You should repeat this calculation for the mass in <xref ref="" text="type-global" />se-netF-a}.  (<xref ref="" text="type-global" />A-netF-a})
\end{sample}
You should note that
\important{<m>F_\mathrm{net} \ (=ma)</m> is always related to the actual acceleration of the object, \\ <m>F_g\ (=mg)</m> is always related to the local acceleration due to gravity.}
You should also note that
\important{the actual acceleration is only equal to the local acceleration due to gravity if the object is in freefall.}
\begin{sample}
</p></li><li><p>\label{se-FNB} If<aside><title>Referenced by</title> <p></p></aside>{\mmr{<xref ref="" />f-firstFBDupdate}}, \mmr{<xref ref="" />f-firstFBDangle}}} \studentB<idx><h sortby="\studentB">\studentB</h></idx> is not falling, but rather standing safely on the floor, then the gravitational force is still acting.  It can be computed as
    <m> \vec F_g = m \vec g = (\massB) [-(9.81\unitfrac{m}{s^2})\,\jhat] = -\sig{735}{.75}{N} \jhat = -736 \unit N \jhat </m>
    However, since we can see that \hisB\ acceleration is zero, the <m>\vec F_\mathrm{net}</m> <em></em>must be zero}.  The only way that can happen, though is if there is another force acting upwards on \studentB.  What could possibly be pushing up on \himB?  <xref ref="" text="type-global" />A-floor}.  Whatever it is pushing up on \himB, it is supplying a support force, which can be calculated since <m>\vec F_\mathrm{net} = \vec F_g + \vec F_\mathrm{support}</m> and we can solve for
    <me> \vec F_\mathrm{support} = \vec F_\mathrm{net} - \vec F_g = m\left(0\unitfrac{m}{s^2}\right) - \left[ -(\sig{735}{.8}{N}) \jhat\right] = +736\unit N \jhat </me>
    Because it is in the direction opposite to <m>\vec F_g</m>, it is upwards <m>(+\jhat)</m>.

    Can you identify <em></em>why} the support force is equal in magnitude and opposite in direction to the gravitational force?
    \TWO{Newton's second law}{Newton's third law}{A-second}{A-third}
\end{sample}
As was mentioned earlier, the value of the acceleration due to gravity also varies across the surface, although this is less than about a percent or so (see~<xref ref="" />t-gworld}).
Nonetheless, this means that your weight can change even when your mass remains the same.
\begin{sample}
</p></li><li><p>\label{se-gworld} While talking to your friend \studentB<idx><h sortby="\studentB">\studentB</h></idx>, you learn that \hisB\ parents, \studentE<idx><h sortby="\studentE">\studentE</h></idx> and \studentF<idx><h sortby="\studentF">\studentF</h></idx>, grew up in Norway, visited Puerto Rico, and climbed Mount Everest before settling in the United States.  Using <xref ref="" />t-gworld}, compute \studentE's weight are each location, assuming \hisE\ mass is \massE.
<ol>
</p></li><li><p>[Norway] <m>F_g = mg = (\massE)(9.825\unitfrac{m}{s^2}) \ = \ \sig{933}{.4}{N}</m>
</p></li><li><p>[Puerto Rico] <m>F_g = mg = (\massE)(9.782\unitfrac{m}{s^2}) \ = \ \sig{929}{.3}{N}</m>
</p></li><li><p>[Mount Everest] <m>F_g = mg = (\massE)(9.763\unitfrac{m}{s^2}) \ = \ \sig{927}{.5}{N}</m>
</ol>
\end{sample}
%
Because the variation is small, throughout this text when we are considering situations <q>at the surface of the Earth</q>, we will assume that
\important{the acceleration due to gravity is <m>9.81\unitfrac{m}{s^2}</m> to three significant figures.}
%
\begin{table}[bhtp]
\hrule\hrule
\begin{center}
\caption[Comparison of <m>g</m> at a few places on Earth]{\label{t-gworld} Comparison of <m>g</m> at a few places on Earth.  {\color{gray} [While both the latitude-longitude and the local value of <m>g</m> were found using the
<url href=""></url>{https://www.wolframalpha.com/}{WolframAlpha<m>^R</m> computational knowledge engine},
these <m>g</m> values do not necessarily correspond to these coordinates.  The <m>g</m> values are based on a theoretical model of the Earth.]}
You should look for a pattern as the latitude increases.  (<xref ref="" text="type-global" />A-gworld})
You might notice the values for  Mount Everest and Denver; Can you explain any peculiarity?  (<xref ref="" text="type-global" />A-gpeaks})
<assemblage>Return to: </assemblage> <xref provisional="" />{se-gworld}
}
\begin{tabular}{lccr}
Location & latitude & longitude & local <m>g (\!\!\unitfrac{m}{s^2})</m> \\ \hline
San Juan, Puerto Rico & <m>18^\circ 26' 24</q> \unit{N}</m>  & <m>66^\circ 7' 48</q> \unit W</m> & <m>9.782 \unitfrac{m}{s^2}</m> \\
Brownsville, TX & <m>26^\circ 1' 6</q> \unit{N}</m>  & <m>97^\circ 27' 14</q> \unit W</m> & <m>9.788 \unitfrac{m}{s^2}</m> \\
Mount Everest & <m>27^\circ 59' 17</q> \unit{N}</m>  & <m>86^\circ 55' 31</q> \unit E</m> & <m>9.763\unitfrac{m}{s^2}</m> \\
Cincinnati, OH & <m>39^\circ 8' 24</q> \unit{N}</m>  & <m>84^\circ 30' 23</q> \unit W</m> & <m>9.801\unitfrac{m}{s^2}</m> \\
Denver, CO & <m>39^\circ 45' 43</q> \unit{N}</m>  & <m>104^\circ 52' 50</q> \unit W</m> & <m>9.798\unitfrac{m}{s^2}</m> \\
Paris, France & <m>48^\circ 51' 36</q> \unit{N}</m>  & <m>2^\circ 20' 24</q> \unit E</m> & <m>9.813\unitfrac{m}{s^2}</m> \\
Oslo, Norway & <m>59^\circ 54' 36</q> \unit{N}</m>  & <m>10^\circ 45' \phantom{24</q>} \unit E</m> & <m>9.825\unitfrac{m}{s^2}</m> \\
Anchorage, AK & <m>61^\circ 10' 39</q> \unit{N}</m>  & <m>149^\circ 16' 28</q> \unit E</m> & <m>9.826\unitfrac{m}{s^2}</m>
\end{tabular}
\end{center}
\hrule\hrule
\end{table}
%



</section><section><title></title>Fundamental Forces}\label{s-fundamental}<idx></idx>{Force!Fundamental}<!-- -->\new{v2.1}{Started the section on fundamental interactions.  Link ahead, rather than detailling here.}

The previous section describes our (macroscopic) experience of the gravitational interaction when standing on the surface of the Earth.  This is essentially the same across the surface, but does change with altitude and the difference can be measured on mountain tops and in caves.  In fact, one can use the differences from one location to another to predict where we might find a a pocket of oil.<!-- -->\new{v2.2}{Filled out the detail.  Changed the approach.}

In later <p xml:id=""></p>d-fundamental}{sections}, we will consider this and other interactions that depend on the physical properties, such as mass and charge.  All particles with the property of mass (which we will start to call gravitational charge) will interact according to the gravitational force; however, this description is better described by the mathematics in <xref ref="" />c-gravity}.  All particles with the property of electrical charge will interact according to the electrical force.  The basic theory will be discussed in <xref ref="" />c-electric}.  A more complicated version that incorporates quantum mechanics is called quantum electrodynamics (QED) and this will be touched on in <xref ref="" />ss-QED}.  Particles like protons and neutrons (hadrons) are actually made up of other particles (quarks) that are held together by an interaction that is sometimes called the strong nuclear force (<xref ref="" />ss-strong}) and is described by the theory of quantum chromodynamics (QCD); this will be touched on in <xref ref="" />ss-QCD}.  Finally, in <xref ref="" />ss-weak} another fundamental force, called the weak nuclear force, will be discussed.

For the most part, these theories describe the interaction between microscopic particles, so we will not discuss them in detail here.  However, the gravitational interaction is exception in a variety of ways.  In particular, the gravitational interaction does affect macroscopic objects.  These fundamental forces have a particular description that allows us to pretend (recall <xref ref="" text="title"></xref>[s-effective2]{effective theories}) that they are action-at-a-distance interactions.  All other forces (introduced next) will require physical contact in order to exert the force.

</section><section><title></title>Normal Force}\label{s-FN}<aside><title>Referenced by</title> <p></p></aside>{\mmr{<xref ref="" />f-firstFBD}}, \mmr{<xref ref="" text="type-global" />A-floor}}, \mmr{<xref ref="" />s-FT}}}<!-- -->\new{v2.2}{Added detail}<idx></idx>{Force!Normal}

The word <q>normal</q> <url href=""></url>{http://etymonline.com/index.php?term=normal}{originates}<idx></idx>{Normal} with the idea of conformity to the pattern.  While in everyday life this the typical state of being, the origins actually refer to a carpenter's square, which put corners into a right angle.  In math and physics, the word is used to mean perpendicular.  In the context of forces,
\important{the normal force is the force that a surface exerts to keep objects from passing through them.  The direction of this force is always in the outward direction, normal (perpendicular) to the surface.}

Let's consider some specific situations<ellipsis />\inlife{} In <xref ref="" text="type-global" />se-FNB}<todo></todo>DO we need to repeat the example here? no?}, \studentB\ felt the downwards gravitational force even while \heB\ was standing on the ground.  We noticed that \heB\ was not falling (and so not accelerating).  Colloquially, we say that the ground is supporting \studentB.  This support force is keeping \studentB\ from passing through the floor; this is a normal force.  The normal force from the floor is acting upwards, which is normal (perpendicular) to the surface of the floor.  <xref ref="" />f-firstFBDupdate} updates the free-body diagrams of <xref ref="" />f-firstFBD} to show how the gravitational and normal forces impact that calculation.
%
\begin{figure}
\hrule\hrule
\caption{\label{f-firstFBDupdate} An updated version of \protect{<xref ref="" />f-firstFBD}}, people pushing a box.}<idx></idx>{Free-Body Diagrams!Images}
Again, we can start by drawing a picture of the situation.  The description is the same as it was for <xref ref="" />f-firstFBD}.  In addition to those forces, each of the three bodies has a downwards gravitational force.  This analogous to the calculation in <xref ref="" text="type-global" />se-weightA}, which was only for \studentA<idx><h sortby="\studentA">\studentA</h></idx>; but you can calculate the weight for the mass in <xref ref="" text="type-global" />se-netF-a} and \studentB<idx><h sortby="\studentB">\studentB</h></idx>'s weight was computed in <xref ref="" text="type-global" />se-FNB}.  In addition to the downward gravitational force (the weight), Newton's second law and the fact that nothing is accelerating up or down together tells us that

\noindent
\begin{minipage}[b]{150pt}
there must also be a normal force on each body.  This is analogous to the calculation in <xref ref="" text="type-global" />se-FNB}, which was only for \studentB; but you can deduce it for the object and for \studentA.
\end{minipage}
\hfill\begin{minipage}[b]{220pt}
\begin{picture}(220,85)(-10,-25)
\put(0,0){\line(1,0){200}}
\put(60,2){\line(1,0){60}}
\drawbox{30}{1}{20}{50} %\studentA
\drawbox{50}{25}{18}{5} %\studentA's arms
\put(30,53){\scriptsize \studentA}
\drawbox{70}{3}{20}{30} % object
\put(70,35){\scriptsize object}
\drawbox{150}{1}{20}{40} %\studentB
\drawbox{134}{25}{16}{5} %\studentB's arms
\put(150,43){\scriptsize \studentB}
\put(90,27.5){\oval(2,2)[r]}
\put(91,27.5){\line(1,0){43}}
\put(60,-12){\begin{minipage}{60pt}
\scriptsize The object is on a sheet of ice.
\end{minipage}}
\end{picture}
\end{minipage}


Now, as in <xref ref="" />f-firstFBD}, we will draw a free-body diagram for each individual separately.  However, this time we will use <xref ref="" text="type-global" />se-weightA} and <xref ref="" text="type-global" />se-FNB} to include the gravitational force (the weight) and the normal force.

\noindent % \textwidth default is 5in for a book
\fbox{\begin{minipage}{1.5in}
\begin{FBD}{10}{25}{15}{80}{\studentA}
\onele{20}{<m>5\unit N</m>}{black}
\onedo{100}{<m>834\unit N</m>}{black}
\oneup{100}{<m>834\unit N</m>}{black}
\end{FBD}
\raggedright
Even with the vertical forces, \studentA\ still has a <m>\vec F_\mathrm{net} = -5.0\unit N \ihat</m>.
\end{minipage}}
\hfill
\fbox{\begin{minipage}{1.5in}
\begin{FBD}{10}{15}{15}{25}{object}
\twori{20}{<m>5\unit N</m>}{black}{16}{<m>4\unit N</m>}{black}
\onedo{35}{<m>20\unit N</m>}{black}
\oneup{35}{<m>20\unit N</m>}{black}
\end{FBD}
\raggedright
Even with the vertical forces, the object still has a <m>\vec F_\mathrm{net} = +9.0\unit N \ihat</m>.
\end{minipage}}
\hfill
\fbox{\begin{minipage}{1.5in}
\begin{FBD}{10}{20}{15}{75}{\studentB}
\onele{16}{<m>4\unit N</m>}{black}
\onedo{88}{<m>736\unit N</m>}{black}
\oneup{88}{<m>736\unit N</m>}{black}
\end{FBD}
\raggedright
Even with the vertical forces, \studentB\ still has a <m>\vec F_\mathrm{net} = -4.0\unit N \ihat</m>.
\end{minipage}}
\flushright
<assemblage>Return to: </assemblage>{\mmr{<xref ref="" />ex-braced}}, \mmr{<xref ref="" />ex-unbraced}}, \mmr{<xref ref="" />s-FN}}, \mmr{<xref ref=""></xref>{d-rope.net}{rope-tension}}, \mmr{<xref ref="" />f-firstFBDangle}}}
\hrule\hrule
\end{figure}

Let's consider some other specific situations<ellipsis /> If you decide to lean against a wall, the wall will provide a normal force that pushes horizontally, keeping you from moving through the wall.<!-- -->\new{v2.3}{Answered \protect{<xref ref="" text="type-global" />se-ladderN}} and its related problems.}
%
\begin{sample}
</p></li><li><p>\label{se-ladderN} \studentC\ leans a <m>22.7\unit{kg}</m> ladder against a wall at an angle of <m>75.5^\circ</m>, consistent with \protect{<url href=""></url>{https://www.osha.gov/}{OSHA}} standard \protect{<url href=""></url>{https://www.osha.gov/pls/oshaweb/owadisp.show_document?p_table=standards&p_id=10839}{1926.1053(a)(1)(ii)}}, so that about <m>\txtfrac{1}{8}</m> of the weight is leaning into the wall.  <ol>
</p></li><li><p> Find the magnitude and direction of the normal force exerted by the wall on the ladder.
</p></li><li><p> Find the magnitude and direction of the normal force exerted by the wall on the ladder.
    </ol>

Since the weight is <m>F_g = mg = (22.7\unit{kg})(9.81\unitfrac{m}{s^2}) = \sig{222}{.69}{N}</m>, an eighth of this is <m>\sig{27.8}{36}{N}</m>.  This force is pressing into the wall (horizontally, which I will choose as the <m>+\ihat</m> direction).  By <xref ref="" text="title"></xref>[ss-NIII]{Newton's third law} if the ladder presses into the wall with <m>\sig{27.8}{36}{N}</m> in the <m>+\ihat</m> direction (this is also a normal force), then the wall pushes the ladder with a normal force of <m>\sig{27.8}{36}{N}</m> in the <m>-\ihat</m> direction.  <em></em>Notice that this is normal (perpendicular) to the surface of the wall.}

Since the full weight of the ladder, <m>F_g = \sig{222}{.69}{N}</m>, is still pressing downwards <m>(-\jhat)</m> into the floor (as a normal force), <xref ref="" text="title"></xref>[ss-NIII]{Newton's third law} says that the floor pushes the ladder upwards <m>(+\jhat)</m> with a normal force of <m>\sig{222}{.69}{N}</m>.  <em></em>Notice that this is normal (perpendicular) to the surface of the floor.}

<xref ref="" />ex-ladder2} goes into the full details of how one calculates the necessary values.
\end{sample}
%
If you lose control of your car and run into a tree, the tree also provides a normal force pushing the car away from the tree; this normal force will stop the car.
%
\begin{sample}
</p></li><li><p>\label{se-tree} \studentZ<idx><h sortby="\studentZ">\studentZ</h></idx> is driving home after a late night of studying at the library.  \HeZ\ is kind of tired and drifts off during the drive.  While traveling <m>\vec v_i = 13.0\unitfrac ms \ihat</m>, \studentZ\ runs into a tree, bringing \hisZ\ car <m>(m=2.1\ten{3}\unit{kg})</m> to a halt in <m>\Delta t = 0.243\unit s</m>.  (\studentZ\ remains unharmed because \heZ\ was awake enough to wear \hisZ\ seatbelt and
\noindent
\begin{minipage}[b]{240pt}
had a car with a functioning airbag.  Whew.)  Find the normal force by the tree on the car. \\

To be clear about what is happening, I will draw the picture. In order to find the force, we will first need to find the acceleration.
\end{minipage}
\hfill\begin{minipage}[b]{130pt}
\begin{picture}(120,80)(-10,-5)
\put(0,0){\line(1,0){100}}
\drawbox{70}{1}{20}{50} %\studentA
\drawbox{10}{5}{30}{20} % object
\put(15,3){\circle{5}}
\put(35,3){\circle{5}}
\put(0,40){\scriptsize <m>v=13.0\unitfrac ms</m>}
\put (10,35){\vector(1,0){30}}
\put(72,33){\scriptsize Tree}
\put(15,15){\scriptsize car}
\end{picture}
\end{minipage}
<me> \vec a = \frac{\vec v_f-\vec v_i}{\Delta t} = \frac{(0\unitfrac ms)-(13.0\unitfrac ms \ihat)}{0.243\unit s} = -\sigfrac{53.4}{9}{m}{s^2}\ihat </me>
That the acceleration is in the direction opposite the velocity corresponds to the object slowing down.  Now we can find the \underline{net force} from Newton's second law:
<me> \vec F_\mathrm{net} = m \vec a = (2.1\ten{3}\unit{kg})(-\sigfrac{53.4}{9}{m}{s^2}\ihat) = -\sig{1.1}{2\ten{5}}{N} \ihat </me>
There are three forces acting on the car, as can be seen in the free-body diagrams of <xref ref="" />f-firstFBDupdate}.  So, we can draw a free-body diagram here as well.  The gravitational force on the car is
<me> \vec F_g = m\vec g = (2.1\ten{3}\unit{kg})(-9.81\unitfrac{m}{s^2}\jhat) = -\sig{2.0}{6\ten{4}}{N}\jhat </me>
Since this in the vertical direction and the net force is in the horizontal direction, there must be an upwards normal force from the ground
\begin{minipage}[b]{240pt}
<me> F_{N,\mathrm{ground}} = \sig{2.0}{6\ten{4}}{N}\jhat</me>.
<em></em>This is normal (perpendicular) to the surface of the ground.} \\

The remaining horizontal force is the normal force from the tree,
<me> \deq F_{N,\mathrm{tree}} = -\sig{1.1}{2\ten{5}}{N} \ihat</me>.
\end{minipage}
\hfill\begin{minipage}[b]{130pt}
\fbox{\begin{minipage}[b]{100pt}
\begin{FBD}{15}{10}{15}{25}{car}
\onele{40}{<m>F_{N,\mathrm{tree}}</m>}{rgb:red,0;green,2;blue,1}
\onedo{30}{<m>F_g</m>}{rgb:red,0;green,2;blue,1}
\oneup{30}{<m>F_{N,\mathrm{ground}}</m>}{rgb:red,0;green,2;blue,1}
\end{FBD}
\end{minipage}}
\end{minipage}
<em></em>This is normal (perpendicular) to the surface of the tree.}
\end{sample}%
(Notice that <xref ref="" text="type-global" />se-tree} also shows why it is not always necessary to consider the vertical forces when we <q>know</q> that they cancel.)  If you throw a ball at the ceiling, the ceiling will provide a normal force downwards, keeping the ball from moving through the surface.
%
\begin{sample}
</p></li><li><p>\label{se-ceiling} \studentC<idx><h sortby="\studentC">\studentC</h></idx> recalls that one time \heC\ got bored one day in physics class (what?!?) and tossed a baseball (<m>m_b = 0.145\unit{kg}</m>) at the ceiling<ellipsis /> a little too hard <ellipsis /> as recounted in <xref ref="" />ex-ceiling}.  The acceleration during that collision with the ceiling was <m>\vec a = - \sigfrac{28.0}{9}{m}{s^2} \jhat</m>.  Find the normal force by the ceiling on the ball.

There are five stages to the motion: (a) throwing, (b) falling up, (c) hitting the ceiling, (d) falling down, and (e) catching show the forces involved. \\
\color{lightgray}
\fbox{\begin{minipage}[b]{55pt}
\begin{picture}(50,100)(0,0)
\put(25,25){\circle{10}}
\put(25,26){\vector(0,1){25}}
\put(25,24){\vector(0,-1){15}}
\put(28,35){<m>F_\mathrm{throw}</m>}
\put(28,10){<m>F_g</m>}
\end{picture}
\centering{(a) throwing}
\end{minipage}}
\hfill
\fbox{\begin{minipage}[b]{55pt}
\begin{picture}(50,100)(0,0)
\put(25,50){\circle{10}}
\put(25,50){\vector(0,-1){15}}
\put(28,35){<m>F_g</m>}
\end{picture}
\centering{(b) falling up}
\end{minipage}}
\hfill
\color{rgb:red,0;green,2;blue,1}
\fbox{\begin{minipage}[b]{55pt}
\begin{picture}(50,100)(0,0)
\put(25,95){\circle{10}}
\put(26,95){\vector(0,-1){25}}
\put(24,95){\vector(0,-1){15}}
\put(28,75){<m>F_N</m>}
\put(10,75){<m>F_g</m>}
\end{picture}
\centering{(c) \\ hitting}
\end{minipage}}
\hfill
\color{lightgray}
\fbox{\begin{minipage}[b]{55pt}
\begin{picture}(50,100)(0,0)
\put(25,50){\circle{10}}
\put(25,50){\vector(0,-1){15}}
\put(28,35){<m>F_g</m>}
\end{picture}
\centering{(d) falling down}
\end{minipage}}
\hfill
\fbox{\begin{minipage}[b]{55pt}
\begin{picture}(50,100)(0,0)
\put(25,25){\circle{10}}
\put(25,26){\vector(0,1){25}}
\put(25,24){\vector(0,-1){15}}
\put(28,35){<m>F_\mathrm{catch}</m>}
\put(28,10){<m>F_g</m>}
\end{picture}
\centering{(e) catching}
\end{minipage}}
\color{rgb:red,0;green,2;blue,1}
\\
In this particular problem, we are only concerned with step (c) when the ball hits the ceiling, because that is the only part where the normal force acts. <xref ref="" text="type-global" />se-throw-up} will describe what happens during steps (a) and (e).

During step (c), we have the actual acceleration, which tells us about the net force.  We will also need to know the weight of the baseball, because gravity is still acting during the collision.
<md>
\vec F_N + \vec F_g & = &  \vec F_\mathrm{net} \ = \ m \vec a \\
\vec F_N + m \vec g & = &  m \vec a \\
\vec F_N  & = &  m \vec a - m \vec g \\
\vec F_N  & = &  \left[ (0.145\unit{kg})(-\sigfrac{28.0}{9}{m}{s^2}\jhat) \right] - \left[ (0.145\unit{kg})(-9.81\unitfrac{m}{s^2}\jhat) \right] \\
\vec F_N  & = &  \left[ -\sig{4.07}{3}{N} \jhat \right] - \left[  - \sig{1.42}{2}{N} \jhat \right] \ = \ -\sig{2.65}{1}{N} \jhat
</md>
You can see that the downward normal force <m>(\sig{2.65}{1}{N})</m> combined with the downward gravitational force <m>(\sig{1.42}{2}{N})</m> together create the downward net force <m>(\sig{4.07}{3}{N})</m>.
\end{sample}
%
If you make a <p xml:id=""></p>d-bank-shot}{<q>bank shot</q>} with either a basketball off the backboard or a pool ball<fn xml:id=""></fn>{Resources for \protect{<url href=""></url>{http://wpapool.com/equipment-specifications/\#Balls-and-Ball-Rack}{specifications}} and
\protect{<url href=""></url>{http://c.ymcdn.com/sites/bca-pool.com/resource/resmgr/imported/BCAEquipmentSpecifications_2008.pdf}{a PDF version}}.
These provide:
    weight (<m>5.5\unit{oz}=0.\sig{155}{92}{kg}</m> and <m>6.0\unit{oz}=0.\sig{170}{097}{kg}</m> cue),
    diameter (<m>2.250\pm 0.005\unit{in}=\sig{5.71}{5}{cm}\pm 0.0127\unit{cm}</m>),
    rail height (<m>63.5 \%</m> of the ball height, <m>= \sig{3.62}{9}{cm}</m>),  and
    dimension limits on the cue stick:
        <m>L_\mathrm{min}=40.00\unit{in}=1.016\unit{m}</m>,
        <m>m_\mathrm{max} = 25.0\unit{oz} = 0.\sig{708}{75}{kg}</m>, and
        tip-width <m>w_\mathrm{max}=1.4\unit{cm}</m>.
You might also consider the information and calculations at
\protect{<url href=""></url>{http://billiards.colostate.edu/technical_proofs/index.html}{Dr.~Dave's site}},
which gives
    slow (<m>1\unit{mph}</m>), medium (<m>3\unit{mph}</m>), and fast (<m>7\unit{mph}</m>);
    coefficient of friction ball-to-ball <m>\mu=0.06</m>; and
    ball-ball collision times as <m>250\unit{\mu s}</m>-<m>300\unit{\mu s}</m>.
}
off the bumper, then the surface provides a normal force that is perpendicular to the surface, in this case redirecting the ball rather than stopping it.  Unfortunately, the actual mechanism is somewhat more complicated than we are ready for; these are considered a little bit in the <xref ref="" />irl-poolcushion} (pg.~\pageref{irl-poolcushion})<todo></todo>\protect{<xref ref="" />irl-poolcushion}} should be moved to a section that has more about friction and angular momentum.  It is too complex for this section.}.
%
\begin{reallife}[bthp]
\hspace{-.2in}
\fcolorbox{black}{green!10}{\begin{minipage}{5.29in} \center
\caption{\label{irl-poolcushion}<idx></idx>{Pool!Real Life} Pool balls and bumpers / cushions.}
\begin{minipage}{4.925in}
\studentD<idx><h sortby="\studentD">\studentD</h></idx> is relaxing with the local physics club, playing pool.  \HeD\ shoots a bank-shot and the ricochet reminds all of you about the normal force from the bumper on the ball.
\end{minipage}
\begin{realtable}
\dna{Find a billiards table}
    {Notice the felt, the bumpers (cushion), and the dimensions of the table}
    {Does the ball roll as far on felt as it does on hardwood?  <xref ref="" text="type-global" />A-pool.roll} \\
     How soft is the bumper? <xref ref="" text="type-global" />A-pool.bumper}}
\dna{Find a set of pool balls}
    {Compare the solid-colored balls, the striped balls, and the cue ball}
    {Are there differences in size of weight? <xref ref="" text="type-global" />A-noncue}}
\dna{Hit the cue-ball off of a bumper in the manner intended for
\protect{<url href=""></url>{http://c.ymcdn.com/sites/bca-pool.com/resource/resmgr/imported/BCAEquipmentSpecifications_2008.pdf}{testing cushions}}.}
    {Compare the angle it leaves the bumper (reflected angle) match the angle at which it came in (incident angle)}
    {Does the spin of the ball matter? <xref ref="" text="type-global" />A-pool.spin}}
\dna{Place a pool ball against the bumper and ricochet the cue ball off the pool ball instead of the bumper itself.}
    {Notice how the pool ball reacts. <xref ref="" text="type-global" />A-pool.later}}
    {Why does the pool ball jump off the bumper? \\
     Does the pool ball move along the wall? \\
     Where did you hit the pool ball?}
\end{realtable}
\begin{minipage}{4.925in}
Billiard tables have a lot of interesting physics, which can help us see a wide variety of physics, for example:
<xref ref="" text="title"></xref>[irl-poolfriction]{friction}, <xref ref="" text="title"></xref>[irl-poolelastic]{elastic versus inelastic collisions}, <xref ref="" text="title"></xref>[irl-poolrotmot]{rotational motion}, and <xref ref="" text="title"></xref>[irl-poolangmom]{angular momentum}.
\end{minipage}

\flushright
<assemblage>Return to: </assemblage> <xref provisional=""></xref>[pool]{d-bank-shot}
\end{minipage}}
\end{reallife}
%

</subsection><subsection><title></title>Bathroom Scales Measure the Normal Force}\label{ss-scales}<aside><title>Referenced by</title> <p>Discussion of <xref provisional=""></xref></p></aside>[uses of <m>F=ma</m>]{d-usesofFma}

To get a good sense of what how the normal force works, it helps to consider the way a bathroom scale works.  Consider the concepts presented in the <xref ref="" />irl-scale} (pg.~\pageref{irl-scale}).
%
\begin{reallife}[bthp]
\hspace{-.2in}
\fcolorbox{black}{green!10}{\begin{minipage}{5.29in} \center
\caption{\label{irl-scale}<idx></idx>{Force!Normal} Playing with a scale.}
\begin{minipage}{4.925in}
While speaking to your friend, \studentB<idx><h sortby="\studentB">\studentB</h></idx> about \hisB\ recent accomplishment of losing <m>45\unit{N}</m>, you mention that your scale always gives a different number than the one in the doctor's office.  You suggest \heB\ gets on your scale to verify the calibration.  \studentB\ currently has a mass of <m>\massB</m>.
\end{minipage}
\begin{realtable}
\dna{Try to lose <m>45\unit{N}</m>.}
    {Compare this to your weight}
    {Is this a lot of weight to lose? <xref ref="" text="type-global" />A-weight.loss}}
\dna{Place your toe on the scale while \studentB\ weighs \himselfB}
    {This increases the value the scale reads}
    {Does \studentB\ weigh more? <xref ref="" text="type-global" />A-weight.gain}}
\dna{With your hands, press down on \studentB's shoulders while \heB\ stands on the scale}
    {Control the value read by the scale.  Increase the reading by <m>20\unit{N}</m>, <m>30\unit{N}</m>, etc.}
    {Does \studentB's weight change?  <xref ref="" text="type-global" />A-weight.gain} Are you adding weight to the scale? <xref ref="" text="type-global" />A-scale.increase}}
\dna{Have \studentB\ lean on a nearby table or counter while \heB\ stands on the scale}
    {Control the value read by the scale.  Decrease the reading by <m>20\unit{N}</m>, <m>30\unit{N}</m>, etc.}
    {Does \studentB's weight change?  <xref ref="" text="type-global" />A-weight.gain} }
\dna{Hold the scale against the wall and press into it.}
    {Control the value read by the scale.  Increase the reading by <m>20\unit{N}</m>, <m>30\unit{N}</m>, etc.}
    {What is the scale measuring? <xref ref="" text="type-global" />A-scale.measure}}
\dna{Imagine placing a scale on a ramp that can be laid flat or raised to any angle up to a vertical (making it a wall)}
    {Imagine standing on the scale on the ramp while it is lifted from horizontal (like a floor) to vertical (like a wall)}
    {Does the scale always read the same value while it is raised to different angles? <xref ref="" text="type-global" />A-scale.ramp}}
\end{realtable}
%\begin{minipage}{4.925in}
%If you can control the value read by the scale while at the same time not changing your actual mass, does the scale literally measure the weight of the object on the scale?  <xref ref="" text="type-global" />A-scale.measure}
%\end{minipage}

\flushright
<assemblage>Return to: </assemblage> <xref provisional="" />{ss-scales}
\end{minipage}}
\end{reallife}
%
Some digital scales are inconvenient for understanding how they work because they don't display the value until it has come to something close to equilibrium.  If you have access to an analog scale, then you can watch the value change as it settles down and it might be easier to build your intuition.

As you consider the values that you read on the scale, you should consider what happens if you jump off of or land upon a scale.  <em></em>Note that actually doing this can decalibrate your scale, if not break it entirely.  Scales are not meant to be handled this way.} While you are jumping from your scale, it must provide not only the force necessary to support your weight, but also the upwards force require to accelerate you upwards.  While you are landing on the scale, it musty provide not only the force necessary to support your weight, but also the upwards force necessary to decelerate you.

Bathroom scales use leverage (i.e., <xref ref="" text="title"></xref>[s-leverarm]{torque}) and a <xref ref="" text="title"></xref>[s-springs]{spring}-system to balance the force pressing into them.  The mechanism can be seen at <url href=""></url>{http://home.howstuffworks.com/inside-scale.htm}{How Stuff Works}.

</section><section><title></title>Tension}\label{s-FT}<aside><title>Referenced by</title> <p>Discussion of <xref provisional=""></xref></p></aside>[<m>F=ma</m>]{d-fma}<idx></idx>{Force!Tension}

Where the <xref ref="" text="title"></xref>[s-FN]{normal force} is appropriate for pushing against surfaces,
\important{tension is the pulling force that is transmitted through materials \\ such as cable, chain, or rope.}
Tension is closely related to the compression force experienced by support beams.  One can simplistically think of tension as pulling<todo></todo>add a link to (and the section itself) to a section on the modulus and stress/strain.}{} and compression as pushing<todo></todo>Maybe add an IRL about a house settling and the compression forces.  Loading a pick-up truck and watching the bed sag as weight is added.  Hammock as an example of adding weight and increasing the tension force.}{} on the intermediate object that transmits force between the objects at either end.<fn xml:id=""></fn>{It doesn't usually make sense to talk about the compression of a rope or chain.}
When engineers design the skeleton of bridges and buildings, one of the primary considerations is the tension and compression of the steel beams.  You can build your intuition by considering the <xref ref="" />irl-tension} (pg.~\pageref{irl-tension}).<todo></todo>Still need to update the \protect{<xref ref="" />irl-tension}}.}
%
\begin{reallife}[bthp]
\hspace{-.2in}
\fcolorbox{black}{green!10}{\begin{minipage}{5.29in} \center
\caption{\label{irl-tension}<idx></idx>{Force!Tension} Pull my finger.}
\begin{minipage}{4.925in}
We talk about tension and stress in our daily lives.  This is an analogy to the physical version of tension, stress, and strain.  While \protect{<url href=""></url>{http://etymonline.com/index.php?allowed_in_frame=0&search=stress}{stress}} and \protect{<url href=""></url>{http://etymonline.com/index.php?search=strain&searchmode=&p=0&allowed_in_frame=0}{strain}} come from the the concept of tightening, tension \protect{<url href=""></url>{http://etymonline.com/index.php?allowed_in_frame=0&search=tension}{comes from}} the concept of stretching.
\end{minipage}
\begin{realtable}
\dna{Sit on a swing }
    {Notice the tightness of the support ropes/chains}
    {How tight are the supports when the swing is empty? When a small child is in the swing? When a full-sized adult is in the swing? <xref ref="" text="type-global" />A-swing.tension}}
\dna{Install a fan or light fixture that hangs from the ceiling}
    {You don't want the fan to be supported by the electrical wires, but rather by the metal shaft}
    {How is the fan supported? <xref ref="" text="type-global" />A-fan.tension}}
\dna{Pull on a doorknob}
    {Imagine replacing the knobs (inside and outside) with large knots on a rope that runs through the hole the doorknob used to occupy.}
    {What if the doorknob were replaced with a rope, knotted on either side of the door? [Answer]}
\dna{Take a dog for a walk on a leash}
    {Try to pay attention to Newton's second and third law when the dog changes its level of enthusiasm for pulling on the leash.}
    {If the dog pulls very hard on the leash and you balance that force without allowing the dog to move away from you, then describe the way the force connects you to the dog. [Answer]}
\end{realtable}
%\begin{minipage}{4.925in}
%If you can control the value read by the scale while at the same time not changing your actual mass, does the scale literally measure the weight of the object on the scale?  <xref ref="" text="type-global" />A-scale.measure}
%\end{minipage}

\flushright
<assemblage>Return to: </assemblage> <xref provisional="" />{s-FT}
\end{minipage}}
\end{reallife}
%

When considering the tension in the rope, the context is generally that the rope is connecting two objects that are trying to pull on each other.  It is convenient to recognize that each object only <q>sees</q> the rope, not the object at the far side.  This can be seen in a couple of contexts.<!-- -->\new{v2.4}{Modified}

We will start with the <xref ref="" text="title"></xref>[s-effective2]{simplistic approximation} of ropes that only transmit the force.  As your understanding improves, we will add some examples where the tension in the rope also affects the rope itself.  In that more complicated situation, the tension will change across the rope<todo></todo>maybe add links}{} and the rope may stretch<todo></todo>maybe add links}{}.  Since ropes and cables are twisted strands while chains are links, ropes and cables can also introduce a <xref ref="" text="title"></xref>[s-torsion]{torsion}\foreshadow{} that tend not to occur in chains.

</subsection><subsection><title></title>Tension as a Support Force}\label{ss-tension.support}

Ropes and chains (and beams) can use tension to support (from above) dangling objects.
\begin{sample}
</p></li><li><p>\label{se-purse} \studentD<idx><h sortby="\studentD">\studentD</h></idx> hangs her purse <m>(m=1.36\unit{kg})</m> on a hook.  How much tension is in the shoulder strap to keep it from falling?

The strap connects the hook to the purse.  We can consider the interaction between the hook and the strap or between the purse and the strap.  We will consider the latter since we don't know anything about the hook.  Considering the forces on the purse, we know that there is a downwards gravitational force of <m>\deq F_g = (1.36\unit{kg})(9.81\unitfrac{m}{s^2}) = \sig{13.3}{4}{N}</m> and that the net force must zero (because the purse is not accelerating). So, the strap must provide an upwards (tension) force.
<md>
\vec F_T + \vec F_g & = & m \zero{\vec a}{0} \\
\vec F_T + (-\sig{13.3}{4}{N} \jhat) & = & 0\unit N \\
\vec F_T & = & +\sig{13.3}{4}{N} \jhat
</md>
This is the upwards force that the strap applies to the purse; however, the tension strap is doing two jobs: It is pulling up on the purse (as indicated above) <em></em>and} it is pulling down on the hook.
\end{sample}
The important thing to take away from <xref ref="" text="type-global" />se-purse} is not that we can compute the value (although that is, of course, a useful skill), but rather that
\important{the tension is conveying the force between the two objects.}  In the same way that the <xref ref="" text="title"></xref>[s-FN]{normal force} on a scale does not measure your weight, but rather the amount you press into the scale, the tension passes force on to the attached object.  The hook doesn't feel the weight of the purse, but does feel the tension required to support the purse.

In <xref ref="" text="title"></xref>[sss-multiple.mass]{an upcoming section}, we will consider what happens when multiple masses are hung from the rope.

</subsubsection><subsubsection><title></title>How Physicists Use the Words (Vocabulary)}

You can probably think of several examples of objects dangling: a purse on a hook, a flag on a pole, a shop sign attached to a post, a pendulum,<todo></todo>Add an image of an immovable surface to that section}{} \\
\begin{minipage}{4.25in}
a swing set, etc.  Since these are all similar in some ways (although different in other ways), <em></em>we can treat all of them as a mass at the end of a rope}.  Typically, because we do not want to deal with the complications that come from sagging supports, we will use the <xref ref="" text="title"></xref>[s-effective2]{approximation} of an <q><em></em>immovable support}.</q>  This will be indicated by hashing the surface.
\end{minipage}
\hfill
\begin{minipage}{30pt}
\begin{picture}(35,80)
\put(0,70){\line(1,0){25}}
\multiput(5,70)(5,0){4}{\line(1,1){5}}
\put(12.5,70){\line(0,-1){50}}
\put(7.5,20){\line(1,0){10}}
\put(7.5,20){\line(0,-1){10}}
\put(17.5,10){\line(-1,0){10}}
\put(17.5,10){\line(0,1){10}}
\end{picture}
\end{minipage}

</subsection><subsection><title></title>Tension as Dragging Force}\label{ss-tension.drag}

We can also consider the <p xml:id=""></p>d-rope.net}{tension} in a rope used to drag an object across the floor.  You may recall that in <xref ref="" />f-firstFBD} (and the updated version, <xref ref="" />f-firstFBDupdate}) \studentB<idx><h sortby="\studentB">\studentB</h></idx> pulled a box across a sheet of ice.  It is possible that  \studentB\ was grabbing the object itself, but it is more likely that \heB\ was pulling on a rope that was attached to the object.  In that case, the tension in the rope was <m>4.0 \unit N</m>.  This tension is what pulled \studentB\ leftwards <em></em>and} what pulled the object rightwards.

We can further update this by considering the case where \studentB\ pulls the rope up at an angle.  In that case, some of the tension is used to drag the box and some is used to reduce the normal force.  In <xref ref="" />f-firstFBDangle}, we will have \studentA\ continue to push with <m>5.0\unit{N}</m> horizontally and have \studentB\ pull with <m>4.0\unit{N}</m> at a <m>14^\circ</m> angle above the horizontal.
%
\begin{figure}
\hrule\hrule
\caption{\label{f-firstFBDangle} An updated version of \protect{<xref ref="" />f-firstFBDupdate}}, people pushing a box.}<idx></idx>{Free-Body Diagrams!Images}
Again, we can start by drawing a picture of the situation.  The description is the same as it was for <xref ref="" />f-firstFBDupdate} except that \studentB\ pulls at a slight

\noindent
\begin{minipage}[b]{150pt}
angle upwards.  We will again need the gravitational force for \studentA<idx><h sortby="\studentA">\studentA</h></idx> (<xref ref="" text="type-global" />se-weightA}) and \studentB<idx><h sortby="\studentB">\studentB</h></idx> (<xref ref="" text="type-global" />se-FNB}).  As before, since nothing is accelerating up or down together, there must also be a normal force on each body.
\end{minipage}
\hfill\begin{minipage}[b]{220pt}
\begin{picture}(220,85)(-10,-25)
\put(0,0){\line(1,0){200}}
\put(60,2){\line(1,0){60}}
\drawbox{30}{1}{20}{50} %\studentA
\drawbox{50}{25}{18}{5} %\studentA's arms
\drawbox{70}{3}{20}{30} % object
\drawbox{150}{1}{20}{40} %\studentB
\drawbox{134}{25}{16}{5} %\studentB's arms
\put(90,16.5){\oval(2,2)[r]}
\put(91,16.5){\line(4,1){44}}
\put(30,53){\scriptsize \studentA}
\put(70,35){\scriptsize object}
\put(150,43){\scriptsize \studentB}
\put(60,-12){\begin{minipage}{60pt}
\scriptsize The object is on a sheet of ice.
\end{minipage}}
\end{picture}
\end{minipage}


Now, as in <xref ref="" />f-firstFBDupdate}, we will draw a free-body diagram for each individual separately.  However, this time we will put the tension of the rope at the appropriate angle.  We will need to do a small calculation to find the value of the normal forces.

\noindent % \textwidth default is 5in for a book
\fbox{\begin{minipage}{1.5in}
\begin{FBD}{10}{25}{15}{80}{\studentA}
\onele{20}{<m>5\unit N</m>}{black}
\onedo{100}{<m>834\unit N</m>}{black}
\oneup{100}{<m>834\unit N</m>}{black}
\end{FBD}
\raggedright
The forces on \studentA\ have not changed.
\end{minipage}}
\hfill
\begin{minipage}{1.5in}
\fbox{\begin{minipage}{1.5in}
\begin{FBD}{10}{15}{15}{25}{object}
\oneri{20}{}{black}\put(43,30){\color{black}\tiny  <m>5\unit N</m>}
\onedo{35}{<m>20\unit N</m>}{black}
\oneup{35}{<m>F_N</m>}{black}
\put(26,41){\color{black} \vector(4,1){20}}
\put(43,46){\color{black} \tiny <m>4 \unit{N}</m>}
\end{FBD}
\raggedright
The forces on the object <em></em>have} changed.
\end{minipage}}
\begin{picture}(100,60)
\put(0,10){\line(4,1){80}}
\put(0,10){\line(1,0){80}}
\put(80,10){\line(0,1){20}}
\put(15,10){\oval(5,8)[rt]}
\put(25,11){\tiny <m>14^\circ</m>}
\put(35,28){\tiny <m>F_T=4.0 \unit{N}</m>}
\put(82,30){\tiny <m>F_{Ty}=</m>}
\put(82,20){\tiny <m>= F_T\,\sin 14^\circ</m>}
\put(82,10){\tiny <m> = 0.\sig{96}{8}{N}</m>}
\put(5,0){\tiny <m>F_{Tx}=F_T \, \cos 14^\circ = \sig{3.8}{8}{N}</m>}
\end{picture}
\end{minipage}
\hfill
\fbox{\begin{minipage}{1.5in}
\begin{FBD}{10}{20}{15}{75}{\studentB}
%\onele{16}{<m>4\unit N</m>}{black}
\onedo{88}{<m>736\unit N</m>}{black}
\oneup{88}{<m>F_N</m>}{black}
\put(24,94){\color{black} \vector(-4,-1){20}}
\put(0,92){\color{black} \tiny <m>4\unit{N}</m>}
\end{FBD}
\raggedright
The forces on \studentB\ <em></em>have} changed.
\end{minipage}}

\noindent
<em></em>For the object}: Since the y-component of the net force is zero, we can find the normal force to be <m>F_N = -[(-20\unit{N})+(+0.\sig{96}{8}{N})] = 19\unit{N}</m>.
The x-component of the net force is <m>F_{\mathrm{net},x}=(5.0\unit N)+(\sig{3.8}{8}{N}) = \sig{8.8}{8}{N}</m>.

<em></em>For \studentB}: Since the y-component of the net force is zero, we can find the normal force to be <m>F_N = -[(-736\unit{N})+(-0.\sig{96}{8}{N}) = 737\unit{N}</m>.
The x-component of the net force is <m>F_{\mathrm{net},x}=(-\sig{3.8}{8}{N})</m>.

\flushright
<assemblage>Return to: </assemblage> <xref provisional=""></xref>[rope-tension]{d-rope.net}
\hrule\hrule
\end{figure}
%
You should note that since the tension on the object is pulling up, helping the normal force, this allows the normal force (what a scale would read) to be a little smaller.
You should also note that since the tension on \studentB\ is pulling down, counter-acting the normal force, this requires the normal force (what a scale would read) to be a little larger.

</subsection><subsection><title></title>Pulleys}

While the flexibility of ropes makes them inconvenient for pushing, their flexibility makes them <em></em>very useful} for changing the direction of the pull.  The mechanism for changing the direction is the pulley.  Furthermore, by allowing us to change the direction of the pull, we are also able to double, triple, or further improve the strength of the pull.  The term for this is <q>the mechanical advantage</q> of a pulley-system.

First we will consider three simple cases of redirecting the force.  In each of these cases, I will <xref ref="" text="title"></xref>[s-effective2]{assume} that the pulley and rope have no mass and that there is no friction in the turning of the pulley (assume it is trivially easy to spin).  If we do not make this assumption, then the problem gets significantly more complicated.<todo></todo>add a reference to the section (problem?) where this is considered.}{}

\begin{minipage}[c]{3.25in}
\begin{sample}
</p></li><li><p>\studentA<idx><h sortby="\studentA">\studentA</h></idx> decides to hold a box that weighs <m>20\unit N</m> using a pulley-system.  What is the tension in the rope?

Since the mass is in equilibrium, the net force is zero and the tension must balance the weight.  This tells us that the tension in the rope is <m>20 \unit N</m>.

If the pulley were difficult to turn (had friction) that stickiness could help support the mass and the tension on \studentA's side might be less than <m>20\unit N</m>; but since we assumed the pulley to be frictionless, \studentA\ must provide the full <m>20\unit N</m> of tension to the rope.
\end{sample}
\end{minipage}
\hfill
\begin{minipage}{1in}
\begin{picture}(100,120)(0,7)
\put(31,105){\oval(36,36)[t]}
\put(31,105){\circle{33}}
\put(31,106){\line(0,1){29}}
\put(49,105){\line(0,-1){62}}
\put(13,105){\line(0,-1){70}}
\put(-30,7){\line(1,0){100}} % floor
\put(0,135){\line(1,0){62}} % ceiling
\multiput(5,135)(10,0){6}{\line(1,1){5}} % immovable
%
\drawbox{-26}{8}{20}{50} %\studentA
\drawbox{-6}{32}{18}{5} %\studentA's arms
\put(-26,60){\scriptsize \studentA}
%
%\drawbox{5}{19}{16}{16}
%\put(6,25){\small <m>m_1</m>}
%
\drawbox{41}{19}{16}{24}
\put(42,25){\small <m>m</m>}
%
%\put(49,-1){\vector(0,1){20}}
%\put(49,-1){\vector(0,-1){20}}
%\put(51,-1){\tiny <m>12\unit m</m>}
\end{picture}
\end{minipage}
%

\noindent
The interesting aspect is that \studentA\ must pull <em></em>down} in order to produce the <em></em>upward} tension on the box.  This means that both \studentA\ and the mass are pulling down.  Since the rope is draped over the pulley, the pulley feels <m>40\unit N</m> downwards, <m>20\unit N</m> from the tension supporting the mass and <m>20\unit N</m> from \studentA\ who is creating the tension that supports the mass.  This means that the second rope that is connecting the pulley to the ceiling must be supporting the full <m>40\unit N</m> in order to keep the pulley in equilibrium.



</subsection><subsection><title></title>Interesting Complications}

</subsubsection><subsubsection><title></title>What is the net force on the rope itself?}
The answer to this depends on how complicated you want the answer to be (recall the discussion about effective theories in <xref ref="" />s-effective2}).  Some reasonable answers are:
<ul>
</p></li><li><p> If the rope (and the attachments) are static, then the net force on the rope must be zero even while it maintains the tension.  It is also possible that the rope is accelerating, in which case the net force on the rope while it transfers the forces between the objects at each end is whatever is necessary to produce the acceleration <m>\vec F_\mathrm{net} = m_\mathrm{rope} \vec a_\mathrm{rope}</m>.
</p></li><li><p> A different answer is to assume that the mass of the rope is small enough that whether it is in equilibrium or accelerating, it does not require a net force and it merely passes its tension through to the object at the other end.
</ul>

</subsubsection><subsubsection><title></title>Multiple Masses}\label{sss-multiple.mass}<aside><title>Referenced by</title> <p><xref provisional="" /></p></aside>{ss-tension.support}

Now that we have a few examples of tension under our belts, we can consider some more interesting examples.

<xref ref="" />ex-multiweight.tension} considers the case of hanging multiple masses, which extends the ideas of <xref ref="" />ss-tension.support}.
%
\begin{example}[hbpt]
\fcolorbox{black}{yellow!10}{\begin{minipage}{4.925in}
\caption{\label{ex-multiweight.tension} How many weights?}
While preparing to hang some ornament on a tree, you chain them from a hook on the wall.  You hang ornament 1 from ornament 2 from ornament 3.  What is the tension in each subsequent string?

\color{blue}
The first thing we should do is notice what information is given to us and make sure that everything is in consistent units.  I will convert everything to <xref ref="" text="title"></xref>[ss-convertunits]{SI units}.

\color{black}
<assemblage>Return to: </assemblage> <xref provisional="" />{sss-multiple.mass}
\end{minipage}}
\end{example}
%
<xref ref="" />ex-multidrag.tension} considers the case of dragging multiple masses, which extends the ideas of <xref ref="" />ss-tension.drag}.
%
\begin{example}[hbpt]
\fcolorbox{black}{yellow!10}{\begin{minipage}{4.925in}
\caption{\label{ex-multidrag.tension} Caravan}
While pulling a sled on which your son sits, your son's sled is tied to a sled on which your dog sits.  Your dog's sled is then connected to a sled with provisions for the day.  What is the tension in each subsequent string?

\color{blue}
The first thing we should do is notice what information is given to us and make sure that everything is in consistent units.  I will convert everything to <xref ref="" text="title"></xref>[ss-convertunits]{SI units}.

\color{black}
<assemblage>Return to: </assemblage> <xref provisional="" />{sss-multiple.mass}
\end{minipage}}
\end{example}
%
You should note that these examples are essentially expressing the same idea in two different contexts.

</subsubsection><subsubsection><title></title>Atwood's Machine}\label{sss-Atwood}

The<todo></todo>imported a homework problem from Giordano.  Need to modify it to fit my purposes.}{} two crates in the figure (p. 114) hang over a pulley (in what is called an <q>Atwood's machine</q>).  I will select <m>m_1=35\unit{kg}</m> (because it looks smaller) and <m>m_2=85\unit{kg}</m> (because it looks bigger).  We will assume that the pulley is massless and frictionless (so that the tension is the same throughout the rope).  Find the acceleration and the time it takes <m>m_2</m> to accelerate down for the <m>12\unit m</m> to the floor.

\begin{minipage}{1in}
\begin{picture}(100,150)(0,-50)
\put(31,80){\oval(36,36)[t]}
\put(31,80){\circle{33}}
\put(31,81){\line(0,1){29}}
\put(49,80){\line(0,-1){62}}
\put(13,80){\line(0,-1){70}}
%
\put(5,-6){\line(0,1){16}}
\put(5,-6){\line(1,0){16}}
\put(21,10){\line(0,-1){16}}
\put(21,10){\line(-1,0){16}}
\put(6,0){\small <m>m_1</m>}
%
\put(41,-6){\line(0,1){24}}
\put(41,-6){\line(1,0){16}}
\put(57,18){\line(0,-1){24}}
\put(57,18){\line(-1,0){16}}
\put(42,0){\small <m>m_2</m>}
%
\put(49,-26){\vector(0,1){20}}
\put(49,-26){\vector(0,-1){20}}
\put(51,-26){\tiny <m>12\unit m</m>}
\end{picture}
\end{minipage}
\hfill
\begin{minipage}{4.5in}
The easy way to do this is to say that <m>m_1</m> pulls down on the left with <m>F_{g1} = (35\unit{kg})(9.81\unitfrac{m}{s^2})=\sig{34}{3.4}{N}</m> and <m>m_2</m> pulls down on the right with <m>F_{g2}=(85\unit{kg})(9.81\unitfrac{m}{s^2})=\sig{83}{3.5}{N}</m> for a difference of <m>F_{net} = \sig{49}{0}{N}</m> down to the right.  Since this has to move both <m>m_1</m> and <m>m_2</m>, the acceleration is
<me> a = \frac{F_{\rm net}}{m_1+m_2} = \frac{\sig{49}{0}{N}}{(35\unit{kg})+(85\unit{kg})} = \frac{\sig{49}{0}{N}}{\sig{120}{}{kg}} = \sigfrac{4.0}{87}{m}{s^2} </me>
This acceleration then causes <m>m_2</m> to drop and the time it takes is found from the equation that include distance and time, \\
<me>y_f \ = \  y_i + v_i \, t + \frac{1}{2} a \,  t^2 </me>
<me> (0\unit m) \ = \ (12\unit m) + (0\unitfrac ms) \, t + \frac{1}{2} (-\sigfrac{4.0}{9}{m}{s^2}) \,  t^2 </me>
\end{minipage}
which we can solve for time:
<me> t \ = \ \sqrt{ \frac{-(12\unit m)}{\frac{1}{2} (-\sigfrac{4.0}{9}{m}{s^2})} } \ = \  \sqrt{ \sig{5.8}{7}{s^2}} \ = \ \sig{2.4}{2}{s} </me>

\footnoterule
\small
However, this does not show what the tension is, and many students make a mistake with the tension.  So, I will also answer the question about the tension. We can draw three free-body diagrams. The equation for <m>m_1</m> is as follows, where I am putting the sign in by
<!-- -->\newpar

\begin{minipage}{4.5in}
hand to indicate the direction: \hfill
<m>\displaystyle (-F_{g1}) + (+F_T) = m_1 (+a) </m> \\
The equation for <m>m_2</m> is as follows: \hfill
<m>\displaystyle (-F_{g2}) + (+F_T) = m_2 (-a) </m> \\
Since we know the weights and the masses, these two equations and two unknowns can be written as
<md>
(-\sig{34}{3}{N}) + (+F_T) & = & (35\unit{kg}) (+a) \\
(-\sig{83}{3}{N}) + (+F_T) & = & (85\unit{kg}) (-a)
</md>
There are many ways to solve two equations and two unknowns.
If we subtract the second equation from the first, then we get the equation on the left.
But, if we solve the first equation for <m>a</m> and plug it into the second, then we get the equation on the right
\end{minipage}
\hfill
\begin{minipage}{1in}
\begin{picture}(100,150)(0,-50)
%\put(31,80){\oval(36,36)[t]}
\put(31,80){\circle{33}}
\put(31,81){\vector(0,1){50}}
\put(47.5,80){\vector(0,-1){30}}
\put(14.5,80){\vector(0,-1){30}}
\put(50,55){\tiny <m>F_T</m>}
\put(15,55){\tiny <m>F_T</m>}
%
\put(5,-6){\line(0,1){16}}
\put(5,-6){\line(1,0){16}}
\put(21,10){\line(0,-1){16}}
\put(21,10){\line(-1,0){16}}
\put(13,4){\vector(0,1){30}}
\put(13,0){\vector(0,-1){20}}
\put(14,20){\tiny <m>F_T</m>}
\put(14,-15){\tiny <m>F_{g1}</m>}
%
\put(49,8){\vector(0,1){30}}
\put(49,4){\vector(0,-1){40}}
\put(41,-6){\line(0,1){24}}
\put(41,-6){\line(1,0){16}}
\put(57,18){\line(0,-1){24}}
\put(57,18){\line(-1,0){16}}
\put(51,30){\tiny <m>F_T</m>}
\put(51,-15){\tiny <m>F_{g1}</m>}
\end{picture}
\end{minipage}

<me> \begin{array}{ccc}
\deq
(-\sig{34}{3}{N}) - (-\sig{83}{3}{N}) \ = \ \left[ (35\unit{kg}) + (85\unit{kg}) \right] (a) &&
\deq
(-\sig{83}{3}{N}) + (F_T) \ = \  - (85\unit{kg}) \left[ \frac{(-\sig{34}{3}{N}) + (F_T)}{(35\unit{kg})} \right] \\
\deq
a \ = \ \frac{\sig{49}{0}{N}}{(35\unit{kg})+(85\unit{kg})} = \sigfrac{4.0}{87}{m}{s^2} &&
\deq
F_T \ = \ \frac{-(35\unit{kg})(-\sig{83}{3}{N})-(85\unit{kg})(-\sig{34}{3}{N})}{[(35\unit{kg})+(85\unit{kg})]} \ = \ \sig{48}{6}{N}
\end{array} </me>
The acceleration is as above.  The tension is not enough to support <m>m_2</m> (so it falls) and more than enough to lift <m>m_1</m> (so it rises).
You should note that
<m>\left[(\sig{48}{6}{N}-\sig{34}{3}{N})/(35\unit{kg})=\sigfrac{4.0}{9}{m}{s^2}\right]</m>
\hfill and \hfill
<m>\left[(\sig{83}{3}{N}-\sig{48}{6}{N})/(85\unit{kg})=\sigfrac{4.0}{9}{m}{s^2}\right]</m>.

\normalsize

</subsubsection><subsubsection><title></title>Surface Tension}

As a <p xml:id=""></p>d-surf.tension}{final note}, <xref ref="" text="title"></xref>[s-surface.tension]{surface tension} is something else entirely.  See <xref ref="" />sss-tea} for a comment on the contribution to hot versus cold spoon noises.

</section><section><title></title>Frictional Force}\label{s-Ff}<aside><title>Referenced by</title> <p></p></aside>{\mmr{<xref ref="" text="type-global" />A-chair2}}, \mmr{<xref ref="" text="type-global" />A-chair6}}, \mmr{<xref ref="" text="type-global" />A-chair7}}, \mmr{<xref ref="" />A-fly.balls}}}

%
\begin{reallife}[bthp]
\hspace{-.2in}
\fcolorbox{black}{green!10}{\begin{minipage}{5.29in} \center
\caption{\label{irl-poolfriction}<idx></idx>{Pool!Real Life} Rolling pool balls and friction.}
\begin{minipage}{4.925in}
\studentD<idx><h sortby="\studentD">\studentD</h></idx> is relaxing with the local physics club, playing pool.  \HeD\ hits the cue ball and counts the number of walls \heD\ can hit in one shot.
\end{minipage}
\begin{realtable}
\dna{Hit the cue-ball off of a bumper in the manner intended for
\protect{<url href=""></url>{http://c.ymcdn.com/sites/bca-pool.com/resource/resmgr/imported/BCAEquipmentSpecifications_2008.pdf}{testing cushions}}.}
    {Compare the strength of the hit to the distance travelled}
    {How much is the total distance affected by the number of bumpers hit? \\
     Does it matter if you shoot along the length of the table versus the width of the table?  \\
     Why does friction slow the ball down instead of just make it turn <m>v=\omega r</m> (no slip)}
\end{realtable}
\begin{minipage}{4.925in}
Billiard tables have a lot of interesting physics, which can help us see a wide variety of physics, for example:
<xref ref="" text="title"></xref>[irl-poolnormal]{normal force}, <xref ref="" text="title"></xref>[irl-poolelastic]{elastic versus inelastic collisions}, <xref ref="" text="title"></xref>[irl-poolrotmot]{rotational motion}, and <xref ref="" text="title"></xref>[irl-poolangmom]{angular momentum}.
\end{minipage}

%\flushright
%<assemblage>Return to: </assemblage> <xref provisional=""></xref>[pool]{d-bank-shot}
\end{minipage}}
\end{reallife}
%

</section><section><title></title>Spring Force}\label{s-springs}<aside><title>Referenced by</title> <p></p></aside>{\mmr{<xref ref=""></xref>{d-fma}{<m>F=ma</m>}}, \mmr{<xref ref=""></xref>{d-usesofFma}{uses of <m>F=ma</m>}}, \mmr{<xref ref="" />ss-scales}}}

</section><section><title></title>Applied Force}

The term <q>an applied force</q> is used to describe any force applied by any object when there isn't really a formula to find it.  So this is kind of a <q>any other force</q> category.  I will use this type of force to describe forces exerted by people.  We have seen some examples where a person throws an object.  We can now revisit those examples and consider the force exerted (applied) by the person who threw the object.
\begin{sample}
</p></li><li><p>\label{se-throw-up} \studentC<idx><h sortby="\studentC">\studentC</h></idx> recalls that one time \heC\ got bored one day in physics class (what?!?) and tossed a baseball (<m>m_b = 0.145\unit{kg}</m>) at the ceiling<ellipsis /> a little too hard <ellipsis /> as recounted in <xref ref="" />ex-ceiling}.  Recall that <xref ref="" text="type-global" />se-ceiling} found the normal force by the ceiling on the ball.  Please now find the force \studentC\ applied while throwing and catching the ball assuming that the throw took <m>0.200\unit{s}</m> to gain the speed of <m>5.00\unitfrac ms</m> and the catch took <m>0.250\unit s</m> to slow the ball from <m>4.73\unitfrac ms</m> to rest.

There are five stages to the motion: (a) throwing, (b) falling up, (c) hitting the ceiling, (d) falling down, and (e) catching show the forces involved. \\
\fbox{\begin{minipage}[b]{55pt}
\begin{picture}(50,100)(0,0)
\put(25,25){\circle{10}}
\put(25,26){\vector(0,1){25}}
\put(25,24){\vector(0,-1){15}}
\put(28,35){<m>F_\mathrm{throw}</m>}
\put(28,10){<m>F_g</m>}
\end{picture}
\centering{(a) throwing}
\end{minipage}}
\hfill
\color{lightgray}
\fbox{\begin{minipage}[b]{55pt}
\begin{picture}(50,100)(0,0)
\put(25,50){\circle{10}}
\put(25,50){\vector(0,-1){15}}
\put(28,35){<m>F_g</m>}
\end{picture}
\centering{(b) falling up}
\end{minipage}}
\hfill
\fbox{\begin{minipage}[b]{55pt}
\begin{picture}(50,100)(0,0)
\put(25,95){\circle{10}}
\put(26,95){\vector(0,-1){25}}
\put(24,95){\vector(0,-1){15}}
\put(28,75){<m>F_N</m>}
\put(10,75){<m>F_g</m>}
\end{picture}
\centering{(c) \\ hitting}
\end{minipage}}
\hfill
\fbox{\begin{minipage}[b]{55pt}
\begin{picture}(50,100)(0,0)
\put(25,50){\circle{10}}
\put(25,50){\vector(0,-1){15}}
\put(28,35){<m>F_g</m>}
\end{picture}
\centering{(d) falling down}
\end{minipage}}
\hfill
\color{rgb:red,0;green,2;blue,1}
\fbox{\begin{minipage}[b]{55pt}
\begin{picture}(50,100)(0,0)
\put(25,25){\circle{10}}
\put(25,26){\vector(0,1){25}}
\put(25,24){\vector(0,-1){15}}
\put(28,35){<m>F_\mathrm{catch}</m>}
\put(28,10){<m>F_g</m>}
\end{picture}
\centering{(e) catching}
\end{minipage}}
\\
In this particular problem, we are only concerned with steps (a) and (e) because that's where \studentC\ throws and catches the ball. In each case, we need the acceleration: \\
\begin{minipage}[b]{150pt}
<md>
\vec a_\mathrm{throw} & = & \frac{(+5.00\unitfrac ms \jhat)-(0\unitfrac ms \jhat)}{0.200\unit s} \\
& = & +\sigfrac{25.0}{0}{m}{s^2} \jhat
</md>
\end{minipage}
\hfill
\begin{minipage}[b]{150pt}
<md>
\vec a_\mathrm{catch} & = & \frac{(0\unitfrac ms \jhat)-(-4.73\unitfrac ms \jhat)}{0.250\unit s} \\
& = & +\sigfrac{18.9}{2}{m}{s^2} \jhat
</md>
\end{minipage}

During each step, we have the actual acceleration, which tells us about the net force.  We will also need to know the weight of the baseball <m>F_g=\sig{1.42}{2}{N}</m>, because gravity is still acting during the collision.  Let's consider the throwing part first.
<md>
\vec F_N + \vec F_g & = &  \vec F_\mathrm{net} \ = \ m \vec a \\
\vec F_A  & = &  m \vec a - \vec F_g \\
\vec F_A  & = &  \left[ (0.145\unit{kg})(+\sigfrac{25.0}{0}{m}{s^2}\jhat) \right] - \left[  - \sig{1.42}{2}{N} \jhat \right] \\
\vec F_A  & = &  \left[ +\sig{3.62}{5}{N} \jhat \right] - \left[  - \sig{1.42}{2}{N} \jhat \right] \ = \ +\sig{5.04}{7}{N} \jhat
</md>
You can see that the upward applied force <m>(\sig{5.04}{7}{N})</m> has to be large enough so that when it is combined with the downward gravitational force <m>(\sig{1.42}{2}{N})</m> they can together result in the necessary (but smaller) upward net force <m>(\sig{3.62}{5}{N})</m> to get it going upwards.

For the catching part, the ball is moving downwards and needs to be stopped, so the catching applied force must be upwards.
<md>
\vec F_A + \vec F_g & = &  \vec F_\mathrm{net} \ = \ m \vec a \\
\vec F_A  & = &  m \vec a - \vec F_g \\
\vec F_A  & = &  \left[ (0.145\unit{kg})(+\sigfrac{18.9}{2}{m}{s^2}\jhat) \right] - \left[  - \sig{1.42}{2}{N} \jhat \right] \\
\vec F_A  & = &  \left[ +\sig{2.74}{3}{N} \jhat \right] - \left[  - \sig{1.42}{2}{N} \jhat \right] \ = \ +\sig{4.16}{5}{N} \jhat
</md>
You can see that the upward applied force <m>(\sig{4.16}{5}{N})</m> has to be large enough so that when it is combined with the downward gravitational force <m>(\sig{1.42}{2}{N})</m> they can together result in the necessary upward net force <m>(\sig{2.74}{3}{N})</m> to stop it from continuing downwards.
\end{sample}

</section><section><title></title>Putting it Together, <m>F_\mathrm{net}</m>}\label{s-Fnet}

</subsection><subsection><title></title>Translational Equilibrium}

blah blah blah
\phantomsection\label{ss-transeq} Translational equilibrium: <m>F_\mathrm{net} = m \zero{a}{0}</m>.  blah blah blah

</subsection><subsection><title></title>Static Equilibrium}

</subsection><subsection><title></title>Dynamic Equilibrium}


</section><section><title></title>Summary and Homework}

</subsection><subsection><title></title>Summary of Concepts and Equations}<!-- -->\new{v2.3}{Created this section}

<ellipsis />

</subsection><subsection><title></title>Conceptual Questions}<!-- -->\new{v2.3}{Added two conceptual problems.}
%\vspace{-24pt}
<ol>
</p></li><li><p>\label{c-weightmass} Estimate, preferably without using the internet, the mass of the following: (a) a four-door sedan, (b) dishwasher, (c) a pair of glasses, (d) a cell phone.  You should be able to estimate to within one significant digit.
</p></li><li><p>\label{c-massweight} List at least one object, preferably without using the internet, that has the following mass: (a) <m>2500\unit{kg}</m> (b) <m>41\unit{kg}</m>, (c) <m>3\unit{kg}</m>, (d) <m>50\unit{g}</m>.
</ol>
</subsection><subsection><title></title>Problems}<!-- -->\new{v2.3}{Created section.}<todo></todo>Add more problems.}
%\vspace{-24pt}
<ol>
 </p></li><li><p><ellipsis />
</ol>


</chapter><chapter><title></title>Energy and the Transfer of Energy}

<p xml:id=""></p>d-energynoun}{Energy is a noun}<idx></idx>{Energy!noun}; objects can <em></em>have} energy.  <p xml:id=""></p>d-workverb}{Work is a verb}<idx></idx>{Work!verb}<aside><title>Referenced by</title> <p>Discussion of <xref provisional=""></xref></p></aside>[heat as a verb]{d-heatverb}; doing work is the process of <em></em>exchanging} energy.

</section><section><title></title>Objects Can Have Energy}

</section><section><title></title>A Force Can Transfer Energy} \label{s-work}<aside><title>Referenced by</title> <p>Discussion of <xref provisional=""></xref></p></aside>[the direction of forces]{d-pushvector}

</section><section><title></title>Dissipating Energy} \label{s-Wfr}

pool balls on cushion/bumper

</section><section><title></title>Conserving Energy} \label{s-PE}

%
\begin{reallife}[bthp]
\hspace{-.2in}
\fcolorbox{black}{green!10}{\begin{minipage}{5.29in} \center
\caption{\label{irl-poolelastic}<idx></idx>{Pool!Real Life} 1-D elastic collisions of pool balls.  inelastic collisions off the bumper.}
\begin{minipage}{4.925in}
\studentD<idx><h sortby="\studentD">\studentD</h></idx> is relaxing with the local physics club, playing pool.  \HeD\ hits the cue ball and counts the number of walls \heD\ can hit in one shot.
\end{minipage}
\begin{realtable}
\dna{collide balls.}
    {where does it hit}
    {<m>90^\circ</m> output}
\end{realtable}
\begin{minipage}{4.925in}
Billiard tables have a lot of interesting physics, which can help us see a wide variety of physics, for example:
<xref ref="" text="title"></xref>[irl-poolnormal]{normal force}, <xref ref="" text="title"></xref>[irl-poolelastic]{elastic versus inelastic collisions}, <xref ref="" text="title"></xref>[irl-poolrotmot]{rotational motion}, and <xref ref="" text="title"></xref>[irl-poolangmom]{angular momentum}.
\end{minipage}

%\flushright
%<assemblage>Return to: </assemblage> <xref provisional=""></xref>[pool]{d-bank-shot}
\end{minipage}}
\end{reallife}
%

</subsection><subsection><title></title>Gravitational Potential Energy}\label{ss-PEg}<aside><title>Referenced by</title> <p><xref provisional="" /></p></aside>{s-PEG}
See also <xref ref="" text="type-global" />s-PEG}
</subsection><subsection><title></title>Spring Potential Energy}\label{ss-PEs}
</subsection><subsection><title></title>Conservative Forces in General}

\part{Interesting Uses of Motion, Force, and Energy}

</chapter><chapter><title></title>Momentum: A Better Way to Describe Force}\label{c-momentum}<aside><title>Referenced by</title> <p></p></aside>{\mmr{<xref ref=""></xref>{d-objectinmotion}{objects in motion}}, \mmr{<xref ref="" />sss-inertia}}, \mmr{<xref ref="" />ss-NIII}}, \mmr{<xref ref="" text="type-global" />A-chair6}}}

Useful to include?
<url href=""></url>{https://www.wired.com/2017/06/physics-bullets-versus-wonder-womans-bracelets/}{The Physics of Bullets Vs. Wonder Woman's Bracelets}

</section><section><title></title>Revising Newton's First and Second Laws}

</subsection><subsection><title></title>Inertia and Momentum}\label{ss-inertia}<aside><title>Referenced by</title> <p><xref provisional="" /></p></aside>{sss-inertia}
Recall <xref ref="" />sss-inertia}.

</section><section><title></title>Revising Newton's Third Law: Conservation of Momentum}\label{s-conservemom}<aside><title>Referenced by</title> <p><xref provisional="" /></p></aside>{ss-NIII}

</section><section><title></title>Two-Dimensional Collisions}\label{s-2Dcollisions}<aside><title>Referenced by</title> <p><xref provisional="" /></p></aside>{sss-vectorequations}

pool balls?  What about rolling?
%
\begin{reallife}[bthp]
\hspace{-.2in}
\fcolorbox{black}{green!10}{\begin{minipage}{5.29in} \center
\caption{\label{irl-pool2Dcollision}<idx></idx>{Pool!Real Life} 2-D collisions of pool balls.}
\begin{minipage}{4.925in}
\studentD<idx><h sortby="\studentD">\studentD</h></idx> is relaxing with the local physics club, playing pool.  \HeD\ hits the cue ball and counts the number of walls \heD\ can hit in one shot.
\end{minipage}
\begin{realtable}
\dna{collide balls.}
    {where does it hit}
    {<m>90^\circ</m> output}
\end{realtable}
\begin{minipage}{4.925in}
Billiard tables have a lot of interesting physics, which can help us see a wide variety of physics, for example:
<xref ref="" text="title"></xref>[irl-poolnormal]{normal force}, <xref ref="" text="title"></xref>[irl-poolelastic]{elastic versus inelastic collisions}, <xref ref="" text="title"></xref>[irl-poolrotmot]{rotational motion}, and <xref ref="" text="title"></xref>[irl-poolangmom]{angular momentum}.
\end{minipage}

%\flushright
%<assemblage>Return to: </assemblage> <xref provisional=""></xref>[pool]{d-bank-shot}
\end{minipage}}
\end{reallife}
%


</chapter><chapter><title></title>Rotational Motion}

</section><section><title></title>The Equations of Rotational Motion}

%
\begin{reallife}[bthp]
\hspace{-.2in}
\fcolorbox{black}{green!10}{\begin{minipage}{5.29in} \center
\caption{\label{irl-poolrotmot}<idx></idx>{Pool!Real Life} Rolling pool balls.}
\begin{minipage}{4.925in}
\studentD<idx><h sortby="\studentD">\studentD</h></idx> is relaxing with the local physics club, playing pool.  \HeD\ hits the cue ball and counts the number of walls \heD\ can hit in one shot.
\end{minipage}
\begin{realtable}
\dna{Roll a striped ball along the table.}
    {Use the stripe to notice the rate of rotation}
    {How does the rotation compare to the translation?}
\dna{Roll a striped ball along the table.}
    {Notice the distance the ball travels}
    {Why does friction slow the ball down instead of just make it turn <m>v=\omega r</m> (no slip)}
\end{realtable}
\begin{minipage}{4.925in}
Billiard tables have a lot of interesting physics, which can help us see a wide variety of physics, for example:
<xref ref="" text="title"></xref>[irl-poolnormal]{normal force}, <xref ref="" text="title"></xref>[irl-poolelastic]{elastic versus inelastic collisions}, <xref ref="" text="title"></xref>[irl-poolrotmot]{rotational motion}, and <xref ref="" text="title"></xref>[irl-poolangmom]{angular momentum}.
\end{minipage}

%\flushright
%<assemblage>Return to: </assemblage> <xref provisional=""></xref>[pool]{d-bank-shot}
\end{minipage}}
\end{reallife}
%

</section><section><title></title>Angular Momentum}

%
\begin{reallife}[bthp]
\hspace{-.2in}
\fcolorbox{black}{green!10}{\begin{minipage}{5.29in} \center
\caption{\label{irl-poolangmom}<idx></idx>{Pool!Real Life} Rolling pool balls.}
\begin{minipage}{4.925in}
\studentD<idx><h sortby="\studentD">\studentD</h></idx> is relaxing with the local physics club, playing pool.  \HeD\ hits the cue ball and counts the number of walls \heD\ can hit in one shot.
\end{minipage}
\begin{realtable}
\dna{Roll a striped ball along the table.}
    {Use the stripe to notice the rate of rotation}
    {How does the rotation compare to the translation?}
\end{realtable}
\begin{minipage}{4.925in}
Billiard tables have a lot of interesting physics, which can help us see a wide variety of physics, for example:
<xref ref="" text="title"></xref>[irl-poolnormal]{normal force}, <xref ref="" text="title"></xref>[irl-poolelastic]{elastic versus inelastic collisions}, <xref ref="" text="title"></xref>[irl-poolrotmot]{rotational motion}, and <xref ref="" text="title"></xref>[irl-poolangmom]{angular momentum}.
\end{minipage}

%\flushright
%<assemblage>Return to: </assemblage> <xref provisional=""></xref>[pool]{d-bank-shot}
\end{minipage}}
\end{reallife}
%


</section><section><title></title>Non-inertial Rotational Reference Frames} \label{s-noninertial}<aside><title>Referenced by</title> <p></p></aside>{\mmr{<xref ref="" />ss-noninertial}}, \mmr{<xref ref=""></xref>{d-NewtonInertial}{non-inertial reference frames}}, \mmr{<xref ref="" />ss-NI}}}
<idx></idx>{Reference Frames!Inertial}
<idx></idx>{Reference Frames!Non-inertial}

Because the Earth <p xml:id=""></p>d-noninertial}{rotates}<aside><title>Referenced by</title> <p><xref provisional="" /></p></aside>{ss-NII}, we are actually in a non-inertial reference frame.  In fact, we can prove that the Earth rotates by observing the effects, such as the <xref ref=""></xref>{d-coriolis}{Coriolis effect}, that in our non-inertial frame seem to require unexplainable forces but which, in a non-rotating frame, follow the expected laws of physics.

</subsection><subsection><title></title>The Coriolis Effect}\label{ss-coriolis}<aside><title>Referenced by</title> <p></p></aside>{\mmr{<xref ref=""></xref>{d-NewtonInertial}{non-inertial reference frames}}, \mmr{<xref ref=""></xref>{d-noninertial}{Non-inertial Rotational Reference Frames}}}

<p xml:id=""></p>d-coriolis}{weather, etc}
<!-- -->\newpar

In her podcast<!-- -->\new{v2.0}{<em></em>Spacepod}}, <em></em>Spacepod}<fn xml:id=""></fn>{Nugent, Carrie (Producer, Host). <em></em>Spacepod} [Audio podcast], episode 89 (19 May, 2017).  Retrieved from <xref ref="" text="title"></xref>{http://spacepod.libsyn.com/}{T4LTFdOxHD5WWzdD}{99}{\nolinkurl{http://spacepod.libsyn.com/}}
on 9 Apr. 2017.} Dr. Carrie Nugent interviews Dr. Andy Thompson about <q>underwater flying objects</q> that investigate the ocean.  He notes that ocean waters, because they are such a large-scale system, can see the effect of the rotation of the Earth.

</subsection><subsection><title></title>The Foucault Pendulum}\label{ss-Foucault}

See <url href=""></url>{https://www.youtube.com/watch?v=sWDi-Xk3rgw}{youtube video} by <url href=""></url>{http://sixtysymbols.com/}{Sixty Symbols}.<!-- -->\new{v2.0}{Foucault video}




</chapter><chapter><title></title>Circular Motion and Centripetal Force}

</section><section><title></title>Circular Motion}
</section><section><title></title>Centripetal Force}\label{s-centripetal}<aside><title>Referenced by</title> <p>Discussion of <xref provisional=""></xref></p></aside>[<m>F=ma</m>]{d-fma}




</chapter><chapter><title></title>Torque and the <m>F=ma</m> of Rotations}\label{c-torque}<aside><title>Referenced by</title> <p><xref provisional="" /></p></aside>{a-NIIIaction}<!-- -->\new{v2.3}{Added an example that is computable here, but helps introduce normal force in \protect{<xref ref="" />s-FN}}.}

</section><section><title></title>Leverage}\label{s-leverarm}<aside><title>Referenced by</title> <p><xref provisional="" /></p></aside>{ss-scales}

</section><section><title></title>Putting it all together, <m>\tau_\mathrm{net}</m>}

</subsection><subsection><title></title>Rotational Equilibrium}

blah blah blah
\phantomsection\label{ss-roteq} Rotational equilibrium: <m>\tau_\mathrm{net} = I \zero{\alpha}{0}</m>.  blah blah blah

</subsection><subsection><title></title>Static (Rotational) Equilibrium}

</subsection><subsection><title></title>Dynamic (Rotational) Equilibrium}

<!-- -->\new{v2.3}{Answered \protect{<xref ref="" />ex-ladder2}} and its related problems.}
\begin{example}[p]
\fcolorbox{black}{yellow!10}{\begin{minipage}{4.925in}\setlength{\parskip}{3pt}
\caption{\label{ex-ladder2} \studentC<idx><h sortby="\studentC">\studentC</h></idx> uses a ladder}
\begin{quote}
\studentC\ leans a <m>22.7\unit{kg}</m> ladder against a wall at an angle of <m>75.5^\circ</m>, consistent with \protect{<url href=""></url>{https://www.osha.gov/}{OSHA}} standard \protect{<url href=""></url>{https://www.osha.gov/pls/oshaweb/owadisp.show_document?p_table=standards&p_id=10839}{1926.1053(a)(1)(ii)}}.
The coefficient of friction between the ladder and the floor is <m>\mu_f=0.31</m>.
The coefficient of friction between the ladder and the wall is <m>\mu_w=0.19</m>.
Use the rotational and translational equilibrium to determine if the ladder slides.
\end{quote}

Since we are asked to distinguish between two cases that cannot both be true, we should assume one (the easier one to calculate is that the ladder does not slip) and then verify that the result is consistent with that assumption.

<em></em>What do we know?}
We know that the floor has a normal force <m>(F_{Nf})</m> upwards and a frictional force <m>(F_{ff})</m> to the left.
We know that the wall
\\[2pt]
\begin{minipage}{3.2in}
has a normal force <m>(F_{Nw})</m> to the right and a frictional force <m>(F_{fw})</m> up (keeping the ladder from sliding down).
We know the weight is
<m> F_g = mg = (22.7\unit{kg})(9.81\unitfrac{m}{s^2}) = \sig{222}{.69}{N} </m>
<em></em>What do we want to know?}  We want to know about the the magnitudes of both normal
\end{minipage}
\hfill
\begin{minipage}{100pt}
\begin{picture}(100,100)(-10,-5)
% Dimensions and offset: (width,height)(x offset,y offset)
% Insert picture commands (\line,\circle, etc...) here:
\put(0,0){\line(0,1){100}}
\put(0,0){\line(1,0){75}}
\put(20,0){\color{blue}\line(-1,4){20}}     % ladder
\put(10,40){\color{red}\vector(0,-1){30}}   % Fg
\put(20,1){\color{red}\vector(0,1){25}}     % FNf
\put(19,1){\color{red}\vector(-1,0){12}}
\put(1,80){\color{red}\vector(1,0){12}}
\put(1,81){\color{red}\vector(0,1){8}}
\end{picture}
\end{minipage}
%\hfill {}
\\[3pt]
forces and both frictional forces.
Can we easily deduce the magnitude of <m>F_{Nf}</m>? <xref ref="" text="type-global" />A-ladderNf}.

<em></em>How are these related?}  The forces acting on any body are related by static <xref ref="" text="title"></xref>[ss-transeq]{translational equilibrium}
<md>
x: \hspace{.5cm} 0 & = & \zero{F_{gx}}{0} + \zero{F_{Nfx}}{0} + F_{ffx} + F_{Nwx} + \zero{F_{fwx}}{0} \\
y: \hspace{.5cm} 0 & = & F_{gy} + F_{Nfy} + \zero{F_{ffy}}{0} + \zero{F_{Nwy}}{0} + F_{fwy}
</md>
and static <xref ref="" text="title"></xref>[ss-roteq]{rotational equilibrium}, assuming the pivot point is at the ground, and using the relationship <m>F_f=\mu F_N</m>, we find
<md>
0 & = & \tau_{g} + \zero{\tau_{Nf}}{0} + \zero{\tau_{ff}}{0} + \tau_{Nw} + \tau_{fw} \\
0 & = & \left[ F_g \frac{l}{2} \sin 14.5^\circ \right] + \left[ - F_{Nw} l \sin(75.5^\circ) \right] + \left[ - F_{fw} l \sin(14.5^\circ) \right] \\
F_{Nw} & = & \left[ F_g \frac{l}{2} \sin 14.5^\circ \right] / \left[  l \sin(75.5^\circ) + \mu_w l \sin(14.5^\circ) \right]
</md>
%<em></em>Free-Body Diagrams:}  Since the picture is so simple, we will not draw the free-body diagram.


{}\hfill {\footnotesize\autoref*{ex-ladder2} continued on next page<ellipsis />}
\end{minipage}}
\end{example}
\begin{example}[p]
\fcolorbox{black}{yellow!10}{\begin{minipage}{4.925in}\setlength{\parskip}{3pt}
{\footnotesize \autoref*{ex-ladder2} continued from previous page<ellipsis />}

<em></em>Concepts to Consider:}  First, the length of the ladder cancels from the expression; what matters is the angle at which it is propped.

Second, every force value will be linearly dependent on the mass of the ladder.  So once we solve this problem, we can easily scale the answers to any mass.

Third, the friction with the wall is, by far, the smallest effect and it might be interesting to approximate all of this with <m>\mu_w=0</m>.  You can check your calculation against <xref ref="" text="type-global" />A-nowall}.

<em></em>Solution to the example:}  When we worry about significant figures,
<md>
F_{Nw} & = & \frac{\left[ (\sig{222}{.7}{N})(\txtfrac{1}{2}) (0.\sig{250}{4}{}) \right]}{\left[  (0.\sig{968}{2}{}) + (0.19) (0.\sig{250}{4}{}) \right]}
\ = \ \frac{\left[ (\sig{27.8}{8}{N})\right]}{\left[  (0.\sig{968}{2}{}) + (0.0\sig{47}{6}{}) \right]} \\
F_{Nw} & = & \frac{\left[ (\sig{27.8}{8}{N})\right]}{\left[  (\sig{1.015}{7}{}) \right]}
\ = \ \sig{27.4}{4}{N} \\
F_{fw,\mathrm{max}} & = & (0.19)(\sig{27.4}{4}{N}) \ = \ \sig{5.2}{15}{N} \\
F_{Nf} & = & F_g - F_{fw} = (\sig{222}{.7}{N})-(\sig{5.2}{15}{N}) \ = \ \sig{217}{.5}{N} \\
F_{ff,\mathrm{max}} & = & (0.31)(\sig{217}{.5}{N}) \ = \ \sig{672}{.4}{N}
</md>
Since <m>F_{ff} >F_{Nw}</m>, the friction is sufficient to hold the ladder in place, as assumed.

%\begin{quote}
<em></em>Aside:} Since <m>F_{ff}</m> only needs to be <m>\sig{27.4}{4}{N}</m> to hold the ladder in place, it is possible for the ladder to not slide on a floor that only has
<m>\mu_\mathrm{min} = (\sig{27.4}{4}{N})/(\sig{217}{.5}{N}) = 0.\sig{126}{2}{}</m>; but that would not allow a person to climb the ladder.

<em></em>Homework:} Homework problem~<xref ref="" text="type-global" />h-ladderC} asks you to determine if the ladder slides when \studentC\ climbs to different locations on the ladder.
%\end{quote}
\flushright
<assemblage>Return to: </assemblage>{\mmr{<xref ref="" text="type-global" />se-ladderN}}, \mmr{<xref ref="" />ss-roteq}}}
\end{minipage}}
\end{example}

</section><section><title></title>Torsion}\label{s-torsion}<aside><title>Referenced by</title> <p><xref provisional="" /></p></aside>{s-FT}<!-- -->\new{v2.4}{Created this section}

</section><section><title></title>Summary and Homework}

</subsection><subsection><title></title>Summary of Concepts and Equations}<!-- -->\new{v2.3}{Created this section}

<ellipsis />

</subsection><subsection><title></title>Conceptual Questions}<todo></todo>Add conceptual problems.}
%\vspace{-24pt}
<ol>
</p></li><li><p><ellipsis />
</ol>
</subsection><subsection><title></title>Problems}<!-- -->\new{v2.3}{Added problems.}<todo></todo>Add more problems.}
%\vspace{-24pt}
<ol>
 </p></li><li><p>\label{h-ladderC} \studentC\ leans a <m>22.7\unit{kg}</m> ladder against a wall at an angle of <m>75.5^\circ</m>, consistent with \protect{<url href=""></url>{https://www.osha.gov/}{OSHA}} standard \protect{<url href=""></url>{https://www.osha.gov/pls/oshaweb/owadisp.show_document?p_table=standards&p_id=10839}{1926.1053(a)(1)(ii)}}.<!-- -->\new{v2.3}{Answered \protect{<xref ref="" text="type-global" />h-ladderC}} and its related problems.}
The coefficient of friction between the ladder and the floor is <m>\mu_f=0.31</m>.
The coefficient of friction between the ladder and the wall is <m>\mu_w=0.19</m>.
Use the rotational and translational equilibrium to determine if the ladder slides when \studentC\ (<m>\massC</m>) climbs to
<ol>
</p></li><li><p> the third-rung from the top of the ladder, so that he is <m>1.53\unit m</m> from the bottom of the ladder.
    (See <xref ref="" text="type-global" />A-nowallC} for that answers if <m>\mu_w = 0</m>.)
\begin{ForMe}
\color{blue} Answers:
<md>
F_{Nw} & = & \sig{163}{.9}{N} \\
F_{fw,\mathrm{max}} & = & (0.19)(\sig{163}{.9}{N}) \ = \ \sig{31}{.14}{N} \\
F_{Nf} & = & \sig{1074}{.4}{N} \\
F_{ff,\mathrm{max}} & = & \sig{333}{.0}{N} < \sig{163}{.9}{N}
</md>
<m>\mu_\mathrm{min} = 0.\sig{152}{56}{}</m>
\color{black}
\end{ForMe}
</p></li><li><p> the third-rung from the bottom of the ladder, so that he is <m>0.914\unit m</m> from the bottom of the ladder.
\begin{ForMe}
\color{blue}
Answers:
<md>
F_{Nw} & = & \sig{108}{.97}{N} \\
F_{fw,\mathrm{max}} & = & (0.19)(\sig{108}{.97}{N}) \ = \ \sig{20}{.70}{N} \\
F_{Nf} & = & \sig{1084}{.9}{N} \\
F_{ff,\mathrm{max}} & = & \sig{336}{.3}{N} < \sig{108}{.97}{N}
</md>
<m>\mu_\mathrm{min} = 0.\sig{100}{45}{}</m>

If <m>\mu_w = 0</m>.
<md>
F_{Nw} & = & \sig{114}{.3}{N} \\
F_{fw,\mathrm{max}} & = & 0 \unit N \\
F_{Nf} & = & \sig{1105}{.6}{N} \\
F_{ff,\mathrm{max}} & = & \sig{342}{.7}{N} < \sig{114}{.3}{N}
</md>
<m>\mu_\mathrm{min} = 0.\sig{103}{4}{}</m>
\color{black}
\end{ForMe}
</ol>
</ol>


</chapter><chapter><title></title>Energy of Rotating Objects}
</section><section><title></title>Rotational Kinetic Energy}
pool balls

</chapter><chapter><title></title>The Gravitational Force on a Large Scale}\label{c-gravity}<aside><title>Referenced by</title> <p></p></aside>{\mmr{<xref ref=""></xref>{d-accgrav}{freefall}}, \mmr{<xref ref=""></xref>{d-fundamental}{fundamental forces}}}

</section><section><title></title>Gravitational Force and Field}\label{s-Gfield}<aside><title>Referenced by</title> <p>Discussion of <xref provisional=""></xref></p></aside>[<m>F=ma</m>]{d-fma}<!-- -->\new{v2.3}{Added some placeholders}

The value of the acceleration due to gravity  varies according to the mass and size of any celestial body.<todo></todo>Reference a table of <m>g</m> on other planets and compute the weight of a space craft at each planet.}
This means that, as was seen in <xref ref="" text="type-global" />se-gworld}, your weight can change even when your mass remains the same.
\begin{sample}
</p></li><li><p>\label{se-gplanets} In conversation with a visiting alien, \studentX<idx><h sortby="\studentX">\studentX</h></idx>, you find that \studentX\ has been to the moon and several planets both within and outside of our solar system.  In addition to the Earth, \studentX\ has visited our moon, Mars, Pluto, and Planet X.  Using <xref ref="" />t-gplanets}, compute \studentX's weight are each location, assuming \hisX\ mass is \massX.
<ol>
</p></li><li><p>[Earth] <m>F_g = (\massX)\left[ \frac{ G M_E}{R_E^2} \right] = (\massX)(9.825\unitfrac{m}{s^2}) \ = \ \sig{933}{.4}{N}</m>
</p></li><li><p>[moon] <m>F_g = (\massX)\left[ \frac{ G M_m}{R_m^2} \right] = (\massX)(9.782\unitfrac{m}{s^2}) \ = \ \sig{929}{.3}{N}</m>
</p></li><li><p>[Mars] <m>F_g = (\massX)\left[ \frac{ G M_M}{R_M^2} \right] = (\massX)(9.763\unitfrac{m}{s^2}) \ = \ \sig{927}{.5}{N}</m>
</p></li><li><p>[Pluto] <m>F_g = (\massX)\left[ \frac{ G M_P}{R_P^2} \right] = (\massX)(9.763\unitfrac{m}{s^2}) \ = \ \sig{927}{.5}{N}</m>
</p></li><li><p>[Planet X] <m>F_g = (\massX)\left[ \frac{ G M_X}{R_X^2} \right] = (\massX)(9.763\unitfrac{m}{s^2}) \ = \ \sig{927}{.5}{N}</m>
</ol>
\end{sample}
%
\begin{table}[bhtp]
\hrule\hrule
\begin{center}
\caption[Properties of various celestial bodies]{\label{t-gplanets} Properties of various celestial bodies.
<assemblage>Return to: </assemblage> <xref provisional="" />{se-gplanets}
}
\begin{tabular}{lccr}
Planet & Mass (kg) & Mean Radius (m) & <m>g (\unitfrac{m}{s^2})</m> \\
\end{tabular}
\end{center}
\hrule\hrule
\end{table}
%


</subsection><subsection><title></title>Inertial Mass versus Gravitational Mass}\label{ss-equivmm}<aside><title>Referenced by</title> <p><xref provisional="" /></p></aside>{ss-weightmass}<!-- -->\new{v2.2}{Moved this here, might need to move it back.}

</section><section><title></title>Gravitational Potential Energy} \label{s-PEG}<aside><title>Referenced by</title> <p><xref provisional="" /></p></aside>{ss-PEg}

Recall <xref ref="" text="type-global" />ss-PEg}

</section><section><title></title>Making Connections}\label{s-Gconnection}<aside><title>Referenced by</title> <p><xref provisional="" /></p></aside>{s-Econnection}

<me> \begin{array}{ccccc}
& & \vec F = m \vec g & & \\
& \deq F = G \frac{m_1 m_2}{R^2} & \leftrightarrow & \deq g = G \frac{m}{R^2} & \\
\Delta \PE = -\vec F \cdot \Delta\vec x & \updownarrow & & \updownarrow & \mbox{\scriptsize [for later]} \\
& \deq \PE = G \frac{m_1 m_2}{R} & \leftrightarrow & \mbox{[for later]} & \\
& & \mbox{\scriptsize [for later]} & &
\end{array} </me>
(Look ahead to the parallel with the electrical interaction in <xref ref="" />s-Econnection}.)

</section><section><title></title>Orbits}


\part{Making Waves}

</chapter><chapter><title></title>Fluids}<!-- -->\new{v2.2}{Placeholder}
</section><section><title></title>Density}\label{s-density}<idx></idx>{Density}<aside><title>Referenced by</title> <p><xref provisional="" /></p></aside>{ss-weightmass}

</section><section><title></title>Surface Tension}\label{s-surface.tension}<aside><title>Referenced by</title> <p>Discussion of <xref provisional=""></xref></p></aside>{d-surf.tension}



</chapter><chapter><title></title>Oscillations}\label{c-SHM}
</section><section><title></title>Oscillating Springs}\label{c-SHMspring}<aside><title>Referenced by</title> <p>Discussion of <xref provisional=""></xref></p></aside>[<m>F=ma</m>]{d-fma}
</section><section><title></title>Oscillating Pendulums}\label{c-SHMpend}

</section><section><title></title>Other Examples of Oscillations}\label{s-SHMother}

On 13 April, 2017,<!-- -->\new{v2.3}{New source of info}
<url href=""></url>{http://www.cbc.ca/podcasting}{CBC Broadcasting} published a
<url href=""></url>{http://www.cbc.ca/podcasting/includes/quirks.xml}{<em></em>Quirks and Quarks}} episode discussing how we can find
<url href=""></url>{https://podcast-a.akamaihd.net/mp3/podcasts/quirks_20170415_12100.mp3}{solutions to health issues caused by swaying office towers and vibrating floors}.

</chapter><chapter><title></title>Sound}
</subsection><subsection><title></title>Musical Instruments}\label{ss-stringed.instruments} <aside><title>Referenced by</title> <p><xref provisional="" /></p></aside>{A-swing.tension}



\part{Is It Hot in Here?}

</chapter><chapter><title></title>The flow of thermal energy}

\phantomsection\label{find-heatwarm}
Energy is a noun<idx></idx>{Energy!noun}; objects can <em></em>have} energy.  <p xml:id=""></p>d-heatverb}{Heat is a verb}<idx></idx>{Heat!verb}; heating is a process of <em></em>exchanging} energy.  Recall our <xref ref=""></xref>{d-forcenoun}{discussions of force}<idx></idx>{Force!noun} and <xref ref=""></xref>{d-workverb}{work}<idx></idx>{Work!verb}.

</section><section><title></title>Specific Heat Capacity}\label{s-specificheat}

<p xml:id=""></p>d-heatwarm}{Heating (positive <m>Q</m>)} can warm (positive <m>\Delta T</m>) a material.
<men>\label{eq-Q=mcDT}
Q = m c \, \Delta T
</men>
but <xref ref="" />eq-Q=mL} (as one example) shows that it is possible to heat (positive <m>Q</m>) a material without warming it (constant <m>T</m>). When we get to <xref ref="" />s-PV} we will see other examples of <q>isothermal processes</q> that have a non-zero <m>Q</m> (heat the system or heat the surroundings) without warming or cooling the system.

</section><section><title></title>Latent Heat}

Heating might also change the phase of a material.<aside><title>Referenced by</title> <p>Discussion of <xref provisional=""></xref></p></aside>[heating versus warming]{d-heatwarm}
<men>\label{eq-Q=mL}
Q = \pm mL
</men>

</section><section><title></title>The Flow of Thermal Energy}

</subsection><subsection><title></title>Thermal Conductivity}\label{ss-thermalconductivity}<aside><title>Referenced by</title> <p><xref provisional="" /></p></aside>{s-story}

<men>\label{eq-thermalconductivity}
\frac{Q}{\Delta t} = \kappa A \, \frac{\Delta T}{\Delta x}
</men>

\begin{example}
\fcolorbox{black}{yellow!10}{\begin{minipage}{4.925in}
\caption{\label{ex-baking}\studentA\protect{<idx><h sortby="\studentA">\studentA</h></idx>} warms \hisA\ oven.}
\studentA\protect{<idx><h sortby="\studentA">\studentA</h></idx>} decides to bake some bread for the dinner party at \studentB\protect{<idx><h sortby="\studentB">\studentB</h></idx>}'s house, but \heA\ is on a tight schedule.  In order to set \hisA\ schedule, \heA\ needs to know how long it will take \hisA\ oven to <xref ref="" text="title"></xref>[find-heatwarm]{warm up}.

<assemblage>Return to: </assemblage> <xref provisional="" />{s-story}
\end{minipage}}
\end{example}

</subsection><subsection><title></title>Convection}
</subsection><subsection><title></title>Radiation}

</chapter><chapter><title></title>Ideal Gas Law}
</section><section><title></title><m>P</m>-<m>V</m> Diagrams}\label{s-PV}<aside><title>Referenced by</title> <p>Discussion of <xref provisional=""></xref></p></aside>[heating versus warming]{d-heatwarm}

\part{Let There be Light!}

</chapter><chapter><title></title>The Electrical Interaction}\label{c-electric}<aside><title>Referenced by</title> <p>Discussion of <xref provisional=""></xref></p></aside>[fundamental forces]{d-fundamental}
</section><section><title></title>Electrical Charge}\label{s-Echarge}<!-- -->\new{v2.1}{Decide where this should go.}

</section><section><title></title>The Big Picture}

</subsection><subsection><title></title>Electric Forces and Fields}\label{ss-Efield}<aside><title>Referenced by</title> <p></p></aside>{\mmr{<xref ref="" />sss-vectorequations}}, \mmr{<xref ref=""></xref>{d-fma}{<m>F=ma</m>}}}

pst-electricfield

</subsubsection><subsubsection><title></title>Examples}

</subsection><subsection><title></title>Electric Forces, Fields, and Potential Energy}

</subsection><subsection><title></title>Electric Fields, Potential Energy, and Potential}

</section><section><title></title>Making Connections}\label{s-Econnection}<aside><title>Referenced by</title> <p><xref provisional="" /></p></aside>{s-Gconnection}

<me> \begin{array}{ccccc}
& & \vec F = q \vec E & & \\
& \deq F = k \frac{q_1 q_2}{r^2} & \leftrightarrow & \deq E = k \frac{q}{r^2} & \\
\Delta \PE = -\vec F \cdot \Delta\vec x & \updownarrow & & \updownarrow & \Delta V = -\vec E \cdot \Delta\vec x  \\
& \deq \PE = k \frac{q_1 q_2}{r} & \leftrightarrow & \deq V = k \frac{q}{r} & \\
& & \Delta \PE = q \Delta V & &
\end{array} </me>
(Recall the parallel with the gravitational interaction in <xref ref="" />s-Gconnection}.)

</chapter><chapter><title></title>Electricity}

</chapter><chapter><title></title>The Magnetic Interaction}

pst-magneticfield

</chapter><chapter><title></title><q>Magnicity?</q>}

</chapter><chapter><title></title>Light}

</chapter><chapter><title></title>Optics}

\part{What Have You Done for Me Lately?}

</chapter><chapter><title></title>Relativity}
</chapter><chapter><title></title>Quantum Mechanics}<!-- -->\new{v2.1}{Decide if these subsections should be chapters in and of themselves.  These are now labeled.}
</section><section><title></title>Atomic Physics} </subsection><subsection><title></title>The Periodic Table and Quantum Numbers}
</section><section><title></title>Nuclear Physics} </subsection><subsection><title></title>Nuclear Decay}\label{ss-nucleardecay}
</subsection><subsection><title></title>The Strong Nuclear Force}\label{ss-strong}<aside><title>Referenced by</title> <p>Discussion of <xref provisional=""></xref></p></aside>[fundamental forces]{d-fundamental}
</subsection><subsection><title></title>The Weak Nuclear Force}\label{ss-weak}<aside><title>Referenced by</title> <p>Discussion of <xref provisional=""></xref></p></aside>[fundamental forces]{d-fundamental}
</section><section><title></title>Particle Physics}\label{s-particle}
</subsection><subsection><title></title>Field Theory}
</subsection><subsection><title></title>Quantum Electrodynamics}\label{ss-QED}<aside><title>Referenced by</title> <p>Discussion of <xref provisional=""></xref></p></aside>[fundamental forces]{d-fundamental}
</subsection><subsection><title></title>Quantum Chromodynamics}\label{ss-QCD}<aside><title>Referenced by</title> <p>Discussion of <xref provisional=""></xref></p></aside>[fundamental forces]{d-fundamental}
</subsection><subsection><title></title>The Standard Model}\label{ss-StandardModel}
</subsection><subsection><title></title>Particle Decay}\label{ss-particledecay}
</chapter><chapter><title></title>Condensed Matter}
</chapter><chapter><title></title>Astronomy}
</chapter><chapter><title></title>Cosmology}

\part{Supplements}

</chapter><chapter><title></title>Deeper Dive}\label{c-revisted}<!-- -->\new{v2.1}{This chapter should mirror \protect{<xref ref="" />c-physics}}.}

This is where I will put the fuller explanations.

</subsection><subsection><title></title>The Sun}\label{sss-sun}
The bright, shiny sun, which keeps us all alive, is a nice example of a rather complex system that allows us to verify our various theories of the world around us.  We can consider the existence of a star in three phases: the birth of a star, the life of the star, and the death of the star.

</subsubsection><subsubsection><title></title>The Birth of a Star}
</subsubsection><subsubsection><title></title>The Life of a Star}
</subsubsection><subsubsection><title></title>The Death of a Star}


</subsection><subsection><title></title>Kitchen Appliances}
</subsubsection><subsubsection><title></title>Oven}
</subsubsection><subsubsection><title></title>Refrigerator}
</subsubsection><subsubsection><title></title>Microwave}
</subsubsection><subsubsection><title></title>Television}

</subsection><subsection><title></title>Automobile}
</subsubsection><subsubsection><title></title>Coolant and Antifreeze}
</subsubsection><subsubsection><title></title>Tires}
</subsubsection><subsubsection><title></title>Torque}

</subsection><subsection><title></title>Cool Ideas}
</subsubsection><subsubsection><title></title>Black Holes}\label{sss-blackhole2}<aside><title>Referenced by</title> <p><xref provisional="" /></p></aside>{ss-weightmass}

On 7 April, 2017,<!-- -->\new{v2.3}{New source of info}
<url href=""></url>{http://www.cbc.ca/podcasting}{CBC Broadcasting} published a
<url href=""></url>{http://www.cbc.ca/podcasting/includes/quirks.xml}{<em></em>Quirks and Quarks}} episode discussing how we can
<url href=""></url>{https://podcast-a.akamaihd.net/mp3/podcasts/quirks_20170408_51226.mp3}{turn our planet into a giant telescope to get a photo of a black hole}.
The results should be available by the early 2018.<todo></todo>Follow-up in 2018 to find the results.}

</subsubsection><subsubsection><title></title>Quantum Mechanics}
</subsubsection><subsubsection><title></title>Relativity}
</subsubsection><subsubsection><title></title>String Theory}



</chapter><chapter><title></title>Podcasts and Videos}\label{c-videos}\label{c-podcasts}

</section><section><title></title>Podcasts}\label{s-podcasts}
<xref ref="" text="title"></xref>{http://spacepod.libsyn.com/}{T4LTFdOxHD5WWzdD}{99}{Spacepod with Carrie Nugent} \\
<url href=""></url>{http://www.sciencefriday.com/}{Science Friday with Ira Flatow}

</section><section><title></title>Videos}\label{s-videos}
<url href=""></url>{http://physicsfootnotes.com/}{Physics Footnotes} \\
<url href=""></url>{http://sixtysymbols.com/}{Sixty Symbols}

</section><section><title></title>Websites}\label{s-websites}
<url href=""></url>{http://www.aldakavlilearningcenter.org/practice/flame-challenge}{The Flame Challenge}

</chapter><chapter><title></title>Answers to Interactive Questions}

\begin{AIQ}
</p></li><li><p>\label{A-hbf} There are forces acting on it.  You should be able to tell this because you are exerting one of the forces. While it is true that there are forces on it, it is also true that there is no <em></em>net force}.  If you are exerting an upward force on the book, can you guess (<xref ref="" text="type-global" />A-gravity}) what the downward force is?   <assemblage>Return to: </assemblage> <xref provisional="" />{IQ-holdbook}
</p></li><li><p>\label{A-chair1} If we refer to <q>motion</q> as describing the velocity, then no. Force causes a <em></em>change in} velocity. When you stop pushing, the chair stops because there is a force from the carpet acting to oppose the force you apply while you push the chair. <assemblage>Return to: </assemblage> <xref provisional="" />{irl-NI}
</p></li><li><p>\label{A-chair2} This is essentially the same as <xref ref="" text="type-global" />A-chair1}, but the carpet exerts more force than the tile.  In either case, force causes a <em></em>change in} velocity. You are trying to speed the chair up and the floor is trying to slow the chair down.  (Both are trying to change the velocity, but cancel to result in a constant velocity.)  When you stop pushing, the chair stops moving because there is a force from the tile acting to oppose the force you apply while you push the chair; when you let go, this force slows the chair until the chair stops and then the force stops acting. (See <xref ref="" />s-Ff} for more details.) <assemblage>Return to: </assemblage> <xref provisional="" />{irl-NI}
</p></li><li><p>\label{A-weight.loss} Since \studentB<idx><h sortby="\studentB">\studentB</h></idx> weighs <m>(\massB)(9.81\unitfrac{m}{s^2})=736\unit{N}</m>, <m>45\unit{N}</m> is about <m>6\%</m> of her weight.  This is fairly substantial.  You should compute how much <m>6\%</m> of your weight is and convert that to kilograms and Newtons.  <assemblage>Return to: </assemblage> <xref provisional="" />{irl-scale}
</p></li><li><p> \label{A-ladderNf} Since the full weight of the ladder, <m>F_g = \sig{222}{.69}{N}</m>, is still pressing downwards into the floor (as a normal force), it is tempting to say that <xref ref="" text="title"></xref>[ss-NIII]{Newton's third law} implies that the floor pushes the ladder upwards with a normal force of <m>\sig{222}{.69}{N}</m> but this would not account for the frictional force on the wall, <m>F_{fw}</m>.  If there were no friction between the ladder and the wall, then we could deduce <m>F_{Nf}</m>, but at this point, we cannot. <assemblage>Return to: </assemblage> <xref provisional="" />{ex-ladder2}
</p></li><li><p>\label{A-hbnof}  It is true that while you hold the book, there is no <em></em>net force}, but that does not mean that there is no force acting.  If there were no forces on the book, then your hand would not need to be there.  In fact, if you remove the force your hand provides, then the book falls. This shows that there is an upward force (by your hand on the book) and a downward force (of gravity by the Earth on the book).  <assemblage>Return to: </assemblage> <xref provisional="" />{IQ-holdbook}
</p></li><li><p>\label{A-netF-a} Since the object in <xref ref="" text="type-global" />se-netF-a} has a mass of <m>2.0\unit{kg}</m>, we can find the weight by
    <m> \vec F_g = m \vec g = (2.0\unit{kg}) [-(9.81\unitfrac{m}{s^2})\,\jhat] = -\sig{19}{.62}{N} \jhat = -20 \unit N \jhat </m>
    <assemblage>Return to: </assemblage> <xref provisional="" />{se-weightA}
</p></li><li><p>\label{A-chair3} For a chair with wheels being pushed across a tile floor, when you stop pushing it probably continues to move across the floor for at least a short distance.  <assemblage>Return to: </assemblage> <xref provisional="" />{irl-NI}
</p></li><li><p>\label{A-weight.gain} When one person stands on the scale, the scale provides just enough of an upwards normal force to keep that person in equilibrium<todo></todo>link equilibrium?}.  In that case, the upwards force is balancing the weight of the person.  This gives the impression that the scale is telling you your weight; however, when you press down or help support whomever is standing on the scale, the scale adjusts the amount it must provide.  The scale is not trying to tell you your weight.  Rather the scale is trying to create equilibrium by balancing whatever force(s) are pressing into it.  Your weight is determined by the gravitational force<todo></todo>link the gravitational force?}{} and does not change when you press harder or lighter onto the scale.  <assemblage>Return to: </assemblage> <xref provisional="" />{irl-scale}
</p></li><li><p>\label{A-nowall} If we consider <m>\mu_w\rightarrow 0</m>, then <m>F_{fw}=0\unit N</m>,  <m>\vec F_{Nf} = -\vec F_g = \sig{222}{.7}{N} \jhat</m>, and <m>\vec F_{Nw} = - \vec F_{ff} = \sig{28.7}{9}{N} \ihat</m>.  In this case, <m>\mu_f</m> could be as small as <m>0.\sig{129}{3}{}</m> and still hold the ladder in place, unless \studentC<idx><h sortby="\studentC">\studentC</h></idx> climbs the ladder, in which case see <xref ref="" text="type-global" />A-nowallC}.  <assemblage>Return to: </assemblage> <xref provisional="" />{ex-ladder2}
</p></li><li><p>\label{A-true1} It is in equilibrium.  When the acceleration is zero, then the net force must be zero and those properties are what define equilibrium. <assemblage>Return to: </assemblage> <xref provisional="" />{IQ-holdbook}
</p></li><li><p>\label{A-chair4} The chair continues to move for the same reason that the chair without wheels and the chair on carpet <em></em>all} continued to move when you let go.  The reason is that this is <em></em>how all objects behave; they maintain their velocity when allowed to act without interference.}  (This is why Newton's first law says what it does.)   Because the chair with wheels has much less friction there is a smaller force trying to interfere with the motion and so it continues to move for a noticeable distance. The other chairs slowed to a stop almost immediately.  The wheel-less chair on tile might have continued for a short distance if it was moving fast enough that it required a long enough time to change its velocity to zero.  <assemblage>Return to: </assemblage> <xref provisional="" />{irl-NI}
</p></li><li><p>\label{A-scale.increase} When you press down on \studentB's shoulders, you are not adding weight.  Weight has a specific definition: it is specifically the value that the gravitational force<todo></todo>link the gravitational force}{} pulls on any object.  Pushing the person does not change their weight; it does, however, change the amount that they press into the Earth.  That is to say, it increases their downwards normal force, but not their weight.  <assemblage>Return to: </assemblage> <xref provisional="" />{irl-scale}
</p></li><li><p>\label{A-nowallC} If we consider <m>\mu_w\rightarrow 0</m> with \studentC<idx><h sortby="\studentC">\studentC</h></idx> (<m>m=\massC</m>) at the third-rung-from-the-top of the ladder, (<m>1.53\unit m</m> up the ladder), then <m>F_{fw}=0\unit N</m>,  <m>\vec F_{Nf} = \sig{1105}{.6}{N} \jhat</m>, and <m>\vec F_{Nw} = - \vec F_{ff} = \sig{171}{.97}{N} \ihat</m>.  In this case, <m>\mu_f</m> could be as small as <m>0.\sig{155}{5}{}</m> and still hold the ladder in place. <assemblage>Return to: </assemblage>{\mmr{<xref ref="" text="type-global" />A-nowall}}, \mmr{<xref ref="" />ex-ladder2}}}
</p></li><li><p>\label{A-gworld} Because the Earth was spinning as it cooled (forming the crust), it formed an oblate spheroid<fn xml:id=""></fn>{The equator is slightly further from the center than the poles are.}.  Since the strength of the gravitational interaction depends (among other things) on how far you are from the center (slightly weaker further away), the acceleration due to gravity is smaller when you are at smaller latitudes (closer to the equator).  <assemblage>Return to: </assemblage>{\mmr{<xref ref="" text="type-global" />A-gpeaks}}, \mmr{<xref ref="" />t-gworld}}}
</p></li><li><p>\label{A-false1} The definition of equilibrium is that the forces balance.  The result of this is that the net force must be zero and the acceleration is then zero.  You can tell this is true because the velocity is <em></em>not changing}.  It is not important that the velocity is zero, what is important is that the velocity <em></em>stays} zero.  While you hold it, the book is in equilibrium. <assemblage>Return to: </assemblage> <xref provisional="" />{IQ-holdbook}
</p></li><li><p>\label{A-chair5} No. But if it does not matter what you do after you let go of the chair, then why do coaches (in basketball free-throws, tennis serves and swings, baseball pitches, and all manner of arm and leg propulsion) tell you to pay attention to your <q>follow through</q>? \TWO{They have been fooled; follow-through doesn't matter}{they are right; follow-through does matter!}{A-noFT}{A-FT} <assemblage>Return to: </assemblage> <xref provisional="" />{irl-NI}
</p></li><li><p>\label{A-scale.measure} Since your weight is a force pulling downwards, having the scale on the wall shows that the scale cannot be balancing weight.  Since you are pushing into the wall, you are exerting a normal force into the scale and the scale is exerting a normal force back at you.  Both of these forces are horizontal (assuming the wall is plumb).  <assemblage>Return to: </assemblage> <xref provisional="" />{irl-scale}
</p></li><li><p>\label{A-gpeaks} In addition to being an oblate spheroid (<xref ref="" text="type-global" />A-gworld}), the Earth has mountains and valleys.  Since the strength of the gravitational interaction depends (among other things) on how far you are from the center (slightly weaker further away), the acceleration due to gravity is smaller when you are at at high altitudes, such as Denver, CO and Mount Everest.  <assemblage>Return to: </assemblage> <xref provisional="" />{t-gworld}
</p></li><li><p>\label{A-falls}  Both <q>Yes</q> and <q>No</q> bring you to this answer.  Yes, there is \underline{a force} on the book while it falls (the force of gravity), but no, there are not force\underline{s} (plural).  There is only one force.  <q>But, wait!</q> you say, <q>What about the force of air resistance?</q>  Aha!   You are correct; there is a force of air resistance, but in this case, it is negligible and we will not consider it.  Please read <xref ref="" />ss-airresistance} for more information about deciding when to use or ignore this phenomenon.  <assemblage>Return to: </assemblage> <xref provisional="" />{IQ-holdbook}
</p></li><li><p>\label{A-chair6} No.  When you throw a ball very high into the air, you can dance a jig or do any manner of things and it will obviously not affect the ball.  The force you exerted on the chair goes away the instant you stop touching the chair.  It is, however, true that your force gave the chair some velocity (actually <xref ref="" text="title"></xref>[c-momentum]{momentum}) and Newton's first law (inertia) says that the chair would prefer to keep that velocity.  Unfortunately, the friction with the ground slows it down.  The careful way to describe the situation is that your force gave the chair some velocity (actually <xref ref="" text="title"></xref>[c-momentum]{momentum}) and its characteristic inertia made it difficult for the <xref ref="" text="title"></xref>[s-Ff]{frictional force} to slow it down rapidly. <assemblage>Return to: </assemblage> <xref provisional="" />{irl-NI}
</p></li><li><p>\label{A-fly.balls} TOOK <assemblage>Return to: </assemblage> <xref provisional="" />{irl-nonparabolic}
</p></li><li><p>\label{A-hitY} The book is accelerating.  The velocity \underline{is changing} from <q>moving downwards</q> to <q>stopped</q>.  The book is not in equilibrium.  <assemblage>Return to: </assemblage> <xref provisional="" />{IQ-holdbook}
</p></li><li><p>\label{A-noFT} They haven't been fooled, but follow-through matters in a different way.  What does matter is not literally how you move <em></em>after} the release, but rather how you move <em></em>before} you release the ball. By paying attention to your follow-through, you are also changing the way you move before you release or impact the ball.  You want a smooth flow throughout the motion and a sloppy follow-through often implies a sloppy initiation of the motion.  <assemblage>Return to: </assemblage> <xref provisional="" />{A-chair5}
</p></li><li><p>\label{A-scale.ramp} When the scale is on the flat, horizontal floor, it balances your full weight.  When the scale is on the vertical wall it does not carry any of your weight.  At any angle in between those values, it carries some fraction of your weight while friction keeps you from sliding down the ramp<todo></todo>link to the section on friction and ramps}.  It will turn out that since the cosine function<todo></todo>link to the trig section}{} behaves in just the right way, we can use<todo></todo>link <q>can use</q> to the section on ramps}{} the cosine to find the component of the weight that the normal force from the scale has to support.   <assemblage>Return to: </assemblage> <xref provisional="" />{irl-scale}
</p></li><li><p>\label{A-pitches.side} TOOK  <assemblage>Return to: </assemblage> <xref provisional="" />{irl-nonparabolic}
</p></li><li><p>\label{A-hitN} While it is hitting the desk, the velocity is changing from <q>moving downwards</q> to <q>stopped</q>.  Since the velocity is changing, \underline{the book is accelerating}.  Since it is accelerating, the book is not in equilibrium.  <assemblage>Return to: </assemblage> <xref provisional="" />{IQ-holdbook}
</p></li><li><p>\label{A-chair7} The force that the chair feels after you release it is <xref ref="" text="title"></xref>[s-Ff]{friction}.  For the carpet, there is a lot of friction and the chair slows down very quickly (essentially instantaneously).  For the wheel-less chair on the tile floor, the chair slows rapidly although it may leave your hand.  The wheels provide the least amount of friction and that chair goes the furthest.  You may note that the friction slowing the chair-with-wheels is primarily between the rolling wheel and its axel (where it connects to the non-rolling chair leg) rather than between the wheel and the floor (although the friction between the wheel and the floor also plays a role).  This is discussed in more detail in <xref ref="" />s-Ff}. <assemblage>Return to: </assemblage> <xref provisional="" />{irl-NI}
</p></li><li><p>\label{A-pitches.top} TOOK <assemblage>Return to: </assemblage> <xref provisional="" />{irl-nonparabolic}
</p></li><li><p>\label{A-landedY} It is in equilibrium.  The book is at rest and <em></em>continues to be} at rest on the desk. There are forces acting, but they cancel each other, resulting in no net force.  <assemblage>Return to: </assemblage> <xref provisional="" />{IQ-holdbook}
</p></li><li><p>\label{A-gravity}  It is the force of gravity. <assemblage>Return to: </assemblage> <xref provisional="" />{A-hbf}
</p></li><li><p>\label{A-chair8} If there were no friction, then you could start the chair and it would move on its own at a constant speed; you wouldn't need to continue pushing to keep it moving.  On the other hand, if you did continue to push, then the chair would continue to speed up and you would have to run faster and faster to keep up with it. On the other hand, if the chair were not experiencing friction, then you probably wouldn't either and you couldn't get enough traction to keep up with the chair, so it would sail away almost immediately, being then described by Newton's first law! <assemblage>Return to: </assemblage> <xref provisional="" />{irl-NI}
</p></li><li><p>\label{A-pool.roll} First, you should not roll a pool ball across just any floor; there is felt on the pool table for a reason.  However, if you have a clean, smooth surface and are able to reproduce your rolling speed, you will find that the pool ball rolls further on the stiff, nonyielding surface than it will on the felt.  The reason for this is beyond the scope of this textbook, but you can read more from <url href=""></url>{http://stacks.iop.org/0031-9120/30/i=3/a=009}{<q>Sliding and rolling: the physics of a rolling ball,</q> J. Hierrezuelo and C. Carnero, Physics Education, Volume 30, Number 3} (unofficially at <url href=""></url>{http://billiards.colostate.edu/physics/Hierrezuelo_PhysEd_95_article.pdf}{this PDF}). <assemblage>Return to: </assemblage> <xref provisional="" />{irl-poolcushion}
</p></li><li><p>\label{A-landedN} After it has landed, the book stops moving.  Once the book comes to rest on the desk, it <em></em>continues to stay at rest}.  This says that the velocity is not changing, so the book is not accelerating.  That means that the book is in equilibrium. There are forces acting, but they cancel each other, resulting in no net force. <assemblage>Return to: </assemblage> <xref provisional="" />{IQ-holdbook}
</p></li><li><p>\label{A-FT} What does matter is not literally how you move <em></em>after} the release, but rather how you move <em></em>before} you release the ball. By paying attention to your follow-through, you are also changing the way you move before you release or impact the ball.  You want a smooth flow throughout the motion and a sloppy follow-through often implies a sloppy initiation of the motion.  <assemblage>Return to: </assemblage> <xref provisional="" />{A-chair5}
</p></li><li><p>\label{A-pool.bumper} The cushion (sometimes called a bumper) is pretty still to the touch, but it is made of a springy rubber that allows the balls to bounce reasonably well.  The \protect{<url href=""></url>{http://c.ymcdn.com/sites/bca-pool.com/resource/resmgr/imported/BCAEquipmentSpecifications_2008.pdf}{document}} indicates that you should be able to firmly strike a ball at some angle to the far wall and have it bounce around the table four to four-and-a-half times.  If the bumpers were perfectly <xref ref="" text="title"></xref>[s-elastic]{elastic}, then the normal force would be normal to the restign surface; but since the bumper has some flexibility, when the ball hits the bumper with a glancing blow, then bumper bends inwards and the normal force is directed in a way that depends on the shape of the dent.  <assemblage>Return to: </assemblage> <xref provisional="" />{irl-poolcushion}.
</p></li><li><p>\label{A-zero} Recall the situation when you were holding the book.  Gravity is still pulling the book down and the desk is holding the book up.  There are two forces acting on the book while it is at rest on the desk. <assemblage>Return to: </assemblage> <xref provisional="" />{IQ-holdbook}
</p></li><li><p>\label{A-firstfall} TOOK  <assemblage>Return to: </assemblage> <xref provisional="" />{irl-freefall}
</p></li><li><p>\label{A-floor}  I hope you guessed the floor.  That is the only thing pushing up on \studentB<idx><h sortby="\studentB">\studentB</h></idx>.  One useful way to think about it is that the floor is the thing keeping \himB\ from falling.  The direction of this force is <em></em>normal} (perpendicular) to the horizontal floor, so it is in the vertical direction.  This will be discussed in more detail in <xref ref="" />s-FN}. <assemblage>Return to: </assemblage> <xref provisional="" />{se-FNB}
</p></li><li><p>\label{A-firstwhy} TOOK  <assemblage>Return to: </assemblage> <xref provisional="" />{irl-freefall}
</p></li><li><p>\label{A-noncue} The <url href=""></url>{http://wpapool.com/equipment-specifications/\#Balls-and-Ball-Rack}{specifications} show that there is no difference between the solids and stripes, but the cue ball weighs <m>9\%</m> more that the other balls (<m>6.0\unit{oz}</m> versus <m>5.5\unit{oz}</m>).  The colored balls and the cue ball are otherwise identical.  <assemblage>Return to: </assemblage> <xref provisional="" />{irl-poolcushion}
</p></li><li><p>\label{A-one} If there were only one force on the book, it could not be a balanced force, so the book could not be in equilibrium and the book would be accelerating.  The book is not accelerating, so there are either two forces (<xref ref="" text="type-global" />A-two}) or no forces (<xref ref="" text="type-global" />A-zero}).  <assemblage>Return to: </assemblage> <xref provisional="" />{IQ-holdbook}
</p></li><li><p>\label{A-fallv} TOOK  <assemblage>Return to: </assemblage> <xref provisional="" />{irl-freefall}
</p></li><li><p>\label{A-pool.spin} Because the bumper is covered in felt, it has a small grip on the ball.  Because the bumper has some give to it, it dents in when hit and provide more surface area, which increases the grip.  Both of these mean that the spin of the ball gets transferred to the pool table somewhat and change the way a spinning ball exits from the bumper collision.  <assemblage>Return to: </assemblage> <xref provisional="" />{irl-poolcushion}
</p></li><li><p>\label{A-second} You can tell that it is Newton's second law <m>(\vec F_\mathrm{net} = m \vec a)</m> because the forces we are considering are acting on the <em></em>same} object.  In this case, the gravitational force is caused by the Earth and the normal force is caused by the floor by they are both felt by \studentB<idx><h sortby="\studentB">\studentB</h></idx>.  These forces happen to be equal and opposite because \heB\ happens to be in equilibrium. \HeB\ does not <em></em>have to be} in equilibrium, such as when \heB\ jumps, in which case the forces would not be equal and might not be opposite. <assemblage>Return to: </assemblage> <xref provisional="" />{se-FNB}
</p></li><li><p>\label{A-falla} TOOK  <assemblage>Return to: </assemblage> <xref provisional="" />{irl-freefall}
</p></li><li><p>\label{A-pool.later} This answer is getting \important{too complex for the section} it is in.  I need to move the IRL before I finish considering how to answer this question. <assemblage>Return to: </assemblage> <xref provisional="" />{irl-poolcushion}
</p></li><li><p>\label{A-two} There are two forces acting on the book while it is at rest on the desk. Similar to the situation when you were holding the book, gravity is pulling the book down and the desk is holding the book up.  <assemblage>Return to: </assemblage> <xref provisional="" />{IQ-holdbook}
</p></li><li><p>\label{A-third} If it were Newton's third law, then the two forces we were discussing would be acting on different objects and would be unrelated to the fact that the object (in this case, \studentB<idx><h sortby="\studentB">\studentB</h></idx>) is in equilibrium.  The gravitational force and the normal force in this case are both acting on \studentB, so although they happen to be equal and opposite, this is not due to Newton's third law.

    You should, however, note that the force that is reaction-paired to the gravitational force on \studentB<idx><h sortby="\studentB">\studentB</h></idx> by the Earth is a gravitational force on the Earth by \studentB.  Similarly, the reaction-paired force to the normal force on \studentB\ by the floor is a normal force on the floor by \studentB.  (Please note the <q>on</q> and <q>by</q> in each case.) <assemblage>Return to: </assemblage> <xref provisional="" />{se-FNB}

</p></li><li><p>\label{A-swing.tension} To make this comparison, let's consider a swing that is supported by chains.  If you are sitting in the swing and take hold of the chains at about shoulder height, you should be able to shake them in (towards your chest) and out (away from you, towards your neighbor swings).  You can do this same motion while standing next to the swing.  If you do this when the swing is empty, it is very easy to do this.  If you ask a series of successively larger people to sit in the swing, you will notice that it gets progressively more difficult to extend them very far.  The chains are increasing in tension; they are pulled more taut.  Your ability to move the chain in this way is exactly analogous to the way a bow draws across a violin or the way your fingers pluck a guitar, as described in <xref ref="" />ss-stringed.instruments}. <assemblage>Return to: </assemblage>{\mmr{<xref ref="" />irl-tension}}, \mmr{<xref ref="" text="type-global" />A-chandelier.tension}}}
</p></li><li><p>\label{A-fan.tension} You might also consider <xref ref="" text="type-global" />A-chandelier.tension}, which discusses the case of hanging a light fixture from the ceiling. If you have ever installed a fan in your house, then you will notice that you have to support the fan while the wires are connected.  Usually the fan has a shaft that connects to the ceiling at one end and the fan at the other and provides a mechanism for supporting the fan while you manage the wires, which pass through the shaft.  Since the fan houses the motor, it is usually reasonably heavy.   The nice property of use a metal shaft to support the fan is that it doesn't stretch or wiggle like a chain might.  The difficulty in this example is that it is more difficulty to notice the tension in the shaft.  If you are the person hanging the fan, then one thing you might be able to notice is that if you flick the metal with a finger when it is not supporting anything, it will have a slightly different <q>ting</q> than when it is supporting the fan.<todo></todo>Is this sufficiently noticeable?} <assemblage>Return to: </assemblage> <xref provisional="" />{irl-tension}
</p></li><li><p>\label{A-chandelier.tension} If you have ever installed a chandelier in your house, then you will notice that the light has to be supported between the joists of the ceiling.  There will be an electrical box with a screw to which you will attach the support for the chain that holds the chandelier.  The wires will run through the support chain.  The heavier the chandelier, the tauter the chain, much as described in <xref ref="" text="type-global" />A-swing.tension}. This tension is much easier to see than the tension in the shaft of the fan.  <assemblage>Return to: </assemblage> <xref provisional="" />{A-fan.tension}
\end{AIQ}

</chapter><chapter><title></title>Adventures}


Throughout the book, there are examples and adventures.  The follow-up stories are contained below.
\begin{Story}
</p></li><li><p>\label{a-parkandwalk}  TOOK
</p></li><li><p>\label{a-NIIIaction} As \studentC<idx><h sortby="\studentC">\studentC</h></idx> gets pushed, you notice that \heC\ was not aware of the pending doom.  \HeC\ is standing casually with \hisC\ feet set to support his own weight, but not to brace \himC\ against the sideways force.  When \heC\ gets pushed from the side, \hisC\ feet stay in place and \hisC\ torso topples, rotating \himC\ about \hisC\ center of mass\Foreshadow{The physics of why an object (or person) rotates when they fall over is discussed with \protect{(<assemblage>Return to: </assemblage> <xref ref="" />.)[c-torque]{torque}}.}{} as \heC\ falls to the ground. \studentD\ points to \hisD\ phone and says, <q>I recorded the whole thing!</q>  If you respond, <q>Awesome! Can I watch the part about how \studentZ<idx><h sortby="\studentZ">\studentZ</h></idx> acts?</q>, please read <xref ref="" text="type-global" />a-NIIIreaction}.  If you respond, <q>Awesome! Let's show the psychology and physics faculty our cool video!</q>, please read <xref ref="" text="type-global" />a-NIIIfaculty}. If you respond, <q>Yeah, we probably should have intervened before this happened instead of just watching.  Let's go talk to Campus Security.</q>, please read <xref ref="" text="type-global" />a-NIIIsecurity}.
</p></li><li><p>\label{a-coastindrive} TOOK
</p></li><li><p>\label{a-NIIIreaction} As \studentZ<idx><h sortby="\studentZ">\studentZ</h></idx> pushes, you notice that because \heZ\ was being intentional, \heZ\ put one foot behind \himZ\ to brace \hisZ\ body during the push.  \HeZ\ leans into the push and stays standing.  You are intrigued.  If you decide to do a follow-up experiment by pushing \studentD<idx><h sortby="\studentD">\studentD</h></idx> over without bracing yourself, then read <xref ref="" text="type-global" />a-NIIIexperiment}.  If you decide to exercise self-restraint, then read <xref ref="" text="type-global" />a-NIIIrestraint}.
</p></li><li><p>\label{a-coastinneutral} TOOK
</p></li><li><p>\label{a-NIIIconcern} Being the thoughtful and considerate person you are, you rush over and startle \studentC<idx><h sortby="\studentC">\studentC</h></idx> out of \hisC\ reverie.  \studentZ<idx><h sortby="\studentZ">\studentZ</h></idx> is quite angry and now focuses \hisZ\ attention on you!  \HeZ\ rushes towards you and shoves as hard as he can.  You go <em></em>flying} backwards and land on your tailbone while he just stands there laughing.  \studentC\ and \studentD<idx><h sortby="\studentD">\studentD</h></idx> both rush over to help you while \studentZ\ wanders off.  Surprisingly, \studentC\ has an icepack, which helps.  If you go speak to your faculty members about this, please read <xref ref="" text="type-global" />a-NIIIfaculty}. If you decide to talk to Campus Security, please read <xref ref="" text="type-global" />a-NIIIsecurity}.
</p></li><li><p>\label{a-NIdrive} TOOK
</p></li><li><p>\label{a-NIIIexperiment} You turn and push \studentD<idx><h sortby="\studentD">\studentD</h></idx> over.  Like \studentC<idx><h sortby="\studentC">\studentC</h></idx>, \heD\ did not expect it and was not braced, so \heD\ falls over. Similarly, you decided not to brace yourself and in pushing \studentD, you fall over backwards!  \studentD\ did not have to <em></em>choose} to push on you.  The act of you deciding to push \himD\ necessarily and simultaneously produces a force on you, equal in magnitude and opposite in direction.  Unfortunately, \studentD\ doesn't think this was a useful exercise and shouts <q>I have the whole thing on video!</q> and storms off to Campus Security.  You are arrested for assault, miss your physics class for a couple of weeks and ultimately fail all of your classes. I certainly hope this was all happening in your head and not in real life!  You learned something about physics, but at what cost to your humanity?  (<assemblage>Return to: </assemblage> <xref ref="" />.)[cyoa-NIII]{The end!}
</p></li><li><p>\label{a-nogas} TOOK
</p></li><li><p>\label{a-NIIIrestraint} \studentD<idx><h sortby="\studentD">\studentD</h></idx> points to \hisD\ phone and says, <q>I recorded the whole thing!</q>  If you respond, <q>Awesome! Can I watch the part about how \studentC<idx><h sortby="\studentC">\studentC</h></idx> acts?</q>, please read <xref ref="" text="type-global" />a-NIIIaction}.  If you respond, <q>Awesome! Let's show the psychology and physics faculty our cool video!</q>, please read <xref ref="" text="type-global" />a-NIIIfaculty}. If you respond, <q>Yeah, we probably should have intervened before this happened instead of just watching.  Let's go talk to Campus Security</q>, please read <xref ref="" text="type-global" />a-NIIIsecurity}.
</p></li><li><p>\label{a-parkandwalk2}  TOOK
</p></li><li><p>\label{a-NIIIfaculty} The psychology faculty member speaks to you both about how to be good citizens and about the psychological effects of bullies both on the bully and on the recipient.  If you decide to learn more about this, please read <url href=""></url>{https://www.psychologytoday.com/basics/bullying}{Psychology Today}.  The physics faculty member points out that when one person pushes another, the person being pushed does not brace \himselfC, whereas the person doing the pushing does.  Furthermore one might imagine what would happen if you did not brace yourself when you pushed each other, such as in <xref ref="" text="type-global" />a-NIIIexperiment}.  You are asked to review both <xref ref="" />ex-braced}  (pg.~\pageref{ex-braced}) and <xref ref="" />ex-unbraced}  (pg.~\pageref{ex-unbraced}) before the next exam.  On your way out the door, you hear a voice suggest <q><ellipsis /> and you <em></em>might} want to talk to (<assemblage>Return to: </assemblage> <xref ref="" />.)[a-NIIIsecurity]{Campus Security} about the incident<ellipsis /></q>
</p></li><li><p>\label{a-intosunset} TOOK
</p></li><li><p>\label{a-NIIIsecurity} You speak with Campus Security about the incident and \studentZ<idx><h sortby="\studentZ">\studentZ</h></idx> gets taken in for assault.  The Dean thanks you for being brave enough to speak up. (<assemblage>Return to: </assemblage> <xref ref="" />.)[cyoa-NIII]{The end!}
</p></li><li><p>\label{a-NIresult} TOOK
</p></li><li><p>\label{a-guilty} You feel guilty for letting \studentZ<idx><h sortby="\studentZ">\studentZ</h></idx> push \studentC<idx><h sortby="\studentC">\studentC</h></idx> down despite your amazing score on the next physics test.  It wasn't worth it.  (<assemblage>Return to: </assemblage> <xref ref="" />.)[cyoa-NIII]{The end!}
</p></li><li><p>\label{a-nogas2} TOOK
</p></li><li><p>\label{a-intosunset2} TOOK
\end{Story}

%%%%%%%%%%%%%%%%%%%%%%%%%%%%%%%%%%%%%%%%%%%%%%%%%%%%%%%%%%%%%%%%%%%%%%%%%%%%%%%%%%%%%%%%%%%%%%%%%%%%%%

</chapter><chapter><title></title>Characters}

This textbook has five characters who follow you throughout the book.  They appear in the examples and some homework problems.  They also remember previous experiences.  I need to adjust the examples in <xref ref="" />c-force} such that the people pushing boxes are helping the reader rearrange furniture.

The index lists<todo></todo>The index will recognize the people in two different formats.  One is by my name for them, which is <backslash />studentX (where X is A, B, C, D, <ellipsis /> Z).  The other is by the name assigned to that variable.  So these show up in different places in the Index.}{} the pages that the characters appear.  The point of this chapter is to highlight some of the primary adventures of the characters according to their own perspectives.  <em></em>None of the links in this chapter will be given a corresponding return link.}  This chapter is for me to track relationships and will likely go away when the book is ready for publication.
%
I can, at the header of the code, define the name, gender, mass, and dimensions of each individual.\dothis[inline]{<url href=""></url>{http://malveyauthor.com/}{Madeline Alvey}, the author of  \protect{<url href=""></url>{http://escapepod.org/2017/03/09/ep566-honey-and-bone-artemis-rising-3/}{<q>Honey and Bone</q> at EscapePod}} is a physics and English undergraduate student at UK in Lexington.  I might consider hiring(?) her to help storyboard the characters.}

</section><section><title></title>\studentA<idx></idx>{\studentA!inside}}<idx></idx>{\studentA!outside}<todo></todo>The index-call that is <em></em>outside} of the section title registers as <backslash />studentA, which puts the name alphabetically under <backslash />studentA, rather than \studentA.  The index-call that is <em></em>inside} of the section title registers as \studentA, which puts the name alphabetically under \studentA, rather than <backslash />studentA.}
<idx></idx>{\studentA|(} % Begin page-range
<ul>
</p></li><li><p> In <xref ref="" />s-forcewords}, \studentB{} gives \studentA{} a good-natured shove in the arm in order to get the language clarified and begin the conversation about the on-by notation.
</p></li><li><p> In <xref ref="" text="type-global" />se-FBD-AB} \studentA{} helps \studentB{}<ellipsis />
    <ul>
    </p></li><li><p> (in the current version) push an object to make it accelerate and feel a reaction force causing \himA{} to accelerate backwards.
    </p></li><li><p> (in the future version) will help the reader move into or out of their residence hall by pushing on heavier furniture.
    </p></li><li><p>[NOTE:] This is all drawn in <xref ref="" />f-firstFBD}, which is updated in <xref ref="" />f-firstFBDupdate}.
    </ul>
</p></li><li><p> In <xref ref="" text="type-global" />se-weightA}, \studentA{} falls from a small height.  (maybe he is jumping off a short ledge while taking a short-cut to class?)
</p></li><li><p> In <xref ref="" />ex-baking}, \studentA{} decides to bake some bread for a party at \studentB's house, measuring the time it takes to warm his oven.
</ul>

<idx></idx>{\studentA|)} % end page-range
</section><section><title></title>\studentB<idx><h sortby="\studentB">\studentB</h></idx>}
<idx></idx>{\studentB|(} % Begin page-range

<ul>
</p></li><li><p> \studentB{} is a passenger in the reader's car in <xref ref="" />ex-slowcar} when the reader runs out of gas and coasts to a stop.
</p></li><li><p> \studentB{} is a passenger in the reader's car in <xref ref="" />ex-coasting} and speculates about how fast to go before putting the car in neutral to coast to a stop.
</p></li><li><p> \studentB{} joins the reader on a road trip in <xref ref="" />cyoa-NI} and runs out of gas.  This results in multiple possible adventures:
<ul>
    </p></li><li><p> <xref ref="" text="type-global" />a-parkandwalk}, which leads to either an end at <xref ref="" text="type-global" />a-nogas} or an end at <xref ref="" text="type-global" />a-intosunset}.
    </p></li><li><p> <xref ref="" text="type-global" />a-coastindrive}, which leads to either <xref ref="" text="type-global" />a-NIdrive} (choose <xref ref="" text="type-global" />a-coastinneutral} or end with <xref ref="" text="type-global" />a-intosunset2}) or <xref ref="" text="type-global" />a-parkandwalk2} (choose <xref ref="" text="type-global" />a-intosunset} or end at <xref ref="" text="type-global" />a-nogas2})
    </p></li><li><p> <xref ref="" text="type-global" />a-coastinneutral}, which leads to an end at <xref ref="" text="type-global" />a-NIresult}.
</ul>
</p></li><li><p> In <xref ref="" />s-forcewords}, \studentB{} gives \studentA{} a good-natured shove in the arm in order to get the language clarified and begin the conversation about the on-by notation.
</p></li><li><p> In <xref ref="" />se-FBD-AB}, \studentB{} helps \studentA{}<ellipsis />
    <ul>
    </p></li><li><p> (in the current version) pull an object to make it accelerate and feel a reaction force causing \himB{} to accelerate backwards.
    </p></li><li><p> (in the future version) will help the reader move into or out of their residence hall by pushing on heavier furniture.
    </p></li><li><p>[NOTE:] This is all drawn in <xref ref="" />f-firstFBD}, which is updated in <xref ref="" />f-firstFBDupdate}.
    </ul>
</p></li><li><p> In <xref ref="" text="type-global" />se-FNB}, \studentB{} has a normal force supporting \himB.  (This touches <xref ref="" text="type-global" />A-floor}, <xref ref="" text="type-global" />A-second}, and <xref ref="" text="type-global" />A-third}.)
</p></li><li><p> At some point, \studentB{} has a party, because in <xref ref="" />ex-baking}, \studentA{} decides to bake some bread for a party at \studentB's house.
</ul>

<idx></idx>{\studentB|)} % end page-range
</section><section><title></title>\studentC}
<idx></idx>{\studentC|(} % Begin page-range

<idx></idx>{\studentC|)} % end page-range
</section><section><title></title>\studentD}
<idx></idx>{\studentD|(} % Begin page-range

<idx></idx>{\studentD|)} % end page-range
</section><section><title></title>\studentE}
<idx></idx>{\studentE|(} % Begin page-range

<idx></idx>{\studentE|)} % end page-range
</section><section><title></title>\studentF}
<idx></idx>{\studentF|(} % Begin page-range

<idx></idx>{\studentF|)} % end page-range
</section><section><title></title>\studentZ}
<idx></idx>{\studentZ|(} % Begin page-range

<idx></idx>{\studentZ|)} % end page-range



%%%%%%%%%%%%%%%%%%%%%%%%%%%%%%%%%%%%%%%%%%%%%%%%%%%%%%%%%%%%%%%%%%%%%%%%%%%%%%%%%%%%%%%%%%%%%%%%%%%%%%

\addcontentsline{toc}{chapter}{Index}
%\printindex
\input{ABIP.ind}

<!-- -->\newpage
Note: You can do some more fancy indexing with the formatting found at
\url{https://en.wikibooks.org/wiki/LaTeX/Indexing}

\end{document}

<!-- -->\newpage

%\verb[\marginparwidth]:
\printinunitsof{in}\prntlen{\marginparwidth}

%\verb[\marginparwidth]:
\printinunitsof{mm}\prntlen{\marginparwidth}

%\verb[\marginparwidth]:
\printinunitsof{pt}\prntlen{\marginparwidth}

\pagediagram


