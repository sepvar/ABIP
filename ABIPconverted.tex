\documentclass[11pt,letter,openany,makeidx]{book}
\usepackage{amsmath}
\usepackage{macros}
\usepackage{comment}
\usepackage{graphicx}
\usepackage{microtype}
\usepackage{gfsdidot}
\usepackage[T1]{fontenc}
\usepackage{booktabs}
\usepackage{underscore}
\usepackage{caption}
\usepackage[within=chapter,chapterlistsgap=6pt]{newfloat}
\usepackage{tocloft}
\usepackage{xpicture}
\usepackage{xcolor}
%\usepackage[dvips]{xcolor}
%\GetGinDriver  % for xcolor to work well with hyperref
%\usepackage[\GinDriver]{hyperref}
\usepackage{ulem}%for \sout (done)
\usepackage[colorinlistoftodos]{todonotes}
%\usepackage[disable,colorinlistoftodos]{todonotes}
%\usepackage{layouts}
%\usepackage{showframe}
\usepackage{coordsys}
\usepackage[pdftex]{hyperref}

%\usepackage{cellpage}

\hypersetup{colorlinks=true,bookmarks=true,pdftitle=Algebra-Based Introductory Physics,pdfauthor=J.Christensen,pdfdisplaydoctitle}
% If using a bibliography, then include "backref" in list of \hypersetup items
% linkcolor=color of internal links (red); anchorcolor = color of anchor text (black); citecolor = bibliographic citations (green); filecolor = color for local URL files (cyan); menucolor = Acrobat menu (red); urlcolor = external links (magenta); hidelinks = remove all color
% citebordercolor = color of box for citations (0 1 0); fileborder = links to files box (0 .5 .5); linkbordercolor = normal links (1 0 0); menuborder; urlborder; allbordercolors; pdfborder

\includecomment{ForMe}
\includecomment{ForReviewer}
\includecomment{ForPublic}

\makeindex

\newlistof{example}{loe}{List of Examples}
\DeclareFloatingEnvironment[fileext=loe,listname="List of Examples",name=Example]{example}
\setlength{\cftexamplenumwidth}{1cm}
\newcounter{sample}
\newcounter{carrysample}
\renewcommand{\thesample}{Simple Example \arabic{sample}}
\renewcommand{\thecarrysample}{Simple Example \arabic{carrysample}}
\newenvironment{sample}{\color{rgb:red,0;green,2;blue,1}\begin{list}{\textbf{\thesample}:}{\usecounter{sample} \setcounter{sample}{\value{carrysample}} \leftmargin 12pt}}{\end{list}\setcounter{carrysample}{\value{sample}}}
\newcommand{\THREE}[6]{\vspace{-3pt}\begin{flushright} Select one:  \mbox{#1 (\ref{#4})},  \mbox{#2 (\ref{#5})}, or \mbox{#3 (\ref{#6})}.\end{flushright}}
\newcommand{\TWO}[4]{\begin{flushright} Select one:  \mbox{#1 (\ref{#3})} or \mbox{#2 (\ref{#4})}.\end{flushright}}
\newcommand{\YN}[2]{\TWO{Yes}{No}{#1}{#2}}
\newcommand{\TF}[2]{\TWO{True}{False}{#1}{#2}}
\newcommand{\return}[1]{{} \hfill \mbox{Return to \ref{#1}.}}
\newcommand{\autoreturn}[1]{{} \hfill \mbox{Return to \autoref{#1}.}}
\newcommand{\linkreturn}[2][a related idea]{{}\hfill \mbox{Return to the discussion of \protect{\hyperlink{#2}{#1}}.}}
\newcommand{\mmr}[1]{\mbox{[\protect{#1}]}}
\newcommand{\multireturn}[1]{{}\hfill Return to one of the following locations: \newline #1.}
\newcounter{AtIQ}
\renewcommand{\theAtIQ}{Answer \arabic{AtIQ}}
\newenvironment{AIQ}{\begin{list}{\textbf{Interactive \theAtIQ}:}{\usecounter{AtIQ} \leftmargin 12pt}}{\end{list}}

% Related (return), but not part of...
\newcommand{\mreturn}[1]{\note{Return to \protect{\ref{#1}}.}}
\newcommand{\mlinkreturn}[2][a related idea]{\note{Return to the discussion of \protect{\hyperlink{#2}{#1}}.}}
\newcommand{\mautoreturn}[1]{\note{Return to \protect{\autoref{#1}}.}}
\newcommand{\mmultireturn}[1]{\note{Return to one of the following locations: \newline #1.}}


\newlistof{adventure}{loa}{List of Adventures}
\DeclareFloatingEnvironment[fileext=loa,listname="List of Adventures",name=Adventure]{adventure}
\setlength{\cftadventurenumwidth}{1cm}
\newcounter{CYOA}
\renewcommand{\theCYOA}{Plan \Alph{CYOA}}
\newenvironment{CYOA}{\begin{list}{\textbf{\theCYOA}:}{\usecounter{CYOA}}}{\end{list}}
\newcounter{storyline}
\renewcommand{\thestoryline}{Storyline \arabic{storyline}}
\newenvironment{Story}{\begin{list}{\textbf{\thestoryline}:}{\usecounter{storyline} \leftmargin 12pt}}{\end{list}}

\newlistof{reallife}{irl}{List of Real Life Patterns}
%\DeclareFloatingEnvironment[fileext=irl,listname="List of Real Life Patterns",chapterlistsgaps=off,name=Real Life Patterns]{reallife}
\DeclareFloatingEnvironment[fileext=irl,listname="List of Real Life Patterns",name=Real Life Patterns]{reallife}
\setlength{\cftreallifenumwidth}{1cm}
\newcounter{IRL}
%\renewcommand{\theIRL}{\arabic{IRL}}
\newenvironment{realtable}{%\renewcommand{\arraystretch}{2}
                           %\hspace{-.2in}
                            \begin{tabular}{@{}lll@{}} \toprule Do This & Notice This & Ask This  \\ }
                            {\bottomrule \end{tabular} }%\renewcommand{\arraystretch}{.5}}
\newcommand{\dna}[3]{\midrule \begin{minipage}{4cm}\raggedright #1 \end{minipage}
                   & \begin{minipage}{4cm}\raggedright #2 \end{minipage}
                   & \begin{minipage}{4cm}\raggedright #3 \end{minipage} \\ }
\newcommand{\multidna}[1]{\multicolumn{3}{|c|}{\begin{minipage}{13cm}\center #1 \end{minipage}} \\ \midrule }


\newlistof{story}{los}{The Stories of the Equations}
\DeclareFloatingEnvironment[fileext=los,listname="The Stories of the Equations",name=This Equation's Story]{story}
\setlength{\cftstorynumwidth}{1cm}
\newcommand{\thestoryof}[1]{\marginpar{\raggedright \footnotesize The story of \\ \fcolorbox{black}{yellow}{\begin{minipage}[c]{1.5in} \center $\deq #1$ \end{minipage}}}}
\newcommand{\EqStory}[2]{\left[ {\color{rgb:red,1;green,1;blue,4} \begin{minipage}{#1}\raggedright\begin{center} #2 \end{center}\end{minipage}} \right]}
\newcommand{\EqStoryOver}[3]{\overbrace{\EqStory{#1}{#2}}^{\displaystyle #3}}
\newcommand{\EqStoryUnder}[3]{\underbrace{\EqStory{#1}{#2}}_{\displaystyle #3}}
\newcommand{\EqStoryFrac}[5]{\frac{\overbrace{\EqStory{#1}{#2}}^{\displaystyle #3}}
                                 {\underbrace{\EqStory{#1}{#4}}_{\displaystyle #5}}}


%%%%%%%%%%%%%%%%%%%%%%%%%%%%%%%%%%%%%%%%%%%%%%%%%%%%%%%%%%%%
%
%\presetkeys{todonotes}{fancyline,color=blue!15}{}
\presetkeys{todonotes}{color=blue!15,linecolor=blue!75,size=\footnotesize}{}
%
\newcounter{todocounter}
\newcommand{\dothis}[2][]
{\stepcounter{todocounter}\todo[color=green!30, #1]{\thetodocounter: #2}}
\newcommand{\docaption}[3][]
{\stepcounter{todocounter}\todo[color=green!30, prepend, caption={\thetodocounter: \underline{#2}}, #1]{#3}}
\newcommand{\addlink}[2][]
{\stepcounter{todocounter}\todo[prepend, caption={\thetodocounter: \underline{Add Link}}, #1]{#2}}
\newcounter{todourgentcounter}
\newcommand{\urgent}[2][]
{\stepcounter{todourgentcounter}\todo[color=orange!50, #1]{\thetodourgentcounter: #2}}
\newcommand{\urgcap}[3][]
{\stepcounter{todourgentcounter}\todo[color=orange!50, prepend, caption={\thetodourgentcounter: \underline{#2}}, #1]{#3}}
\newcommand{\done}[2][]
{\todo[color=yellow!10, #1]{\sout{#2}}}
%
%\newcommand{\new}[2]{}%
\newcommand{\new}[2]{\marginpar{\raggedright \footnotesize New to #1 \\ \fcolorbox{blue}{yellow!10}{\begin{minipage}[c]{1.5in} \center {\color{blue} #2 } \end{minipage}}}}%
%%%%%%%%%%%%%%%%%%%%%%%%%%%%%%%%%%%%%%%%%%%%%%%%%%%%%%%%%%%%


%%%%%%%%%%%%%%%%%%%%%%%%%%%%%%%%%%%%%%%%%%%%%%%%%%%%%%%%%%%%
%
%\newcommand{\deq}{\displaystyle}
%\newcommand{\txtfrac}[2]{{}^{#1}\!/_{\!#2}}
%
%%%%%%%%%%%%%%%%%%%%%%%%%%%%%%%%%%%%%%%%%%%%%%%%%%%%%%%%%%%%



%%%%%%%%%%%%%%%%%%%%%%%%%%%%%%%%%%%%%%%%%%%%%%%%%%%%%%%%%%%%
%
% PEOPLE AND PRONOUNS
%
% According to https://www.cdc.gov/nchs/fastats/body-measurements.htm
% Measured average height, weight, and waist circumference for adults ages 20 years and over
% Men:
% Height (inches): 69.3                 = 1.760 m
% Weight (pounds): 195.5                = 88.86 kg
% Waist circumference (inches): 39.7    = 1.01 m
% Women:
% Height (inches): 63.8                 = 1.621 m
% Weight (pounds): 166.2                = 75.55 kg
% Waist circumference (inches): 37.5    = 0.9525 m
% Source: Anthropometric Reference Data for Children and Adults: United States, 2007-2010, tables 4, 6, 10, 12, 19, 20[PDF - 1.7 MB]
%  https://www.cdc.gov/nchs/data/series/sr_11/sr11_252.pdf
%
\newcommand{\studentA}{Abdul}       \newcommand{\massA}{\mbox{$85.0\unit{kg}$}}
\newcommand{\studentB}{Beth}        \newcommand{\massB}{\mbox{$75.0\unit{kg}$}}
\newcommand{\studentC}{Carl}        \newcommand{\massC}{\mbox{$90.0\unit{kg}$}}
\newcommand{\studentD}{Diane}       \newcommand{\massD}{\mbox{$80.0\unit{kg}$}}
\newcommand{\studentE}{Erik}        \newcommand{\massE}{\mbox{$95.0\unit{kg}$}}
\newcommand{\studentF}{Frances}       \newcommand{\massF}{\mbox{$85.0\unit{kg}$}}
\newcommand{\studentX}{Xerxes}       \newcommand{\massX}{\mbox{$62.5\unit{kg}$}}
\newcommand{\studentZ}{Zambert}     \newcommand{\massZ}{\mbox{$95.0\unit{kg}$}}
% Male
\newcommand{\heA}{he}\newcommand{\himA}{him}\newcommand{\hisA}{his}\newcommand{\himselfA}{himself}
\newcommand{\HeA}{He}\newcommand{\HimA}{Him}\newcommand{\HisA}{His}
\newcommand{\heC}{he}\newcommand{\himC}{him}\newcommand{\hisC}{his}\newcommand{\himselfC}{himself}
\newcommand{\HeC}{He}\newcommand{\HimC}{Him}\newcommand{\HisC}{His}
\newcommand{\heE}{he}\newcommand{\himE}{him}\newcommand{\hisE}{his}\newcommand{\himselfE}{himself}
\newcommand{\HeE}{He}\newcommand{\HimE}{Him}\newcommand{\HisE}{His}
\newcommand{\heZ}{he}\newcommand{\himZ}{him}\newcommand{\hisZ}{his}\newcommand{\himselfZ}{himself}
\newcommand{\HeZ}{He}\newcommand{\HimZ}{Him}\newcommand{\HisZ}{His}
% Female
\newcommand{\heB}{she}\newcommand{\himB}{her}\newcommand{\hisB}{her}\newcommand{\himselfB}{herself}
\newcommand{\HeB}{She}\newcommand{\HimB}{Her}\newcommand{\HisB}{Her}
\newcommand{\heD}{she}\newcommand{\himD}{her}\newcommand{\hisD}{her}\newcommand{\himselfD}{herself}
\newcommand{\HeD}{She}\newcommand{\HimD}{Her}\newcommand{\HisD}{Her}
\newcommand{\heF}{she}\newcommand{\himF}{her}\newcommand{\hisF}{her}\newcommand{\himselfF}{herself}
\newcommand{\HeF}{She}\newcommand{\HimF}{Her}\newcommand{\HisF}{Her}
%
\newcommand{\heX}{\studentX}\newcommand{\himX}{\studentX}\newcommand{\hisX}{\studentX's}\newcommand{\himselfX}{the person of \studentX}
\newcommand{\HeX}{\studentX}\newcommand{\HimX}{\studentX}\newcommand{\HisX}{\studentX's}
%%%%%%%%%%%%%%%%%%%%%%%%%%%%%%%%%%%%%%%%%%%%%%%%%%%%%%%%%%%%%


%%%%%%%%%%%%%%%%%%%%%%%%%%%%%%%%%%%%%%%%%%%%%%%%%%%%%%%%%%%%
%
% Book macros
%
\newcommand{\aside}[2]{\marginpar{\raggedright \footnotesize\textbf{#1}: #2}}
\newcommand{\important}[1]{\\ \fcolorbox{black}{yellow}{\begin{minipage}[c]{4.925in} \center #1 \end{minipage}}\\}
\newcommand{\inlife}{\marginpar[\scriptsize \raggedright How you might observe $\Rightarrow$ this in your life.]
                               {\scriptsize \raggedleft $\Leftarrow$ How you might observe this in your life.}}
\newcommand{\touchstone}{\marginpar[\scriptsize \raggedright Where have I seen this $\Rightarrow$ before?]
                                   {\scriptsize \raggedleft $\Leftarrow$ Where have I seen this before?}}
\newcommand{\foreshadow}{\marginpar[\scriptsize \raggedright When will I ever use this? $\Rightarrow$]
                                   {\scriptsize \raggedleft $\Leftarrow$ When will I ever use this?}}
\newcommand{\foreshadowR}{\reversemarginpar
                          \marginpar[\scriptsize \raggedright When will I ever use this? $\Rightarrow$]
                                    {\scriptsize \raggedleft $\Leftarrow$ When will I ever use this?}}
\newcommand{\Touchstone}[1]{\marginpar[\scriptsize \raggedright Where have I seen this $\Rightarrow$ \\ before? #1]
                                      {\scriptsize \raggedleft $\Leftarrow$ Where have I seen this before? #1}}
\newcommand{\Foreshadow}[1]{\marginpar[\scriptsize \raggedright When will I ever use this? $\Rightarrow$ \\ #1]
                                      {\scriptsize \raggedleft $\Leftarrow$ When will I ever use this? #1}}
%
%%%%%%%%%%%%%%%%%%%%%%%%%%%%%%%%%%%%%%%%%%%%%%%%%%%%%%%%%%%%


\begin{document}

%\title{Algebra-Based Introductory Physics}
%\author{J Christensen}
%\date{Jan 2017}
%\maketitle
%\pagestyle{cellpage}

\begin{titlepage}
	\centering
%	\includegraphics[width=0.15\textwidth]{example-image-1x1}\par\vspace{1cm}
	{\Huge\bfseries Physics Connected\par}
	\vspace{1cm}
	{\Large\bfseries An Algebra-Based Introductory Physics Textbook\par}
	\vspace{1cm}
	{\large Learn like you think: an interconnected view of physics\par}
	\vspace{2cm}
	{\Large\itshape by: J Christensen\par}
	\vfill
\begin{ForReviewer}
	Version 2.3\par
	{\footnotesize
    \begin{itemize}
    \item Ideas yet to implement:
        \begin{itemize}
        \item The examples are phrased as descriptions, not examples like the homework problems.  Need to consider rephrasing these, not calling them examples, or adding actual examples that better show how to respond to the way homework problems are written.
        \item Define a different page dimension that fits on a cell phone display.  (Enhance possible cell-phone reading.)
        \end{itemize}
    \item version 2.3: June 16-28, 2017
        \begin{itemize}
        \item Updated Section 81. $F=mg$ and Section 8.2 Normal Force
        \item Added specific list of Flame Challenges
        \item Rearranged some of the subsections in the ``Seeing Physics'', added references
        \item Equations of motion for constant acceleration (Need the Story Of)
        \item Added a section to Chap 5 (1-D motion) that gives examples of solutions that require multiple steps  (one equation is insufficient)
        \item Developed the weight and mass discussion and examples
        \item Ladder leaning example in torque, plus some homework problems
        \item Added some Conceptual Homework to weight/mass
        \item Added placeholders to the Gravity chapter
        \item Removed indicators of v1.7 changes
        \end{itemize}
    \item version 2.2: June 16, 2017
        \begin{itemize}
        \item Created conversation about $F=mg$ for Chapter on types of forces.  Caused modifications in lots of places
            \begin{itemize}
            \item Added freefall to the motion chapter
            \item Created IRL and Example dropping objects to see acceleration in $F=mg$, then moved to freefall section -- new Answers to interactive questions
            \item Commented on air resistance
            \item Comments about precision in language (need to do more with precision in mathematics)
            \item Started a couple of ideas about effective theories.  (need to decide where it goes)
            \item Added detail about SI, and specifically the pound-force, pound-mass, and kilogram. to sections \ref{s:SI-MKS} and \ref{ss:weightmass}
            \item Added NIST and BIPM references (found in Wikipedia and then searched further)
            \item conversation about weight and mass.  (required reference to the chapter on Fluids and density)
            \item Moved Google search about significant figures
            \end{itemize}
        \item Added comments about fundamental forces to the section on types of force
        \item Removed indicators of v1.5 and v1.6 changes
        \end{itemize}
    \item version 2.1: June 10, 2017
        \begin{itemize}
        \item Re-commented the $\backslash$new command
        \item Started the chapters on Seeing Physics [\autoref{c:physics}] and Deeper Dive [\autoref{c:revisted}] (These should be renamed)
        \item Moved some sections on fundamental interactions
        \end{itemize}
    \item version 2.0: April 10, 2017
        \begin{itemize}
        \item Re-enabled v1.8 hides
        \item Added a link to \textit{Spacepod}, \textit{Physics Footnotes}, and \textit{Sixty Symbols}
        \item Fixed a $\backslash$dothis that was inside an $\backslash$important, causing a compile error.
        \item Removed indicators of v1.4 changes
        \end{itemize}
    \item version 1.8: April 1, 2017
        \begin{itemize}
        \item Prepare for "the public": "Disabled" the To-Do items, "Hid" the $\backslash$new revision notes, Hid the List of Tables (have none yet)
        \end{itemize}
    \end{itemize}
    }
\end{ForReviewer}
\begin{ForPublic}
{\flushleft
\textbf{Note to the reviewers:}\new{v1.8}{Added the note}
My goal with this book is to create an electronically viewable book that makes use of the advantages of being electronic.  While current e-books have the advantage of being viewable on various devices with having to carry a physical book around, most e-textbooks do not take advantage of hyperlinked text.  With this book I hope to integrate links both forward and backward.  The forward links will be used to motivate curious students.  The backward links will be used to support students who lose track of previous topics.  The integration of these will also provide a convenient opportunity for students to browse through topics they are interested in.
\newpar

At this time, I am providing a single chapter to gauge the viability.  The chapter I am providing is on Newton's Laws.  However, as you read this document, you will find many, many more partially written chapters.  All of the partial chapters and sections are intended to be place-holders for the forward- and backward-links that \autoref{c:force} depends upon.
\newpar

I created this as a PDF that, I believe, can be easily viewed on a computer or tablet.  Since some of my students also seem to read on their phone, I verified that I am also able to view the text in a reasonable manner on my Samsung phone in landscape mode.  In each case, the links should be active and easily manageable.

}
\end{ForPublic}
	\vfill

% Bottom of the page
	{\large \today\par}
\end{titlepage}

\tableofcontents
\newpage
\begin{ForReviewer}
\listoftables
\vfill
\end{ForReviewer}
%\newpage
\listoffigures
\vfill
%\newpage
%\listofstorys
%\newpage
\listofexamples
\vfill
%\newpage
\listofadventures
\vfill
%\newpage
\listofreallifes
\vfill
\newpage

\listoftodos

\newpage



\part{Prerequisites}

<chapter><title>The Story of Science</title>

</chapter><chapter><title></title>Seeing Physics}\label{c-physics}

</chapter><chapter><title></title>Why so much math?}

</chapter><chapter><title></title>Estimating and Uncertainty}

\part{Introducing Motion, Force, and Energy}

</chapter><chapter><title></title>One-Dimensional Motion}\label{c-motion}

</chapter><chapter><title></title>Two-Dimensional Motion}

</chapter><chapter><title></title>Force}\label{c-force}
</section><section><title></title>How Physicists Use the Words (Notation)}\label{s-forcewords}
</section><section><title></title>Connecting the Concepts: Newton's Laws}\label{s-Newton}<aside><title>Referenced by</title> <p>Discussion of <xref provisional=""></xref></p></aside>[how to describe forces]{d-Newtonahead}
    </subsection><subsection><title></title>Translating Newton's First Law: The Law of Inertia}\label{ss-NI}<aside><title>Referenced by</title> <p>Discussion of <xref provisional=""></xref></p></aside>[how to describe forces]{d-Newtonahead}
    </subsection><subsection><title></title>Translating Newton's Second Law: The Equation Law}\label{ss-NII}<aside><title>Referenced by</title> <p></p></aside>{\mmr{<xref ref="" />sss-vectorequations}}, \mmr{<xref ref=""></xref>{d-Newtonahead}{how to describe forces}}, \mmr{<xref ref=""></xref>{d-atrestinmotion}{Newton's first law}}, \mmr{<xref ref="" />d-Fgball}}}
    </subsection><subsection><title></title>Translating Newton's Third Law: Action \& Reaction}\label{ss-NIII}<aside><title>Referenced by</title> <p></p></aside>{\mmr{<xref ref=""></xref>{d-Newtonahead}{how to describe forces}}, \mmr{<xref ref="" />ex-braced}}, \mmr{<xref ref="" />ex-unbraced}}}
</section><section><title></title>Examples} \label{s-NewtonExamples}<aside><title>Referenced by</title> <p><xref provisional="" /></p></aside>{sss-NIItogether}
</section><section><title></title>Summary and Homework}

</chapter><chapter><title></title>The Many Types of Force}\label{c-forcetype}<aside><title>Referenced by</title> <p>Discussion of <xref provisional=""></xref></p></aside>[subscript notation of forces]{d-interaction}
</section><section><title></title>Gravity at the Surface of the Earth}\label{s-Fg}<aside><title>Referenced by</title> <p></p></aside>{\mmr{<xref ref=""></xref>{d-accgrav}{freefall}}, \mmr{<xref ref="" />f-firstFBD}}}<!-- -->\new{v2.2}{Adding detail}
    </subsection><subsection><title></title>Weight versus Mass}\label{ss-weightmass}<aside><title>Referenced by</title> <p></p></aside>{\mmr{<xref ref="" />ss-convertunits}}, \mmr{<xref ref="" />s-sigfig}}}<idx></idx>{Weight}<!-- -->\new{v2.2}{Added detail.  Moved the previous version to \protect{<xref ref="" />s-sigfig}} to smooth the transition to \protect{<xref ref="" />ss-equivmm}}.}
    </subsection><subsection><title></title>Calculating the weight}\label{ss-local-mg}<!-- -->\new{v2.2}{renamed this section and added detail}
</section><section><title></title>Fundamental Forces}\label{s-fundamental}<idx><h></h><h></h></idx>{Force!Fundamental}<!-- -->\new{v2.1}{Started the section on fundamental interactions.  Link ahead, rather than detailling here.}
</section><section><title></title>Normal Force}\label{s-FN}<aside><title>Referenced by</title> <p></p></aside>{\mmr{<xref ref="" />f-firstFBD}}, \mmr{<xref ref="" text="type-global" />A-floor}}, \mmr{<xref ref="" />s-FT}}}<!-- -->\new{v2.2}{Added detail}<idx><h></h><h></h></idx>{Force!Normal}
</section><section><title></title>Tension}\label{s-FT}<aside><title>Referenced by</title> <p>Discussion of <xref provisional=""></xref></p></aside>[<m>F=ma</m>]{d-fma}<idx><h></h><h></h></idx>{Force!Tension}
    </subsection><subsection><title></title>Tension as a Support Force}\label{ss-tension-support}
    </subsection><subsection><title></title>Tension as Dragging Force}\label{ss-tension-drag}
    </subsection><subsection><title></title>Pulleys}
    </subsection><subsection><title></title>Interesting Complications}
</section><section><title></title>Frictional Force}\label{s-Ff}<aside><title>Referenced by</title> <p></p></aside>{\mmr{<xref ref="" text="type-global" />A-chair2}}, \mmr{<xref ref="" text="type-global" />A-chair6}}, \mmr{<xref ref="" text="type-global" />A-chair7}}, \mmr{<xref ref="" />A-fly.balls}}}
</section><section><title></title>Spring Force}\label{s-springs}<aside><title>Referenced by</title> <p></p></aside>{\mmr{<xref ref=""></xref>{d-fma}{<m>F=ma</m>}}, \mmr{<xref ref=""></xref>{d-usesofFma}{uses of <m>F=ma</m>}}, \mmr{<xref ref="" />ss-scales}}}
<section xml:id="s-FA"><title>Applied Force</title>
</section><section><title></title>Putting it Together, <m>F_\mathrm{net}</m>}\label{s-Fnet}
</section><section><title></title>Summary and Homework}

</chapter><chapter><title></title>Energy and the Transfer of Energy}


\part{Interesting Uses of Motion, Force, and Energy}

</chapter><chapter><title></title>Momentum: A Better Way to Describe Force}\label{c-momentum}<aside><title>Referenced by</title> <p></p></aside>{\mmr{<xref ref=""></xref>{d-objectinmotion}{objects in motion}}, \mmr{<xref ref="" />sss-inertia}}, \mmr{<xref ref="" />ss-NIII}}, \mmr{<xref ref="" text="type-global" />A-chair6}}}

</chapter><chapter><title></title>Rotational Motion}

</chapter><chapter><title></title>Circular Motion and Centripetal Force}

</chapter><chapter><title></title>Torque and the <m>F=ma</m> of Rotations}\label{c-torque}<aside><title>Referenced by</title> <p><xref provisional="" /></p></aside>{a-NIIIaction}<!-- -->\new{v2.3}{Added an example that is computable here, but helps introduce normal force in \protect{<xref ref="" />s-FN}}.}

</chapter><chapter><title></title>Energy of Rotating Objects}




\part{Making Waves}

</chapter><chapter><title></title>Fluids}<!-- -->\new{v2.2}{Placeholder}
</chapter><chapter><title></title>Oscillations}\label{c-SHM}
</chapter><chapter><title></title>Sound}

\part{Is It Hot in Here?}

</chapter><chapter><title></title>The flow of thermal energy}

\part{Let There be Light!}

</chapter><chapter><title></title>The Electrical Interaction}\label{c-electric}<aside><title>Referenced by</title> <p>Discussion of <xref provisional=""></xref></p></aside>[fundamental forces]{d-fundamental}

\part{What Have You Done for Me Lately?}

</chapter><chapter><title></title>Relativity}
</chapter><chapter><title></title>Quantum Mechanics}<!-- -->\new{v2.1}{Decide if these subsections should be chapters in and of themselves.  These are now labeled.}
</chapter><chapter><title></title>Condensed Matter}
</chapter><chapter><title></title>Astronomy}
</chapter><chapter><title></title>Cosmology}


\part{Supplements}

</chapter><chapter><title></title>Deeper Dive}\label{c-revisted}<!-- -->\new{v2.1}{This chapter should mirror \protect{<xref ref="" />c-physics}}.}
</chapter><chapter><title></title>Podcasts and Videos}\label{c-videos}\label{c-podcasts}

SKIPPED TWO CHAPTERS

</chapter><chapter><title></title>Characters}




% YOU ARE HERE



</chapter><chapter><title></title>Answers to Interactive Questions}

\begin{AIQ}
</p></li><li><p>\label{A-hbf} TOOK <assemblage>Return to: </assemblage> <xref provisional="" />{IQ-holdbook}
</p></li><li><p>\label{A-chair1} TOOK <assemblage>Return to: </assemblage> <xref provisional="" />{irl-NI}
</p></li><li><p>\label{A-chair2} TOOK  <assemblage>Return to: </assemblage> <xref provisional="" />{irl-NI}
</p></li><li><p>\label{A-weight.loss} TOOK <assemblage>Return to: </assemblage> <xref provisional="" />{irl-scale}
</p></li><li><p> \label{A-ladderNf} TOOK <assemblage>Return to: </assemblage> <xref provisional="" />{ex-ladder2}
</p></li><li><p>\label{A-hbnof}  TOOK  <assemblage>Return to: </assemblage> <xref provisional="" />{IQ-holdbook}
</p></li><li><p>\label{A-netF-a} TOOK <assemblage>Return to: </assemblage> <xref provisional="" />{se-weightA}
</p></li><li><p>\label{A-chair3} TOOK <assemblage>Return to: </assemblage> <xref provisional="" />{irl-NI}
</p></li><li><p>\label{A-weight.gain} TOOK <assemblage>Return to: </assemblage> <xref provisional="" />{irl-scale}
</p></li><li><p>\label{A-nowall} TOOK  <assemblage>Return to: </assemblage> <xref provisional="" />{ex-ladder2}
</p></li><li><p>\label{A-true1} TOOK  <assemblage>Return to: </assemblage> <xref provisional="" />{IQ-holdbook}
</p></li><li><p>\label{A-chair4} TOOK <assemblage>Return to: </assemblage> <xref provisional="" />{irl-NI}
</p></li><li><p>\label{A-scale.increase} TOOK  <assemblage>Return to: </assemblage> <xref provisional="" />{irl-scale}
</p></li><li><p>\label{A-nowallC} TOOK <assemblage>Return to: </assemblage>{\mmr{<xref ref="" text="type-global" />A-nowall}}, \mmr{<xref ref="" />ex-ladder2}}}
</p></li><li><p>\label{A-gworld} TOOK <assemblage>Return to: </assemblage>{\mmr{<xref ref="" text="type-global" />A-gpeaks}}, \mmr{<xref ref="" />t-gworld}}}
</p></li><li><p>\label{A-false1} TOOK <assemblage>Return to: </assemblage> <xref provisional="" />{IQ-holdbook}
</p></li><li><p>\label{A-chair5} TOOK <assemblage>Return to: </assemblage> <xref provisional="" />{irl-NI}
</p></li><li><p>\label{A-scale.measure} TOOK  <assemblage>Return to: </assemblage> <xref provisional="" />{irl-scale}
</p></li><li><p>\label{A-gpeaks} TOOK  <assemblage>Return to: </assemblage> <xref provisional="" />{t-gworld}
</p></li><li><p>\label{A-falls}  TOOK   <assemblage>Return to: </assemblage> <xref provisional="" />{IQ-holdbook}
</p></li><li><p>\label{A-chair6} TOOK <assemblage>Return to: </assemblage> <xref provisional="" />{irl-NI}
</p></li><li><p>\label{A-fly.balls} TOOK <assemblage>Return to: </assemblage> <xref provisional="" />{irl-nonparabolic}
</p></li><li><p>\label{A-hitY} TOOK  <assemblage>Return to: </assemblage> <xref provisional="" />{IQ-holdbook}
</p></li><li><p>\label{A-noFT} TOOK <assemblage>Return to: </assemblage> <xref provisional="" />{A-chair5}
</p></li><li><p>\label{A-scale.ramp} TOOK   <assemblage>Return to: </assemblage> <xref provisional="" />{irl-scale}
</p></li><li><p>\label{A-pitches.side} TOOK  <assemblage>Return to: </assemblage> <xref provisional="" />{irl-nonparabolic}
</p></li><li><p>\label{A-hitN} TOOK <assemblage>Return to: </assemblage> <xref provisional="" />{IQ-holdbook}
</p></li><li><p>\label{A-chair7} TOOK <assemblage>Return to: </assemblage> <xref provisional="" />{irl-NI}
</p></li><li><p>\label{A-pitches.top} TOOK <assemblage>Return to: </assemblage> <xref provisional="" />{irl-nonparabolic}
</p></li><li><p>\label{A-landedY} TOOK  <assemblage>Return to: </assemblage> <xref provisional="" />{IQ-holdbook}
</p></li><li><p>\label{A-gravity}  TOOK <assemblage>Return to: </assemblage> <xref provisional="" />{A-hbf}
</p></li><li><p>\label{A-chair8} TOOK <assemblage>Return to: </assemblage> <xref provisional="" />{irl-NI}
</p></li><li><p>\label{A-pool-roll} TOOK <assemblage>Return to: </assemblage> <xref provisional="" />{irl-poolcushion}
</p></li><li><p>\label{A-landedN} TOOK <assemblage>Return to: </assemblage> <xref provisional="" />{IQ-holdbook}
</p></li><li><p>\label{A-FT} TOOK  <assemblage>Return to: </assemblage> <xref provisional="" />{A-chair5}
</p></li><li><p>\label{A-pool-bumper} TOOK <assemblage>Return to: </assemblage> <xref provisional="" />{irl-poolcushion}.
</p></li><li><p>\label{A-zero} TOOK <assemblage>Return to: </assemblage> <xref provisional="" />{IQ-holdbook}
</p></li><li><p>\label{A-firstfall} TOOK  <assemblage>Return to: </assemblage> <xref provisional="" />{irl-freefall}
</p></li><li><p>\label{A-floor}  TOOK  <assemblage>Return to: </assemblage> <xref provisional="" />{se-FNB}
</p></li><li><p>\label{A-firstwhy} TOOK  <assemblage>Return to: </assemblage> <xref provisional="" />{irl-freefall}
</p></li><li><p>\label{A-noncue} TOOK <assemblage>Return to: </assemblage> <xref provisional="" />{irl-poolcushion}
</p></li><li><p>\label{A-one} TOOK  <assemblage>Return to: </assemblage> <xref provisional="" />{IQ-holdbook}
</p></li><li><p>\label{A-fallv} TOOK  <assemblage>Return to: </assemblage> <xref provisional="" />{irl-freefall}
</p></li><li><p>\label{A-pool-spin} TOOK <assemblage>Return to: </assemblage> <xref provisional="" />{irl-poolcushion}
</p></li><li><p>\label{A-second} TOOK  <assemblage>Return to: </assemblage> <xref provisional="" />{se-FNB}
</p></li><li><p>\label{A-falla} TOOK  <assemblage>Return to: </assemblage> <xref provisional="" />{irl-freefall}
</p></li><li><p>\label{A-pool-later} TOOK <assemblage>Return to: </assemblage> <xref provisional="" />{irl-poolcushion}
</p></li><li><p>\label{A-two} TOOK  <assemblage>Return to: </assemblage> <xref provisional="" />{IQ-holdbook}
</p></li><li><p>\label{A-third} TOOK  <assemblage>Return to: </assemblage> <xref provisional="" />{se-FNB}

</p></li><li><p>\label{A-swing-tension} TOOK <assemblage>Return to: </assemblage>{\mmr{<xref ref="" />irl-tension}}, \mmr{<xref ref="" text="type-global" />A-chandelier-tension}}}
</p></li><li><p>\label{A-fan-tension} TOOK <assemblage>Return to: </assemblage> <xref provisional="" />{irl-tension}
</p></li><li><p>\label{A-chandelier-tension} TOOK   <assemblage>Return to: </assemblage> <xref provisional="" />{A-fan-tension}
\end{AIQ}

</chapter><chapter><title></title>Adventures}


Throughout the book, there are examples and adventures.  The follow-up stories are contained below.
\begin{Story}
</p></li><li><p>\label{a-parkandwalk}  TOOK
</p></li><li><p>\label{a-NIIIaction} TOOK
</p></li><li><p>\label{a-coastindrive} TOOK
</p></li><li><p>\label{a-NIIIreaction} TOOK
</p></li><li><p>\label{a-coastinneutral} TOOK
</p></li><li><p>\label{a-NIIIconcern} TOOK
</p></li><li><p>\label{a-NIdrive} TOOK
</p></li><li><p>\label{a-NIIIexperiment} TOOK
</p></li><li><p>\label{a-nogas} TOOK
</p></li><li><p>\label{a-NIIIrestraint} TOOK
</p></li><li><p>\label{a-parkandwalk2}  TOOK
</p></li><li><p>\label{a-NIIIfaculty} TOOK
</p></li><li><p>\label{a-intosunset} TOOK
</p></li><li><p>\label{a-NIIIsecurity} TOOK
</p></li><li><p>\label{a-NIresult} TOOK
</p></li><li><p>\label{a-guilty} TOOK
</p></li><li><p>\label{a-nogas2} TOOK
</p></li><li><p>\label{a-intosunset2} TOOK
\end{Story}

%%%%%%%%%%%%%%%%%%%%%%%%%%%%%%%%%%%%%%%%%%%%%%%%%%%%%%%%%%%%%%%%%%%%%%%%%%%%%%%%%%%%%%%%%%%%%%%%%%%%%%


%%%%%%%%%%%%%%%%%%%%%%%%%%%%%%%%%%%%%%%%%%%%%%%%%%%%%%%%%%%%%%%%%%%%%%%%%%%%%%%%%%%%%%%%%%%%%%%%%%%%%%

\addcontentsline{toc}{chapter}{Index}
%\printindex
\input{ABIP.ind}

<!-- -->\newpage
Note: You can do some more fancy indexing with the formatting found at
\url{https://en.wikibooks.org/wiki/LaTeX/Indexing}

\end{document}

<!-- -->\newpage

%\verb[\marginparwidth]:
\printinunitsof{in}\prntlen{\marginparwidth}

%\verb[\marginparwidth]:
\printinunitsof{mm}\prntlen{\marginparwidth}

%\verb[\marginparwidth]:
\printinunitsof{pt}\prntlen{\marginparwidth}

\pagediagram


